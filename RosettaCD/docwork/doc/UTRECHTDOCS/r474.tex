
\documentstyle{Rosetta}
\begin{document}
   \RosTopic{Rosetta3.doc.dictionary.English}
   \RosTitle{Lexical Entry Specification: English BNOUN}
   \RosAuthor{Margreet Sanders, Petra de Wit}
   \RosDocNr{474}
   \RosDate{December 12, 1991}
   \RosStatus{approved}
   \RosSupersedes{-}
   \RosDistribution{Project}
   \RosClearance{Project}
   \RosKeywords{BNOUN, dictionary, Lexical Entry, English}
   \MakeRosTitle
%
%

\section{Introduction}

This document describes the attributes of the English basic noun (BNOUN) 
records and their interpretation. 
It also serves as a guide for dictionary filling, with a few 
examples showing how to decide on specific attribute values.
Part of the document is a copy of a similar 
document for Dutch BNOUNs, written by Franciska de 
Jong (R0284: {\em The filling of BNOUN entries\/}).
 
\section{The attributes of the BNOUN record}
In Appendix A, the definition of the BNOUN record in the English Domain T is 
given. The attributes mentioned there are described in more detail here. The 
full set of possible attribute values (which may include values not used for 
BNOUN itself) is given in Appendix B. The attributes are:

\begin{description}
\item[plurforms] This attribute indicates which form the plural of the noun 
takes. There are nine possible values:
\begin{description}
  \item [onlyplur] For nouns that are always plural (e.g. \ {\em scissors, 
cattle\/}).
  \item [noplur] For nouns that are always singular (e.g. \ {\em gold, 
milk\/}).
  \item [irrplur] For nouns that have an irregular plural (e.g. \ {\em child, 
ox\/}).
  \item [irrSplur] For nouns 
ending in {\em -y} that form their plural by adding an {\em -s} (e.g. \ {\em 
Germany\/}).
  \item [regplur] For nouns with a regular plural (e.g. \ {\em cat, dog\/}).
  \item [regEplur] For nouns that form a regular plural with {\em -es} (e.g. \ {\em 
hero, tomato\/}).
  \item [voicingplur] For nouns with plurals that voice the last element of the 
stem (e.g. \ {\em wife, calf\/}).
  \item [latplur] For nouns that form a latinate plural (e.g. \ {\em antenna, 
stimulus\/}).
  \item [singandplur] For nouns that have the same form in plural and 
singular (e.g. \ {\em sheep, salmon\/}).
\end{description}
 No attribute value has been defined yet for nouns 
obligatorily taking a `periphrastic' plural, like {\em a (loaf of) bread - two 
loaves of bread\/}. 

\item[animate] This attribute separates inanimate from animate nouns. Nouns are 
considered animate when they can be frightened physically. Under this 
definition plants are not animate, whereas nouns such as {\em council} are. In 
Rosetta3, the value of this attribute has not been used in any rule yet. 
Possible values are:
  \begin{description}
  \item[yesanimate] For nouns that refer to animate beings (e.g.\ {\em cat, 
child, doctor, worm, public, council\/}).
  \item[noanimate] For nouns that do not refer to animate beings (e.g.\ {\em 
house, tree\/}).
  \end{description}
The value {\bf omegaAnimate} should not be used for BNOUNs.

\item[genders] This attribute is used for inanimate nouns, to 
indicate which anaphora and reflexives they take. The attribute is a set value, 
but only for historical reasons. The possible values are:
  \begin{description}
  \item[omegagender] For nouns that are referred to by {\em it\/}. This is the default setting.
  \item[masculinegender] For nouns that may be referred to by {\em he\/} 
(no examples found yet).
  \item[femininegender] For nouns that may be referrred to by {\em she\/} 
(e.g.\ {\em ship\/} in {\em Do you see that ship? She rides the waves like a 
queen\/}).
  \end{description}

\item[sexes] This attribute is used for animate nouns to indicate
by which pronoun the noun must be referred to. Lower animals receive no sex,
since they usually are referred to by {\em it\/}. The attribute is a set value,
but maybe each entry should be split up so that only one value results.
The possible values are:
  \begin{description}
  \item[{[ ]}] For inanimate nouns and lower 
animals (e.g.\ {\em table, sea urchin\/}).
  \item[masculine] For beings that are inherently masculine (e.g.\ {\em bull
\/}).
  \item[feminine] For beings that are inherently feminine (e.g.\ {\em 
mare\/}).
  \item[masculine,feminine] For beings that are animate, but can be of 
either sex (e.g.\ {\em baby, dog\/}).
  \end{description}
It is not clear that English needs both genders and sexes. This distinction was 
made for Dutch to account for the fact that the neutgender word {\em meisje} 
can be referred to by {\em she}. No examples of this kind have been found for 
English.
\item[human] For animate nouns, this attribute indicates whether the noun 
refers to a human or not. On the basis of this distinction, the choice for a 
relative pronoun is made: humans are referred to by {\em who\/}, non-humans by 
{\em which\/}. Possible values are:
  \begin{description}
  \item[nohuman] For animate nouns that are not human (e.g.\ {\em cat, worm
\/}, but also for group nouns like {\em public\/} in cases where they take 
grammatical concord i.s.o.\ 
notional concord ({\em The public, which demanded more action\/}), and for 
nouns like {\em 
baby\/} when the are referred to as {\em it\/} or {\em which\/} ({\em The baby
 which she showed me\/})).
  \item[human] For animate nouns that are human (e.g.\ {\em doctor, child\/}, 
but also for group nouns like {\em Government\/} in cases where they take 
notional concord rather than grammatical concord ({\em The Government, who are 
cutting their losses\/}), and for pet animals like {\em horse\/} when they are 
referred to as {\em who\/} or {\em he/she\/} ({\em The horse who cut his hind
leg\/})).
  \end{description}
The value {\bf omegahuman} should not be assigned to BNOUNs.

\item[posscomas] This attribute separates mass and count nouns. Tests to 
distinguish between the two are:
  \begin{itemize}
  \item In English, count nouns take {\em many\/} as determiner, while mass 
nouns take {\em much\/} (e.g.\ much water, many roses).
  \item Mass nouns may occur without a determiner (e.g.\ {\em There is bread on 
the table\/}; $^{*}${\em There is rose in the vase\/}).
  \item Most count nouns may take the indefinite article {\em a(n)\/} (e.g.\ 
{\em a plant, an hour\/} vs.\ $^{*}${\em a milk\/}).
  \end{itemize}
Contrary to the strategy adopted in Dutch, words that have both a mass and a 
count reading (e.g.\ {\em success\/}) are not split into two lemmas but are 
assigned the value {\bf [mass,count]}. In these cases the NP-grammar should 
make the proper distinctions. This strategy was chosen for efficiency reasons 
(splitting up the dictionary was too time consuming). It is not clear whether
there are other principled reasons to handle the mass/count distinctions 
in the NP-grammar.
Possible 
values are:
  \begin{description}
  \item[count] For count nouns (e.g.\ {\em house, rose\/}, but also an
inherently plural noun like {\em cattle\/} ({\em He has 20 cattle on the farm
\/})).
  \item[mass] For mass nouns (e.g. {\em gold, water\/} and the inherently 
plural {\em brains\/} (as in {\em The brains of an elephant are very large\/}))
.
  \item[mass,count] For nouns that have both a mass and a count reading (e.g.\ 
{\em success, chance\/}).
  \end{description}

\item[subcs] This set attribute is used to subclassify nouns. It is not certain 
that all distinctions made are really necessary for English. Some of them have 
simply been copied from Dutch. The values currently possible are:
  \begin{description}
  \item[professionnoun] For nouns that indicate a profession (e.g.\ {\em 
policeman\/}). This value is not relevant for English in the sense that 
the indefinite article before professionnouns, contrary to Dutch, cannot be 
deleted, except when the reference is to a unique function. (e.g.\ {\em He is a 
doctor, He is king\/}.
  \item[vocativenoun] For nouns that may be used 
when addressing a person ({\em e.g.\ 
doctor, miss\/}, and perhaps also {\em (Mr) chairman\/}).
  \item[relationnoun] For nouns that indicate a family (or other close) 
relationship (e.g.\ {\em brother, granddaughter\/}, but also {\em 
colleague\/}). This attribute is used in Dutch to distinguish which nouns can 
receive genetive case, but since in English this set is much larger than 
relationnouns, this value might be superfluous.
  \item[unitnoun] For nouns that may be used when forming a `Measure Phrase'
(e.g.\ {\em metre, kilo\/}).
  \item[othernoun] The default value for ordinary nouns.
  \end{description}

\item[personal] This Boolean attribute indicates whether the noun is a 
personal noun or not. A noun is considered to be non-personal when it can 
occur in non-referential {\em It\/}-constructions like:
\begin{itemize} 
\item {\em It is N today\/}.
\end{itemize}. The following values are distinguished:
\begin{description}
  \item [false] For non-personal nouns (e.g. \ {\em fall, Christmas\/})
  \item [true] For personal nouns (e.g. \ {\em boy, fair\/}). This is also the 
               default value. 
\end{description}
Notice that the set of nouns that can occur in non-referential 
{\em It\/}-constructions is much smaller than in Dutch. Words such as {\em fair
, sale, party} corresponding to Dutch {\em kermis, uitverkoop, feest} can only 
occur in constructions starting with {\em there} as in {\em There is a sale}
\item[thetanp] This attribute indicates the number of arguments that a noun 
selects. Notice that
all arguments selected by a noun are considered to be optional. The following 
values are distinguished:
\begin{description}
  \item [omegathetanp] For nouns that do not select any arguments.
  \item [np120] For nouns that select two arguments.
  \item [np123] For nouns that select three arguments.
\end{description}
The values {\bf np000} and {\bf np100} are superfluous and therefore not used in the 
dictionary.
\item[nounpatterns] This attribute indicates how the arguments a noun may 
take are realised syntactically. Up till now, the following patterns have 
been distinguished for English :
\begin{description}
  \item [synPREPNP] e.g. {\em desire} as in {\em his desire for wealth}.
  \item [synTHATSENT] e.g. {\em observation} as in {\em the observation that pigs 
can fly}.
  \item [synOPENTOSENT] e.g. {\em order} as in {\em his order to buy 20 cattle}
.
  \item [synQSENT] e.g. {\em uncertainty} as in {\em the uncertainty whether she 
knows}.
  \item [synPREPCLOSEDGERUND] e.g. {\em danger} as in {\em the danger of him 
finding out}. 
  \item [synPREPOPENGERUND] e.g. {\em custom} as in {\em the custom of giving 
presents at Christmas}.
  \item [synPREPNP\_THATSENT] e.g. {\em message} as in {\em the message to me 
that he would be late}.
  \item [synPREPNP\_OPENTOSENT] e.g. {\em offer} as in {\em his offer to me to 
buy the house}.
\end{description}
For a more detailed description, see De Jong (to appear). Notice that the 
examples given here of patterns {\bf prepclosedgerund} and {\em prepopengerund}
can also be analyzed as simple gerunds where the preposition is introduced 
syncategorematically.

\item[prepkey] This attribute indicates the S-key of the preposition 
a noun may take as a prepositional phrase complement.
The default value is zero. Non defaults are e.g.\ {\em dislike of, answer to,
feeling for\/}.

\item[poss] This Boolean attribute indicates whether the noun may take a 
genitive {\em 's\/} (e.g.\ {\em father's\/}). The default value is {\bf true}. 
A few nouns have received the value {\bf 
false}, like {\em call\/} and {\em demand\/}, but no good test has been found 
yet to decide on which grounds the item can be filled.

\item[temporal] This Boolean attribute is used to mark any noun that can have a 
temporal interpretation. Tests to find out whether a noun has such an 
interpretation are:
  \begin{itemize}
  \item The NP can be combined with a temporal preposition (during: {\em During 
the thunderstorm\/}).
  \item The NP can be followed by a temporal adverb (ago: {\em Three hours ago
\/}).
  \end{itemize}
The default setting for the attribute is {\bf false}.

\item[class] For nouns with the value {\bf true} for 
the attribute {\em temporal\/}, this attribute indicates what kind 
of temporal reference is established. The attribute has a set value, but only 
for historical reasons. The possible values are:           
  \begin{description}
  \item[omegaTimeAdvClass] the default value, for non-temporal nouns.
  \item[duration] For nouns that indicate the duration of a time interval 
(e.g.\ {\em hour\/} in {\em for three hours\/}).
  \item[reference] For nouns that indicate a moment or closed interval in time 
(e.g.\ {\em year\/} in {\em Last year we went to France\/}).
  \item[frequential] For nouns that indicate repetition in time (e.g. 
{\em time\/} in {\em three times\/}).
  \end{description}

\item[deixis] Also for nouns with the value {\bf true} for 
the attribute {\em temporal\/}. The noun must also have the value {
\bf reference} for the attribute {\em class\/}. The deixis attribute indicates 
what kind of deictic reference is established. For further explanation, 
see docs.\ 53 and 263 by 
Lisette Appelo ({\em Temporal Expressions in Rosetta3\/} and {\em Documentation 
of the rules for the translation of temporal expressions in Rosetta3, part 1\/} 
respectively). The attribute is not 
used in Rosetta3 yet for BNOUNs. Possible values are:
  \begin{description}
  \item[omegadeixis] For non-deictic nouns. All nouns now have this default 
value.
  \item[presentdeixis] For nouns that are inherently `present' (perhaps the 
noun {\em present\/} as in {\em At present, there are no unicorns\/}).
  \item[pastdeixis] For nouns that are inherently `past' (perhaps the 
noun {\em past\/} as in {\em In the past, there were unicorns\/}).
  \item[futuredeixis] For nouns that are inherently `future' (perhaps the 
noun {\em future\/} as in {\em In the future, there will be tetracorns\/}).
  \end{description}

\item[aspect] Also for nouns with the value {\bf true} for 
the attribute {\em temporal\/}. This attribute indicates what
kind of inherent `aspect' a noun has. For further explanation, see docs.\ 53 and 263 
by 
Lisette Appelo ({\em Temporal Expressions in Rosetta3\/} and {\em Documentation 
of the rules for the translation of temporal expressions in Rosetta3, part 1\/} 
respectively). Only the default value is used in Rosetta3 for BNOUNs. Possible 
values are:
  \begin{description}
  \item[omegaAspect] For non-temporal nouns. This is the default value.  
  \item[habitual] For nouns that describe habituality.
  \item[imperfective] For nouns that specify a time interval.
  \item[perfective] For nouns that specify a moment in time.
  \end{description}

\item[retro] Also for nouns with the value {\bf true} for 
the attribute {\em temporal\/}. This Boolean attribute indicates 
whether the noun has an inherently retrospective aspect. Only the default value 
(which is {\bf false}) is used in Rosetta3 for BNOUNs.

\item[req, env] These attributes are used to indicate the 
    {\em  polarity\/} properties of the noun, i.e.\ whether it requires a 
    question, an emphatic positive or a negated sentence, or creates
    such an environment for other elements (e.g.\ for the determiner {\em any\/
}).
    The attribute has a set value, but only for historical reasons. 
    The attributes are not used in English noun rules yet. 
    The possible values are the same as for BVERBs, and all 
    nouns now have the default value, {\bf [pospol, negpol, omegapol]}.

\item[KEY] This attribute indicates the language specific S-key of the BNOUN, 
which is related to an interlingual M-key.

\end{description}

\newpage
\section{Consistency Checks on the filling of BNOUNs}

The following consistency checks are currently performed:

\begin{verbatim}
< BNOUN
: IMPLIES((human = yeshuman),(animate = yesanimate)) 
           "yeshuman implies yesanimate"
: IMPLIES(true,(animate <> omegaanimate))           
           "there are no omegaanimate bnouns"
: IMPLIES(true,(human   <> omegahuman))              
           "there are no omegahuman bnouns"
: IMPLIES ( (sexes <> []), (animate=yesanimate))
           "sexes specified require animate=yesanimate"
: IMPLIES( (human = yeshuman),(sexes <> []) )
           "there are no sexless human beings"
: IMPLIES( (genders <> [omegagender]), (sexes = []))
           "Only sexless objects may have a gender"

: IMPLIES ( (class <> omegatimeadvclass), (temporal =true) )
         "nondefault specification of class requires temporal =true"
: IMPLIES ( (deixis <> omegadeixis), (temporal =true) )
         "nondefault specification  of deixis requires temporal =true"
: IMPLIES ( (aspect <> omegaaspect), (temporal =true) )
         "nondefault specification  of aspect requires temporal =true"
: IMPLIES ( (retro<> false), (temporal =true) )
         "nondefault specification  of retro requires temporal =true"

: IMPLIES ( (subcs * [vocativenoun, professionnoun, relationnoun] <> []),
            (human = yeshuman) )
        "vocativenouns, professionnouns, relationnouns must be yeshuman"
: IMPLIES ( (human = nohuman),
           (subcs * [vocativenoun, professionnoun, relationnoun] = []) )
        "nohuman nouns cannot be vocativenoun, professionnoun, or relationnoun"

: IMPLIES( ((posscomas = [mass]) AND ([Noplur] <> plurforms )), false)
         " Mass nouns do not have a plural"
:IMPLIES((prepkey <> 0), (nounpatterns * LSAUXDOM_prepobjvps <> []))
            "prepkey <> 0 requires appropriate nounpattern"

:IMPLIES((thetanp <> omegathetanp), (nounpatterns <> []))
            "thetanp <> omegathetanp requires nounpattern(s)"

:IMPLIES((thetanp=np120), (nounpatterns <= LSAUXDOM_vp120vps))
           "nounpatterns and thetanp incompatible"

:IMPLIES((thetanp=np123), (nounpatterns <= LSAUXDOM_vp123vps))
           "nounpatterns and thetanp incompatible"

\end{verbatim}
\vspace{3 ex}
The following cases are marked as being PECULIAR:

\begin{verbatim}

:IMPLIES( ((count IN posscomas) AND (Noplur IN plurforms)), false)
         $P "most count nouns do have a plural"
>

\end{verbatim}

\appendix
\newpage
\section{The BNOUN record}
\begin{verbatim}
BNOUNrecord   =
		<
		 req:              polarityEFFSETtype:[pospol, negpol, 
						  omegapol]    
		 env:              polarityEFFSETtype:[pospol, negpol, 
						  omegapol]    
		 plurforms:        plurformSETtype:[regplur]
		 genders:          genderSETtype:[omegagender] 
		 class:            timeadvclasstype:omegaTimeAdvClass
		 deixis:           deixistype:omegadeixis
		 aspect:           aspecttype:omegaAspect
		 retro:            retrotype:false
		 sexes:            sexSETtype:[]
		 subcs:            nounsubcSETtype:[othernoun]
		 temporal:         temporaltype:false
		 animate:          animatetype:omegaanimate
		 human:            humantype:omegahuman
		 posscomas:        posscomaSETtype:[count]
		 thetanp:          thetanptype:omegathetanp
		 nounpatterns:     synpatternSETtype:[]
		 prepkey:          keytype:0
		 personal:         personaltype:true
		 poss:             posstype:true               
		 KEY              
		>
\end{verbatim}

\newpage
\section{Types used in the BNOUN record}
\begin{verbatim}
  animatetype         = (yesanimate, noanimate,omegaAnimate);
  Aspecttype          = (habitual, imperfective, perfective, omegaAspect);
  Deixistype          = (omegadeixis, presentdeixis, pastdeixis,
                         futuredeixis );                     
  gendertype          = (femininegender, masculinegender, omegagender);
  genderSETtype       = SET OF gendertype;               
  humantype           = (yeshuman, nohuman, omegahuman);  
  keytype             = INTEGER;
  nounsubctype        = (city, country, firstname, surname, institutename,
                         holidayname, vocativenoun, professionnoun, 
                         relationnoun, unitnoun, othernoun, monthnoun, 
                         daynoun, clocktimenoun, othername);
  nounsubcSETtype     = SET OF nounsubctype;
  nounsubcEFFSETtype  = {EFF}SET OF nounsubctype;
  personaltype        = BOOLEAN;
  plurformtype        = (onlyplur, noplur, irrplur, irrSplur, regplur, 
                         regEplur, latplur, voicingplur, singandplur);
  plurformSETtype     = SET OF plurformtype;
  polaritytype        = (pospol, negpol, omegapol);
  polaritySETtype     = SET OF polaritytype;             
  polarityEFFSETtype  = {EFF}SET OF polaritytype;         
  posscomatype        = (count, mass);
  posscomaSETtype     = SET OF posscomatype;
  posstype            = BOOLEAN;
  Retrotype           = BOOLEAN;   
  sexSETtype          = SET OF sextype;
  sextype             = (masculine,feminine);
  Synpatterntype      = (synASIFSENT, synBE, 
                         synCLAUSE, synCLOSEDADJPPROP, 
                         synCLOSEDADJPPROP_EMPTY, synCLOSEDADJPPROP_PREPNP, 
                         synCLOSEDGERUND, synCLOSEDINFSENT, 
                         synCLOSEDNPPROP, synCLOSEDNPPROP_EMPTY, 
                         synCLOSEDNPPROP_PREPNP,               
                         synCLOSEDTOSENT, synCLOSEDVERBPPROP,
                         synDIRCLOSEDPREPPPROP, synDIROPENPREPPPROP,
                         synDONP_DIROPENPREPPPROP, synDONP_EMPTY,
                         synDONP_LOCOPENPREPPPROP, 
                         synDONP_OPENADJPPROP, 
                         synDONP_OPENGERUND,   
                         synDONP_OPENNPPROP, 
                         synDONP_OPENTOSENT, 
                         synDONP_OTHEROPENPREPPPROP,       
                         synDONP_PREPNP, 
                         synDONP_PREPOPENADJPPROP,         
                         synDONP_PREPOPENGERUND,           
                         synDONP_PREPOPENNPPROP, 
                         synDONP_PREPOTHEROPENPREPPPROP,   
                         synDONP_PREPQSENT,                
                         synDONP_PROSENT,                  
                         synDONP_QSENT,                    
                         synDONP_THATSENT,                 
                         synEMPTY, synEMPTY_DONP, 
                         synEMPTY_CLOSEDTOSENT,            
                         synEMPTY_MEASUREPHRASE, 
                         synEMPTY_OPENGERUND,              
                         synEMPTY_OPENTOSENT, 
                         synEMPTY_PREPOPENGERUND,          
                         synEMPTY_PROSENT,                 
                         synEMPTY_PREPNP,                  
                         synEMPTY_PREP2NP,                 
                         synEMPTY_QSENT, synEMPTY_THATSENT, 
                         synFOREMPTY,                      
                         synFORTOSENT, synFRONTSOPROSENT,  
                         synIOEMPTY_DONP, synIOEMPTY_THATSENT,
                         synIOEMPTY_QSENT,                    
                         synIONP_DONP, synIONP_EMPTY,         
                         synIONP_MEASUREPHRASE, 
                         synIONP_OPENINFSENT,                 
                         synIONP_OPENNPPROP,                  
                         synIONP_OPENTOSENT, synIONP_PREPCLOSEDADJPPROP, 
                         synIONP_PREPNP, 
                         synIONP_PREPOPENGERUND,              
                         synIONP_PROSENT, synIONP_QSENT, 
                         synIONP_SOPROSENT, 
                         synIONP_THATSENT, synITTHATSENT, 
                         synLOCCLOSEDPREPPPROP, synLOCEMPTY,  
                         synLOCOPENPREPPPROP, synLOCPREPP,    
                         synMEASUREPHRASE, synNOTPROSENT,     
                         synNoVpArgs, synNP, 
                         synnoadjpargs,                       
                         synOPENADJPPROP, synOPENGERUND, 
                         synOPENGERUND_PREPNP,                
                         synOPENINFSENT,
                         synOPENTOINFSENTPROOBJ,         
                         synOPENNPPROP, synOPENTOSENT,   
                         synOPENVERBPPROP,
                         synOTHERCLOSEDPREPPPROP, 
                         synOTHERCLOSEDPREPPPROP_EMPTY,  
                         synOTHERCLOSEDPREPPPROP_PREPNP, 
                         synOTHEROPENPREPPPROP,
                         synPREPCLOSEDADJPPROP, synPREPCLOSEDGERUND, 
                         synPREPCLOSEDNPPROP, synPREPCLOSEDTOSENT, 
                         synPREPEMPTY,                   
                         synPREPMEASUREPHRASE,           
                         synPREPNP, 
                         synPREPNP_CLOSEDTOSENT,         
                         synPREPNP_EMPTY,                
                         synPREPNP_ITOPENTOSENT,         
                         synPREPNP_OPENTOSENT, synPREPNP_PREPNP, 
                         synPREPNP_PREPOPENGERUND,       
                         synPREPNP_QSENT, synPREPNP_THATSENT, 
                         synPREPOPENGERUND,              
                         synPREPOPENNPPROP, synPREPOTHERCLOSEDPREPPPROP,
                         synPREPOPENTOSENT,              
                         synPREPQSENT, synPREPTHATSENT, synPROSENT, 
                         synQSENT, synSOPROSENT, 
                         synSOPROSENT_EMPTY,             
                         synSOPROSENT_PREPNP,            
                         synTHATSENT, 
                         synTHATSENT_EMPTY,              
                         synTHATSENT_LOCOPENPREPPPROP,   
                         synTONP, synTONP_DONP,          
                         synTONP_THATSENT, synTONP_QSENT,
                         synVERBPPROP,                   
                         synDONP_OPENINFSENT,
                         vpid1, vpid2, vpid3, vpid4, vpid5,
                         vpid6, vpid7, vpid8, vpid9, vpid10,
                         vpid11, vpid12, vpid13, vpid14, vpid15,
                         vpid16, vpid17, vpid18, vpid19, vpid20,
                         vpid21, vpid22, vpid23, vpid24, vpid25,
                         vpid26, vpid27, vpid28, vpid29, vpid30,
                         vpid31, vpid32, vpid33, vpid34, vpid35,
                         vpid36, vpid37, vpid38, vpid39, vpid40,
                         vpid41, vpid42, vpid43, vpid44, vpid45,
                         vpid46, vpid47, vpid48, vpid49, vpid50,
                         vpid51, vpid52, vpid53, vpid54, vpid55,
                         vpid56, vpid57, vpid58, vpid59, vpid60,
                         vpid61, vpid62, vpid63, vpid64, vpid65,
                         vpid66, vpid67, vpid68, vpid69, vpid70,
                         vpid71, vpid72, vpid73, vpid74, vpid75,
                         vpid76, vpid77, vpid78, vpid79, vpid80,
                         vpid81, vpid82, vpid83, vpid84, vpid85,
                         vpid86, vpid87, vpid88, vpid89, vpid90,
                         vpid91, vpid92, vpid93, vpid94, vpid95,
                         vpid96, vpid97, vpid98, vpid99, vpid100,
                         vpid101, vpid102, vpid103, vpid104, vpid105,
                         vpid106, vpid107, vpid108, vpid109, vpid110,
                         vpid111, vpid112, vpid113, vpid114, vpid115,
                         vpid116, vpid117, vpid118, vpid119, vpid120,
                         vpid121, vpid122, vpid123, vpid124, vpid125,
                         vpid126, vpid127, vpid128, vpid129, vpid130,
                         vpid131, vpid132, vpid133, vpid134, vpid135,
                         vpid136, vpid137, vpid138, vpid139, vpid140,
                         vpid141, vpid142, vpid143, vpid144, vpid145,
                         vpid146, vpid147, vpid148, vpid149, vpid150,
                         vpid151, vpid152, vpid153, vpid154, vpid155,
                         vpid156, vpid157, vpid158, vpid159, vpid160,
                         vpid161, vpid162, vpid163, vpid164, vpid165,
                         vpid166, vpid167, vpid168, vpid169, vpid170,
                         vpid171, vpid172, vpid173, vpid174, vpid175,
                         vpid176, vpid177, vpid178, vpid179, vpid180,
                         vpid181, vpid182, vpid183, vpid184, vpid185,
                         vpid186, vpid187, vpid188, vpid189, vpid190,
                         vpid191, vpid192, vpid193, vpid194, vpid195,
                         vpid196, vpid197, vpid198, vpid199, vpid200,
                         );
  SynpatternSETtype   = SET OF synpatterntype;
  SynpatternEFFSETtype= {EFF}SET OF synpatterntype;
  temporaltype        = BOOLEAN;
  thetanptype         = (omegathetanp, np000, np100, np120, np123); 
  Timeadvclasstype    = (duration, reference, frequential,     
                         omegaTimeAdvClass);  
                >

\end{verbatim}
\end{document}
