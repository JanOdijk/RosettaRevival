\documentstyle{Rosetta}
\begin{document}
\newcommand{\com}[1]{"{\em #1 }"\\}
\newcommand{\tcmone}[1]{\>"{\em #1 }"\\}
\newcommand{\tcmtwo}[1]{\>\>"{\em #1 }"\\}
\newcommand{\tcmthr}[1]{\>\>\>"{\em #1 }"\\}
\newcommand{\tcmfou}[1]{\>\>\>\>"{\em #1 }"\\}
\newcommand{\tcmfiv}[1]{\>\>\>\>\>"{\em #1 }"\\}
\newcommand{\tcmsix}[1]{\>\>\>\>\>\>"{\em #1 }"\\}
\newcommand{\stpw}[2]{"{\bf {(#1)} {\em #2 }}"\\}
\newcommand{\expl}[2]{{\noindent
\bf {(#1)} {\em #2 :}}\\}
\newcommand{\tcdone}[1]{\> #1 \\}
\newcommand{\tcdtwo}[1]{\>\> #1 \\}
\newcommand{\tcdthr}[1]{\>\>\> #1 \\}
\newcommand{\tcdfou}[1]{\>\>\>\> #1 \\}
\newcommand{\tcdfiv}[1]{\>\>\>\>\> #1 \\}
\newcommand{\tcdsix}[1]{\>\>\>\>\>\> #1 \\}
\newcommand{\tcdsev}[1]{\>\>\>\>\>\>\> #1 \\}
\newenvironment{code}{\begin{tabbing}
===\====\====\====\====\====\====\=\kill}{\end{tabbing}}
\RosTopic{Rosetta3.formalism}
\RosTitle{Syntax and Semantics of the Rosetta3 Segmentation Rules}
\RosAuthor{J. Rous}
\RosDocNr{265}
\RosDate{900124}
\RosStatus{concept}
\RosSupersedes{-}
\RosDistribution{Project}
\RosClearance{Project}
\RosKeywords{Morphology, segmentation rules, syntax, semantics}
\MakeRosTitle
\section{Introduction}
This document contains a definition of the syntax and semantics of the
segmentation rules.
It is strongly recommended to read document nr. 210
before an attempt is made to read this document.

On a global level the syntax of the segmentation rules is the same for all
the different kind of rules. In section 2, 3 and 4 a description can be
found of the global syntax definitions. Section 4.1 to 4.5 give a detailed
specification of the syntax and semantics of each rule type.


The definitions in this document contain quantifications over variables. In
each of these definitions a specification of the domain of the quantified
variable has been omitted. Instead the a naming convention has been used.
Variables $\alpha_{1}$, $b$ are of type string. Variable $\kappa$
is either a prefix key or a suffix key and variable $m$ is a natural number.

\section{General}
The formalism prescribes 5 different segmentation rule types, namely,
left, right and middle glue rules and left (prefix) and right (suffix)
segmentation rules.

A complete specification of all rules of one type consists of three parts,
one part in which `variables' are declared, another part in which `alias' names
can be defined for the variables and a third part in which
all rules are specified.

\small
\begin{code}
{\bf rule specification} :: [ `VAR' {\bf variable definition part} ] \\
\>\>\>\>[ `ALIAS' {\bf alias definition part } ] \\
\>\>\>\>`TABLE' {\bf rule definition part } `END' \\
\end{code}
\normalsize

\section{Variable and Alias definition part}

The syntax of the variable definition part and the alias definition part is
the same for all
rule types. A variable definition introduces a variable name and the set of
( string ) values that the variable can take ( the domain of the variable ).

\small
\begin{code}
{\bf variable definition part} :: \{ {\bf variable name} `=' `[' {\bf variable value } \\
\>\>\>\>\{ `,' {\bf variable value} \} [ `,' ] `]' `;' \}\\
{\bf variable value} :: {\bf single string} $|$ ( {\bf left string} $<>$ {\bf right string} )\\
{\bf left string} :: {\bf string} \\
{\bf right string} :: {\bf string} \\
{\bf single string} :: {\bf string} \\
\end{code}
\normalsize

\small
\begin{code}
Example: \\
\>\>Vowels = [ a, o, u, e, i, a$<>$A, o$<>$O, u$<>$U, e$<>$E, i$<>$I ]; \\
\end{code}
\normalsize

The meaning of left string, right string and single string will be explained in
the next section. A single string is in fact an abbreviation for
a left and right string with the same value. For example, `ee' is
interpreted as `ee$<>$ee'. The empty string is also a possible variable value.
In case the empty string is one of the possible variable values the closing
bracket `]' of the variable definition must be preceded by a comma `,'.

An alias definition introduces a set of alternative
names for a specified variable.
These names can be used instead of the variable name. The alias definition
part has been introduced in order to increase the readability of the
segmentation rules.

\small
\begin{code}
{\bf alias definition part} :: \{ {\bf variable name} `=' {\bf string} `,' \{ {\bf string} \} `;' \}\\
\end{code}
\normalsize

\small
\begin{code}
Example: \\
\>\>Vowels = LeftVowel, RightVowel;
\end{code}
\normalsize


\section{Rule definition part}

On a global level all rules in the rule definition part of each type are of
the same structure, that is:\\
\small
\begin{code}
{\bf rule definition part } :: {\bf rule definition} \{ {\bf rule definition} \} \\
{\bf rule definition } :: {\bf lefthand side} `::' {\bf righthand side} `;'\\
\end{code}
\normalsize

The specification of the {\bf left hand side} ( to be called {\bf lhs} ) and
{\bf right hand side } ( to be called {\bf rhs} ) will be different for each rule
type. In the following sections a more detailed specification will be given of
the {\bf rule definition } for each type of rule.

All rule types have in
common that the {\bf lhs}-part contains {\em at least} one and the {\bf rhs}-part contains
{\em exactly } one so-called M-string ( the M- is an abbreviation for match- ).
An M-string is a (possibly empty) string consisting of non-blank characters and `variables'.
A `variable' in an M-string is specified by the variable name enclosed by an
open `(' and closing `)' bracket. The string {\em a(K1)bc } is an example of
an M-string in which the variable {\em `K1'} is used.
Each variable
which appears in the {\bf lhs} of a rule must also appear in the {\bf rhs} of a rule
and vice versa.

During a rule application the input-string
is matched with the M-strings of a rule. If the match is successful
for all M-strings in the rule the rule application will be successful.
During the process of matching, all variables in each M-string of the rule
( both the {\bf lhs} and {\bf rhs} M-strings ) are instantiated with each value of
their domain. More formally, the semantics of a rule application can be defined
by means of the function $\Phi$:
\small
\begin{code}
$\Phi(\alpha) =_{def} \{ r | \exists v_{1}, \ldots, v_{n}, d_{1}, \ldots, d_{n}:$\\
\>\>\>$\forall i: 1 \leq i \leq n: v_{i} \in VARS(\Phi) \wedge d_{i}\in D^{v_{i}} \wedge r \in R(\alpha,d_{1}, \ldots ,d_{n}) \}$\\
\end{code}
\normalsize
Here, parameter $\alpha$ contains the input string that has to be matched. The
type of the result $r$ differs for the different
type of rules. Also the analytical and generative interpretation of the rules
will require a different type for $r$ as we will see in the next
sections. The values $d_{i}$ of domain $D^{v_{i}}$ of variable $v_{i}$ are in
fact left and right
string pairs $<l_{i}, r_{i}>$. The actual rule is defined as the function $R$
which substitutes value $r_{i}$ for
each occurence of the variable $v_{i}$ in the {\bf rhs} M-string and $l_{i}$ for
each occurence of the variable $v_{i}$ in the {\bf lhs} M-strings of the rule
before it starts the matching process. Each rule type has its own
definition of the function $R$. These definitions will be given
in the next sections.

It is obvious from the preceding definition
that a segmentation or glue rule in fact defines a set of rules.
The cardinality of this rule set depends on the size of the domains of the
variables, as follows:
\small
\begin{code}
\>$\prod_{i=1}^{n} \#D^{v_{i}}$\\
\end{code}
\normalsize
This formula expresses the threat of a combinatorial
explosion, e.g. a rule with 3 different variables, each having a domain
of 10 elements in fact defines 10$^{3}$ different rules.
In the section on implementation of these rules I will describe how this danger
is averted.
\subsection{Left segmentation rules}
A left segmentation rule has one of the next two forms:
\small
\begin{code}
{\bf rule definition } :: {\bf perfect match} $|$ {\bf left match }\\
{\bf perfect match } :: {\bf PF-K} `+' {\bf q}  `::' {\bf p} [`,' {\bf F}] `;'\\
{\bf left match } :: {\bf PF-K} `+' {\bf q}`*'  `::' {\bf p}`*' [`,' {\bf F}] `;'
\end{code}
\normalsize
Here, PF-K is a prefix key, $p$ and $q$ are M-strings and $F$ is the
identification of a phonetical rule.

\small
\begin{code}
Example: \\
\>VAR\\
\>\>A = [a, b, c, d, f, g, h, j, k, l, m, n, o, p, q, r, s, t, v, w, x, y, z];\\
\>TABLE\\
\>\>PFKge + (A)* :: ge(A)*;\\
\>\>PFKge + e* :: ge\"{e}*;\\
\end{code}
\normalsize

The difference between the left match and the perfect match ruletype is that
the first ruletype requires
an exact match between the M-string $p$ and the input string \footnote{in
generation the match will be with $q$}, whereas the second type requires that
$p$ matches with the left part of the input string.

In the previous section
we have seen that the semantics of the rule can be defined by means of the
function $\Phi$. I will define now the function $R$ which was used for the
definition of $\Phi$ for each of the two forms of a left segmentation rule
and also for the analytical and generative version of each form.
The semantics associated with the analytical version of the
{\bf left match} rule is:
\small
\begin{code}
$R^{A}_{l}(\alpha,<l_{1},r_{1}>, \ldots ,<l_{n},r_{n}>) =_{def}$\\
\>$\{ r | \exists \alpha_{1}, b : \alpha = \alpha_{1}\bullet b \wedge \alpha_{1} \equiv p\{r_{1}/v_{1},\ldots,r_{n}/v_{n}\} \wedge$\
\
\>$\neg( b = \epsilon \wedge p = \epsilon ) \wedge r = <PFK, F, q\bullet b\{l{1}/v_{1},\ldots,l_{n}/v_{n}\}> \}$\\
\end{code}
\normalsize
In this definition $p$, $q$, $PFK$ and $F$ are rule-defined
constants. They correspond with $p$, $q$, $PFK$ and $F$ in the
syntax definition.
The notation $p\{r_{1}/v_{1},\ldots,r_{n}/v_{n}\}$ means "the string $p$
in which for each occurence of the variable $v_{i}$ the value $r_{i}$ has been
substituted. The `$\bullet$' symbol is used to express a concatenation operation between
two strings.

The semantics associated with the analytical version of the
 perfect match rule ( indicated by subscript $p$ ) is:
\small
\begin{code}
$R^{A}_{p}(\alpha,<l_{1},r_{1}>, \ldots ,<l_{n},r_{n}>) =_{def}$\\
\>$\{ r | \alpha \equiv p\{r_{1}/v_{1},\ldots,r_{n}/v_{n}\} \wedge$\\
\>$r = <PFK, F, \{l_{1}/v_{1},\ldots,l_{n}/v_{n}\}> \}$\\
\end{code}
\normalsize
In generation the semantics of the two rule forms is as follows:
\small
\begin{code}
$R^{G}_{l}(\alpha,<l_{1},r_{1}>, \ldots ,<l_{n},r_{n}>) =_{def}$\\
\>$\{ r | \exists \alpha_{1}, b, \kappa : \alpha = <\kappa ,\alpha
_{1}\bullet b> \wedge \alpha_{1} \equiv q\{l_{1}/v_{1},\ldots,l_{n}/v_{n}\} \wedge$\\
\>$\neg( b = \epsilon \wedge q = \epsilon ) \wedge \kappa = PFK \wedge r = <F, p\bullet b\{r_{1}/v_{1},\ldots,r_{n}/v_{n}\}> \}$\\
\end{code}
\normalsize
\small
\begin{code}
$R^{G}_{p}(\alpha,<l_{1},r_{1}>, \ldots ,<l_{n},r_{n}>) =_{def}$\\
\>$\{ r | \exists \alpha_{1}, \kappa : \alpha = <\kappa ,\alpha_{1} \wedge \alpha_{1} \equiv q\{l_{1}/v_{1},\ldots,l_{n}/v_{n}\} \wedge$\\
\>$\kappa = PFK \wedge r = <F, p\{r_{1}/v_{1},\ldots,r_{n}/v_{n}\}> \}$\\
\end{code}
\normalsize
\subsection{Right segmentation rules}
A right segmentation rule has one of the next two forms:
\small
\begin{code}
{\bf rule definition } :: {\bf perfect match} $|$ {\bf right match} \\
{\bf perfect match } :: {\bf q} `+' {\bf SF-K} `::' {\bf p} [`,' {\bf F}] [, {\bf CC}] `;'\\
{\bf right match } :: `*'{\bf q} `+' {\bf SF-K} `::' `*'{\bf p} [`,' {\bf F}] [, {\bf CC}] `;' \\
\end{code}
\normalsize
Here, SF-K is a suffix key, $p$ and $q$ are M-strings, F is the
identification of a phonetical rule and CC is the identification of a
context condition.

The difference between these two kind of rules is of the same type as the
difference between the two kinds of left segmentation rules.

\small
\begin{code}
Example: \\
\>VAR\\
\>\>C = [b, c, d, f, g, h, j, k, l, m, n, p, q, r, s, t, v, w, x, y, z];\\
\>\>V = [a, e, i, o, u];\\
\>TABLE\\
\>\>*(C)(V)b + SFKje :: *(C)(V)bbetje, FONleegsjwa;\\
\end{code}
\normalsize

The definition of the analytical right match and perfect match
semantics are:
\small
\begin{code}
$R^{A}_{r}(\alpha,<l_{1},r_{1}>, \ldots ,<l_{n},r_{n}>) =_{def}$\\
\>$\{ r | \exists \alpha_{1}, b : \alpha = b\bullet\alpha_{1} \wedge \alpha_{1} \equiv p\{r_{1}/v_{1},\ldots,r_{n}/v_{n}\} \wedge$\\
\>$\neg( b = \epsilon \wedge p = \epsilon ) \wedge r = <SFK, F, CC, b\bullet q\{l_{1}/v_{1},\ldots,l_{n}/v_{n}\}> \}$\\
\end{code}
\normalsize
\small
\begin{code}
$R^{A}_{p}(\alpha,<l_{1},r_{1}>, \ldots ,<l_{n},r_{n}>) =_{def}$\\
\>$\{ r | \alpha \equiv p\{r_{1}/v_{1},\ldots,r_{n}/v_{n}\} \wedge$\\
\>$r = <SFK, F, CC, q\{l_{1}/v_{1},\ldots,l_{n}/v_{n}\}> \}$\\
\end{code}
\normalsize
The definition of the generative right match and perfect match
semantics are:
\small
\begin{code}
$R^{G}_{r}(\alpha,<l_{1},r_{1}>, \ldots ,<l_{n},r_{n}>) =_{def}$\\
\>$\{ r | \exists \alpha_{1}, b, \kappa : \alpha = <\kappa , b\bullet\alpha_{1}> \wedge \alpha_{1} \equiv q\{l_{1}/v_{1},\ldots,l_{n
}/v_{n}\} \wedge$\\
\>$\neg( b = \epsilon \wedge p = \epsilon ) \wedge \kappa = SFK \wedge r = <F, CC, b\bullet p\{r_{1}/v_{1},\ldots,r_{n}/v_{n}\}> \}$
\\
\end{code}
\normalsize
\small
\begin{code}
$R^{G}_{p}(\alpha,<l_{1},r_{1}>, \ldots ,<l_{n},r_{n}>) =_{def}$\\
\>$\{ r |\exists \alpha_{1}, \kappa : \alpha = <\kappa , alpha_{1}> \wedge \alpha_{1} \equiv q\{l_{1}/v_{1},\ldots,l_{n}/v_{n}\} \wedge$\\
\>$\kappa = SFK \wedge r = <F, CC, p\{r_{1}/v_{1},\ldots,r_{n}/v_{n}\}> \}$\\
\end{code}
\normalsize
\subsection{Right glue rules }
A right glue rule has the following form:
\small
\begin{code}
{\bf rule definition } :: `*'{\bf q$_{1}$} \{ `+' {\bf q$_{i}$ \}$_{i=2}^{M}$} `::' `*'{\bf p} `;' \\
\end{code}
\normalsize
Here, $p$ and $q_{1},\ldots,q_{M}$ are M-strings. Notice that one cannot
express the requirement for a perfect match with right glue rules.

\small
\begin{code}
Example: \\
\>TABLE\\
\>\>* + will  :: *'ll;\\
\>\>* + had   :: *'d;\\
\>\>* + would :: *'d;\\
\end{code}
\normalsize

The semantics of
the analytical and generative rule-application are:
\small
\begin{code}
$R^{A}(\alpha,<l_{1},r_{1}>, \ldots ,<l_{n},r_{n}>) =_{def}$\\
\>$\{ r | \exists \alpha_{1}, b : \alpha = b\bullet\alpha_{1} \wedge \alpha_{1} \equiv p\{r_{1}/v_{1},\ldots,r_{n}/v_{n}\} \wedge$\\
\>$r = <M, <b\bullet q_{1},\ldots,q_{M}>\{l_{1}/v_{1},\ldots,l_{n}/v_{n}\}> \}$\\
\end{code}
\normalsize
\small
\begin{code}
$R^{G}(\alpha,<l_{1},r_{1}>, \ldots ,<l_{n},r_{n}>) =_{def}$\\
\>$\{ r | \exists \alpha_{1}, b, m : \alpha = <m, b\bullet\alpha_{1}> \wedge $\\
\>$\alpha_{1} \equiv q_{1}\bullet"|"\bullet\ldots\bullet "|"\bullet q_{M}\{l_{1}/v_{1},\ldots,l_{n}/v_{n}\} \wedge$\\
\>$m = M \wedge r = b\bullet p\{r_{1}/v_{1},\ldots,r_{n}/v_{n}\} \}$\\
\end{code}
\normalsize
For implementation reasons the input parameter $\alpha$ of the generative version of
the function $R$ is not a tuple of strings. Instead, the strings have
been concatenated to one big string in which the individual strings are
separated by the `$|$' character.  In order to perform the matching process
the M-strings in the lefthand side of the rule must be concatenated in the
same way.

\subsection{Left glue rules }
A left glue rule has the following form:
\small
\begin{code}
{\bf rule definition } :: \{ {\bf q$_{i}$} `+' \}$_{i=1}^{M-1}$ {\bf q$_{M}$}`*' `::' {\bf p}`*' `;' \\
\end{code}
\normalsize
Here, $p$ and $q_{1},\ldots,q_{M}$ are M-strings. Notice that one cannot
express the requirement for a perfect match with left glue rules.

\small
\begin{code}
Example: \\
\>TABLE\\
\>\>tegen     + * :: tegen*;\\
\>\>tegenover + * :: tegenover*;\\
\>\>tekeer    + * :: tekeer*;\\
\end{code}
\normalsize

The semantics of the analytical and generative rule-application are:
\small
\begin{code}
$R^{A}(\alpha,<l_{1},r_{1}>, \ldots ,<l_{n},r_{n}>) =_{def}$\\
\>$\{ r | \exists \alpha_{1}, b : \alpha = \alpha_{1}\bullet b \wedge \alpha_{1} \equiv p\{r_{1}/v_{1},\ldots,r_{n}/v_{n}\} \wedge$\
\
\>$r = <M, <q_{1},\ldots,q_{M}\bullet b>\{l_{1}/v_{1},\ldots,l_{n}/v_{n}\}> \}$\\
\end{code}
\normalsize
\small
\begin{code}
$R^{G}(\alpha,d_{1}, \ldots ,d_{n}) =_{def}$\\
\>$\{ r | \exists \alpha_{1}, b, m : \alpha = <m, \alpha_{1}\bullet b> \wedge $\\
\>$\alpha_{1} \equiv q_{1}\bullet"|"\bullet\ldots\bullet "|"\bullet q_{M}\{l_{1}/v_{1},\ldots,l_{n}/v_{n}\} \wedge$\\
\>$m = M \wedge r = p\bullet b\{r_{1}/v_{1},\ldots,r_{n}/v_{n}\} \}$\\
\end{code}
\normalsize
\subsection{Middle glue rules }
A middle glue rule has the following form:
\small
\begin{code}
{\bf rule definition } :: \{ {\bf q$_{i}$} `+' \}$_{i=1}^{M-1}$ {\bf q$_{M}$} `::' {\bf p} `;' \\
\end{code}
\normalsize
Here, $p$ and $q_{1},\ldots,q_{M}$ are M-strings. Notice that one can only
express the requirement for a perfect match with these rules.

\small
\begin{code}
Example: \\
\>VAR
\>\>A = [se,te,me,le,lo,la,os,nos,les,los,las];\\
\>\>B = [se,te,me,os,nos];\\
\>\>C = [te,me,le,lo,la,os,nos,les,los,las];\\
\>TABLE\\
\>\>ten + (A) :: ten(A);\\
\>\>se  + (B) + (C) :: se(B)(C);\\
\end{code}
\normalsize

The semantics of the analytical and generative rule-application are:
\small
\begin{code}
$R^{A}(\alpha,<l_{1},r_{1}>, \ldots ,<l_{n},r_{n}>) =_{def}$\\
\>$\{ r | \alpha \equiv p\{r_{1}/v_{1},\ldots,r_{n}/v_{n}\} \wedge$\\
\>$r = <M, <q_{1},\ldots,q_{M}>\{l_{1}/v_{1},\ldots,l_{n}/v_{n}\}> \}$\\
\end{code}
\normalsize
\small
\begin{code}
$R^{G}(\alpha,<l_{1},r_{1}>, \ldots ,<l_{n},r_{n}>) =_{def}$\\
\>$\{ r | \exists m : \alpha = <m, \alpha_{1}> \wedge$\\
\>$\alpha_{1} \equiv q_{1}\bullet"|"\bullet\ldots\bullet "|"\bullet q_{M}\{l_{1}/v_{1},\ldots,l_{n}/v_{n}\} \wedge$\\
\>$m= M \wedge r = p\{r_{1}/v_{1},\ldots,r_{n}/v_{n}\} \}$\\
\end{code}
\normalsize

\end{document}
ROSETTA.sty
\typeout{Document Style 'Rosetta'. Version 0.4 - released  24-DEC-1987}
% 24-DEC-1987:  Date of copyright notice changed
\def\@ptsize{1}
\@namedef{ds@10pt}{\def\@ptsize{0}}
\@namedef{ds@12pt}{\def\@ptsize{2}}
\@twosidetrue
\@mparswitchtrue
\def\ds@draft{\overfullrule 5pt}
\@options
\input art1\@ptsize.sty\relax


\def\labelenumi{\arabic{enumi}.}
\def\theenumi{\arabic{enumi}}
\def\labelenumii{(\alph{enumii})}
\def\theenumii{\alph{enumii}}
\def\p@enumii{\theenumi}
\def\labelenumiii{\roman{enumiii}.}
\def\theenumiii{\roman{enumiii}}
\def\p@enumiii{\theenumi(\theenumii)}
\def\labelenumiv{\Alph{enumiv}.}
\def\theenumiv{\Alph{enumiv}}
\def\p@enumiv{\p@enumiii\theenumiii}
\def\labelitemi{$\bullet$}
\def\labelitemii{\bf --}
\def\labelitemiii{$\ast$}
\def\labelitemiv{$\cdot$}
\def\verse{
   \let\\=\@centercr
   \list{}{\itemsep\z@ \itemindent -1.5em\listparindent \itemindent
      \rightmargin\leftmargin\advance\leftmargin 1.5em}
   \item[]}
\let\endverse\endlist
\def\quotation{
   \list{}{\listparindent 1.5em
      \itemindent\listparindent
      \rightmargin\leftmargin \parsep 0pt plus 1pt}\item[]}
\let\endquotation=\endlist
\def\quote{
   \list{}{\rightmargin\leftmargin}\item[]}
\let\endquote=\endlist
\def\descriptionlabel#1{\hspace\labelsep \bf #1}
\def\description{
   \list{}{\labelwidth\z@ \itemindent-\leftmargin
      \let\makelabel\descriptionlabel}}
\let\enddescription\endlist


\def\@begintheorem#1#2{\it \trivlist \item[\hskip \labelsep{\bf #1\ #2}]}
\def\@endtheorem{\endtrivlist}
\def\theequation{\arabic{equation}}
\def\titlepage{
   \@restonecolfalse
   \if@twocolumn\@restonecoltrue\onecolumn
   \else \newpage
   \fi
   \thispagestyle{empty}\c@page\z@}
\def\endtitlepage{\if@restonecol\twocolumn \else \newpage \fi}
\arraycolsep 5pt \tabcolsep 6pt \arrayrulewidth .4pt \doublerulesep 2pt
\tabbingsep \labelsep
\skip\@mpfootins = \skip\footins
\fboxsep = 3pt \fboxrule = .4pt


\newcounter{part}
\newcounter {section}
\newcounter {subsection}[section]
\newcounter {subsubsection}[subsection]
\newcounter {paragraph}[subsubsection]
\newcounter {subparagraph}[paragraph]
\def\thepart{\Roman{part}} \def\thesection {\arabic{section}}
\def\thesubsection {\thesection.\arabic{subsection}}
\def\thesubsubsection {\thesubsection .\arabic{subsubsection}}
\def\theparagraph {\thesubsubsection.\arabic{paragraph}}
\def\thesubparagraph {\theparagraph.\arabic{subparagraph}}


\def\@pnumwidth{1.55em}
\def\@tocrmarg {2.55em}
\def\@dotsep{4.5}
\setcounter{tocdepth}{3}
\def\tableofcontents{\section*{Contents\markboth{}{}}
\@starttoc{toc}}
\def\l@part#1#2{
   \addpenalty{-\@highpenalty}
   \addvspace{2.25em plus 1pt}
   \begingroup
      \@tempdima 3em \parindent \z@ \rightskip \@pnumwidth \parfillskip
      -\@pnumwidth {\large \bf \leavevmode #1\hfil \hbox to\@pnumwidth{\hss #2}}
      \par \nobreak
   \endgroup}
\def\l@section#1#2{
   \addpenalty{-\@highpenalty}
   \addvspace{1.0em plus 1pt}
   \@tempdima 1.5em
   \begingroup
      \parindent \z@ \rightskip \@pnumwidth
      \parfillskip -\@pnumwidth
      \bf \leavevmode #1\hfil \hbox to\@pnumwidth{\hss #2}
      \par
   \endgroup}
\def\l@subsection{\@dottedtocline{2}{1.5em}{2.3em}}
\def\l@subsubsection{\@dottedtocline{3}{3.8em}{3.2em}}
\def\l@paragraph{\@dottedtocline{4}{7.0em}{4.1em}}
\def\l@subparagraph{\@dottedtocline{5}{10em}{5em}}
\def\listoffigures{
   \section*{List of Figures\markboth{}{}}
   \@starttoc{lof}}
   \def\l@figure{\@dottedtocline{1}{1.5em}{2.3em}}
   \def\listoftables{\section*{List of Tables\markboth{}{}}
   \@starttoc{lot}}
\let\l@table\l@figure


\def\thebibliography#1{
   \addcontentsline{toc}
   {section}{References}\section*{References\markboth{}{}}
   \list{[\arabic{enumi}]}
        {\settowidth\labelwidth{[#1]}\leftmargin\labelwidth
         \advance\leftmargin\labelsep\usecounter{enumi}}}
\let\endthebibliography=\endlist


\newif\if@restonecol
\def\theindex{
   \@restonecoltrue\if@twocolumn\@restonecolfalse\fi
   \columnseprule \z@
   \columnsep 35pt\twocolumn[\section*{Index}]
   \markboth{}{}
   \thispagestyle{plain}\parindent\z@
   \parskip\z@ plus .3pt\relax
   \let\item\@idxitem}
\def\@idxitem{\par\hangindent 40pt}
\def\subitem{\par\hangindent 40pt \hspace*{20pt}}
\def\subsubitem{\par\hangindent 40pt \hspace*{30pt}}
\def\endtheindex{\if@restonecol\onecolumn\else\clearpage\fi}
\def\indexspace{\par \vskip 10pt plus 5pt minus 3pt\relax}


\def\footnoterule{
   \kern-1\p@
   \hrule width .4\columnwidth
   \kern .6\p@}
\long\def\@makefntext#1{
   \@setpar{\@@par\@tempdima \hsize
   \advance\@tempdima-10pt\parshape \@ne 10pt \@tempdima}\par
   \parindent 1em\noindent \hbox to \z@{\hss$^{\@thefnmark}$}#1}


\setcounter{topnumber}{2}
\def\topfraction{.7}
\setcounter{bottomnumber}{1}
\def\bottomfraction{.3}
\setcounter{totalnumber}{3}
\def\textfraction{.2}
\def\floatpagefraction{.5}
\setcounter{dbltopnumber}{2}
\def\dbltopfraction{.7}
\def\dblfloatpagefraction{.5}
\long\def\@makecaption#1#2{
   \vskip 10pt
   \setbox\@tempboxa\hbox{#1: #2}
   \ifdim \wd\@tempboxa >\hsize \unhbox\@tempboxa\par
   \else \hbox to\hsize{\hfil\box\@tempboxa\hfil}
   \fi}
\newcounter{figure}
\def\thefigure{\@arabic\c@figure}
\def\fps@figure{tbp}
\def\ftype@figure{1}
\def\ext@figure{lof}
\def\fnum@figure{Figure \thefigure}
\def\figure{\@float{figure}}
\let\endfigure\end@float
\@namedef{figure*}{\@dblfloat{figure}}
\@namedef{endfigure*}{\end@dblfloat}
\newcounter{table}
\def\thetable{\@arabic\c@table}
\def\fps@table{tbp}
\def\ftype@table{2}
\def\ext@table{lot}
\def\fnum@table{Table \thetable}
\def\table{\@float{table}}
\let\endtable\end@float
\@namedef{table*}{\@dblfloat{table}}
\@namedef{endtable*}{\end@dblfloat}


\def\maketitle{
   \par
   \begingroup
      \def\thefootnote{\fnsymbol{footnote}}
      \def\@makefnmark{\hbox to 0pt{$^{\@thefnmark}$\hss}}
      \if@twocolumn \twocolumn[\@maketitle]
      \else \newpage \global\@topnum\z@ \@maketitle
      \fi
      \thispagestyle{plain}
      \@thanks
   \endgroup
   \setcounter{footnote}{0}
   \let\maketitle\relax
   \let\@maketitle\relax
   \gdef\@thanks{}
   \gdef\@author{}
   \gdef\@title{}
   \let\thanks\relax}
\def\@maketitle{
   \newpage
   \null
   \vskip 2em
   \begin{center}{\LARGE \@title \par}
      \vskip 1.5em
      {\large \lineskip .5em \begin{tabular}[t]{c}\@author \end{tabular}\par}
      \vskip 1em {\large \@date}
   \end{center}
   \par
   \vskip 1.5em}
\def\abstract{
   \if@twocolumn \section*{Abstract}
   \else
      \small
      \begin{center} {\bf Abstract\vspace{-.5em}\vspace{0pt}} \end{center}
      \quotation
   \fi}
\def\endabstract{\if@twocolumn\else\endquotation\fi}


\mark{{}{}}
\if@twoside
   \def\ps@headings{
      \def\@oddfoot{Rosetta Doc. \@RosDocNr\hfil \@RosDate}
      \def\@evenfoot{Rosetta Doc. \@RosDocNr\hfil \@RosDate}
      \def\@evenhead{\rm\thepage\hfil \sl \rightmark}
      \def\@oddhead{\hbox{}\sl \leftmark \hfil\rm\thepage}
      \def\sectionmark##1{\markboth {}{}}
      \def\subsectionmark##1{}}
\else
   \def\ps@headings{
      \def\@oddfoot{Rosetta Doc. \@RosDocNr\hfil \@RosDate}
      \def\@evenfoot{Rosetta Doc. \@RosDocNr\hfil \@RosDate}
      \def\@oddhead{\hbox{}\sl \rightmark \hfil \rm\thepage}
      \def\sectionmark##1{\markboth {}{}}
      \def\subsectionmark##1{}}
\fi
\def\ps@myheadings{
   \def\@oddhead{\hbox{}\sl\@rhead \hfil \rm\thepage}
   \def\@oddfoot{}
   \def\@evenhead{\rm \thepage\hfil\sl\@lhead\hbox{}}
   \def\@evenfoot{}
   \def\sectionmark##1{}
   \def\subsectionmark##1{}}


\def\today{
   \ifcase\month\or January\or February\or March\or April\or May\or June\or
      July\or August\or September\or October\or November\or December
   \fi
   \space\number\day, \number\year}


\ps@plain \pagenumbering{arabic} \onecolumn \if@twoside\else\raggedbottom\fi




% the Rosetta title page
\newcommand{\MakeRosTitle}{
   \begin{titlepage}
      \begin{large}
     \begin{figure}[t]
        \begin{picture}(405,100)(0,0)
           \put(0,100){\line(1,0){404}}
           \put(0,75){Project {\bf Rosetta}}
           \put(93.5,75){:}
           \put(108,75){Machine Translation}
           \put(0,50){Topic}
           \put(93.5,50){:}
           \put(108,50){\@RosTopic}
           \put(0,30){\line(1,0){404}}
        \end{picture}
     \end{figure}
     \bigskip
     \bigskip
     \begin{list}{-}{\setlength{\leftmargin}{3.0cm}
             \setlength{\labelwidth}{2.7cm}
             \setlength{\topsep}{2cm}}
        \item [{\rm Title \hfill :}] {{\bf \@RosTitle}}
        \item [{\rm Author \hfill :}] {\@RosAuthor}
        \bigskip
        \bigskip
        \bigskip
        \item [{\rm Doc.Nr. \hfill :}] {\@RosDocNr}
        \item [{\rm Date \hfill :}] {\@RosDate}
        \item [{\rm Status \hfill :}] {\@RosStatus}
        \item [{\rm Supersedes \hfill :}] {\@RosSupersedes}
        \item [{\rm Distribution \hfill :}] {\@RosDistribution}
        \item [{\rm Clearance \hfill :}] {\@RosClearance}
        \item [{\rm Keywords \hfill :}] {\@RosKeywords}
     \end{list}
      \end{large}
      \title{\@RosTitle}
      \begin{figure}[b]
     \begin{picture}(404,64)(0,0)
        \put(0,64){\line(1,0){404}}
        \put(0,-4){\line(1,0){404}}
        \put(0,59){\line(1,0){42}}
        \begin{small}
        \put(3,48){\sf PHILIPS}
        \end{small}
        \put(0,23){\line(0,1){36}}
        \put(42,23){\line(0,1){36}}
        \put(21,23){\oval(42,42)[bl]}
        \put(21,23){\oval(42,42)[br]}
        \put(21,23){\circle{40}}
        \put(4,33){\line(1,0){10}}
        \put(9,28){\line(0,1){10}}
        \put(9,36){\line(1,0){6}}
        \put(12,33){\line(0,1){6}}
        \put(29,13){\line(1,0){10}}
        \put(34,8){\line(0,1){10}}
        \put(28,10){\line(1,0){6}}
        \put(31,7){\line(0,1){6}}

        \put(1,21){\line(1,0){0.5}}
        \put(1.5,21.3){\line(1,0){0.5}}
        \put(2,21.6){\line(1,0){0.5}}
        \put(2.5,21.9){\line(1,0){0.5}}
        \put(3,22.1){\line(1,0){0.5}}
        \put(3.5,22.3){\line(1,0){0.5}}
        \put(4,22.5){\line(1,0){0.5}}
        \put(4.5,22.7){\line(1,0){0.5}}
        \put(5,22.8){\line(1,0){0.5}}
        \put(5.5,22.9){\line(1,0){0.5}}
        \put(6,23){\line(1,0){0.5}}
        \put(6.5,22.9){\line(1,0){0.5}}
        \put(7,22.8){\line(1,0){0.5}}
        \put(7.5,22.7){\line(1,0){0.5}}
        \put(8,22.5){\line(1,0){0.5}}
        \put(8.5,22.3){\line(1,0){0.5}}
        \put(9,22.1){\line(1,0){0.5}}
        \put(9.5,21.9){\line(1,0){0.5}}
        \put(10,21.6){\line(1,0){0.5}}
        \put(10.5,21.3){\line(1,0){0.5}}

        \put(1,23){\line(1,0){0.5}}
        \put(1.5,23.3){\line(1,0){0.5}}
        \put(2,23.6){\line(1,0){0.5}}
        \put(2.5,23.9){\line(1,0){0.5}}
        \put(3,24.1){\line(1,0){0.5}}
        \put(3.5,24.3){\line(1,0){0.5}}
        \put(4,24.5){\line(1,0){0.5}}
        \put(4.5,24.7){\line(1,0){0.5}}
        \put(5,24.8){\line(1,0){0.5}}
        \put(5.5,24.9){\line(1,0){0.5}}
        \put(6,25){\line(1,0){0.5}}
        \put(6.5,24.9){\line(1,0){0.5}}
        \put(7,24.8){\line(1,0){0.5}}
        \put(7.5,24.7){\line(1,0){0.5}}
        \put(8,24.5){\line(1,0){0.5}}
        \put(8.5,24.3){\line(1,0){0.5}}
        \put(9,24.1){\line(1,0){0.5}}
        \put(9.5,23.9){\line(1,0){0.5}}
        \put(10,23.6){\line(1,0){0.5}}
        \put(10.5,23.3){\line(1,0){0.5}}

        \put(1,25){\line(1,0){0.5}}
        \put(1.5,25.3){\line(1,0){0.5}}
        \put(2,25.6){\line(1,0){0.5}}
        \put(2.5,25.9){\line(1,0){0.5}}
        \put(3,26.1){\line(1,0){0.5}}
        \put(3.5,26.3){\line(1,0){0.5}}
        \put(4,26.5){\line(1,0){0.5}}
        \put(4.5,26.7){\line(1,0){0.5}}
        \put(5,26.8){\line(1,0){0.5}}
        \put(5.5,26.9){\line(1,0){0.5}}
        \put(6,27){\line(1,0){0.5}}
        \put(6.5,26.9){\line(1,0){0.5}}
        \put(7,26.8){\line(1,0){0.5}}
        \put(7.5,26.7){\line(1,0){0.5}}
        \put(8,26.5){\line(1,0){0.5}}
        \put(8.5,26.3){\line(1,0){0.5}}
        \put(9,26.1){\line(1,0){0.5}}
        \put(9.5,25.9){\line(1,0){0.5}}
        \put(10,25.6){\line(1,0){0.5}}
        \put(10.5,25.3){\line(1,0){0.5}}

        \put(11,21){\line(1,0){0.5}}
        \put(11.5,20.7){\line(1,0){0.5}}
        \put(12,20.4){\line(1,0){0.5}}
        \put(12.5,20.1){\line(1,0){0.5}}
        \put(13,19.9){\line(1,0){0.5}}
        \put(13.5,19.7){\line(1,0){0.5}}
        \put(14,19.5){\line(1,0){0.5}}
        \put(14.5,19.3){\line(1,0){0.5}}
        \put(15,19.2){\line(1,0){0.5}}
        \put(15.5,19.1){\line(1,0){0.5}}
        \put(16,19){\line(1,0){0.5}}
        \put(16.5,19.1){\line(1,0){0.5}}
        \put(17,19.2){\line(1,0){0.5}}
        \put(17.5,19.3){\line(1,0){0.5}}
        \put(18,19.5){\line(1,0){0.5}}
        \put(18.5,19.7){\line(1,0){0.5}}
        \put(19,19.9){\line(1,0){0.5}}
        \put(19.5,20.1){\line(1,0){0.5}}
        \put(20,20.4){\line(1,0){0.5}}
        \put(20.5,20.7){\line(1,0){0.5}}

        \put(11,23){\line(1,0){0.5}}
        \put(11.5,22.7){\line(1,0){0.5}}
        \put(12,22.4){\line(1,0){0.5}}
        \put(12.5,22.1){\line(1,0){0.5}}
        \put(13,21.9){\line(1,0){0.5}}
        \put(13.5,21.7){\line(1,0){0.5}}
        \put(14,21.5){\line(1,0){0.5}}
        \put(14.5,21.3){\line(1,0){0.5}}
        \put(15,21.2){\line(1,0){0.5}}
        \put(15.5,21.1){\line(1,0){0.5}}
        \put(16,21){\line(1,0){0.5}}
        \put(16.5,21.1){\line(1,0){0.5}}
        \put(17,21.2){\line(1,0){0.5}}
        \put(17.5,21.3){\line(1,0){0.5}}
        \put(18,21.5){\line(1,0){0.5}}
        \put(18.5,21.7){\line(1,0){0.5}}
        \put(19,21.9){\line(1,0){0.5}}
        \put(19.5,22.1){\line(1,0){0.5}}
        \put(20,22.4){\line(1,0){0.5}}
        \put(20.5,22.7){\line(1,0){0.5}}

        \put(11,25){\line(1,0){0.5}}
        \put(11.5,24.7){\line(1,0){0.5}}
        \put(12,24.4){\line(1,0){0.5}}
        \put(12.5,24.1){\line(1,0){0.5}}
        \put(13,23.9){\line(1,0){0.5}}
        \put(13.5,23.7){\line(1,0){0.5}}
        \put(14,23.5){\line(1,0){0.5}}
        \put(14.5,23.3){\line(1,0){0.5}}
        \put(15,23.2){\line(1,0){0.5}}
        \put(15.5,23.1){\line(1,0){0.5}}
        \put(16,23){\line(1,0){0.5}}
        \put(16.5,23.1){\line(1,0){0.5}}
        \put(17,23.2){\line(1,0){0.5}}
        \put(17.5,23.3){\line(1,0){0.5}}
        \put(18,23.5){\line(1,0){0.5}}
        \put(18.5,23.7){\line(1,0){0.5}}
        \put(19,23.9){\line(1,0){0.5}}
        \put(19.5,24.1){\line(1,0){0.5}}
        \put(20,24.4){\line(1,0){0.5}}
        \put(20.5,24.7){\line(1,0){0.5}}

        \put(21,21){\line(1,0){0.5}}
        \put(21.5,21.3){\line(1,0){0.5}}
        \put(22,21.6){\line(1,0){0.5}}
        \put(22.5,21.9){\line(1,0){0.5}}
        \put(23,22.1){\line(1,0){0.5}}
        \put(23.5,22.3){\line(1,0){0.5}}
        \put(24,22.5){\line(1,0){0.5}}
        \put(24.5,22.7){\line(1,0){0.5}}
        \put(25,22.8){\line(1,0){0.5}}
        \put(25.5,23.9){\line(1,0){0.5}}
        \put(26,23){\line(1,0){0.5}}
        \put(26.5,22.9){\line(1,0){0.5}}
        \put(27,22.8){\line(1,0){0.5}}
        \put(27.5,22.7){\line(1,0){0.5}}
        \put(28,22.5){\line(1,0){0.5}}
        \put(28.5,22.3){\line(1,0){0.5}}
        \put(29,22.1){\line(1,0){0.5}}
        \put(29.5,21.9){\line(1,0){0.5}}
        \put(30,21.6){\line(1,0){0.5}}
        \put(30.5,21.3){\line(1,0){0.5}}

        \put(21,23){\line(1,0){0.5}}
        \put(21.5,23.3){\line(1,0){0.5}}
        \put(22,23.6){\line(1,0){0.5}}
        \put(22.5,23.9){\line(1,0){0.5}}
        \put(23,24.1){\line(1,0){0.5}}
        \put(23.5,24.3){\line(1,0){0.5}}
        \put(24,24.5){\line(1,0){0.5}}
        \put(24.5,24.7){\line(1,0){0.5}}
        \put(25,24.8){\line(1,0){0.5}}
        \put(25.5,24.9){\line(1,0){0.5}}
        \put(26,25){\line(1,0){0.5}}
        \put(26.5,24.9){\line(1,0){0.5}}
        \put(27,24.8){\line(1,0){0.5}}
        \put(27.5,24.7){\line(1,0){0.5}}
        \put(28,24.5){\line(1,0){0.5}}
        \put(28.5,24.3){\line(1,0){0.5}}
        \put(29,24.1){\line(1,0){0.5}}
        \put(29.5,23.9){\line(1,0){0.5}}
        \put(30,23.6){\line(1,0){0.5}}
        \put(30.5,23.3){\line(1,0){0.5}}

        \put(21,25){\line(1,0){0.5}}
        \put(21.5,25.3){\line(1,0){0.5}}
        \put(22,25.6){\line(1,0){0.5}}
        \put(22.5,25.9){\line(1,0){0.5}}
        \put(23,26.1){\line(1,0){0.5}}
        \put(23.5,26.3){\line(1,0){0.5}}
        \put(24,26.5){\line(1,0){0.5}}
        \put(24.5,26.7){\line(1,0){0.5}}
        \put(25,26.8){\line(1,0){0.5}}
        \put(25.5,26.9){\line(1,0){0.5}}
        \put(26,27){\line(1,0){0.5}}
        \put(26.5,26.9){\line(1,0){0.5}}
        \put(27,26.8){\line(1,0){0.5}}
        \put(27.5,26.7){\line(1,0){0.5}}
        \put(28,26.5){\line(1,0){0.5}}
        \put(28.5,26.3){\line(1,0){0.5}}
        \put(29,26.1){\line(1,0){0.5}}
        \put(29.5,25.9){\line(1,0){0.5}}
        \put(30,25.6){\line(1,0){0.5}}
        \put(30.5,25.3){\line(1,0){0.5}}

        \put(31,21){\line(1,0){0.5}}
        \put(31.5,20.7){\line(1,0){0.5}}
        \put(32,20.4){\line(1,0){0.5}}
        \put(32.5,20.1){\line(1,0){0.5}}
        \put(33,19.9){\line(1,0){0.5}}
        \put(33.5,19.7){\line(1,0){0.5}}
        \put(34,19.5){\line(1,0){0.5}}
        \put(34.5,19.3){\line(1,0){0.5}}
        \put(35,19.2){\line(1,0){0.5}}
        \put(35.5,19.1){\line(1,0){0.5}}
        \put(36,19){\line(1,0){0.5}}
        \put(36.5,19.1){\line(1,0){0.5}}
        \put(37,19.2){\line(1,0){0.5}}
        \put(37.5,19.3){\line(1,0){0.5}}
        \put(38,19.5){\line(1,0){0.5}}
        \put(38.5,19.7){\line(1,0){0.5}}
        \put(39,19.9){\line(1,0){0.5}}
        \put(39.5,20.1){\line(1,0){0.5}}
        \put(40,20.4){\line(1,0){0.5}}
        \put(40.5,20.7){\line(1,0){0.5}}

        \put(31,23){\line(1,0){0.5}}
        \put(31.5,22.7){\line(1,0){0.5}}
        \put(32,22.4){\line(1,0){0.5}}
        \put(32.5,22.1){\line(1,0){0.5}}
        \put(33,21.9){\line(1,0){0.5}}
        \put(33.5,21.7){\line(1,0){0.5}}
        \put(34,21.5){\line(1,0){0.5}}
        \put(34.5,21.3){\line(1,0){0.5}}
        \put(35,21.2){\line(1,0){0.5}}
        \put(35.5,21.1){\line(1,0){0.5}}
        \put(36,21){\line(1,0){0.5}}
        \put(36.5,21.1){\line(1,0){0.5}}
        \put(37,21.2){\line(1,0){0.5}}
        \put(37.5,21.3){\line(1,0){0.5}}
        \put(38,21.5){\line(1,0){0.5}}
        \put(38.5,21.7){\line(1,0){0.5}}
        \put(39,21.9){\line(1,0){0.5}}
        \put(39.5,22.1){\line(1,0){0.5}}
        \put(40,22.4){\line(1,0){0.5}}
        \put(40.5,22.7){\line(1,0){0.5}}

        \put(31,25){\line(1,0){0.5}}
        \put(31.5,24.7){\line(1,0){0.5}}
        \put(32,24.4){\line(1,0){0.5}}
        \put(32.5,24.1){\line(1,0){0.5}}
        \put(33,23.9){\line(1,0){0.5}}
        \put(33.5,23.7){\line(1,0){0.5}}
        \put(34,23.5){\line(1,0){0.5}}
        \put(34.5,23.3){\line(1,0){0.5}}
        \put(35,23.2){\line(1,0){0.5}}
        \put(35.5,23.1){\line(1,0){0.5}}
        \put(36,23){\line(1,0){0.5}}
        \put(36.5,23.1){\line(1,0){0.5}}
        \put(37,23.2){\line(1,0){0.5}}
        \put(37.5,23.3){\line(1,0){0.5}}
        \put(38,23.5){\line(1,0){0.5}}
        \put(38.5,23.7){\line(1,0){0.5}}
        \put(39,23.9){\line(1,0){0.5}}
        \put(39.5,24.1){\line(1,0){0.5}}
        \put(40,24.4){\line(1,0){0.5}}
        \put(40.5,24.7){\line(1,0){0.5}}
        \begin{large}
           \put(60,45){Philips Research Laboratories}
           \put(60,30){\copyright\ 1988 Nederlandse Philips Bedrijven B.V.}
        \end{large}
     \end{picture}
      \end{figure}
      \newpage
      \pagenumbering{roman}
      \tableofcontents
      \newpage
      \pagenumbering{arabic}
   \end{titlepage}
}
\title{}
\topmargin 0pt
\oddsidemargin 36pt
\evensidemargin 36pt
\textheight 600pt
\textwidth 405pt
\pagestyle{headings}
\newcommand{\@RosTopic}{General}
\newcommand{\@RosTitle}{-}
\newcommand{\@RosAuthor}{-}
\newcommand{\@RosDocNr}{}
\newcommand{\@RosDate}{\today}
\newcommand{\@RosStatus}{informal}
\newcommand{\@RosSupersedes}{-}
\newcommand{\@RosDistribution}{Project}
\newcommand{\@RosClearance}{Project}
\newcommand{\@RosKeywords}{}
\newcommand{\RosTopic}[1]{\renewcommand{\@RosTopic}{#1}}
\newcommand{\RosTitle}[1]{\renewcommand{\@RosTitle}{#1}}
\newcommand{\RosAuthor}[1]{\renewcommand{\@RosAuthor}{#1}}
\newcommand{\RosDocNr}[1]{\renewcommand{\@RosDocNr}{#1 (RWR-102-RO-90#1-RO)}}
\newcommand{\RosDate}[1]{\renewcommand{\@RosDate}{#1}}
\newcommand{\RosStatus}[1]{\renewcommand{\@RosStatus}{#1}}
\newcommand{\RosSupersedes}[1]{\renewcommand{\@RosSupersedes}{#1}}
\newcommand{\RosDistribution}[1]{\renewcommand{\@RosDistribution}{#1}}
\newcommand{\RosClearance}[1]{\renewcommand{\@RosClearance}{#1}}
\newcommand{\RosKeywords}[1]{\renewcommand{\@RosKeywords}{#1}}

