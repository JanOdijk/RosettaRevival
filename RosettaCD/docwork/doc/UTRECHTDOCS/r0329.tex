\documentstyle{Rosetta}
\begin{document}
   \RosTopic{Rosetta3.doc.morphology}
   \RosTitle{Rosetta3 Dutch Morphology, inflection}
   \RosAuthor{Harm Smit}
   \RosDocNr{329}
   \RosDate{\today}
   \RosStatus{concept}
   \RosSupersedes{134}
   \RosDistribution{Project}
   \RosClearance{Project}
   \RosKeywords{Dutch, morphology, inflection, documentation}
   \MakeRosTitle


\section{Preface}

This document is a new version of document 134. It gives an impression
of the state of the art of the Dutch inflection of Rosetta at the moment. 
Another document deals with derivation, and 
separate documents gives all segmentation rules and W-rules (for inflection 
as well as derivation).

\newpage
\section{Introduction}

This document consists of several parts, which are:
\begin{itemize}
  \item domain T for Dutch (section 3),
  \item general explanation of the rules for segmentation, the GLUE-rules and 
        the W-rules (section 4),
  \item explanation of the way verbs are dealt with in Dutch morphology 
        (section 5),
  \item idem, for nouns (section 6),
  \item idem, for adjectives and adverbs (section 7),
  \item idem, for the remaining categories (section 8),
  \item final remarks (section 9) with suggestions for further refinement and
        for changes in future versions of the Dutch morphology.
\end{itemize} 


\newpage
\section{Domain T of Dutch morphology}

This section contains domain T of the Dutch morphology as far as it is
relevant for inflection: all categories and category-records, that are used
in inflection, are specified here. Also, the attributes (together with the 
possible attribute-values) and the suffix- and prefix-keys are listed here.

{\bf Categories:}

\begin{tabbing}
BVERB, \ \ \ \ \ \ \ \ \ \ \ \ \ \  \=   SUBVERB,  \ \ \ \ \ \ \ \ \ \ \ \ \ \ 
 \=   VERB,  \\ 
BNOUN,       \>   SUBNOUN,      \>   NOUN,  \\ 
BPROPERNOUN, \>   PROPERNOUN,   \>          \\
BADJ,        \>  SUBADJ,        \>   ADJ,   \\  
BADV,        \>  SUBADV,        \>   ADV,   \\
BPERSPRO,    \>  PERSPRO,       \>          \\
BDET,        \>    DET,         \>          \\ 
BINDEFPRO,   \>  INDEFPRO,      \>          \\
POSSADJ,     \>  \>  \\ 
POSSPRO,     \>  \>  \\ 
BWHPRO,      \>  WHPRO,         \>          \\
DEMPRO,      \>  \>  \\
SFCAT,       \>  \>  \\ 
PFCAT,       \>  \>  \\ 
\end{tabbing}

{\bf Keys:}

\begin{tabbing}
SFKen \ \ \ \ \ \ \ \ \ \ \ \ \ \ \ \ \  \= 
SFKt  \ \ \ \ \ \ \ \ \ \ \ \ \ \ \ \ \  \=  
SFKe  \ \ \ \ \ \ \ \ \ \ \ \ \ \ \ \ \  \= 
SFKIrrottenk0 \\
SFKIrrottenk2 \>  SFKIrrottenk3 \>  SFKIrrott4 \> SFKIrrott5 \\  
SFKIrrottmv \>  SFKIrrovtenk \>  SFKIrrovt5 \> SFKIrrovtmv \\  
SFKIrrgb \>  SFKIrrconjott \>  SFKIrrconjovt \> SFKovt1 \\  
SFKovtvd1 \>  SFKovtvd2 \>  SFKovt2 \>  SFKovt3 \\  
SFKdete \> SFKdt \>  SFKIrrvd \>  SFKvd1 \\  
SFKvd2 \> SFKmvs \>  SFKaTOaa \>  SFKaTOee \\  
SFKeTOee \>  SFKeiTOee \>  SFKiTOee \> SFKoTOoo \\  
SFKeren \>  SFKien \>  SFKden \>  SFKnen \\  
SFKieAccent \>  SFKlui \> SFKlieden \>  SFKLat \\  
SFKenIrreg \>  SFKsIrreg \>  SFKLatIrreg \>  SFKgens \\  
SFKonrege \>  SFKer \>  SFKonreger \>  SFKst \\  
SFKonregst \>  SFKadjs \> SFKsgnom \>  SFKsgnomred \\  
SFKsgaccdat \>  SFKsgaccdatred \>  SFKplnom \>  SFKplnomred \\ 
SFKplaccdat \>  SFKplaccdatred \>  SFKpldat \>  SFKplacc \\  
SFKsgpossadj \>  SFKsgpossadjred \> SFKplpossadjons \>  SFKplpossadjonze \\  
SFKplpossadj \>  SFKsgpossadjgen \> SFKplpossadjgen \>  SFKsgposs \\  
SFKplposs \>  SFKsgpossnvorm \>  SFKplpossnvorm \> SFKpossadjwiens \\  
SFKpossadjwier \>  SFKpossadjdiens \> SFKeDet \>  SFKenDet \\ 
SFKdelD \>  \>  \>  \\
\end{tabbing}

and:

\begin{tabbing}
PFKge  \ \ \ \ \ \ \ \ \ \ \  \=  PFKaller \\
\end{tabbing}

{\bf Records (inherent attributes are bold printed):}
\\
\begin{tabular}{lll}
             &                    &              \\
\end{tabular}
\\
\begin{tabular}{lll}
BVERBrecord: &                    &              \\
             & {\bf conjclasses:} & subset of ( 0, 1, 2, 3, 4, 5, 6,  \\
             &                          & 7, 8, 9, 10, 11, 12, 13, 15, 16 ) \\
             & {\bf particle:}    & key                               \\
\end{tabular}
\\
\begin{tabular}{lll}
SUBVERBrecord: &                    &              \\
               & {\bf conjclasses:} & subset of ( 0, 1, 2, 3, 4, 5, 6, \\
               &                        & 7, 8, 9, 10, 11, 12, 13, 15, 16 ) \\
               & {\bf particle:}    & key                               \\
\end{tabular}
\\
\begin{tabular}{lll}
VERBrecord:  &                 &              \\
             & {\bf conjclasses:} & subset of ( 0, 1, 2, 3, 4, 5, 6,  \\
             &                          & 7, 8, 9, 10, 11, 12, 13, 15, 16 ) \\
             & {\bf particle:}    & key                               \\
             & tense:          & omegatense, presenttense, pasttense        \\
             & modus:          & indicative, subjunctive, imperative,       \\ 
             &                 & infinitive, prespart, pastpart, omegamodus \\
             & number:                  & singular, plural, omeganumber     \\
             & persons:                 & subset of ( 0, 1, 2, 3, 4, 5 )    \\
             & eORenForm:               & NoForm, eForm, enForm             \\
\end{tabular}
\\
\begin{tabular}{lll}
BNOUNrecord: &                 &              \\
             & {\bf pluralforms:} & subset of (enPlural, sPlural, aTOaaPlural, \\
             &                 & aTOeePlural, eTOeePlural, eiTOeePlural,    \\
             &                 & iTOeePlural, oTOooPlural, erenPlural,      \\
             &                 & ienPlural, denPlural, nenPlural,           \\
             &                 & ieAccentPlural, luiPlural, liedenPlural,   \\
             &                 & LatPlural, enIrregPlural, sIrregPlural,    \\
             &                 & LatIrregPlural, NoPlural, OnlyPlural)      \\
             & {\bf possgeni:} & true, false                          \\
\end{tabular}
\\
\begin{tabular}{lll}
SUBNOUNrecord: &               &              \\
             & {\bf pluralforms:} & subset of (enPlural, sPlural, aTOaaPlural,  \\
             &                 & aTOeePlural, eTOeePlural, eiTOeePlural,    \\
             &                 & iTOeePlural, oTOooPlural, erenPlural,      \\
             &                 & ienPlural, denPlural, nenPlural,           \\
             &                 & ieAccentPlural, luiPlural, liedenPlural,   \\
             &                 & LatPlural, enIrregPlural, sIrregPlural,    \\
             &                 & LatIrregPlural, NoPlural, OnlyPlural)      \\  
             & {\bf possgeni:} & true, false                          \\
\end{tabular}
\\
\begin{tabular}{lll}
NOUNrecord:  &                 &              \\
             & {\bf pluralforms:} & subset of (enPlural, sPlural, aTOaaPlural,  \\
             &                 & aTOeePlural, eTOeePlural, eiTOeePlural,    \\
             &                 & iTOeePlural, oTOooPlural, erenPlural,      \\
             &                 & ienPlural, denPlural, nenPlural,           \\
             &                 & ieAccentPlural, luiPlural, liedenPlural,   \\
             &                 & LatPlural, enIrregPlural, sIrregPlural,    \\
             &                 & LatIrregPlural, NoPlural, OnlyPlural)      \\
             & {\bf possgeni:} & true, false                          \\
             & number:               & singular, plural, omeganumber        \\
             & geni:                 & true, false                          \\
\end{tabular}
\\
\begin{tabular}{lll}
BPROPERNOUNrecord: &           &              \\
             & {\bf pluralforms:} & subset of (enPlural, sPlural, aTOaaPlural,  \\
             &                 & aTOeePlural, eTOeePlural, eiTOeePlural,    \\
             &                 & iTOeePlural, oTOooPlural, erenPlural,      \\
             &                 & ienPlural, denPlural, nenPlural,           \\
             &                 & ieAccentPlural, luiPlural, liedenPlural,   \\
             &                 & LatPlural, enIrregPlural, sIrregPlural,    \\
             &                 & LatIrregPlural, NoPlural, OnlyPlural)      \\
             & {\bf possgeni:} & true, false                          \\
\end{tabular}
\\
\begin{tabular}{lll}
PROPERNOUNrecord: &            &              \\
             & {\bf pluralforms:} & subset of (enPlural, sPlural, aTOaaPlural,  \\
             &                 & aTOeePlural, eTOeePlural, eiTOeePlural,    \\
             &                 & iTOeePlural, oTOooPlural, erenPlural,      \\
             &                 & ienPlural, denPlural, nenPlural,           \\
             &                 & ieAccentPlural, luiPlural, liedenPlural,   \\
             &                 & LatPlural, enIrregPlural, sIrregPlural,    \\
             &                 & LatIrregPlural, NoPlural, OnlyPlural)      \\
             & {\bf possgeni:} & true, false                          \\
             & number:         & singular, plural, omeganumber        \\
             & geni:           & true, false                          \\
\end{tabular}
\\
\begin{tabular}{lll}
BADJrecord: &                     &              \\
            & {\bf uses:} & subset of (attributive, predicative, nominalised) \\
            & {\bf eFormation:}   & regEformation, irregEformation, noFormation \\
            & {\bf sFormation:}   & true, false \\
            & {\bf eNominalised:} & true, false \\
            & {\bf comparatives:} & subset of (erComp, erIrregComp, meerComp, \\
            &                           &     NoComp) \\
            & {\bf superlatives:} & subset of (stSup, stIrregSup, allerSup, \\
            &                     &  allerIrregSup, meestSup, noSup) \\
\end{tabular}
\\
\begin{tabular}{lll}
SUBADJrecord: &                   &              \\
            & {\bf uses:} & subset of (attributive, predicative, nominalised) \\
            & {\bf eFormation:}   & regEformation, irregEformation, noFormation \\
            & {\bf sFormation:}   & true, false \\
            & {\bf eNominalised:} & true, false \\
            & {\bf comparatives:} & subset of (erComp, erIrregComp, meerComp, \\
            &                     &     NoComp) \\
            & {\bf superlatives:} & subset of (stSup, stIrregSup, allerSup, \\
            &                     &  allerIrregSup, meestSup, noSup)  \\
\end{tabular}
\\
\begin{tabular}{lll}
ADJrecord: &                      &              \\
            & {\bf uses:} & subset of (attributive, predicative, nominalised) \\
            & {\bf eFormation:}   & regEformation, irregEformation, noFormation \\
            & {\bf sFormation:}   & true, false \\
            & {\bf eNominalised:} & true, false \\
            & {\bf comparatives:} & subset of (erComp, erIrregComp, meerComp, \\
            &                           &                           NoComp) \\
            & {\bf superlatives:} & subset of (stSup, stIrregSup, allerSup, \\
            &                           &  allerIrregSup, meestSup, noSup)   \\
            & form:                     & positive, sPositive, comparative,  \\
            &                           & sComparative, superlative, allerSuperlative \\
            & eORenForm:                & NoForm, eForm, enForm \\
\end{tabular}
\\
\begin{tabular}{lll}
BADVrecord: &                     &              \\
            & {\bf comparatives:} & subset of (erComp, erIrregComp, meerComp, \\
            &                     &                            NoComp) \\
            & {\bf superlatives:} & subset of (stSup, stIrregSup, allerSup, \\
            &                     & allerIrregSup, meestSup, noSup)    \\
\end{tabular}
\\
\begin{tabular}{lll}
SUBADVrecord: &                   &              \\
            & {\bf comparatives:} & subset of (erComp, erIrregComp, meerComp, \\
            &                     &                            NoComp) \\
            & {\bf superlatives:} & subset of (stSup, stIrregSup, allerSup, \\
            &                     & allerIrregSup, meestSup, noSup) \\
\end{tabular}
\\
\begin{tabular}{lll}
ADVrecord: &                      &              \\
            & {\bf comparatives:} & subset of (erComp, erIrregComp, meerComp, \\
            &                     &                         NoComp)    \\
            & {\bf superlatives:} & subset of (stSup, stIrregSup, allerSup, \\
            &                     & allerIrregSup, meestSup, noSup) \\
            & form:               & positive, sPositive, comparative, \\
            &                   & sComparative, superlative, allerSuperlative \\
            & eORenForm           & NoForm, eForm, enForm \\
\end{tabular}
\\
\begin{tabular}{lll}
BPERSPROrecord: &                 &              \\
                & {\bf number:}   & singular, plural, omeganumber   \\
                & {\bf gender:}   & mascgender, femgender, neutgender, \\
                &                 &                     omegagender \\
                & {\bf person:}   & 1 ... 5                  \\
\end{tabular}
\\
\begin{tabular}{lll}
PERSPROrecord:  &                 &              \\
                & {\bf number:}   & singular, plural, omeganumber   \\
                & {\bf gender:}   & mascgender, femgender, neutgender, \\
                &                 &                     omegagender \\
                & {\bf person:}   & 1 ... 5                  \\
                & persprocases:  & subset of (nominative, dative, accusative) \\
                & reduced:        & true, false           \\
\end{tabular}
\\
\begin{tabular}{lll}
BINDEFPROrecord: &                &              \\
                 & {\bf number:}  & singular, plural, omeganumber \\
                 & {\bf posgeni:} & true, false           \\
\end{tabular}
\\
\begin{tabular}{lll}
INDEFPROrecord: &                 &              \\
                & {\bf number:}   & singular, plural, omeganumber \\
                & {\bf posgeni:}  & true, false           \\
                & geni:                 & true, false           \\
\end{tabular}
\\
\begin{tabular}{lll}
BDETrecord: &                    &              \\
            & {\bf eFormation:}  & regEformation, irregEformation, noFormation  \\
            & {\bf enFormation:} & true, false           \\
\end{tabular}
\\
\begin{tabular}{lll}
DETrecord: &                    &              \\
           & {\bf eFormation:}  & regEformation, irregEformation, noFormation  \\
           & {\bf enFormation:} & true, false           \\
           & eORenForm:         & eForm, enForm, NoForm \\
\end{tabular}
\\
\begin{tabular}{lll}
POSSADJrecord: &                  &              \\
               & eORenForm:       & eForm, enForm, NoForm \\
               & reduced:         & true, false           \\
               & mood:            & wh, declxpmood, relativexpmood \\
               & geni:            & true, false           \\
\end{tabular}
\\
\begin{tabular}{lll}
POSSPROrecord: &                   &              \\
               & nForm:            & true, false \\
\end{tabular}
\\
\begin{tabular}{lll}
BWHPROrecord: &                  &              \\
             & {\bf sexes:}     & subset of (masculine, feminine) \\
             & {\bf number:}    & singular, plural, omeganumber   \\
\end{tabular}
\\
\begin{tabular}{lll}
WHPROrecord: &                  &              \\
             & {\bf sexes:}     & subset of (masculine, feminine) \\
             & {\bf number:}    & singular, plural, omeganumber   \\
\end{tabular}
\\
\begin{tabular}{lll}
DEMPROrecord: &                 &              \\
              & {\bf sexes:}    & subset of (masculine, feminine) \\
              & {\bf number:}   & singular, plural, omeganumber   \\
\end{tabular}
\\
\begin{tabular}{lll}
SFCATrecord: &                 &              \\
             & {\bf key:}      & key     \\
\end{tabular}
\\
\begin{tabular}{lll}
PFCATrecord: &                 &              \\
             & {\bf key:}      & key     \\
\end{tabular}
\\

Note: the attribute `person' of PERSPRO's is {\em not} a set; it is 
      {\em not} the same as the attribute `persons' of VERBs.

Note: not {\em all} values of these attributes are mentioned {\em always}: in
      some cases values that are not relevant for the morphology fail (for 
      instance the values `Rcase' and `genitive' of the attribute 
      `persprocases'). In other cases, however, values {\em are} mentioned
      although they do not occur in the morphology or {\em never} 
      occur.\footnote{This is the case when only part of a set of values is
      relevant for a specific category,
      like the value `enForm' of the attribute `eORenForm' in the ADVrecord:
      adverbs can have `eForm': {\em (hij komt het) vaakste}, but not `enForm':
      *{\em (het) vaaksten}. The value `enForm' has been added to the set 
      for other categories, like adjectives (ADJ).}

\newpage

\section{Rules}

In the Dutch morphological component of ROSETTA3 the following rules are 
used:

\begin{itemize}
  \item GLUE rules, for contractions of several kinds (see section 4.1),
  \item segmentation rules, that combine (in generation) or split (in analysis) 
        stem and suffixes (see section 4.2),
  \item phonological rules, that act as conditions on segmentation (see 
        section 4.3),
  \item W-rules, that -analytically speaking- build S-trees out of a 
        word and a suffix (see section 4.4).
\end{itemize}

In the following sections, the way these rules work will be illustrated.

\subsection{GLUE rules}

In analysis, GLUE rules split words; the parts (which are {\em strings}) 
are bound by a `GLUE':

\begin{tabbing}
erheen \ \ \ \ \ \  \= $\longrightarrow$ \ \ \  \= er \ \  \= + \ \  \= GLUE \ \  
\= + \ \  \= heen    \\
ingelopen    \> $\longrightarrow$ \> in \> + \> GLUE \> + \> gelopen   \\
\end{tabbing}

In generation, words bound by a GLUE are sticked together:

\begin{tabbing}
er \ \ \  \= + \ \  \= GLUE \ \  \= + \ \  \= heen \ \ \ \ \ \   \= 
$\longrightarrow$ \ \ \ \= erheen \\
in \> + \> GLUE \> + \> gelopen \> $\longrightarrow$ \> ingelopen \\
\end{tabbing}

For Dutch, GLUE rules are used for the following phenomenons:
\begin{itemize}
 \item contraction of the words {\em er}, {\em daar}, {\em hier}, {\em ergens},
 {\em nergens}, {\em overal} and {\em waar} with prepositions or adverbs 
       ({\em ertoe}, {\em daarheen}, \ldots ),
 \item contraction of prepositions and verb (see also section 5.6),
 \item contraction of {\em van} and {\em de} to: {\em der}.
\end{itemize}

The GLUE rules are listed in a separate document.

\subsection{Segmentation rules}

\subsubsection{About the rules}

Segmentation rules split -analytically speaking- a word into a stem and a 
prefix or suffix. The stem is a {\em string}, 
the prefix or suffix is represented as a {\em key}. 

Examples (`$\star$' stands for arbitrary strings):

\begin{tabbing}
$\star$oo  \ \ \ \ \ \ \ \   \= + \ \  \= SFKt \ \ \ \ \ \ \ \ \ \ \ \ \ \ 
\= :: \ \  \= $\star$oot;  \ \ \ \ \ \ \ \   \= ((hij) shampoo{\em t}) \\
$\star$giet \> + \> SFKovtvd1 \> :: \> $\star$goot; \> ((hij) {\em goot}, (hij) 
be{\em goot}) \\
PFKge       \> + \> ij$\star$ \> :: \> geij$\star$; \> ({\em ge}ijld)      \\
ik          \> + \> SFKsgaccdat  \> :: \> mij;      \> ({\em mij})         \\
\end{tabbing}

As we see here, segmentation rules are used for `real' suffixes ({\em -t} for
3rd person singular in present) and prefixes ({\em ge-} for past participles),
but also for ablaut (=change of stem vowel), like 
{\em giet}$\longrightarrow${\em goot}, 
and even for the change of complete strings, as in the last example.

Segmentation rules always handle one suffix or prefix at a time;
often we need more than one segmentation rule to `make' a word, which is
illustrated by the following example:

\begin{tabbing}
help  \ \ \ \ \ \ \ \   \= + \ \  \= SFKvd1  \ \ \ \ \ \ \ \ \ \ \ \ \ \    
\= :: \ \  \= holp; \ \ \ \ \ \ \ \   \= (ablaut)       \\
holp   \> + \> SFKen  \> :: \> holpen;   \> ({\em -en}-suffix) \\
PFKge  \> + \> holpen \> :: \> geholpen; \> ({\em ge-}-prefix) \\
\end{tabbing}

This means that the set of segmentation rules must be called recursively.
Some rules however, especially the ones that do {\em not} 
result in a lengthening
of the string (as in the above given rules for ablaut), cannot be applied
recursively because of danger of `looping' during execution. 
To prevent unnecessary or unwanted sequences of rule-applications,
the order in which the rules can be applied has been put down in a 
`regular expression'. This regular expression gives all {\em possible}
sequences of suffix- and prefix-keys. 

The regular expression for Dutch: 

\begin{tabbing}
[ \  \= \ \  \=  \ \ \ \ \ \ \ \ \   \ \ \ \ \ \ \ \ \   \ \ \ \ \ \ \ \ \  
 \ \ \ \ \ \ \ \ \   \ \ \ \ \ \ \ \ \   \ \ \ \ \ \ \ \ \  
 \ \ \ \ \ \ \ \ \   \ \ \ \ \ \ \ \ \   \ \ \ \ \ \ \ \ \  
 \ \ \ \ \ \ \ \ \   \ \ \ \ \ \ \ \ \   \ \ \ \= (verb)\\
( \>   \> (SFKen $\mid$  SFKt $\mid$  SFKe $\mid$  SFKdelD)
\> (1) \\
  \> $\mid$  \> (SFKIrrottenk0 $\mid$  SFKIrrottenk2 $\mid$  SFKIrrottenk3 
$\mid$  SFKIrrott4 \> (2) \\
  \>   \>  $\mid$  SFKIrrott5  $\mid$  SFKIrrottmv   $\mid$  SFKIrrovtenk  
$\mid$  SFKIrrovt5 \> (2) \\
  \>   \>  $\mid$  SFKIrrovtmv   $\mid$  ([SFKt] SFKIrrgb) 
$\mid$  SFKIrrconjott \> (2) \\
  \>   \>  $\mid$  SFKIrrconjovt)  \> (2) \\
  \> $\mid$  \> ([SFKen $\mid$  SFKt] (SFKovt1 $\mid$  SFKovtvd1 
$\mid$  SFKovtvd2 $\mid$  SFKovt2  \>     \\
  \>   \>  $\mid$  SFKovt3 $\mid$  SFKdete))  \> (3) \\
  \> $\mid$  \> ([PFKge] [SFKe $\mid$  SFKen] (SFKdt $\mid$  SFKen 
$\mid$  SFKovtvd2 $\mid$  SFKIrrvd)) \> (4)\\
  \> $\mid$  \> ([PFKge] [SFKe $\mid$  SFKen] SFKen (SFKovtvd1 $\mid$  SFKvd1 
$\mid$  SFKvd2)) \> (4) \\
  \> $\mid$  \> ([SFKe $\mid$  SFKen] SFKdt (SFKen $\mid$  SFKIrrottmv))   
       \> (5) \\
) \>   \>                                                             \>     \\
$\mid$  \>   \>                                                         
   \> (noun)\\
( \>   \> SFKen $\mid$  SFKmvs $\mid$  SFKaTOaa $\mid$  SFKaTOee 
$\mid$  SFKeTOee $\mid$  SFKeiTOee \> (6) \\
  \>   \> $\mid$  SFKiTOee $\mid$  SFKoTOoo $\mid$  SFKeren $\mid$  SFKien 
$\mid$  SFKden $\mid$ 
 SFKnen  \> (6) \\
  \>   \> $\mid$  SFKieAccent $\mid$  SFKlui $\mid$  SFKlieden $\mid$  SFKLat 
$\mid$  SFKenIrreg    \> (6) \\
  \>   \> $\mid$  SFKsIrreg $\mid$  SFKLatIrreg                                 
  \> (6) \\
  \>   \> $\mid$  (SFKgens  [SFKsIrreg $\mid$  SFKmvs])                         
  \> (7) \\
) \>   \>                                                             \>    
 \\
$\mid$  \>   \>                                                             
\> (adj)\\
( \>   \> ([[SFKen] (SFKe $\mid$  SFKonrege)]  (SFKer $\mid$  SFKonreger)) 
       \> (8) \\
  \> $\mid$  \> ([PFKaller] [[SFKen] (SFKe $\mid$  SFKonrege)] (SFKst 
$\mid$  SFKonregst))
 \> (9) \\
  \> $\mid$  \> (SFKadjs  [SFKer $\mid$  SFKonreger])                     
       \> (10) \\ 
  \> $\mid$  \> ([SFKen]  (SFKe $\mid$  SFKonrege))                       
       \> (11) \\
) \>   \>                                                            
\>      \\
$\mid$  \>   \>                                                       
    \> (other)\\
( \>   \> SFKsgnom $\mid$  SFKsgnomred $\mid$  SFKsgaccdat $\mid$  SFKsgaccdatred  
    \> (12) \\
  \>   \> $\mid$  SFKplnom $\mid$  SFKplnomred $\mid$  SFKplaccdat 
$\mid$  SFKplaccdatred    \> (12) \\
  \>   \> $\mid$  SFKpldat $\mid$  SFKplacc                             
         \> (12) \\
  \>   \> $\mid$  SFKsgpossadj $\mid$  SFKsgpossadjred $\mid$  SFKplpossadjons   
      \> (13) \\
  \>   \> $\mid$  SFKplpossadjonze $\mid$  SFKplpossadj                   
       \> (13) \\
  \>   \> $\mid$  SFKsgpossadjgen $\mid$  SFKplpossadjgen                 
       \> (13) \\
  \>   \> $\mid$  SFKsgposs $\mid$  SFKplposs $\mid$  SFKsgpossnvorm 
$\mid$  SFKplpossnvorm  \> (14) \\
  \>   \> $\mid$  SFKpossadjwiens $\mid$  SFKpossadjwier $\mid$  SFKpossadjdiens  
     \> (15) \\
) \>   \>                                                            \>      \\
$\mid$  \>   \>                                                      
      \>      \\
  \>   \> (SFKeDet $\mid$  SFKenDet)                                
       \> (16) \\
] \>   \>                                                         
   \>      \\
\end{tabbing}

Explanation of the regular expression (compare with the relevant segmentation 
and W-rules in the document on rules):

Line 1 gives regular present tense inflection of verbs: lev{\em en}, 
leef{\em t}, lev{\em e}, and deletion of `d' in forms like: hou, snij. 
The lines marked by `2' give all forms of (very)
irregular verbs: {\em zul}, {\em ben},  {\em is},  {\em bent},  {\em zijt}, 
{\em zijn},  {\em was},  {\em waart}, etc. The lines marked by `3' give 
past tense forms of weak and strong verbs: {\em liep}, {\em liept},
{\em liepen}, etc. The lines marked by `4' give the past participles, like:
{\em ge}lok{\em t}, {\em ge}lok{\em te}, {\em ge}lok{\em ten}, etc.
Line 5 gives all present participle forms.

The lines marked by `6' give the plural forms of nouns and line `7' the 
genitive forms.

The lines 8-11 give the inflection of adjectives and adverbs:
line `8' gives all comparatives, line 9 all superlatives, line 10 the 
s-forms (like: goed{\em s}, beter{\em s}), and line 11 all positive forms
(like: mooi{\em e}, mooi{\em en}).

The lines marked by `12' give inflected forms of the category PERSPRO:
{\em ikke}, {\em 'k}, {\em mij}, {\em me}, {\em ons}, {\em ze}, etc.

The lines marked by `13' give POSSADJ-forms, like: {\em mijn}, {\em m'n},
{\em ons}, {\em onze}, and: {\em mijner}, {\em onzer}, etc.

Line 14 gives POSSPRO-forms, like: {\em mijne}, {\em onze}
{\em mijnen}, {\em onzen}, etc.

Line 15 gives the POSSADJ-forms {\em wier}, {\em wiens}, and {\em diens}. 

Line 16 gives DET-forms, like: {\em vele}, {\em velen}.

More information about the segmentation rules can be found in the document 
where prefix- and suffix-rules are listed.


\subsubsection{Data}

A large amount of data underlies the segmentation rules. Often, the stem and
the string belonging to the affix interact (think of: consonant doubling,
change or ommission of letters, etc.). Several sources have been used
to collect the relevant data: ANS, the tape of Van Dale N-N, the `grammaticaal
compendium' of Van Dale, reverse dictionaries for Dutch (Nieuwborg, Martin),
`Praktische Cursus Spelling' (Klein \& Visscher). Of course, data influenced
many choices about attributes and their values.

The segmentation rules are listed in a separate document; in general, 
the rules will be self-explanatory (besides, all segmentation rules are followed
by examples that will help to understand).


\subsection{Phonological Rules}

\subsubsection{pronunciation-attributes}

In the Dutch morphology, two pronunciation-attributes are used:
`sjwa' en `wissel', both with the values `true' and `false'.

The attribute `sjwa' is `true' when the last syllable of the word contains
a sjwa. In Dutch, this can only be the case when {\em e}, {\em i} or {\em ij}
is written (all of these letters also have other pronunciations, and in that 
case `sjwa' is `false' of course). Example: {\em schot\underline{e}l}
(sjwa=true) 
(compare: {\em hot\underline{e}l} with sjwa=false). Other examples: 
{\em dikk\underline{e}}, {\em dikk\underline{e}rd}, {\em monn\underline{i}k}, 
{\em verrukkel\underline{ij}k}, {\em hand\underline{e}l}, etc. 
The attribute `sjwa' 
is `false' in all other cases (and also in: {\em race}, {\em bridge}).

The attribute `wissel' is `true' when the last letter of the word is a {\em s}
or {\em f} but only if these can change into {\em z} resp. {\em v}
during inflection. 
Examples: {\em doo\underline{f}}, that can change into {\em do\underline{v}e};
{\em doo\underline{s}}, that can change into {\em do\underline{z}en}, etc. 
The value of `wissel' is `false' in 
all other cases.

The attribute `wissel' differs from the attribute `sjwa', because it is not
directly connected to pronunciation: the {\em s} and {\em f} 
in words with `{\tt +}wissel' and in words 
with `{\tt -}wissel' are pronounced equally. But, a relation exists between
pronunciation and the attribute `wissel':
In Dutch, voiced consonants at the end of words become voiceless, and 
often the spelling doesn't follow this rule. This holds for {\em b},
that is pronounced as {\em p} at the end of a word (e.g.: {\em web}), 
and {\em d}, that is pronounced as {\em t} at the end of a word 
(e.g.: {\em dood}). For {\em v} and {\em z} both pronunciation 
and spelling change: {\em reizen}$\longrightarrow${\em reis}, 
{\em roven}$\longrightarrow${\em roof}.
The ROSETTA system only deals with {\em written} sentences, so no
problems arise when ambiguities in pronunciation are encountered, as long as 
there is difference in spelling: {\em lood} - {\em loot};
{\em krab} - {\em krap}. Therefore, no 
measures are needed for words ending in {\em b} or {\em d}.
Problems do arise for {\em v} and {\em z}, because 
here the written forms are equal 
too: the stem {\em golf} is ambiguous between the verbs {\em golven} 
and {\em golfen}. Here,
special measures have to be taken, which is done by introducing the 
attribute `wissel'.


\subsubsection{Check on pronunciation}

In some cases, it is important to have a check on pronunciation when words
undergo inflection; we need it, for example, to distinguish between 
$'${\em bedelen} and {\em be}$'${\em delen}
(the same holds for: $'${\em beteren} vs. {\em be}$'${\em teren}, 
and: $'${\em legeren} vs. {\em le}$'${\em geren}), and
to prevent that the plural form of the stem {\em wandel} will be written as
{\em wandellen} instead of {\em wandelen}, etc.

Sometimes however, the check on pronunciation is not needed for the spelling
of words; it doesn't make any difference, for example, whether or not the
{\em ij} is pronounced as sjwa, because this doesn't cause variation in 
spelling.

In the following situations `{\tt +}sjwa'
and `{\tt -}sjwa' words are spelled different 
(examples of verbs and nouns both with `{\tt +}sjwa' and with `{\tt -}sjwa' 
are given, `$<$cons$>$' stands for a consonant):

\begin{tabbing}
  $\star$$<$cons$>$el \ \ \ \ \ \ \= \= (wandel{\em en} \ \ \  \= vs. 
                       \= bel{\em len}) \= \\
  $\star$$<$cons$>$em \> \> (adem{\em en}    \> vs. \> rem{\em men}) \> \\
  $\star$$<$cons$>$en \> \> (teken{\em en}   \> vs. \> ren{\em nen}) \> \\
  $\star$$<$cons$>$er \> \> (vorder{\em en}  \> vs. \> sper{\em ren}) \> \\
  $\star$$<$cons$>$es \> \> (hannes{\em en}  \> vs. \> fles{\em sen}) \> \\
  $\star$$<$cons$>$ig \> \> (stenig{\em en}  \> vs. \> lig{\em gen}) \> \\
  $\star$$<$cons$>$ik \> \> (hinnik{\em en}  \> vs. \> stik{\em ken}) \> \\
  $\star$$<$cons$>$et \> \> (lemmet{\em en}  \> vs. \> smet{\em ten}  \> 
                       (only for nouns)) \\
  $\star$$<$cons$>$it \> \> (kievit{\em en}  \> vs. \> pit{\em ten}   \> 
                       (only for nouns)) \\
\end{tabbing}

In these cases, different rules for {\tt +}sjwa and {\tt -}sjwa words 
{\em must} be made.

In other cases, it is {\em not} relevant for spelling whether {\em e} or
{\em i} is 
pronounced as sjwa. Example (where the {\em e} is not preceded by a consonant): 
the stemverbs {\em lui\underline{e}r}(`{\tt +}sjwa') and {\em kier}(`{\tt
-}sjwa') 
both get identically formed present plural forms (with simply {\em en} sticked 
to the stem). In these cases, different SUFFIX rules for `{\tt +}sjwa' and 
`{\tt -}sjwa'
words are possible, but {\em not} necessary.

The attribute `wissel' is almost always relevant; nearly {\em every}
word ending in {\em f} or {\em s} can meet rules where differences in 
spelling occur when it undergoes inflection.


\subsubsection{List of phonological rules}

In this section, we will specify all phonological rules of the Dutch ROSETTA 
morphology.

Actually, every segmentation rule that deals with suffixes is accompanied by
a phonological rule. The `default'-rule `FONleegleeg' however, is never
written. Every rule consists of a condition and an action; in the condition
the values of the attributes `sjwa' and `wissel' can be tested, in the action
new values can be assigned. Attributes keep their original value when no new
value is assigned to them in the action.

In theory, many combinations attribute-value pairs are possible in both 
condition and action. In practice, the number of rules is limited. One important
reduction results from the fact that the combination of {\tt +}sjwa 
and {\tt +}wissel does {\em not} exist in Dutch words 
(except for the word {\em gannef}, see Van Dale N~--~N, which has the strange 
plural form {\em ganneven}; this plural should be treated as an irregular 
form); therefore, the `sjwa:=true' assignment can always be made together with
the `wissel:=false' assignment (and the other way around).

The rules are:

\begin{tabbing}
rule  \ \ \ \ \  \= FONleegleeg \\
C:   \> true        \\
A:   \> @           \\
\end{tabbing}

Note: this `default-rule' is in practice never written; the condition is 
always true, the action doesn't change anything.

\begin{tabbing}
rule   \ \ \ \ \  \= FONleegsjwa   \ \ \ \ \  \= ( example: verg-vergen ) \\
C:   \> true                  \>                          \\
A:   \> fonuit.sjwa:=true     \>                          \\
     \> fonuit.wissel:=false  \>                          \\
\end{tabbing}

\begin{tabbing}
rule \ \ \ \ \  \= FONleegonwissel \ \ \ \ \  \= ( lees-leest, 
leef-[ge]leefd ) \\
C:   \> true                  \>                                \\
A:   \> fonuit.wissel:=false  \>                                \\
\end{tabbing}

\begin{tabbing}
rule \ \ \ \ \   \= FONsjwasjwa           \ \ \ \ \   \= ( wandel-wandelen ) \\
C:   \> fonin.sjwa=true       \>                     \\
A:   \> fonuit.sjwa:=true     \>                     \\
\end{tabbing}

\begin{tabbing}
rule \ \ \ \ \   \= FONonsjwasjwa \ \ \ \ \   \= ( bel-bellen )   \\
C:   \> fonin.sjwa=false      \>                          \\
A:   \> fonuit.sjwa:=true     \>                          \\
     \> fonuit.wissel:=false  \>                          \\
\end{tabbing}

\begin{tabbing}
rule \ \ \ \ \   \= FONsjwaonsjwa \ \ \ \ \   \= ( bewandel-bewandelbaar )    \\
C:   \> fonin.sjwa=true       \>                          \\
A:   \> fonuit.sjwa:=false    \>                          \\
\end{tabbing}

\begin{tabbing}
rule \ \ \ \ \   \= FONwisselsjwa \ \ \ \ \   \= ( blaas-blazen )         \\
C:   \> fonin.wissel=true     \>                          \\
A:   \> fonuit.sjwa:=true     \>                          \\
     \> fonuit.wissel:=false  \>                          \\
\end{tabbing}

\begin{tabbing}
rule \ \ \ \ \   \= FONonwisselsjwa   \ \ \ \ \   \= ( hees-hesen )           \\
C:   \> fonin.wissel=false    \>                          \\
A:   \> fonuit.sjwa:=true     \>                          \\
     \> fonuit.wissel:=false  \>                          \\
\end{tabbing}

\begin{tabbing}
rule \ \ \ \ \   \= FONwisselonwissel  \ \ \ \ \   \= ( gons-[ge]gonsd ) \\
C:   \> fonin.wissel=true     \>                          \\
A:   \> fonuit.wissel:=false  \>                          \\
\end{tabbing}

\begin{tabbing}
rule \ \ \ \ \  \= FONonwisselonwissel \ \ \ \ \  
\= ( dans-[ge]danst; hijs-hees ) \\
C:   \> fonin.wissel=false    \>                                 \\
A:   \> fonuit.wissel:=false  \>                                 \\
\end{tabbing}

\begin{tabbing}
rule \ \ \ \ \   \= FONonwisselwissel     \ \ \ \ \   \= ( hef-hief ) \\
C:   \> fonin.wissel=false    \>                          \\
A:   \> fonuit.wissel:=true   \>                          \\
     \> fonuit.sjwa:=false    \>                          \\
\end{tabbing}

\begin{tabbing}
rule \ \ \ \ \   \= FONwisselwissel \ \ \ \ \   \= ( rijs-rees )   \\
C:   \> fonin.wissel=true     \>                          \\
A:   \> fonuit.wissel:=true   \>                          \\
     \> fonuit.sjwa:=false    \>                          \\
\end{tabbing}

\subsection{W-rules}

W-rules build lexical S-trees out of a word and one or more derivational
or inflectional affixes. In the morphology of ROSETTA, we have three types of
categories for the main categories: BCAT (BVERB, BNOUN, BADJ, BADV), SUBCAT 
(SUBVERB, SUBNOUN, etc.) and CAT (VERB, NOUN, etc.). The first type is in the 
dictionary, the second type is made out of the first by combining the word
with one or more derivational affixes (like: {\em ex-} or {\em -baar}), and 
the third is made out of the second by applying inflection.

Categories other than the ones mentioned here have only two kinds of categories 
in ROSETTA: BCAT and CAT (for inflectional affixes), or only one: CAT.

The W-rules for derivation produce trees like:

\setlength{\unitlength}{1ex}
\begin{picture}(65,29)(0,0)
\put(39,24){\makebox(0,0)[b]{SUBCAT}}
\put(30,13){\makebox(0,0)[b]{BPREFIX}}
\put(48,13){\makebox(0,0)[b]{SUBCAT}}
\put(39,2){\makebox(0,0)[b]{BCAT}}
\put(57,2){\makebox(0,0)[b]{BSUFFIX}}
\put(39,23){\line(-1,-1){7}}
\put(39,23){\line(1,-1){7}}
\put(48,12){\line(-1,-1){7}}
\put(48,12){\line(1,-1){7}}
\end{picture}

There may be several levels with SUBCAT nodes and the derivational affixes 
will be represented as nodes.
The W-rules for inflection make a CAT out of a SUBCAT:

\begin{picture}(65,15)(0,0)
\put(39,12){\makebox(0,0)[b]{CAT}}
\put(39,3){\makebox(0,0)[b]{SUBCAT}}
\put(39,6){\line(0,1){6}}
\end{picture}

Inflectional suffixes are not represented as nodes in the S-tree; the
information about the inflectional form is represented by means of 
attribute-values.

W-rules are condition-action pairs; the action will be done only
if the condition is true. Each rule consists of two of these pairs: one
for analysis and one for generation. The condition-action pair is called 
`comp' for analysis and `decomp' for generation.

In Dutch, we have five types of rules:
\begin{itemize}
 \item rules that make a SUBCAT out of a BCAT;
 \item rules that make a SUBCAT out of a SUBCAT;
 \item rules that make a CAT out of a SUBCAT;
 \item rules that make a CAT out of a BCAT;
 \item rules that make a CAT out of a CAT.
\end{itemize}

The first two types are derivational rules; the last three are inflectional
rules, and will be illustrated here.

In principle, the inflectional W-rules for Dutch build words at one go. 
There is, for instance, a rule that builds the past participle {\em gelopen} 
out of the stem {\em loop}, and another rule that builds the infinitive 
{\em lopen} out of {\em loop}, and both rules are independent (an alternative 
would be to use the rule for {\em lopen} as input for the rule for 
{\em gelopen}).
This principle has an important advantage: W-rules always take 
single stems or stems 
with derivational affixes as input, and never `obscure' (partially inflected) 
forms (such an obscure form would be: {\em wroken}, which is neither a stem of 
a Dutch word, nor is it a Dutch word in itself, but
could be an intermediate form between the stem {\em wreek} and the past 
participle
{\em gewroken}). The disadvantages are: often rules overlap, and in some cases,
rules will be rather complex.

Four rules do not obey this `at-one-go'-principle; by making these four
`CAT-to-CAT'-rules, the number of rules `SUBCAT-to-CAT'-rules could be
reduced considerable. One of these rules sticks {\em -e} or {\em -en} 
to things of the category VERB, in the following situation:

\begin{description}
  \item [verbs] as past participle + {\em -e}   ({\em gevraagd\underline{e}});
  \item [verbs] past participle + {\em -en}  ({\em gevraagd\underline{en}});
  \item [verbs] present participle + {\em -e}   ({\em vragend\underline{e}});
  \item [verbs] present participle + {\em -en}  ({\em vragend\underline{en}});
\end{description}

For the category ADJ, there are two rules; one sticks {\em -e} to an adjective
in case of:

\begin{description}
  \item [adjectives] in positive     + {\em -e} ({\em grot\underline{e}});
  \item [adjectives] in comparative  + {\em -e} ({\em groter\underline{e}});
  \item [adjectives] in superlative  + {\em -e} ({\em grootst\underline{e}}):
  \item [adjectives] in allersuperlative + {\em -e} 
        ({\em allergrootst\underline{e}});
\end{description}

Note: the other forms of adjectives (`sPositive' and `sComparative') 
do not get the suffix {\em -e}: *{\em (iets) grootse}, *{\em (iets) groterse}.

Another rule sticks {\em -en} to an adjective in the same cases (so again,
the `s-forms' are excluded):

\begin{description}
  \item [adjectives] in positive     + {\em -en} ({\em grot\underline{en}});
  \item [adjectives] in comparative  + {\em -en} ({\em groter\underline{en}});
  \item [adjectives] in superlative  + {\em -en} ({\em grootst\underline{en}});
  \item [adjectives] in allersuperlative + {\em -en} 
        ({\em allergrootst\underline{en}});
\end{description}

The fourth rule sticks {\em -e} to objects of the category ADV, in case of:

\begin{description}
  \item [adverbs] in superlative      + {\em -e}  ({\em vaakst\underline{e}});
  \item [adverbs] in allersuperlative + {\em -e}  
        ({\em allervaakst\underline{e}});
\end{description}

When all these phenomena would be handled 
in rules according to the `at-one-go'-principle, the number of rules would 
be nearby doubled.

Note that in these rules:

\begin{itemize}
  \item the input is always something that is a word of Dutch already (which 
        means that the input is never `obscure'),
  \item often, the resulting word functions, in some sense, like a word of 
        another category ({\em gevraagde}, for instance, functions as 
        adjective; {\em grote} in some sense functions like a noun).
\end{itemize}

Only one attribute changes (in analysis) its value in these 
`CAT$\longrightarrow$CAT'-rules: `eORenForm' 
will change from `NoForm' to `eForm' or `enForm'
(see sections 5.4, 7.3, and 7.6), and in one rule, from `eForm' to `enForm'.
This change also guarantees that the
`CAT$\longrightarrow$CAT'-rules will not get into a loop:
three of them can
-in analysis- only be applied if `NoForm' holds, and they yield `eForm' or
`enForm'; the fourth can only be applied if `eForm' holds and it yields
`enForm'.

One should note that inflectional phenomena with the {\em -e}-suffix,
{\em other} than those mentioned above,
like the present tense of the subjunctive in Dutch ({\em men nem\underline{e}}, 
{\em lev\underline{e} de koningin}, etc.) are handled in 
W-rules of the type `SUBCAT changes into
CAT'. The same holds for other inflectional phenomena with the {\em -en}-suffix.

\newpage

\section{Verbs}
\subsection{Conjugationclasses}
In Dutch, we can distinguish the following verb types:

\begin{itemize}
 \item verbs with limited inflection,
 \item weak verbs,
 \item strong verbs,
 \item irregular verbs.
\end{itemize}

As we will see, the set of verbs of three of these four types can 
be subdivided 
into groups of verbs, that have resemblant conjugation. These groups are
the `conjugationclasses' of the Dutch morphology for ROSETTA.
In the next sections we will discuss each of these types, and the 
conjugationclasses corresponding to them. The inherent attribute `conjclasses'
gives the conjugationclass(es) of each verb in ROSETTA.


\subsubsection{Verbs with limited inflection}

This verb type corresponds to one conjugationclass, namely class 0. All
verbs belonging to this conjugationclass are limited in their inflection;
the only possible forms are: infinitive and present participle. The other
forms are ungrammatical. Example: {\em buikspreken}, which
has the following wellformed forms: {\em buikspreken}(inf.),  
{\em buiksprekend(e)} 
(pres. part.). Other forms, as: {\em sprak buik} or {\em buikspreekte}
(past sing.), are judged ungrammatical by (most) speakers of Dutch.
Other examples: {\em kunstrijden}, {\em wielrijden}, 
{\em langlaufen}, etc.

The existing inflected forms of the verbs of class 0 can easily be obtained 
by using the same W-rules as for weak verbs; therefore, the 
W-rules for infinitive and present participle of weak verbs also work for verbs 
of conjugationclass 0.


\subsubsection{Weak verbs}

The weak verbs correspond to two conjugationclasses, namely class 3 and
class 4. Class 3 verbs form their past participle 
with the {\em ge}-prefix, class 4
verbs, in contrary, do without.

In the present tense, all verbs get regular endings: 

\begin{itemize}
 \item no ending for the first person singular ({\em ik werk}), and the second 
       person singular when in reverse order: {\em werk jij};
 \item {\em -t} for the second person singular (but not in reverse order: 
       *{\em werkt jij}) and the 
       third person singular, but not when the stem ends in {\em t} already 
       (*{\em hij plantt});
 \item {\em -en} for the plural forms of the first, second and third person 
       ({\em wij/jullie/zij werken}); and {\em -t} for the fourth and the 
       fifth person ({\em u/gij werkt}).
\end{itemize}

Both in the past tense and in past participle
forms, two different suffixes exist in Dutch: {\em -de} and {\em -te}
 for past forms,
{\em -d} and {\em -t} for the past participle. 
The {\em -de} and {\em -d} forms are used 
adjacent to the following vowels and consonants: a, b, d, dge 
({\em bridge}), ee, ie, oe, ue, f, g, i, ij, l, m, n, o, r, s, u, w, y; 
the {\em -te} and {\em -t} form are used when 
the stem ends in: ce ({\em race}), ch, f, k, p, s, sh, sj, t, x.

When the stem of a verb ends in {\em f} or {\em s}, which are included in 
both groups, {\em -de}/{\em -d} is chosen when {\em f} changes into {\em v} 
or {\em s} into {\em z} in the inflectional process (for instance in the 
infinitive form: {\em roof}-{\em roven} 
and {\em -te}/{\em -t} is chosen when {\em f} and {\em s} never change in 
inflection ({\em ruis}-{\em ruisen})). Verbs with change of {\em f} and
{\em s} are marked by a phonological attribute.
Some verbs ending in {\em f} and {\em s} have both suffixes: {\em sponsde}, 
{\em sponste} (in fact, there are two infinitives: {\em sponzen} and 
{\em sponsen}). Sometimes, verbs with the same stem form have different
suffixes: the {\em f} in {\em golf} changes when the verb means {\em `to wave'},
in the verb {\em golf} with the meaning {\em `to play golf'}, however, it 
doesn't change.

The past participle doesn't get a suffix when the stem already ends in {\em t}
or {\em d} ({\em plant}-{\em geplant}, {\em dood}-{\em gedood}).

The plural form of the past tense get the regular plural ending {\em -(e)n} 
extra for the first, second and third person; all 
other forms (infinitive, imperative, etc.) have regular inflection too.


\subsubsection{Strong verbs}
Verbs of this type have {\em ablaut} (change of vowel) in the past tense and/or
past participle forms; they can belong to several conjugationclasses:

\begin{itemize}
  \item [5] ablaut in past tense; past participle with {\em -en} ending and 
        {\em ge-}-prefix;

        (dragen-droeg-gedragen, blazen-blies-geblazen)

  \item [6] as class 5, but past participle without {\em ge-}; 

        (verdragen-verdroeg-verdragen)

  \item [7] (same) ablaut in past tense and past participle; past participle 
        with {\em -en}-ending and {\em ge-}-prefix;

        (blinken-blonk-geblonken)
  \item [8] as class 7, but past participle without {\em ge-}; 

        (verbinden-verbond-verbonden)

  \item [9] (same) ablaut in past tense and past participle; past participle 
        with regular ending and {\em ge-}-prefix;

        (denken-dacht-gedacht)

  \item [10] idem, but past participle without  {\em ge-};

        (verdenken-verdacht-verdacht)

  \item [11] ablaut in past tense, and (a different) ablaut  in past participle;
        past participle with {\em -en} ending. 

        (helpen-hielp-geholpen)

  \item [12] idem, but past participle without {\em ge-}. 

        (bederven-bedierf-bedorven)

  \item [13] ablaut in past tense; past participle doesn't exists.
        Often, verbs of this class have also a (full) weak conjugation,
        and therefore belong to class 3 too. 

        (durven-dorst (reg.:durven-durfde-gedurfd), plegen-placht (no reg.))

  \item [14] for special form for 0th and 1st person singular and 
        imperative singular; verbs of this class have a full conjugation in
        another conjugationclass.

        (hou, snij, etc. (reg. houd-hield-gehouden resp. snijd-sneed-gesneden))

  \item [15] verbs with a regular past tense, and past participle with ablaut 
        and {\em -en} ending. 

        Actually, some of these verbs have a past participle that does
        not have ablaut (but do have {\em -en} ending); of course these verbs
        could have formed a separate class, but for convenience they are 
        combined with the other verbs of this class; as a consequence of this
        incorporation, we now have segmentation rules of the type:

\begin{itemize}
          \item [] $\star$bak + SFKvd2 :: $\star$bak;
\end{itemize}

        where the string at the left is identical with the one at the right.

        (bakken-bakte-gebakken, wreken-wreekte-gewroken)

  \item [16] idem, but past participle without {\em ge-}. 
        (vermalen-vermaalde-vermalen)

\end{itemize}
        

All strong verbs have regular conjugation for the present tense. Also,
they have regular infinitive, imperative and present participle forms.
Actually, the only way these verbs differ from the weak ones, is that they
have ablaut instead of the regular ending for the past form and/or ablaut 
(often combined with {\em -en} ending) for the past participle, again instead of
the regular ending. As we have seen, some strong verbs even have weak
past forms (class 15,16). Because of this rather great similarity between
strong and weak verbs, it is possible to write one set of 
W-rules for both. This in contrary to the irregular verbs, 
that have their own set of W-rules.

A separate list of strong verbs will not be given in this section;
as a matter of fact, the segmentation rules give such a list, but one should
note that several verbs are not listed there while they would cause redundant 
paths. The verb {\em bezwijken}, for instance, is {\em not}
listed, because of the rule
for {\em $\star$wijken} (`$\star$' 
denotes `variable string'(the empty string included)). Also, the rule for 
{\em $\star$wijken} works for {\em afwijken}. Most `simple' strong verbs can 
get prefixes as: {\em her-}, {\em ver-}, {\em re-}, etc., and therefore every 
rule has a `$\star$'.

The number of segmentation rules could have been limited by combining verbs
with identical ablaut (and stem ending): the rules for {\em $\star$wijken} and 
{\em $\star$kijken}
could have been joined in a single rule for {\em $\star$ijken}, 
because both end in {\em k} and have {\em ij}$\longrightarrow${\em ee} 
ablaut, but for 
reasons of readability we have decided not 
to do so. Now, every segmentation rule for ablaut has as its left hand member
a (minimal) verbstem, preceded by `$\star$'.

A complete list of `simple' verbs that are not listed because their stem
contains another `simple' verb with identical ablaut:

\begin{tabbing}
      bezwijken  \ \ \ \ \ \ \ \ \    \=  (see: \=  wijken) \\
      braden      \>  (see: \>  raden, class 15) \\
      blijken     \>  (see: \>  lijken) \\
      drijven     \>  (see: \>  rijven) \\
      glijden     \>  (see: \>  lijden) \\
      knijpen     \>  (see: \>  nijpen) \\
      krijgen     \>  (see: \>  rijgen) \\
      krijten     \>  (see: \>  rijten) \\
      kwijten     \>  (see: \>  wijten) \\
      prijzen     \>  (see: \>  rijzen) \\
      schrijden   \>  (see: \>  rijden) \\
      schrijven   \>  (see: \>  rijven) \\
      stijgen     \>  (see: \>  tijgen, class 7) \\
      strijden    \>  (see: \>  rijden) \\
      vliegen     \>  (see: \>  liegen) \\
      wrijven     \>  (see: \>  rijven) \\
      zwerven     \>  (see: \>  werven) \\
\end{tabbing}

The rules for {\em $\star$krijs} and {\em $\star$rijs} (class 7) are 
a special case: they do 
not overlap, because they are restricted by different phonological conditions.

The stem {\em tijg} belongs to two different verbs, one with the past tense form
{\em teeg (aan)} and the past participle {\em (aan)getegen}, and one with
{\em toog} and {\em getogen}.
The first form seems to be restricted to the verb 
{\em aantijgen}, the other occurs in {\em tijgen}, $'${\em overtijgen} and
{\em over}$'${\em tijgen}.
Due to the type of ablaut, both stems would be in the same conjugationclass, 
but this would give problems because the stems belong to different verbs 
(with different meanings) and are {\em not} spelling variants. 
Therefore the first stem (that also works for frequent verbs like 
{\em stijgen}, {\em opstijgen}, etc.) has been put in class 7, 
(which is the most appropriate class); the other stem 
is put in class 11 and -thus- is the only member of this class with the 
same ablaut for both past tense and past participle.

\subsubsection{Irregular verbs}

The irregular verbs differ from the weak and strong verbs 
because they have irregular forms. They have conjugationclasses 1 
(verbs that have a past participle with {\em ge-}) or 2 
(verbs with past participles without {\em ge-}).

The following combinations of person, number, tense and modus are listed 
{\em separately}
for each of the verbs (but only if the form exists, of course):

\begin{itemize}
  \item for {\bf present tense, indicative}:
     \begin{description}
       \item [] persons: {\bf 0},  number:  {\bf singular}
       \item [] persons: {\bf 1},  number:  {\bf singular}
       \item [] persons: {\bf 2},  number:  {\bf singular}
       \item [] persons: {\bf 3},  number:  {\bf singular}
       \item [] persons: {\bf 4},  number:  {\bf singular} 
       \item [] persons: {\bf 5},  number:  {\bf singular}
       \item [] persons: {\bf 1\ldots 4},  number:  {\bf plural}
       \item [] persons: {\bf 5},  number:  {\bf plural}
     \end{description}  
  \item for {\bf past tense, indicative}:
     \begin{description}
       \item [] persons: {\bf 1\ldots 4}, number: {\bf singular}
       \item [] persons: {\bf 5}, number: {\bf singular}
       \item [] persons: {\bf 1\ldots 4}, number: {\bf plural}
       \item [] persons: {\bf 5}, number: {\bf plural}
     \end{description}
   \item other forms (all with persons: {\bf {[\ ]}}):
     \begin{description}
       \item [] number: {\bf omeganumber}, 
tense: {\bf omegatense}, modus: {\bf pastpart}
       \item [] number: {\bf omeganumber}, 
tense: {\bf omegatense}, modus: {\bf prespart}
       \item [] number: {\bf omeganumber}, 
tense: {\bf omegatense}, modus: {\bf infinitive}
       \item [] number: {\bf singular}, 
tense: {\bf omegatense}, modus: {\bf imperative}
       \item [] number: {\bf plural}, 
tense: {\bf omegatense}, modus: {\bf imperative}
       \item [] number: {\bf omeganumber}, 
tense: {\bf presenttense}, modus: {\bf subjunctive}
       \item [] number: {\bf omeganumber}, 
tense: {\bf pasttense}, modus: {\bf subjunctive}
     \end{description}
\end{itemize}

Note: the subjunctive past form exists only for the verb {\em zijn}.

For almost any of these forms a separate suffix-key has been introduced. 
Only few verbs (like {\em zullen}, {\em kunnen}) have an `extra' 0-form, etc.

The problem that {\em zullen} doesn't have a past participle, will be
handled in the M-grammar. In the morphology, the W-rule that deals
with past participles of irregular verbs can handle {\em zullen} in principle
(but only if there is a segmentation rule for the past participle of {\em 
zullen} too).

The stems of irregular verbs can be split into
three groups; the first group consists of the verbs {\em zijn}, {\em hebben},
verbs that can act as auxiliary,
and the verb {\em zeggen} (with the irregular past form {\em zeiden}), 
the second one consists of verbs with infinitives ending in {\em -an} and 
the verb {\em doen}, and the third group consists of verbs with a difference in 
length of the stem
vowel between singular and plural in the past tense as, for instance, {\em at} 
vs. {\em aten}. The verb {\em komen} has this difference in the present tense 
too and the verb {\em eten} has also an irregular past participle. 


The following {\em stems} are listed in the segmentation rules:

\begin{tabbing}
{\bf verb:}   \ \ \ \ \ \ \ \ \ \ \ \     \=   
{\bf past tense:}  \ \ \ \ \ \ \ \ \ \ \ \    \=    
{\bf past participle:}  \ \ \ \ \ \ \ \ \ \ \ \   \\
          \>              \>            \\
ben       \>   was        \>    geweest \\
heb       \>   had        \>    gehad   \\
kan       \>   kon        \>    gekund  \\
mag       \>   mocht      \>    gemogen \\
wil       \>   wou/wilde  \>    gewild  \\
zal       \>   zou        \>    -       \\
zeg       \>   zei        \>    gezegd  \\
          \>              \>            \\
doe       \>   deed       \>    gedaan  \\
ga        \>   ging       \>    gegaan  \\
sla       \>   sloeg      \>    geslagen \\
sta       \>   stond      \>    gestaan \\
          \>              \>            \\
beveel    \>   beval      \>    bevolen \\
bid       \>   bad        \>    gebeden \\
breek     \>   brak       \>    gebroken \\
eet       \>   at         \>    gegeten \\
geef      \>   gaf        \>    gegeven \\
genees    \>   genas      \>    genezen \\
kom       \>   kwam       \>    gekomen \\
lees      \>   las        \>    gelezen \\
lig       \>   lag        \>    gelegen \\
meet      \>   mat        \>    gemeten \\
neem      \>   nam        \>    genomen \\
spreek    \>   sprak      \>    gesproken \\
steek     \>   stak       \>    gestoken \\
steel     \>   stal       \>    gestolen \\
treed     \>   trad       \>    getreden \\
vergeten  \>   vergat     \>    vergeten \\
vreten    \>   vrat       \>    gevreten \\
zie       \>   zag        \>    gezien   \\
zit       \>   zat        \>    gezeten  \\
\end{tabbing}

The list also contains 
some verbs that belong to conjugationclass 2 because they do not have
past participle forms with {\em ge-}, like {\em bevelen}, {\em genezen}
and {\em vergeten}. These verbs do not have a counterpart in class 1.

Verbs of class 2 that have a counterpart in class 1 are for instance 
{\em verbreken},
{\em vergeven}, {\em bekomen}, {\em vergaan}, 
{\em vernemen}, {\em verslaan}, {\em beslaan}, {\em herzien},
{\em vermogen}, etc. (Note: the past participles of {\em vermogen} and 
{\em mogen} do not have similar ending).



\subsubsection{Remarks on the strong and irregular verbs}

In this section, lists of irregular verbs of various kinds are given; e.g. a 
list of verbs with extra past forms, verbs, different in meaning but 
with the same stem form, etc. These lists are meant to give an indication of 
the complexities and ambiguities that can occur; there is no guarantee that 
they are complete.

\paragraph{Sources}

It is difficult to get a {\em complete} list of the strong and irregular verbs.
Of course, ANS and the `Van Dale' have been used; the number of differences
between these two sources is remarkable, however.

The following verbs were missing in the `grammaticaal compendium' of `Van 
Dale' (for the meaning of numbers --as in `wuiven/1'-- see table in 
`different verbs with identical stems' later in this section):

\begin{tabbing}
{\bf verb:}   \ \ \ \ \ \ \ \ \ \ \ \     \=   
{\bf past tense:}  \ \ \ \ \ \ \ \ \ \ \ \    \=    
{\bf past participle:}  \ \ \ \ \ \ \ \ \ \ \ \   \\
             \>               \>                               \\
meten        \>   mat         \>    gemeten                    \\
hoeven       \>   hoefde      \>    gehoeven/gehoefd           \\
rijzen       \>   rees        \>    gerezen                    \\
spouwen      \>   spouwde     \>    gespouwen                  \\
weven        \>   weefde      \>    geweven                    \\
wuiven/1     \>   woof        \>    gewoven                    \\
\end{tabbing}

Missing in the `grammaticaal compendium' of `van Dale' and missing in ANS
(but present in the alfabetical list of `Van Dale'):

\begin{tabbing}
{\bf verb:}   \ \ \ \ \ \ \ \ \ \ \ \     \=   
{\bf past tense:}  \ \ \ \ \ \ \ \ \ \ \ \    \=    
{\bf past participle:}  \ \ \ \ \ \ \ \  \=  \\
        \>                   \>                      \>                    \\
bersten \>   borst(berstte)  \>  geborsten    \> (={\em vaneensplijten})   \\
rijven  \>   reef            \>  gereven  \> (={\em harken}, {\em raspen}) \\
vijzen  \>   vees            \>  gevezen      \> (={\em schroeven})        \\
zeiken  \>   zeek(zeikte)    \>  gezeken(gezeikt) \>                       \\
zweten  \>   zweette         \>  gezweten         \>                       \\
\end{tabbing}

and also the following verbs, of which the irregular forms are not common in 
written language, however
(and therefore, this group is {\em not} included in our Dutch morphology):

\begin{tabbing}
{\bf verb:}   \ \ \ \ \ \ \ \ \ \ \ \     \=   
{\bf past tense:}  \ \ \ \ \ \ \ \ \ \ \ \    \=    
{\bf past participle:}  \ \ \ \ \ \ \ \ \ \ \ \   \\
             \>                   \>                                   \\
erven        \>   erfde           \> georven (geerfd)                  \\
vrijen       \>   vree(vrijde)    \> gevree\"{e}n (gevrijd)            \\
breien       \>   bree(breide)    \> gebree\"{e}n (gebreid)            \\
uitscheiden  \>   schee(d) uit    \> uitgeschee\"{e}n, uitgescheden    \\
             \>   (scheidde uit)  \> (uitgescheiden)                   \\
\end{tabbing}

ANS mentions the following extra form: {\em gekorven}(=past participle of 
{\em kerven}),
which is missing in the list in `Van Dale' but not in the alfabetical 
section).

Note that the alfabetical section of `Van Dale' is not always consistent 
with the list in the `grammaticaal compendium': {\em vragen} has two past 
forms in the alfabetical section ({\em vraagde} and {\em vroeg}), but only 
one in the list ({\em vroeg}). 

The majority of the irregular verbs that are {\em not} mentioned in the 
`grammaticaal compendium' {\em do} occur in the alfabetical section.

\paragraph{Verbs with more than one conjugation class}

Sometimes a single verb has more than one conjugationclass;
a list of these verbs is:

\begin{itemize}

\item Extra past and past participle:

\begin{tabbing}
{\bf verb:}   \ \ \ \ \ \ \ \ \ \ \ \ \     \=   
{\bf past tense:}  \ \ \ \ \ \ \ \ \ \ \ \ \ \    \=    
{\bf past participle:}  \ \ \ \ \ \ \ \ \  \=
{\bf class:} \\
           \>                 \>                     \>                   \\
dunken     \> docht/dacht/dunkte \>  gedocht/gedunkt \>    [9, 3, 13]     \\
kerven     \> korf/kerfde     \> gekorven/gekerfd    \>    [7, 3]         \\
krijsen    \> krees/krijste   \> gekresen/gekrijst   \>    [7, 3]         \\
spugen     \> spoog/spuugde   \> gespogen/gespuugd   \>    [7, 3]         \\
scheren/3  \> schoor/scheerde \> geschoren/gescheerd \>    [7, 3]         \\
wuiven/1   \> woof/wuifde     \> gewoven/gewuifd     \>    [7, 3]         \\ 
zeiken     \> zeek/zeikte     \> gezeken/gezeikt     \>    [7, 3]         \\
\end{tabbing}

Note: the OVT-form {\em dacht} (of {\em dunken}) is only mentioned in 
Van Dale and not in ANS.

\item Extra past only:

\begin{tabbing}
{\bf verb:}   \ \ \ \ \ \ \ \ \ \ \ \ \      \=   
{\bf past tense:}  \ \ \ \ \ \ \ \ \ \ \ \ \ \    \=    
{\bf past participle:}  \ \ \ \ \ \ \ \ \  \=
{\bf class:} \\
           \>               \>                   \>                   \\
bersten    \> borst/berstte \> geborsten  \> [7, 16] \\
delven     \> dolf/delfde   \> gedolven   \> [7, 15] \\
durven     \> dorst/durfde  \> gedurfd    \> [13, 3] \\
jagen      \> joeg/jaagde   \> gejaagd    \> [13, 3] \\
leggen     \> lei/legde     \> gelegd     \> [13, 3] \\
melken     \> molk/melkte   \> gemolken   \> [7, 15] \\ 
raden/2    \> ried/raadde   \> geraden    \> [5, 15] \\
stoten     \> stiet/stootte \> gestoten   \> [5, 15] \\
vragen     \> vroeg/vraagde \> gevraagd   \> [13, 3] \\
waaien     \> woei/waaide   \> gewaaid    \> [13, 3] \\
wassen/2   \> wies/waste    \> gewassen   \> [5, 15] \\
zeggen     \> zei/zegde     \> gezegd     \> [1, 3]  \\
zweren/2   \> zwoor/zweerde \> gezworen   \> [7, 15] \\
\end{tabbing}

Note:
Irregular verbs can have extra past forms, but do not always have
more conjugationsclasses (extra forms can be added easily for irregular
verbs). Example: {\em willen} with: {\em wou} and {\em wilde} (class 1).

The verb {\em varen} has a second past form ({\em vaarde}) when it is not used 
in its simple form. Compare with: {\em gelachen} versus: {\em geglimlacht}, 
{\em gegrimlacht}, etc.

Also there seems to be difference in meaning between {\em zegde} en {\em zei};
{\em zegde} is always used in compound verbs or verbs combined with particles,
whereas the use of {\em zei} is limited to the simple verb {\em zeggen}. 
Compare:
{\em toezeggen}, {\em afzeggen}, etc. vs. {\em zeggen}.

The form {\em vraagde} will presumably not be correct for every native 
speaker of Dutch.

\item Extra form for the 0th and 1st person singular, and for the imperative 
singular (verbs of class 14):

\begin{tabbing}
{\bf verb:}   \ \ \ \ \ \ \ \ \ \ \ \ \      \=   
{\bf extra form:}  \ \ \ \ \ \ \ \ \ \ \ \ \ \    \=    
{\bf class:} \\
                  \>               \>                   \\
glijd             \> glij          \> [7, 14] \\
rijd              \> rij           \> [7, 14] \\
snijd             \> snij          \> [7, 14] \\
houd              \> houd          \> [5, 14] \\
\end{tabbing}

Note: of course variant forms of verbs can exist for class 8 (like {\em berij},
{\em doorsnij}, or 6 {\em onthou}.
\end{itemize}

\paragraph{Different verbs with identical stems}

Sometimes, two different verbs have identical stems. This doesn't mean that
they have similar conjugation too. List of verbs with
identical stems, but different conjugationclasses:

\begin{tabbing}
{\bf verb:}   \ \ \ \ \ \ \ \ \ \ \ \      \=   
{\bf past tense:}  \ \ \ \ \ \ \ \ \ \ \ \  \=    
{\bf past participle:}  \ \ \ \ \ \ \  \=
{\bf class:} \ \ \= 
{\bf meaning:} \\
           \>             \>              \>        \>          \\
brouwen/1  \> brouwde     \>  gebrouwen   \>  [15]  \> {\em (bier) bereiden} \\
brouwen/2  \> brouwde     \>  gebrouwd    \>  [3]   \> 
{\em spec. uitspreken van `r'} \\
           \>             \>              \>        \>          \\
krijten/1  \> kreet       \>  gekreten    \>  [7]   \> {\em luid roepen} \\
krijten/2  \> krijtte     \>  gekrijt     \>  [3]   \> 
{\em met krijt bewerken} \\
           \>             \>              \>        \>          \\
plegen/1   \> placht      \>      -       \>  [13]  \> {\em gewoon zijn} \\
plegen/2   \> pleegde     \>  gepleegd    \>  [3]   \> {\em doen, bedrijven} \\
           \>             \>              \>        \>          \\
pluizen/1  \> ploos       \>  geplozen    \>  [7]   \> {\em onderzoeken} \\
pluizen/2  \> pluisde     \>  gepluisd    \>  [3]   \> {\em pluisjes afgeven} \\
           \>             \>              \>        \>          \\
prijzen/1  \> prees       \>  geprezen    \>  [7]   \> {\em loven} \\
prijzen/2  \> prijsde     \>  geprijsd    \>  [3]   \> 
{\em van 'n prijs voorzien} \\
           \>             \>              \>        \>          \\
raden/1    \> raadde      \>  geraden     \>  [15]   \> {\em gissen} \\
raden/2    \> ried/raadde \>  geraden     \>  [15,5] \> {\em raad geven} \\
           \>             \>              \>        \>          \\
scheppen/1 \> schiep      \>  geschapen   \>  [11]  \> {\em maken} \\
scheppen/2 \> schepte     \>  geschept    \>  [3]   \> {\em putten} \\
           \>             \>              \>        \>          \\
scheren/1  \> schoor      \>  geschoren   \>  [7]   \> {\em afsnijden} \\
scheren/2  \> scheerde    \>  gescheerd   \>  [3]   \> {\em laag vliegen} \\
scheren/3  \> schoor/scheerde \> geschoren/gescheerd \>  [7,3] \> 
{\em spannen} \\
           \>             \>              \>        \>          \\
schrikken/1  \> schrok      \>  geschrokken \>  [7]   \> {\em bang worden} \\
schrikken/2  \> schrikte    \>  geschrikt   \>  [3]   \> {\em plots afkoelen} \\
           \>             \>              \>        \>          \\
schuilen/1 \> school      \>  gescholen   \>  [7]   \> {\em zich verbergen} \\
schuilen/2 \> schuilde    \>  geschuild   \>  [3]   \> 
{\em beschutting zoeken} \\
           \>             \>              \>        \>          \\
stijven/1  \> steef       \>  gesteven    \>  [7]   \> {\em stijf maken} \\
stijven/2  \> stijfde     \>  gestijfd    \>  [3]   \> {\em sterken} \\
           \>             \>              \>        \>          \\
wassen/1   \> wies        \>  gewassen    \>  [5]   \> {\em groeien} \\
wassen/2   \> wies/waste  \>  gewassen    \>  [5,15] \> {\em schoonmaken} \\
wassen/3   \> waste       \>  gewast      \>  [3]   \> {\em in de was zetten} \\
           \>             \>              \>        \>          \\
wuiven/1   \> woof/wuifde \>  gewoven/gewuifd \> [7,3] \> 
{\em (met hand) zwaaien} \\
wuiven/2   \> wuifde      \>  gewuifd     \>  [3]   \> {\em zwenken} \\
           \>             \>              \>        \>          \\
zinnen/1   \> zon         \>  gezonnen    \>  [7]   \> {\em peinzen} \\
zinnen/2   \> zinde       \>  gezind      \>  [3]   \> 
{\em naar de zin zijn van} \\ 
           \>             \>              \>        \>          \\
zweren/1   \> zwoer       \>  gezworen    \>  [11]  \> {\em een eed afleggen} \\
zweren/2   \> zwoor/zweerde \>  gezworen  \>  [7,15] \> {\em etteren}         \\
\end{tabbing}

This list is based mainly on the `grammatical compendium' of Van Dale. The 
alfabetical section and the `grammatical compendium' of Van Dale differ 
considerably, and therefore, a full investigation of the alfabetical section 
will result in another list.

\subsection{Tense and modus}

The attribute `tense' in ROSETTA has three values: 
`presenttense', `pasttense' and `omegatense'.
The values `presenttense' and `pasttense' can be combined with the values
`indicative' and `subjunctive' (for the attribute `modus') only. 

The attribute `modus' in ROSETTA has the following values: 
`indicative', `subjunctive', `imperative', `infinitive', `prespart', 
`pastpart', and `omegamodus'.

\subsection{Number and persons}

The attribute `number' in ROSETTA has three values: 
`plural', `singular' and `omeganumber'.
The following `persons' exist in Dutch morphology of ROSETTA: 
\begin{itemize}
  \item singular:
    \begin{itemize} 
      \item [0] second person singular in present 
                (but only in cases of reverse order),
      \item [1] first person singular
      \item [2] second person singular
      \item [3] third person singular
      \item [4] `u'-form singular
      \item [5] `gij'-form singular
    \end{itemize}
  \item plural:
    \begin{itemize}
      \item [1] first person plural
      \item [2] second person plural
      \item [3] third person plural
      \item [4] `u'-form plural
      \item [5] `gij'-form plural
  \end{itemize}
\end{itemize}

The attribute `persons' may contain a {\em set} of the above given values.


\subsection{eORenForm}

The attribute `eORenForm' has three values: `NoForm', `eForm' and `enForm'.
The values `eForm' and `enForm' appear only in case of past participles and 
present participles. For instance: {\em gevraagde} and {\em slapende} have 
the value `eForm', {\em gevraagden} and {\em slapenden} have the value `enForm'.

The morphology only makes the forms: in case of the value `eForm' it
adds {\em e}, and in case of `enForm' it adds {\em en}. 
In case of `NoForm', nothing is added. The decision, whether or not 
{\em e} or {\em en} should be added, is beyond the reach of morphology because 
it depends on the context of a word (like: the 
preceding article, etc.). Therefore, the M-grammar must yield the value of
`eORenForm'. 
In case of {\em -e} or {\em -en} suffixes for past participles, 
the conjugationclass of the verb is of importance, because
irregularities appear when the past participle ends in {\em -en} instead of 
{\em -d} or {\em -t}. 
Compare: {\em de gesteld{\em e} vraag}, {\em het bedacht{\em e} verhaal}, and:
{\em de verzonnen kwestie}.
 
The following cases exist:

\begin{itemize}
  \item The conjugationclasses 5, 6, 7, 8, 11, 12, 15, 16 have a past participle
        ending in {\em -en}. For these classes, 
        both `eForm' and `enForm' exist, so the morphology makes them both:
        {\em de bedrogene}, {\em het verzonnene}, {\em de verzonnenen}, {\em de
        bedrogenen}. There seems to be a problem, however; compare:

        \begin{itemize}
          \item [-] {\em Zie je die broeken? Zoek de gekrompen er maar uit.}
          \item [-] {\em Zie je die broeken? Zoek de gekrompenen (?) er maar 
                  uit.}
        \end{itemize}

        It is not always clear whether or not the {\em en}-form should be used.

        Note that -for this group- there seems to be a resemblance to 
        adjectives like `tevreden', where
        the suffix {\em -e} only occurs in nominalised forms.


        \begin{itemize}
          \item [-] {\em De bedrogen man. De} *{\em bedrogene man. 
                De bedrogen\underline{e}.}
          \item [-] {\em De tevreden man. De} *{\em tevredene man. 
                De tevreden\underline{e}.}
        \end{itemize}

  \item The conjugationclasses 3, 4, 9 and 10 have past participles ending in 
        {\em -d} or {\em -t}. Verbs of these classes can always have `eForm' 
        and `enForm':

        \begin{itemize}
          \item [class 3:] {\em de gevraagde man, de gevraagde(n), 
                            het gevraagde.}
          \item [class 9:] {\em de gebrachte man, de gebrachte(n), 
                            het gebrachte.}
        \end{itemize}

  \item The conjugationclasses 1 and 2 
        (=irregular verbs) have {\em both} past participles ending in 
        {\em -d}/{\em -t} {\em and} past particles ending in {\em -en}
        (and also in {\em -aan}: {\em gegaan}, {\em gestaan} and 
        {\em gedaan}, which do get {\em -e}, like those ending in 
        {\em -d} and {\em -t}); here, we need to enumerate the {\em keys} 
        of verbs that do have past participle ending
        in {\em -en} (the best choice to do this will probably be enumeration 
        of the complement of this set, the verbs that {\em don't} have a 
        past participle ending in {\em -en}, 
        because this is a smaller set). The morphology also makes forms like:
        {\em geweeste} and {\em gehadde}; these should be excluded by the 
        M-grammar. The same holds for some other past participles (e.g. 
        {\em gekunde}). The verb {\em zullen} doesn't have a past participle at 
        all (see: 5.1.4); this is also handled by the M-grammar.

  \item Verbs of conjugationclass 0 (e.g. {\em buikspreken}) and 13 don't 
        have a past participle, and therefore no problems will arise here.

\end{itemize}

No problems arise when {\em -e} or {\em -en} is added to present participles.

\subsection{verbs with particles}

A lot of verbs are accompanied by a particle; examples are: {\em losmaken}, 
{\em doodgaan}, {\em weglopen}, etc. The number of particles is limited
(and quite small); the number of verbs that they can accompany, however, is
not. 

In ROSETTA, the relation between particle and verb is handled by
derivation rules. Thus, it is possible to deal with complex forms like
{\em onoplosbaar}, where the particle should be sticked to the stem before the 
derivational prefix {\em on-}.

It is quite difficult to get a complete list of particles for Dutch; grammars
always give the most frequent one's only. By using the N-N tape of `Van Dale', 
it was possible to make a list of verbs, of which the -in the dictionary 
mentioned- past participle didn't begin with the inflectional prefix `ge'. 
From this set, we 
could derive a rather large set of particles. Of some verbs, however, the
past participle is not mentioned in `Van Dale'(although a lot of these verbs do 
have a complete inflection), so it's possible that we missed some particles.
A particle that was not found by using the tape was: {\em les}; 
the entry {\em lesgeven} didn't show the past participle form {\em lesgegeven}.
Of course, the set of verbs that is given by dictionaries like `Van Dale' is 
arbitrary to a certain extent: the word {\em theedrinken} is included, but 
{\em koffiedrinken}, however, is not.

In most cases, the bare stem of a verb with particle is a verb too: 
{\em losmaken} is a verb, but the bare stem of it, {\em maken} is a verb too.
The same holds for {\em doodgaan} and {\em gaan}, {\em weglopen} and 
{\em lopen}, {\em koffiedrinken} and {\em drinken}, etc. There are exceptions,
however, like: {\em aankondigen}, {\em uitmonden}, {\em opdoffen}, etc.

\subsection{verbs with prepositions bound by a GLUE}

Sometimes, prepositions can be sticked to adjacent verbs; for instance in
constructions like: {\em het bos inlopen}, {\em het kanaal overzwemmen},
{\em het meer overzeilen}, etc., and also: {\em het bos ingelopen}, 
{\em het kanaal overgezwommen}, {\em het meer overgezeild}, etc.
These prepositions are not participles:
\begin{itemize}
   \item [-] *{\em het ingelopen bos},
   \item [-] *{\em het overgezwommen kanaal},
   \item [-] *{\em het overgezeilde meer},
\end{itemize}
but postponed prepositions (compare: {\em het bos in},
{\em het kanaal over}, etc.) and therefore should not be treated as particles.
(The verb {\em inlopen} is ambiguous, however; compare: {\em het bos inlopen}
and {\em de schoenen inlopen}. In the second sentence {\em in} is particle of
the verb {\em inlopen}). The postponed prepositions can be handled by
GLUE-rules because it is impossible that the process of sticking the 
preposition to the verb is followed by any form of derivation.

\newpage

\section{Nouns}

\subsection{Plural forms}

The normal plural endings in Dutch are the suffixes {\em -s} and {\em -(e)n}; 
the majority of Dutch nouns takes one of these two suffixes (and some nouns take
both, like: {\em appel}). Other endings also exist: {\em -eren} 
({\em kinderen}), {\em -ien} ({\em koeien}, {\em vlooien}), etc.

Some nouns lack a plural form (like: {\em heelal}, {\em verdriet}), and others 
are plural themselves already (like: {\em hersenen}, {\em notulen}, 
{\em onkosten}).

The following plural forms exist (each of them is `value' to the inherent 
attribute `plurforms' of nouns in ROSETTA):

\begin{tabbing}
{\bf No:} \ \ \  \= {\bf value:} \ \ \ \ \ \ \ \ \ \ \ \ \ \ \  \= 
{\bf suffix:} \ \ \ \ \ \ \ \ \ \ \ \ \ \   \=  {\bf example: }\\
    \>                \>             \>             \\
1  \> enPlural        \> SFKen       \> boeken, appelen, koloni\"{e}n    \\
2  \> sPlural         \> SFKmvs      \> etalages, appels, boompjes       \\
3  \> aTOaaPlural     \> SFKaTOaa    \> daken, baden, verdragen          \\
4  \> aTOeePlural     \> SFKaTOee    \> steden                           \\
5  \> eTOeePlural     \> SFKeTOee    \> bevel, gebrek, weg               \\
6  \> eiTOeePlural    \> SFKeiTOee   \> waarheden, heren, schelen        \\
7  \> iTOeePlural     \> SFKiTOee    \> leden, schepen, smeden           \\
8  \> oTOooPlural     \> SFKoTOoo    \> goden, geboden, motoren          \\
9  \> erenPlural      \> SFKeren     \> goederen, lammeren, liederen     \\
10 \> ienPlural       \> SFKien      \> vlooien, koeien                  \\
11 \> denPlural       \> SFKden      \> roeden, treden                   \\
12 \> nenPlural       \> SFKnen      \> lendenen, redenen                \\
13 \> ieAccentPlural  \> SFKieAccent \> knie\"{e}n, voetbalknie\"{e}n, 
                                        antipathie\"{e}n   \\
14 \> luiPlural       \> SFKlui      \> werklui                          \\
15 \> liedenPlural    \> SFKlieden   \> brandweerlieden                  \\
16 \> LatPlural       \> SFKlat      \> cycli, schemata, matrices, spectra,
                                        bases \\
17 \> enIrregPlural   \> SFKenIrreg  \> bamboezen, blaren                \\
18 \> sIrregPlural    \> SFKsIrreg   \> vlaas, eegaas, gevoelens          \\ 
19 \> LatIrregPlural  \> SFKlatIrreg \> tempora, casus                   \\
20 \> NoPlural        \>    -        \> verdriet, heelal                 \\
21 \> OnlyPlural      \>    -        \> hersenen, notulen, onkosten      \\
\end{tabbing}

Of course, the last two groups -with the values `NoPlural' and `OnlyPlural'
differ from the rest.

Groups 3, 4, 5, 6, 7 and 8 are irregular variants of the {\em -en} plurals; all 
these groups have {\em ablaut} (change of the stemvowel). 
The groups 9, 10 ,11 and 12 are also variants of {\em -en} plural, but without 
ablaut. 
Group 13 is also a special case of {\em -en} plural. 
Groups 14 and 15 are special plurals for words 
ending in {\em -man}; group 16 consists of loan-words with their original plural.
Groups 17, 18 and 19 consist of irregular plurals.

Nouns belonging to all these groups:

\begin{itemize}
  \item [3] {\em bad-baden, bedrag-bedragen, blad-bladen, dag-dagen, dak-daken, 
            dal-dalen, gat-gaten, pad-paden, slag-slagen, stag-stagen, 
            titan-titanen, vat-vaten, verdrag-verdragen, staf-staven, 
            graf-graven, glas-glazen,} and compounds with these words as rigth
            part. 

  \item [4] {\em stad-steden} and compounds with the noun {\em stad} as rigth 
            part.

  \item [5] {\em bevel-bevelen, gebed-gebeden, gebrek-gebreken, gen-genen, 
            spel-spelen, tred-treden, weg-wegen} and compounds with these 
            nouns.

  \item [6] all words with derivational suffix {\em -heid}, as well as: 
            {\em heir-heren, scheil-schelen}.

  \item [7] {\em lid-leden, schip-schepen, smid-smeden, spit-speten, rif-reven}
            and compounds with these nouns.

  \item [8] {\em alcohol-alcoholen, deuteron-deuteronen, fenol-fenolen,
            gebod-geboden, god-goden, hof-hoven, hertog-hertogen, hol-holen, 
            lot-loten, oorlog-oorlogen, motor-motoren, schot-schoten, 
            slot-sloten, verlof-verloven, kolchoz-kolchozen, ion-ionen} 
            and their compounds.

  \item [9] {\em been-beenderen, berd-berderen, blad-bladeren, ei-eieren,
            gelid-gelederen, gemoed-gemoederen, goed-goederen, hoen-hoenderen, 
            kalf-kalveren, kind-kinderen, kleed-klederen, lam-lammeren, 
            lied-liederen, rad-raderen, rund-runderen, volk-volkeren} and
            compounds of these nouns.

            {\em Vederen} might be the plural of {\em veer}, but it is also 
            possible that {\em vederen} is the plural of {\em veder} only,
            and that {\em veren} is the (only) plural of {\em veer}.

  \item [10] {\em vlo-vlooien, koe-koeien} and compounds.

  \item [11] {\em jaargetij-jaargetijden, la-laden, roe-roeden, ree-reden, 
             stee-steden, tree-treden} and compounds.

             The members of this group have variant forms with the same 
             plural form (but regular!): {\em lade-laden, 
             roede-roeden, rede-reden, stede-steden, trede-treden}.

  \item [12] {\em regio-regionen, lende-lendenen, rede-redenen} (e.g. 
             {\em lijkrede}) and compounds.

             The members of group 12 have other plural forms too: 
             {\em regio-regio's/regiones, lende-lenden, rede-redes}.

  \item [13] consists of nouns ending in the {\em stressed} syllable {\em -ie}: 
             {\em knie-knie\"{e}n, antipathie-antipathie\"{e}n}; 
             also compounds with 
             this kind of nouns as the rightmost part belong to this group: 
             {\em voetbal{\em knie}-voetbal{\em knie\"{e}n}}, etc. 

             The members of this group differ from the nouns ending in 
             {\em -ie} but 
             {\em without} stress, which have regular {\em -en}-ending: 
             {\em kolonie-koloni\"{e}n, referentie-referenti\"{e}n}, etc.
             Another difference is formed by the fact that nouns of group 13 
             never have `sPlural', in contrast with nouns ending in stressless 
             {\em -ie}, that normally have `sPlural' too: *{\em knies}, 
             *{\em antipathies}, 
             but: {\em kolonies, referenties}. 

             A third group of nouns ending in {\em -ie} is formed by words like 
             {\em fraaie}, {\em mooie}, etc. These words have a sjwa 
             (the {\em e}), and therefore, the {\em ie}-syllable is in fact a 
             diphthong. The plural of these words can be formed regularly (the 
             phonological attribute `sjwa' can be used to distinguish 
             stressless {\em -ie} with sjwa from stressless {\em -ie} 
             without sjwa).

  \item [14] consists of nouns ending in {\em -man} with plural ending 
             {\em -lui} that comes instead of {\em -man}: {\em werkman-werklui}.

  \item [15] consists of nouns ending in {\em -man} with plural ending 
             {\em -lieden} that comes instead of {\em -man}: 
             {\em brandweerman-brandweerlieden}, etc.

  \item [16] consists of original plural forms of loan-words, that have become 
             common in Dutch. Most of the plural forms stem from Latin or 
             Greek, but plurals of words from other languages have been added
             too. For convenience sake, the complete group has been called 
             `LatPlural' (=Latin Plural). The following plural forms belong to 
             this group:
    \begin{itemize}
      \item [a] words ending in {\em -ma}, where {\em -ma} changes into 
                {\em -mata} ({\em schema-schemata}),
      \item [b] words ending in {\em -ca}, {\em da}, {\em ga}, {\em ia}, 
                {\em la}, {\em na}, {\em ra}, 
                {\em sa}, {\em ta}, and {\em va}, 
                which get an extra {\em -e} ({\em mensa-mensae}),
      \item [c] words ending in {\em -e}; {\em -e} changes into {\em -ia} 
                ({\em impersonale-impersonalia}),
      \item [d] words ending in {\em -aal}; {\em -aal} changes into {\em -alia} 
                ({\em regaal-regalia}),
      \item [e] words ending in {\em -um}; {\em -um} changes into {\em -a} 
                ({\em datum-data}),
      \item [f] words ending in {\em -on}; {\em -on} changes into {\em -a} 
                ({\em protozo\"{o}n-protozoa}),
      \item [g] words ending in {\em -en}; {\em -en} changes into {\em -ina} 
                ({\em examen-examina}),
      \item [h] words ending in {\em -o}; {\em -o} changes into {\em -i} 
                ({\em saldo-saldi}),
      \item [i] words ending in {\em -or}; {\em -or} changes into {\em -ores} 
                ({\em pastor-pastores}),
      \item [j] words ending in {\em -aur}; {\em -aur} changes into {\em -auri} 
                ({\em centaur-centauri}),
      \item [k] words ending in {\em -uur}; {\em -uur} changes into {\em -ures} 
                ({\em paruur-parures}),
      \item [l] words ending in {\em -as}; {\em -as} changes into {\em -ates}
                ({\em civitas-civitates}),
      \item [m] words ending in {\em -is}; {\em -is} changes into {\em -es} 
                ({\em basis-bases}),
      \item [n] words ending in {\em -ns}; {\em -ns} changes into {\em -ntia} 
                ({\em reagens-reagentia}),
      \item [o] words ending in {\em -os}; {\em -os} changes into {\em -oi} 
                ({\em topos-topoi}),
      \item [p] words ending in {\em -ps}; {\em -ps} changes into {\em -pora} 
                ({\em corps-corpora}),
      \item [q] words ending in {\em -rs}; {\em -rs} changes into {\em -rtes} 
                ({\em pars-partes}),
      \item [r] words ending in {\em -us}; {\em -us} changes into {\em -i}
                ({\em cyclus-cycli}),
      \item [s] words ending in {\em -nt}; {\em -nt} changes into {\em -ntia} 
                ({\em deodorant-deodorantia}),
      \item [t] words ending in {\em -ut}; {\em -ut} changes into {\em -ita} 
                ({\em caput-capita}),
      \item [u] words ending in {\em -ex}; {\em -ex} changes into {\em -ices} 
                ({\em codex-codices}),
      \item [v] words ending in {\em -ix}; {\em -ix} changes into {\em -ices} 
                ({\em matrix-matrices}),
      \item [w] words ending in {\em -ox}; {\em -ox} changes into {\em -oces} 
                ({\em vox-voces}),
      \item [x] words ending in {\em -ux}; {\em -ux} changes into {\em -uces} 
                ({\em crux-cruces}).
    \end{itemize}
  \item [17] consists of irregular {\em -en} plurals: {\em bamboe-bamboezen, 
             blad-blaren} and compounds.
  \item [18] consists of irregular {\em -s} plurals: {\em la-laas, ra-raas, 
             vla-vlaas, eega-eegaas, gevoel-gevoelens}, etc. and compounds.
  \item [19] consista of irregular `latinate' plurals: {\em stotinka-stotinki, 
             mamma-mammae, regio-regiones, virgo-virgines, homo-homines, 
             dactylus-dactylen, genus-genera, opus-opera, tempus-tempora,
             corpus-corpora, jus-jura, modulus-modulen, casus-casus, 
             lapsus-lapsus, singularis-singularia, pluralis-pluralia, 
             lapis-lapides, pelvis-pelvis, glacis-glacis, permis-permis, 
             hospes-hospites, praeses-praesides, genius-geni\"{e}n, epos-epen, 
             mecenas-mecenaten, maecenas-maecenaten, simplex-simplicia},
             and compounds.
\end{itemize}

Some nouns, like {\em heir} and {\em rif} have variant forms: {\em heer} 
and {\em reef}.
Note that the mentioned pluralforms ({\em heren} and {\em reven}) 
are regular with respect to these variants.

Many words have more than one pluralform, like {\em volk}, which has the regular 
plural form {\em volken} and the irregular form {\em volkeren}.

Many pluralforms are ambigious: {\em staven} is plural of both {\em staaf} and 
{\em staf}, {\em sloten} of both {\em slot} and {\em sloot}, {\em treden} of 
{\em trede} and {\em tree}, etc.:

One should note that it is possible that many of the groups mentioned above
can be incomplete: it is hard to
find all the nouns with a certain plural form. Often no systematic strategy is
possible except of checking each noun of the dictionary separately. Besides, in
a certain way every group of plurals is infinite: almost any noun can be used to
form compounds with. In general, the compound has the same plural form(s) as its
rightmost part has when it occurs as `simple' noun. Thus, in the segmentation 
rules, every string starts with a `$\star$' 
and no other measures have to be taken.

(Basic) nouns that are plural form themselves (like {\em hersenen}) have the 
value
`OnlyPlural' and of course, cannot have other plural forms; nouns that can occur
as singular only have the value `NoPlural'. Consequently, the values
`OnlyPlural' and `NoPlural' never combine with other values. 

A consequence of this solution is that the full plural form of the 
nouns with `OnlyPlural' is listed in 
the dictionary; an alternative solution would be to put an (artificial) 
singular (like {\em hersen}) in the dictionary, and to use the normal rules for
plural forms (which would be obligatory for these nouns, of course). A 
motivation for this alternative would be forms like {\em hersenstam}, 
{\em hersentumor}, etc., where there seems to be compounding with the stem 
{\em hersen}. To my opinion, this is {\em not} a regular proces, because such
compounds do not exist for many other nouns with `OnlyPlural' (like {\em 
notulen, onkosten}). Besides, there is no guarantee that {\em hersen-} in 
{\em hersenstam} has anything to do with the singular form of {\em hersenen}; 
often the compounding-stem differs from the singular: {\em kinderarts} instead 
of {\em kindarts}, {\em raderwerk} instead of {\em radwerk}, {\em vlooiebeet}
instead of {\em vlobeet}, {\em scheepsarts} instead of {\em schipsarts}, etc.
Therefore, there is no reason to choose this alternative. The way the nouns
with `OnlyPlural' have been handled now is elegant because the dictionary 
contains only basic nouns that -as dictionary form- can be found in Dutch 
sentences, and it does not contain any `abstract' forms.





\subsection{Number}

The attribute `number' has the values: `singular' (for singular nouns), 
`plural' (for plural nouns) and `omeganumber' (when `number' is irrelevant).


\subsection{Genitive forms (attributes `possgeni' and `geni')}

The attribute `possgeni' says whether or not the noun can get a genitive form.
It has the values `true' and `false'.

The attribute `geni' is `true' when the noun has the genitive-form. The 
genitive in Dutch is formed by:
\begin{itemize}
  \item the ending {\em -s}, after: {\em b} , {\em c}, {\em d}, {\em e},
        {\em f}, {\em g}, {\em h}(but only if preceded by a consonant), 
        {\em k}, {\em l}, {\em m}, {\em n}, {\em p}, {\em q}, {\em r}, {\em t}, 
        {\em v}, {\em w}, {\em y}(but only if preceded by a vowel);
  \item an apostrophe (after {\em sj}, {\em s}, {\em x}, {\em z});
  \item by {\em -'s} (after {\em a}, {\em \'{e}}, {\em h}(but only if preceded 
        by a vowel), {\em i}, {\em ij}, {\em o}, {\em u} and 
        {\em y}(but only if preceded by a consonant).
\end{itemize}

The genitive suffix is marked by `SFKgens'; examples can be found in the 
document that contains the Dutch rules.

In general, only proper names (Jan, Tineke, Jans, Cruijff), words that 
express family-relations (vader, tante, ouders) or words that can be used to 
accost a person (dominee, buurman) can get genitive form in Dutch. Plural
genitive forms are rare ({\em mijn ouders' huis}, 
{\em mijn zoontjes' rapporten}),
but not impossible. Genitive of a plural form seems to be possible only when the
plural ending is a {\em -s}: *{\em mijn zonens rapporten} vs. 
{\em mijn zoons' rapporten}.
Therefore, the W-rules accept genitive forms of plurals only when the
plural is formed by (regular or irregular) s-plural.


\newpage
\section{Adjectives and adverbs}

This section is about adjectives and adverbs. All attributes are given for
adjectives; adverbs have the same attributes (with the same values), except for:
`uses', `eFormation', `eNominalisation' and `sFormation', which they have not.

\subsection{Form}

The attribute `form' is an non-inherent attribute of adjectives and adverbs,
that shows the form of the adjective: positive ({\em verstandig}), comparative
({\em verstandiger}), superlative ({\em verstandigst}) and a special type 
of superlative, the allersuperlative ({\em allerverstandigst}); both positive
and comparative can have a special form: sPositive ({\em verstandigs}) and 
sComparative ({\em verstandigers}). Adjectives and
adverbs occur in positive-form in the dictionary; the other forms are derived
by inflectional suffixes (like: {\em -er} for comparatives, {\em -st} for 
superlatives, 
etc.) or by words like {\em meer} (for comparatives) or {\em meest} (for 
superlatives).


\subsection{Uses}

The inherent attribute `uses' is used for adjectives only and tells us whether 
or not the adjective can function {\bf attributively} (e.g. as modifier of a 
noun), {\bf predicatively} (e.g. as a subject complement) or 
{\bf nominalised} (as a noun). The distinction between these three
functions of adjectives is important because only attributively used adjectives
can get {\em -e} ending and only nominalised adjectives have both {\em -e} and 
{\em -en} 
endings (see also 7.3. and 7.6.).

Adjectives that cannot be used predicatively are for instance: 
{\em huidig, vaderlands}; examples of adjectives that can be used only 
predicatively are: {\em bekaf, onwel}.

\subsection{eORenForm}

The non-inherent attribute `eORenForm' has the values `NoForm', `eForm' 
and `enForm'; it has 
`eForm' when the adjective has {\em -e} (like: {\em grote} and {\em grootste}),
`enForm' when it has {\em -en} (like in nominalised forms: {\em groten}, 
`grootsten')
and `NoForm' when it has neither {\em -e} nor {\em -en} (like: {\em groot} and 
{\em grootst}).

Adverbs have this attribute because of constructions like 
{\em het liefst\underline{e}} (from the adverb {\em graag}) and 
{\em het vaakst\underline{e}}. The value `enForm' doesn't occur for adverbs.
Because adverbs have `eForm' only in combination with (aller)superlatives
preceded by {\em het}, and all (aller)superlatives end in {\em -st}, 
no adverbs are excluded from getting this {\em -e}. Therefore we don't need
inherent attributes like `eFormation' or `eNominalisation' for adverbs.

\subsection{Comparatives}

The attribute `comparatives' contains information about the way the 
comparative of a adjective (or an adverb) can be formed; sometimes, adjectives 
have several comparative forms. The attribute has a set as value; this set 
contains one or more of the following values:

\begin{description}
  \item [erComp]: for regular comparatives with: {\em -(d)er}; segmentation 
                  rules with SFKer.
  \item [erIrregComp]: for irregular comparatives; segmentation rules with 
                  SFKonreger.
  \item [meerComp]: for comparatives with {\em meer}; there are no special 
                  segmentation rules for this type of comparative, because
                  the word {\em meer} doesn't interact with the string of the 
                  adjective.
  \item [NoComp]: adjective (or adverb) doesn't have a comparative at all.

\end{description}

Examples: {\em leuker} is the regular comparative of {\em leuk} (`erComp');
  {\em grover} and {\em beter} are irregular comparatives (`erIrregComp') 
  of resp. {\em grof} and {\em goed};
  {\em meer tevreden} and {\em meer priv\'{e}} are `meer'-comparatives 
  (`meerComp') of   resp. {\em tevreden} and {\em priv\'{e}}. 
  Adjectives like {\em plastic, houten, huidig}, etc. do not have a comparative 
  (`NoComp'). 


\subsection{Superlatives}

The attribute `superlatives' contains information about the way the 
superlative of an adjective (or an adverb) can be formed; sometimes, adjectives 
have several superlative forms. The attribute has a set as value; this set 
contains one or more of the following values: 

\begin{description}
  \item [stSup]: for regular superlatives with: {\em -st}; 
                 segmentation rules with SFKst.
  \item [stIrregSup]: for irregular superlatives; segmentation rules 
                 with SFKonregst.
  \item [allerSup]: for regular superlatives with: prefix 
                 {\em aller-}(SFKaller) + {\em -st}(SFKst).
  \item [allerIrregSup]: for irregular superlatives with {\em aller-} and 
                 {\em -st}.
  \item [meestSup]: for superlatives with {\em meest} (no special segmentation 
                 rules).
  \item [NoSup]: adjective (or adverb) doesn't have a superlative at all.
\end{description}

The values with `allerSup' and `allerIrregSup' have been introduced because
of the fact that, in principle, derivation should take place {\em before} 
inflection. Superlatives with {\em aller-} seem to be a counterexample,
because forms other than superlatives cannot have {\em aller-}:
*{\em allermooi}, *{\em allermooier}, *{\em allermoois}, etc.
As a ad hoc solution, {\em aller-} is treated as a inflectional suffix,
comparable to {\em ge-}, but this is not really sufficient, because 
{\em aller-} can be sticked to a superlative more than once:
{\em allerallergrootst, allerallerallergrootst}, etc. (but it is not clear 
whether or not the {\em -st}-ending has also been applied repeatedly,
because probably the rule `$\star$st + SFKst :: $\star$st' holds). 
These forms are not possible in the current version.

Examples of superlatives: 
{\em leukst} is the regular superlative of {\em leuk} (`stSup'), 
{\em best} is the irregular superlative of {\em goed} (`erIrregSup'), 
{\em meest priv\'{e}} is the
`meest'-superlative of {\em priv\'{e}} (`meestSup') and adjectives like 
{\em houten}, {\em gouden} do not have a superlative form (`NoSup').

Examples of allersuperlatives: {\em allerleukst}, {\em allerbest}.
Of course, adjectives with the value `NoSup' can't have an allersuperlative.

\subsection{sFormation}

The attribute `sFormation' says whether or not the adjective can get a `sForm';
it has the values `false' and `true'.


\subsection{eFormation and eNominalisation}

The adjectives can be split into three types:

\begin{description}
  \item [group 1:] Adjectives ending in: {\em -a, -c, -o, -en, -\'{e},
        -i,-er, -im, -is}, like: {\em lila, plastic, macho, franco, gouden,
        houten, priv\'{e}, doubl\'{e}, gummi, quasi, rechter, rubber,
        interim, gratis}. These adjectives do not have comparative or
        superlative forms. They never get the suffix {\em -e}   
        or {\em -en} when used attributively or nominalised:

\begin{description}
  \item [] {\em De mica tafel.}
  \item [] {\em Een gouden armband.}
  \item [] {\em Geef mij de linker maar.}   (nominalised)
  \item [] {\em De houten is van hem.}      (nominalised)
\end{description}

  \item [group 2:] Adjectives ending in {\em -y, -en}, like 
        {\em trendy, sexy} and {\em tevreden, ervaren}. 

        These adjectives 
        have {\em -e} in nominalised forms: {\em de tevredene,
        de tevredenere}, and in superlative when used attributively:
        {\em de tevredenste man}.

        They do not have {\em -e} in comparative, positive when used 
        attributively:
        {\em de tevreden man} (vs. *{\em de tevredene man}) and:
        {\em een ervarener man} (vs. *{\em een ervarenere man}).

        When the superlative is formed with {\em meest}, the adjective 
        is treated as positive, so it doesn't get {\em -e} when used 
        attributively: {\em de meest ervaren man} (vs. *{\em de meest ervarene
        man}).

        The suffix {\em -en} can be used whenever {\em -e} can be used in 
        nominalised adjectives.

  \item [group 3:] Adjectives ending in {\em -er, -e}, like:
        {\em zeker, lekker, luxe, timide, chique} and others, like
        {\em verstandig, leuk, mooi, goed}, etc. These adjectives get {\em -e}
        both when attributively used and when nominalised used. 

        A special group is formed by words on 
        {\em -e} (=sjwa): {\em luxe}, {\em timide}, etc.; these adjectives are 
        regular, but the extra {\em -e} in the positive and nominalised 
        positive cannot stick to the stem because the stem ends in {\em -e} 
        already; this implies that for this group there is a
        segmentation rule like:

        \begin{itemize}
             \item [\ ]  $\star$e   + SFKe  :: $\star$e, FONsjwasjwa;
        \end{itemize}

\end{description}


Note: the fact that {\em het sexy\"{e}} is accepted in Dutch, but
{\em de sexy\"{e}} 
(referring to a person) seems to be unacceptable, is not considered in the 
morphology at the moment.

The {\em -en}-ending for nominalised forms is possible whenever 
`eNominalisation'
is `true': {\em de zekeren}, {\em de tevredenen}, {\em de verstandigen}, etc. 
(the fact that {\em de sexy\"{e}n}, {\em de trendy\"{e}n}, etc. seem to be 
unacceptable is (again) not considered at the moment).

Note that `eFormation' and `eNominalisation' only work for positive forms;
it seems that comparatives can always have both forms: {\em de ervarener man, 
de ervarenere man} (although the opinions of native speakers about these 
facts vary).

Note that the attributes eNominalisation and eFormation go in fact partly 
beyond the reach of morphology, because it is impossible find out whether or 
not a form is nominalised or attributively used.


\newpage

\section{Other categories}

Until now, we have seen the way the Dutch morphology deals with the main
categories: verb, noun, adjectives and adverbs. Of course, all other categories
are handled in morphology too; most of these, however, are treated trivially:
the only thing that is done, is: linking BCAT and CAT without change of form. 
In a few cases more happens than a simple linking process; in the next
sections, these categories will be paid attention to. A SUBCAT level is
not needed for other categories than verb, noun, adjectives and adverbs, 
because derivation is limited to the main categories.


\subsection{Propernouns}

Propernouns have only two `levels': BPROPERNOUN and PROPERNOUN. The only type 
of inflection of propernouns is the genitive form; the relevant attributes,
`possgeni' and `geni', work the same as those for nouns.

Forms like {\em het zonnige Itali\"{e}}, {\em het romantische Duitsland},
{\em de Kennedy's}, {\em de beide Duitslanden}, etc. are treated as derivation:
a derivation rule will make a SUBNOUN out of a PROPERNOUN. This also explains
why BPROPERNOUNs have attributes like e.g. `gender': this inherent attribute 
guarantees that the SUBNOUN has `gender' too (and therefore has the same 
properties as a SUBNOUN derived from a BNOUN.

Problematic are constructies like: {\em de Antillen}, {\em de Fillipijnen},
{\em de Verenigde Staten}, etc. They will be treated as fixed idioms.



\subsection{PERSPRO's}

PERSPRO's will be derived from BPERSPRO's, which are in the dictionary; in 
Dutch, the following BPERSPRO's exist (with inherent attributes: `number', 
`gender', `person' and, of course, `key'):

\begin{tabbing}
BPERSPRO: \ \ \  \= number: \ \ \ \ \ \ \ \ \  \= gender:  
 \ \ \ \ \ \ \ \ \ \ \    \= person: \ \  \ \ \ \ \ \ \  \= key: \\
          \>          \>            \>         \>      \\
ik        \> singular \> omegagender \> \ \ \  1 \>    \ldots \\
jij       \> singular \> omegagender \> \ \ \  2 \>    \ldots \\
u         \> singular \> omegagender \> \ \ \  4 \>    \ldots \\
gij       \> singular \> omegagender \> \ \ \  5 \>    \ldots \\
hij       \> singular \> mascgender  \> \ \ \  3 \>    \ldots \\
zij       \> singular \> femgender   \> \ \ \  3 \>    \ldots \\
het       \> singular \> neutgender  \> \ \ \  3 \>    \ldots \\
          \>          \>            \>         \>      \\
wij       \> plural \>  omegagender \> \ \ \  1 \>     \ldots \\
jullie    \> plural \>  omegagender \> \ \ \  2 \>     \ldots \\
u         \> plural \>  omegagender \> \ \ \  4 \>     \ldots \\
gij       \> plural \>  omegagender \> \ \ \  5 \>     \ldots \\
zij       \> plural \>  omegagender \> \ \ \  3 \>     \ldots \\
\end{tabbing}

and the following generic used BPERSPRO's:

\begin{tabbing}
BPERSPRO: \ \ \  \= number: \ \ \ \ \ \ \ \ \  \= gender:  
 \ \ \ \ \ \ \ \ \ \ \    \= person: \ \  \ \ \ \ \ \ \  \= key: \\
          \>          \>            \>         \>      \\
je        \> singular \>  omegagender \> \ \ \  2 \>      \ldots \\ 
ze        \> plural   \>  omegagender \> \ \ \  3 \>      \ldots \\ 
\end{tabbing}


Note that some of the BPERSPRO's have the same (string)form; because the 
attribute `number' always differs for such homonyms, it is easy to 
distinguish between them. Such homonyms also occur in forms that are derived 
from BPERSPRO's: {\em uw} (singular/plural), {\em uwe} (singular/plural), etc. 
As a 
consequence, suffix-keys for pronouns are split into two groups; one group
consists of suffix-keys that work for singular forms only, the other of
suffix-keys that work for plurals only.

All PERSPRO's can be derived from one of these BPERSPRO's. PERSPRO's also have
two non-inherent attributes: `persprocases' (with the values `nominative', 
`dative' and
`accusative') and `reduced' (with the values `true' and `false'). In Dutch, the
following PERSPRO's exist:

\begin{itemize}
  \item The following PERSPRO's are {\em directly} derived (compare with the 
        BPERSPRO's in the last table):

\begin{tabbing}
PERSPRO: \ \ \  \= from: \ \ \ \ \ \ \ \ \= persprocases: \ \ \ \ \ \ \  
\ \ \ \ \ \  \= reduced: \\
         \>                \>                    \>       \\
ik       \> ik             \> nominative         \> false \\
jij      \> jij            \> nominative         \> false \\
u        \> u              \> nominative         \> false \\
gij      \> gij            \> nominative         \> false \\
hij      \> hij            \> nominative         \> false \\
zij      \> zij            \> nominative         \> false \\
het      \> het            \> nominative         \> false \\
         \>                \>                    \>       \\
wij      \> wij            \> nominative         \> false \\
jullie   \> jullie         \> nominative         \> false \\
u        \> u              \> nominative         \> false \\
gij      \> gij            \> nominative         \> false \\
zij      \> zij            \> nominative         \> false \\
\end{tabbing}

and the following generic used PERSPRO's:

\begin{tabbing}
PERSPRO: \ \ \  \= from: \ \ \ \ \ \ \ \ \= persprocases: \ \ \ \ \ \ \  
\ \ \ \ \ \  \= reduced: \\
          \>               \>                    \>       \\
je        \> je            \>  nominative        \> true \\ 
ze        \> ze            \>  nominative        \> true \\ 
\end{tabbing}


  \item Some nominative forms are derived from the 
        BPERSPRO's {\em with suffix-keys} (note: the forms
        {\em ikke}, {\em jelui} and {\em gijlieden} can only be 
        analysed and \underline{not} generated):

\begin{tabbing}
PERSPRO: \ \ \  \= from: \ \ \ \ \ \ \ \ \= persprocases: \ \ \ \ \ \ \  
\ \ \ \ \ \  \= reduced: \\
         \>                \>                    \>       \\
'k       \> ik             \> nominative         \> true  \\
ikke     \> ik             \> nominative         \> false \\
je       \> jij            \> nominative         \> true  \\
ge       \> gij            \> nominative         \> true  \\
ie       \> hij            \> nominative         \> true  \\
ze       \> zij            \> nominative         \> true  \\
't       \> het            \> nominative         \> true  \\
         \>                \>                    \>       \\
we       \> wij            \> nominative         \> true  \\
jelui    \> jullie         \> nominative         \> false \\
ge       \> gij            \> nominative         \> true  \\
gijlieden\> gij            \> nominative         \> false \\
ze       \> zij            \> nominative         \> true  \\
\end{tabbing}

  \item Also, {\em accusative} and {\em dative} forms are derived from the 
        BPERSPRO's {\em with suffix-keys} (note: the forms
        {\em 'r}, {\em ze} (as alternative for {\em haar}), 
        {\em hen} (as dative) and {\em hun} (as accusative) can only be 
        analysed and \underline{not} generated):

\begin{tabbing}
PERSPRO: \ \ \  \= from: \ \ \ \ \ \ \ \ \= persprocases: \ \ \ \ \ \ \  
\ \ \ \ \ \  \= reduced: \\
         \>                \>                    \>       \\
mij      \> ik             \> dative or accusative  \> false \\
me       \> ik             \> dative or accusative  \> true  \\
jou      \> jij            \> dative or accusative  \> false \\
je       \> jij            \> dative or accusative  \> true  \\
u        \> u              \> dative or accusative  \> false \\
hem      \> hij            \> dative or accusative  \> false \\
'm       \> hij            \> dative or accusative  \> true  \\  
haar     \> zij            \> dative or accusative  \> false \\
'r       \> zij            \> dative or accusative  \> true  \\
d'r      \> zij            \> dative or accusative  \> true  \\
ze       \> zij            \> dative or accusative  \> true  \\
het      \> het            \> dative or accusative  \> false \\
't       \> het            \> dative or accusative  \> true  \\  
         \>                \>                       \>       \\
ons      \> wij            \> dative or accusative  \> false \\
jullie   \> jullie         \> dative or accusative  \> false \\
u        \> u              \> dative or accusative  \> false \\
hen      \> zij            \> accusative            \> false \\
hun      \> zij            \> dative                \> false \\
hen      \> zij            \> dative                \> false \\
hun      \> zij            \> accusative            \> false \\
ze       \> zij            \> dative or accusative  \> true  \\
\end{tabbing}

and the following generic used PERSPRO's:

\begin{tabbing}
PERSPRO: \ \ \  \= from: \ \ \ \ \ \ \ \ \= persprocases: \ \ \ \ \ \ \  
\ \ \ \ \ \  \= reduced: \\
          \>         \>                       \>       \\
je        \>     je  \>  dative or accusative \> true \\ 
ze        \>     ze  \>  dative or accusative \> true \\ 
\end{tabbing}

{\em Gij} has nominative forms only; these are used in analysis (only). The 
accusative
and dative forms of {\em gij}, as well as the from {\em gij} 
derived POSSADJ's, are the
same as for {\em u}; in all these cases, however, we will analyse them as 
forms of 
{\em u} only, because forms with {\em gij} are not very usual in Dutch. In 
generation, only {\em u} and derived forms of {\em u} will be made. For 
POSSADJ's, this approach may lead to problems for sentences like:
\begin{itemize}
   \item [-] Gij kunt uw kinderen dagelijks bezoeken
\end{itemize}
where {\em gij} and {\em uw} refer to the same person, and {\em uw} is a 
bound anaphor. 
Special rules in the M-grammar should handle this type of sentences.

\end{itemize}

Not included in the set of PERSPRO's are obsolete forms like {\em haar}(plural)
and forms that do not belong to standard Dutch, like {\em hunnie}, 
{\em hullie}, {\em zullie}, etc.

There are no rules for the relation between {\em het} en {\em zijn} resp. 
{\em haar}, as
in sentences like:

\begin{itemize}
  \item [-] {\em Het} ziet {\em zijn} vader
  \item [-] Het meisje beweerde dat {\em het haar} vader nog nooit gezien had
\end{itemize}

As a consequence, the second sentence will never yield the translation:

\begin{itemize}
  \item [-] The girl asserted that {\em she} had never seen {\em her} father
\end{itemize}

where {\em she} and {\em her} are referring to the same person. Extensions at 
this point will be needed in future versions of ROSETTA, in particular with 
respect 
to interactive disambiguation.

The difference between reduced forms and non-reduced forms is important
because of the fact that reduced forms often can be used generic, as the 
following sentences illustrate:
\begin{itemize}
  \item [-] In Scheveningen kun je Engeland zien liggen     
            ({\em je}  = subject)
  \item [-] In Scheveningen kunnen ze Engeland zien liggen  
            ({\em ze}  = subject)
  \item [-] Ze ontslaan je hier zomaar                      ({\em je}  = object)
  \item [-] Men gooit ze er daar zomaar uit                 ({\em ze}  = object)
\end{itemize}

For reasons of efficiency, all attributes of BPERSPRO's 
(and the attribute-values) will be copied to the PERSPROrecord in the 
W-rules.

\subsection{POSSADJ's}

Most POSSADJ's are (also) derived from BPERSPRO's; they have the following 
non-inherent categories: `reduced' (with values: `true', `false'), 
`eORenForm' (with: `eForm', `enForm' and `NoForm'), `mood' (with: `wh', 
`declarative') and `genitive' (with: `true', `false'). The attributes of the 
categories under the POSSADJ's (BPERSPRO, WHPRO, DEMPRO) will never be 
copied to the POSSADJ-level.

\begin{tabbing}
POSSADJ: \ \ \  \= from:\ \ \ \ \  \= reduced: \ \ \ \ \  \= eORenForm: 
\ \ \ \ \  \= mood:  \ \ \ \ \ \ \ \ \ \    \= geni: \\
         \>       \>          \>            \>           \>       \\
mijn    \> ik    \> false    \> NoForm    \> declarative \> false \\
m'n     \> ik    \> true     \> NoForm    \> declarative \> false \\
jouw    \> jij   \> false    \> NoForm    \> declarative \> false \\
je      \> jij   \> true     \> NoForm    \> declarative \> false \\
uw      \> u     \> false    \> NoForm    \> declarative \> false \\
zijn    \> hij   \> false    \> NoForm    \> declarative \> false \\
z'n     \> hij   \> true     \> NoForm    \> declarative \> false \\
haar    \> zij   \> false    \> NoForm    \> declarative \> false \\
'r      \> zij   \> true     \> NoForm    \> declarative \> false \\
d'r     \> zij   \> true     \> NoForm    \> declarative \> false \\
        \>       \>          \>           \>             \>       \\
ons     \> wij   \> false    \> NoForm    \> declarative \> false \\
onze    \> wij   \> false    \> eForm     \> declarative \> false \\
jullie  \> jullie\> false    \> NoForm    \> declarative \> false \\
uw      \> u     \> false    \> NoForm    \> declarative \> false \\
hun     \> zij   \> false    \> NoForm    \> declarative \> false \\
\end{tabbing}

\begin{tabbing}
POSSADJ: \ \ \  \= from:\ \ \ \ \  \= reduced: \ \ \ \ \  \= eORenForm: 
\ \ \ \ \  \= mood:  \ \ \ \ \ \ \ \ \ \    \= geni: \\
         \>       \>          \>            \>           \>       \\
je      \> je    \> true      \> NoForm    \> declarative \> false \\
hun     \> ze    \> false     \> NoForm    \> declarative \> false \\
\end{tabbing}

The attribute `eORenForm' is needed for the distinction between {\em ons} and 
{\em onze}; it has two values, `true' for {\em onze} and `false' for {\em ons}
and all other POSSADJ's. In idiomatic constructions obsolete `e-forms' of 
other POSSADJ's can be found: {\em mijn\underline{e} heren}, etc.

The attribute `genitive' is used for (not very frequent) genitive forms
(all these forms will be {\em analysed} only):

\begin{tabbing}
POSSADJ: \ \ \  \= from:\ \ \ \ \  \= reduced: \ \ \ \ \  \= eORenForm: 
\ \ \ \ \  \= mood:  \ \ \ \ \ \ \ \ \ \    \= geni: \\
         \>       \>          \>            \>           \>       \\
         \>       \>          \>            \>        \>       \\
mijner  \> ik    \> false    \> NoForm \> declarative  \> true \\
jouwer  \> jij   \> false    \> NoForm \> declarative  \> true \\
uwer    \> u     \> false    \> NoForm \> declarative  \> true \\
zijner  \> hij   \> false    \> NoForm \> declarative  \> true \\
harer   \> zij   \> false    \> NoForm \> declarative  \> true \\
        \>       \>          \>        \>              \>      \\
onzer   \> wij   \> false    \> NoForm \> declarative  \> true \\
uwer    \> u     \> false    \> NoForm \> declarative  \> true \\
hunner  \> zij   \> false    \> NoForm \> declarative  \> true \\
\end{tabbing}

The BPERSPRO {\em jullie} doesn't have a corresponding genitive form of
the POSSADJ. This will not lead to problems, because all genitives of
POSSADJ are {\em analysed only}; in generation paraphrases like 
{\em van mij(n)}, {\em van jou(w)}, {\em van ons}, {\em van onze}, etc. will 
be used, and, with respect to these paraphrases, {\em jullie} is regular: 
{\em van jullie}.

Some POSSADJ's, however, are derived from other (basic) categories;
{\em wiens} and {\em wier} are both derived from the WHPRO {\em wie}; and
 {\em diens} is derived from the DEMPRO {\em die}:

\begin{tabbing}
POSSADJ: \ \ \  \= from:\ \ \ \ \  \= reduced: \ \ \ \ \  \= eORenForm: 
\ \ \ \ \  \= mood:  \ \ \ \ \ \ \ \ \ \    \= geni: \\
         \>       \>          \>              \>              \>       \\
wiens   \> wie   \> false    \> NoForm   \> wh           \> false \\
wier    \> wie   \> false    \> NoForm   \> wh           \> false \\
        \>       \>          \>          \>              \>       \\
diens   \> die   \> false    \> NoForm   \> declarative  \> false \\
\end{tabbing}

Note: the WHPRO and DEMPRO, that are `under' the POSSADJ, have attributes
`sexes' (that gives information about the gender of the 
person the POSSADJ refers to) and the attribute `number': the value `masculine'
or `[\ ]' (=unknown) combined with `singular' yields {\em wiens} resp. 
{\em diens}; 
the value `feminine' combined with `singular' or `plural' and the value 
`masculine' or `[\ ]' combined with `plural' yield {\em wier} as 
WHPRO. If these 
attributes have to be evaluated in the M-grammar, the tree has to be traversed
down to the BCAT-level.

For the POSSADJ's derived from {\em het} and {\em gij}, see section 8.2.


\subsection{POSSPRO's}

POSSPRO's can be derived from BPERSPRO too; they have the
non-inherent attribute `nvorm', with the values: `true', `false'.
No attributes are copied from the input BPERSPRO; if these
attributes have to be evaluated, one has to traverse the tree to the 
BCAT-level.


\subsubsection{nvorm}

The value `false' is used for the following POSSPRO's: mijne (from: ik), 
jouwe (from: jij), uwe (from: u), zijne (from: hij), hare (from: zij), 
onze (from: wij), uwe (from: u(=plural)), hunne (from: zij).

The value `true' for: mijnen (from: ik), jouwen (from: jij), uwen (from: u), 
zijnen (from: hij), haren (from: zij), onzen (from: wij), 
uwen (from: u(=plural)), hunnen (from: zij).

There is no POSSPRO (irrespective of the value of `nvorm') that corresponds
with the BPERSPRO {\em jullie}. Here, we need a paraphrase. (Compare 8.3., 
POSSADJ's
in genitive form).


\subsection{DET's}

The following words of the category BDET exist: {\em alle,
beide, elk, enig, enkel, ettelijke, hoeveel, ieder, meerdere, menig, sommige,
veel, verscheidene, verschillende, voldoende, weinig, welk, zoveel, zulk}.

Two inherent attributes give 
information about whether or not it is possible to stick the suffixes {\em -e} 
resp. {\em -en} to a BDET: `eFormation' and `enFormation'.

The following possibilities exist:

\begin{tabbing}
DET: \ \ \ \ \ \ \ \ \ \  \= `eForm': \ \ \ \ \ \ \ \ \ \  \= `enForm': \\
                          \>                               \>           \\
alle         \>   alle          \>     allen              \\
beide        \>   beide         \>     beiden             \\
elk          \>   elke          \>     -                  \\
enig         \>   enige         \>     enigen             \\
enkel        \>   enkele        \>     enkelen            \\
ettelijke    \>   ettelijke     \>     -                  \\
hoeveel      \>   hoevele       \>     hoevelen           \\
ieder        \>   iedere        \>     -                  \\
meerdere     \>   meerdere      \>     -                  \\
menig        \>   menige        \>     -                  \\
sommige      \>   sommige       \>     sommigen           \\
veel         \>   vele          \>     velen              \\
verscheidene \>   verscheidene  \>     verscheidenen      \\
verschillende \>  verschillende \>     verschillenden     \\
voldoende    \>   voldoende     \>     -                  \\
weinig       \>   weinige       \>     weinigen           \\
welk         \>   welke         \>     -                  \\
zoveel       \>   zovele        \>     zovelen            \\
zulk         \>   zulke         \>     -                  \\
\end{tabbing}

Note: 

\begin{itemize}

  \item some DET's, like: {\em voldoende, menige, elke, iedere}, etc. do 
        not have a `en'-form. For these words, the attribute 
        `enFormation' will be \underline{false}.

  \item some DET's, like: {\em voldoende, verschillende, sommige}, etc. have a 
        base form that ends in {\em -e} already, and therefore have the same 
        string in the first two columns. In order to generalise as much as 
        possible, all these words are ambiguous between `eORenForm = 
        NoForm' and `eORenForm = eForm'.

\end{itemize}


\subsection{INDEFPRO's}

INDEFPRO's can have a genitive {\em -s}, as for instance in 
{\em iemand\underline{s}}.
Because
it is considered to be inflection, W-rules treat this phenomenon.
Note that no special segmentationrules are needed; the `normal' genitive-marker 
can be used (see: 6.3). The inherent attribute `possgeni' says whether or not
the INDEFPRO can have a genitive form.

\newpage

\section{Final remarks}

The following {\em verbs} cannot be handled by the spelling rules because
they don't have vowel reduction in present plural forms, subjunctive and 
infinitive, like in `normal' Dutch verbs: {\em carpoolen}, {\em croonen}, 
{\em freewheelen}, {\em inzoomen}, {\em keepen}, {\em poolen}, {\em screenen},
{\em uitzoomen}, {\em zoomen} (list derived from Van Dale's alfabetical 
section). 
If we consider these verbs as `irregular' (conjugationclass 1 or 2), this
problem can easily be dealt with, because all inflected forms of irregular
verbs are `listed' in the segmentation rules: the only thing we have to do is 
to add all forms of {\em carpoolen, croonen,} etc. to the segmentation rules.
Another solution would be to give these verbs new conjugationclasses and make 
appropriate segmentation and W-rules for these classes.

The verb {\em verwelkomen} is problematic, because it is irregular in the 
present tense, like {\em komen}, but differs from {\em komen} in the past tense
en past participle (which are regular for {\em verwelkomen}). It cannot
be added to the list of verbs of class 1 and 2, because {\em komen} must have a
`$\star$' in the suffix rules (due to verbs like {\em bekomen}, {\em ontkomen},
etc.). At the moment, {\em verwelkomen} is not possible.\footnote{
Note that the same problem seems to be present for {\em eten} vs. 
{\em vreten} and
{\em vergeten} ({\em eten} has the deviating form {\em ge\underline{g}eten}),
but in this case {\em eten} didn't need a `$\star$', so these verbs could still 
be combined in the same class.}

Some changes in the morphology could reduce the amount of ambiguities (e.g.
when this is desirable because of efficiency problems). Possible changes are:
\begin{itemize}
  \item to give up the difference between plural and singular of verbal forms
        for persons 4 and 5; the inflection of the verb is similar for plural 
        and singular (e.g. {\em u komt} is ambiguous between singular and
        plural).
  \item to subdivide the `irregular verbs' (class 1 and 2) into different 
        classes (e.g. one for verbs like {\em breken, meten}, etc., one for
        verbs like {\em gaan, staan}, etc. and one for {\em zijn, hebben}, 
        etc.). Thus, forms like {\em breek} and {\em meet} are not as ambiguous 
        as they are now (which is caused by all the different forms of verbs
        like {\em zijn} and {\em hebben}.
\end{itemize} 

Another document deals with derivational suffixes; among these is the
diminutive suffix (as in: {\em klein\underline{tje}}). Of
course, this implies that diminutive-formation is not considered to be an 
inflectional phenomena.

\newpage


\section{References}

\begin{description}
  \item [ANS (GEERTS, G. e.a.)]
     1984, {\bf Algemene Nederlandse Spraakkunst}, Groningen, Wolters-Noordhoff.

  \item [VAN DALE (VAN STERKENBURG, P.G.J. e.a.)]
     1984, {\bf Groot woordenboek van hedendaags Nederlands}, Utrecht, Van Dale 
           Lexicografie b.v.

  \item [NIEUWBORG, E.R.]
     1978, {\bf Retrograde Woordenboek van de Nederlandse Taal}, 
           Deventer, Kluwer Technische Boeken b.v.

  \item [MARTIN, W.]
     1971, {\bf Inverte Frequentielijst van het Nederlands}, 
           Leuven, Instituut voor toegepaste Linguistiek.

  \item [KLEIN, M. and VISSCHER, M.]
     1985, {\bf Praktische Cursus Spelling}, Groningen, Wolters-Noordhoff.

\end{description}
\end{document}
