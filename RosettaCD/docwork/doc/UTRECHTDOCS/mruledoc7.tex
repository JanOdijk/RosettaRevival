\documentstyle{Rosetta}
\begin{document}
   \RosTopic{Rosetta3.doc.Mrules.English}
   \RosTitle{Rosetta3 English M-rules: PREPPPROPtoFORMULA}
   \RosAuthor{Margreet Sanders}
   \RosDocNr{381}
   \RosDate{November 14, 1989}
   \RosStatus{approved}
   \RosSupersedes{-}
   \RosDistribution{Project}
   \RosClearance{Project}
   \RosKeywords{English, documentation, Mrules, PREPPPROPtoFORMULA}
   \MakeRosTitle
%
%

\section{Introduction}
As all main category grammars, the English PREPPPROP grammar is divided in 
three parts. First, {\bf PREPPPROPformation} forms the PROP-structure. Then, {
\bf PREPPPROPtoFORMULA} turns this PROP into an intermediate structure, called 
PREPPFORMULA
(comparable to the CLAUSE in the sentence grammar). Finally, {\bf 
PREPPFORMULAtoPROP} makes an open or closed PROP-structure, comparable to the 
SENTENCE level in the sentence grammar. Prior to PREPPPROPformation, there is 
the {\bf PrepDerivation} grammar, which was discussed in doc.\ 316, {\em 
Rosetta3 English M-rules: Derivation Subgrammars\/}. The final open or closed 
PREPPPROP is usually input to the Proposition Substitution Rules in the 
XPPROPtoCLAUSE subgrammar. A `bare' PREPPPROP may also form an expression on 
its own with help of the copula {\em be\/}. In that case, it leaves the 
PREPPPROP grammars after PREPPPROPformation, and 
is input to the ClauseFormation Rules of the XPPROPtoCLAUSE subgrammar. 
There is no rule to make a complete Utterance of a closed PREPPPROP.

The current document describes the contents of the second PREPPPROP subgrammar, 
PREPPPROPtoFORMULA (the first subgrammar was discussed in doc.\ 380, {\em 
Rosetta3 English M-rules: PrepppropFormation\/}, and the last one is discussed 
in doc.\ 382). The subgrammar consists of 
a number of rule classes and transformation classes. A rule class in its turn
consists of a number of rules and a transformation class of a number of 
transformations. The relative ordering of the rules and transformations in the
(sub)grammar is indicated by a {\em control expression}. A summary of this
control expression (i.e.\ a listing of the ordering of the rule classes, 
without explicit mentioning of the rules themselves) is also included here, 
and the initial (= head), import and export categories are given. 

In the section on the rules and transformations, only the rule names are given, 
but not the exact rule formulation. What is attempted 
is to provide a detailed overview of the workings of the subgrammar, and 
how the different rule classes achieve this,
together with some comments on the problems still to be solved, the reasons 
behind certain choices, and perhaps possible alternatives. For every rule, an 
example is given. If it is uncertain whether the example is correct (either 
because it may not be an example of the phenomenon in question, or because it 
may not be correct English), it is preceded by a question mark. Note that all 
explanation of rules and transformations is given from a generative viewpoint
only, unless explicitly stated otherwise. Often, the information given in this 
document is based strongly on the comment already present in the documentation 
of the rules themselves. Discrepancies between what is stated here and what is 
said in the rule itself are usually caused by the fact that the rule file has 
not  been updated, although insights have changed. The semantics of the rules 
has been left unspecified in the current documentation, since it is not at all 
clear.

Whenever the current implementation differs widely from the strategy that was 
devised in the definition phase of Rosetta3 (as laid down for English in docs.\ 
150, {\em Subgrammars of English\/}, 153, {\em Rule and Transformation Classes 
of English\/}, and 155, {\em Rule and Transformation Classes common to all 
languages\/}, all written by Jan Odijk), this will be indicated explicitly in 
the current document. Conditions on crucial orderings of rule classes will be 
repeated here, even if they do not differ from the original strategy, to make 
the document as self-contained as possible.

Finally note that the rules described in this document have NOT been tested 
properly. English analysis is not possible yet (there is no Surface Parser), and 
English generation has only been tested in as far as the construction was the 
translation of a Dutch sentence to be tested.

\newpage
\section{PREPPPROPtoFORMULA}
The exact contents of the PREPPPROPtoFORMULA subgrammar are mainly determined 
by the requirements of isomorphy with the XPPROPtoCLAUSE subgrammar. Thus, 
most rule classes have only one rule, providing some sort of 
`default' value. Transformations are hardly needed, since the surface structure 
and many attribute values will be determined by the sentence the PROP is 
substituted in. 

In doc.\ 150, it was assumed that this subgrammar would also contain a rule 
class for proposition substitution. Since there are no sentential variables in 
the PrepppropFormation grammar (see the comment in the pattern rules there), 
this class no longer seems necessary, nor does the related Control 
transformation class. Of course, this makes the current subgrammar only {\em 
partially\/} isomorphic to the XPPROPtoCLAUSE subgrammar: only if no rules from 
this optional PropSubst class occur in the derivation tree can an isomorphic 
derivation be found in the current subgrammar.

As in the sentence grammar, the FormulaFormation rules mentioned in doc.\ 150 
have been separated into 
a number of rule classes, each performing its own task (viz.\ aspect and 
superdeixis assignment). There is no rule class for EMPTY substitution, since 
it is assumed that preps do not take an EMPTY argument (in that case, they are 
simply intransitive).

\section{Subgrammar Specification}
The subgrammar definition can be found in the file which also contains all the 
rules of this subgrammar, {\bf english:PrepppropToFormula.mrule}, 
which is {\em mrules87.mrule\/}.

\begin{verbatim}
%SUBGRAMMAR PrepppropTOformula


   ( RC_pppFormulaFormation )
.  ( RC_pppAspect )
.  ( RC_pppSuperdeixis )
.  ( TC_pppCaseAssign )

\end{verbatim}

\begin{description}
  \item[Head]  PREPPPROP  \ \ \ \ FROM (PREPPPROPformation)
  \item[Export] PREPPFORMULA
  \item[Import] --
\end{description}

\newpage
\section{Rules and Transformations}

\subsection{RC\_pppFormulaFormation}
\begin{description}
\item[Kind] Obligatory Rule Class
\item[Task] To turn a PREPPPROP into a PREPPFORMULA (which has exactly the same 
record as a PREPPPROP). This rule is needed only to form a counterpart in the 
isomorphic scheme for the ClauseFormation rules of the XPPROPtoCLAUSE 
subgrammar.

\vspace{1 cm}
\begin{description}
\item[Name] RPrepToFormula
\item[Task] To turn a PREPPPROP into a PREPPFORMULA (which has exactly the same 
record as a PREPPPROP). 
\item[File] english:PrepppropToFormula.mrule (mrules87.mrule)
\item[Semantics]
\item[Example] $_{PREPPPROP}$[x1 behind x2] $\rightarrow$ $_{PREPPFORMULA}$[x1 
behind x2] (She hid behind the curtain)
\item[Remarks] 
\end{description}

\end{description}

\newpage
\subsection{RC\_pppAspect}
\begin{description}
\item[Kind] Obligatory Rule Class
\item[Task] To spell out the aspect of the PREPPFORMULA, i.e.\ the aspect 
relation between the interval E and a reference interval R. It is assumed that 
this relation is always {\em imperfective\/} here; there may not be any 
temporal adverbial present. The rule is needed for isomorphy reasons with 
the XPPROPtoCLAUSE subgrammar, e.g.\ when translating a small clause in a full 
sentence: {\em He seemed against the proposal - It seemed that he was 
against the proposal\/}

It is assumed that appositive PREPPFORMULAs containing a temporal adverbial are 
derived from infinite sentences with deletion of {\em being\/}: {\em The man, 
against the proposal last week, ...\/} $\leftarrow$ {\em The man, being 
against the proposal last week, ...\/}.

\vspace{1 cm}
\begin{description}
\item[Name] RpppAspectImperf
\item[Task] see above
\item[File] english:PrepppropToFormula.mrule (mrules87.mrule)
\item[Semantics]
\item[Example] [x1 behind x2]$_{omegaaspect}$ $\rightarrow$ [x1 behind x2]$_{
imperfective}$
\item[Remarks]
\end{description}

\end{description}

\newpage
\subsection{RC\_pppSuperdeixis}
\begin{description}
\item[Kind] Obligatory Rule Class
\item[Task] To provide the PREPPFORMULA with a value for the attribute {\bf 
superdeixis}. This rule is needed because of isomorphy reasons. It uses a 
parameter, {\em superpar\/}, to determine whether a present or past superdeixis 
value should be assigned.

If rules are added to introduce propositions or sentences in the preppprop 
subgrammar, it may be necessary to add superdeixis adaptation transformations 
following the current rule class. These should reset the value of the 
superdeixis attribute to something the Surface Parser can cope with.

For remarks on the relative ordering of modification rules and superdeixis 
determination rules, see doc.\ 380.

\vspace{1 cm}
\begin{description}
\item[Name] RpppSuperdeixis
\item[Task] see above
\item[File] english:PrepppropToFormula.mrule (mrules87.mrule)
\item[Semantics]
\item[Example] [x1 behind x2]$_{omegasuperdeixis}$ $\rightarrow$ [x1 behind 
x2]$_{pastsuperdeixis}$ (He hid behind the door)
\item[Remarks]
\end{description}

\end{description}

\newpage
\subsection{TC\_pppCaseAssign}
\begin{description}
\item[Kind] Obligatory Transformation Class
\item[Task] To assign oblique case to NP/CNVAR elements in the predicate, if 
there are any.

\vspace{1 cm}
\begin{description}
\item[Name] TNoPPCaseAssign
\item[Task] Default rule, to let structures where there is no NP/CNVAR element 
in the predicate pass this transformation class.
\item[File] english:PrepppropToFormula.mrule (mrules87.mrule)
\item[Semantics] --
\item[Example] x1 out
\item[Remarks]
\end{description}

\vspace{1 cm}
\begin{description}
\item[Name] TPPCaseAssign
\item[Task] To assign oblique case to NP/CNVAR elements in the predicate
\item[File] english:PrepppropToFormula.mrule (mrules87.mrule)
\item[Semantics] --
\item[Example] x1 in x2$_{[]}$ $\rightarrow$ x1 in x2$_{[accusative]}$ (These 
rags are tents, and those people live in them)
\item[Remarks] 
\end{description}

\end{description}


\end{document}

