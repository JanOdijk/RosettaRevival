
\documentstyle{Rosetta}
\begin{document}
   \RosTopic{General}
   \RosTitle{Notulen Rosetta vergadering 11-5-87}
   \RosAuthor{Margreet Sanders}
   \RosDocNr{0203}
   \RosDate{11-6-87}
   \RosStatus{approved}
   \RosSupersedes{-}
   \RosDistribution{Project}
   \RosClearance{Project}
   \RosKeywords{minutes}
   \MakeRosTitle
%
%
\begin{description}
\item[Aanwezig:] Carel Fellinger, Natalia Grygierczyk,
                 Chris Hazenberg, Franciska de Jong, 
                 Lilian Kopinga, Jan Landsbergen, Ren\'{e} Leermakers, 
                 Jeroen Medema, Elly van Munster, 
                 Jan Odijk, Ans Post, Joep Rous, Margreet Sanders (not),
                 Andr\'{e} Schenk, 
                 Harm Smit 
\item[Afwezig:]  Lisette Appelo
\item[Agenda:]\mbox{}
  \begin{enumerate}
  \item Opening en notulen
  \item Mededelingen
  \item Verslag MT dag in Luik
  \item Besproken en/of nieuw verschenen documenten
  \item Rondvraag en sluiting
  \item Bespreking PAHO-systeem (SPANAM)
  \end{enumerate}
\end{description}

\section{Opening en notulen}
De notulen van de vorige vergadering worden met een aantal wijzigingen 
aangenomen. 

\section{Mededelingen}
\begin{enumerate}
  \item Het stuk {\em `Dynamic Predicate Logic'} van Jeroen Groenendijk kan op 
dond.\  14 mei met hem besproken worden door ge\"{i}nteresseerden.
  \item Aan de heren Ahles en Vos van {\em VNU} is een tegenbezoek 
gebracht. Ze hebben o.a.\  een databestand van de Spectrum Encyclopedie op tape 
(gestructureerd; 300.000 termen, vnl. geografisch/staatkundig). Hun 
belangstelling voor een Q/A-systeem als toegang tot het bestand lijkt 
vooralsnog ge\-ring. Nadere contacten volgen nog. Ook met {\em KLuwer} en {\em 
Elsevier} wordt nog verder overlegd. Jan L. zal nieuwe ontwikkelingen melden.
  \item Twee mensen van het {\em IBM National Language Services Centre} zijn op 
6 mei op een ori\"{e}nterend bezoek geweest (zie notulen 13-4). Een tegenbezoek 
volgt waarschijnlijk rond augustus. Er zijn geen concrete plannen te verwachten 
naar aanleiding van dit contact.
  \item Jan L. heeft contact met Louis Dumontier van {\em Philips Canada}. De 
tekst\-ver\-werker-fabriek wil de bestaande terminologie-banken een goede 
interface geven voor menselijke vertalers (Fr/Eng).
  \item Het nieuwe tijdschrift {\em Language and Technology} van INK int. is 
uit. We zullen een abonnement nemen. Nummers komen op de leesplank in k. 345.
  \item Jan L. hoopt in september naar {\em Japan} te gaan en daar, naast een 
eventueel bezoek aan een conferentie over MT, een tournee langs instituten 
en bedrijven te maken. Dit alles misschien nog gecombineerd met een bezoek aan 
Montreal.
  \item Natalia wijst op het uitkomen van een nieuw {\em combinatorisch 
woordenboek}. Jan L. bekijkt of we dit zullen aanschaffen.
  \item Joep en Ren\'{e} zijn gastsprekers geweest op een {\em colloquium} van 
Theo Janssen over autom. vertaalsystemen, en hebben de software van Rosetta 
belicht. Ook isomorfie is nog eens verduidelijkt.
  \item De planning van de software-groep is als volgt:\\
  \begin{tabular}{lll}
  Chris: & mei/juni/juli: & M-regel compiler\\
  Ren\'{e}: & tot 15 mei: & S-parser in het systeem brengen\\
            & mei/juni/juli: & M-regel compiler\\
  Carel: & mei: & M-parser\\
         & juni/juli: & RBS\\
  Jeroen: & mei: & correspondentie Van Dale N-E en E-N\\
          & juni/juli: & RBS\\
  Joep:  & mei: & Woordenboek - Organisatie - Editor - consistentie\\
         & juni/juli: & Documentatie\\
  \end{tabular}

\end{enumerate}

\section {Verslag MT-dag in Luik}
Margreet doet verslag van de {\em Athena-dag over Automatisch Vertalen} die op 
5 mei bij de Universiteit van Luik werd gehouden in het kader van de promotie 
van nieuwe technologie\"{e}n in Walloni\"{e}. Van de $\pm$150 deelnemers was het 
grootste deel vertaler. Sprekers waren Nagao (univ.\ Kyoto), Gross (univ.\ 
Parijs VII), Melby (Brigham Young univ.), Guilbaud (GETA),
Nitta (Hitachi), Byrd (IBM) en Michiels (Eurotra). Veel nadruk werd gelegd op de
belangrijke rol van de menselijke vertaler als pre- en/of post-editor. Ook de
organisatie van woordenboeken en de inhoud ervan kwam meermalen aan de orde.
Gewezen werd op de noodzaak tot samenwerking bij het structuren van
woordenboeken en databestanden . \\
In de wandelgangen probeerde Longman zijn CD-Roms te slijten. Er waren 
demonstraties van Logos en LADS.

\section{Besproken en/of nieuw verschenen documenten}
\begin{description}
  \item [Besproken:]\mbox{}

  \begin{itemize}
  \item Joep Rous: CF Control Grammars (180). Dit alternatief voor de huidige 
methode levert niet een duidelijke winst op en wordt daarom niet ingevoerd.
  \item Jeroen Medema: RESIDE proposal (198). Stapsgewijs zal een
omgeving (RESIDE) worden ingevoerd die beter voldoet aan hogere performance-
eisen en meer faciliteiten biedt dan RBS.
\end{itemize}
  \item [Verschenen:]\mbox{}

  \begin{itemize}
  \item Chris Hazenberg: nieuwe versie van doc. 188: Auxiliary Domain Compiler.
Bespreking geschiedt na gebleken behoefte.
  \item Jeroen Medema: System to Create Retrograde Vocabularies (190). Dit 
verslag van de gang van zaken wordt verder niet besproken.
  \item Andr\'{e} Schenk: Idioms and the Dictionary (194). Wordt door de 
linguisten op 12 mei besproken, en door de informatici op 15 mei.
  \item Jan Odijk en Andr\'{e} Schenk: Dictionaries in Rosetta3 (201). Wordt 
door de linguisten op 21 mei besproken; de informatici doen het samen met het 
vorige stuk.
  \item Het stuk van Joep Rous: Conversion Problems with B-LEX (196) wordt 
nader besproken als Joep een concreet voorstel heeft gemaakt.
  \end{itemize}
\end{description}

\section{Rondvraag en sluiting}
\begin{description}
  \item [Carel] geeft een  verzoek van Frank Stoots door om bij het integreren 
\verb2\today2 uit de {\em dochead} te vervangen door de echte datum.
  \item [Joep] vermeldt een idee van Theo Janssen voor een SPIN-subsidie: het 
omzetten van Nederlands naar de taal {\em Blizz} (een plaatjestaal) voor afasie
-pati\"{e}nten.
\end{description}
Hiermee wordt de vergadering om 14.55 uur gesloten.

\section*{Bespreking PAHO-systeem (SPANAM)}
Om 15.03 wordt de vergadering informeel voortgezet met een bespreking van 
SPANAM en ENGSPAN door Elly en Jan L., aan de hand van een hand-out.
\end{document}

