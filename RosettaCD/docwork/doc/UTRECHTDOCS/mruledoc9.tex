\documentstyle{Rosetta}
\begin{document}
   \RosTopic{Rosetta3.doc.Mrules.English}
   \RosTitle{Rosetta3 English M-rules: PREPPformation}
   \RosAuthor{Margreet Sanders}
   \RosDocNr{383}
   \RosDate{November 15, 1989}
   \RosStatus{approved}
   \RosSupersedes{-}
   \RosDistribution{Project}
   \RosClearance{Project}
   \RosKeywords{English, documentation, Mrules, PREPPformation}
   \MakeRosTitle
%
%

\section{PreppFormation}
PREPPs and ADVPs are used in Rosetta3 as non-argument locatives and temporals.
Contrary to what was assumed in the definition phase of Rosetta3 (see e.g.\ 
doc.\ 150, {\em Subgrammars of English\/}), it is no longer considered 
necessary to introduce these categories as full PREPPPROPs or ADVPPROPs that 
have to be pruned. For a full discussion of this subject, see the document on 
the Treatment of Adverbs by Jan Odijk (to appear).

Since the PREPPformation subgrammar need NOT be isomorphic with other XPPROP or 
sentence grammars
(because it is impossible anyway in Rosetta3 to translate a simple 
phrase (PREPP or ADVP) into something with a propositional structure),
the exact contents of the PreppFormation subgrammar are mainly determined 
by the requirements posed by the sentence the PREPP is 
substituted in. Note that the 
subgrammar is isomorphic with the ADVPformation subgrammar, since PREPs and 
ADVs may have to translate into each other. Input to this grammar is a SUBPREP, 
coming from the PrepDerivation subgrammar (see doc.\ 316, {\em Rosetta3 English 
M-rules: Derivation Subgrammars\/}). For testing purposes, the PREPP formed in 
the current subgrammar can also be given a top node UTT in the Utterance 
grammar, to allow translation of bare PREPPs.
Since in the definition phase of Rosetta3 it was thought that there would only 
be PROP categories, the current subgrammar was not mentioned in the general 
lay-out, described for English in doc.\ 150, {\em Subgrammars of English\/}.

As all subgrammars, the PREPP subgrammar consists of 
a number of rule classes and transformation classes. A rule class in its turn
consists of a number of rules and a transformation class of a number of 
transformations. The relative ordering of the rules and transformations in the
(sub)grammar is indicated by a {\em control expression}. A summary of this
control expression (i.e.\ a listing of the ordering of the rule classes, 
without explicit mentioning of the rules themselves) is also included here, 
and the initial (= head), import and export categories are given. 

In the section on the rules and transformations, only the rule names are given, 
but not the exact rule formulation. Most rules are a copy of the rules in the 
three PREPPPROP subgrammars, and the description of these rules here has 
often been copied from the documents on these grammars. 
For every rule, an example is given. If 
it is uncertain whether the example is correct (either 
because it may not be an example of the phenomenon in question, or because it 
may not be correct English), it is preceded by a question mark. Note that all 
explanation of rules and transformations is given from a generative viewpoint
only, unless explicitly stated otherwise. The semantics of the rules 
has been left unspecified in the current documentation, since it is not at all 
clear.


Finally note that the rules described in this document have NOT been tested 
properly. English analysis is not possible yet (there is no Surface Parser), and 
English generation has only been tested in as far as the construction was the 
translation of a Dutch sentence to be tested.

\newpage
\section{Subgrammar Specification}
The subgrammar definition can be found in the file which also contains all the 
rules of this subgrammar, {\bf english:Prepp.mrule}, 
which is {\em mrules90.mrule\/}.

\begin{verbatim}
%SUBGRAMMAR PreppFormation


   ( RC_StartPreppRules )
.  ( TC_PreppPatternRules )
.  ( RC_PreppSuperdeixis )
.  [ RC_PreppModRules ]
.  ( TC_PreppCaseAssign )
.  { RC_PreppSubst }


\end{verbatim}

\begin{description}
  \item[Head]  SUBPREP  \ \ \ \ FROM (Prepderivation)
  \item[Export] PREPP
  \item[Import] NPVAR, CNVAR, ADJPPROPVAR, ADVPPROPVAR, NPPROPVAR, 
PREPPPROPVAR, VERBPPROPVAR, SENTENCEVAR, CLAUSEVAR, EMPTYVAR, PROSENTVAR, 
ADVP, NP, PREPP
\end{description}

\newpage
\section{Rules and Transformations}

\subsection{RC\_StartPreppRules}
\begin{description}
\item[Kind] Obligatory Rule Class
\item[Task] To provide a SUBPREP with its correct number of PREPP argument 
variables and build a PREPP above it. The SUBPREP-level itself is deleted, 
just leaving the PREP. The reason for the existence of the SUBPREP level is 
explained in doc.\ 316, {\em Rosetta3 English M-rules: Derivation Subgrammars}. 

\vspace{1 cm}
\begin{description}
\item[Name] RStartPrepp100
\item[Task] To build a PREPP for an intransitive SUBPREP, so with no arguments 
in the PREPP.
\item[File] english:Prepp.mrule (mrules90.mrule)
\item[Semantics]
\item[Example] inside $\rightarrow$ inside (We played inside)
\item[Remarks]
\end{description}

\vspace{1 cm}
\begin{description}
\item[Name] RStartPrepp120
\item[Task] To build a PREPP for a transitive SUBPREP, so with one argument in 
the PREPP. The attributes {\bf mood} and {\bf 
specQ} of the PREPP are determined on basis of the variable in it.
\item[File] english:Prepp.mrule (mrules90.mrule)
\item[Semantics]
\item[Example] in + x1 $\rightarrow$ in x1 (He danced in the house)
\item[Remarks]
\end{description}

\item[Remark] No rules have been written for `path' preps, like {\em from ... 
to ... (via ...)\/}. If a way can be found to deal with the explosion of 
ambiguities caused by the introduction of such double preps, rules should be 
added for them.

\end{description}

\newpage
\subsection{TC\_PreppPatternRules}
\begin{description}
\item[Kind] Obligatory Transformation Class
\item[Task] To check the category of the argument variable and the relation 
it bears in the PREPP against the preppatterns specified for the PREP. The {\bf 
synppefs} attribute of the PREPP is set at the value 
actually chosen from the synpps of the PREP.

In doc.\ 150, this transformation class is mentioned too, although it is 
described only in the document on Dutch rule classes (doc.\ 152). There, it is 
assumed that preps taking a prepp or advp argument must introduce a whole 
preppprop or advpprop and then prune it. This is no longer considered 
necessary. `Approximate 
expressions', like {\em over a hundred, about five thousand\/}, are still 
thought to be examples of modification of numerals, and should be dealt with in 
the DETP grammar. Note that Dutch has many more of these prepp modifiers, often 
translating into an English adverb: {\em tegen de/rond de/om en nabij de 
vijftig - about/approximately fifty\/}.

\vspace{1 cm}
\begin{description}
\item[Name] TPPPrepPattern0
\item[Task] To let preps that have no PREPP arguments pass this transformation 
class
\item[File] english:Prepp.mrule (mrules90.mrule)
\item[Semantics]
\item[Example] inside (We played inside)
\item[Remarks]
\end{description}

\vspace{1 cm}
\begin{description}
\item[Name] TPPPrepPattern1
\item[Task] To specify the relation name (which must always be {\em objrel\/})
and the category of the argument in the PREPP 
\item[File] english:Prepp.mrule (mrules90.mrule)
\item[Semantics]
\item[Examples] \mbox{}\\
in argrel/VAR $\rightarrow$ in objrel/NPVAR  (He'll arrive in an hour)\\
in argrel/VAR $\rightarrow$ in objrel/CNVAR ((the car) in which he arrived)\\
from  argrel/VAR $\rightarrow$ from objrel/PREPPVAR (She has been waiting from 
before noon)\\
until argrel/VAR $\rightarrow$ until objrel/ADVPVAR (He will not be 
there until tomorrow)
\item[Remarks] 
\end{description}

\end{description}

\newpage
\subsection{RC\_PreppSuperdeixis}
\begin{description}
\item[Kind] Obligatory Rule Class
\item[Task] To provide the PREPP with a value for the attribute {\bf 
superdeixis}. This rule is needed here because of the modification rules that 
come right after this rule class: 
the superdeixis of the PREPP must be known to be able to check its 
compatibility with that of the modifier. Cf.\ the PREPPPROP grammar, where 
the superdeixis rule is placed after modification, in the next subgrammar, 
which is in fact incorrect but demanded because of isomorphy with the sentence 
grammar.

The rule uses a 
parameter, {\em super\/}, to determine whether a present or past superdeixis 
value should be assigned.

\vspace{1 cm}
\begin{description}
\item[Name] RppSuperdeixis
\item[Task] see above
\item[File] english:Prepp.mrule (mrules90.mrule)
\item[Semantics]
\item[Example] [behind x1]$_{omegasuperdeixis}$ $\rightarrow$ [behind 
x1]$_{presentsuperdeixis}$ (He sleeps behind the couch)
\item[Remarks]
\end{description}

\end{description}

\newpage
\subsection{RC\_PreppModRules}
\begin{description}
\item[Kind] Optional Rule Class
\item[Task] To introduce modifiers (of the PREP), as some kind of 
degree modification. The rule includes a check on superdeixis (see comment on
previous rule class).

\vspace{1 cm}
\begin{description}
\item[Name] RPPAdvpMod
\item[Task] To introduce an ADVP degree modifier of the PREP
\item[File] english:Prepp.mrule (mrules90.mrule)
\item[Semantics]
\item[Example] ahead of x1 + dead $\rightarrow$ dead ahead of x1 (It crawled
dead ahead of him)
\item[Remarks]
\end{description}

\vspace{1 cm}
\begin{description}
\item[Name] RPPNpMod
\item[Task] To introduce a (unitnoun) NP modifier of the PREP
\item[File] english:Prepp.mrule (mrules90.mrule)
\item[Semantics]
\item[Example] behind x1 + three foot $\rightarrow$ three foot behind 
x1 (She always walks three foot behind her husband)
\item[Remarks]
\end{description}

\end{description}


\newpage
\subsection{TC\_PreppCaseAssign}
\begin{description}
\item[Kind] Obligatory Transformation Class
\item[Task] To assign oblique case to NP/CNVAR elements in the PREPP, if 
there are any.

\vspace{1 cm}
\begin{description}
\item[Name] TPreppCA0
\item[Task] Default rule, to let structures where there is no NP/CNVAR object 
in the PREPP pass this transformation class.
\item[File] english:Prepp.mrule (mrules90.mrule)
\item[Semantics] --
\item[Example] out
\item[Remarks]
\end{description}

\vspace{1 cm}
\begin{description}
\item[Name] TPREPPca1
\item[Task] To assign oblique case to NP/CNVAR objects in the PREPP
\item[File] english:Prepp.mrule (mrules90.mrule)
\item[Semantics] --
\item[Example] in x1$_{[]}$ $\rightarrow$ in x1$_{[accusative]}$ (These 
rags are tents, and those people eat in them)
\item[Remarks] 
\end{description}

\end{description}

\newpage
\subsection{RC\_PreppSubst}
\begin{description}
\item[Kind] Iterative Rule Class
\item[Task] To substitute non-sentential expressions for their variable. There 
is no condition on substitution order (there is only one VAR anyway).
Conditions on genericity and superdeixis are the same as in the comparable 
class in the CLAUSEtoSENTENCE subgrammar (see doc.\ 370, p.\ 14). Reflexives 
etc.\ have not yet been excluded from the current rules explicitly. 

The rules use the system parameter LEVEL to check whether the index of the 
variable is consistent with (the level of) the variable that is to be 
substituted for according to the rule parameter.

There still is no way to limit the workings of the substitution rules in this 
grammar to those cases where it is really needed, i.e.\ where there will not be 
any substitution in the higher clause. Thus, many ambiguities arise caused by
substitution having applied in different cycles. See section 5 of doc.\ 368, {
\em Scope in Rosetta3\/}, for a discussion of this problem.

\vspace{1 cm}
\begin{description}
\item[Name] RppSubstitution1
\item[Task] To substitute a non-generic NP for its objrel VAR
\item[File] english:Prepp.mrule (mrules90.mrule)
\item[Semantics]
\item[Example] [after x1] + the lunch break $\rightarrow$ [after the lunch 
break]
\item[Remarks]
\end{description}

\vspace{1 cm}
\begin{description}
\item[Name] RppSubstitution3
\item[Task] To substitute a PREPP for its objrel VAR
\item[File] english:Prepp.mrule (mrules90.mrule)
\item[Semantics]
\item[Example] [from x1] + behind the door $\rightarrow$ [from behind 
the door] (She called him from behind the door)
\item[Remarks] 
\end{description}

\vspace{1 cm}
\begin{description}
\item[Name] RppSubstitution4
\item[Task] To substitute an ADVP for its objrel VAR
\item[File] english:Prepp.mrule (mrules90.mrule)
\item[Semantics]
\item[Example] [until x1] + tomorrow $\rightarrow$ [until tomorrow]
\item[Remarks] 
\end{description}

\end{description}


\end{document}

