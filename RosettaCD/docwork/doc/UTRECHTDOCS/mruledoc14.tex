\documentstyle{Rosetta}
\begin{document}
   \RosTopic{Rosetta3.doc.Mrules.English}
   \RosTitle{Rosetta3 English M-rules: NPPROPformation}
   \RosAuthor{Margreet Sanders, Franciska de Jong}
   \RosDocNr{392}
   \RosDate{December 4, 1989}
   \RosStatus{concept}
   \RosSupersedes{-}
   \RosDistribution{Project}
   \RosClearance{Project}
   \RosKeywords{English, documentation, Mrules, NPPROPformation}
   \MakeRosTitle
%
%

\section{Introduction}
NPPROP structures are structures with an NP as predicative head (under predrel).
The derivation of NPPROP structures is different from the derivation of all
other XPPROP and SENTENCE structures, and the isomorphy with other 
main category grammars is only partial. 
The reason for this deviation is the following.

Two kinds  of NPs may be distinguished on the level of
surface structures: (1) non-predicative NPs, 
that may function a.o. as subject, VP-complement and PREPP-complement, and 
(2) predicative NPs that occur in predicative position, e.g. in the context 
of verbs as {\em become\/} and {\em consider\/}. 
For most major categories (i.e.\ for X = ADJ, ADV, VERB, PREP), 
a predicate (XP) and a propositional structure (XPPROP) are built around the 
X in the XPPROPformation subgrammar. For X = NOUN, the situation is more 
complex: only for the derivation of predicative NPs is the 
creation of the NPPROP level semantically correct. 
Non-predicative NPs must be derived by a separate grammar, with its own 
idiosyncratic division into subgrammars (a.o CNformation, NPformation), not 
resulting in an NPPROP.

However, even when dealing with predicative NPs only there is a problem with 
isomorphy.
Many semantic and syntactic features of NP structures do not match the 
processes known for other XPs.
For example, NP determination (the introduction of determiners) has no clear
counterpart in any other main category grammar, and most 
modifications within the NP pertain to the NP-head, whereas in the  
other XPPROP grammars, modification pertains to the XPPROP-level, at 
least semantically. Hence, a derivation of 
NPPROPs that parallels the derivation of other XPPROPs is 
problematic, at least under the present implementation of XPPROP grammars.

Pending the development of a more satisfying approach to the mapping of 
NPPROPs and XPPROPs,  the present approach to the derivation of 
NPPROPs is one that has relaxed the conditions on isomorphy with other 
XPPROPsubgrammars, and requires a minimum number of rules specific for 
NPPROPs. Instead of building the NPPROP around the SUBNOUN provided by a 
derivation subgrammar, NPPROPs now take an NP formed by the NPformation 
subgrammar as their head. Thus, the rules existing for determination, 
modification etc.\ of 
non-predicative NPs need not be duplicated in the NPPROPformation subgrammar.
(Note that almost all NPs that may occur 
in non-predicative position may occur in predicative position as well. 
NPs consisting of a bare count noun ({\em 
Juliana  became queen in 1948}), which are excluded from all other contexts, may
also occur as predicate.) 
As a consequence, it is excluded that an NPPROP is translated into 
any other XPPROP. 

The rules working in the NPPROP subgrammars could not be written completely 
independent of the other XPPROP subgrammars, however. When an NPPROP is input 
to another subgrammar, it must have a general structure and attribute values 
compatible with what is expected there. Thus, there still is a general 
resemblence between the rules for NPPROPs and those for other XPPROPs.
Presently, of the three subgrammars that constitute the other XPPROP grammars 
only two have a counterpart in the NPPROP grammar.
The subgrammar {\bf NPPROPformation} introduces the NPPROP-level with the 
`finished' NP as predicate (predrel). 
Its output is either input to the clause formation rules of the 
{\bf XPPROPtoCLAUSE} subgrammar 
(in which sentences with the copula {\em be\/}, 
such as {\em She is a linguist\/} are derived), or 
it is input to the remaining part of the NPPROP grammar, called {\bf toNPPROP}.
For other XPPROPgrammars this remaining part consists of two subgrammars, 
XPPROPtoXPFORMULA and XPFORMULAtoXPPROP. In the `middle' subgrammar, aspect and 
superdeixis are dealt with. Since these have already been dealt with in the 
NP-subgrammar, they can be skipped for NPPROPs, and there is no need for an 
NPFORMULA 
level. The final subgrammar {\bf toNPPROP} turns the NPPROP directly into 
an open or a closed NPPROP structure.
This NPPROP is then import to the proposition substitution rules of
the subgrammar {\bf XPPROPtoCLAUSE}, and is incorporated into the sentential 
structure. Recently, the introduction of a `minimal' NPPROPtoNPFORMULA 
subgrammar has been proposed, to make translations between a small clause 
NPPROP like {\em He (seems) an imposter\/} and a full clause like {\em (It 
seems) that he is an imposter\/} possible. These clauses share the 
NP-derivation 
path, and only need parallel paths in the NPPROPtoNPFORMULA and XPPROPtoCLAUSE 
subgrammars to be possible. This change will probably be added in the near 
future.

The current document describes the contents of the first NPPROP subgrammar, 
{\bf NPPROPformation} (the other subgrammar is discussed in doc.\ 393, 
{\em Rosetta3 English M-rules: toNPPROP\/}, in which the introduction to the 
NPPROP subgrammars as given here is repeated in full).
The subgrammar consists of 
a small number of rule classes. A rule class in its turn
consists of a number of rules.
The relative ordering of the rules in the
subgrammar is indicated by a {\em control expression}. A summary of this
control expression (i.e.\ a listing of the ordering of the rule classes, 
without explicit mentioning of the rules themselves) is also included here, 
and the initial (= head), import and export categories are given. 

In the section on the rules, only the rule names are given, 
but not the exact rule formulation. For every rule, an 
example is given. If it is uncertain whether the example is correct (either 
because it may not be an example of the phenomenon in question, or because it 
may not be correct English), it is preceded by a question mark. Note that all 
explanation of the rules is given from a generative viewpoint
only, unless explicitly stated otherwise. Often, the information given in this 
document is based strongly on the comment already present in the documentation 
of the rules themselves. Discrepancies between what is stated here and what is 
said in the rule itself are usually caused by the fact that the rule file has 
not  been updated, although insights have changed. The semantics of the rules 
has been left unspecified in the current documentation, since it is not at all 
clear.

In doc.\ 150, {\em Subgrammars of English\/} (in which the definition of 
Rosetta3 for English was presented), the contents of the NPPROP subgrammar were 
not specified. Instead,
reference was made to an earlier document by Franciska de Jong, doc.\ 117: {\em 
The Subgrammars specific to Nominal Constituents\/}. For NPPROPs, this document 
specifies only two rule classes, Startrules and Modification rules, together 
with a transformation class for Pattern rules. Only the first of these three 
classes has found its place in the current implementation. Hence, there will be 
no comparison of the rule classes described in this document with the contents 
of doc.\ 117.

Finally note that the rules described in this document have NOT been tested 
properly. English analysis is not possible yet (there is no Surface Parser), and 
English generation has only been tested in as far as the construction was the 
translation of a Dutch sentence to be tested.

\newpage
\section{NPPROPformation}
The exact contents of the NPPROPformation subgrammar are mainly determined 
by the requirements posed by the XPPROPtoCLAUSE subgrammar for all its heads,
so they pertain to those XPPROPs that will become a full sentence by means of 
the copula {\em be\/}.
Thus, one rule class is needed to provide the variables expected in the 
XPPROPtoCLAUSE subgrammar, while transformations are not needed at all.

The NPPROP may be pruned in the sentence grammar, just leaving an
NP. However, since the head of the NPPROPformation subgrammar is a complete 
NP (and not a SUBX, as in comparable grammars), there is no need to specify 
all kinds of attributes of the NP predicate (they have already been assigned a 
value in the NPformation subgrammar). This also means that the task of 
the startrules is limited. The predicate already exists, and only the 
PROP-structure has to be defined.

The NPPROPformation subgrammar is not meant for identificational or 
existential NPPROPs. These are covered in separate subgrammars (see the 
documentation on these grammars: doc.\ 396 on IdentPropFormation and doc.\ 395 
on ExistPropFormation).

\section{Subgrammar Specification}
The subgrammar definition can be found in the file which also contains all the 
rules of this subgrammar, {\bf english:NPsubgrammars.mrule}, 
which is {\em mrules56.mrule\/}.

\begin{verbatim}
%SUBGRAMMAR NPPROPformation


   ( RC_NPPROPformation )
.  { RC_NPAdvVar }

\end{verbatim}

\begin{description}
  \item[Head]  NP \ \ \ \ FROM (NPformation)
  \item[Export] NPPROP
  \item[Import] NPVAR, CNVAR, ADJPPROPVAR, ADVPPROPVAR, NPPROPVAR, 
PREPPPROPVAR, VERBPPROPVAR, SENTENCEVAR, EMPTYVAR, PROSENTVAR, CLAUSEVAR,
ADVPVAR, PREPPVAR 
\end{description}

\newpage
\section{Rules and Transformations}

\subsection{RC\_NPPROPformation}
\begin{description}
\item[Kind] Obligatory Rule Class
\item[Task] To build an NPPROP around the head NP and its subject argument. The 
rule class contains only one rule.

In the rule, the Aktionsarts of the NPPROP are set to {\em [stative]\/};
no separate transformation class is needed to to determine this `standard'
value. 

The rule excludes subject arguments consisting only of the word {\em it, 
this\/} or {\em that\/}; these forms should be dealt with by the 
identificational 
NPPROP subgrammar. This also implies that the derivation process for these 
structures is different, since the 
identificational subgrammar takes an NPVAR as predicate, not a full NP. Thus, 
in the CLAUSEtoSENTENCE subgrammar, substitution will be needed.
Existential NPPROPs contain no subject at all, 
so they can never be confused with the current subgrammar. The NPPROPformation 
subgrammar is not suited yet to deal with NPs that do not take any argument, 
like {\em winter: It is winter\/} (with a non-referential {\em it\/}) cannot 
be derived now. Such a rule will be added, however.

The superdeixis of the NP is passed on to the NPPROP and removed from the NP; 
the genericity of the NP is set to omegageneric (both have to do with the 
inability of the Surface Parser to find the correct value during analysis).

\vspace{1 cm}
\begin{description}
\item[Name] RNPPROPformation
\item[Task] To provide a PROP structure for a non-generic NP that takes a 
subject argument and has {\bf posspred} = {\em true\/}.
\item[File] english:NPsubgrammars.mrule (mrules56.mrule)
\item[Semantics]
\item[Example] a doctor + x1 $\rightarrow$ x1 a doctor (She became a doctor)
\item[Remarks]
\end{description}

\item[Remark] In doc.\ 117, certain determiners ({\em some, most, every\/}) and 
measure noun heads ({\em three bottles of wine\/}) were mentioned as probably 
not being allowed to form predicative NPPROPs. Since they have received the 
value {\em false\/} for the attribute {\bf posspred}, they are indeed excluded 
from the current rule.\\
Since there are no rules for NPs that take any other argument besides a 
subject, there 
is no need for separate Pattern Rules following this rule class. All 
constituents
going with the NP have already been introduced in the NPformation subgrammar.\\
Since no rules have been written yet to build NPs out of bare count nouns, a 
structure like {\em She became queen\/} cannot be made yet.


\end{description}

\newpage
\subsection{RC\_NPAdvVar}
\begin{description}
\item[Kind] Iterative Rule Class
\item[Task] To introduce variables for adverbials. The classes of adverbials 
accounted for here are: TempAdvs, SentAdvs and LocAdvs. Also, variables for 
adverbial subordinate sentences (ConjsentAdvs) are introduced here. 
The rule class is iterative, but the rules are 
formulated in such a way that in analysis only one ordering of 
var-deletion is possible 
for the different adverbials. This is to prevent unnecessary ambiguities in 
analysis. The ordering chosen is the same as in all other English 
XPPROPformation subgrammars:\\
- first, the tempadvvars are removed (first the retrospective one, than the 
non-retrospective referential one),\\
- then, the locadvvars are removed (from right to left), and\\
- finally, the (conj)sentadvvars are removed (also from right to left).\\
In generation, any order of var-insertion is allowed, so it can be dictated 
completely by the source language.

\subsubsection{TempAdvVar}
\begin{description}
\item[Task] To introduce variables for time adverbials. 
The introduction of temporal adverbials is necessary only for those NPPROPs 
that will take the copula {\em be\/} and form an expression on their own. 
NPPROPs that go on to the {\bf toNPPROP} subgrammar are not supposed to receive 
their own time adverbials. 

The variable may be for a sentence, a prepp or an advp. 

\vspace{1 cm}
\begin{description}
\item[Name] RNPrefvarInsert
\item[Task] To introduce a variable for a referential time adverbial that is 
not retrospective. 
\item[File] english:NPsubgrammars.mrule (mrules56.mrule)
\item[Semantics]
\item[Example] x1 a junkie + refVAR $\rightarrow$ x1 a junkie refVAR 
(That man is a junkie)
\item[Remarks]
\end{description}

\vspace{1 cm}
\begin{description}
\item[Name] RNPretrovarinsertion
\item[Task] To introduce a variable for a retrospective time adverbial. 
\item[File] english:NPsubgrammars.mrule (mrules56.mrule)
\item[Semantics]
\item[Example] x1 captives + retroVAR $\rightarrow$ x1 captives retroVAR 
(They have been captives for three years now)
\item[Remarks]
\end{description}

\item[Remark] There is no rule for a durative variable. Perhaps it is 
needed: {\em They were prisoners for three years\/}.

\end{description}

\subsubsection{(Conj)SentAdvVar}
\begin{description}
\item[Task] To introduce variables for adverbial subordinate sentences in 
different positions and sentence or causal adverbials in initial position. The 
conjunction may also be a Preposition. No rules have been written yet for 
abstract conjunctions.

The IL strategy in mapping the different 
conjsent rules is to preserve the surface order used in the source language as 
much as possible, because it may be of importance for pronominal reference. 
However, this restriction is expressed only in the mapping of the substitution 
rules. The var-introduction rules are all mapped onto each other.

\vspace{1 cm}
\begin{description}
\item[Name] RNPConjsentVar
\item[Task] To introduce a variable for an adverbial subordinate sentence (or 
sentential PREPP) in initial (leftdislocrel) position.
\item[File] english:NPsubgrammars.mrule (mrules56.mrule)
\item[Semantics]
\item[Example] \mbox{}\\
x1 your new neighbours + advSENTENCEVAR $\rightarrow$ \\
advSENTENCEVAR x1 your new neighbours\\
(Although they do not live here yet, these people are your new neighbours)\\
x1 our best candidate + advPREPPVAR $\rightarrow$ \\
advPREPPVAR x1 our best candidate\\
(Without wanting to make you nervous, this man is our best candidate)
\item[Remarks] The comma that is probably obligatory after the adverbial 
sentence is not introduced here, but in the proposition substitution rules of 
the XPPROPtoCLAUSE subgrammar (see RConjSentSubst and RConjPrepNPSubst). \\
Note that the 
second example sentence given here cannot actually be derived yet, since there 
is no controller for the verb in the subordinate sentence.
\end{description}

\vspace{1 cm}
\begin{description}
\item[Name] RNPFinalConjsentVar
\item[Task] To introduce a variable for an adverbial subordinate sentence (or 
sentential PREPP) in final (postsentadvrel) position.
\item[File] english:NPsubgrammars.mrule (mrules56.mrule)
\item[Semantics]
\item[Example] \mbox{}\\
x1 your new neighbours + advSENTENCEVAR $\rightarrow$ \\
x1 your new neighbours advSENTENCEVAR \\
(These people are your new neighbours, although they do not live here yet)\\
x1 our best candidate + advPREPPVAR $\rightarrow$ \\
x1 our best candidate advPREPPVAR \\
(This man is our best candidate, without considering all the details of his 
application)
\item[Remarks] Note that the 
second example sentence given here cannot actually be derived yet, since there 
is no controller for the verb in the subordinate sentence. \\
In the proposition substitution rules, a comma between the main sentence and 
the subordinate one is accepted, but not generated (see RFinalConjSentSubst and 
RFinalConjPREPNPSubst).
\end{description}

\vspace{1 cm}
\begin{description}
\item[Name] RNPSentadvVar
\item[Task] To introduce a variable for a causal or sentence adverbial or a 
causal prepp in 
initial position. No rules have been written to account for any other position 
of the adverbial.
\item[File] english:NPsubgrammars.mrule (mrules56.mrule)
\item[Semantics]
\item[Example] \mbox{}\\
x1 our new manager + sentADVVAR $\rightarrow$ sentADVVAR x1 our new manager 
(Probably, this man is our new manager)\\
x1 the champions + causPREPPVAR $\rightarrow$ causPREPPVAR x1 the champions
(For that reason, the club of Chelsea are the champions)
\item[Remarks] The comma that is obligatory after most sentence adverbs 
is not introduced here, but in the substitution rules of 
the CLAUSEtoSENTENCE subgrammar (see RSentPreppSubst and RSentAdvSubst). \\
\end{description}

\end{description}

\subsubsection{LocAdvVar}
\begin{description}
\item[Task] To introduce variables for non-argument locatives (ADVP or PREPP) 
at a fixed position (locadvrel) following the NP. No rules have been 
written to account for any other position 
of the locative.

\vspace{1 cm}
\begin{description}
\item[Name] RNPLocAdvVar
\item[Task] To introduce a variable for a non-argument locative ADVP 
\item[File] english:NPsubgrammars.mrule (mrules56.mrule)
\item[Semantics]
\item[Example] x1 the king + locADVVAR $\rightarrow$ x1 the king locADVVAR
(He was the king there)
\item[Remarks]
\end{description}

\vspace{1 cm}
\begin{description}
\item[Name] RNPLocPreppVar
\item[Task] To introduce a variable for a non-argument locative PREPP
\item[File] english:NPsubgrammars.mrule (mrules56.mrule)
\item[Semantics]
\item[Example] x1 the king + locADVVAR $\rightarrow$ x1 the king locADVVAR
(He was the king at the club)
\item[Remarks]
\end{description}

\end{description}

\end{description}
\end{document}

