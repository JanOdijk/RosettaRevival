\documentstyle{Rosetta}
\begin{document}
   \RosTopic{General}
   \RosTitle{Notulen Groepsvergadering 12-3-1990}
   \RosAuthor{Petra de Wit}
   \RosDocNr{431}
   \RosDate{25-4-1990}
   \RosStatus{approved}
   \RosSupersedes{-}
   \RosDistribution{Project}
   \RosClearance{Project}
   \RosKeywords{minutes}
   \MakeRosTitle



\hyphenation{be-oordelen}
\hyphenation{samen-werking}
\hyphenation{woorden-boek}
\begin{itemize}
  \item {\bf aanwezig}: Andr\'{e} Schenk, Jan Landsbergen, Lisette Appelo,
                     Franciska de Jong, Petra de Wit, Elly van Munster, 
                     Elena Pinillos, Joep Rous, Jan Odijk, Harm Smit,
                     Ren\'{e} Leermakers, Frank Uittenbogaard.

  \item {\bf afwezig}: Josien Willems, Harold Leurs 
  \item {\bf Agenda}:
    \begin{enumerate}
       \item Opening en notulen
       \item Diversen
       \item CAP
       \item Boek
       \item Voortgang CRE:
         \begin{enumerate} 
  	 \item Rosetta 3D
         \item Localros 
         \item Conjugator
         \item Demo Brievenvertaler
         \end{enumerate}
       \item Rondvraag
    \end{enumerate}
\end{itemize}

\section {Opening en notulen}
De notulen van de vergaderingen op 19 februari, 26 februari en 2 maart worden 
met enkele wijzigingen aangenomen.

\section {Diversen}
\begin{enumerate}
   \item Jan L. doet verslag van de volgende punten uit de sectorvergadering:
   \begin{enumerate}
      \item Er is momenteel een discussie gaande over wie de werknemers in 
vakgroepen 6 en 7 moet beoordelen. Tot nu toe werd dit gedaan door 
de desbetreffende adjunct directeur; met ingang van heden worden niet-academici
in 
vakgroep 6 door de groepsleider beoordeeld . Het voorstel is om in de toekomst
iedereen in vakgroep 6 en 7 door de groepsleider te laten beoordelen.
      \item Het groepsbudget is als volgt samengesteld: 
           \begin{enumerate}
   	   \item investeringen: Hiervoor is 120.000 gulden beschikbaar, met 
name voor de conversie van SUN3 naar SUN4 workstations.
           \item kleine instrumentaria:
              \begin{enumerate}
              \item laserprinter : deze is al besteld
              \item VT330 : gaat niet door
              \item PCMAC : blijven staan
              \end{enumerate}
           \end{enumerate}
   \item Het budget voor reizen naar conferenties bestaat uit 1 
Amerika-reis 
en 3 Europa-reizen. Jan L. zal een lijstje laten rondgaan waarop iedereen zijn 
voorkeur voor een bepaald congres aan kan geven. Utrechters kunnen wellicht via
de STT reizen aanvragen. Jan L. dringt aan op matiging.
   \end{enumerate}
   \item Elly en Elena zijn inmiddels verhuisd naar de derde verdieping. In 
         ruil voor de vrijgekomen kamer 226 krijgen wij nu kamer 331 en niet 
         "de grote kamer" zoals gehoopt.
   \item Jan L. vraagt of er nog reacties zijn op de publicatielijst. Franciska
         meldt dat ze hem nog niet heeft doorgenomen en Ren\'{e} mist 
         paginanummers. Sinds het verschijnen van deze lijst zijn er zijn 
nog twee publicaties bijgekomen, namelijk \'{e}\'{e}n 
         van Franciska in het blad \underline{Informatie} en \'{e}\'{e}n van 
Jan L. in het Philips blad \underline{Horizon}. Franciska zal proberen 
reprints te bestellen van haar artikel.
   \item De Esprit II projecten PLUS en STEM hebben van de EEG de c van 
         'considered' gekregen. Het SRI project gaat door; misschien kunnen wij 
dat project volgen.
   \item Acquilex : Er is een tape naar
Amsterdam gestuurd met daarop een verrijkte Van Dale en een lijst van engelse 
werkwoorden waarover wij graag informatie uit de Longman zouden ontvangen.   
   \item Er wordt een cursus "Natuurwetenschappelijk Japans" aangeboden. 
Belangstellenden kunnen zich binnen twee weken bij Jan L. melden.
\end{enumerate}

\section {CAP}

Meneer Khatchadourian van CAP is op bezoek geweest om te praten over 
samenwerking bij het ontwikkelen van een interactief vertaalsysteem op basis 
van Rosetta. Het te ontwikkelen vertaalsysteem zou brieven en berichten moeten 
kunnen vertalen op een geavanceerde PC. Voor deze inspanning is 3 maal 10 
manjaar gepland. Over de vorm van samenwerking moet nog worden nagedacht. 
CAP zou vooral bijdragen aan de software engineering en uiteindelijk als 
distributiekanaal dienen. Waumans 
vindt deze ontwikkeling boeiend. CAP zal deze maand nog reageren.

\section {Boek}

Waumans heeft positief gereageerd op het plan een boek over Rosetta te 
schrijven. Hij beseft dat het schrijven van het boek zeker 1-1,5 manjaar 
gaat kosten, maar merkt op dat van de auteurs ook een bijdrage in de vrije tijd 
gevraagd mag worden.
Leidraad bij onze planning is de prognose dat het schrijven van een 
bladzijde ongeveer een dag werktijd mag kosten. In principe gaat iedereen 
hier mee 
akkoord.

Theo Janssen is op bezoek geweest om samen met Jan L. en Lisette over het boek 
te praten. In deze vergadering zijn inhoudelijke kritiekpunten besproken, 
alsmede enkele misverstanden uit de weg geholpen. In dit gesprek was Theo niet 
erg enthousiast
om het boek met de fictieve auteur Dr. Rosetta Stone te publiceren. 

Inmiddels meent Jan L. dat het 
initiatief niet bij Theo moet liggen, maar bij ons. Hij heeft hierover met Theo 
gesproken en onder deze omstandigheden heeft Theo geen bezwaar tegen de
fictieve auteur Dr. Rosetta Stone. Het plan is om in juni met een 
tegenvoorstel te komen.
Het boek zou dan in juni 1991 klaar moeten zijn, wat inhoudt dat de eerste 
versies, met name die van H1-H10, rond de kerst klaar moeten zijn.

\section {CRE}

\subsection {Rosetta 3D} 

Joep meldt dat de volgende woordenboektaken af zijn:
   \begin{enumerate}
   \item Integratie testwoordenboeken Ned/En/Sp (95\%)
   \item Vullen Nederlandse Adj's + splitsen
   \item Crash-vrij maken Engelse werkwoorden
   \item Mapping Adj's Nederlands-Engels
   \item Verwijderen Adv's met waarde irrel 
   \end{enumerate}
Er wordt gewerkt aan de generatie van engelse adj's (HS),  het nalopen van de 
Engelse stammen (AS) en het toekennen van waarden voor het attribuut adjuncts 
(JO,PdW,EP). De tweede en verdere betekenissen van Nederlandse werkwoorden in 
het Dict woordenboek zijn gemarkeerd.

Skeydef en Mkeydef zijn beide door Joep opgedeeld in twee files, een "big" en 
een "small" file. Nieuwe mkeys en skeys dienen aan respectievelijk smallmkeydef 
en smallskeydef toegevoegd te worden. 

Jan O. meldt dat hij aan "gelieve" , coordinatie en data heeft gewerkt.
Taken L19, L20, L25 komen te vervallen.

Jan L. vraagt zich af of we op de CRE een briefje kunnen vertalen. Jan 0. merkt 
op dat als we van te voren bepalen welk briefje dat is, dit mogelijk is. Er 
zijn twee mogelijkheden om dit op de CRE te presenteren. We kunnen Rosetta zelf 
draaien, zodat we zin voor zin moeten vertalen of gebruik maken van de al 
bestaande brieven-interface. Een combinatie van beide op \'{e}\'{e}n terminal 
lijkt niet zo'n goed idee.


\subsection {Localros} 

Joep is bezig met een interface voor Localros. De werking van Localros met het 
grote woordenboek moet nog getest worden. 

\subsection {Conjugator}

Lisette meldt dat Elly zich zal bezighouden met het bijwerken van de Spaanse 
morfologie en het toevoegen van uitspraak informatie aan de woordenboeken 
(voor zover dit nog niet gedaan is). Elena zal zich bezig houden met de 
werkwoorden
en het testen hiervan. Ren\'{e} heeft een programma geschreven waarmee 
werkwoorden 
die in een file staan getest kunnen worden. Met Jan O. zal worden bepaald 
welke regels uit
de grammatica relevant zijn. Het idee is dat de conjugator verschijnselen zoals 
clitics, passieven en negatie aan moet kunnen. Op 16 april moet het geheel 
werken. Er komt een aparte user voor de Conjugator.

\subsection {Demo Brievenvertaler}

Frank zal een mooie stand alone demo maken, waarin het vertaalproces wordt 
weergegeven. 

\section {Rondvraag}

Er waren geen vragen voor de rondvraag.

\end{document}
