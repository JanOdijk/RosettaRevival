\hyphenation{groeps-bij-een-komst}
Notulen softwaregroepbijeenkomst d.d. 3 November 1986. \\ \\ 
Aanwezig: Rene Leermakers, Joep Rous, Nick Spaen, Jan Stevens en Jeroen 
Medema.\\ 
\begin{itemize}
 \item {\bf Aktiepunten}\\
 1. Het installeren van de performance tool blijft staan voor de volgende 
 vergadering.
 \item {\bf Mededelingen}
 \begin{itemize}
  \item De VAX is uitgebreid met geheugen en disc. Dit heeft evenwel twee 
        inplaats van een week geduurd.
  \item De L(ange) T(ermijn) P(lanning) is vanwege de extra week 
        onbeschikbaarheid van de computer en de drie weken onbeschikbaarheid 
        van Carel aangepast. De LTP is te vinden achteraan deze notulen.
  \item Daar de bevroren versie van RBS toch iets te koud was, is het syteem 
        weer ontdooid door het blokkeren van IBUILD en de creatie van twee 
        batch queues voor de executie van Builds: S(mall)BUILD voor kleine jobs 
        en L(arge)BUILD voor de grotere.
  \item Er is ook een nachtbatch voor integrates nodig.
  \item Voor problemen met RBS moet je zijn bij of Carel of Joep. De notulist 
        vraagt zich af waarom deze opmerking in de notulen moet: er zijn toch 
        nooit problemen met RBS?
  \item De interface van windows gaat veranderd worden.
  \item Mensen die windows gebruiken moeten hun tabsettings standaard hebben 
        staan.
  \item Een fout in de Pascal-compiler is gevonden: 175 geneste IF's kan deze 
        niet meer aan. Dit is gemeld aan DEC.
  \item De stagiaires hebben (eindelijk) een terminal.
 \end{itemize}
 \item {\bf Software}\\
  Er is een Van Dale interface gemaakt voor het makkelijker schrijven van 
  programma's die gebruik maken van deze woordeboeken.
 \item {\bf Korte termijn planning}\\ 
 Het volgende is afgesproken voor de periode tot 17 November:\\
 \begin{tabular}{lcl}
  Jan S. &:&RMS interface onderzoek.                      \\
  Nick   &:&Woordenboek editor.                           \\
  Joep   &:&Morfologische componenten en co\"{o}rdinator. \\
  Carel  &:&Ziek en I/O interface (windows).              \\
  Rene   &:&Lextree-regel compilatie en vakantie.         \\
  Jeroen &:&I/O interface (windows).                      \\
 \end{tabular}
 \item {\bf Aktiepunten}\\
 \begin{tabular}{llcl}
  1.&Installeren performance tool&:&Jan S.\\
  2.&Approved maken notulen      &:&Joep en Jeroen\\
  3.&Paging file aanpassen       &:&Jan S.
 \end{tabular}
\end{itemize}
