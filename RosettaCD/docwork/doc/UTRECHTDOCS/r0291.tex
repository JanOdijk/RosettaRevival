 \documentstyle{Rosetta}
 \begin{document}
      \RosTopic{Rosetta3.Linguistics.minutes}
      \RosTitle{Notulen Linguistenvergaderingen 10-11-88 en 24-11-88}
      \RosAuthor{Margreet Sanders}
      \RosDocNr{R0291}
      \RosDate{1-12-88}
      \RosStatus{approved}
      \RosSupersedes{-}
      \RosDistribution{Linguists, Joep Rous}
      \RosClearance{Project}
      \RosKeywords{minutes, imperatieven, finite-infinite mapping, Getal, 
          locatieve argumenten, `allowed - toegestaan', BIGPRO, `het - er',
          subjectszinnen}
      \MakeRosTitle

\begin{description}
  \item[Aanwezig:] Lisette Appelo, Franciska de Jong, Elly van Munster, 
Jan Odijk, Margreet Sanders, Andr\'{e} Schenk, Harm Smit
  \item[Afwezig:] --
  \item[Agenda 10-11-88:] \mbox{}
    \begin{enumerate}
    \item Imperatieven
    \item Integreren
    \item Vertaling van Infiniete in Finiete zinnen
    \end{enumerate}
  \item[Agenda 24-11-88:]
    \begin{enumerate} \mbox{}
    \setcounter{enumi}{3}
    \item Notulen 20-10 en 3-11
    \item Vertaling van Nouns: Getal
    \item Locatieve argumenten
    \item ADJs die in VERBs vertalen
    \item BIGPRO
    \item Subjectszinnen: Het, er
    \end{enumerate}
\end{description}

\section{Imperatieven}
Op dit moment zijn de imperatiefregels uit de verschillende talen niet op elkaar 
afgestemd (vooral de onderliggende subjectsvariabelen niet). Na enige discussie
wordt de volgende mapping afgesproken:\\
Imp.\ sg.: jij kom / you(sg) come / tu ven\\
Imp.\ plur.: jullie komt / you(pl) come / vosotros(-as) venid\\
Imp.\ polite: U kom (!) / you(sg, pl) come / usted venga(n)\\
Natuurlijk wordt het subject uiteindelijk gedeleerd. De beleefdere vormen als
{\em Komt U binnen, Do come in\/} etc.\ stellen we nog even uit, tot de juiste
vertaling in alle talen duidelijk is.

\section{Integreren}
Lisette verzoekt dringend om geen integraties te starten die v\'{o}\'{o}r 18.30
uur gaan consolideren, omdat mensen die een lokale versie van het systeem hebben
dan niets meer kunnen builden zonder helemaal overnieuw te moeten beginnen.
Een verzoek aan Ren\'{e} en Joep of zij iets kunnen maken zodat de verschillende
talen tegelijk kunnen integreren lijkt op dit moment geen haast te hebben, 
en wordt daarom nog maar even uitgesteld.

Wat wel aan hen gevraagd zal worden is om een lijstje te maken van wat er nu
precies eerst gebuild moet worden voordat je bepaalde dingen als domeinen en 
m-regels lokaal kunt testen.

\section{Vertaling van Infiniete in Finiete zinnen}
Sommige open infiniete zinnen laten zich gemakkelijk omzetten in een finiete 
zin: 
(Hij meende) {\em iets te horen $\rightarrow$ dat hij iets hoorde\/}. Voor 
andere zinnen ligt het een stuk moeilijker, omdat de open infiniete
zin eigenlijk iets modaals heeft: (Hij beloofde) {\em (om) dit te doen
$\rightarrow$ dat hij dit {\em zou} doen\/}. In het Nederlands lijkt dit
verschil bepaald te worden door welk soort infinitief de bijzin heeft:
een {\em teinf\/} is een niet-modale bij\-zin, een {\em 
omteinf\/} (met een optioneel {\em om\/}) is wel modaal. De mapping van de
{\em teinf\/} moodregel naar IL gaat daarom zowel naar finiete als naar 
infiniete moodregels, terwijl de {\em omteinf\/} op Linjunsub wordt gemapt,
een soort kruising tussen een injunctief en een subjunctief. Lisette merkt op
dat een dergelijke {\em omteinf\/} bijzin ook nooit onafhankelijke tijd heeft.

In het Spaans en Engels lijkt er echter niet zo'n duidelijk attribuut te zijn 
wat het verschil tussen beide soorten infinieven aangeeft, en is de vertaling 
van Linjunsub dan ook een probleem. Een eerste voorstel van Jan om voor het
Engels een nieuwe `fortoinf' moodregel te schrijven leidt tot enige verwarring.
Daarom wordt besloten dat Jan O., Margreet en Elly eerst eens zullen kijken
om welke Nederlandse werkwoorden het gaat, en wat hun vertaling is in het 
Engels en Spaans, met welke soort bijzinnen. Misschien blijkt er toch enige
systematiek te bestaan.

\section{Notulen 20-10 en 3-11}
Margreet zal nog wat kleine wijzigingen aanbrengen, maar verder worden de 
notulen goedgekeurd. 

N.a.v.\ punt 1 (testzinnen) merkt Margreet op dat er inmiddels gewerkt wordt 
aan een 
document waar alle zinnen en constructies in staan + wat ze testen + wat de
vertaling is (voor Engels en Spaans). De relevante files zijn allemaal te 
vinden op de nla150, onder [schenk.margreet]testbank*.tex. Degenen die deze
gegevens nog niet duidelijk hadden opgeschreven, wordt verzocht de relevante
files zelf aan te passen.

N.a.v.\ punt 4 zegt Lisette dat inmiddels is besloten de regels 
voor superdeixis niet optioneel te maken. Verder geldt voortaan dat in de 
XP-grammatica's modificatieregels verplicht n\'{a} de superdeixisregels moeten 
komen. In de XPPROP-grammatica's is dat ook het handigst, maar als dat echt 
niet kan en de modificatieregels geordend zijn v\'{o}\'{o}r superdeixis, dan 
moeten in de betreffende subgrammatica ook superdeixis-adaptation 
transformaties worden geschreven.

\section{Vertaling van Nouns: Getal}
Franciska deelt een document uit waarin een voorstel staat voor de vertaling
van singular en plural nouns. Op het moment wordt nog aangenomen dat alle nouns 
die de waarde {\em OnlyPlur\/} hebben voor het attribuut {\bf pluralforms} in 
feite singular zijn. Voor {\em NoPlurals\/} lag de situatie helemaal 
onduidelijk. In het nieuwe voorstel zijn er voorlopig twee regels: de eerste 
vertaalt alle `semantische enkelvouden', d.w.z. hij werkt alleen op count nouns
die singular zijn (boek $\rightarrow$ books, speeltje $\rightarrow$ speeltjes; 
maar ook voor het NoPlur vent $\rightarrow$ bloke).
De tweede regel werkt voor gewone meervouden (boeken $\rightarrow$ books, 
speeltjes $\rightarrow$ toys), maar ook voor woorden die een mass betekenis 
hebben en syntactisch al dan niet een meervoudsvorm vertonen: speelgoed 
$\rightarrow$ toys, hersenen1 $\rightarrow$ brains.

Er zijn nog geen enkelvoudsregels voor woorden die OnlyPlur zijn \`{e}n een 
count-achtige betekenis hebben: limbs = ledematen, limb = ??. Ook zijn er nog 
geen regels voor woorden die een extra woord nodig hebben in het meervoud,
zoals {\em 2 stel hersenen\/}. 
Overigens zijn de regels die extra woorden in het enkelvoud toevoegen er 
ook nog niet (bv.\ voor {\em 2 pair(s) of glasses, 2 loaves of bread, (een stel) 
hersenen\/}) en er is ook nog geen attribuut dat zegt welk woord nodig is.
In hoeverre het nodig is om per taal al onderscheid te maken tussen count en 
mass versies van hetzelfde woord (brood1 - brood2; hersenen1 - hersenen2) blijft 
nog even onduidelijk. Voor het Engels zijn er ook nog regels nodig voor het 
behandelen van SingAndPlur nouns zoals {\em sheep\/} en nouns die zowel als 
singular als als plural kunnen functioneren in een zin, zoals {\em team\/} (The
team is ... - The team are ...).

Franciska zal in elk geval de twee regels zoals die in het voorstel staan 
implementeren en alle woorden uit het stencil in de woordenboeken zetten, zodat 
deze gevallen getest kunnen worden. Het lijkt erop dat het Spaans geen nieuwe 
verschijnselen heeft die weer een aparte behandeling vereisen.

\section{Locatieve argumenten}
Naar aanleiding van de vertaling van het adjectief {\em gewend\/} in het 
werkwoord {\em to settle down\/} ontdekte Franciska het volgende. 
Kennelijk zijn de afspraken over de categorie van locatieve en directionele 
adverbiale (en prepositionele)
argumenten nooit goed hard gemaakt, want in de zinsgrammatica wordt
aangenomen dat het altijd ADVPPROPs zijn ({\em x1 woont [x1 hier]\/}), terwijl 
in de ADJPPROP-grammatica wordt verondersteld dat het het ADVPs zijn ({\em 
x1 [hier] gewend\/}). Jan O.\ noemt als argument voor de keuze voor PROPs
het feit dat ze niet zomaar kunnen passivizeren: $^{*}${\em er wordt door de 
asbak op de grond gelegen, ?Er wordt hier gewoond\/}.  
Het argument van con\-trole speelt eigenlijk niet zo erg, en dan nog het meest
bij `resultatieve' props als {\em Hij zet de boeken in de kast\/}.

Aangezien veranderingen in de zinsgrammatica nogal drastisch lijken, wordt 
besloten dan maar de ADJPPROP grammatica te veranderen. Er moeten nieuwe 
patroonregels komen, substitutieregels, en controleregels. Jan O.\ biedt 
Franciska zijn hulp aan.

Lisette merkt op dat bij de evaluatie van Rosetta3 de beslissing voor de PROP-
categorie van deze argumenten nog eens goed onder de loep moet worden genomen.

\section{ADJs die in VERBs vertalen}
Voor een algemene discussie over de problemen met de adjectieven {\em 
toegestaan, allowed, permitido\/} vs.\ de verbs {\em toestaan, allow, 
permitir\/} zie de notulen van de vorige keer, sectie 5: Modalen. De oplossing 
die toen gekozen is maakt alle adjectieven 2-plaatsig, en alle verbs 
3-plaatsig. Bij het vullen van de woordenboeken ontdekten Franciska en Elly 
echter een probleem: de zin {\em Het is hem toegestaan dit te doen\/}, met het 
2-plaatsige adjectief {\em toegestaan\/}, moet in het Spaans vertalen naar een
werkwoord, en dat moet 3-plaatsig zijn: {\em Se le permite hacerlo\/}. Enige 
discussie volgt, maar het probleem wordt niet opgelost. Daar Jan O.\ zich kan 
herinneren dat er nog een ander probleem was maar niet meer precies weet welk,
wordt de verdere discussie uitgesteld tot een volgende vergadering.

Overigens wil het Spaans voor bepaalde ADJs die een QSENT nemen als argument 
nog wat extra woorden in de zin: {\em Es curioso {\em por saber} quien ...\/}
(Hij is er benieuwd naar wie ...), en ook {\em Es curioso por saber\/} (Hij is 
er benieuwd naar). Afhankelijk van het aantal ADJs dat dit gedrag vertoont
worden ze als idioom behandeld, of moet er een apart synpattern voor komen 
(synPORSABERQSENT of zoiets...)

\section{BIGPRO}
Jan O.\ meldt dat de oplossing die hij een vorige vergadering noemde voor het 
onthouden van het antecedent van BIGPROs d.m.v.\ een attribuut {\em index\/}
(zie de notulen van de vorige keer, sectie 6) op een praktisch probleem stuit:
een BIGPRO is een basisexpressie, en het is niet netjes om in de attributen van 
een basisexpressie te gaan veranderen. Daarom stelt hij voor een BBIGPRO in te 
voeren naast een BIGPRO, waarbij BBIGPRO weer gewoon alleen een KEY heeft 
(naast alle andere BPERSPRO attributen), en 
BIGPRO het attribuut {\em index\/}. Jan zal deze verandering zelf 
doorvoeren, en belooft dat niemand er last van zal hebben.

\section{Subjectszinnen: Het, er}
Na de lunchpauze gaan we verder met het laatste punt: een documentje dat Jan 
O.\ heeft geschreven over de behandeling van zinnen als subject en als 
complement. 
Aanleiding tot dit stuk was een vraag van de lexico's wanneer een synpattern nu 
HETTHATSENT is, en wanneer THATSENT, etc.

Alle zinnen in {\em subjects\/}positie (heeft dus niets met de synpatterns te 
maken)
moeten daar weg; ze worden \`{o}f ge\"{e}traponeerd, \`{o}f ze gaan verder naar 
voren en krijgen een leftdislocrel. In het laatste geval wordt er een dummy 
subject {\em het \/} voor in de plaats gezet. Een analyse met subjectszinnen 
voor werkwoorden als {\em verbazen, ergeren, boeien\/} etc.\ is overigens 
enigszins controversieel. Een probleem is ook, dat sommige werkwoorden, zoals 
{\em impliceren\/}, een voorkeur lijken te hebben voor leftdislocrel: {\em ?Het 
impliceert niets dat hij ziek is\/} vs.\ {\em Dat hij ziek is impliceert niets
\/}. In ingebedde zinnen kan leftdislocatie meestal niet.

Complementzinnen worden in het Nederlands altijd ge\"{e}xtraponeerd, en de hoofdzin 
krijgt dan al naar gelang het synpattern wel of geen {\em het\/} ervoor in de 
plaats in het geval van een direct object; voor een prepositioneel object is 
{\em het\/} verplicht. Als er geen {\em het\/} wordt ingevuld en het gaat om 
een ergatief werkwoord, dan wordt de subjectspositie uiteindelijk gevuld door het
dummy subject {\em er\/}.

Franciska zal nakijken of de behandeling die zinscomplementen in de 
ADJPPROP-grammatica krijgen in overeenstemming is met de gang van zaken zoals 
in het stuk van Jan O.\ staat beschreven. Als een zin rond een adjectief ook 
{\em Er\/} als subject kan krijgen, moet dat adj dus ergatief zijn.\\[2 cm]

{\Large Rondvraag}\\
Margreet vraagt of het stuk van Franciska over de vulling van de NOUNs nog 
besproken gaat worden. Dit zal op de volgende linguistenvergadering gebeuren.
\end{document}
