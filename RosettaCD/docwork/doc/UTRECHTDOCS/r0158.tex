\documentstyle{Rosetta}
\begin{document}
   \RosTopic{linguistics}
   \RosTitle{The DETP-subgrammar}
   \RosAuthor{Franciska de Jong}
   \RosDocNr{0158}
   \RosDate{\today}
   \RosStatus{concept}
   \RosSupersedes{-}
   \RosDistribution{linguists}
   \RosClearance{Project}
   \RosKeywords{DETP, BNUM, QP}
   \MakeRosTitle
%
%   \input{R0000_Chap1}
%   \input{R0000_App1}
%

\section{Introduction}

This document is an extension and updating of section 4 and 5 of doc. 117.
It discusses the subgrammar that accounts for the vcarious kinds of 
expressions that can occur in prenominal position and express 
the detrel-relation (determiner-relation).
\\ \\
The (relevant part of the) proposals of doc. 41 will be incorporated. That 
is, except DETP-nodes dominating a single determiner, possibly modified by 
what traditionally is called predeterminer (cf. structure (1), 
we distinguish partitive 
determiners of the scheme: {\em DETP van/of/de DETP}. Cf. structure (2).
A consequence of the analysis of partitive NPs by means of a partitive
DETP analysis, embedding of NPs is a more restricted phenomenon than under
alternative analyses base don the scheme {DETP-Empty Noun-PP}, or 
{\em DETP-van/of/de-DETP}.

\begin{verbatim}
(1)            NP

     det               head

     DETP               CN

                (mod)  (mod)   head

                ...     ...    NOUN 


 (2)                  NP

           det               head

           DETP               CN

   (mod)  (mod)   head

    DETP     of   DETP
\end{verbatim}

In section 2, the control-expression for the DETP-subgrammar and its
various rule classes are discussed. 
Section 3 discusses the peculiarities of the
proposal with respect to the account of the synonomy 
of NPs with a PP-modifier expressing a possessive
relation, e.g. {\em het boek van Jan}, and NPs with a possessive determiner,
e.g. {\em Jans boek}. 
In order to allow an isomorphic treatment, both possessive genitives
(which occur as determiner) as well as postnominal {\em van/of/de}-PPs are
introduced by one and the same ruleclass of the NP(Prop)-subgrammar.
(Cf. doc. R0157). How these modifiers occuur in their correct form in the
appropriate surface position is discussed in section 3 of the present 
document. 

\section{The organisation of the DETPsubgrammar}
\subsection{The control expression}

\begin{verbatim}
DUTCH
import: (BDET, BNUM, QP), DETP
export: DETP

control expression:

RC: makedetprules.[RC: detpmodrules].[RC: partformatrules]

ENGLISH
import: (BDET, BNUM, QP), DETP
export: DETP

control expression:

RC: makedetprules.[RC: detpmodrules].[RC: partformatrules]

SPANISH
import: (BDET, BNUM, QP), DETP
export: DETP

control expression:

RC: makedetprules.[RC: detpmodrules].[RC: partformatrules]
\end{verbatim}

\subsection{Comments on the various rule classes }
\begin{verbatim}
a. RC: makedetprules
import: (BDET, BNUM, QP)
export: DETP 
\end{verbatim}
This ruleclass creates DETP-nodes. (As there are no detpatterns the
name startdetprules might be confusing). Primarily DETP-nodes 
dominate basic expressions
such as the determiners {\em alle}, ??{\em de}, {\em sommige}, {\em beide}, 
etc., the numerals
{\em vele}, 
{\em enkele}, {\em weinige}, etc. plus the cardinals. A DETP-node can also
dominate quantifier phrases such as the simplex {\em meer} and {\em genoeg}, 
the
complemented 
{\em meer dan ...}, {\em de meeste van allemaal}, and the complex QPs
formed with the shape of an NP, such as {\em een (klein) beetje}. The class of
simple QPs is distinguished from the basic determiners that are traditionally
considered 
to be quantifiers as well, such as {\em alle} and {\em some}. The reason is
that QPs have a very specific distribution: they can also occur as degree
modifier to adverbs and adjectives, as postnominal modifier, and as predicate.
These positions are not accessible to the quantifying determiners. (Probably
this distribution favours an analysis in terms of QPProp. This issue is ignored
here.)
\\ \\
A special case in English is the formation of a DETP out of the complex string
{\em such a}. Semantically the element {\em such} is a modifier to either the
nominal head ({\em such a fool}) or to a prenominal modifier ({\em  such a
foolish guy}). Though this certainly is not accounting for this semantic role,
the string might be treated as a complex basic expression (fixed idiom), 
with {\em zo'n}
and {\em zulke} as its translation. Alternatively we might opt for a more 
sophisticated account, and consider both {\em a} and {\em such} as a basic
expression and extend the control expression with 
a transformation class to account for their shifting. Which of the two 
approaches is to be preferred must be decided by a more thorough comparison
of the translations in Dutch and Spanish which is postponed to the rule-writing 
stage. 
\\ \\ 
Numerals are supposed to be dominated by a DETP in both prenominal positions
they may appear in. That is, in the position where a numeral functions as
(or in case of a partitive, see below: in) a determiner to the noun and where
it is a sister to node CN, and in the position where a numeral is a modifier to
the noun and a daugther to node CN. Cf. structure (1). Under this
position a NUMP-subgrammar becomes superfluous.

\begin{verbatim}
b. RC: detpmodrules 
import: (DETP), ADVP, DETP
export: DETP
\end{verbatim}

This ruleclass introduces modifiers to DETP's. Both DETP's headed by a BNUM as
well as DETP's headed by a BDET can be modified. Examples: {\em minstens
twintig} (van de) boeken, {\em bijna iedere} film, {\em al} haar werk. As in
general is the case with modification not every kind of modifier can be
attached to any kind of DETP. For example, {\em minstens} can not modify {\em
iedere}. It is not clear to me at the moment whether partitive DETP's,
constructed in a rule class ordered after RC: detpmodrules, can be modified as
a whole. However, in order to avoid ambiguity (between a status as NP-modifier
or as partitive-DETP- modifier), it is preferable to pursue an analysis
without this possibility. 

\begin{verbatim}
c. RC: partformatrules 
import: (DETP), DETP
export: DETP
\end{verbatim}

This ruleclass combines certain kinds of DETP's into a complex DETP that in
combination with a noun might form a so-called partitive NP. (Some DETPs headed
by a QP are excluded from this possibility.) The general scheme assumed for
partitive NPs given in (2) involves the occurrence of {\em van} (Dutch), {\em
of} (English) or {\em de} (Spanish) in between the two DET's combined. This
preposition-like element is not considered a basic expression. It is to be
introduced syncategorematically. 

Note that it is assumed here that of the two DETP connected by {\em van/of/de}
the rightmost is considered to be the head. The reason why the RC:
partformatrules do combine two DETP's instead of one DETP and one BDET (or
BNUM), or two BDET's (or a BDET and a BNUM) is the fact that both elements 
independently allow modification. 

\section{An isomorphic treatment of [VAR's N] and [de N van VAR]}

The three Rosetta-languages differ with respect to the possibility to express
possessive relations by means of prenominal phrases with genitive case, i.e.
with morpheme {\em 's} (or some variant). In Spanish the genitive possessive
does not exist at all, while English and Dutch differ with respect to the
frequency of the construction. Therefore Rosetta should in general provide an
isomorphic treatment for phrases with a prenominal genitive and phrases with a
postnominal modifier expressing a possesive relation. The isomorphy should be
realized not only "between" subgrammar(s), but also within the subgrammar(s) of
Dutch and English, in order to account for the general synonymy of the scheme's
{\em NP's N} and {\em the N of NP}/{\em de N van NP}. In addition to the
mapping of these two scheme's, the rules should account for the Spanish
construction in (3).

\begin{verbatim}
  (3) a. El libro mio/tuyo/suyo/nuestro/vuestro
      b. Los libros mios/nuestros/... etc.
      c. Las casa(s) mio/mia/.....etc.
\end{verbatim}

The possessive pronouns occurring in these examples syntactically behave as
if they were adjectives. Note the number/gender agreement. However, as they
correspond to postnominal PP-modifiers or prenominal determiners in English and
Dutch, and as their semantic effect is slightly more complicated than for
adjectives they are to be introduced by a specific subset of rules which,
unlike the rules for adjectival modification, is not to be applied more than
once.

An important aspect of the data to account for is the fact that the synonymy of
the {\em NP's N} cases with the other forms is restricted to the examples with
a definite article/determiner: {\em mijn/Juans boek} has the same meaning as
{\em het boek van mij/Juan}, but not as {\em een boek van mij/Juan}. (Note also
that phrases like {\em mijn/Juans boek} behave in all respects as definite
nominal phrases.) On the other hand one would like a uniform account of the
part of the NP corresponding to {\em boek van mij/Juan}, irrespective of the
kind of determiner preceding this part. The present proposal attempts to
realize this by an NP-analysis that involves a syncategorematic introduction
for the definite article {\em the}. In generation this introduction is ordered
after the rules that deal with the formation of a CN-level, just like all other
rules for the introduction of determiners. As the possessive relation
semantically should be dealt with on a par with other noun modifications, e.g.
as expressed by adjectives, the introduction of possessive elements is supposed
to belong to the domain of the modification rules that precede CN-formation.
(Cf. doc. R0157.) In order not to complicate the syntactic effect of these 
modification rules separate rules are distinguished for the introduction
of possesive modifiers finally that are meant to occur in prenominal 
position, and those that are will surface as a postnominal PP. (But note 
that both are 
mapped  onto the same meaning rule.)
\\ \\
The following derivations are meant as a global illustration of this idea.
In view of the Rosetta approach to the introduction of NPs into
sentences, all these phrases should be generated with a VAR for the genitive NP.
Cf. the title of this section. For the sake of transparancy the eventual
substituents i.e. {\em mij} and {\em Juan} are used in the trees to follow.
\begin{verbatim}
(4) data: DEFINITES: mijn/Juans boek, het boek van mij/Juan
                     my/Juan's book, the book of ?mine/Juan 
                     mi libro, el libro mio, el libro de Juan

          INDEFINITES: een boek van mij/Juan
                       a book of mine/Juan's
                       un libro mio/de Juan
\end{verbatim}
\newpage
\begin{verbatim}
         Derivation trees                     S-trees      

DUTCH DEFINITES:
                                              NP{+def}

                                         detrel       CN

                                       NP/POSSADJ       N

                                       Juan{gen-s}    boek
                                       mijn
             R npformation1,2

                                              NP{+def}

                                        DETP            CN

                                        het       NOUN          PP

                                                  boek      van Juan
                                                            van  mij


                                                  CN
                                            
                                          NOUN          NP

                                          boek         Juan              
                                                        ik
             R modposs1,2

                                                  CN

                                          NOUN          PP

                                          boek          van Juan
                                                        van mij
     R cnformation    R npformation

     boek                 Juan                           
                           ik
\end{verbatim}
\newpage
\begin{verbatim}

ENGLISH DEFINITES
                                              NP{+def}

                                         detrel       CN

                                       NP/POSSADJ      N

                                       Juan{gen-s}    book
                                       my
             R npformation1,2

                                              NP{+def}

                                        DETP            CN

                                        the       NOUN          PP

                                                  book      of Juan
                                                            of  my


                                                  CN
                                            
                                          NOUN          NP

                                          book         Juan              
                                                        I
             R modposs1,2

                                                  CN

                                          NOUN          PP

                                          book          of Juan
                                                        of mine
     R cnformation    R npformation

     book                 Juan                           
                           I
\end{verbatim}
\newpage
\begin{verbatim}
SPANISH DEFINITES
                                              NP{+def}

                                       detrel         CN

                                        POSSADJ       N

                                        mi          libro
                                       
             R npformation1,2

                                              NP{+def}

                                        DETP            CN

                                        el        NOUN       PP/POSSADJ

                                                  libro      de Juan
                                                             mio


                                                  CN
                                            
                                          NOUN          NP

                                         libro          yo
                                                        
             R modposs1,2
                                                  CN

                                          NOUN          PP/POSSADJ

                                          libro         de Juan
                                                        mio

     R cnformation    R npformation

     libro                 Juan                           
                           yo

\end{verbatim}
\newpage
\begin{verbatim}
DUTCH INDEFINITES

             R npformation                       NP{-def}
                                                 
                                       DETP          CN

                                       een       N          PP

                                                boek       van Juan
                                                           van mij         
             R modposs2

     R cnformation    R npformation

     boek                 Juan                           
                           ik


ENGLISH INDEFINITES

             R npformation                       NP{-def}
                                                 
                                       DETP          CN

                                        a       N          PP

                                                book       of Juan
                                                           of mine
             R modposs2

     R cnformation    R npformation

     book                 Juan                           
                           I
\end{verbatim}
\newpage
\begin{verbatim}
SPANISH INDEFINITES

             R npformation                       NP{-def}
                                                 
                                       DETP          CN

                                       een       N          PP/POSSADJ

                                                libro      de Juan
                                                           mio
             R modposs2

     R cnformation    R npformation

      libro            Juan                           
                       yo                             

\end{verbatim}

For partitive NPs with a possessive genitive occurring e.g. {\em twee van Juans
boeken}/{\em twee van de boeken van Juan}, this analysis is adequate as well.
Up to the formnation of CN things are fully equal. For the introduction of the
complex determiner a decision is to be made. For the non-partitive definite
cases a syncategorematic introduction of {\em the }, {\em het}, {\em de} etc,
is assumed. The {\em DETP van}-part of the partitive DETP requires that at
least the DETP, for example {\em twee} in the example above, be introduced via
the introduction of a basic expression. For the formation of partitive DETPs it
was uptil now assumed that two DETPs each derived from basic expressions, were
involved. However, if we treat the second determiner in the DETP as a basic
expression it is to be deleted in case we wish to generate {\em een van Juans
boeken}. The alternative, a syncategorematic introduction of this definite
determiner for the semantic equivalent {\em een van de boeken van Juan}, would
require special provisions within the the rules for the formation of a partive
determiner.  Another possibility is to assume an abstract definite determiner,
which is to be spelled out by a transformation of the NP-subgrammar under
certain conditions. I propose to adopt the latter alternative. (Not yet
incorporated in the preceding sections.)\\ \\

A special case that is requirea a separate discussion is the Dutch combination
of a definite article and a POSSPRO. For example: {\em de mijne}, {\em de
jouwe}, {\em de onze}, {\em het hare}. Here it assumed that these strings are
examples of headless NPs and that they are to be translated into English NPs
with {\em one} substituted for an empty head and a POSSADJ.  E.g. {\em my one(s
)}. 
\footnote{Given the fact that the selection of the article {\em het} or {\em 
de} is context dependent the translation from English to Dutch will 
necessarily consist of double output, as sentences provide no information
to determine the apropriate grammatical gender. }
Under these assumptions the preceding treatment are of use here too. The
only extra requirement to be made is that RC:NPformationrules should contain a
rule that takes  a CN consisting of EN plus a NP dominated PERSPRO as imput,
and yields a structure with EN preceded by a complex DETP dominating a DET plus
a POSSPRO. (The EN will be deleted by the TC:emptyheaddeletion.) Schematicaly:
\begin{verbatim}

             R npformation1                       NP{+def}
                                                 
                                           DETP             CN

                                     DET       POSSPRO      EN
                                     de/het    mijne     

                                           
                                                  CN
             R modposs1
                                             EN         NP     
     R cnformation    R npformation
                                                        ik
     EN                   ik

\end{verbatim}
In fact forms such as {\em de/het mijne} are not the only possible translations
of {\em my one}. An adequate alternative would be {\em die van mij}. 
The treatment of possessive modification descibed above proposed can account
for the synonymy of {\em de mijne} and {\em die van mij} fairly easily.
It requires that RC:NP-formation contains a rule that syncategorematically
introduces {\em die} as a determiner preceding the CN-string {\em EN van mij}.
Scematically:
\begin{verbatim}

             R npformation3                       NP{+def}
                                                 
                                           DETP          CN

                                           die       EN       PP

                                                            van mij  
                                           
                                                  CN
             R modposs2
                                             EN         PP
     R cnformation    R npformation
                                                      van mij
     EN                   ik

\end{verbatim}
\end{document}
