\documentstyle{Rosetta}
\begin{document}
   \RosTopic{General}
   \RosTitle{Vierde halfjaarlijkse verslag aan Van Dale Lexicografie}
   \RosAuthor{H.E. Smit, J.P. Medema}
   \RosDocNr{0256}
   \RosDate{February 29, 1988}
   \RosStatus{informal}
   \RosSupersedes{-}
   \RosDistribution{Project}
   \RosClearance{Project}
   \RosKeywords{dictionary, Van Dale report}
   \MakeRosTitle
%
%
\hyphenation{woor-den-boe-ken werk-zaam-he-den hoe-veel-heid mor-fo-lo-gische
       scheid-baar pro-ble-men gram-ma-ti-caal ver-voe-gings-in-for-ma-tie}

\section{Inleiding}

In dit vierde halfjaarlijkse verslag zal een beknopt 
overzicht gegeven worden van de 
werkzaamheden aan de N-N, N-E en E-N bestanden in de periode van september 1987
tot en met februari 1988 (zie sectie 2).

Tevens is weer een lijst van gevonden fouten en inconsistenties toegevoegd
(zie sectie 3).

\section{Overzicht van de werkzaamheden}

Zoals in het vorige halfjaarlijkse verslag al was aangegeven, kunnen slechts 
enkele `ROSETTA-attributen' automatisch gevuld worden op grond van informatie 
in de bestanden. Het gaat hierbij om morfologische informatie (zoals de 
vervoeging van werkwoorden, de meervoudsvorming van nouns, grammaticaal 
geslacht van nouns, e.d.) die --voor het Nederlands-- voorkomt in de N-N, en 
--voor het Engels-- in de E-N. 
Door de bestanden N-N en N-E aan elkaar te koppelen, kan de verbinding tussen
Nederlandse morfologische informatie en de vertaalslag gelegd worden.

Om de Engelse morfologische informatie te kunnen gebruiken, zou een link
moeten worden gelegd tussen N-E (samen met N-N) enerzijds en E-N anderzijds. Dit
blijkt echter problematisch: zeer veel Engelse vertalingen uit de N-E
zijn {\em niet} als ingang opgenomen in de E-N, en ook tussen de Nederlandse 
vertalingen uit de E-N en de ingangen van de N-E en N-N lijkt weinig 
overeenkomst te zijn. Van het leggen van deze link is derhalve afgezien; dit is
mede mogelijk omdat de Engelse morfologie --in tegenstelling tot de 
Nederlandse--
z\'{o} regelmatig is dat de 
attribuutwaarden vrijwel probleemloos `default' gevuld kunnen worden (waarbij
slechts een vrij beperkte hoeveelheid uitzonderingen ge\"{e}valueerd moet 
worden).

Het leggen van de link tussen N-N en N-E is gelukkig wat minder problematisch 
maar ook niet triviaal, zoals uit een van de volgende secties zal blijken.

De stand van zaken is momenteel als volgt: de automatisch afleidbare
morfologische attributen voor het Nederlands worden op basis van de N-N gevuld,
en de vertaalslag komt uit de N-E. Alle overige Nederlandse en {\em alle} 
Engelse attributen worden `handmatig' gevuld. De E-N wordt alleen gebruikt als
{\em informatiebron} voor onregelmatige vormen; bij een evaluatie bleek dat de
zo verkregen informatie nauwelijks verschilde van wat er in de gangbare
grammaticaboeken te vinden was.

In de volgende secties zal nader worden ingegaan op de hierboven geschetste 
activiteiten.

\subsection{Automatisch vullen van Nederlandse morfologische attributen}

Naast het programma dat de meervoudsvorming van {\em zelfstandige naamwoorden},
zoals die in de N-N is aangegeven, omzet in de overeenkomstige ROSETTA-waarden,
is er nu een programma geschreven voor de {\em werkwoorden}.

Dit programma herschrijft de vervoeging van werkwoorden in de N-N in een 
gemakkelijk naar ROSETTA-attribuutwaarden omzetbare code. Tevens wordt de stam 
van het werkwoord gegenereerd, en wordt genoteerd of het werkwoord een 
scheidbaar partikel en/of een reflexief ({\em `zich'}) in de vervoeging heeft.
Ook wordt onthouden of de voltooide tijd van het werkwoord met {\em hebben}, met
{\em zijn} of met {\em hebben} \`{e}n {\em zijn} gevormd wordt.
Het programma heeft inmiddels over de werkwoorden gedraaid; als resultaat 
hiervan zijn o.a. 347 fouten en inconsistenties in de vervoegingsinformatie
gevonden.

\subsection{Link tussen N-N en N-E}

Het leggen van een link tussen N-N en N-E bleek het eenvoudigst te realiseren 
door
de doorsnede van beide bestanden te bepalen. Omdat beide bestanden op dezelfde
woordenschat zijn gebaseerd, vallen er in zowel N-N als N-E maar weinig
woorden {\em niet} in deze doorsnede.

Er is een programma geschreven dat de doorsnede van de N-N en de N-E genereert;
d.w.z. alle ingangswoorden die in beide bestanden voorkomen (waarbij 
gecontroleerd wordt of de hoofdcategorie klopt) worden --met morfologische 
informatie \`{e}n vertaling-- weggeschreven naar een
file. Daarnaast worden woorden die slechts in \'{e}\'{e}n der beide bestanden 
voorkomen
naar aparte files weggeschreven zodat onderzocht kan worden welke woorden
bij het maken van de intersectie wegvallen. 
Woorden die in de beide bestanden in hun lemma verschillen in de romeinse 
onderverdeling leveren problemen op omdat de morfologische informatie per
romeinse ingang kan verschillen. Ook deze woorden worden naar een aparte file 
weggeschreven en zullen ge\"{e}valueerd moeten worden (het betreft veelal
hoogfrequente woorden die in de uiteindelijke woordenboeken niet mogen 
ontbreken).

\subsection{Subset van de intersectie van N-E en N-N}

Omdat slechts een aantal morfologische attributen automatisch van hun waarde
kan worden voorzien, zullen veel attributen handmatig gevuld moeten worden.
Omdat het aantal woorden in de intersectie van N-N en N-E te groot is om in 
een tijdsverloop van enkele maanden gevuld te kunnen worden, zal in eerste
instantie alleen een {\em subset} van de woorden gebruikt worden. Het ligt in 
de bedoeling voor deze subset de woorden van ROSETTA2 te gebruiken
(het gaat om circa 5000 woorden), met dien 
verstande dat deze woorden dan wel met al hun vervoegingen en verbuigingen, en
in al hun betekenissen opgenomen worden. Om de ROSETTA2-woorden uit de 
intersectie van N-N en N-E te kunnen bepalen is een programma geschreven.

\subsection{Gebruik E-N}

Uit de E-N zijn lijsten afgeleid met onregelmatige Engelse werkwoorden
en nouns. Deze lijsten bleken weinig te verschillen van overeenkomstige lijsten
uit grammatica's. 

Verder is er een lijst van alle adjectieven waarbij `grammaticaal
commentaar' staat (dit zijn niet altijd inflectioneel onregelmatige 
adjectieven; het 
commentaar kan bijvoorbeeld ook op het {\em gebruik} van het adjectief slaan).
Overigens is ook voor de adjectieven uit de N-N een dergelijke lijst gemaakt.

\subsection{Document over de afbeelding van Van Dale op ROSETTA-woordenboeken}

Er is een document gemaakt waarin beschreven is hoe de Van Dale bestanden
voor ROSETTA gebruikt kunnen worden; dit betreft de globale struktuur van het
lemma (waarbij de N-E als basis dient) en de morfologische informatie (uit
de N-N). In het document is aangegeven hoe een lemma uit de N-E omgezet zou 
kunnen worden in een (of meer) lemma's voor de ROSETTA-woordenboeken, waarbij
met name lemma's met spellings- en vormvarianten, romeinse onderverdeling, e.d.
tot ingewikkelde afbeeldingen aanleiding geven.

\section{Fouten en inconsistenties}

Naast de al genoemde fouten in vervoegingen bij werkwoorden, waarvan de lijst
te lang zou worden om hier op te noemen, zijn er ook nog andere fouten gevonden:
\begin{itemize}
   \item het woord {\em geluidsmontage} is in de N-N fout gespeld; er staat:
         {\em geluismontage}.
   \item het werkwoord {\em hoeden} heeft in de N-N de hoofdcategorie `1' na de
         `GR'-code. De vervoeging is alleen na `I' vermeld, en niet na `II'.
   \item het woord {\em jus} heeft in de N-E het cijfer `1' i.p.v. `I' bij het
         eerste romeinse sublemma.
   \item in het lemma van het bijvoeglijk naamwoord {\em koperen} is in de N-E
         een fout geslopen: in het laatste voorbeeld is in de gedrukte versie
         niet het  bedoelde `1/2' gekomen maar `I' gevolgd door een punthaak.
   \item het woord {\em kijker} heeft in de N-E voor de variant vorm 
         {\em kijkster} `0.17' staan.
   \item bij het woord {\em acre} staat in de E-N bij de grammaticale 
         informatie in de gedrukte versie `$\rightarrow$t', wat overeen zou 
         komen met een verwijzing naar `t' in het grammaticaal compendium (op
         tape wordt de `t' voorafgegaan door de code `CI').

         Overigens zijn er nog meer woorden waarbij dit verschijnsel optreedt.

   \item bij het woord {\em iemand} in de N-E is als vertaling bij 0.1 het 
         Engelse woord {\em some{\em n}one} i.p.v. {\em someone}.
   \item in de N-N staat bij {\em aerobics} in de betekenisomschrijving
         {\em zuurtsof} i.p.v. {\em zuurstof}.
\end{itemize}

\end{document}
