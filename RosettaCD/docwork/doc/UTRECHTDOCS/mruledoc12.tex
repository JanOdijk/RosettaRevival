\documentstyle{Rosetta}
\begin{document}
   \RosTopic{Rosetta3.doc.Mrules.English}
   \RosTitle{Rosetta3 English M-rules: ADVPFORMULAtoPROP}
   \RosAuthor{Margreet Sanders}
   \RosDocNr{388}
   \RosDate{\today}
   \RosStatus{concept}
   \RosSupersedes{-}
   \RosDistribution{Project}
   \RosClearance{Project}
   \RosKeywords{English, documentation, Mrules, ADVPFORMULAtoPROP}
   \MakeRosTitle
%
%

\section{Introduction}
As all main category grammars, the English ADVPPROP grammar is divided in 
three parts. First, {\bf ADVPPROPformation} forms the PROP-structure. Then, {
\bf ADVP\-PROPtoFORMULA} turns this PROP into an intermediate structure, called 
ADVPFORMULA
(comparable to the CLAUSE in the sentence grammar). Finally, {\bf 
ADVPFORMULAtoPROP} makes an open or closed PROP-structure, comparable to the 
SENTENCE level in the sentence grammar. Prior to ADVPPROPformation, there is 
the {\bf AdvDerivation} grammar, which was discussed in doc.\ 316, {\em 
Rosetta3 English M-rules: Derivation Subgrammars\/}. The final open or closed 
ADVPPROP is usually input to the Proposition Substitution Rules in the 
XPPROPtoCLAUSE subgrammar. An OPENADVPPROP may also be used in RC\_AdvVar, for 
subject-oriented verb modifying adverbs. Theoretically, a
 `bare' ADVPPROP may also form an expression on 
its own with help of the copula {\em be\/}. In that case, it would leave the 
ADVPPROP grammars after ADVPPROPformation (so it would not be input to the 
subgrammar described in this document), and 
be input to the ClauseFormation Rules of the XPPROPtoCLAUSE subgrammar. 
However, no example of a predicatively used adverb could be found, and the 
relevant Clause Formation rule has not been written.
In English, there also is no rule to make a complete Utterance of a closed 
ADVPPROP.

The current document describes the contents of the third ADVPPROP subgrammar, 
ADVPFORMULAtoPROP (the first subgrammar was discussed in doc.\ 386, {\em 
Rosetta3 English M-rules: ADVPROPformation\/}, and the second one 
in doc.\ 387, {\em Rosetta3 English Mrules: ADVPPROPtoFORMULA\/}). 
The subgrammar consists of 
a number of rule classes and transformation classes. A rule class in its turn
consists of a number of rules and a transformation class of a number of 
transformations. The relative ordering of the rules and transformations in the
(sub)grammar is indicated by a {\em control expression}. A summary of this
control expression (i.e.\ a listing of the ordering of the rule classes, 
without explicit mentioning of the rules themselves) is also included here, 
and the initial (= head), import and export categories are given. Conditions on 
crucial orderings of rule classes, if they exist in the current subgrammar, are 
mentioned explicitly.

In the section on the rules and transformations, only the rule names are given, 
but not the exact rule formulation. What is attempted 
is to provide a detailed overview of the workings of the subgrammar, and 
how the different rule classes achieve this,
together with some comments on the problems still to be solved, the reasons 
behind certain choices, and perhaps possible alternatives. For every rule, an 
example is given, if one could be found. 
If it is uncertain whether the example is correct (either 
because it may not be an example of the phenomenon in question, or because it 
may not be correct English), it is preceded by a question mark. Note that all 
explanation of rules and transformations is given from a generative viewpoint
only, unless explicitly stated otherwise. Note that this 
document is an adapted copy of doc.\ 382 on the PREPPFORMULAtoPROP subgrammar 
(they are very similar).
Discrepancies between what is stated here and what is 
said in the documentation to the rule itself are usually caused by 
the fact that the rule file has 
not  been updated, although insights have changed. The semantics of the rules 
has been left unspecified in the current documentation, since it is not at all 
clear.

Finally note that the rules described in this document have NOT been tested 
properly. English analysis is not possible yet (there is no Surface Parser), and 
English generation has only been tested in as far as the construction was the 
translation of a Dutch sentence to be tested.

\newpage
\section{ADVPFORMULAtoPROP}
The exact contents of the ADVPFORMULAtoPROP subgrammar are mainly determined 
by the requirements of at least partial isomorphy with the CLAUSEtoSENTENCE 
and PREPPFORMULAtoPROP subgrammars (see doc.\ 386 on the PREPPPROPformation 
subgrammar for a more extensive discussion of the isomorphy relations). 

In doc.\ 150, it was assumed that the rules of the current subgrammar would 
mirror all important rule classes of the CLAUSEtoSENTENCE subgrammar, including 
rules for reciprocal and reflexive spelling. However, it seems doubtful whether 
these rules are useful here, and they have not been written. 

Most rule classes have only one rule, just providing some sort of 
`default' value. It is not impossible that more rules will have to be added 
once the treatment of adverbs in Rosetta3 has been worked out in more detail.

\section{Grammar Specification}
The grammar definition can be found in the file which also contains all the 
rules of this subgrammar, {\bf AdvpSubgrammars.mrule}, 
which is {\em mrules84.mrule\/}.

\begin{verbatim}
%SUBGRAMMAR PreppformulaTOprop


   ( TC_advppPROstatus)
.  { RC_advppSubst }
.  ( TC_advppAspectNeutralization )
.  ( RC_advppMood )
.  ( RC_advppPunc )

\end{verbatim}

\begin{description}
  \item[Head]  ADVPFORMULA  \ \ \ \ FROM (ADVPPROPtoFORMULA)
  \item[Export] OPENADVPPROP, CLOSEDADVPPROP
  \item[Import] NP
\end{description}

\newpage
\section{Rules and Transformations}

\subsection{TC\_advppPROstatus}
\begin{description}
\item[Kind] Obligatory Transformation Class
\item[Task] To assign a value to the attribute {\bf PROsubject}. The default 
value is {\em false\/}. In generation, there is a free choice between the two 
rules, and both an OPENADVPPROP (PROsubject = {\em true\/}) and a 
CLOSEDADVPPROP (PROsubject = {\em false\/}) can be made. If a subject 
substitution rule is applied (see below), only the version with PROsubject = {
\em false\/} is allowed; hence, the transformation class is ordered crucially 
before the substitution rules.

\vspace{1 cm}
\begin{description}
\item[Name] TadvppNoPROsubject
\item[Task] Vacuous rule, leaving the {\bf PROsubject} attribute of the 
ADVPFORMULA at its default value, which is {\em false\/}.
\item[File] english:AdvpSubgrammars.mrule (mrules84.mrule)
\item[Semantics] --
\item[Example] ? [x1 southwards]$_{PROsubject=false}$ (They live southwards) 
\item[Remarks] 
\end{description}

\vspace{1 cm}
\begin{description}
\item[Name] TadvppPROsubj
\item[Task] To set the {\bf PROsubject} attribute of the ADVPFORMULA at the 
value {\em true\/}.
\item[File] english:AdvpSubgrammars.mrule (mrules84.mrule)
\item[Semantics] --
\item[Example] [x1 enthousiastically x2]$_{PROsubject=false}$ $\rightarrow$ 
[x1 enthousiastically x2]$_{PROsubject=true}$ (She greeted him 
enthousiastically) 
\item[Remarks] 
\end{description}

\end{description}


\newpage
\subsection{RC\_advppSubst}
\begin{description}
\item[Kind] Iterative Rule Class
\item[Task] To substitute non-sentential expressions for their variable, in the 
order dictated by analysis and consistent with the substitution order 
conditions. There only is a rule for subject substitution.

The rule uses the system parameter LEVEL to check whether the index of the 
variable is consistent with (the level of) the variable that is to be 
substituted for according to the rule parameter.

For more information on the rule class, see the documentation to the comparable 
class in the CLAUSEtoSENTENCE subgrammar (doc.\ 370, p.\ 14). The same 
constraints hold, except that reflexives etc.\ have not yet been excluded from 
the current rule explicitly. 

There still is no way to limit the workings of the substitution rules in this 
grammar to those cases where it is really needed, i.e.\ where there will not be 
any substitution in the higher clause. Thus, many ambiguities arise caused by
substitution having applied in different cycles.

\vspace{1 cm}
\begin{description}
\item[Name] RadvppSubjSubst
\item[Task] To substitute a non-generic NP for its subjrel VAR
\item[File] english:AdvpSubgrammars.mrule (mrules84.mrule)
\item[Semantics]
\item[Example] ? [x1 home] + the man $\rightarrow$ [the man home] (The man 
seemed home)
\item[Remarks] This rule was added because in Spanish, some constructions never 
get a subject in the main clause (there only is something like {\em It seemed 
that the man was home\/})
\end{description}

\newpage
\subsection{TC\_advppAspectNeutralization}
\begin{description}
\item[Kind] Obligatory Transformation Class
\item[Task] To reset the attribute {\bf aspect} of the ADVPFORMULA, which was  
set at the value {\em imperfective\/} in the Aspect Rule Class (see previous 
subgrammar), to the 
value {\em omegaaspect\/}. This is needed because the Surface Parser cannot 
decide the correct value in analysis, while the Aspect Rule Class is an 
obligatory class in the sentence grammar.

\vspace{1 cm}
\begin{description}
\item[Name] TadvppAspectNeutralization
\item[Task] see above
\item[File] english:AdvpSubgrammars.mrule (mrules84.mrule)
\item[Semantics] --
\item[Example] [x1 southwards]$_{imperfective}$ $\rightarrow$ [x1 
southwards]$_{omegaaspect}$ (He fled southwards)
\item[Remarks]
\end{description}

\end{description}

\newpage
\subsection{RC\_advppMood}
\begin{description}
\item[Kind] Obligatory Rule Class
\item[Task] To turn an ADVPFORMULA into a closed or open ADVPPROP, depending 
on the value of the attribute {\bf PROsubject}. This rule is the 
`XPPROP-version' of the Mood Rules in the sentence grammar, but there are no 
new attributes going with the new topnode in the current subgramamar.

\vspace{1 cm}
\begin{description}
\item[Name] ROpenAdvppMood
\item[Task] To turn an ADVPFORMULA into an OPENADVPPROP
\item[File] english:AdvpSubgrammars.mrule (mrules84.mrule)
\item[Semantics] 
\item[Example] $_{ADVPFORMULA}$[x1 southwards] $\rightarrow$ 
$_{OPENADVPPROP}$[x1 southwards]
\item[Remarks]
\end{description}

\vspace{1 cm}
\begin{description}
\item[Name] RClosedAdvppMood
\item[Task] To turn an ADVPFORMULA into a CLOSEDADVPPROP
\item[File] english:AdvpSubgrammars.mrule (mrules84.mrule)
\item[Semantics] 
\item[Example] ? $_{ADVPFORMULA}$[the man home] $\rightarrow$ 
$_{OPENADVPPROP}$[the man home] (The man seemed home)
\item[Remarks] 
\end{description}

\end{description}

\newpage
\subsection{RC\_advppPunc}
\begin{description}
\item[Kind] Obligatory Rule Class
\item[Task] Vacuous rule, to serve as counterpart in the isomorphic scheme for 
the punctuation rules in the sentence subgrammar. 

\vspace{1 cm}
\begin{description}
\item[Name] RadvppNoPunc
\item[Task] see above
\item[File] english:AdvpSubgrammars.mrule (mrules84.mrule)
\item[Semantics] 
\item[Example] the man home
\item[Remarks]
\end{description}

\end{description}


\end{document}

