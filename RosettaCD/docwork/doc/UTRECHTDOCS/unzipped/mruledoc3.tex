
\documentstyle{Rosetta}
\begin{document}
   \RosTopic{Rosetta3.doc.Mrules.English}
   \RosTitle{Rosetta3 English Mrules: XPPROPtoCLAUSE}
   \RosAuthor{Margreet Sanders, Lisette Appelo}
   \RosDocNr{326}
   \RosDate{December 4, 1989}
   \RosStatus{approved}
   \RosSupersedes{-}
   \RosDistribution{Project}
   \RosClearance{Project}
   \RosKeywords{English, documentation, Mrules, XPPROPtoCLAUSE}
   \MakeRosTitle
%
%

\section{Introduction}
The English sentence grammar is divided in three parts, in the same way as all
other main category grammars. First, there is a subgrammar providing the 
PROP-structure, 
called {\bf VerbppropFormation}. Then, {\bf XPPROPtoCLAUSE} turns this prop
into a clause. Finally, {\bf ClauseToSentence} makes a full sentence of this
clause. Around the Sentence grammar there are two other, very small grammars.
Prior to VerbppropFormation, there is the {\bf VerbDerivation} grammar, and
following ClauseToSentence, there is the {\bf Utterance} grammar.

The current document describes the contents of 
the second Sentence subgrammar, XPPROPtoCLAUSE (the first sentence subgrammar 
has been described in doc.\ 310, {\em Rosetta3 English Mrules: 
Verbppropformation\/}). The subgrammar consists of 
a number of rule classes and transformation classes. A rule class in its turn
consists of a number of rules and a transformation class of a number of 
transformations. The relative ordering of the rules and transformations in the
(sub)grammar is indicated by a {\em control expression}. A summary of this
control expression (i.e.\ a listing of the ordering of the rule classes, 
without explicit mentioning of the rules themselves) is also included here, 
and the initial (= head), import and export categories are given. Basically, 
the document describes the system as it was when the document was first 
written, i.e.\ in June 1989. Modifications that took place later have been 
added in footnotes.

In the section on the rules and transformations, only the rule names are given, 
but not the exact rule formulation. What is attempted 
is to provide a detailed overview of the workings of the subgrammar, and 
how the different rule classes achieve this,
together with some comments on the problems still to be solved, the reasons 
behind certain choices, and perhaps possible alternatives. For every rule, an 
example is given. If it is uncertain whether the example is correct (either 
because it may not be an example of the phenomenon in question, or because it 
may not be correct English), it is preceded by a question mark. Note that all 
explanation of rules and transformations is given from a generative viewpoint
only, unless explicitly stated otherwise. Often, the information given in this 
document is based strongly on the comment already present in the documentation 
of the rules themselves. Discrepancies between what is stated here and what is 
said in the rule itself are usually caused by the fact that the rule file has 
not  been updated, although insights have changed. The semantics of the rules 
has been left unspecified, since it is not clear at all how to formulate it 
correctly.

Whenever the current implementation differs widely from the strategy that was 
devised in the definition phase of Rosetta3 (as laid down for English in docs.\ 
150, {\em Subgrammars of English\/}, 153, {\em Rule and Transformation Classes 
of English\/}, and 155, {\em Rule and Transformation Classes common to all 
languages\/}, all written by Jan Odijk), this will be indicated explicitly in 
the current document. Conditions on crucial orderings of rule classes will be 
repeated here, even if they do not differ from the original strategy, to make 
the document as self-contained as possible.

Finally note that the rules described in this document have NOT been tested 
properly. English analysis is not possible yet (there is no Surface Parser), and 
English generation has only been tested in as far as the construction was the 
translation of a Dutch sentence to be tested.

In principle, the authors of the rule classes have documented their own 
classes: Lisette Appelo wrote the sections on 
RC\_Aspect, RC\_Retro, TC\_Finiteness, RC\_Deixis, TC\_HaveModalAdaptation,
and TC\_SuperdeixisAdaptation. The rest of the document was written by Margreet 
Sanders.


 \newpage
\section{XPPROPtoCLAUSE}
In this second sentence subgrammar, all XPPROPs are turned into a CLAUSE. The 
most usual head for the subgrammar is a Verbpprop, but other possible heads are 
an ADJPPROP, an NPPROP (which may also be identificational or existential), a 
PREPPPROP and 
an ADVPPROP. Since these head categories all function in basically the same way 
as an 
ordinary Verbppprop once the copula {\em be\/} is introduced, it was decided to 
have only one XPPROPtoCLAUSE instead of five (one for each category X). This in 
contrast to what was stated in doc.\ 150, where all five subgrammars are 
mentioned explicitly. The only rule class that has to be able to deal with all 
these different head categories is the one for Proposition Substitution. Right 
after that, the copula {\em be\/} is introduced in the ClauseFormation rule 
class.

Also in this subgrammar are rules to take care of substitution of empty 
arguments, calculation of Aspect and (super)deixis, and introduction of 
VP-modifying adverbials. Transformations  move the leftmost verb 
out of the VERBP and the particle to its correct position, take care of 
extraposition, control and superdeixis 
adaptation, deal with raising, agreement and case assignment, and spell out 
finiteness, argument reflexives and reciprocals.

The organisation of the XPPROPtoCLAUSE subgrammar has changed somewhat when 
compared to the general lay-out as presented in doc.\ 150. Especially,
some rule classes have been split up and many transformation classes have been 
added. All this will be 
indicated below, in the comment on the rule classes. The Transformation Class 
TC\_PropOK has disappeared from the current subgrammar specification. Its 
functions have been taken over by several other rule and transformation 
classes: in RC\_PropSubst, restructuring of the clause for modals takes place; 
in TC\_ControlRules, the embedded subject of open propositions is deleted and 
the embedded proposition is pruned; in TC\_SentOKrules, care is taken of 
extraposition. New in the subgrammar is RC\_Deixis. For more information on the 
background of this rule class, see docs.\ 263, {\em Documentation of the rules 
for the translation of temporal expressions in Rosetta3, part I\/}, and 320, {
\em Superdeixis in Rosetta3\/}, both by Lisette Appelo.

\newpage
\section{Subgrammar Specification}
The subgrammar definition can be found in file {\bf 
english:XPPROPtoCLAUSE.mrule}, which is {\em mrules4.mrule\/}.

\begin{verbatim}
%SUBGRAMMAR XPPROPtoCLAUSE

    { RC_PropositionSubstitution: mrules41 and 40 }
.   ( RC_ClauseFormation: mrules39 
      . ( TC_ComparIncorp: mrules95 ) )
.   ( RC_Aspect: mrules38 )
.   [ RC_Retro: begin mrules76 ]
.   ( TC_Finiteness: mrules37 )
.   ( RC_Deixis: mrules35, 71, 72, 78 )
.   ( TC_HaveModalAdaptation: end mrules76 )
.   ( TC_SuperdeixisAdaptation: mrules73 and 89 )
.   ( TC_VerbLeft: mrules24 )
.   ( TC_SentOKrules: mrules33 )
.   ( TC_ControlRules: mrules32, 31 and 23 )
.   ( TC_FinControl: mrules23 )
.   { RC_VPAdv: mrules93 }
.   { RC_EMPTYsubst: mrules30 }
.   ( TC_ObjectOKrules: mrules29 )
.   { TC_ConjSentControl: mrules94 }
.   ( TC_SubjVerbAgr: begin mrules28 )
.   ( TC_ThatDel: rest mrules28 )
.   { TC_CaseAssignment: mrules27 and 26 }
.   { TC_ArgReflSpelling: mrules18 }
.   { RC_ReciprocalSpelling: mrules17 }
.   ( TC_ParticleHop: mrules25 )
\end{verbatim}
\begin{description}
  \item[Head] \mbox{}\\
    \begin{tabular}{ll}
VERBPPROP  & FROM (VerbppropFormation)\\
ADJPPROP   & FROM (AdjppropFormation)\\
NPPROP     & FROM (NppropFormation, IdentPropFormation, ExistPropFormation)\\
PREPPPROP  & FROM (PrepppropFormation)\\
ADVPPROP   & FROM (AdvppropFormation)
    \end{tabular}
  \item[Export] CLAUSE
  \item[Import] CLOSEDNPPROP, OPENNPPROP, CLOSEDADJPPROP, OPEN\-ADJP\-PROP,
CLOSEDPREPPPROP, OPENPREPPPROP, CLOSED\-VERBP\-PROP, OPENVERBPPROP, 
SENTENCE, CLOSEDADVPPROP, OPENADVPPROP, NP, EMPTY             
\end{description}


\newpage
\section{Rules and Transformations}
\subsection{RC\_PropositionSubstitution}

\begin{description}
\item[Kind] Iterative Rule Class
\item[Task] To substitute a proposition for its variable. 
This transformation class is ordered crucially before RC\_ClauseFormation for 
reasons of isomorphism: in Dutch, the {\em graag\/}-substitution rule must have 
applied before a clause can be formed. The head of this Transformation 
class can be any XPPROP. In case nothing is specified in the documentation 
below, the rule works for Verbpprops only.

There are 40 rules in total. Perhaps more are needed when new verbpatterns will 
be discovered, but this does not seem likely. Still missing in the rule 
class are rules for free adjuncts (OPEN and CLOSED XPPROPS) ({\em he ate the 
meat raw, he came in angry\/}), because the treatment of adjuncts has not been 
decided on yet.

All substitution rules have a parameter, LEVEL, which is used to indicate the 
depth of embedding of the substituent. This parameter is passed on to the 
target language, to ensure a proper mapping of variables and substituents. The 
exact value of this parameter is determined by the system itself, and is not 
set in the rules explicitly. There should also be some restriction on the order
of substitution in analysis. The argument substitution rules 
(in RC\_Substitution) use a special 
function QUOTE\_substordercondition for this, but that is not needed here, 
since many propositions cannot co-occur anyway, and since there is no need to 
look into the substituent. The conditions to be added are fairly simple and 
should mainly concern the substitution order of adverbial sentences, subject 
sentences and complement sentences.

For those rules that are not restricted to Verbpprops, there may be parallel 
rules in 
the other subgrammars (esp.\ for ADJPPROPtoFORMULA, which also has a set of
proposition substitution rules). Discrepancies or 
equivalences between those rules and the ones used in the current subgrammar 
have not been indicated, since they do not really matter for a good 
understanding of how things are organised in XPPROPtoCLAUSE.

\vspace{1 cm}
\begin{description}
\item[Name] RLocClosedPreppPropSubst
\item[Task] To substitute a locative CLOSEDPREPPPROP for its variable in 
loc\-arg\-rel. In analysis, the actsubcefs of the PREPPPROPVAR is set at {\em 
[loc]\/}.
\item[File] english:RC\_PropSubst1.mrule (mrules41.mrule)
\item[Semantics]
\item[Example] x1 get x2 $\rightarrow$ x1 get [the two sides round the table]
\item[Remarks]
\end{description}

\vspace{1 cm}
\begin{description}
\item[Name] RLocOpenPrepPPropSubst
\item[Task] To substitute a locative OPENPREPPPROP for its variable in 
locargrel. In analysis, the actsubcefs of the PREPPPROPVAR is set at 
{\em [loc]\/}. 
\item[File] english:RC\_PropSubst1.mrule (mrules41.mrule)
\item[Semantics]
\item[Example] x1 put x2 x3 $\rightarrow$ x1 put x2 [x2 on the table]
\item[Remarks] Control rules to relate the embedded subject variable to 
another variable and delete it follow later.
\end{description}

\vspace{1 cm}
\begin{description}
\item[Name]   RDirClosedPrepPPropSubst
\item[Task] To substitute a directional CLOSEDPREPPPROP for its variable in 
dirargrel. In analysis, the actsubcefs of the PREPPPROPVAR is set at 
{\em [dir]\/}.
\item[File] english:RC\_PropSubst1.mrule (mrules41.mrule)
\item[Semantics]
\item[Example] ? x1 see x2 $\rightarrow$ x1 see [they to the door]
\item[Remarks]
\end{description}

\vspace{1 cm}
\begin{description}
\item[Name] RDirOpenPrepPPropSubst
\item[Task] To substitute a directional OPENPREPPPROP for its variable in 
dirargrel. In analysis, the actsubcefs of the PREPPPROPVAR is set at 
{\em [dir]\/}. 
\item[File] english:RC\_PropSubst1.mrule (mrules41.mrule)
\item[Semantics]
\item[Example] x1 jump x2 $\rightarrow$ x1 jump [x1 off the fence]
\item[Remarks] Control rules to relate the embedded subject variable to 
another variable and delete it follow later.
\end{description}

\vspace{1 cm}
\begin{description}
\item[Name]   ROtherClosedPrepPPropSubst
\item[Task] To substitute a non-locative and non-directional CLOSEDPREPPPROP 
for its variable in 
complrel. In analysis, the actsubcefs of the PREPPPROPVAR is set at the set of 
`other subcs' that preps may have.
\item[File] english:RC\_PropSubst1.mrule (mrules41.mrule)
\item[Semantics]
\item[Example] ? seem x1 $\rightarrow$ seem [she with child]
\item[Remarks]
\end{description}

\vspace{1 cm}
\begin{description}
\item[Name] ROtherOpenPrepPPropSubst
\item[Task] To substitute a non-locative and non-directional OPENPREPPPROP for 
its variable in 
complrel. In analysis, the actsubcefs of the PREPPPROPVAR is set at the set of 
`other subcs' that preps may have.
\item[File] english:RC\_PropSubst1.mrule (mrules41.mrule)
\item[Semantics]
\item[Example] no examples found
\item[Remarks] Control rules to relate the embedded subject variable to 
another variable and delete it follow later.
\end{description}

\vspace{1 cm}
\begin{description}
\item[Name]   RLocClosedAdvPPropSubst
\item[Task] To substitute a locative CLOSEDADVPPROP for its variable in 
locargrel. In analysis, the actsubcefs of the ADVPPROPVAR is set at {\em 
[locadv]\/}.
\item[File] english:RC\_PropSubst1.mrule (mrules41.mrule)
\item[Semantics]
\item[Example] ? x1 get x2 $\rightarrow$ x1 get [they home]
\item[Remarks]
\end{description}

\vspace{1 cm}
\begin{description}
\item[Name] RLocOpenAdvPPropSubst
\item[Task] To substitute a locative OPENADVPPROP for its variable in 
locargrel. In analysis, the actsubcefs of the ADVPPROPVAR is set at 
{\em [locadv]\/}.  
\item[File] english:RC\_PropSubst1.mrule (mrules41.mrule)
\item[Semantics]
\item[Example] x1 live x2 $\rightarrow$ x1 live [x1 westwards]
\item[Remarks] Control rules to relate the embedded subject variable to 
another variable and delete it follow later.
\end{description}

\vspace{1 cm}
\begin{description}
\item[Name]   RDirClosedAdvPPropSubst
\item[Task] To substitute a directional CLOSEDADVPPROP for its variable in 
dirargrel. In analysis, the actsubcefs of the ADVPPROPVAR is set at 
{\em [diradv]\/}.
\item[File] english:RC\_PropSubst1.mrule (mrules41.mrule)
\item[Semantics]
\item[Example] no examples found
\item[Remarks]
\end{description}

\vspace{1 cm}
\begin{description}
\item[Name] RDirOpenAdvPPropSubst
\item[Task] To substitute a directional OPENADVPPROP for its variable in 
dirargrel. In analysis, the actsubcefs of the ADVPPROPVAR is set at 
{\em [diradv]\/}. 
\item[File] english:RC\_PropSubst1.mrule (mrules41.mrule)
\item[Semantics]
\item[Example] ? x1 swim x2 $\rightarrow$ x1 swim [x1 southwards]
\item[Remarks] Control rules to relate the embedded subject variable to 
another variable and delete it follow later.
\end{description}

\vspace{1 cm}
\begin{description}
\item[Name]   ROtherClosedAdvPPropSubst
\item[Task] To substitute a non-locative and non-directional CLOSEDADVPPROP 
for its variable in 
complrel. In analysis, the actsubcefs of the ADVPPROPVAR is set at the set of 
`other subcs' an adverb may have.
\item[File] english:RC\_PropSubst1.mrule (mrules41.mrule)
\item[Semantics]
\item[Example] no examples found
\item[Remarks]
\end{description}

\vspace{1 cm}
\begin{description}
\item[Name] ROtherOpenAdvPPropSubst
\item[Task] To substitute a non-locative and non-directional OPENADVPPROP for 
its variable in 
complrel. In analysis, the actsubcefs of the ADVPPROPVAR is set at the set of 
`other subcs' an adverb may have.
\item[File] english:RC\_PropSubst1.mrule (mrules41.mrule)
\item[Semantics]
\item[Example] no examples found
\item[Remarks] Control rules to relate the embedded subject variable to 
another variable and delete it follow later.
\end{description}

\vspace{1 cm}
\begin{description}
\item[Name]   RClosedPrepNPPropSubst
\item[Task] To substitute a CLOSEDNPPROP for its variable as object of 
a prepobj. 
\item[File] english:RC\_PropSubst1.mrule (mrules41.mrule)
\item[Semantics]
\item[Example] no examples found
\item[Remarks]
\end{description}

\vspace{1 cm}
\begin{description}
\item[Name] ROpenPrepNPPropSubst
\item[Task] To substitute an OPENNPPROP for its variable as object of a 
prepobj. 
\item[File] english:RC\_PropSubst1.mrule (mrules41.mrule)
\item[Semantics]
\item[Example] x1 come across as x2 $\rightarrow$ x1 come across as [x1 a nice 
person]
\item[Remarks] Control rules to relate the embedded subject variable to 
another variable and delete it follow later.
\end{description}

\vspace{1 cm}
\begin{description}
\item[Name]   RClosedPrepAdjPPropSubst
\item[Task] To substitute a CLOSEDADJPPROP for its variable as object of a
prepobj.
\item[File] english:RC\_PropSubst1.mrule (mrules41.mrule)
\item[Semantics]
\item[Example] ? strike x1 as x2 $\rightarrow$ strike x1 as [he pompous]
\item[Remarks]
\end{description}

\vspace{1 cm}
\begin{description}
\item[Name] ROpenPrepAdjPPropSubst
\item[Task] To substitute an OPENADJPPROP for its variable as object of a 
prepobj.
\item[File] english:RC\_PropSubst1.mrule (mrules41.mrule)
\item[Semantics]
\item[Example] x1 regard x2 as x3 $\rightarrow$ x1 regard x2 as [x2 clever]
\item[Remarks] Control rules to relate the embedded subject variable to 
another variable and delete it follow later.
\end{description}

\vspace{1 cm}
\begin{description}
\item[Name]   ROpenPrepPrepPPropSubst
\item[Task] To substitute a non-directional, non-locative OPENPREPPPROP for 
its variable as object of a prepobj. In analysis, the actsubcefs of the 
PREPPPROPVAR is set to the set of `other subcs' a prep may have.
\item[File] english:RC\_PropSubst1.mrule (mrules41.mrule)
\item[Semantics]
\item[Example] x1 regard x2 as x3 $\rightarrow$ x1 regard x2 as [x2 without 
principles]
\item[Remarks] Control rules to relate the embedded subject variable to 
another variable and delete it follow later.
\end{description}

\vspace{1 cm}
\begin{description}
\item[Name]  RClosedNPPropSubst
\item[Task] To substitute a CLOSEDNPPROP for its variable in complrel. 
\item[File] english:RC\_PropSubst2.mrule (mrules40.mrule)
\item[Semantics]
\item[Example] become x1 $\rightarrow$ become [she President]
\item[Remarks] It is unclear how the further treatment of CLOSEDNPPROPs (and 
other closed XPPROPs as well) that have a SENTENCE as subject should be (as 
e.g.\ in {\em That she is ill (seems) his main problem\/}). Currently, in 
the control rules only NP(VAR) and CNVAR are allowed as subject of embedded 
closed XPPROPS, since they are made into the object of the higher clause when 
the PROP node is deleted. In case the rules 
were extended and a sentence were allowed (which would then probably have to be 
made a complrel under the VERBP of the main clause, not an objrel), 
there still would be a 
problem with the ObjectOKrules (see documentation below). A new rule would have 
to be added to turn this complrel/SENTENCE into the subject of the higher 
clause (and, subsequently, 
into a leftdislocrel, also introducing a dummy subject THAT for agreement; cf.\ 
the documentation to RldislocSubjSentSubst below), but this rule would have to 
be restricted to complrel sentences coming from the subject position of an 
XPPROP, and leave ordinary argument sentences alone (to prevent 
malconstructions like {\em That he comes seems\/}). This problem has not been 
solved yet. For a more extensive discussion of this problem, see the remarks 
under RSentenceSubj in doc.\ 314 on the Dutch XPPROPtoCLAUSE subgrammmar.
\end{description}

\vspace{1 cm}
\begin{description}
\item[Name]  ROpenNPPropSubst
\item[Task] To substitute an OPENNPPROP for its variable in 
complrel. 
\item[File] english:RC\_PropSubst2.mrule (mrules40.mrule)
\item[Semantics]
\item[Example] x1 cost x3 x2 $\rightarrow$ x1 cost x3 [x1 a fortune]\\
x1 call x2 x3 $\rightarrow$ x1 call x2 [x2 a bastard]
\item[Remarks] Control rules to relate the embedded subject variable to 
another variable and delete it follow later.
\end{description}

\vspace{1 cm}
\begin{description}
\item[Name]   RClosedAdjPPropSubst
\item[Task] To substitute a CLOSEDADJPPROP for its variable in 
complrel. 
\item[File] english:RC\_PropSubst2.mrule (mrules40.mrule)
\item[Semantics]
\item[Example] x1 consider x2 $\rightarrow$ x1 consider [this foolish]
\item[Remarks] For remarks on the problems with a closed XPPROP which has a 
SENTENCE as subject, see under RClosedNPPropSubst above.
\end{description}

\vspace{1 cm}
\begin{description}
\item[Name] ROpenAdjPPropSubst
\item[Task] To substitute an OPENADJPPROP for its variable in 
complrel. 
\item[File] english:RC\_PropSubst2.mrule (mrules40.mrule)
\item[Semantics]
\item[Example] x1 paint x2 x3 $\rightarrow$ x1 paint x2 [x2 green]
\item[Remarks] Control rules to relate the embedded subject variable to 
another variable and delete it follow later.
\end{description}

\vspace{1 cm}
\begin{description}
\item[Name]   RClosedVerbPPropSubst
\item[Task] To substitute a CLOSEDVERBPPROP for its variable in 
complrel. 
\item[File] english:RC\_PropSubst2.mrule (mrules40.mrule)
\item[Semantics]
\item[Example] x1 have x2 $\rightarrow$ x1 have [a house built] (by a famous 
architect)
\item[Remarks] For remarks on the problems with a closed XPPROP which has a 
SENTENCE as subject, see under RClosedNPPropSubst above (although this probably 
will not occur for VERBPPROPs).
\end{description}

\vspace{1 cm}
\begin{description}
\item[Name] ROpenVerbPPropSubst
\item[Task] To substitute an OPENVERBPPROP for its variable in 
complrel. 
\item[File] english:RC\_PropSubst2.mrule (mrules40.mrule)
\item[Semantics]
\item[Example] no examples found
\item[Remarks] NO control rules have been written yet to relate the embedded 
subject variable to another variable and delete it.
\end{description}

\vspace{1 cm}
\begin{description}
\item[Name]   RPrepSENTSubst
\item[Task] To substitute a non-adverbial 
SENTENCE for its variable as object of a prepobj.
\item[File] english:RC\_PropSubst2.mrule (mrules40.mrule)
\item[Semantics]
\item[Example] \mbox{}\\
x1 count on x2 $\rightarrow$ x1 count on [the weather being fine]\\
x1 count on x2 $\rightarrow$ x1 count on [that you will be there]
\item[Remarks] Extraposition rules (introducing the dummy object {\em it\/}) 
follow later.
\end{description}

\vspace{1 cm}
\begin{description}
\item[Name] RbySENTSubst
\item[Task] To substitute a wh- or yes/no-SENTENCE for its variable as object 
of a by-phrase in a passive Verbpprop.
\item[File] english:RC\_PropSubst2.mrule (mrules40.mrule)
\item[Semantics]
\item[Example] puzzled x2 by x1 $\rightarrow$ puzzled x2 by [whether or not he
should go]
\item[Remarks]
\end{description}

\vspace{1 cm}
\begin{description}
\item[Name]   RComplSENTSubst
\item[Task] To substitute a  non-adverbial SENTENCE for its variable in 
complrel. 
\item[File] english:RC\_PropSubst2.mrule (mrules40.mrule)
\item[Semantics]
\item[Example] \mbox{}\\
x1 expect x2 $\rightarrow$ x1 expect [he to be your friend]\\
x1 know x2 $\rightarrow$ x1 know [(that) John will be late]\\
appear x1 $\rightarrow$ appear [as if he will win]\\
x1 promise x3 x2 $\rightarrow$ x1 promise x3 [x1 to come]
\item[Remarks] The rule contains a check on the presence of the conjunction 
{\em that\/} in case the complement sentence is a THATSENT: it may be absent 
only if the verb allows it, or if the embedded sentence is
a finite complement sentence that has a (shifted) wh-element. 
These subsentences do not contain any conjunction (see RIndicWhMoodSub in the 
next subgrammar): {\em Whom did you say ($^{*}$that) left?, What do you think 
($^{*}$that) she had bought?\/}.
A problem exists for verbs that have the value {\em nodel\/} for 
the attribute {\bf thatdel} when they must be combined with such a finite 
wh-complement: {\em $^{*}$Whom did they acknowledge (that) was defeated?\/}.
This problem has not been solved.
\end{description}

\vspace{1 cm}
\begin{description}
\item[Name] RLdislocSubjSENTSubst
\item[Task] To substitute a non-adverbial SENTENCE for its variable in 
subjrel. The substituted SENTENCE is put in a special `leftdislocrel', and a 
dummy subject {\em That\/} is inserted (for agreement; see below). This rule 
works for all input categories, not just for Verbpprops.
\item[File] english:RC\_PropSubst2.mrule (mrules40.mrule)
\item[Semantics]
\item[Example] \mbox{}\\
 x1 surprise x2 $\rightarrow$ [That he went away so early] THAT surprise x2\\
x1 unbelievable $\rightarrow$ [For a bridge to collapse like that] THAT 
unbelievable
\item[Remarks] For a discussion of an alternative treatment of subject 
sentences, by means of a more formal use of the pronoun THAT, see the remarks 
on the Dutch RSentenceSubj in doc.\ 314, p.\ 10.
\end{description}

\vspace{1 cm}
\begin{description}
\item[Name]   RExtrapSubjSENTSubst
\item[Task] To substitute a non-adverbial SENTENCE for its variable in 
subjrel. The substituted SENTENCE is put in complrel (and will be extraposed 
later, together with insertion of a dummy subject).
\item[File] english:RC\_PropSubst2.mrule (mrules40.mrule)
\item[Semantics]
\item[Example] x1 [surprise x2] $\rightarrow$ -- [surprise x2 (that) he never 
told you]
\item[Remarks]
\end{description}

\vspace{1 cm}
\begin{description}
\item[Name]   RModalComplSentSubst1
\item[Task] To substitute a non-adverbial SENTENCE for its variable in 
complrel, when the head verb is a one-place modal. The higher VERBP is pruned, 
and replaced by the embedded VERBP, and the SENTENCE node is deleted, so that 
the modal becomes an auxiliary. The voice of the new VERBPPROP is copied from 
the embedded sentence ({\em He can be hit\/} is thus considered a passive, in 
the same way as {\em He will be hit\/}). All 
this is needed because in surface structure, no difference exists between 
sentences like {\em Can you see him\/} and {\em Did you see him\/}. In deep 
structure, a difference is assumed for reasons of isomorphism, since some modal 
constructions may have to translate into full subsentences: {\em I can swim\/} 
has as one of its meanings {\em Het is mogelijk dat ik zwem\/}.
\item[File] english:RC\_PropSubst2.mrule (mrules40.mrule)
\item[Semantics]
\item[Example] can x1 ( + x2 see x3) $\rightarrow$ x2 can see x3
\item[Remarks] \mbox{}
\begin{itemize}
\item Part of the functions of TC\_PropOK as mentioned in doc.\ 153 has been 
incorporated in the present rule.
\item The distribution of adverbs over the two sentences still has to 
be solved: it is not at all clear whether the adverbs of the embedded sentence 
and the adverb variables of the main sentence
should all end up in the main clause, and in what order, nor whether there are 
any restrictions on the occurrence of adverbs in either the main or the 
embedded sentence. Presently, the only restriction that has been formulated is 
that the modal may not be accompanied by durational or retrospective temporal 
expressions.
\item The occurrence of a negation in the embedded clause still must be 
restricted somehow to modals having narrow negation scope like {\em may: He may 
not come\/} has as one of its meanings {\em It is possible that he does not 
come\/}. A modal like {\em can\/} does not have this option: {\em He can not 
come\/} does not mean {\em It is possible that he does not come\/}. Presently, 
the negation is always allowed. Perhaps, a condition on polarity should be 
added too: if the modal requires a negpol environment, like {\em need\/}, 
the negation should be present in the main clause. The current rule allows an 
embedded negation to move upwards, so that the polarity check finds nothing 
wrong with the sentence although it actually has not been negated.
\end{itemize}
\end{description}

\vspace{1 cm}
\begin{description}
\item[Name] RModalComplSentSubst2
\item[Task] To substitute a non-adverbial SENTENCE for its variable in 
complrel, when the head verb is a two-place modal. The higher VERBP is pruned, 
and replaced by the embedded VERBP, and the SENTENCE node is deleted, so that 
the modal becomes an auxiliary. The embedded subject is deleted. 
The voice of the new VERBPPROP is copied from 
the embedded sentence ({\em He can be hit\/} is thus considered a passive, in 
the same way as {\em He will be hit\/}). All 
this is needed because in surface structure, no difference exists between 
sentences like {\em Can you see him\/} and {\em Did you see him\/}. In deep 
structure, a difference is assumed for reasons of isomorphism, since some modal 
constructions may have to translate into full subsentences: {\em I can swim\/} 
has as one of its meanings {\em Ik ben in staat om te zwemmen\/}.
\item[File] english:RC\_PropSubst2.mrule (mrules40.mrule)
\item[Semantics]
\item[Example] x1 can x2 ( + x1 see x3 ) $\rightarrow$ x1 can see x3
\item[Remarks] \mbox{}
\begin{itemize}
\item Part of the functions of TC\_PropOK as mentioned in doc.\ 153 has 
been incorporated in the present rule.
\item The distribution of adverbs over the two sentences still has to 
be solved. See the comment in the previous rule.
\item For 2-place modals, no negation is allowed in the embedded sentence: 
{\em He can not come\/} does not mean {\em He is able not to come\/}, nor does 
{\em He may not come\/} mean {\em He is allowed not to come\/}.
\end{itemize}

\end{description}

\vspace{1 cm}
\begin{description}
\item[Name]   RObjNPSentSubst
\item[Task] To substitute a closed sentence-like NP for its variable in 
(ind)objrel. This rule is ordered among the proposition substitution rules so 
that the NP can be translated into real sentences in other languages. Open 
structures are considered NP-like SENTENCEs, and are dealt with in the ordinary 
complrel SENTENCE substitution.
\item[File] english:RC\_PropSubst2.mrule (mrules40.mrule)
\item[Semantics]
\item[Example] x1 forgive x3 x2 $\rightarrow$ x1 forgive x3 [his being late]
\item[Remarks]
\end{description}

\vspace{1 cm}
\begin{description}
\item[Name] RSubjNPSentSubst
\item[Task] To substitute a sentence-like NP (open or closed) for its variable 
in subj\-rel. This rule is ordered among the proposition substitution rules so 
that the NP can be translated into real sentences in other languages. The rule 
works for all XPPROPs allowed here, not just for Verbpprops.
\item[File] english:RC\_PropSubst2.mrule (mrules40.mrule)
\item[Semantics]
\item[Example] x1 wrong $\rightarrow$ [Eating people] wrong
\item[Remarks] Although the example sentence given here can be dealt with by 
the present rule, it will be stopped in the control rules: there is no
controller for the OpenIng, and no rules have been written yet to deal with 
such cases. SentNPs (like {\em Their making such a mess\/}) can also be dealt 
with here, and do not need control, so they will not be stopped. 
\end{description}

\vspace{1 cm}
\begin{description}
\item[Name]   RPrepObjNPSentSubst 
\item[Task] To substitute a sentence-like NP (closed or open) for its variable 
as object of a prepobj. The rule 
works for all XPPROPs allowed here, not just for Verbpprops.
\item[File] english:RC\_PropSubst2.mrule (mrules40.mrule)
\item[Semantics]
\item[Example] \mbox{}\\
x1 talk x2 out of x3 $\rightarrow$ x1 talk x2 out of [jumping from the bridge]
\\
x1 count on x2 $\rightarrow$ x1 count on [your being there in time]
\item[Remarks]
\end{description}

\vspace{1 cm}
\begin{description}
\item[Name]   RConjSentSubst
\item[Task] To substitute an adverbial non-temporal SENTENCE starting with a 
conjunction for its variable in 
leftdislocrel, and to introduce a separating comma between the two sentences.
The rule 
works for all XPPROPs allowed here, not just for Verbpprops.
\item[File] english:RC\_PropSubst2.mrule (mrules40.mrule)
\item[Semantics]
\item[Example] x4 x1 see x2 $\rightarrow$ [Although she ignored him], x1 
see x2 
\item[Remarks] \mbox{}
\begin{itemize}
\item Open wh-infinitives cannot be dealt with yet. 
\item This rule has already been provided with a condition on substitution 
order: it is applicable only before all other propsubst rules in analysis 
(no variables for propositions may occur to the right of the conjsent).
\item Control rules to relate the embedded subject variable of open sentences 
to another variable and delete it follow later, in TC\_ConjSentControl.
\item Because the surface order of adverbial sentences may have influence on 
the interpretation of pronominal expressions ({\em The doctor$_{i}$ came out of 
my sister's room. My father$_{j}$ cried 
because he$_{i,j}$ had raped her - The doctor$_{i}$ came out of 
my sister's room. Because he$_{i}$ had raped her, my 
father$_{j}$ cried\/}), the strategy in IL is to map substitution rules for the
same position onto each other. The Dutch `middle' conjsent is mapped onto an 
English initial one, because English does not have this position. However, 
since there may already be something there in leftdislocrel, and since only one 
leftdislocrel is allowed, it is also mapped onto a final conjsent, with a 
lowered bonus. The rules that introduce the variables for conjsents have no 
restriction on position: they all map onto each other (see the previous 
subgrammar, VERBPPROPformation).
\item The rule also has a condition on the attribute {\bf distri} of the 
conjunction: it must contain the value {\em initial\/}. For the moment, it is 
assumed that all CONJs can be used to introduce both an initial conjsent and a 
final one (they all have the value {\em [initial, final]\/}), but if examples 
are found of CONJs with restricted use, they can be dealt with here. However, 
there is NO mapping of initial conjsents on final ones in transfer yet, so the
effect of the restriction is that no translation whatsoever will be produced in 
case there was only a conjunction with the `wrong' distribution as translation.
\end{itemize}
\end{description}

\vspace{1 cm}
\begin{description}
\item[Name] RConjPrepNPSubst
\item[Task] To substitute an adverbial non-temporal\footnote
{The rule excludes temporal sentential PREPPs by checking the attribute {\bf 
actsubcefs} of the PREPP for the presence of the value {\em temp\/}. PREPPs do 
not have an attribute {\bf temporal}.} sentential PREPP (i.e.\ an adverbial
sentence-like NP preceded by a prep) for its variable in 
leftdislocrel, and to introduce a separating comma between the two sentences.
The rule works for all XPPROPs, not just for Verbpprops.
\item[File] english:RC\_PropSubst2.mrule (mrules40.mrule)
\item[Semantics]
\item[Example] x4 x1 see x2 $\rightarrow$ [Without realising it], x1 see x2
\item[Remarks] \mbox{}
\begin{itemize}
\item This rule has already been provided with a condition on substitution 
order: it is applicable only before all other propsubst rules in analysis 
(no variables for propositions may occur to the right of the conjprepnp).
\item Control rules to relate the embedded subject variable of an OpenIng NP 
to another variable and delete it follow later, in TC\_ConjSentControl.
\item For conditions on the mapping of different conjsent substitution rules 
with respect to position of the conjsent see the previous rule.
\end{itemize}
\end{description}

\vspace{1 cm}
\begin{description}
\item[Name] RFinalConjSentSubst
\item[Task] To substitute an adverbial non-temporal SENTENCE starting with a 
conjunction for its variable in 
postsentadvrel. In analysis, a separating comma between the two sentences is 
accepted.
\item[File] english:RC\_PropSubst2.mrule (mrules40.mrule)
\item[Semantics]
\item[Example]  x1 see x2 x4 $\rightarrow$ x1 see x2 [although she ignored him]
\item[Remarks] \mbox{}
\begin{itemize}
\item The rule does not need explicit conditions on substitution order: it 
should be applied last in analysis (since it is the last element in the 
sentence, there cannot be any variables for propositions on 
the right of the conjsent anyway). 
\item Control rules to relate the embedded subject variable of open sentences 
to another variable and delete it follow later, in TC\_ConjSentControl.
\item For conditions on the mapping of different conjsent substitution rules 
with respect to position of the conjsent see RConjSentSubst.
\item The rule has a check on the value {\em final\/} for the CONJ-attribute 
{\bf distri}. However, 
there is NO mapping of initial conjsents on final ones in transfer yet, so the
effect of the restriction is that no translation whatsoever will be produced in 
case there was only a conjunction with the `wrong' distribution as translation.
\end{itemize}
\end{description}

\vspace{1 cm}
\begin{description}
\item[Name] RFinalConjPrepNPSubst
\item[Task] To substitute an adverbial non-temporal sentential PREPP 
(i.e.\ an adverbial
sentence-like NP preceded by a prep) for its variable in 
postsentadvrel. In analysis, a separating comma between the two sentences is 
accepted. The rule works for all XPPROPs, not just for Verbpprops.
\item[File] english:RC\_PropSubst2.mrule (mrules40.mrule)
\item[Semantics]
\item[Example] x1 see x2 x4 $\rightarrow$ x1 see x2 [without his realising it]
\item[Remarks] \mbox{}
\begin{itemize}
\item The rule does not need explicit conditions on substitution order: it 
should be applied last in analysis (since it is the last element in the 
sentence, there cannot be any variables for propositions on 
the right of the conjsent anyway). 
\item Control rules to relate the embedded subject variable of an OpenIng NP 
to another variable and delete it follow later, in TC\_ConjSentControl.
\item For conditions on the mapping of different conjsent substitution rules 
with respect to position of the conjsent see RConjSentSubst.
\end{itemize}
\end{description}

\vspace{1 cm}
\begin{description}
\item[Name]   RAdjppComplSentSubst
\item[Task] To substitute a (non-adverbial) SENTENCE for its variable under the 
predrel ADJP of an ADJPPROP.
\item[File] english:RC\_PropSubst2.mrule (mrules40.mrule)
\item[Semantics]
\item[Example] x1 able x2 $\rightarrow$ x1 able [to swim]
\item[Remarks] The set of relations now allowed for the SENTENCE in the ADJP 
may be too large. This will have to be checked when the 
documentation for the ADJPPROPformation subgrammar for English is available.\\
Extraposition of complrel sentences from the ADJP is accounted 
for in TPredExtrapos1 (in TC\_SentOKrules). 
\end{description}

\vspace{1 cm}
\begin{description}
\item[Name] RxppExtrapSubjSentSubst
\item[Task] To substitute a non-adverbial SENTENCE for its variable in 
subjrel of any XPPROP that is not a Verbpprop. The subject sentence is put in 
extraposrel, but no dummy subject {\em It\/} is inserted here (that is done in 
TC\_ObjectOK, as for all other extraposed sentences).
The rule is an alternative to the VERBPPROP rule RldislocSubjSENTsubst, but 
more kinds of sentences (esp.\ (for)toinfs and OpenIngs) are allowed to 
extrapose here.
\item[File] english:RC\_PropSubst2.mrule (mrules40.mrule)
\item[Semantics]
\item[Example] x1 impossible $\rightarrow$ impossible [that you have not heard 
this before]
\item[Remarks]
\end{description}

\vspace{1 cm}
\begin{description}
\item[Name] RTempAdvSentSubstitution
\item[Task] To substitute a temporal SENTENCE for its variable in 
tempadvrel of any XPPROP. 
\item[File] english:RC\_TempVar.mrule (mrules36.mrule)
\item[Semantics]
\item[Example] x1 dance x4 $\rightarrow$ x1 dance [when he is in love]
\item[Remarks] Several extensions to this rule or new rules are still needed: \\
1) This rule works for all XPPROPs that have the tempadvrel outside the 
predicate, which means that it does not work for ADJPPROPs yet\\
2) it uses the ordinary substorder condition, instead of a condition for 
subst\-order of propositions. This must be changed.\\
3) the model refers explicitly to a predicate, hence the rule does not work 
for existential and identificational NPPROPs yet (they have an objrel and an 
idrel, respectively)\\
4) it only works for temporal SENTENCEs, not for temporal PREPPs like {\em 
Before going home,...\/}\\
Also, the rule assumes that all temporal sentences are in the same position as 
ordinary tempadvs. However, they might also occur in leftdislocrel, and perhaps 
the final position of temporal sentences is postsentadvrel rather than 
tempadvrel. This still has to be decided, and the relevant transformations to 
move the sentence have to be added.
\end{description}

\item[Remark]
\end{description}

\newpage
\subsection{RC\_ClauseFormation}

\begin{description}
\item[Kind] Obligatory Rule Class
\item[Task] To generate a new top node CLAUSE for any XPPROP.
The new top node has attributes to allow for different aspect and 
time settings, finiteness, modus, etc.

For clauses formed out of ADJPPROPs, an obligatory transformation class 
TC\_ComparIncorp has been added, to account for inflectional comparatives and 
superlatives (see documentation on that transformation class below).

No rule has been written yet to form a CLAUSE out of an ADVPPROP, since no 
relevant examples could be found. However, this rule can be added easily if 
necessary.

In doc.\ 153, the term `clause formation' was used as an abbreviation for a 
number of rules and transformations, including those that would deal with the 
introduction of time and aspect. Presently, these phenomena 
are covered in separate rule and transformation classes, since the conditions 
are much too complex 
to be incorporated elsewhere, and since a specific order of application of the 
rules must be guaranteed. 

Originally, the clause formation rules were also intended to account for the 
difference between Spanish {\em ser\/} and {\em estar\/}. Now that this 
distinction is made elsewhere, there is no reason to keep calling the current 
rules a rule class instead of a transformation class. For practical reasons, 
however, the status has not been changed.

\vspace{1 cm}
\begin{description}
\item[Name] RActClauseFormation
\item[Task] To form a CLAUSE out of an active VERBPPROP. This VERBPPROP may 
also be passive in case of modals that combined with a passive sentence
({\em He can be hit\/} is treated the same as {\em He will be hit\/}, i.e.\ as 
a passive, but it does not need an extra auxiliary of the passive because that 
was already present in the embedded sentence).
\item[File] english:RC\_ClauseFormation.mrule (mrules39.mrule)
\item[Semantics]
\item[Example] \mbox{}\\
x1 consider he a fool $\rightarrow$ x1 consider he a fool\\
x1 must be published $\rightarrow$ x1 must be published
\item[Remarks]
\end{description}

\vspace{1 cm}
\begin{description}
\item[Name] RPasClauseFormation
\item[Task] To form a CLAUSE out of a passive VERBPPROP that does not contain a 
modal and to introduce the auxiliary of the passive {\em be\/}. (For the 
treatment of modals see above:
{\em He can be hit\/} is treated the same as {\em He will be hit\/}, i.e.\ as 
a passive, but it does not need an extra auxiliary of the passive because that 
was already present in the embedded sentence).
\item[File] english:RC\_ClauseFormation.mrule (mrules39.mrule)
\item[Semantics]
\item[Example] \mbox{}\\
considered he a fool by x1 $\rightarrow$ be considered he a fool by x1\\
built x2 by x1 $\rightarrow$ be built x2 by x1
\item[Remarks]
\end{description}

\vspace{1 cm}
\begin{description}
\item[Name]   RAdjpClauseFormation
\item[Task] To form a CLAUSE out of an ADJPPROP, introducing the copula 
{\em be\/}. The CLAUSE receives the default voice, which is {\em active\/}.
\item[File] english:RC\_ClauseFormation.mrule (mrules39.mrule)
\item[Semantics]
\item[Example] \mbox{}\\
x1 consider he a fool $\rightarrow$ x1 consider he a fool\\
x1 must be published $\rightarrow$ x1 must be published
\item[Remarks] Following this rule, an obligatory transformation class has been 
added to account for inflectional comparative and superlative adjectives (see 
below under TC\_ComparIncorp).
\end{description}

\vspace{1 cm}
\begin{description}
\item[Name]   RNpClauseFormation
\item[Task] To form a CLAUSE out of an NPPROP, introducing the copula 
{\em be\/}. 
The CLAUSE receives the default voice, which is {\em active\/}. This rule 
is only for `real', predicative NPPROPs, not for identificationals or 
existentials. The CLAUSE node receives the NPPROP setting for superdeixis.
\item[File] english:RC\_ClauseFormation.mrule (mrules39.mrule)
\item[Semantics]
\item[Example] x1 a fool $\rightarrow$ x1 be a fool
\item[Remarks]
\end{description}


\vspace{1 cm}
\begin{description}
\item[Name] RPreppClauseFormation
\item[Task] To form a CLAUSE out of a PREPPPROP, introducing the copula 
{\em be\/}. The CLAUSE receives the default voice, which is {\em active\/}. 
The rule 
is only for `real' PREPPPROPs, not for locatives or directionals (these should 
never become a full sentence by means of a copula, but take a main verb 
{\em be\/}: {\em He is in the garden\/} may translate into 
{\em Hij zit in de tuin\/})\footnote{This had been adapted somewhat by the 
time this document was approved: in generation, locative and directional preps
are allowed, to be able to make full relative clauses as translation of a 
simple locative modifiers: {\em Het meisje in de tuin\/} now also translates 
into {\em The girl who is in the garden\/}. }.
\item[File] english:RC\_ClauseFormation.mrule (mrules39.mrule)
\item[Semantics]
\item[Example] x1 against x2 $\rightarrow$ x1 be against x2 (He was against the 
proposal)
\item[Remarks]
\end{description}


\vspace{1 cm}
\begin{description}
\item[Name]   RExistNPClauseFormation
\item[Task] To form a CLAUSE out of an existential NPPROP, introducing the 
copula {\em be\/}. The CLAUSE receives the default voice, which is 
{\em active\/}, and the default superdeixis, which is {\em omegadeixis\/}.
\item[File] english:RC\_ClauseFormation.mrule (mrules39.mrule)
\item[Semantics]
\item[Example] x1 $\rightarrow$ be x1 (There was a doctor)
\item[Remarks] Note that there is no predrel NPVAR under the newly created 
VERBP node for identificationals, just an objrel NPVAR.
\end{description}


\vspace{1 cm}
\begin{description}
\item[Name] RIdentNPClauseFormation
\item[Task] To form a CLAUSE out of an identificational NPPROP, introducing the 
copula {\em be\/}. The CLAUSE receives the default voice, which is 
{\em active\/}, and the default superdeixis, which is {\em omegadeixis\/}.
\item[File] english:RC\_ClauseFormation.mrule (mrules39.mrule)
\item[Semantics]
\item[Example] this x1 $\rightarrow$ this be x1
\item[Remarks] In the rule, the idrel NPVAR of the NPPROP is changed to a 
predrel NPVAR under the VERBP.
\end{description}

\end{description}

\newpage
\subsection{TC\_ComparIncorp}

\begin{description}
\item[Kind] Obligatory Transformation Class, only called after 
RAdjpClauseFormation
\item[Task] To check whether there are any degree modifiers {\em more\/} 
and {\em most\/}, and to incorporate them 
into the adjective, if the adjective allows an inflectional comparative/
superlative.

This transformation class is only called after the Adjp Clause Formation rule.
The class is needed because the incorporation is otherwise done in the 
ADJPPROPtoADJPFORMULA subgrammar. If it is decided to 
move the transformation class from there to ADJPPROPformation 
(e.g.\ straight after the 
degree modification, where modifiers as {\em more\/} and {\em most\/} are 
introduced even for adjectives allowing inflectional comparatives), the class 
can be removed here.

Although the transformation class has always been planned in 
ADJPPROPtoADJPFORMULA (see doc.\ 109, {\em The organisation of the 
ADJPPROP-sub\-gram\-mar\/}, by Franciska de Jong), a parallel version in the 
clause subgrammar was 
not included in the general lay-out as given in doc.\ 150.

\vspace{1 cm}
\begin{description}
\item[Name] TComparIncorp
\item[Task] To replace the degree modifier {\em more\/} by an inflectional 
comparative of the adjective.
\item[File] english:TC\_ComparIncorp.mrule (mrules95.mrule)
\item[Semantics] --
\item[Example] x1 be more long $\rightarrow$ x1 be longer
\item[Remarks] 
\end{description}

\vspace{1 cm}
\begin{description}
\item[Name] TSuperIncorp
\item[Task] To replace the degree modifier {\em the most\/} by an inflectional 
superlative of the adjective and a definite article.
\item[File] english:TC\_ComparIncorp.mrule (mrules95.mrule)
\item[Semantics] --
\item[Example] x1 be the most long $\rightarrow$ x1 be the longest
\item[Remarks] 
\end{description}


\vspace{1 cm}
\begin{description}
\item[Name] TNoIncorp
\item[Task] Vacuous rule, to let adjectives which do not allow for inflectional 
comparatives or superlatives, and structures without {\em more\/} or {\em 
most\/}, pass this transformation class.
\item[File] english:TC\_ComparIncorp.mrule (mrules95.mrule)
\item[Semantics] --
\item[Example] x1 be long $\rightarrow$ x1 be long; x1 be more beautiful 
$\rightarrow$ x1 be more beautiful
\item[Remarks] 
\end{description}

\end{description}


\newpage
\subsection{RC\_Aspect}

\begin{description}
\item[Kind] Obligatory Rule Class
\item[Task] This rule class spells out the aspect of the clause, 
i.e.\  the aspect relation between the interval E and a reference interval R.

In doc.\ 153, it was assumed that the realisation of aspect would be done in 
the Clause Formation Rules, but it was decided that Aspect is a separate 
phenomenon, deserving its own (meaningful and translational relevant) rule 
class. See also docs.\ 53 and 263.

\vspace{1 cm}
\begin{description}
\item[Name] RaspectImperfective1
\item[Task] To spell out an imperfective aspect relation in case of non-stative 
Aktionsart for verbs that allow a 
progressive and that are not modals or the auxiliaries {\em have}, {\em do}, 
or the progressive {\em be}, 
and to introduce the progressive auxiliary {\em be\/}, giving
the original verb an -ingform.
\item[File] english:RC\_Aspect.mrule (mrules38.mrule)
\item[Semantics] Imperfective aspect relation between interval E and reference 
interval R.
\item[Example] \mbox{}\\
x1 work $\rightarrow$ x1 be working\\
x1 be hit $\rightarrow$ x1 be being hit

*be `canning',
*be `willing',
*be being working,
*be having worked,
*be doing worked
\item[Remarks] 
This rule verifies that a durational adverbial is absent. Presence of a 
durational adverbial makes the aspect of an event perfective.\\
In Dutch, achievements are excluded, but it is not clear if that was a right 
decision, so they are allowed here for the time being.\\
It should be noted that the progressive in English also has another meaning,
i.e.\ {\em aan het ...zijn} which will be introduced by other rules in the 
future. I will refer to this meaning as `stative progressive'.
\end{description}

\vspace{1 cm}
\begin{description}
\item[Name]   RaspectImperfective2
\item[Task] To spell out an imperfective aspect relation for verbs not allowing 
a progressive or for stative CLAUSEs. No progressive auxiliary {\em be\/} is 
introduced; the only thing that is done is setting the CLAUSE attribute {\bf
aspect} at {\em imperfective\/}.
\item[File] english:RC\_Aspect.mrule (mrules38.mrule)
\item[Semantics] Imperfective aspect relation between interval E and reference 
interval R.
\item[Example] \mbox{}\\
x1 can [see x2] $\rightarrow$ x1 can [see x2] (no ing)\\
x1 be ill $\rightarrow$ x1 be ill (stative)\\
will [x1 come] $\rightarrow$ will [x1 come] (no ing)\\
x1 be working $\rightarrow$ x1 be working (stative progressive)\\
x1 have a book $\rightarrow$ x1 have a book (no ing, stative)\\
x1 do work $\rightarrow$ x1 do work (auxiliary do, no ing)\\
x1 have worked $\rightarrow$ x1 have worked (no ing)
\item[Remarks] 
This rule verifies that a durational adverbial is absent. Presence of a 
durational adverbial makes the aspect of an event perfective.\\
In Dutch, achievements are excluded, but it is not clear if that was a right 
decision, so they are allowed here for the time being.\\
Note that this rule will apply to the `stative progressive' that will be added 
in the future.
\end{description}

\vspace{1 cm}
\begin{description}
\item[Name] RaspectPerfective
\item[Task] To spell out a perfective aspect relation by setting the CLAUSE 
attribute {\bf aspect} at {\em perfective\/}.
\item[File] english:RC\_Aspect.mrule (mrules38.mrule)
\item[Semantics] Perfective aspect relation between interval E and reference 
interval R.
\item[Example] x1 work $\rightarrow$ x1 work
\item[Remarks] 
\end{description}

\vspace{1 cm}
\begin{description}
\item[Name] Raspecthabitual
\item[Task] To spell out an habitual aspect relation for clauses containing 
a variable for a frequential time phrase, by setting the CLAUSE attribute {\bf
aspect} at {\em habitual\/}.
\item[File] english:RC\_Aspect.mrule (mrules38.mrule)
\item[Semantics] Habitual aspect relation between interval E and reference 
interval R.
\item[Example] x1 work x4(=every day) $\rightarrow$ x1 work x4(= every day)
\item[Remarks] This rule demands the presence of a frequential 
adverbial.\\
The rule has not been added to the control expression and transfer yet, so it 
has not been tested yet. 
\end{description}

\item[Remark]

\item [Filters] No associated filters.
\item [Speed rules] No associated speed rules.
\item [Not in Control Expression] Raspecthabitual
\end{description}

\newpage
\subsection{RC\_Retro}

\begin{description}
\item[Kind] Optional Rule Class
\item[Task] This rule class spells out the retrospective aspect of the clause, 
i.e.\  the `until' relation between a reference interval Re and the reference 
interval Rs.

In doc.\ 153, it was assumed that the realisation of aspect would be done in 
the Clause Formation Rules, but it was decided that Aspect is a separate 
phenomenon, deserving its own (meaningful and translational relevant) rule 
class. Retrospective aspect is separated from the other Aspect Rules because 
that again is a separate phenomenon, with its own semantics (it is not about 
the relationship between E and R, but between Re and Rs).
See also docs.\ 53 and 263.
\vspace{1 cm}
\begin{description}
\item[Name] Rretro
\item[Task] To spell out the retrospective aspect of the clause, if there is 
any retrospective time indication, i.e.\ if there is a variable for it, and if 
the first verb is not one of the modals {\em can, may\/} or {\em must\/}.
The auxiliary verb {
\em have\/} is introduced, and the following verb is made a past participle
form. The 
CLAUSE attribute {\bf retro} is set to {\em true\/}.
\item[File] english:RC\_Retro.mrule (mrules76.mrule)
\item[Semantics] Between the reference interval Re and the reference interval 
Rs holds a relation `until': RETROSPECTIVE.
\item[Example] x1 work here x4(=since 1980) $\rightarrow$ x1 have worked here
x4(=since  1980)\\
x1 be working here x4(=for 15 years) $\rightarrow$ x1 have been working here
x4(=for 15 years)
\item[Remarks] Modal verbs are excluded in this rule.
\end{description}

\vspace{1 cm}
\begin{description}
\item[Name] Rretrospec 
\item[Task] To spell out the retrospective aspect of the clause, if there is 
any retrospective time indication, i.e.\ if there is a variable for it,
and if the first verb is one of the modals {\em can, may\/} or {\em must\/}.
The auxiliary verb {\em have\/} is introduced, 
but the modal verb is not made a participle. The 
CLAUSE attribute {\bf retro} is set to {\em true\/}.


\item[File] english:RC\_Retro.mrule (mrules76.mrule)
\item[Semantics]Between the reference interval Re and the reference interval 
Rs holds a relation `until': RETROSPECTIVE.

\item[Example] x1 may work x4(=since 1980) $\rightarrow$ x1 have may work
 x4(=since 1980)

\item[Remarks] The rule is needed because in English, 
these modals do not form a participle;
when they must be combined with {\em have\/}, they reverse places: {\em He has 
`can' worked\/} $\rightarrow$ {\em He could have worked\/}. 
In the current rule, only 
{\em have\/} is introduced. In a later transformation, Thavemodaladaptation 
(following a past (super)deixis rule), the  
correct surface order is determined. However, note that this transformation
has NOT been added to the control expression yet!!.\\
For the treatment of modal verbs in Rosetta3, see doc.\ 327.
\end{description}
\item [Filters] No associated filters.
\item [Speed rules] No associated speed rules.
\item [Not in Control Expression] Rretrospec has not been added to the control 
exression and transfer yet.
\end{description}

\newpage
\subsection{TC\_Finiteness}

\begin{description}
\item[Kind] Obligatory Transformation Class
\item[Task] To mark whether is CLAUSE is finite or infinite. The attribute {\bf
finiteness} of the CLAUSE is set at the appropriate value.

Note that finiteness is not a Rule Class: finite sentences may have to 
translate into infinitives or {\em vice versa\/}. Note also that in almost all 
cases a finite and the infinite transformation are applicable in generation!

In doc.\ 153, it was assumed that the realisation of finiteness would be done 
in the Clause Formation Rules. However, the interaction between the 
introduction of other auxiliaries and the auxiliary {\em do\/} warrant a 
separation of all the different phenomena. 
See also docs.\ 53 and 263.

\vspace{1 cm}
\begin{description}
\item[Name] Tfinite1
\item[Task] To mark a CLAUSE as finite without introduction of the auxiliary 
{\em do\/}. This is needed in case there already is an auxiliary, or a modal
(for the copula {\em be\/}, {\em do\/} is introduced (see below), since 
{\em be\/} is considered a main verb there).
\item[File] english:TC\_Finiteness.mrule (mrules37.mrule)
\item[Semantics] --
\item[Example] \mbox{}\\
 x1 may work $\rightarrow$ x1 may work\\
x1 be working $\rightarrow$ x1 be working
\item[Remarks] 
\end{description}

\vspace{1 cm}
\begin{description}
\item[Name] Tfinite2
\item[Task] To mark a CLAUSE as finite and introduce the auxiliary 
{\em do\/}. The rule is applied when there is no auxiliary or modal.
\item[File] english:TC\_Finiteness.mrule (mrules37.mrule)
\item[Semantics] --
\item[Example] \mbox{}\\
 x1 work $\rightarrow$ x1 do work\\
x1 be a fool $\rightarrow$ x1 do be a fool
\item[Remarks] In case the main verb is {\em do, be\/} or {\em have\/}, the 
auxiliary {\em do\/} must be deleted again. This will be done at the end of the 
next subgrammar in TC\_DoBeDeletion and TC\_DoDeletion. By introducing the 
auxiliary here, there always will be a non-head verb and a head verb in every 
VERBP, which makes the subsequent rules easier to formulate.
\end{description}

\vspace{1 cm}
\begin{description}
\item[Name] Tinfinite
\item[Task] To mark any CLAUSE as infinite (not for modals).
\item[File] english:TC\_Finiteness.mrule (mrules37.mrule)
\item[Semantics] --
\item[Example] \mbox{}\\
 x1 work $\rightarrow$ x1 work\\
x1 be working $\rightarrow$ x1 be working
\item[Remarks] 
\end{description}

\item[Remark] This transformation class is ordered crucially before RC\_Deixis, 
because the (super)deixis rules refer to the finiteness attribute.
\item [Filters] No associated filters.
\item [Speed Rules] no associated speed rules.
\end{description}

\newpage
\subsection{RC\_Deixis}

\begin{description}
\item[Kind] Obligatory Rule Class
\item[Task] To set 
the attributes {\bf deixis} and {\bf superdeixis} of the CLAUSE at the 
appropriate values, and
to spell out the tense of the first verb in the VERBP for finite clauses. 

The rule class was not mentioned in doc.\ 150; there, it was assumed that tense 
could be dealt with in the Clause Formation Rules. However, as was said before 
of Aspect Rules (relation E - R) and Retro Rules (relation Re - Rs), the 
deixis rules also have their own semantics: to specify the relation between Rs 
and S. For more information on this subject,
see docs.\ 263, {\em Documentation of the rules 
for the translation of temporal expressions in Rosetta3, part I\/}, and 320, 
{\em Superdeixis in Rosetta3\/}, both by Lisette Appelo.

The rule class contains parallel rules for deixis and for superdeixis of the 
CLAUSE, since it can be dependent  
of a higher clause for its time (superdeixis) or not (deixis).
It is ordered crucially after the finiteness transformations, since for 
infinite sentences, the VERB tense need not be set.

First the deixis rules will be treated (A), then the superdeixis rules (B).

\vspace{1 cm}
\subsubsection{A. Deixis Rules}
\begin{description}
\item[Name] RfinPresentDeixis
\item[Task] The rule is for an independent finite CLAUSE with its own (present 
or
neutral) referential time expression. The {\bf deixis} attribute of this clause
is marked as {\em presentdeixis\/}, and the tense of the leftmost verb is set
at {\em presenttense\/} if allowed by that verb. 
For clauses with a modal, the {\bf superdeixis}
attribute of the clause (which had been copied from the embedded sentence) is 
set to {\em omegadeixis\/} again. 
\item[File] english:RC\_Deixis.mrule (mrules35.mrule)
\item[Semantics] To establish that the relation between Rs and S is 
simultaneous: PRESENT. 
\item[Example] \mbox{}\\
 x1 do$_{omegatense}$ work $\rightarrow$ x1 do$_{presenttense}$ work \\
 x1 be$_{omegatense}$ working $\rightarrow$ x1 be$_{presenttense}$ working
\item[Remarks] 
The superdeixis of the clause may be presentdeixis in 
case it was copied from an embedded clause in the proposition substitution 
rules as e.g. with the modal verbs. \\
Conditions on perfective aspect and presence of a presentdeixis adverbial avoid
perfective readings of:
{\em he swims at 3 o'clock} and {\em he swims for four hours}
(only  habitual interpretations are allowed!)\\
Note that the neutral referential time expression is only allowed here  in case 
of imperfective or habitual aspect and that it has a future interpretation in 
the former case. The only perfective cases allowed are: {\em He swims} and 
{\em He has swum}.\\
In case of a {\em temporally independent} but {\em syntactically dependent} 
clause only a presentdeixis referential time expression will be allowed. This 
condition is taken care of in the superdeixis adaptation transformations. With 
a neutral referential time expression such a clause will always be interpreted 
as {\em temporally dependent} and will be treated in {\em 
RfinPresentSuperdeixis}.  
\end{description}

\vspace{1 cm}
\begin{description}
\item[Name] RfinPresFutDeixis
\item[Task] The rule is for an independent finite CLAUSE that has a
(present, future or omega) referential time expression. 
The {\bf deixis} attribute of the clause is set to {\em presentdeixis\/}. 
The auxiliary of the future {\em will\/} is introduced and given {\em 
presenttense\/}, the following verb is given {\em infinitive\/} modus, 
and if the auxiliary {\em do\/} was present, it is deleted. In case of {\em 
will}, {\em will} is not introduced.
If the {
\bf superdeixis} attribute of the clause is not {\em omegadeixis\/} (because 
it copied the superdeixis of an embedded  clause, in the proposition 
substitution rules), it is set at omega here.
\item[File] english:RC\_Future.mrule (mrules74.mrule)
\item[Semantics] The relation between Rs and S is simultaneous: PRESENT, but 
with S `shifted' into the future
\item[Example] \mbox{}\\
x1 do work x4(=tomorrow) $\rightarrow$ x1 will work x4(=tomorrow)\\
x1 will work x4(=at three o'clock) $\rightarrow$ x1 will work x4(=at 
three o'clock)\\
x1 do work x4(=for four hours) $\rightarrow$ x1 will work x4(=for four hours)\\
\item[Remarks] 
The superdeixis of the clause may already be presentdeixis in 
case it was copied from an embedded  clause in the proposition substitution 
rules.\\
The rule is split into three subrules, 1) no auxiliary, 2) {\em do} present and 
3) {\em will} present. Other auxiliaries are excluded.\\
In case of a presentdeixis adverbial, there are conditions on the perfective 
aspect and presence of a durational adverbial to avoid {\em she will work (now)} 
and {\em she will be working (now)} (only modal interpretation) as translation 
of {\em zij werkt (nu)}. This combination should receive lower bonus since 
the habitual reading of such sentences is more plausible!\\
Note that there is another interpretation of {\em will}, corresponding to Dutch 
{\em zullen}, which is treated as a modal verb. See doc.\ 327 for the 
treatment of modal verbs in Rosetta3.
\end{description}

\vspace{1 cm}
\begin{description}
\item[Name] RfinPastDeixis
\item[Task] The rule is for an independent finite CLAUSE with a (past
or
neutral) referential time expression. The {\bf deixis} attribute of this clause
is marked as {\em pastdeixis\/}, and the tense of the leftmost verb is set
at {\em pasttense\/} if allowed by that verb. 
For clauses with a modal, the (past) {\bf superdeixis}
attribute of the clause (which had been copied from the embedded sentence) is 
set to {\em omegadeixis\/} again. 
\item[File] english:RC\_Deixis.mrule (mrules35.mrule)
\item[Semantics] To establish that the relation between Rs and S is 
before: PAST
\item[Example] \mbox{}\\
 x1 do$_{omegatense}$ work $\rightarrow$ x1 do$_{pasttense}$ work\\
 x1 be$_{omegatense}$ working $\rightarrow$ x1 be$_{pasttense}$ working
\item[Remarks] 
The superdeixis of the clause may be pastdeixis in 
case it was copied from an embedded clause in the proposition substitution 
rules as e.g. with the modal verbs. \\
Note that the neutral referential time expression is only allowed here because 
Rosetta3 translates {\em isolated} sentences, e.g. {\em He worked} or {\em He 
came at 3 o'clock}.
In case of a {\em temporally independent} but {\em syntactically dependent} 
clause only a pastdeixis referential time expression will be allowed. This 
condition is taken care of in the superdeixis adaptation transformations. With 
a neutral referential time expression such a clause will always be interpreted 
as {\em temporally dependent} and will be treated in {\em 
RfinPastSuperdeixis}.  

\end{description}

\vspace{1 cm}
\begin{description}
\item[Name]   RinfinPastDeixisSpec
\item[Task] The rule is for an independent infinite CLAUSE with a (past)
referential time expression. The {\bf deixis} attribute of this clause
is marked as {\em pastdeixis\/}, and the verb {\em have\/} is introduced to 
express `the past tense'. The verb directly following {\em have\/} is given a 
{\em past participle\/} form. 
For clauses with a modal, the (past) {\bf superdeixis}
attribute of the clause (which had been copied from the embedded sentence) is 
set to {\em omegadeixis\/} again. 
\item[File] english:RC\_Deixis2.mrule (mrules71.mrule)
\item[Semantics] To establish that the relation between Rs and S is 
before: PAST. 
\item[Example] \mbox{}\\
x1 work$_{omegatense}$ yesterday $\rightarrow$ x1 have$_{omegatense}$ 
worked yesterday  (I believe him to have worked yesterday)
\item[Remarks] 
The superdeixis of the clause may be pastdeixis in 
case it was copied from an embedded clause in the proposition substitution 
rules as e.g. with the modal verbs. \\
This rule has the same semantics as RfinPastdeixis.
\end{description}

\vspace{1 cm}
\begin{description}
\item[Name] RimpDeixis
\item[Task] The rule is for an independent finite CLAUSE that will get {\em 
imperativemood\/} or {\em letsmood\/} in the moodrules. It does not matter 
whether it has its own 
referential time expression, as long as it is not pastdeixis. 
The {\bf deixis} attribute of this clause is given {\em presentdeixis\/}; 
the {\bf modus} attribute is set to {\em imperative\/}.
\item[File] english:RC\_Deixis2.mrule (mrules71.mrule)
\item[Semantics] The relation between Rs and S is not relevant.
\item[Example] \mbox{}\\
(x1 do work)$_{omegamodus, omegadeixis}$ $\rightarrow$ (x1 do 
work)$_{imperative, presentdeixis}$
\item[Remarks] This rule is only meant to let imperatives pass the deixis rule 
class. Syntactically, they are seen as finite present tense main clauses, 
because otherwise other rules would have to make exceptions for them.\\
The rule should also allow for a future referential adverbial, which is not the 
case now\footnote{This had been remedied by the time the document was 
approved.}.
\end{description}

\vspace{1 cm}
\begin{description}
\item[Name]   Rfinirrmodaldeixis
\item[Task] The rule is for an independent finite CLAUSE with perfective, 
non-retrospective aspect that contains one of 
the modals {\em can\/} or {\em must\/}, and has its own (past)
referential time expression. Contrary to the ordinary past deixis rule, this 
rule is for an irrealis construction. The 
modal is given past tense if it takes one ({\em can - could\/}); otherwise, it 
is replaced by another modal ({\em must\/} $\rightarrow$ {\em should\/}).
The {\bf deixis} attribute of the clause is set to {\em pastdeixis\/}. If the {
\bf superdeixis} attribute of the clause is not {\em omegadeixis\/} (because 
it copied the superdeixis of an embedded clause, in the proposition 
substitution rules), it is set at omega here.
\item[File] english:RC\_Deixis2.mrule (mrules71.mrule)
\item[Semantics] The relation between Rs and S is simultaneous, but in an 
irrealis context.
\item[Example] \mbox{}\\
x1 can$_{omegatense}$ come $\rightarrow$ x1 can$_{pasttense}$ come\\
x1 must$_{omegatense}$ come $\rightarrow$ x1 should$_{pasttense}$ come
\item[Remarks] This rule has NOT been added to the control expression and 
transfer yet! So it has not been tested.\\
Syntactically, the clause is seen as a finite past tense main sentence. Other 
adverbials than referential time adverbials should be considered to belong 
to the embedded argument clause. This condition has not been added yet. 
\end{description}

\newpage
\subsubsection{B. Superdeixis Rules}
\begin{description}
\item[Name]   RfinPresentSuperdeixis
\item[Task] The rule is for a dependent finite CLAUSE that does not contain its 
own (present, past or future) referential time expression. 
The leftmost verb is given {\em presenttense\/} if allowed by that verb;
the {\bf superdeixis} attribute of the clause is set to {\em presentdeixis\/}, 
if it still was omega. 
\item[File] english:RC\_SuperDeixis.mrule (mrules72.mrule)
\item[Semantics] The `indirect' relation between Rs and S is simultaneous: 
PRESENT
\item[Example] \mbox{}\\
x1 do$_{omegatense}$ work $\rightarrow$ x1 do$_{presenttense}$ work (He claims 
that he works)\\
x1 be$_{omegatense}$ working $\rightarrow$ x1 be$_{presenttense}$ working (He 
claims that he is working)
\item[Remarks] The superdeixis of the clause may already be  presentdeixis in 
case it was copied from an embedded  clause in the proposition substitution 
rules as e.g. with the modal verbs. \\
Conditions on perfective aspect and the presence of a duration adverbial avoid:
{\em (he says) that he swims for four hours} and {\em (he says) that he swims 
at 4 o'clock} which have only habitual interpretations! The only 
perfective clauses allowed are: {\em (he says) that he swims} (which should get 
a lower bonus in the future) and {\em (he says) that he has swum}.\\
The combination perfective, -- retro + duration adverbial is accounted for in 
Rfinpresfutsuperdeixis.
\end{description}

\vspace{1 cm}
\begin{description}
\item[Name] RfinPastSuperdeixis
\item[Task] The rule is for a dependent finite CLAUSE with non-habitual aspect
that does not contain a
(present, past or future) referential time expression. 
The leftmost verb is given {\em pasttense\/} if allowed by the verb;
the {\bf superdeixis} attribute of the clause is set to {\em pastdeixis\/}, 
if it still was omega. 
\item[File] english:RC\_SuperDeixis.mrule (mrules72.mrule)
\item[Semantics] The `indirect' relation between Rs and S is before: PAST
\item[Example] \mbox{}\\
x1 do$_{omegatense}$ work $\rightarrow$ x1 do$_{pasttense}$ work (He claimed
that he worked)\\
x1 do$_{omegatense}$ work at 3 o'clock $\rightarrow$ x1 do$_{pasttense}$ 
work at 3 o'clock (He claimed that he worked at 3 o'clock)\\
x1 be$_{omegatense}$ working $\rightarrow$ x1 be$_{pasttense}$ working (He 
claimed that he was working)
\item[Remarks] 
The superdeixis of the clause may already be  pastdeixis in 
case it was copied from an embedded  clause in the proposition substitution 
rules as e.g. with the modal verbs. \\
Note that sentences without an adverbial have the interpretation `past-
simultanuous' ({\em He said that he worked\/}) and with an adverbial 
`past-before' ({\em He said that he worked (for 4 hours) last X-mas\/}). 
In the first interpretation 
the rule corresponds to Rinfinpastsuperdeixis, in the second to 
Rinfinpastsuperdeixisspec. This difference is not accounted for yet. For this 
purpose an IL-expression {\em Lpastpastsuperdeixis} will be added and the 
current
rule will be split up into {\em RfinPastSuperdeixis1} (corresponding to 
Lpastsuperdeixis and RinfinPastSuperdeixis) and {\em RfinPastSuperdeixis2} 
(corresponding to Lpastpastsuperdeixis and RinfinPastSuperdeixisSpec), 
requiring an 
omegadeixis or pastdeixis reference adverbial or a duration adverbial 
(+ perfective, -- retro aspect). RfinPastSuperdeixis1 will have similar 
conditions on the absence of a combination perfective, -- retro aspect 
+ duration adverbial as 
RfinPresentSuperdeixis (see above), since it is its past tense counterpart. 
The specific combination of + perfective, -- retro aspect, + duration adverbial 
is accounted for in RfinPastFutSuperdeixis (see below).\\
So, RfinPastSuperdeixis1 will account for: {\em that he worked/ he was working/
he had worked (for 4 hours)} and RfinPastSuperdeixis2 for: {\em that he worked 
for 4 hours/ he worked (for 4 hours) with X-mas/yesterday/ he was working with 
X-mas/yesterday/ he had worked (for 4 hours) with X-mas/yesterday}.\\
RfinPastFutSuperdeixis accounts for: {\em that he would work for 4 hours/ he 
would work (for 4 hours) with X-mas/tomorrow}.\\
RinfinPastSuperdeixis accounts for: {\em to work (for 4 hours) (with X-mas/
tomorrow)/ to be working (with X-mas/tomorrow)/ to have worked (for 4 hours)
(with X-mas/tomorrow)}.\\
RinfinPastSuperdeixisSpec accounts for {\em to have worked (for 4 hours) with 
X-mas}.

\end{description}

\vspace{1 cm}
\begin{description}
\item[Name]   RinfinPresentSuperdeixis
\item[Task] The rule is for a dependent infinite CLAUSE that does not contain 
a present, past or future referential time expression. 
The {\bf superdeixis} attribute of the clause is set to {\em presentdeixis\/}, 
if it still was omega. 
\item[File] english:RC\_SuperDeixis2.mrule (mrules78.mrule)
\item[Semantics] The `indirect' relation between Rs and S is simultaneous: PRESENT
\item[Example]
(x1 work)$_{omegadeixis}$ $\rightarrow$ (x1 work)$_{presentdeixis}$ (He likes
to work)
\item[Remarks] 
The superdeixis of the clause may already be  presentdeixis in 
case it was copied from an embedded  clause in the proposition substitution 
rules as e.g. with the modal verbs. \\
The rule does not allow for future time adverbials now, but will do so in the 
future, e.g.\ {\em He likes to work tomorrow\/}\footnote{By the time this 
document was approved, future time adverbials had been added.}.\\
Note that the presence of  an omegadeixis adverbial also yields a kind of future 
interpretation: {\em He believes to come at 3 o'clock}.
\end{description}

\vspace{1 cm}
\begin{description}
\item[Name] RinfinPastSuperdeixis
\item[Task] The rule is for a dependent infinite CLAUSE that does not contain its 
own (present or past) referential time expression. 
The {\bf superdeixis} attribute of the clause is set to {\em pastdeixis\/}, 
if it still was omega. 
\item[File] english:RC\_SuperDeixis2.mrule (mrules78.mrule)
\item[Semantics] The `indirect' relation between Rs and S is before: PAST
\item[Example] 
(x1 work)$_{omegadeixis}$ $\rightarrow$ (x1 work)$_{presentdeixis}$ (He liked
to work)
\item[Remarks] 
The superdeixis of the clause may already be  pastdeixis in 
case it was copied from an embedded  clause in the proposition substitution 
rules as e.g. with the modal verbs. \\
The rule does not allow for future time adverbials now, but will do so in the 
future, e.g. {\em He liked to work tomorrow\/}\footnote{By the time this 
document was approved, future time adverbials had been added.}.\\
Note that the presence of an 
omegadeixis time adverbial also yields some kind of future interpretation:
{\em He believed to come at 3 o'clock}.
\end{description}

\vspace{1 cm}
\begin{description}
\item[Name]   RinfinPastSuperdeixisSpec
\item[Task] The rule is for a dependent infinite CLAUSE with perfective aspect
that has 
an omegadeixis referential time expression. 
The {\bf superdeixis} attribute of the clause is set to {\em pastdeixis\/}, 
if it still was omega, and the verb {\em have\/} is introduced to express the 
past tense. The following verb is made a {\em participle\/}.
\item[File] english:RC\_SuperDeixis2.mrule (mrules78.mrule)
\item[Semantics] The `indirect' relation between Rs and S is before: PAST
\item[Example] 
x1 buy x2 x4(=at three o'clock) $\rightarrow$ x1 have bought x2 x4(=at three 
o'clock)  (He believed her to have bought the car at three o'clock)
\item[Remarks] 
The superdeixis of the clause may already be  pastdeixis in 
case it was copied from an embedded  clause in the proposition substitution 
rules as e.g. with the modal verbs. \\
This rule will correspond to Lpastpastsuperdeixis and RfinPastSuperdeixis2 
(see remarks on RFinPastSuperdeixis above).
\end{description}


\vspace{1 cm}
\begin{description}
\item[Name]   RfinPresFutSuperdeixis
\item[Task] The rule is for a dependent finite CLAUSE that has a 
(future or omega) referential time expression. 
The {\bf superdeixis} attribute of the clause is set to {\em presentdeixis\/}, 
if it was still omegadeixis.
The auxiliary of the future {\em will\/} is introduced in case {\em will\/} was 
not present yet, and given {\em 
presenttense\/}, the following verb is given {\em infinitive\/} modus, 
and if the auxiliary {\em do\/} was present, it is deleted. 
\item[File] english:RC\_Future.mrule (mrules74.mrule)
\item[Semantics] The `indirect' relation between Rs and S is simultaneous: PRESENT, but 
with S `shifted' into the future
\item[Example] \mbox{}\\
x1 do work x4(=tomorrow) $\rightarrow$ x1 will work x4(=tomorrow) (I believe 
that he will work tomorrow)\\
x1 be working x4(=at three o'clock) $\rightarrow$ x1 will be working x4(=at 
three o'clock) (I know that he will be working at three o'clock)\\
x1 will work x4(=tomorrow) $\rightarrow$ x1 will work x4(=tomorrow) 
(I know that he will work tomorrow)
\item[Remarks] 
The superdeixis of the clause may already be  presentdeixis in 
case it was copied from an embedded  clause in the proposition substitution 
rules. \\
The rule is split into three subrules: 1) no auxiliary, 2) {\em do} and 3) {\em 
will}. Other auxiliaries are excluded.\\
Conditions on omegadeixis adverbials and retrospectivity avoid {\em (she says) 
that he will have come at 3 o'clock} as a result of this rule.
\end{description}

\vspace{1 cm}
\begin{description}
\item[Name] RfinPastFutSuperdeixis
\item[Task] The rule is for a dependent finite CLAUSE that has a
(future or omega) referential time expression. 
The {\bf superdeixis} attribute of the clause is set to {\em pastdeixis\/}, 
if it was not yet.
The auxiliary of the future {\em will\/} is introduced in case it is not 
present already, and given {\em 
pasttense\/}, the following verb is given {\em infinitive\/} modus, 
and if the auxiliary {\em do\/} was present, it is deleted. 
\item[File] english:RC\_Future.mrule (mrules74.mrule)
\item[Semantics] The relation between Rs and S is before: PAST, but 
with S `shifted' into the future
\item[Example] \mbox{}\\
x1 do work x4(=tomorrow) $\rightarrow$ x1 will work x4(=tomorrow) (I thought
that he would work tomorrow)\\
x1 be working x4(=at three o'clock) $\rightarrow$ x1 will be working x4(=at 
three o'clock) (I thought that he would be working at three o'clock)
\item[Remarks] 
The superdeixis of the clause may already be  pastdeixis in 
case it was copied from an embedded clause in the proposition substitution 
rules. \\
The rule is split into three subrules: 1) no auxiliary, 2) {\em do} and 3) {\em 
will}. Other auxiliaries are excluded.\\
Conditions on omegadeixis adverbials and retrospectivity avoid {\em (she said) 
that he would have come at 3 o'clock} as a result of this rule.
\end{description}

\vspace{1 cm}
\begin{description}
\item[Name] Rfinirrmodalsuperdeixis
\item[Task] The rule is for a dependent finite CLAUSE that contains one of 
the modals {\em can\/} or {\em must\/}, and does not have a present or 
past
referential time expression. Contrary to the ordinary past superdeixis rule, 
this rule is for an irrealis construction. The 
modal is given past tense if it takes one ({\em can - could\/}); otherwise, it 
is replaced by another modal ({\em must\/} $\rightarrow$ {\em should\/}).
The {\bf superdeixis} attribute of the clause is set to {\em pastdeixis\/}, if
 it still was omega.
\item[File] english:RC\_Deixis2.mrule (mrules71.mrule)
\item[Semantics] The `indirect' relation between Rs and S is before: PAST, but 
in irrealis context.
\item[Example] \mbox{}\\
x1 can$_{omegatense}$ come $\rightarrow$ x1 can$_{pasttense}$ come (He said 
that she could come)\\
x1 must$_{omegatense}$ come $\rightarrow$ x1 should$_{pasttense}$ come (He said 
that she should come)
\item[Remarks] This rule has NOT been added to the control expression and 
transfer yet!! So it has not been tested yet.\\
Syntactically, the clause is seen as a finite past tense sentence. Other adverbials 
are considered to belong to the embedded argument clause. This latter 
condition has not been added yet.
\end{description}

\item[Remark]: Rfinpastsuperdeixis will be split into two rules, one 
corresponding to Rinfinpastsuperdeixis, i.e.\ Rfinpastsuperdeixis1, and the 
other corresponding to Rinfinpastsuperdeixisspec, i.e.\ Rfinpastsuperdeixis2. 
An Interlingua expression Lpastpastsuperdeixis will be added for these rules. 
For examples, see RFinPastSuperdeixis.
\item[Filters]: There are no associated filters.
\item[Speed rules] There are no associated speed rules.
\item[Rules not in control expression]:
Rfinirrmodaldeixis, Rfinirrmodalsuperdeixis.
\end{description}

\newpage
\subsection{TC\_HaveModalAdaptation}

\begin{description}
\item[Kind] Optional Transformation, followed by Obligatory Filter
\item[Task] To reverse the order of the auxiliary of the perfect {\em have\/} 
and the modal verbs {\em can, may\/} and {\em must\/}, and to change the modals 
to past tense, if they can take one ({\em can, may\/}) or replace them by 
another modal ({\em must\/} $\rightarrow$ {\em should\/}). See RRetroSpec above.

The transformation class is ordered crucially after a special Retro rule (where 
{\em have\/} is introduced without giving the following verb a {\em participle
\/} modus) and the (past) deixis rules.

\vspace{1 cm}
\begin{description}
\item[Name] Thavemodaladaptation
\item[Task] To reverse the order of the auxiliary of the perfect {\em have\/} 
and the modal verbs {\em can, may\/} and {\em must\/}, and to change the modals 
to past tense, if they can take one ({\em can, may\/}) or replace them by 
another modal ({\em must\/} $\rightarrow$ {\em should\/}). 
\item[File] english:RC\_Retro.mrule (mrules76.mrule)
\item[Semantics] --
\item[Example] \mbox{}\\
x1 have$_{pasttense}$ can come $\rightarrow$ x1 can$_{pasttense}$ have come \\
x1 have$_{pasttense}$ must come $\rightarrow$ x1 should$_{pasttense}$ have come 
\item[Remarks] This rule has NOT been added to the control expression yet!!
so it has not been tested.
\end{description}

\item[Remark] This transformation class was not mentioned in docs.\ 150 or 53.
\item[Filters]: the following filter follows this transformation:

\vspace{1 cm}
\begin{description}
\item[Name] Fhavemodaladaptation
\item[Task] To stop any construction in generation that should have gone 
through the (optional) transformation ThavemodalAdaptation
\item[File] english:RC\_Retro.mrule (mrules76.mrule)
\item[Example] 
\item[Remarks] This filter has NOT been added to the control expression yet!!
So it has not been tested.
\end{description}

\item[Speed rules]: there are no associated speed rules.
\item[Not in control expression] Thavemodaladaptation, Fhavemodaladaptation
\end{description}

\newpage
\subsection{TC\_SuperdeixisAdaptation}
\begin{description}
\item[Kind] Obligatory Transformation Class
\item[Task] 
To check the superdeixis of the embedded 
propositional phrases and sentences with the (super)deixis 
of the clause, and to adapt superdeixis (and deixis)
of embedded propositional phrases 
and sentences on behalf of the surface parser.\\
In generation, the value {\em presentdeixis\/} or {\em pastdeixis\/} of 
the {\bf superdeixis} 
attribute of an embedded finite clause is replaced by {\em omegadeixis\/} 
and vice versa in analysis (on basis of its deixis value, see below).
This is needed for reversibility reasons: in generation, the embedded clause 
was given a value for {\bf superdeixis} in RC\_Deixis (see section 4.7), but 
in analysis, the Surface 
Parser is not able to assign Superdeixis to an embedded clause, and hence it 
must be calculated somewhere in the M-grammar prior to RC\_Deixis.

In generation, the {\bf deixis} of an embedded finite temporally dependent 
sentence is set to 
{\em presentdeixis\/} or {\em pastdeixis\/}.
In analysis, such an embedded sentence receives {\em omegadeixis\/} for 
its {\bf deixis} attribute. Again, this is for reversibility 
reasons: in analysis, the Surface Parser assigns deixis even to dependent 
sentences. This must be `undone' somewhere before RC\_Deixis.
 It is assumed that 
the deixis attribute of XPPROPs is irrelevant. 

There are two kinds of superdeixis adaptation transformations: non-iterative 
and iterative. The non-iterative ones
form an {\em obligatory} transformation class 
with a {\em default} transformation which is the negation of all other 
transformations of that class. This class is followed by several 
iterative superdeixis adaptation transformations, each with a filter.
This means that the default transformation of the obligatory non-iterative 
class must also let the iterative cases pass, but of course, these cases cannot 
be formulated as the negation of the iterative rules!!

There are in general 5 subcases for the embedded propositional phrase or 
sentence in every transformation or subrule of a 
transformation, although not all five of them are applicable in every 
transformation:

\begin{enumerate}
\item {\bf independent }:\\
No check, no adaptation.\\
It is checked whether the verb of the CLAUSE can have independent complements
(AUX\_tempindepcomplvps).\\
(Only applies to finite sentences and a special case of infinite sentences)
\item {\bf dependent finite present }:\\
Check + adaptation of deixis and superdeixis.\\
(Only applies to sentences)
\item {\bf dependent finite past}:\\
Check + adaptation of deixis and superdeixis.\\
(Only applies to sentences)
\item {\bf dependent infinite present}:\\
Check + adaptation of superdeixis.
\item {\bf dependent infinite past}:\\
Check + adaptation of superdeixis.
\end{enumerate}

In analysis these transformations cause a lot of ambiguity. Therefore, 
efficiency and Rosetta3-plausibility conditions have been added to these 
transformations:
\begin{itemize}
\item Tests for the presence of adverbials are added to cut off some possible 
paths here instead of in the deixis rules 
(QUOTE\_temprefnotfound). The test for the correct adverbial is 
left to the deixis rules. 
\item It seemed plausible that only infinite sentences that contain {\em have}
 \`{a}nd an overt reference time adverbial can be independent.
For this purpose the transformations for embedded sentences 
have been split into two subrules: one that tests for the presence of {\em 
have} and one that tests for the absence of {\em have}.
\item The combinations present (super)deixis of the clause with present deixis 
(i.e.\ independent!) without an overt adverbial in the embedded sentence and 
past 
(super)deixis of the clause with past deixis (i.e.\ independent!) without an 
overt adverbial in the embedded sentence are not considered plausible in an 
application like Rosetta3 and are ruled out in these transformations.
\end{itemize}

The motivation and historical reasons for the (actual form of the) 
transformations can be found in doc.\ R263, and a description of the treatment 
of superdeixis throughout the whole grammar in doc.\ R320.\\
Of course, this transformation class was not foreseen in doc.\ 150.\\

The two files that contain the superdeixis adaptation transformations are 
organized roughly in the following way: tc\_superdeixisadaptation contains the 
transformations with embedded {\em sentences} and tc\_superdeixisadaptation2 
the ones with embedded {\em xpprops\/} and {\em xps\/} (including gerunds).\\
First the non-iterative obligatory rule class (A) will be treated and then the 
iterative cases (B, C) and finally the special filters (D).


\end{description}

\subsubsection{TC\_SuperdeixisAdaptation A}
\begin{description}
\item[Kind] Obligatory Transformation Class

\vspace{1 cm}
\begin{description}
\item[Name] TNoSuperdeixisAdaptation
\item[Task] `Default' rule for the case there was no proposition substitution 
before the time rules of the non-iterative kind that is treated in the other 
rules of this transformation
class (cases mentioned in the matchconditions; see below) and for the case 
there was or was not proposition 
substitution before the time rules of the iterative kind (not-mentioned cases).
\\
Cases:
\begin{enumerate}
\item No SENTENCE under CLAUSE with relation {\em complrel, 
      prepobjrel, subjrel, extraposrel}.
\item No SENTENCE under VERBP with relation {\em complrel, prepobjrel}.
\item No NP under CLAUSE with relation {\em subjrel} and NPhead {\em SentNP, 
      OpenIngNP}.
\item No OPEN- or CLOSEDXPPROP (with X = ADJ, N, ADV, VERB or PREP) under 
  VERBP with relation {\em complrel, dirargrel, locargrel\/} or {\em predrel}.
\item No PREPP under VERBP with relation {\em prepobjrel} if it has an OPEN- 
  or CLOSEDXPPROP (with X = ADJ, N, ADV, VERB or PREP) or a SENTENCE with 
  relation {\em objrel} or 
  {\em complrel\/}\footnote{In fact, in the actual system
a complrel cannot occur under a PREPP, and it will be removed from the set.} 
or an NP with relation {\em objrel} 
and NPhead {\em SentNP\/} or {\em OpenIngNP\/} as daughter.
\item No ADJP under VERBP with relation {\em predrel} if it has an NP, ADVP, 
PREP or QP with relation {\em forobjrel, hoprel or degreemodrel} as daughter,  
or an argument SENTENCE.
\item No NP under VERBP with relation {\em subjrel, objrel, indobjrel} and 
NPhead {\em SentNP\/} or {\em OpenIngNP\/}\footnote{A subjrel should 
not occur under a VERBP at all and will be removed from the set.}.
\end{enumerate}

\item[File] english:TC\_SuperdeixisAdaptation.mrule (mrules73.mrule)
\item[Semantics] --
\item[Example] x1 does sing\\
x1 does eat an apple\\
x1 is ill\\
x1 is ill (enough to go home)
\item[Remarks] 
\end{description}

\vspace{1 cm}
\begin{description}
\item[Name] TSuperdeixisAdaptation1
\item[Task] 
Check superdeixis of a temporally dependent complement SENTENCE that is a 
daughter of CLAUSE and has a relation {\em complrel, prepobjrel, subjrel\/} or 
{\em extraposrel}, with the superdeixis or deixis value of the higher CLAUSE.
The superdeixis of the complement SENTENCE is set omega or given a 
value present or past (in generation and analysis resp.) and the deixis value 
is given a value present or past (gener.) or
set omega (anal.) in the finite cases. (This is needed for efficiency 
reasons of the surface parser.) For temporally independent sentences this rule 
does nothing.
\item[File] english:TC\_SuperdeixisAdaptation.mrule (mrules73.mrule)
\item[Semantics] --
\item[Example] \mbox{}\\
independent: be$_{present}$ possible (that he 
sang)$_{\frac{omegasuperdeixis}{pastdeixis}}$ $\rightarrow$
be$_{present}$ possible (that he sang)$_{\frac{omegasuperdeixis}{pastdeixis}}$ 
\\
dependent, infinite: be$_{past}$ able (he 
to sing)$_{\frac{pastsuperdeixis}{omegadeixis}}$ $\rightarrow$
be$_{past}$ able (he to sing)$_{\frac{omegasuperdeixis}{omegadeixis}}$ 
\\
dependent, finite: be$_{past}$ possible (that
he sang)$_{\frac{pastsuperdeixis}{omegadeixis}}$ $\rightarrow$
be$_{past}$ possible (that he sang)$_{\frac{omegasuperdeixis}{pastdeixis}}$ 
\item[Remarks] 
 The transformation has been split into two subrules, one for the 
absence of {\em have} in the embedded sentence, in which case the independent 
case does not apply 
to infinite SENTENCEs, and one for the presence of {\em have}, in which 
case an independent infinite SENTENCE is possible in case a referential 
adverbial is present (NOT QUOTE\_temprefnotfound).\\
This rule will be split up in TSuperdeixisAdaptation1a 
(for embedded sentences following the VERBP, viz.\ in extraposrel; complrel and 
prepobjrel are allowed too but should never occur outside a VERBP) and 
TSuperdeixisAdaptation1b (for embedded sentences preceding the VERBP, viz.\ for 
subjrel; however, a sentence will never occur in subject position: it is moved 
to leftdislocrel or extraposrel)\footnote{This has been done by the time this 
document was approved.}. Leftdislocrel sentences should be covered 
in this latter transformation, but are still (for historical reasons) 
dealt with in the iterative 
transformation TSuperdeixisAdaptation11. The rule will be split up to make 
reference to a modal verb easier: if a modal verb is present, the superdeixis 
of the embedded sentence has already been dealt with in a previous cycle, and 
should not be covered here. Rather than modify TNoSuperdeixisAdaptation, this 
case will be allowed to pass through the `independent path' of the current 
rule. For 
a problem concerning the modal {\em will\/}, see the remarks to 
TSuperdeixisAdaptation2 below. 

\end{description}

\vspace{1 cm}
\begin{description}
\item[Name] TSuperdeixisAdaptation2
\item[Task] 
Check superdeixis of a temporally dependent complement SENTENCE that is a 
daughter of 
VERBP and has relation {\em complrel\/} or {\em 
prepobjrel\/}\footnote{In fact, a 
prepobjrel will never have a sentence as node, but always a PREPP, so this 
relation will be removed from the set.},
 with the superdeixis or 
deixis value of the higher CLAUSE. 
The superdeixis of the complement SENTENCE is set omega (gener.) or given a 
value present or past (anal.) and the deixis value is given a value present or 
past (gener.) or set omega (anal.) in the finite cases. (This is needed for 
efficiency 
reasons of the surface parser.) For temporally independent sentences this rule 
does nothing.
\item[File] english:TC\_SuperdeixisAdaptation.mrule (mrules73.mrule)
\item[Semantics] --
\item[Example] \mbox{}\\
independent: x1 do$_{present}$ know (that he 
sang)$_{\frac{omegasuperdeixis}{pastdeixis}}$ $\rightarrow$
x1 do$_{present}$ know (that he sang)$_{\frac{omegasuperdeixis}{pastdeixis}}$ 
\\
dependent: x1 do$_{past}$ know (that he 
sang)$_{\frac{pastsuperdeixis}{omegadeixis}}$ $\rightarrow$
x1 do$_{past}$ know (that he sang)$_{\frac{omegasuperdeixis}{pastdeixis}}$ 
\item[Remarks] 
The transformation has been split into two subrules, one for the 
absence of {\em have\/} in the embedded sentence, in which case the independent 
case does not apply 
to infinite SENTENCEs, and one for the presence of {\em have\/}, in which 
case an independent infinite SENTENCE is possible in case a referential 
adverbial is present (NOT QUOTE\_temprefnotfound).\\
The rule will be adapted to let CLAUSEs which contain a 
modal verb also pass this transformation (see the remarks to 
TSuperdeixisAdaptation1)\footnote{This has been done by the time this document 
was approved.}. However, when the modal is the syncategorematically 
introduced verb {\em will\/} (see RFinPresFut(super)Deixis and 
RFinPastFutSuperdeixis: {\em hij moet morgen zwemmen - he {\em will} have to 
swim tomorrow\/}), it is not true that superdeixis was already taken care 
of in a previous cycle, because there was no previous cycle. 
Since at the moment there is no way to find out whether 
{\em will\/} was added syncategorematically or not, this problem has not been 
solved yet.
\end{description}

\vspace{1 cm}
\begin{description}
\item[Name]   TSuperdeixisAdaptation3
\item[Task] 
Check superdeixis of a temporally dependent XPPROP in the VERBP with 
superdeixis or deixis of the higher CLAUSE.
The superdeixis of the complement phrase is set omega (gener.) or given a 
value present or past (anal.). (This is needed for efficiency 
reasons of the surface parser.) 
\item[File] english:TC\_SuperdeixisAdaptation2.mrule (mrules89.mrule)
\item[Semantics] --
\item[Example] \mbox{}\\
dependent: 
x1 do$_{past}$ consider (him ill)$_{pastsuperdeixis}$ $\rightarrow$
x1 do$_{past}$ consider (him ill)$_{omegasuperdeixis}$
\item[Remarks] 
There is a subrule for every OPEN- or CLOSEDXPPROP with X= ADJ, N etc. This 
rule could have been
written down in a shorter way, but at the time this rule was 
written the necessary notation with N.REC did not exist yet.
\end{description}

\vspace{1 cm}
\begin{description}
\item[Name]   TSuperdeixisAdaptation4
\item[Task] 
 Check superdeixis of temporally dependent XPPROP as object of a prepositional 
phrase with superdeixis or deixis of the higher CLAUSE.
The superdeixis of the complement phrase is set omega (gener.) or given a 
value present or past (anal.). (This is needed for efficiency 
reasons of the surface parser.) 
\item[File] english:TC\_SuperdeixisAdaptation2.mrule (mrules89.mrule)
\item[Semantics] --
\item[Example] \mbox{}\\
dependent: 
x1 do$_{past}$ regard x2 as (x2 ill)$_{pastsuperdeixis}$ $\rightarrow$
x1 do$_{past}$ regard x2 as (x2 ill)$_{omegasuperdeixis}$
\item[Remarks]
There is a subrule for every OPEN- or CLOSEDXPPROP with X= ADJ, N etc., except 
for X= ADV, which is supposed not to occur under a PREPP anyway. This 
rule could have been
written down in a shorter way, but at the time this rule was 
written the necessary notation with N.REC did not exist yet.
\end{description}

\vspace{1 cm}
\begin{description}
\item[Name] TSuperdeixisAdaptation5
\item[Task] 
Check superdeixis of a temporally dependent complement SENTENCE which is a 
daughter of {\em prepobjrel}/PREPP and has relation {\em objrel},
 with the superdeixis or 
deixis value of the higher CLAUSE.
The superdeixis of the complement SENTENCE is set omega (gener.) or given a 
value present or past (anal.) and the deixis value is given a value present or 
past (gener.) or
set omega (anal.) in the finite cases. (This is needed for efficiency 
reasons of the surface parser.) For temporally independent sentences this rule 
does nothing. 
\item[File] english:TC\_SuperdeixisAdaptation.mrule (mrules73.mrule)
\item[Semantics] --
\item[Example] \mbox{}\\
independent: x1 do$_{present}$ count on (that he 
sang)$_{\frac{omegasuperdeixis}{pastdeixis}}$ $\rightarrow$
x1 do$_{present}$ count on (that he 
sang)$_{\frac{omegasuperdeixis}{pastdeixis}}$ 
\\
dependent: x1 do$_{past}$ count on (that he 
sang)$_{\frac{pastsuperdeixis}{omegadeixis}}$ $\rightarrow$
x1 do$_{past}$ count on (that he 
sang)$_{\frac{omegasuperdeixis}{pastdeixis}}$ 
\item[Remarks] 
The transformation has been split into two subrules, one for the 
absence of {\em have} in the embedded sentence, in which case the independent 
case does not apply 
to infinite SENTENCEs, and one for the presence of {\em have}, in which 
case an independent infinite SENTENCE is possible in case a referential 
adverbial is present (NOT QUOTE\_temprefnotfound).
\end{description}

\vspace{1 cm}
\begin{description}
\item[Name]   TSuperdeixisAdaptation6
\item[Task] To perform superdeixis adaptation for clauses which have an 
embedded sentence-like NP (i.e.\ an NP which has a SENTENCE as its head) 
in their VERBP in {\em objrel\/} or {\em indobjrel\/}\footnote{The rule allows 
a {\em subjrel\/} as well, but that should never occur in the VERBP and the 
relation will be removed from the set.}. 
\item[File] english:TC\_SuperdeixisAdaptation2.mrule (mrules89.mrule)
\item[Semantics] --
\item[Example] \mbox{}\\
dependent: 
x1 do$_{past}$ hear (his singing)$_{pastsuperdeixis}$ $\rightarrow$
x1 do$_{past}$ hear (his singing)$_{omegasuperdeixis}$
\item[Remarks] 
The rule assumes that this sentence is always temporally dependent.
\end{description}

\vspace{1 cm}
\begin{description}
\item[Name] TSuperdeixisAdaptation7
\item[Task] To perform superdeixis adaptation for clauses which have an 
embedded sentence-like NP (i.e.\ an NP which has a SENTENCE as its head) 
as a prepositional object in their VERBP. 
\item[File] english:TC\_SuperdeixisAdaptation2.mrule (mrules89.mrule)
\item[Semantics] --
\item[Example] \mbox{}\\
dependent: 
x1 do$_{past}$ count on (your being there)$_{pastsuperdeixis}$ $\rightarrow$
x1 do$_{past}$ count on (your being there)$_{omegasuperdeixis}$
\item[Remarks] 
The rule assumes that this sentence is always temporally dependent.
\end{description}

\vspace{1 cm}
\begin{description}
\item[Name]   TsuperdeixisAdaptation8
\item[Task] To perform superdeixis adaptation for clauses which have an 
embedded superdeixis-specified XP (ADVP, PREPP, NP, QP) or an argument 
SENTENCE under an 
ADJP-node in the VERBP. The superdeixis of ADVP etc. is set omega (gener.) or 
given a value present or past (anal.).
(Modifier sentences are dealt with in an iterative rule, 
TSuperdeixisAdaptation9).
\item[File] english:TC\_SuperdeixisAdaptation2.mrule (mrules89.mrule)
\item[Semantics] --
\item[Example] \mbox{}\\
independent: 
x1 do$_{present}$ be glad (that she came 
  yesterday)$_{\frac{pastsuperdeixis}{omegadeixis}}$ $\rightarrow$
x1 do$_{present}$ be glad (that she came
  yesterday)$_{\frac{omegasuperdeixis}{pastdeixis}}$
\\
dependent: 
x1 do$_{past}$ be (very ill)$_{pastsuperdeixis}$ $\rightarrow$ \\
x1 do$_{past}$ be (very ill)$_{omegasuperdeixis}$
\item[Remarks] 
Note that the rule assumes that XP is 
always temporally dependent, but its deixis is not checked. Embedded 
sentences can be both dependent or independent.\\
In case there is a 
modal present, superdeixis has already been taken care of in a previous cycle.
Thus, modals will have to be allowed to pass here through the 
`independent path'\footnote{This had been implemented by the time this 
document was approved.}. For a 
problem with the modal {\em will\/}, see the remarks to 
TSuperdeixisAdaptation2.\\
 The reason this adaptation transformation is necessary here is 
that the modification rules with ADVP, PREP, NP and QP in de ADJPPPROP grammar 
are ordered before the (super)deixis rules.\\
There is a subrule for every XP with X =
N, ADV etc. This rule could have written down in a shorter way, but at the time 
this rule was written the necessary notation with N.REC did not exist yet.
\end{description}

\vspace{1 cm}
\begin{description}
\item[Name] TIdSuperdeixisAdaptation
\item[Task] Vacuous rule to let clauses with an embedded propositional 
structure in the VERBP 
that is fully unspecified for (super)deixis pass this transformation class. 
This rule is needed for idioms.
\item[File] english:TC\_SuperdeixisAdaptation.mrule (mrules73.mrule)
\item[Semantics] --
\item[Example] x1 do leave x2 in the lurch
\item[Remarks] 
\end{description}

\vspace{1 cm}
\begin{description}
\item[Name]   Tsuperdeixisadaptation10
\item[Task] To perform superdeixis adaptation for clauses which have a
sentence-like NP (i.e.\ an NP which has a SENTENCE as its head) 
as subject. 
\item[File] english:TC\_SuperdeixisAdaptation2.mrule (mrules89.mrule)
\item[Semantics] --
\item[Example] \mbox{}\\
dependent: 
(x1 eating people)$_{pastsuperdeixis}$ do$_{past}$ be wrong $\rightarrow$
(x1 eating people)$_{omegasuperdeixis}$ do$_{past}$ be wrong
\item[Remarks] The rule assumes that this sentence is always temporally 
dependent.
\end{description}

\end{description}

\newpage
\subsubsection{TC\_SuperdeixisAdaptation B}
\begin{description}
\item[Kind] Iterative Transformation Class, followed by Obligatory Filter

\vspace{1 cm}
\begin{description}
\item[Name] TsuperdeixisAdaptation9
\item[Task] To perform superdeixis adaptation for clauses which have one (or 
more; the class is iterative)
embedded modifier sentence (in toinfmodrel or postmodrel) under an 
ADJP-node in the VERBP. It is assumed that the sentence is 
always temporally dependent, but its deixis is not checked.
\item[File] english:TC\_SuperdeixisAdaptation2.mrule (mrules89.mrule)
\item[Semantics] --
\item[Example] \mbox{}\\
dependent: 
x1 do$_{past}$ be ill enough (to go home)$_{pastsuperdeixis}$ $\rightarrow$
x1 do$_{past}$ be ill enough (to go home)$_{omegasuperdeixis}$
\item[Remarks] This is a {\bf recursive} transformation.\\
It is assumed that this kind of modifier sentence is always 
infinite and temporally dependent, but if it were independent, then this rule 
would not apply, because this is a recursive rule and in independent cases 
it is not necessary to do anything. The efficiency and plausibility 
conditions for those cases should be carried out by special filters.\\
The reason this adaptation transformation is necessary here is 
that the modification rules in de ADJPPPROP grammar 
are ordered before the (super)deixis rules.
 
\end{description}
\item[Filters]: This transformation has an associated filter:
\vspace{1 cm}
\begin{description}
\item[Name] FsuperdeixisAdaptation9
\item[Task] To ensure that the iterative (and hence optional) 
TSuperdeixisAdaptation9 has been applied when it should in generation.
\item[File] english:TC\_SuperdeixisAdaptation2.mrule (mrules89.mrule)
\item[Example] 
\item[Remarks] 
\end{description}

\end{description}


\newpage
\subsubsection{TC\_SuperdeixisAdaptation C}
\begin{description}
\item[Kind] Iterative Transformation Class, followed by Obligatory Filter

\vspace{1 cm}
\begin{description}
\item[Name] TsuperdeixisAdaptation11
\item[Task] To perform superdeixis adaptation for clauses which have an 
embedded non-temporal adverbial or leftdislocated sentence directly under 
S\footnote{For historical reasons, leftdislocrel sentences are still dealt 
with here, although there may be only one of them. They should now go with 
TSuperdeixisAdaptation1b.}. 
The rule assumes that these sentences are always temporally dependent, even if 
the auxiliary {\em have\/} is present in them. Since there may be more of these 
sentences present, the rule is iterative.
\item[File] english:TC\_SuperdeixisAdaptation.mrule (mrules73.mrule)
\item[Semantics] --
\item[Example] \mbox{}\\
dependent: (Although he was ill)$_{\frac{pastsuperdeixis}{omegadeixis}}$ 
x1 do$_{past}$ count on him $\rightarrow$
(Although he was ill)$_{\frac{omegasuperdeixis}{pastdeixis}}$
x1 do$_{past}$ count on him
\item[Remarks] 
This is a {\bf recursive} transformation.\\
For historical reasons, the transformation has been split into two subrules, 
one for the absence of {\em have} in the embedded sentence and one for the 
presence of {\em have}. This is not necessary anymore, because the rule only 
applies in the dependent cases. In the independent cases this recursive 
transformation does not apply. The efficiency and plausibility conditions for 
those cases are formulated in a special filter: Ftempindep1.\\
The splitting into two subrules will be removed\footnote{This had been done by 
the time this document was approved.}.

\end{description}

\item[Filters]: This transformation has the following filter:

\vspace{1 cm}
\begin{description}
\item[Name]  FsuperdeixisAdaptation11
\item[Task] To ensure that the iterative (and hence optional) 
TSuperdeixisAdaptation11 was applied when necessary in generation.
\item[File] english:TC\_SuperdeixisAdaptation.mrule (mrules73.mrule)
\item[Example] 
\item[Remarks] 
the splitting into subrules will be removed (see footnote).
\end{description}

\item[Remark]
\item[Speed rules] There is an efficiency/plausibility filter Ftempindep1.
\end{description}

\newpage
\subsubsection{TC\_SuperdeixisAdaptation D}
\begin{description}
\item[Kind] Obligatory Filter Class, containing only one filter

\begin{description}
\item[Name] Ftempindep
\item[Task] 
To stop temporally independent sentences directly under CLAUSE with relation 
leftdislocrel or postsentadvrel in case:\\
1: present - present, without temporal adverbial in subordinate sentence\\
2: past - past, without temporal adverbial in subordinate sentence\\
These are not considered plausible within Rosetta3.

\item[File] english:TC\_SuperdeixisAdaptation.mrule (mrules73.mrule)
\item[Example] 
\item[Remarks] The splitting into subrules is unnecessary and will be 
removed\footnote{This had been done by the time this document was approved.}.
\end{description}


\end{description}

\newpage
\subsection{TC\_VerbLeft}

\begin{description}
\item[Kind] Obligatory Transformation Class
\item[Task] To provide a uniform structure for all finite clauses, with one 
(auxiliary) verb to the left of the VERBP, always placed in {\em auxrel\/}, and 
the head verb always within the VERBP.
For infinite clauses, all verbs 
remain in the VERBP. Moving the first verb out of the VERBP is easier for 
subject-aux inversion etc.

This transformation class was not mentioned in doc.\ 150, since there it was 
assumed that all auxiliaries would always be outside the VERBP.


\vspace{1 cm}
\begin{description}
\item[Name] TNoVerbLeft
\item[Task] To let infinitives pass this transformation class
\item[File] english:TC\_VerbLeft.mrule (mrules24.mrule)
\item[Semantics] --
\item[Example] x1 sing (I want to sing)
\item[Remarks]
\end{description}

\vspace{1 cm}
\begin{description}
\item[Name] TVerbLeft
\item[Task] 
To provide a uniform structure for all finite clauses, moving the first
(auxiliary) verb in the VERBP to the left of it in S, always placing it in 
{\em auxrel\/}. 
\item[File] english:TC\_VerbLeft.mrule (mrules24.mrule)
\item[Semantics] --
\item[Example] x1 [do sing] $\rightarrow$ x1 do [sing] \\
x1 [be singing] $\rightarrow$ x1 be [singing]
\item[Remarks]
\end{description}

\end{description}

\newpage
\subsection{TC\_SentOKrules}

\begin{description}
\item[Kind] Obligatory Transformation Class, preceded by Obligatory Filter
\item[Task] To extrapose sentences from the VERBP to a clause-final position,
which is specified by {\em extraposrel\/}. In doc.\ 150, it was assumed that 
this could be done in a PROPOK transformation class, together with many other 
tasks. However, these tasks have been split up (see section 2 of this document).

At the moment, it is assumed that extraposed sentences should always be placed 
sentence-finally. However, this probably is incorrect: sentences in {\em 
postsentadvrel\/} should follow extraposed sentences: {\em He told his 
father yesterday $_{extrap}$[that I wouldn't be there for Christmas], 
$_{posts}$[because he didn't want to lie to him]\/}. This will have to be 
altered, and the other rules in the subgrammar must be checked whether they 
indeed allow something to follow an extraposed sentence.

\vspace{1 cm}
\begin{description}
\item[Name] FPreExtrapos
\item[Task] To prevent extraposed sentences from not being put back where they 
belong in analysis (this is a speed filter, to ensure the proper application of 
the following transformation class).
\item[File] english:TC\_SentOKrules.mrule (mrules33.mrule)
\item[Example] 
\item[Remarks] The filter does not stop extraposed sentences in case there is 
no subject while there is a predicate inside the VERBP, since such a structure 
may be dealt with in the proposition 
substitution rules for XPPROPs where (generatively speaking) a sentential 
subject was extraposed straight away, without a `stop-over' as complrel in the 
XP.
\end{description}

\vspace{1 cm}
\begin{description}
\item[Name] TNoExtraposition
\item[Task] Vacuous rule, to let sentences which have nothing to extrapose pass 
this transformation class. The rule works for clauses without embedded 
sentence, with an embedded infinite sentence, and with an embedded THATsent of 
which the conjunction THAT is missing.
\item[File] english:TC\_SentOKrules.mrule (mrules33.mrule)
\item[Semantics] --
\item[Example] (It) do [surprise me he never told you]; He do [sing]
\item[Remarks] 
\end{description}

\vspace{1 cm}
\begin{description}
\item[Name] TExtraposition1
\item[Task] To extrapose finite sentences from the VERBP, including sentences
that came from subject position (see RExtrapSubjSentSubst in RC\_PropSubst 
above). In 
analysis, infinite open to-infs are accepted in extraposrel too.
 \item[File] english:TC\_SentOKrules.mrule (mrules33.mrule)
\item[Semantics] --
\item[Example] \mbox{}\\
x1 do [act [as if he had gone mad]] x4=yesterday $\rightarrow$ \\
x1 do [act] x4=yesterday [as if he had gone mad] \\
-- do [surprise x2 [that he had left]] x4=yesterday $\rightarrow$ \\
-- do [surprise x2] x4=yesterday [that he had left]
\item[Remarks] 
\end{description}

\vspace{1 cm}
\begin{description}
\item[Name]   TExtraposition2
\item[Task] To extrapose a finite THATsentence from a PREPP in the VERBP, and 
to introduce the dummy prepositional object {\em it\/} in its place.
 \item[File] english:TC\_SentOKrules.mrule (mrules33.mrule)
\item[Semantics] --
\item[Example] x1 do [count on [that you would be here]] x4=yesterday 
$\rightarrow$ x1 do [count] on it x4=yesterday [that you would be here]
\item[Remarks] This rule is only for clauses made by the verbpattern 
synPREPTHATSENT.
\end{description}

\vspace{1 cm}
\begin{description}
\item[Name] TPredExtrapos1
\item[Task] To extrapose finite sentences that are in complrel under a 
predicate inside the VERBP. 
In analysis, infinite open to-infs are accepted in extraposrel too.
 \item[File] english:TC\_SentOKrules.mrule (mrules33.mrule)
\item[Semantics] --
\item[Example] x1 do [be [certain [that I would pass the test]]] x4=yesterday 
$\rightarrow$
x1 do [be [certain]] x4=yesterday [that I would pass the test]
\item[Remarks] \mbox{}
\begin{itemize}
\item No rules have been written yet for extraposition of sentences from 
another position than complrel. They probably must be added in the future.
\item In analysis, the rule also works in cases it should not, viz.\ when the 
extraposed sentence must remain where it is until it is put in subject position 
of an XPPROP (X is not VERB) in the proposition substitution. The relevant 
(speed) conditions to prevent this must still be added. 
\end{itemize}
\end{description}

\end{description}

\newpage
\subsection{TC\_ControlRules} 

\begin{description}
\item[Kind] Obligatory Transformation Class
\item[Task] To verify that the subject (PRO-)variable of an embedded 
open 
SENTENCE or XPPROP (if there is any) is identical with a variable in the higher
clause (for an embedded sentence, the attribute {\em controller\/} of its main 
verb determines with which 
variable the subject must correspond: subject, (ind)object or prepobj), 
and to delete the embedded subject, pruning the XPPROP till only a predicate 
remains.  Closed XPPROPs, which do not need control, are also pruned (their 
subject becomes an object of the higher clause). A SENTENCE is not pruned.
The transformation class contains 33 rules, all for different S-trees. If new 
verbpatterns are discovered, perhaps more control rules will be needed, but 
this does not seem very likely.

This transformation class only deals with obligatory control. For 
non-obligatory control (e.g.\ of sentences in subject position, and control by 
EMPTY - see again the remarks made under RClosedNPPropSubst), special rules 
still have to be written. The current rules also function as the `narrow 
reading' rules 
for verbs that have non-obligatory control 
(i.e.\ they deal with the case there is full identity between antecedent and 
variable). For more 
information on the treatment of control, see doc.\ 111, 
{\em The interpretation of non-overt subjects in Rosetta3\/}, by Jan Odijk. 
Also, a new document on Control is to appear, again by Jan Odijk.

In doc.\ 150, no difference was made between obligatory 
and non-obligatory control with respect to rule class. Since the `default' rule
(i.e.\ the rule specifying when no control is needed) 
for `wide' NOC (with only partial identity of antecedent and controllee, and 
never a VAR as controllee) is different from the one for OblControl, it seems 
easiest to separate the two phenomena in two distinct classes. Also, NOC may be 
iterative, while OblControl is not. This has not been taken into account yet at 
all. The rules for FinControl that now occur in file TC\_NOC.mrule 
(see next section) are in 
fact also ordinary control rules. However, the default (=no action needed) rule 
for FinControl is 
quite different from the one in the current rule class, and hence it seemed 
simpler to separate the control of finite sentences from the rest of the rules.
Note that the contents of the Control Transformation Classes still have to be 
extended and revised.

For comment on the absence of control rules dealing with closed XPPROPs that 
have a sentential subject, see the remarks made under RClosedNPPropSubst above.

This rule class is ordered crucially before TC\_ObjectOKrules: the embedded 
subject of an XPPROP must have been made an object of the higher clause 
before it can raise to become subject of the higher clause (for raising 
the subject of a SENTENCE, a special ObjectOKrule exists). In Dutch, the rule 
class is 
ordered crucially after TC\_SentOKrules: extraposed sentences have NOC, while 
sentences that are not extraposed are subject to obligatory control. It is 
unknown to us whether this holds for English too. For more information on this 
subject, see the document on Control by Jan Odijk.

Infinitival adverbial sentences are not covered by the present transformation 
class, but by TC\_ConjSentControl (see below). They take the surface subject as 
controller, not some argument of the clause corresponding with the controller 
type of their verb. Thus, these rules must be ordered 
after TC\_ObjectOKrules.

\vspace{1 cm}
\begin{description}
\item[Name] TNoControl
\item[Task] Vacuous rule, to let structures in which there is no embedded 
XPPROP or open SENTENCE pass this transformation class.
\item[File] english:TC\_ControlRules1.mrule (mrules32.mrule)
\item[Semantics] --
\item[Example] x1 may [come], x1 do [count on the weather being fine], x1 do [
see [he leave]]
\item[Remarks] Since no NOC rules have been written yet, open sentences in 
subject position which are stopped by this transformation class cannot be dealt 
with yet.\\
This rule must 
be adapted to let structures in which there is a modal plus an embedded 
XPPROP or open SENTENCE pass. All other rules must be excluded for 
modals\footnote{This had been done by the time this document was approved.}. 
This is needed because control has already been 
dealt with in an earlier cycle, when the modal was not present yet. However, 
there is one problem with this approach: if the modal is the 
syncategorematically introduced verb {\em will\/}, there was no earlier cycle, 
and control still must be dealt with. This is the same problem as the one 
mentioned in the superdeixis adaptation rules (see the remarks to 
TSuperdeixisAdaptation2), and it has not been solved yet, since there is no way 
to know whether the modal was introduced syncategorematically or not.
\end{description}

\vspace{1 cm}
\begin{description}
\item[Name] TOblObjControlADJP
\item[Task] To verify that the subject (PRO-)variable of an embedded 
open ADJPPROP in the VERBP is identical with the object variable in the 
higher clause 
and to delete the embedded subject, pruning the ADJPPROP till only a predicate 
remains. 
\item[File] english:TC\_ControlRules1.mrule (mrules32.mrule)
\item[Semantics] --
\item[Example] x1 do paint x2 [x2 green] $\rightarrow$ x1 do paint x2 green
\item[Remarks] 
\end{description}

\vspace{1 cm}
\begin{description}
\item[Name] TOblSubjControlADJP
\item[Task] To verify that  the subject (PRO-)variable of an embedded 
open ADJPPROP in the VERBP is identical with the subject variable in the 
higher clause (there is no object)
and to delete the embedded subject, pruning the ADJPPROP till only a predicate 
remains. 
\item[File] english:TC\_ControlRules1.mrule (mrules32.mrule)
\item[Semantics] --
\item[Example] ? x1 do smell [x1 old] $\rightarrow$ x1 do smell old
\item[Remarks] 
\end{description}

\vspace{1 cm}
\begin{description}
\item[Name] TNoControlADJP
\item[Task] To prune the embedded closed ADJPPROP in the VERBP 
(and which does not need control) 
till only a predicate remains. The embedded subject is turned into the object 
of the higher clause.
\item[File] english:TC\_ControlRules1.mrule (mrules32.mrule)
\item[Semantics] --
\item[Example] x1 do consider [he old] $\rightarrow$ x1 do consider he old
\item[Remarks] 
\end{description}

\vspace{1 cm}
\begin{description}
\item[Name]   TOblObjControlPrepADJP
\item[Task] To verify that  the subject (PRO-)variable of an embedded 
open ADJPPROP which is the object of a prepositional phrase in the VERBP is 
identical with the object variable in the higher clause 
and to delete the embedded subject, pruning the ADJPPROP till only a predicate 
remains in the PREPP. 
\item[File] english:TC\_ControlRules1.mrule (mrules32.mrule)
\item[Semantics] --
\item[Example] x1 do regard x2 as [x2 clever] $\rightarrow$ x1 do regard x2 as 
clever
\item[Remarks] 
\end{description}

\vspace{1 cm}
\begin{description}
\item[Name] TOblSubjControlPrepADJP
\item[Task] To verify that  the subject (PRO-)variable of an embedded 
open ADJPPROP which is the object of a prepositional phrase in the VERBP is 
identical with the subject variable in the higher clause (there is no object)
and to delete the embedded subject, pruning the ADJPPROP till only a predicate 
remains in the PREPP. 
\item[File] english:TC\_ControlRules1.mrule (mrules32.mrule)
\item[Semantics] --
\item[Example] ?
\item[Remarks] 
\end{description}

\vspace{1 cm}
\begin{description}
\item[Name]   TNoControlPrepADJP
\item[Task] To prune the embedded closed ADJPPROP which is the object of a 
prepositional phrase in the VERBP 
(and which does not need control) 
till only a predicate remains in the PREPP. The embedded subject is turned 
into the object of the higher clause.
\item[File] english:TC\_ControlRules1.mrule (mrules32.mrule)
\item[Semantics] --
\item[Example] ? do strike x2 as [he pompous] $\rightarrow$ do strike x2 he as 
pompous
\item[Remarks] 
\end{description}

\vspace{1 cm}
\begin{description}
\item[Name] TOblObjControlNP
\item[Task] To verify that  the subject (PRO-)variable of an embedded 
open NPPROP in the VERBP is identical with the object variable in the 
higher clause 
and to delete the embedded subject, pruning the NPPROP till only a predicate 
remains. 
\item[File] english:TC\_ControlRules1.mrule (mrules32.mrule)
\item[Semantics] --
\item[Example] x1 do elect x2 [x2 President] $\rightarrow$ x1 do elect x2 
President
\item[Remarks] 
\end{description}

\vspace{1 cm}
\begin{description}
\item[Name]   TOblSubjControlNP
\item[Task] To verify that  the subject (PRO-)variable of an embedded 
open NPPROP (not an identificational) in the VERBP is 
identical with the subject variable in the higher clause (there is no object)
and to delete the embedded subject, pruning the NPPROP till only a predicate 
remains.
\item[File] english:TC\_ControlRules1.mrule (mrules32.mrule)
\item[Semantics] --
\item[Example] x1 do cost x3 [x1 \$6] $\rightarrow$ x1 do cost x3 \$6
\item[Remarks] 
\end{description}

\vspace{1 cm}
\begin{description}
\item[Name] TNoControlNP
\item[Task] To prune the embedded closed NPPROP (not an identificational) in 
the VERBP (which does not need control) 
till only a predicate remains. The embedded subject is turned into the object 
of the higher clause.
\item[File] english:TC\_ControlRules1.mrule (mrules32.mrule)
\item[Semantics] --
\item[Example] do become [she President] $\rightarrow$ do become she President
\item[Remarks] 
\end{description}

\vspace{1 cm}
\begin{description}
\item[Name]   TOblObjControlPrepNP
\item[Task] To verify that  the subject (PRO-)variable of an embedded 
open NPPROP which is the object of a 
prepositional phrase in the VERBP is identical with the object variable in the 
higher clause 
and to delete the embedded subject, pruning the NPPROP till only an object NP
remains in the PREPP. 
\item[File] english:TC\_ControlRules1.mrule (mrules32.mrule)
\item[Semantics] --
\item[Example] x1 do mistake x2 for [x2 somebody else] $\rightarrow$ x1 do 
mistake x2 for somebody else
\item[Remarks] 
\end{description}

\vspace{1 cm}
\begin{description}
\item[Name] TOblSubjControlPrepNP
\item[Task] To verify that  the subject (PRO-)variable of an embedded 
open NPPROP which is the object of a prepositional phrase in the VERBP is 
identical with the subject variable in the higher clause (there is no object)
and to delete the embedded subject, pruning the NPPROP till only an object NP
remains in the PREPP. 
\item[File] english:TC\_ControlRules1.mrule (mrules32.mrule)
\item[Semantics] --
\item[Example] x1 do come across as [x1 a nice person] $\rightarrow$ x1 do come 
across as a nice person
\item[Remarks] 
\end{description}

\vspace{1 cm}
\begin{description}
\item[Name]   TNoControlPrepNP
\item[Task] To prune the embedded closed NPPROP which is the object of a 
prepositional phrase in the VERBP (and which does not need control) 
till only an object NP remains in the PREPP. The embedded subject is turned 
into the object of the higher clause.
\item[File] english:TC\_ControlRules1.mrule (mrules32.mrule)
\item[Semantics] --
\item[Example] ?
\item[Remarks] 
\end{description}

\vspace{1 cm}
\begin{description}
\item[Name] TNoControlVERBP
\item[Task] To prune the embedded closed VERBPPROP in the VERBP 
(which does not need control) 
till only a predicate remains. The embedded subject is turned into the object 
of the higher clause.
\item[File] english:TC\_ControlRules2.mrule (mrules31.mrule)
\item[Semantics] --
\item[Example] x1 do have [the house built] $\rightarrow$ x1 do have the 
house built
\item[Remarks] This is the only control rule for VERBPPROPs.
\end{description}

\vspace{1 cm}
\begin{description}
\item[Name]   TOblObjControlPREPP
\item[Task] To verify that  the subject (PRO-)variable of an embedded 
(locative, directional or other complement) open PREPPPROP in the VERBP is 
identical with the object variable in the higher clause 
and to delete the embedded subject, pruning the PREPPPROP till only a PREPP
remains. 
\item[File] english:TC\_ControlRules2.mrule (mrules31.mrule)
\item[Semantics] --
\item[Example] x1 do put x2 [x2 on the table] $\rightarrow$ x1 do put x2 on the 
table
\item[Remarks] 
\end{description}

\vspace{1 cm}
\begin{description}
\item[Name] TOblSubjControlPREPP
\item[Task] To verify that  the subject (PRO-)variable of an embedded 
(locative, directional or other complement) open PREPPPROP in the VERBP is 
identical with the subject variable in the higher clause (there is no object)
and to delete the embedded subject, pruning the PREPPPROP till only a PREPP
remains. 
\item[File] english:TC\_ControlRules2.mrule (mrules31.mrule)
\item[Semantics] --
\item[Example] x1 jump [x1 off the fence] $\rightarrow$ x1 jump off the fence
\item[Remarks] 
\end{description}

\vspace{1 cm}
\begin{description}
\item[Name]   TNoControlPREPP
\item[Task] To prune the embedded closed PREPPPROP in the VERBP 
(which does not need control) 
till only a PREPP remains. The embedded subject is turned into the object 
of the higher clause.
\item[File] english:TC\_ControlRules2.mrule (mrules31.mrule)
\item[Semantics] --
\item[Example] x1 do get [the two sides round the table] $\rightarrow$ x1 do 
get the two sides round the table
\item[Remarks] 
\end{description}

\vspace{1 cm}
\begin{description}
\item[Name] TOblObjControlPrepPREPP
\item[Task] To verify that  the subject (PRO-)variable of an embedded 
open PREPPPROP which is the object of a prepositional phrase in the VERBP 
is identical with the object variable in the higher clause 
and to delete the embedded subject, pruning the PREPPPROP till only a predicate 
remains in the PREPP. 
\item[File] english:TC\_ControlRules2.mrule (mrules31.mrule)
\item[Semantics] --
\item[Example] x1 do regard x2 as [x2 without principles] $\rightarrow$ x1 do 
regard x2 as without priciples
\item[Remarks] This is the only control rule for PREPPPROPs in a PREPP.
\end{description}

\vspace{1 cm}
\begin{description}
\item[Name]   TOblObjControlADVP
\item[Task] To verify that  the subject (PRO-)variable of an embedded 
(locative, directional or other complement) open ADVPPROP in the VERBP is 
identical with the object variable in the higher clause 
and to delete the embedded subject, pruning the ADVPPROP till only an ADVP
remains. 
\item[File] english:TC\_ControlRules2.mrule (mrules31.mrule)
\item[Semantics] --
\item[Example] ?
\item[Remarks] 
\end{description}

\vspace{1 cm}
\begin{description}
\item[Name] TOblSubjControlADVP
\item[Task] To verify that  the subject (PRO-)variable of an embedded 
(locative, directional or other complement) open ADJPPROP in the VERBP is 
identical with the subject variable in the higher clause (there is no object)
and to delete the embedded subject, pruning the ADVPPROP till only an ADVP
remains. 
\item[File] english:TC\_ControlRules2.mrule (mrules31.mrule)
\item[Semantics] --
\item[Example] ? x1 swim [x1 southwards] $\rightarrow$ x1 swim southwards
\item[Remarks] 
\end{description}

\vspace{1 cm}
\begin{description}
\item[Name]   TNoControlADVP
\item[Task] To prune the embedded (locative, directional or other complement) 
closed ADVPPROP in the VERBP (which does not need control) 
till only an ADVP remains. The embedded subject is turned into the object 
of the higher clause.
\item[File] english:TC\_ControlRules2.mrule (mrules31.mrule)
\item[Semantics] --
\item[Example] ? x1 do get [them home] $\rightarrow$ x1 do get them home
\item[Remarks] 
\end{description}

\vspace{1 cm}
\begin{description}
\item[Name] TOblObjControlComplSent
\item[Task] To verify that  the subject (PRO-)variable of an embedded 
open SENTENCE in the VERBP is identical with the object or indirect object 
variable in the higher clause (depending on the controller the head 
verb takes) and to delete the embedded subject. The embedded SENTENCE is not 
pruned.
\item[File] english:TC\_ControlRules2.mrule (mrules31.mrule)
\item[Semantics] --
\item[Example] x1 do persuade x2 [x2 to see a doctor] $\rightarrow$ x1 do 
persuade x2 [to see a doctor]
\item[Remarks] 
\end{description}

\vspace{1 cm}
\begin{description}
\item[Name]  TOblSubjControlComplSent
\item[Task] To verify that  the subject (PRO-)variable of an embedded 
open SENTENCE in the VERBP or in a predicate under the VERBP is 
identical with the subject variable in the higher clause (in accordance with 
the subject controller the head verb takes)
and to delete the embedded subject. The embedded SENTENCE is not pruned.
\item[File] english:TC\_ControlRules2.mrule (mrules31.mrule)
\item[Semantics] --
\item[Example] x1 do decide [x1 to leave] $\rightarrow$ x1 do decide [to 
leave]
\item[Remarks] 
\end{description}

\vspace{1 cm}
\begin{description}
\item[Name] TOblPrepobjControlComplSent
\item[Task] To verify that  the subject (PRO-)variable of an embedded 
open SENTENCE in the VERBP is 
identical with the object variable of a prepobj in the higher clause (in 
accordance with the prepobj controller the head verb takes)
and to delete the embedded subject. The embedded SENTENCE is not pruned.
\item[File] english:TC\_ControlRules2.mrule (mrules31.mrule)
\item[Semantics] --
\item[Example] x1 do sign to x2 [x2 to stop] $\rightarrow$ x1 do sign to x2 
[to stop]
\item[Remarks] 
\end{description}

\vspace{1 cm}
\begin{description}
\item[Name]   TOblByobjControlComplSent
\item[Task] To verify that  the subject (PRO-)variable of an embedded 
open SENTENCE in the VERBP is 
identical with the variable in the by-phrase in the higher clause 
(in accordance with the subject controller the head verb takes)
and to delete the embedded subject. The embedded SENTENCE is not pruned.
\item[File] english:TC\_ControlRules2.mrule (mrules31.mrule)
\item[Semantics] --
\item[Example] -- be decided [x1 to leave at six] by x1 $\rightarrow$ -- be 
decided [to leave at six] by x1
\item[Remarks] This rule is not for verbs that only have obligatory control. It 
only serves as a  `narrow reading' control rule for verbs with NOC.
\end{description}

\vspace{1 cm}
\begin{description}
\item[Name]   TOblObjControlPrepSent
\item[Task] To verify that  the subject (PRO-)variable of an embedded 
open SENTENCE which is the object of a prepositional phrase in the VERBP is 
identical with the object variable in the higher clause 
and to delete the embedded subject. The embedded SENTENCE is not pruned.
\item[File] english:TC\_ControlRules3.mrule (mrules23.mrule)
\item[Semantics] --
\item[Example] ? 
\item[Remarks] Probably, all sentences that can be the object of a PREPP are 
sentential NPs. These are dealt with in separate rules (see below).
\end{description}

\vspace{1 cm}
\begin{description}
\item[Name] TOblSubjControlPrepSent
\item[Task] To verify that  the subject (PRO-)variable of an embedded 
open SENTENCE which is the object of a prepositional phrase in the VERBP is 
identical with the subject variable in the higher clause (there is no object)
and to delete the embedded subject. The embedded SENTENCE is not pruned.
\item[File] english:TC\_ControlRules3.mrule (mrules23.mrule)
\item[Semantics] --
\item[Example] ? x1 talk (to x3) about [how x1 to prevent that] $\rightarrow$ x1 
talk (to x3) about [how to prevent that]
\item[Remarks] If this example is considered to be wrong (because the 
embedded constituent is thought to be a sentential NP and not a SENTENCE), the 
rule may be superfluous: no other examples could be found.
\end{description}

\vspace{1 cm}
\begin{description}
\item[Name]   TOblObjControlExtrapSent
\item[Task] To verify that  the subject (PRO-)variable of an extraposed
open SENTENCE is identical with the object or indirect object variable in the 
higher clause (depending on the controller the head verb takes)
and to delete the embedded subject. The SENTENCE is not pruned.
\item[File] english:TC\_ControlRules3.mrule (mrules23.mrule)
\item[Semantics] --
\item[Example] x1 do persuade x2 [x2 to leave] $\rightarrow$ x1 do persuade x2 
[to leave]
\item[Remarks] Although the rule is reversible, it should not be necessary in 
generation, since only finite complement sentences are allowed to be 
extraposed. Infinite subject sentences which are extraposed ({\em -- be unclear 
how to do this\/}) should not go through this rule because the head verb is not 
specified for controller. Special rules for NOC of subject sentences still have 
to be written. See also the following rule for complements extraposed from an 
embedded predicate.
\end{description}

\vspace{1 cm}
\begin{description}
\item[Name] TOblSubjControlExtrapSent
\item[Task] To verify that  the subject (PRO-)variable of an extraposed
open SENTENCE is 
identical with the subject variable in the higher clause (in accordance with 
the subject controller the head verb takes, or in case the head verb is the 
copula {\em be\/})
and to delete the embedded subject. The extraposed SENTENCE is not pruned.
\item[File] english:TC\_ControlRules3.mrule (mrules23.mrule)
\item[Semantics] --
\item[Example] x1 promise x3 [x1 to read the book] $\rightarrow$ x1 promise x3 
[to read the book] \\
x1 be able [x1 to swim] $\rightarrow$ x1 be able [to swim]
\item[Remarks] Although the rule is reversible, it should not be necessary in 
generation, since only finite complement sentences are allowed to be 
extraposed. See what was said in the rule above.
\end{description}

\vspace{1 cm}
\begin{description}
\item[Name]   TOblPrepobjControlExtrapSent
\item[Task] To verify that  the subject (PRO-)variable of an extraposed
open SENTENCE is 
identical with the variable of a prepobj or byobj in the higher clause (in 
accordance with 
the controller the head verb takes; byobj antecedents are allowed only for 
verbs that do not have obligatory control. See TOblByobjControlComplSent), 
and to delete the embedded subject. The extraposed SENTENCE is not pruned.
\item[File] english:TC\_ControlRules3.mrule (mrules23.mrule)
\item[Semantics] --
\item[Example] x1 signed to x3 [x3 to stop] $\rightarrow$ x1 signed to x3 
[to stop] \\
-- be decided by x1 [x1 to leave at six] $\rightarrow$ -- be decided by x1 [to 
leave at six]
\item[Remarks] Although the rule is reversible, it should not be necessary in 
generation, since only finite complement sentences are allowed to be 
extraposed. See what was said in the two rules above.
\end{description}

\vspace{1 cm}
\begin{description}
\item[Name]   TOblSubjControlOpenIngNP
\item[Task] To verify that  the subject (PRO-)variable of an embedded 
open SENTENCE which is the head of a sentential NP in the VERBP is 
identical with the subject variable in the higher clause (there is no object)
and to delete the embedded subject. The NP and SENTENCE are not pruned.
\item[File] english:TC\_ControlRules3.mrule (mrules23.mrule)
\item[Semantics] --
\item[Example] x1 do burst out [x1 singing] $\rightarrow$ x1 do burst out 
[singing]
\item[Remarks] This is the only control rule for complement sentential NPs
\end{description}

\vspace{1 cm}
\begin{description}
\item[Name] TOblObjControlPrepOpenIngNP
\item[Task] To verify that  the subject (PRO-)variable of an embedded 
open SENTENCE which is the head of a sentential NP that is the object of a 
prepositional phrase in the VERBP is identical with the object variable in the 
higher clause 
and to delete the embedded subject. The NP and SENTENCE are not pruned.
\item[File] english:TC\_ControlRules3.mrule (mrules23.mrule)
\item[Semantics] --
\item[Example] x1 do talk x2 out of [x2 killing himself] $\rightarrow$ x1 do 
talk x2 out of [killing himself] 
\item[Remarks] 
\end{description}

\vspace{1 cm}
\begin{description}
\item[Name]   TOblSubjControlPrepOpenIngNP
\item[Task] To verify that  the subject (PRO-)variable of an embedded 
open SENTENCE which is the head of a sentential NP that is the object of a 
prepositional phrase in the VERBP is identical with the subject variable in the 
higher clause (there is no object)
and to delete the embedded subject. The NP and SENTENCE are not pruned.
\item[File] english:TC\_ControlRules3.mrule (mrules23.mrule)
\item[Semantics] --
\item[Example] x1 do succeed in [x1 breaking open the door] $\rightarrow$ x1 do 
succeed in [breaking open the door]
\item[Remarks] 
\end{description}

\end{description}

\newpage
\subsection{TC\_FinControl}

\begin{description}
\item[Kind] Obligatory Transformation Class
\item[Task] To control and spell out the subject of a finite embedded 
sentence in generation, in case of a translation from an infinite open 
sentence (which has no overt subject, so there is no lexical expression in IL 
either) to a finite sentence  (which can never be open, and hence needs a 
subject). 
The rules work in generation only since in analysis, the structure 
is analysed as an ordinary finite sentence, without control. No rules have 
been written yet for other controllers than subject.


\vspace{1 cm}
\begin{description}
\item[Name] TNoFinControl
\item[Task] Vacuous rule, to let structures which do not call for control of an 
embedded finite sentence subject pass this transformation class. The rule also 
passes all structures with a modal in the main sentence, 
since there control has already
been dealt with (but see the second footnote to TSuperdeixisASdaptation2).
\item[File] english:TC\_ControlRules3.mrule (mrules23.mrule)
\item[Semantics] --
\item[Example] x1 do say [that they had left] 
\item[Remarks] 
\end{description}

\vspace{1 cm}
\begin{description}
\item[Name]  TExtrapFinControl
\item[Task] To verify that the the subject (PRO-)variable of an extraposed
finite SENTENCE is identical with the subject variable in the higher clause
(in accordance with the subject controller of the head VERB, or in case the 
head verb is the copula {\em be\/}), 
and to spell it out as a pronoun (PERSPRO or INDEFPRO `one') corresponding 
with the higher subject in number, person, sex and gender.
\item[File] english:TC\_ControlRules3.mrule (mrules23.mrule)
\item[Semantics] --
\item[Example] x1 do think [that x1 will be swimming] $\rightarrow$ x1 do 
think [that he/we/they etc will be swimming] (as translation of Dutch {\em x1 
denkt te zullen zwemmen\/})
\item[Remarks] 
\end{description}

\vspace{1 cm}
\begin{description}
\item[Name] TComplFinControl
\item[Task] To verify that the the subject (PRO-)variable of an embedded 
finite SENTENCE in the VERBP or in a predicate in the VERBP 
is identical with the subject variable in the higher clause
(in accordance with the subject controller of the head verb, or in case the 
head verb is the copula {\em be\/}), 
and to spell it out as a pronoun (PERSPRO or INDEFPRO `one') corresponding 
with the higher subject in number, person, sex and gender.
\item[File] english:TC\_ControlRules3.mrule (mrules23.mrule)
\item[Semantics] --
\item[Example] x1 do [think [x1 will be swimming]] $\rightarrow$ x1 do 
[think [he/we/they etc will be swimming]] (as translation of Dutch {\em x1 
denkt te zullen zwemmen\/})
\item[Remarks] This rule may well be superfluous: in generation, finite 
subsentences always receive the conjunction {\em that\/}, and (thus) are always 
extraposed.
\end{description}

\vspace{1 cm}
\begin{description}
\item[Name] TPrepFinControl
\item[Task] To verify that the the subject (PRO-)variable of an embedded 
finite SENTENCE which is the object of a prepositional or by-phrase in the 
VERBP is identical with the subject variable in the higher clause
(in accordance with the subject controller of the head verb)
and to spell it out as a pronoun (PERSPRO or INDEFPRO `one') corresponding 
with the higher subject in number, person, sex and gender.
\item[File] english:TC\_ControlRules3.mrule (mrules23.mrule)
\item[Semantics] --
\item[Example] ? x1 wondered at [what x1 should do now] $\rightarrow$ 
x1 wondered at [what he/we/they etc should do now] (as translation of 
Dutch {\em x1 vroeg zich af wat nu te doen\/})
\item[Remarks] 
\end{description}

\end{description}

\newpage
\subsection{TC\_NOC}

\begin{description}
\item[Kind] Iterative Transformation Class
\item[Task] To determine partial identity between the subject (PRO-)variable 
of an embedded open SENTENCE (if there is any) with a controller in the higher
clause, (in accordance with the attribute {\bf oblcontrol}, which must be {\em 
nooblcontrol\/}), and to delete the embedded subject. In case the controller is 
an EMPTY, there also is NOC. For more information, see the document on Control 
by Jan Odijk (to appear). 

This transformation class only deals with the `wide reading' of verbs with 
non-obligatory control. The `narrow reading', with full identity of subject and 
antecedent, is covered by the rules for obligatory control. For more comment, 
see what was said in the previous rule class, TC\_ControlRules.

No rules have been written for NOC yet in English, so this transformation class 
is NOT yet present in the control expression. 

\end{description}

\newpage
\subsection{RC\_VPAdv}

\begin{description}
\item[Kind] Iterative Rule Class, preceded by Obligatory Transformation
\item[Task] To insert  vp-modifying adverbials. This is done by 
substituting the clause into an argument slot of an open ADVPPROP, and deleting 
this prop, reorganising the resulting structure. 
The treatment of VPAdvs resembles the approach outlined in doc.\ 
155, but the rule class has been removed from the VerbppropFormation subgrammar 
(where other adverbials are introduced) to its own place in the current 
subgrammar. For more information, see the document on the Treatment of Adverbs 
in Rosetta3, by Jan Odijk (to appear).

The rule class is ordered crucially before EMPTY substitution, since EMPTYs may 
be subjected to the restrictions of agvpadvs, and before TC\_ObjectOKrules, 
since the restrictions hold for the agent, not just for any syntactic subject.

For AgVPAdvs, only those which have the value `false' for Q-status can be dealt 
with adequately. Since these ADVs are not introduced by means of a variable, 
they cannot have any bearing on scope in the current system. A sentence like 
{\em He didn't leave intentionally\/}, where the scope of {\em intentionally\/} 
is ambiguous (wide scope covering {\em not\/} and translating to {\em Hij kwam 
opzettelijk 
niet\/} vs.\ narrow scope and translating to {\em Hij kwam niet opzettelijk\/}) 
cannot be dealt with.

It is assumed that manner adverbs can be dealt with by the current rule class. 
However, adverbials like {\em thus\/} and {\em in this way\/} behave more like 
sentadvs. Perhaps they should be treated differently.

\vspace{1 cm}
\begin{description}
\item[Name] TNoVPAdvs
\item[Task] Obligatory transformation, ensuring the proper application  of this 
iterative (and hence 
optional) rule class in analysis: there may be no VPAdvs left after this 
rule class. The transformation might also have been formulated as a 
(speed) filter.
\item[File] english:RC\_VPAdv.mrule (mrules93.mrule)
\item[Semantics] --
\item[Example] 
\item[Remarks] 
\end{description}

\vspace{1 cm}
\begin{description}
\item[Name] RVPAdv
\item[Task] To insert a vp-modifying adverbial. This is done by 
substituting the clause into the subject slot of an open ADVPPROP, and deleting 
this prop. 
\item[File] english:RC\_VPAdv.mrule (mrules93.mrule)
\item[Semantics]
\item[Example] be hit x1 by x2 + x4 hard $\rightarrow$ be hit x1 hard by x2
\item[Remarks] A special function is used to calculate the deixis of the 
ADVPPROP in analysis. No precautions have been taken yet to assure that if the 
adverb is a superlative, a definite article is introduced ({\em He sang the 
loudest\/}), nor is there a rule to add {\em much\/} if the adverb is {\em very
\/}: {\em He loves her very much\/}.
\end{description}

\vspace{1 cm}
\begin{description}
\item[Name] RAgVPAdvSubj
\item[Task] To insert a subject oriented vp-modifying adverbial, i.e.\ a vp-
modifying adverbial which poses restrictions on the agent of the clause.
This is done by 
substituting the clause into the complement slot of an open ADVPPROP of which 
the subject variable is identical to the subject variable of the clause.
The subjects may be NP/CNVARs or EMPTYVARs. The prop is deleted. 
\item[File] english:RC\_VPAdv.mrule (mrules93.mrule)
\item[Semantics]
\item[Example] x1 do paint x2 + x1 carefully x4 $\rightarrow$ x1 do paint x2 
carefully
\item[Remarks] A special function is used to calculate the deixis of the 
ADVPPROP in analysis. The rule uses a parameter to assure that the correct 
clause is substituted for (needed because the rule class is iterative: there 
may be more agvpadvs in one clause).
\end{description}

\vspace{1 cm}
\begin{description}
\item[Name] RAgVPAdvByobj
\item[Task] To insert a subject oriented vp-modifying adverbial, i.e.\ a vp-
modifying adverbial which poses restrictions on the agent of the clause, in 
case this agent has been moved to a by-phrase.
This is done by 
substituting the clause into the complement slot of an open ADVPPROP of which 
the subject variable is identical to the object variable of a by-phrase in 
the clause.
The subject/byobject may be NP/CNVARs or EMPTYVARs. The prop is deleted. 
\item[File] english:RC\_VPAdv.mrule (mrules93.mrule)
\item[Semantics]
\item[Example] -- be painted x2 by x1 + x1 carefully x4 $\rightarrow$ -- be 
painted x2 carefully by x1
\item[Remarks] A special function is used to calculate the deixis of the 
ADVPPROP in analysis. The rule uses a parameter to assure that the correct 
clause is substituted for (needed because the rule class is iterative: there 
may be more agvpadvs in one clause).
\end{description}

\vspace{1 cm}
\begin{description}
\item[Name]   RAgVPAdvObj
\item[Task] To insert a subject oriented vp-modifying adverbial, i.e.\ a vp-
modifying adverbial which poses restrictions on the agent of the clause, in 
case this agent is the object of an ergative verb.
This is done by 
substituting the clause into the complement slot of an open ADVPPROP of which 
the subject variable is identical to the object variable in the clause.
The subject/object may be NP/CNVARs or EMPTYVARs. The prop is deleted. 
\item[File] english:RC\_VPAdv.mrule (mrules93.mrule)
\item[Semantics]
\item[Example] -- do come in x1 + x1 carefully x4 $\rightarrow$ -- do
come in carefully 
\item[Remarks] A special function is used to calculate the deixis of the 
ADVPPROP in analysis. The rule uses a parameter to assure that the correct 
clause is substituted for (needed because the rule class is iterative: there 
may be more agvpadvs in one clause).
\end{description}

\end{description}

\newpage
\subsection{RC\_EMPTYsubst}

\begin{description}
\item[Kind] Iterative Rule Class
\item[Task] To substitute an abstract EMPTY for an EMPTYVAR; in the output
model, neither the EMPTY nor the VAR occur, and (in 
special cases) a tree corresponding to the strings {\em One\/} or {\em They\/}
is introduced. Several EMPTYs 
have been defined, each with their own number, person and gender. This is
necessary because they may be antecedent for a reflexive or an embedded 
variable (NOC). 

As in all other substitution rules, use is made of a special system parameter, 
LEVEL, to identify the variable to be substituted for by specifying the depth 
of its embedding in the derivation tree. The value of this parameter is 
calculated by the system itself, and not set explicitly in the rules.

In generation, the rules of this (iterative) class are applied in the order 
dictated by the source language (this is passed on by 
means of the parameter). In analysis, a kind of 
substitution order condition is imposed by demanding that no EMPTYVARs may 
occur to the right of the one that is introduced now (hence, EMPTYVARs are 
introduced from left to right). 

This rule class is ordered crucially after TC\_NOC (EMPTYs may be antecedents 
to embedded variables), and before other substitution rules (to account for 
the narrow scope EMPTYs have with respect to other quantificational phrases).

\vspace{1 cm}
\begin{description}
\item[Name] RByEMPTYSubst
\item[Task] To substitute the abstract `oneEMPTY' for an EMPTYVAR in a 
byphrase, and to delete the byphrase.
\item[File] english:RC\_EMPTYsubst.mrule (mrules30.mrule)
\item[Semantics]
\item[Example] -- be built x2 x4=in 1918 by EMPTYVAR + oneEMPTY $\rightarrow$ 
-- be built x2 x4=in 1918
\item[Remarks] 
\end{description}

\vspace{1 cm}
\begin{description}
\item[Name] RObjEMPTYSubst
\item[Task] To substitute the abstract `zeroEMPTY' for an object EMPTYVAR, 
and to delete it.
\item[File] english:RC\_EMPTYsubst.mrule (mrules30.mrule)
\item[Semantics]
\item[Example] x1 do eat EMPTYVAR + zeroEMPTY $\rightarrow$ x1 do eat
\item[Remarks] 
\end{description}

\vspace{1 cm}
\begin{description}
\item[Name]   RIndObjEMPTYSubst
\item[Task] To substitute the abstract `zeroEMPTY' for an indirect object 
EMPTYVAR, and to delete it.
\item[File] english:RC\_EMPTYsubst.mrule (mrules30.mrule)
\item[Semantics]
\item[Example] x1 do cost EMPTYVAR \$6 + zeroEMPTY $\rightarrow$ x1 do cost \$6
\item[Remarks] 
\end{description}

\vspace{1 cm}
\begin{description}
\item[Name] RTheySubjEMPTYSubst
\item[Task] To substitute the abstract `theyEMPTY' for a subject EMPTYVAR of a 
finite clause, and to spell it out as the perspro {\em they\/}\footnote{This 
rule has now been removed from the control expression, since a different 
treatment of EMPTYs (see the footnote to RPrepEmptySubst1) also implied a 
different treatment of the generalized 
personal pronouns {\em je\/} and {\em ze\/} in Dutch, and the indefinite 
pronoun {\em men\/}. These are now considered 
basic expressions, translating to English {\em you, they\/} and {\em one\/} 
respectively. Hence, the constructions in which an EMPTY subject will be found 
are 
now restricted to sentences like {\em Er werd gedanst\/}. For these cases, the 
next rule, ROneSubjEMPTYSubst, seems sufficient.}.
\item[File] english:RC\_EMPTYsubst.mrule (mrules30.mrule)
\item[Semantics]
\item[Example] EMPTYVAR do shoot x2 + theyEMPTY $\rightarrow$ They do shoot x2 
(They shoot horses, don't they?)
\item[Remarks] It is assumed that this `generalized' {\em they\/} is the same 
as an ordinary perspro. The rule should work in generation only: in analysis, 
it is assumed that the pronoun {\em they\/} is not generic. Should severe 
restrictions be found under which the pronoun can only be generic (or is 
probably generic), two different lexical entries {\em they\/} should be made 
for English, in the same way as for Dutch. 

No rules have been written to introduce a generalized {\em you\/} as subject, 
since the Dutch {\em je\/} will always be analysed as a non-generic pronoun as 
well, translating to English {\em you\/}.
\end{description}

\vspace{1 cm}
\begin{description}
\item[Name]   ROneSubjEMPTYSubst
\item[Task] To substitute the abstract `oneEMPTY' for a subject EMPTYVAR of a 
finite clause\footnote{By the time this document was approved, the rule had 
been extended to allow for infinite sentences as well, to allow for 
{\em He let one hear the music\/}. If such a translation is 
considered incorrect English, the condition may be added again, since there now 
is an alternative path for Dutch AanActives through the passive rules: 
{\em He let the music be 
heard\/}.}, and to spell it out as the indefpro {\em one\/}.
\item[File] english:RC\_EMPTYsubst.mrule (mrules30.mrule)
\item[Semantics]
\item[Example] EMPTYVAR be dancing + oneEMPTY $\rightarrow$ One be dancing 
(One was dancing)
\item[Remarks] Probably, more sophisticated rules are needed to spell out the 
subject. In the example given above ({\em Er werd gedanst\/}), 
one would prefer something like 
{\em People\/}. As translation of the Dutch sentence {\em Er werd ge\-klopt\/} 
the form {\em someone\/} seems better ({\em Someone knocked\/}), and for 
{\em Er werd hard gelachen in de kamer naast ons\/} a translation with {\em 
they\/} is quite conceivable. The factors determining the choice have not 
been discovered yet.
\end{description}

\vspace{1 cm}
\begin{description}
\item[Name] RPrepEMPTYSubst1
\item[Task] To substitute the abstract `zeroEMPTY' for an EMPTYVAR that is the 
object of the first or only prepositional phrase the verb takes, and to delete 
it.
\item[File] english:RC\_EMPTYsubst.mrule (mrules30.mrule)
\item[Semantics]
\item[Example] x1 do tell x2 EMPTYVAR + zeroEMPTY $\rightarrow$ x1 do tell x2
(He told the whole story)
\item[Remarks] A difficulty exists when the zeroEMPTY in this rule is to be a 
translation of the Dutch menEMPTY, which happens when a prepobj is translated 
not into a Dutch (prep)obj, but into a Dutch AanObj. Specifically, this occurs 
in the idiom {\em laten zien - show: (Hij liet) het boek aan menEMPTY zien\/} $
\rightarrow$ {\em (He showed) the book (to) zeroEMPTY\/}.
This problem has not been solved yet: it seems strange to make an AanObj a 
zeroEMPTY, since translations with {\em one\/} are perhaps possible : 
{\em hij liet aan EMPTY de solo horen\/} $\rightarrow$ {\em He 
let one hear the solo\/}, or {\em He let the solo be heard (by oneEMPTY)\/}. On 
the other hand, the current rule must work with zeroEMPTYs, because they are 
the translation of Dutch (ind)object and prepobj zeroEMPTYs. Finally,
it is impossible to map zeroEMPTY and oneEMPTY onto each other, 
since that would cause difficulties in Spanish\footnote{This problem has now 
been solved, and all EMPTYs are mapped onto each other. Thus, even the idiom 
can be dealt with now.}.
\end{description}

\vspace{1 cm}
\begin{description}
\item[Name]   RPrepEMPTYSubst2
\item[Task] To substitute the abstract `zeroEMPTY' for an EMPTYVAR that is the 
object of the second prepositional phrase the verb takes, and to delete it.
\item[File] english:RC\_EMPTYsubst.mrule (mrules30.mrule)
\item[Semantics]
\item[Example] x1 do talk to x3 EMPTYVAR + zeroEMPTY $\rightarrow$ x1 do talk to 
x3 (He talked to his father)
\item[Remarks] 
\end{description}

\vspace{1 cm}
\begin{description}
\item[Name] REMPTYinADJPsubst
\item[Task] To substitute the abstract `zeroEMPTY' for an EMPTYVAR that is the 
object of a prepositional phrase or a 
locative argument going with an adjective predicate, and to delete it.
\item[File] english:RC\_EMPTYsubst.mrule (mrules30.mrule)
\item[Semantics]
\item[Example] x1 do be afraid (of) EMPTYVAR + zeroEMPTY $\rightarrow$ x1 do 
be afraid 
\item[Remarks] 
\end{description}

\end{description}

\newpage
\subsection{TC\_ObjectOKrules}

\begin{description}
\item[Kind] Obligatory Transformation Class
\item[Task] To provide a sentence with a subject if there is not one yet, 
either by raising one, or by spelling out a dummy subject.

This rule class is ordered crucially after TC\_ControlRules: the controller 
specified by the verb must still be in its original position when the control 
rules apply, and may not have moved to e.g.\ a higher sentence. Also, 
it is easier when the embedded 
subject of an XPPROP has already been made an object of the higher clause 
before it is raised to become subject of the higher clause (for raising 
the subject of a SENTENCE, a special ObjectOKrule exists). 
The rule class is also ordered crucially after RC\_VPAdv, since 
the restrictions 
posed by the agvpadv hold for the agent of the verb, not just for any 
surface subject.

\vspace{1 cm}
\begin{description}
\item[Name] TSubjOK
\item[Task] Vacuous transformation, to let clauses which already have a 
subject pass this transformation class. In case there is no modal, the 
structure should be either active or not contain a raising verb or a verb that 
is not a case-assigner. Structures with modals are always allowed to pass.
\item[File] english:TC\_ObjectOKrules.mrule (mrules29.mrule)
\item[Semantics] --
\item[Example] x1 do be afraid 
\item[Remarks] 
\end{description}

\vspace{1 cm}
\begin{description}
\item[Name] TObjToSubjRaising
\item[Task] To turn the object of a subjectless clause into a subject. This is 
only allowed when the clause is passive, or when the head verb is ergative and 
not a case-assigner.
\item[File] english:TC\_ObjectOKrules.mrule (mrules29.mrule)
\item[Semantics] --
\item[Example] \mbox{}
-- do escape x2 x1 $\rightarrow$ x1 do escape x2 \\
-- be killed x2 by x1 $\rightarrow$ x2 be killed by x1
\item[Remarks] For passives, the object must be adjacent to the verb (plus its 
particle); for ergatives, an indirect object may intervene. In contrast to 
Dutch, English knows no restrictions on the definiteness or genericity of the 
object.
\end{description}

\vspace{1 cm}
\begin{description}
\item[Name]   TSubjToSubjRaising
\item[Task] To raise the subject of an embedded sentence in the VERBP to be the 
subject of the higher clause. This is 
only allowed when the clause is passive, or when the head verb is a raising 
verb and not a case-assigner.
\item[File] english:TC\_ObjectOKrules.mrule (mrules29.mrule)
\item[Semantics] --
\item[Example] \mbox{}
-- do seem [it to cost a fortune] $\rightarrow$ it do seem [to cost a fortune]\\
-- be believed [he to be the best doctor in town] $\rightarrow$ he be believed 
[to be the best doctor in town]
\item[Remarks] 
\end{description}

\vspace{1 cm}
\begin{description}
\item[Name] TIndobjToSubjRaising
\item[Task] To turn the indirect object of a subjectless clause into a subject. 
This is only allowed when the clause is passive, and the direct object is not a 
simple perspro.
\item[File] english:TC\_ObjectOKrules.mrule (mrules29.mrule)
\item[Semantics] --
\item[Example] -- be given x3 x2 $\rightarrow$ x3 be given x2 (He was given a 
book)
\item[Remarks] The indirect object must be adjacent to the verb (plus its 
particle).
\end{description}

\vspace{1 cm}
\begin{description}
\item[Name]   TPrepobjToSubjRaising
\item[Task] To turn the object of a prepositional phrase in a subjectless 
clause into a subject, leaving the PREP stranded.
This is only allowed when the clause is passive. The prepobj may also be part 
of an idiom.
\item[File] english:TC\_ObjectOKrules.mrule (mrules29.mrule)
\item[Semantics] --
\item[Example] \mbox{}
-- be looked at x2 $\rightarrow$ x2 be looked at \\
-- be caught sight of x2 $\rightarrow$ x2 be caught sight of
\item[Remarks] The prepobj must be adjacent to the verb (plus its 
particle); an NP object may also intervene (needed for idioms). It is well 
possible that more relations and categories should be allowed between the verb 
and 
the prepobj, e.g.\ ADVs (The plan was thought better of), or even all non-VAR 
categories.
\end{description}

\vspace{1 cm}
\begin{description}
\item[Name] TThereSubjInsertion
\item[Task] To insert the dummy subject {\em There\/} when there is no subject.
This is only allowed when the head verb is the copula {\em be\/}, and in 
analysis for ergatives. The direct object must be indefinite, and not 
universally quantifying.
\item[File] english:TC\_ObjectOKrules.mrule (mrules29.mrule)
\item[Semantics] --
\item[Example] \mbox{}
-- do be some doubts about the plan $\rightarrow$ There do be some doubts about 
the plan\\
-- do come x1 $\leftarrow$ There do come x1 (There came a man)
\item[Remarks] The number of the subject NP is set at the same number the object 
NP has.
\end{description}

\vspace{1 cm}
\begin{description}
\item[Name]   TItSubjInsertion
\item[Task] To insert the dummy subject {\em It\/} when there is no subject.
This is only allowed when there is no object, or when the head verb is not 
ergative, and the clause is active (i.e.\ when the subject has been extraposed).
There must be an embedded or extraposed sentence present.
\item[File] english:TC\_ObjectOKrules.mrule (mrules29.mrule)
\item[Semantics] --
\item[Example] \mbox{}
-- do surprise x2 [to see you there] $\rightarrow$ It do surprise x2 [to see 
you there]\\
-- have escaped x2 [that you would be late] $\rightarrow$ It have escaped x2 
[that you would be late]\\
-- be discovered [that you were late] $\rightarrow$ It be discovered [that you 
were late]
\item[Remarks] 
\end{description}

\end{description}

\newpage
\subsection{TC\_ConjSentControl}

\begin{description}
\item[Kind] Iterative Transformation Class, surrounded by two Obligatory 
Filters
\item[Task] To perform obligatory control for infinitival adverbial sentences.
Since this class is ordered crucially after TC\_ObjectOKrules (the controller 
for the adverbial sentence is supposed to be the surface subject, not some
argument in the clause corresponding with the controller 
type of their verb),
it cannot be incorporated into the other Control Rules. Also, the 
transformation class is iterative, since there may be more adverbial sentences 
going with a clause.

No rules have been written yet for other controllers than subjects. It is not 
clear whether they are needed.

In doc.\ 150, this transformation class was described as occurring in the next 
subgrammar, CLAUSEtoSENTENCE, after the reflexive spelling rules and just 
before the shiftrules.

\vspace{1 cm}
\begin{description}
\item[Name] FPreConjSentControl
\item[Task] Speed filter, to prevent embedded adverbial subsentences from 
passing this iterative (and hence optional) class in analysis without receiving 
a subject.
\item[File] english:TC\_NOC.mrule (mrules94.mrule)
\item[Example] 
\item[Remarks] 
\end{description}

\vspace{1 cm}
\begin{description}
\item[Name] TSubjControlConjSent
\item[Task] To verify that the subject VAR 
of a leftdislocrel adverbial subsentence 
is identical to the subject of the main clause, and to delete it.
\item[File] english:TC\_NOC.mrule (mrules94.mrule)
\item[Semantics] --
\item[Example] [In order x1 to pay], x1 do draw x2 $\rightarrow$ 
[In order to pay], x1 do draw x2 (In order to pay, he drew his wallet)
\item[Remarks] 
\end{description}

\vspace{1 cm}
\begin{description}
\item[Name] TSubjControlFinalConjSent
\item[Task] To verify that the subject VAR of a postsentadvrel or tempadvrel 
adverbial subsentence 
is identical to the subject of the main clause, and to delete it.
\item[File] english:TC\_NOC.mrule (mrules94.mrule)
\item[Semantics] --
\item[Example] x1 do draw x2 [in order x1 to pay] $\rightarrow$ x1 do draw x2 
[in order to pay] (He drew his wallet in order to pay)
\item[Remarks] 
\end{description}

\vspace{1 cm}
\begin{description}
\item[Name]   TSubjControlConjPrepNP
\item[Task] To verify that the subject VAR under a leftdislocrel adverbial 
sentential 
PREPP (i.e.\ a PREPP with as its object a sentence-like NP heading a SENTENCE)
is identical to the subject of the main clause, and to delete it.
\item[File] english:TC\_NOC.mrule (mrules94.mrule)
\item[Semantics] --
\item[Example] [Without x1 intending to pay], x1 do draw x2 $\rightarrow$ 
[Without intending to pay], x1 do draw x2 (Without intending to pay, he drew 
his wallet)
\item[Remarks] 
\end{description}

\vspace{1 cm}
\begin{description}
\item[Name] TSubjControlFinalConjPrepNP
\item[Task] To verify that the subject VAR under a postsentadvrel or tempadvrel 
adverbial sentential 
PREPP (i.e.\ a PREPP with as its object a sentence-like NP heading a SENTENCE)
is identical to the subject of the main clause, and to delete it.
\item[File] english:TC\_NOC.mrule (mrules94.mrule)
\item[Semantics] --
\item[Example] x1 do draw x2 [before x1 realising it] $\rightarrow$ 
x1 do draw x2 [before realising it] (He drew his wallet before realising it)
\item[Remarks] 
\end{description}

\vspace{1 cm}
\begin{description}
\item[Name] FPostConjSentControl
\item[Task] Obligatory Filter, to prevent embedded adverbial subsentences from 
passing this iterative (and hence optional) class in generation without 
deletion of the embedded subject.
\item[File] english:TC\_NOC.mrule (mrules94.mrule)
\item[Example] 
\item[Remarks] 
\end{description}

\end{description}

\newpage
\subsection{TC\_SubjVerbAgr}

\begin{description}
\item[Kind] Obligatory Transformation Class
\item[Task] To adapt the number and person of the verb outside the VERBP (if 
there is one) to the number and person of the subject. For this transformation 
class it is crucial that there always is a subject present; hence, it is 
ordered after the ObjectOKrules, and also expects that in the proposition 
substitution rule for subject sentences that are put in leftdislocrel a dummy 
subject THAT was inserted (see RLdislocSubjSentSubst).

\vspace{1 cm}
\begin{description}
\item[Name] TNoAgreement
\item[Task] To let clauses that do not need agreement (infinitives) pass this 
transformation class 
\item[File] english:TC\_SubjVerbAgr.mrule (begin mrules28.mrule)
\item[Semantics] --
\item[Example] x1 go to x2 ( I want to go to the cinema)
\item[Remarks] 
\end{description}

\vspace{1 cm}
\begin{description}
\item[Name] TAgreement
\item[Task] To adapt the number and person of the verb outside the VERBP 
to the number and person of the subject.
\item[File] english:TC\_SubjVerbAgr.mrule (begin mrules28.mrule)
\item[Semantics] --
\item[Example] x1$_{\frac{sing}{3}}$ do$_{\frac{[]}{[]}}$ go to x2
$\rightarrow$ x1$_{\frac{sing}{3}}$ do$_{\frac{[sing]}{[3]}}$ go to x2
(He went to the cinema)
\item[Remarks] 
\end{description}

\end{description}

\newpage
\subsection{TC\_ThatDel}

\begin{description}
\item[Kind] Obligatory Transformation Class
\item[Task] To delete the dummy subject {\em That\/} if one has been inserted 
in Rldisloc\-Subj\-SentSubst (see above). 

This transformation class is ordered crucially after the agreement rules, since 
the dummy subject is needed to indicate that the verb must become a third 
person singular. The class was not mentioned in doc.\ 150.

\vspace{1 cm}
\begin{description}
\item[Name] TNoThatDeletion
\item[Task] Vacuous transformation, to let clauses in which no dummy 
{\em That\/} was inserted pass this transformation class.
\item[File] english:TC\_SubjVerbAgr.mrule (end mrules28.mrule)
\item[Semantics] --
\item[Example] There does be x1; That does be x1; x1 do see x2
\item[Remarks] 
\end{description}

\vspace{1 cm}
\begin{description}
\item[Name] TThatDeletion
\item[Task] To delete the dummy subject {\em That\/} that had been inserted 
in RldislocSubjSentSubst as `singular number carrier' for the agreement rules. 
\item[File] english:TC\_SubjVerbAgr.mrule (end mrules28.mrule)
\item[Semantics] --
\item[Example] [That he went away so early] THAT does surprise x2 $\rightarrow$ 
[That he went away so early] does surprise x2
\item[Remarks] 
\end{description}

\end{description}

\newpage
\subsection{TC\_CaseAssignment}

\begin{description}
\item[Kind] Iterative Transformation Class, in which every (Iterative)
Transformation is surrounded by two matching Obligatory Filters. The PreFilters 
are NOT speed filters here, since cased NPVARs and CNVARs will not be stopped 
anywhere anymore! For a slightly more extensive discussion, see the section on 
Prefilters in doc.\ 367, {\em 
General Remarks on writing M-rules\/}, by Jan Odijk.
\item[Task] To assign case to all NP(VAR)s and CNVAR(s) in the current clause
(excluding the subject NPVAR of an infinitive) and to the subject of an 
embedded infinitive clause. In the transformations, use is made of a special 
percolation 
function for case assignment, which is able to find out whether there is a 
case-demanding constituent (e.g.\ PERSPRO or WHPRO) present under the NP, 
and to mark that for case too (see the document on LSMRUQUO). The need for this 
function was already mentioned in doc.\ 155.

The CaseAssignment rules crucially follow the ObjectOK rules, since only after 
that class the arguments are in the position that decides their case.

\vspace{1 cm}
\begin{description}
\item[Name] FPreObjCaseAssign
\item[Task] To assure the proper application of TObjCaseAssign or 
TErgBeCaseAssign in analysis.
\item[File] english:FC\_CaseAssignment.mrule (mrules26.mrule)
\item[Example] 
\item[Remarks]
\end{description}

\vspace{1 cm}
\begin{description}
\item[Name] TObjCaseAssign
\item[Task] To assign oblique case to NPs, NPVARs and CNVARs that are object or 
indirect object (or predicate, in case they come from the NPPROP grammar), 
given that the main verb is a case-assigner.
\item[File] english:TC\_CaseAssignment.mrule (mrules27.mrule)
\item[Semantics] -- 
\item[Example] \mbox{}
x1 did elect x2$_{[]}$ President$_{[]}$ $\rightarrow$ 
x1 did elect x2$_{[accusative]}$ President$_{[accusative]}$
x1 do see x2$_{[]}$ $\rightarrow$ x1 do see x2$_{[accusative]}$
\item[Remarks] The rule model is formulated in such a way that only one 
ordering of case assignment is tried: from left to right. Alternative to this 
rule is TErgBeCaseAssign (see below).
\end{description}

\vspace{1 cm}
\begin{description}
\item[Name] TErgBeCaseAssign
\item[Task] To assign case to NPs and NP/CNVARs that are object or 
indirect object (or predicate, in case they come from the NPPROP grammar), 
given that the main verb is the verb {\em be\/} or another non-caseassigning 
verb. The case assigned is oblique for non-wh NP(VAR)s, and 
nominative for other NP(VAR)s and for CNVARs.
\item[File] english:TC\_CaseAssignment.mrule (mrules27.mrule)
\item[Semantics] -- 
\item[Example] x1 does be x2=who$_{[]}$ $\rightarrow$ x1 does be 
x2=who$_{[nominative]}$ \\
(Who are you; $^{*}$Whom are you) \\
x1 do be x2$_{[]}$ $\rightarrow$ x1 do be x2$_{[accusative]}$ (You are me)
\item[Remarks] The rule will probably have to be  extended in analysis to cover 
non-wh nominatives too: {\em It was he\/}. In case the predicate is modified, 
nominative case is even obligatory: {\em It was he who came to the door\/}. 
This cannot be handled yet. 

This rule is an alternative to TObjCaseAssign (see above).
\end{description}

\vspace{1 cm}
\begin{description}
\item[Name] FPostObjCaseAssign
\item[Task] To assure the proper application of TObjCaseAssign or 
TErgBeCaseAssign in generation.
\item[File] english:FC\_CaseAssignment.mrule (mrules26.mrule)
\item[Example] 
\item[Remarks]
\end{description}

\vspace{1 cm}
\begin{description}
\item[Name] FPreSubjCaseAssign
\item[Task] To assure the proper application of TSubjCaseAssign in analysis.
\item[File] english:FC\_CaseAssignment.mrule (mrules26.mrule)
\item[Example] 
\item[Remarks]
\end{description}

\vspace{1 cm}
\begin{description}
\item[Name] TSubjCaseAssign
\item[Task] To assign nominative case to the subject NP(VAR) or CNVAR of finite 
sentences. The case of the subject of infinitives is left undetermined, since 
it should be assigned by either Exceptional Case Marking (see below) or by a 
prepositional conjunction (see the moodrule for a Fortoinf)
\item[File] english:TC\_CaseAssignment.mrule (mrules27.mrule)
\item[Semantics] -- 
\item[Example] x1$_{[]}$ did walk $\rightarrow$ x1$_{[nominative]}$ did walk
\item[Remarks]
\end{description}

\vspace{1 cm}
\begin{description}
\item[Name] FPostSubjCaseAssign
\item[Task] To assure the proper application of TSubjCaseAssign in generation.
\item[File] english:FC\_CaseAssignment.mrule (mrules26.mrule)
\item[Example]
\item[Remarks]
\end{description}

\vspace{1 cm}
\begin{description}
\item[Name] FPrePrepCaseAssign
\item[Task] To assure the proper application of TPrepCaseAssign in analysis.
\item[File] english:FC\_CaseAssignment.mrule (mrules26.mrule)
\item[Example] 
\item[Remarks]
\end{description}

\vspace{1 cm}
\begin{description}
\item[Name] TPrepCaseAssign
\item[Task] To assign oblique case to NP(VAR)s and CNVARs that are governed by 
a prep.
\item[File] english:TC\_CaseAssignment.mrule (mrules27.mrule)
\item[Semantics] -- 
\item[Example] x1 did speak to x2$_{[]}$ $\rightarrow$ x1 did speak to 
x2$_{[accusative]}$ (whom did you speak to?)
\item[Remarks] Following the words {\em but, except, than\/} and {\em as\/} 
both nominative and oblique case should be accepted when the constituent is in 
subject position or a subject complement: {\em He is more 
intelligent than she/her; Nobody except she/her can solve our problems\/}. This 
cannot be dealt with yet.
\end{description}

\vspace{1 cm}
\begin{description}
\item[Name] FPostPrepCaseAssign
\item[Task] To assure the proper application of TPrepCaseAssign in generation.
\item[File] english:FC\_CaseAssignment.mrule (mrules26.mrule)
\item[Example] 
\item[Remarks]
\end{description}

\vspace{1 cm}
\begin{description}
\item[Name] FPreExceptCaseAssign
\item[Task] To assure the proper application of TExceptCaseAssign in analysis.
\item[File] english:FC\_CaseAssignment.mrule (mrules26.mrule)
\item[Example] 
\item[Remarks]
\end{description}

\vspace{1 cm}
\begin{description}
\item[Name] TExceptCaseAssign
\item[Task] To assign oblique case to the subject (which may have been shifted) 
of an embedded infinite sentence. This subject did not receive case in an 
earlier cycle (see TSubjCaseAssign).
\item[File] english:TC\_CaseAssignment.mrule (mrules27.mrule)
\item[Semantics] -- 
\item[Example] x1 do believe [he$_{[]}$ to be a fool] $\rightarrow$ x1 do 
believe [him$_{[accusative]}$ to be a fool] 
\item[Remarks]
\end{description}

\vspace{1 cm}
\begin{description}
\item[Name] FPostExceptCaseAssign
\item[Task] To assure the proper application of TExceptCaseAssign in generation.
\item[File] english:FC\_CaseAssignment.mrule (mrules26.mrule)
\item[Example]
\item[Remarks]
\end{description}

\vspace{1 cm}
\begin{description}
\item[Name] FPrePrepExceptCaseAssign
\item[Task] To assure the proper application of TPrepExceptCaseAssign in 
analysis.
\item[File] english:FC\_CaseAssignment.mrule (mrules26.mrule)
\item[Example] 
\item[Remarks]
\end{description}

\vspace{1 cm}
\begin{description}
\item[Name] TPrepExceptCaseAssign
\item[Task] To assign oblique case to the subject (which may have been shifted) 
of an embedded infinite sentence, in case this sentence is in a PREPP.
\item[File] english:TC\_CaseAssignment.mrule (mrules27.mrule)
\item[Semantics] -- 
\item[Example] x1 do count on [he$_{[]}$ to be there] $\rightarrow$ 
x1 do count on [him$_{[accusative]}$ to be there]
\item[Remarks]
\end{description}

\vspace{1 cm}
\begin{description}
\item[Name] FPostPrepExceptCaseAssign
\item[Task] To assure the proper application of TPrepExceptCaseAssign in 
generation.
\item[File] english:FC\_CaseAssignment.mrule (mrules26.mrule)
\item[Example] 
\item[Remarks]
\end{description}

\vspace{1 cm}
\begin{description}
\item[Name] FPreXppObjCaseAssign
\item[Task] To assure the proper application of TxppObjCaseAssign in analysis.
\item[File] english:FC\_CaseAssignment.mrule (mrules26.mrule)
\item[Example] 
\item[Remarks]
\end{description}

\vspace{1 cm}
\begin{description}
\item[Name] TXppObjCaseAssign
\item[Task] To assign oblique case to a direct or indirect object that occurs 
under a predicate in the VP (while the main verb is {\em be\/}).
\item[File] english:TC\_CaseAssignment.mrule (mrules27.mrule)
\item[Semantics] -- 
\item[Example] x1 do be [in [fashion$_{[]}$]] $\rightarrow$ 
x1 do be [in [fashion$_{[accusative]}$]]
\item[Remarks] 
\end{description}

\vspace{1 cm}
\begin{description}
\item[Name] FPostXppObjCaseAssign
\item[Task] To assure the proper application of TxppObjCaseAssign in 
generation.
\item[File] english:FC\_CaseAssignment.mrule (mrules26.mrule)
\item[Example] 
\item[Remarks]
\end{description}

\vspace{1 cm}
\begin{description}
\item[Name] FPreXppPrepCaseAssign
\item[Task] To assure the proper application of TxppPrepCaseAssign in analysis.
\item[File] english:FC\_CaseAssignment.mrule (mrules26.mrule)
\item[Example] 
\item[Remarks]
\end{description}

\vspace{1 cm}
\begin{description}
\item[Name] TXppPrepCaseAssign
\item[Task] To assign oblique case to the object of a (VAR)PREPP that occurs 
under a predicate in the VP (while the main verb is {\em be\/}).
\item[File] english:TC\_CaseAssignment.mrule (mrules27.mrule)
\item[Semantics] -- 
\item[Example] x1 do be [fond [of [she$_{[]}$]]] $\rightarrow$ 
x1 do be [fond [of [her$_{[accusative]}$]]]
\item[Remarks]
\end{description}

\vspace{1 cm}
\begin{description}
\item[Name] FPostXppPrepCaseAssign
\item[Task] To assure the proper application of TxppPrepCaseAssign in 
generation.
\item[File] english:FC\_CaseAssignment.mrule (mrules26.mrule)
\item[Example] 
\item[Remarks]
\end{description}

\end{description}

\newpage
\subsection{TC\_ArgReflSpelling}

\begin{description}
\item[Kind] Iterative Transformation Class, surrounded by two Obligatory 
Filters
\item[Task] To establish identity between two NPVARs, and to spell 
out the second one as a reflexive, 
agreeing with the first NPVAR in person, sex and number.

\vspace{1 cm}
\begin{description}
\item[Name] FPreArgReflSpelling
\item[Task] To assure the proper application of all ArgReflSpelling 
Transformations, i.e.\ there 
may be no reflexive left if the main verb is not reflexive. Like most 
prefilters, this is a speed filter (the analysis with a reflexive would stop 
only at the Reflexive Spelling rules, just preceding the pattern rules).
\item[File] english:TC\_ArgReflSpelling.mrule (mrules18.mrule)
\item[Example] 
\item[Remarks]
\end{description}

\vspace{1 cm}
\begin{description}
\item[Name] TArgReflSpelling1
\item[Task] To spell out the second of two identical NPVARs as a reflexive, 
agreeing with the first NPVAR in person, sex and number. The first NPVAR is in 
S, the second in the VP.
\item[File] english:TC\_ArgReflSpelling.mrule (mrules18.mrule)
\item[Semantics] -- 
\item[Example] x1 did give x1 x2 $\rightarrow$ x1 did give myself/herself etc.\ 
x2
\item[Remarks] In the current implementation, it is assumed that the antecedent 
for the reflexive is always the NPVAR which is just preceding it. No 
alternative rules have been formulated yet, neither for other positions of the 
second NPVAR (in a PREPP or in S), nor for cases where there are more than two 
identical NPVARs (esp.\ when there will also be a reciprocal NP)

Also note that in analysis, reflexives that are caused by an inherently 
reflexive verb are also covered by this rule. This wrong path will only stop at 
the Reflexive Spelling Transformations, just before the pattern rules. Perhaps 
the pre-filter may be extended to demand that in case of a reflexive verb, 
there is no structure which has no reflexive left at all. 
\end{description}

\vspace{1 cm}
\begin{description}
\item[Name] FPostArgReflSpelling
\item[Task] To assure the proper application of TArgReflSpelling1
\item[File] english:TC\_ArgReflSpelling.mrule (mrules18.mrule)
\item[Example] 
\item[Remarks]
\end{description}

\item[Remarks] In doc.\ 155 pp.\ 6-9 several problems with respect to 
reflexives are mentioned. They concern a.o.\ other realisations of 
identical NPVARs 
(as pronoun, possadj, or genitive), and have not been solved yet. Another 
problem that has not been solved is the case where an NPVAR will be substituted 
for by an NP containing an element which should have been spelled out as a 
reflexive ({\em x1 heard x2} $\rightarrow$ {\em x1 heard [stories about x1] }).
There is a rule ordering clash there: the shiftrules must follow the ArgRefl 
rules (otherwise the antecedent for the reflexive-to-be is hard to 
define), but they must precede the substitution rules to account for scope, so 
when shift is taken into consideration, reflexive spelling cannot be delayed 
until after substitution.
For these cases, a second application of the Argument Reflexive Spelling 
transformations, after the substition rules, seems necessary. In docs.\ 150 and 
155, 
where it was assumed that Reflexive and Reciprocal Spelling were part of the 
CLAUSEtoSENTENCE subgrammar, this is already indicated.

Note that the Argument Reflexive rules should also precede the Particle 
Spelling rules, since these refer to reflexives for surface ordering of 
particles and NPs.

\end{description}

\newpage
\subsection{RC\_ReciprocalSpelling}

\begin{description}
\item[Kind] Iterative Rule Class, preceded by Obligatory Filter
\item[Task] To identify an NPVAR as being related to another NP/CNVAR 
antecedent (by means of the attribute {\bf index} of the antecedent, the 
value of which must agree with the value of the parameter {\em antecedent\/}
going with the rule), and to substitute a reciprocal pronoun for this NPVAR. 
For a more extensive discussion of the problems with this rule class (which 
caused the treatment of reciprocals to deviate from what was proposed in doc.\ 
155), see its description for Dutch in doc.\ 314, {\em The Dutch XPPROPtoCLAUSE 
subgrammar\/}.
Use is made of a special function, {\em ReciproAnt\/}, which checks whether the 
antecedent has the correct number (usu.\ plural, but singular is also possible 
in specific cases).

Next to the ordinary LEVEL parameter (used for all substitution rules) and the 
{\em antecedent\/} parameter, the rules use a third parameter, {\em 
multitude\/}, to collapse the models for two different reciprocals: 
{\em each other\/} (two; or more, but in 
analysis only) and {\em one another\/} (more than two).

\vspace{1 cm}
\begin{description}
\item[Name] FpreRecipro1
\item[Task] To assure correct application of the ReciproSubst rules. This speed 
filter takes care that no reciprocal pronoun remains after the rule class has 
been applied in analysis.
\item[File] english:RC\_ReciprocalSpelling.mrule (mrules17.mrule)
\item[Example]
\item[Remarks]
\end{description}

\vspace{1 cm}
\begin{description}
\item[Name] RObjReciproSubst
\item[Task] To substitute an NP heading a reciprocal pronoun for an NPVAR that 
is identical with it, given that there also is some antecedent subject 
NP/CNVAR with the correct index.
\item[File] english:RC\_ReciprocalSpelling.mrule (mrules17.mrule)
\item[Semantics] 
\item[Example] x1 did give x3 x2 + each other $\rightarrow$ x1 did give each 
other x2
\item[Remarks] No rules have been written yet for other antecedents than 
subjects.
\end{description}

\item[Remarks] \mbox{}
\begin{itemize}
\item In doc.\ 150, it was assumed that the Reflexive and Recipro 
Spelling rules would be in the next subgrammar, CLAUSEtoSENTENCE. Again, as for 
reflexives, it may be necessary to apply this rule class again after the 
substitution rules, to account for constructions like {\em They heard stories 
about each other\/}.
\item No use has been made yet of the value of the attribute {\bf reflexive} 
of the main verb: if that value is {\em reciprocal\/}, this means that the verb 
allows deletion of the reciprocal pronoun, as in {\em They kissed\/}. It is not 
clear whether deletion should take place here or in a separate transformation 
class.
\end{itemize}
\end{description}

\newpage
\subsection{TC\_ParticleHop}

\begin{description}
\item[Kind] Obligatory Transformation Class
\item[Task] To put a particle, if any, in its correct surface position.
In generation, the particle must precede any NP except reflexives and personal 
pronouns. This is in fact a simplification: esp.\ when the NP is hardly
modified, the particle might as well follow it: {\em I called the man up\/}. 
Since it would be difficult to formulate an appropriate `heaviness'-condition 
for the NP, it is simply never allowed to precede the particle in generation;
in analysis, however, both orders are accepted. Also note that no substitution 
has taken place yet; hence, it is not possible here to know how `heavy' the NP 
that is going to be substituted for the NPVAR in fact is.

The formulation of the rules and the connected TC\_ParticleSpelling (see doc.\ 
310 on the English VerbppropFormation) is not as was announced in 
doc.\ 153. See doc.\ 310 for comment. Note that 
the transformation class crucially follows Argument Reflexive Spelling, since 
it refers to (any kind of) reflexives for the surface ordering of particle and 
NP.

\vspace{1 cm}
\begin{description}
\item[Name] TNoPartHop
\item[Task] Vacuous transformation, to let clauses in which there is no 
particle or in which the
particle already is in its correct position (not preceding a perspro 
or a reflexive) pass this transformation class.
\item[File] english:TC\_ParticleHop.mrule (mrules25.mrule)
\item[Semantics] -- 
\item[Example] x1 give up x2=my ambitions
\end{description}

\vspace{1 cm}
\begin{description}
\item[Name] TOblPartHop
\item[Task] To hop the particle over any reflexive or personal pronoun 
immediately following it.
\item[File] english:TC\_ParticleHop.mrule (mrules25.mrule)
\item[Semantics] -- 
\item[Example] x1 do give up x2=it $\rightarrow$ x1 do give x2=it up\\
x1 did turn in himself $\rightarrow$ x1 did turn himself in
\end{description}

\vspace{1 cm}
\begin{description}
\item[Name] TOptPartHop
\item[Task] Extra rule to allow for a `deviant' surface order of particle and 
NP in analysis: the particle may follow an NP that is not a reflexive or 
personal pronoun.
\item[File] english:TC\_ParticleHop.mrule (mrules25.mrule)
\item[Semantics] -- 
\item[Example] x1 do give up x2=the boy $\leftarrow$ x1 do give x2=the boy up
\end{description}

\end{description}

\end{document}
