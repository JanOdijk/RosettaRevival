\documentstyle{Rosetta}
\begin{document}
   \RosTopic{Rosetta3.Software}
   \RosTitle{System to Create Retrograde Vocabularies}
   \RosAuthor{Jeroen Medema}
   \RosDocNr{190}
   \RosDate{\today}
   \RosStatus{Informal}
   \RosSupersedes{-}
   \RosDistribution{Software, Harm Smit}
   \RosClearance{Project}
   \RosKeywords{Retrograde vocabulary, System}
   \MakeRosTitle
%
%
\setlength{\parskip}{2mm}
\setlength{\parindent}{0mm}

\section{Target}
\subsection{Definition}
The target of the system is the creation of the retrograde vocabularies for
the Van Dale dictionaries N~--~N \cite{st:groot}, N~--~E \cite{ma:groot1}, 
and E~--~N \cite{ma:groot2}. That is, somewhat more formal, given the 26 files 
(one per letter) of one of these dictionaries, make one file (in which the 
words are retrograde ordered) that, without any conversion, is printable in 
a special form as defined in subsection 1.3.
\subsection{Input Description}
A description of the N~--~N dictionary is \cite{sm:descr} while the 
other two dictionaries aren't described up to now. What is to be known here, is,
however, only a small part of an entire description of these dictionaries. 
First, the 
entryword -that's what we're looking for- is always on the same place in the 
entry. Also, there are, in the entryword, some special codes which need to be 
converted to other characters in the printed version. These conversion codes 
are listed in table~1. 
\begin{center}
\begin{tabular}{|l|l|l|}\hline
 Code             
  & $\alpha \in$       
    & Conversion\\ \hline
 {\tt `}$\alpha$
  & \{a, e, i, o, u, A, E, I, O, U\}  
    & \`{a}, \`{e}, \`{\i}, \`{o}, \`{u}, \`{A}, \`{E}, \`{I}, \`{O}, \`{U}\\
 \~{}$\alpha$
  & \{a, e, i, o, u, A, E, I, O, U\}  
    & \'{a}, \'{e}, \'{\i}, \'{o}, \'{u}, \'{A}, \'{E}, \'{I}, \'{O}, \'{U}\\
 \^{}$\alpha$     
  & \{a, e, i, o, u, A, E, I, O, U\}  
    & \^{a}, \^{e}, \^{\i}, \^{o}, \^{u}, \^{A}, \^{E}, \^{I}, \^{O}, \^{U}\\
 {\tt "}$\alpha$  
  & \{a, e, i, o, u, A, E, I, O, U\}  
    & \"{a}, \"{e}, \"{\i}, \"{o}, \"{u}, \"{A}, \"{E}, \"{I}, \"{O}, \"{U}\\
 {[ps]o[pr]}&          & $^{\rm o}$ \\
 /[h9]      &          & -\\
 //         &          & /\\
 /9         &          & $|$\\
 +$\alpha$1 & \{n, N\} & \~{n}, \~{N}\\
 +$\alpha$4 & \{a, A\} & \aa, \AA \\
 +$\alpha$8 & \{c, C\} & \c{c}, \c{C}\\
 +\%        &          & $^{\rm o}$ \\
 +{\tt '}   &          & {\tt `}\\
 +;         &          & \\ \hline
\end{tabular}\vspace{1mm}

Table 1.
\end{center}
\subsection{Output Description}
\subsubsection{Sorting Order}
 The sorting order of the words is retrograde (which 
 means that they are not ordered with respect to the first characters of a 
 word -as in normal dictionaries- but with respect to the last ones). 
 The ordering is done by smallest first; that is: first ``~'', then 
  ``-'', next ``.'', followed by the letters.
 The following constraints apply to the equivalence of words:
 \begin{enumerate}
  \item characters with an accent (as e.g. \^{e}, \~{n}, \aa, \"{o}) are 
  considered to be equal to the same character without the accent.
  \item words containing the string ``(-)'' are considered to be equal to the 
  same word without that string.
  \item words containing {\em other} characters than letters, ``-'', ``~'', and 
  ``.'' are considered to be equal to the same word without those characters.
  \item the upper and lower case form of a letter are considered to be equal.
 \end {enumerate}
\subsubsection{Printing Format}
 The words are split up in twenty-eight parts: for every letter one part and 
 two additional for the ``-'' and ``.'' (no word ends with a space). Each part 
 starts at a new page. The output more or less looks like the retrograde 
 dictionary of the `Grote Van Dale' \cite{ni:retro}. It is defined follows:
  \begin{itemize}
   \item every page consist of sixty rows, of which the last one is empty, and
         four columns.
   \item in a column, every word is right justified.
   \item if a word has a length greater than 19 (which is the pagewidth -80-
         divided by the number of columns minus 1 for the separation of the 
         columns) it has to be split into two or more consecutive (that is, in 
         one column) printed strings.
   \item if a word (because of its length) does not fit anymore in the column
         a new column is to be used (the same applies for pages).
   \item the order of reading is column first, row second, page last.
   \item the headwords (as stored in the file) are converted from a coded form 
         (see table 1) to printable strings more or less as in \cite{st:groot}.
  \end{itemize}
Appendix A contains an example of two output pages as generated making the 
retrograde wordlist of the E~--~N dictionary.
\section{System Description}
The target of the system was reached by writing several programs, each partly 
doing the job.
First is told which programs are needed and what they do, and after that the 
coupling of them into the system is described.
\subsection{Programs}
For this system three programs were made: SkipXtra (sx), AddReverse (ar), and 
PrintRetro (pr). Two system programs were used to append and sort files: append 
and sort. A description of each of these programs is listed below.
\begin{description}
  \item [SkipXtra] This program will produce one outputfile which is the 
        stripped version of a Van Dale dictionary inputfile. It will only 
        contain the entryword of the entire entry. In the outputfile one word 
        is placed per line. The skipping of the ``+;''-string, which markes the 
        stress in the entryword, is done with this program too.
  \item [AddReverse] This program will with the given input (which is described 
        below) generate output (also described below) which also contains the 
        reversed words (at the beginning of each line). The input consists of 
        26 files which each contain the entrywords as in the Van Dale 
        dictionary on disk but without the string ``+;''. The output will 
        consist of 26 files which contain lines of the following form: first 
        the reversed entryword, then four spaces followed by the normal entry. 
        The reversed entryword is, however, totaly stripped of special 
        characters (e.g. {\tt`},{\tt"},\~{},\^{},+,\%,(-), etcetera). 
  \item [Append] This system command is used to append 26 files onto one, 
        initially empty, file.
  \item [Sort] This system program is used to take as input one file and make a 
        new version of it by sorting it on the first characters of a line.
  \item [PrintRetro]
        This program will make a printfile of the given input. The input 
        consist of one file which contains the (retrograde) sorted list of 
        entries preceeded by the reversed entry and four spaces. The output is 
        one file in which the words are ordered as described in 1.3.2. Every 
        word is made printable (in normal text) by converting the special codes 
        (as defined in table~1).
\end{description}
\subsection{Coupling}
With the previous defined programs the creation of a retrograde wordlist is 
very simple: one runs the program SkipXtra 26 times to get the stripped files. 
From these 26 new files the program AddReverse will generate files with the 
reversed word at front. With Append one file is made by appending all 26 
AddReverse-files which gives one large file. This large file is sorted using the
system program Sort (which is really quick: 3 minutes CPU time for a file with 
90.000 entries). The newly created file is treated by PrintRetro which makes 
the printable file.

The scheme in figure~1 represents the total structure of the system. Double 
arrows represent programs which were especially made for this system, while 
single arrows depict programs which were already available in the operating 
system.
\begin{figure}
 \[ \left.
     \begin{array}{ccc} 
        A_{text} & \stackrel{sx}{\Longrightarrow} & SXA_{word}\\ 
        B_{text} & \stackrel{sx}{\Longrightarrow} & SXB_{word}\\ 
        C_{text} & \stackrel{sx}{\Longrightarrow} & SXC_{word}\\ 
        \vdots   & \stackrel{sx}{\Longrightarrow} & \vdots  \\ 
        Z_{text} & \stackrel{sx}{\Longrightarrow} & SXZ_{word}
     \end{array}
    \right\}
     \stackrel{ar}{\Longrightarrow}
    \left\{
     \begin{array}{cc} 
        ARA_{retro} &             \\
        ARB_{retro} & \searrow    \\
        ARC_{retro} & \rightarrow \\
        \vdots    & \nearrow    \\
        ARZ_{retro}
    \end{array}
   \right.
   Abc_{app} \rightarrow Abc_{sort} \stackrel{pr}{\Longrightarrow} Abc_{prt}
 \]
 \caption{Outline Retrograde Wordlist System}
\end{figure}
\section{Program Implementation}
 Two of the three programs were very simple to write, viz.\ SkipXtra and 
 AddReverse. The straightforward mechanisms for these programs are described 
 below. The third, PrintRetro, was more difficult but gave no real problems.
 \subsection{SkipXtra}
  This program was written more general and determines its input with the aid 
  of another file named INP.SX\@. In this file, which can be filled manually, 
  the name of the dictionary file which is to be treated is given. The output 
  of the program will be in a file named SX.OUT\@. The algorithm used is very 
  simple: per entry in the inputfile one has to: read it, extract the entryword,
  delete occurences of ``+;'', write this entryword in the outputfile.
 \subsection{AddReverse}
  This program is, if possible, even more simple to write (because of the 
  existing modules). It works on 26 inputfiles named SXA.OUT to SXZ.OUT\@. 
  It generates 26 files named LETTERA.OUT to LETTERZ.OUT\@. The algorithm used 
  is very simple: inputfile SX$\alpha$.OUT is used to make outputfile 
  LETTER$\alpha$.OUT; per line of the inputfile, the line generated for the 
  outputfile is made by reversing the inputline, write this reversed string 
  followed by four spaces and the original inputline.
 \subsection{PrintRetro}
  This program works on an input-file named RETRO.TXT and generates an 
  output-file named RETRO.PRT\@. The output is generated (added to the 
  output-file) per page. A page is internally represented by an array of sixty
  lines (which each are: a string of eighty characters and the current length 
  of that string). A page is made by appending the first column to the 
  initially empty lines. A column is made up out of a sequence of words, each 
  right justified and converted. The mechanism of adding columns is repeated 
  four times (four columns), or until a new ending-letter is found, or until 
  there are no entrywords left. Then, the page is written to the output-file 
  and the lengthes of the page-lines are set to zero.
  
\section{Motivation}
\begin{itemize}
  \item {\bf Why?}\\
        The retrograde dictionaries were made to be used by linguists and 
        dictionary programmers at times a set of words with a special kind of 
        ending is to be known. This especially accounts for the morphological 
        parts of the system.
  \item {\bf Why this way?}\\
        It was, of course, possible to make one program that performs the task
        as described. However, this program would be quite more complex than 
        the total of its parts. It also would require a lot of internal memory
        to store the entrywords of the dictionary and its printed form. Further,
        some intermediate results can be useful for other programs (this 
        especially accounts for the files generated by SkipXtra).
\end{itemize}

\begin{thebibliography}{MMM99m}
 \bibitem[MAR84]{ma:groot2} W. Martin, ed. {\em Groot woordenboek 
  Engels -- Nederlands}, Van Dale Lexicografie, 1984
 \bibitem[MAR86]{ma:groot1} W. Martin, ed. {\em Groot woordenboek 
  Nederlands -- Engels}, Van Dale Lexicografie, 1986
 \bibitem[NIE69]{ni:retro} E.R. Nieuwborg, {\em Retrograde woordenboek van de 
  Nederlandse taal}, Universit\'{e} Catholique de Louvain, 1969
 \bibitem[SME87]{sm:descr} Harm Smit \& Jeroen Medema, {\em Description Van 
  Dale dictionary N~--~N}, {\bf Rosetta report 174}, Philips Research 
  Laboratories, 1987
 \bibitem[STE84]{st:groot} P.G.J. van Sterkenburg, ed. {\em Groot woordenboek 
  hedendaags Nederlands}, Van Dale Lexicografie, 1984
\end{thebibliography}
\newpage
\appendix
\section{Output Examples}
\footnotesize
\begin{verbatim}
              Vedic ai                           mystagogic             graphic 
          Samoyedic r-passenger traffic               logic            -graphic 
          Samoyedic         air traffic              -logic         sciagraphic 
             acidic     tourist traffic            dialogic         paragraphic 
           nuclidic            morbific            analogic         telegraphic 
            juridic             Pacific             illogic        calligraphic 
         glucosidic             pacific       pharmacologic          epigraphic 
            Druidic        transpacific        archaeologic     pseudepigraphic 
            Davidic            specific            geologic       stratigraphic 
            scaldic            specific          teleologic       lexicographic 
           heraldic         subspecific         morphologic       palaeographic 
          Icelandic         conspecific      geomorphologic         ideographic 
          Icelandic       interspecific         glaciologic          geographic 
               .                   .                   .                   .
               .                   .                   .                   .
               .                   .                   .                   .
            melodic           honorific           topologic     historiographic 
          psalmodic           honorific           serologic       physiographic 
          spasmodic           saporific           horologic    encephalographic 
      antispasmodic           vaporific          petrologic    crystallographic 
      antispasmodic           soporific          astrologic         holographic 
             anodic           soporific        climatologic        stylographic 
            parodic          torporific        dermatologic         xylographic 
           episodic            terrific          tautologic         demographic 
           prosodic            horrific           sexologic        anemographic 
          rhapsodic             ossific        ichthyologic        cosmographic 
             bardic            beatific         embryologic        planographic 
          Lombardic          scientific           lethargic        stenographic 
          Sephardic        unscientific            allergic        ethnographic 
             Nordic               Cufic            lysergic        iconographic 
             Nordic               Cufic             georgic        phonographic 
              asdic               Kufic             georgic         monographic 
              ludic               sufic            theurgic       chronographic 
           Talmudic             pelagic           demiurgic        pornographic 
              pudic               magic         dramaturgic         typographic 
          aldehydic               magic        thaumaturgic       polarographic 
            nucleic               magic                chic        micrographic 
        ribonucleic              tragic                chic        hydrographic 
   deoxyribonucleic              tragic           stomachic          orographic 
              oleic          paraplegic           stomachic        chorographic 
          dyspnoeic          paraplegic        amphibrachic        reprographic 
         mythopoeic           strategic             bacchic      spectrographic 
       onomatopoeic              -algic          oligarchic        petrographic 
               -fic           neuralgic            diarchic     cinematographic 
            malefic           nostalgic            anarchic        pictographic 
            benefic              Belgic          hierarchic        photographic 
            traffic           pedagogic           autarchic    telephotographic 
            traffic           demagogic             psychic   astrophotographic 
white-slave traffic            anagogic             psychic       cryptographic 
   carrying traffic          hypnagogic             edaphic        cartographic 
  passenger traffic           paragogic            seraphic        dittographic 
                               isagogic             graphic         autographic 
\end{verbatim}
\newpage
\begin{verbatim}
       tachygraphic               ethic            bicyclic             bucolic 
        polygraphic               ethic           alicyclic             bucolic 
         euthrophic              lithic           epicyclic               Eolic 
            trophic          megalithic        heterocyclic              Aeolic 
      gonadotrophic            Eolithic              Gaelic              Aeolic 
      heterotrophic            eolithic              Gaelic          hypergolic 
        autotrophic        palaeolithic         psychedelic              -holic 
       hypertrophic           neolithic         psychedelic          workaholic 
           strophic         heliolithic            Goidelic           potaholic 
       catastrophic         granolithic            Goidelic              cholic 
        geostrophic          monolithic             angelic         melancholic 
        apostrophic          Mesolithic         archangelic         melancholic 
         dystrophic          mesolithic            Gadhelic             -oholic 
            Sapphic            ornithic            Gadhelic           alcoholic 
               .                   .                   .                   .
               .                   .                   .                   .
               .                   .                   .                   .
        theomorphic              Sothic        bibliophilic           fumarolic 
        idiomorphic            autarkic         hydrophilic              frolic 
      theriomorphic              Turkic       psychrophilic              frolic 
     enantiomorphic              Turkic           lyophilic              frolic 
        homomorphic             vocalic             basilic            petrolic 
      actinomorphic         postvocalic              exilic           diastolic 
        monomorphic            vandalic          pre-exilic           apostolic 
         zoomorphic            cephalic              gallic        subapostolic 
    anthropomorphic           -cephalic          pyrogallic              garlic 
      heteromorphic        megacephalic             phallic          peelgarlic 
        mesomorphic          encephalic         ithyphallic           pilgarlic 
         isomorphic       mesencephalic         ithyphallic               aulic 
      robotomorphic     dolichocephalic             thallic           hydraulic 
       protomorphic       orthocephalic            metallic    electrohydraulic 
        polymorphic      megalocephalic          bimetallic           salicylic 
            glyphic       macrocephalic         nonmetallic              beylic 
         anaglyphic       microcephalic      organometallic               hylic 
         triglyphic       microcephalic         intervallic             acrylic 
       hieroglyphic       hydrocephalic            Cyrillic            dactylic 
            myrrhic        mesocephalic            Cyrillic            dactylic 
            pyrrhic       leptocephalic         glycol(l)ic          carboxylic 
            pyrrhic         oxycephalic             idyllic             pyeamic 
            gnathic            pashalic            diabolic               gamic 
             pathic            phthalic            anabolic              agamic 
         telepathic               malic           parabolic        phanerogamic 
           empathic               Salic           catabolic         cryptogamic 
       homoeopathic              italic           metabolic        cleistogamic 
         idiopathic              italic           shambolic           polygamic 
         allopathic            tantalic             embolic           sulphamic 
     anthropopathic              oxalic            symbolic         epithalamic 
        hydropathic         philobiblic            symbolic             Islamic 
       heteropathic              public            carbolic             dynamic 
        neuropathic              public          hyperbolic             dynamic 
       naturopathic            republic               colic            adynamic 
            spathic     banana republic          lead colic m                   
        feldspathic   people's republic          wind colic agnetoplasmadynamic 
         felspathic              cyclic     painter's colic       thermodynamic 
\end{verbatim}
\end{document}
