
\documentstyle{Rosetta}
\begin{document}
   \RosTopic{General}
   \RosTitle{The filling of Dutch BADJ entries}
   \RosAuthor{Franciska de Jong}
   \RosDocNr{425}
   \RosDate{\today}
   \RosStatus{informal}
   \RosSupersedes{-}
   \RosDistribution{Project}
   \RosClearance{Project}
   \RosKeywords{BADJ, attribute value assignment}
   \MakeRosTitle
%
%
\section{Introduction}

This document is primarily meant as an instructive note for the
"lexicographers" working on the entries for BADJs
It may also be seen as provisional 
documentation on the choice of attributes and attribute
values for Dutch BADJs
In the text, `BADJ' and `adjective' are both meant to refer to adjectival
lexical elements.

\section{The BADJ-record in dutch:lsdomaint.dom}
In the dutch domain, the BADJ record is specified as follows:
\begin{verbatim}
BADJrecord       =
                   <
                    req:               polarityEFFSETtype:[pospol, negpol, 
                                                        omegapol]      {28/10}
                    env:               polarityEFFSETtype:[pospol, negpol,
                                                        omegapol]{17/9}{28/10}  

                    class:             timeadvclasstype:omegaTimeAdvClass
                    deixis:            deixistype:omegadeixis
                    aspect:            aspecttype:omegaAspect
                    retro:             retrotype:false

                    uses:             adjuseSETtype:[attributive, predicative,
                                      nominalised] {*morph* ++} 
                    eFormation:       eFormationtype:RegEformation {*morph*,8/5}
                    sFormation:       sFormationtype:true {*morph* ++}
                    eNominalised:     eNominalisedtype:regeNominalised
                                               {*morph* } {8/5}
                    comparatives:     comparativeSETtype:[erComp]  {*morph* } 
                    superlatives:     superlativeSETtype:[stSup]  {*morph* } 

                    temporal:         temporaltype:false
                    subcs:            adjsubcSETtype:[otheradj]
                    reflexivity:      reflexivetype:notreflexive
                    thetaadj:         thetaadjtype:omegathetaadjp
                    adjpatterns:      synpatternSETtype:[]
                    prepkey:          keytype:0
                    possadv:          possadvtype:true
                    KEY
                   >
\end{verbatim}

\newpage
For each attribute an attribute type is specified, plus a default value.
Note that some attributes have a set as their (default) value while
others have a single 
value. 
Except for the  exceptional attributes .env and .req,
this difference is encoded 
in the names of the attributes: attributes with a value set as its value have 
a name that ends in an additional -s.

In filling the lexical entries, the provisional values (or value sets) 
must be replaced by the correct value.
For each attribute the possible attribute values are specified in the
in dutch:lsdomaint.dom, in the set that is declared for each attribute type.
In filling an entry the correct value or valueset must be chosen 
from this set.
For convenience the relevant sets are given here in 
 the following  list. 
\begin{verbatim}

[req]           valueset: [pospol, negpol, omegapol]   

[env]           valueset: [pospol, negpol, omegapol]   

[class]         valueset: [duration, reference, frequential, omegaTimeAdvClass]
 
[deixis:]       valueset: [omegadeixis, presentdeixis, pastdeixis]

[aspect]        valueset: [habitual, imperfective, perfective, omegaAspect]

[retro]         valueset: [true, false]

[uses]          valueset: [attributive, predicative, nominalised] 

[eFormation]    valueset: [NoFormation, RegEFormation, IrregEformation]

[sFormation]    valueset: [true, false]

[eNominalised]  valueset: [noNominalised, regeNominalised, IrregeNominalised]

[comparatives]  valueset: [erComp, erIrregComp, meerComp, NoComp]

[superlatives]  valueset: [stSup, stIrregSup, allerSup, allerIrregSup, 
                           meestSup, noSup]

[temporal]      valueset: [true, false]

[subcs]         valueset: [coloradj, modaladj, measureadj, subjectiveadj,
                          substanceadj, otheradj]

[reflexivity]   valueset: [notreflexive, reflexive, reciprocal]

[thetaadj]      valueset: [omegathetaadjp, adjp000, adjp100, adjp120, 
                          adjp123, adjp012, adjp010, adjp210, adjp021,
                          adjp132, adjp213, adjp231, adjp312, adjp321]
                         
[adjpatterns]   synpatternSETtype:[] (too long to enumerate here)

[prepkey]       keytype: 0 (idem)

[possadv]       valueset: [true, false]
  
[KEY]

\end{verbatim}

Some `omegavalues' are not relevant at the lexical level. This is indicated
in the comment on the individual attributes in the next section.\\

Uptil now the following no constraints have proposed that check the consistency 
of the filling of a BADJ-entry.

\section{Criteria for value assignment}
In the remainder of this instruction the criteria for 
determining the correct value for the BADJ attributes are given.

\newpage

\begin{description}
\item
[req]\mbox{}

valueset: [pospol, negpol, omegapol]\\

.req is a counterexample to the convention that attributes that have a set
as value have a name that ends with an -s. \\

Filling postponed.

\newpage
\item 
[env]\mbox{}

valueset: [pospol, negpol, omegapol]\\

.env is a counterexample to the convention that attributes that have a set
as value have a name that ends with an -s. \\

Filling postponed.

\newpage
\item 
[class]\mbox{}

valueset: [duration, reference, frequential, omegaTimeAdvClass]\\


The exact criteria will be specified by Lisette Appelo, who will probably also
take care of the actual filling. \\
Nondefault specification of .class (default = omegaTimeAdvClass) 
requires the value `true' for .temporal.

\newpage
\item 
[deixis]\mbox{}

valueset: [omegadeixis, presentdeixis, pastdeixis]\\


The exact criteria will be specified by Lisette Appelo, who will probably also
take care of the actual filling. \\
Nondefault specification of .deixis (default = omegadeixis) requires
the value `true' for .temporal.

\newpage
\item 
[aspect]\mbox{}

valueset: [habitual, imperfective, perfective, omegaAspect]\\


The exact criteria will be specified by Lisette Appelo, who will probably also
take care of the actual filling. 
Nondefault specification of .aspect 
(default = omegaAspect) requires the value `true' for .temporal.

\newpage
\item 
[retro]\mbox{}

valueset: [true, false]\\


The exact criteria will be specified by Lisette Appelo, who will probably also
take care of the actual filling. \\

Nondefault specification of .aspect 
(default = false) requires the value `true' for .temporal.

\newpage
\item [uses]\mbox{}

valueset: [attributive, predicative, nominalised] \\

The attribute indicates in which way(s) the adjective may be used.
There are three possibilities:\\
\begin{enumerate}
  \item An adjective has {\em attributive} in the valueset for {\bf .uses}
 if it can occur in prenominal position. For example:
\begin{enumerate}
  \item de {\bf mooie} os
  \item een {\bf bijzonder} huis
  \item het  {\bf voormalige} staatshoofd
\end{enumerate}
  \item An adjective has {\em predicative} in the valueset for {\bf .uses}
 if it can occur in predicative position. For example:
\begin{enumerate}
  \item De os is {\bf gek} op zout 
  \item Het huis werd {\bf bijzonder} gevonden
\end{enumerate}
  \item An adjective has {\em nominalised} in the valueset for {\bf .uses}
 if it can occur in NPs without a nominal head. (NB. If the denoted object
is human, while there is no PREPP-modifier to the right of the adjective, the 
adjective may receive an {em -en}-suffix.) For example:
\begin{enumerate}
  \item de weigerachtigen
  \item *de voormalige(n)
  \item de gele 
\end{enumerate}
\end{enumerate}


\newpage
\item [eFormation]\mbox{}


    valueset: [NoFormation, RegEFormation, IrregEformation]\\

This attribute indicates whether the adjective receives a regular 
flection {\em -e}
when it is used in attributive position (RegEformation), whether it has no 
inflection (NoFormation), or whether it has an irregular 
morphology in attributive position (IrregEformation).
Examples:\\
\begin{description}
  \item RegEformation: mooi, goede, verliefde
  \item Noformation: houten, gouden
  \item IrregEformation: grof (grove) 
\end{description}

\newpage
\item [sFormation]\mbox{}

valueset: [true, false]\\

This attribute indicates whether the adjective may receive a suffix {\em -s}
to construe NPs such as {\em iets leuks}, {\em wat lekkers}.
\newpage
\item 
[eNominalised] \mbox{}

valueset: [noNominalised, regeNominalised, IrregeNominalised]\\

This attribute indicates in what form the adjective may occur in an NP 
without a 
nominal head. Examples:\\
\begin{description}
  \item noNominalised: houten
  \item regEnominalised: geel (de {\bf gele} liggen hier) 
  \item IrregeNominalised:grof ( de {\bf grove} zijn onvindbaar)
\end{description}


\newpage
\item 
[comparatives] \mbox{}

 valueset: [erComp, erIrregComp, meerComp, NoComp]\\

This attribute indicates the possible morphological forms that 
may play a role in the expression of  'comparison'.


\newpage
\item 
[superlatives]\mbox{}

  valueset: [stSup, stIrregSup, allerSup, allerIrregSup, 
                           meestSup, noSup]\\

This attribute indicates the possible morphological forms that 
express 'superlativity'.

\newpage
\item
[temporal]\mbox{}
      valueset: [true, false]\\

This attribute is meant to mark the adjectives
that can have a temporal interpretation.

\newpage
\item
[subcs]\mbox{}

         valueset: [coloradj, modaladj, measureadj, subjectiveadj,
                          substanceadj, otheradj]\\

In the rules, only to a subset of these values reference has been made 
up till now. 
Examples:\\
\begin{description}
  \item [subjectiveadj] In Rosetta reference to this attribute value is made 
to decide whether an {\em om te}-sentence may modify an adjective.
Example:\\
\begin{enumerate}
  \item  dat is {\bf leuk} om naar te kijken
  \item  dat is {\bf verstandig} om te doen
  \item  *dat is {\bf rood} om te gebruiken
\end{enumerate}
NB. A modifying {\em om te} sentence may also be licensed by the
occurrence of a degree adverb such as {\em te}. 
For example in: Die soep is te {\bf warm} om nu te eten.
These examples should NOT be considered as an inidication 
that the  adjective is a {\em subjectiveadj}.

  \item [measureadj]Examp[le:  {\em  halve}. 
It makes {\em de halve dag} a measure NP
and therefore it may function as a durative temporal adverbial, while {\em de 
dag} may not do so.
\end{description}

\newpage
\item
[reflexivity]\mbox{}

   valueset: [notreflexive, reflexive, reciprocal]\\


The attribute value {\em reflexive} is used for 
adjectives that require the co-occurrence of a non-argument 
reflexive pronoun; {\em 
reciprocal} is used for adjectives 
that require the co-ocurrence of a non-argument reciprocal pronoun.
(The non-argument of the relevant reflexives and reciprocals
implies that they are not translated.\\

Examples: 
\begin{description}
  \item[reflexive] Hij is zich {\bf bewust} van het probleem.
  \item[reciprocal] Zij zijn aan elkaar verwant (English: they are familiar)
\end{description}

\newpage
\item 
[thetaadj]\mbox{}

valueset: [omegathetaadjp, adjp000, adjp100, adjp120, 
                          adjp123, adjp012, adjp010, adjp210, adjp021,
                          adjp132, adjp213, adjp231, adjp312, adjp321]\\
                         
Examples:\\
\begin{description}
  \item [adjp000] Het is {\bf koud}/{\bf regenachtig}
  \item [adjp100] Dat is {\bf onwaarschijnlijk}; Het is {\bf onwaarschijnlijk}
dat het gaat regenen; Jan is {\bf ziek}
  \item [adjp120] De is {\bf dol} op zout; Ik ben {\bf blij} dat het regent;
Jan is {\bf verslaafd} aan drop
  \item [adjp123]
Hij is mij een tientje {\bf schuldig}
  \item [adjp012] (ergative adjs) Het antwoord is mij {\bf duidelijk}; Dat hij 
drinkt is ons {\bf bekend}
\end{description}

For the other values no relevant examples have been found yet.

\newpage
\item 
[adjpatterns]\mbox{}
synpatternSETtype:[] (too long to enumerate here)

The set that will turn out to be relevant to BADJs is a subset
of the synpatternSET.

For further documentation, cf. doc. R374 (ADJpatterns of Dutch).

\newpage
\item
[prepkey]\mbox{}

This attribute is meant to indicate whether the adjective can be complemented
by a prepositional object with a fixed preposition. 
This preposition is uniquely identified by its key (of type keytype).
Such a key is indicated indirectly by its name, a string preceded
by \$, as defined in the testing dictionary.\\

Examples:
\begin{itemize}
  \item 
 gehecht{\em aan}; .prepkey = \$aanprepkey
  \item
 verliefd {\em op}; .prepkey = \$opprepkey
\end{itemize}

Nondefault specification of .prepkey
(default = 0) would require a nondefault value for .thetanp.

\newpage
\item 
[KEY]\mbox{}

The attribute .key is meant to uniqely identify a syntactic lexical element.\\

The attribute .key is specified in the top of a lemma by means of a string
preceded by \$s
\end{description}
\end{document}
