
\documentstyle{Rosetta}
\begin{document}
   \RosTopic{Rosetta3.doc.linguistics}
   \RosTitle{Documentation Idioms}
   \RosAuthor{Andr\'{e} Schenk}
   \RosDocNr{273}
   \RosDate{\today}
   \RosStatus{concept}
   \RosSupersedes{-}
   \RosDistribution{Project}
   \RosClearance{Project}
   \RosKeywords{Idioms}
   \MakeRosTitle
%
%

\input{[schenk.univers.dissertation]definitions}

\section{Introduction}
This paper gives the documentation for idioms in Rosetta. 
In section~\ref{idiom} we will introduce idiom classes and show how they are 
represented in the dictionary. In section~\ref{idform} we will present the 
idiom formation rules and in section~\ref{or} we will discuss other rules that 
have been added or adapted for idioms.

\section{Idiomatic Expressions in the Dictionary}
\label{idiom}
This section gives a description of the way idiomatic expressions are listed in
the dictionary. We will introduce the notions idiom, fixed and flexible idiom 
and translation idiom and show how each of these is represented in the 
dictionary. The last part of this section consists of tests for dictionary 
filling.

\subsection{Idioms}

We can loosely define idioms as expressions consisting of more than one word,
for which a literal, i.e.\ compositional, interpretation does not yield the
correct meaning. The classic example is {\em kick the bucket}. Literally this
means {\em to hit a specific vessel with one's foot}. The idiomatic reading is
approximately {\em to die}. It is obvious that this second interpretation
cannot be derived compositionally from the parts of the expression. 

Idioms occur in all languages. In most cases, an idiomatic expression in one
language has an idiomatic translational equivalent in other languages, {\em
kick the bucket} corresponds to {\em de pijp uit gaan} and {\em spill the
beans} to {\em zijn mond voorbij praten}. 

As these examples show, there is no direct relation between surface forms of
idioms in different languages. In some cases, the most adequate translational
equivalent may even be a single word, e.g.\ a translation of the Dutch idiom
{\em de pijp aan Maarten geven} is {\em opt out} and of {\em laten zitten} is
{\em ditch}. 

\subsubsection{Fixed idioms}

Some idioms can be treated as strings, i.e.\ as contiguous rows of words in a
fixed order. 
These `fixed idioms' are expressions consisting of more than one word in 
which the order of the words cannot be changed by syntactic operations and no
words can intervene between the words of the fixed idiom. Furthermore, 
expressions of this type should be assignable to a lexical category, like BNOUN 
or BADJ. An example is {\em red herring}.

These idioms  are treated as though they were simple words without any relevant
internal structure. It is possible to apply morphological operations to the
final part of a fixed idiom, e.g.\ for deriving the plural form {\em red
herrings}. 

Note that there are types of fixed idioms for which it would be hard to assign
an internal syntactic structure, even if one wanted to, because they are
syntactically obsolete or in some other way deficient. {\em Kant en klaar} is
an example of this in Dutch: a noun ({\em kant\/}) and an adjective ({\em
klaar\/}) are coordinated. An English example of syntactic deficiency is {\em
by and large}, where a preposition and an adjective are coordinated. 

\subsubsection{Flexible idioms}

For most idioms it is impossible to treat them as strings.
We will give two arguments against a string treatment here:

(i) The words of an idiom may be scattered over the sentence. For example, in
(\ref{CI13}a) {\em gave} and {\em the finger} have to be interpreted
idiomatically, while the free argument {\em Mary} is intervening (the actual 
parts of the idiom are underlined). In  (\ref{CI13}b) a possessive
pronoun that varies with the subject intervenes between the 
other parts of the idiom. In
(\ref{CI13}c) a temporal adverb is intervening between the verb and its
complement. \\
 

\begin{lxam}
&a&& Pete \underline{gave} Mary \underline{the finger}\\ \label{CI13}
&b&& Pete \underline{lost} his \underline{temper}\\
&c&& Pete \underline{gaf} gisteren \underline{de pijp aan Maarten}\\
&&& (Pete gave yesterday the pipe to Maarten)
\end{lxam}


(ii) Idioms occur in a variety of forms that are accounted for in
transformational grammar by means of syntactic transformations. For example,
the idiom in (\ref{CI18}a) has a passive counterpart in (\ref{CI18}b). 

\begin{lxam}
&a&& Pete \underline{broke} Mary's \underline{heart}\\ \label{CI18}
&b&& Mary's \underline{heart} was \underline{broken} by Pete
\end{lxam}

Other examples of the transformational capacity of idioms are given in
(\ref{CI20}a-b). In (\ref{CI20}a) the verb is in sentence-final position and in 
(\ref{CI20}b) the verb is in the position following the subject. 

\begin{lxam}
&a&& Pete heeft \underline{de pijp aan Maarten gegeven}\\\label{CI20}
&&& (Pete has the pipe to Maarten given)\\
&b&& Pete \underline{gaf de pijp aan Maarten}\\
&&& (Pete gave the pipe to Maarten)
\end{lxam}

These examples show that a string treatment of these 
idioms, which we will call `flexible idioms',  would not account for all of the
data. A representation in the form of an explicit 
syntactic structure is needed, which shows what the parts of the 
idiom are and where free arguments can be inserted.
Furthermore, the fact that transformations apply to idioms suggests that the 
representation of idioms should be in a
canonical form, i.e.\ a form to which no syntactic transformations have applied.
\\

So, an idiom is a basic expression with an explicit constituent
structure. 

\subsection{Translation Idioms}
The way flexible idioms are treated
can also be used to solve other translation problems, in particular if a word 
in one language does not correspond to a single word in the other language, but 
to a larger expression. This larger expression may have a compositional 
semantics, so it need not be an idiomatic 
expression from a monolingual point of view, but in spite of
the compositional semantics a compositional translation is not possible.
We will refer to these expressions as `translation idioms'.

Examples are {\em zachtjes neerleggen}, Italian {\em adagiare};
{\em vroeg opstaan}, Spanish {\em madrugar}.

Another example is the reflexive Spanish verb {\em enamorarse} corresponding to
an idiomatic expression in English {\em fall in love} and a non-idiomatic
expression in Dutch {\em verliefd worden}, that should be treated as a
translation idiom. 

Translation idioms are not only useful for defining the translation relation
between a word and a complex expression, but also between two complex
expressions. Examples are: a combination of a verb, an object and a
prepositional object {\em iemand om brood sturen} has to be translated in a
combination of a verb, an object and a subordinate clause in which the verb
takes an object in Spanish {\em mandar a alguien a buscar pan} (lit. {\em ask
to someone to get bread}), and a combination of a verb, an object and a
prepositional object {\em iemand van het paard helpen} has to be translated in
a combination of a verb, an object and a subordinate clause in which the verb
takes an object in Spanish {\em ayudar a alguien a descender el caballo} (lit.
{\em help to somebody to dismount the horse}). 

The examples mentioned above all involve expressions headed by a verb. The
idiom techniques can be used for other constructions as well. Examples are:
{\em de trein naar Gent}, {\em el tren que va a Gent} (lit. {\em the train that
goes to Gent}) and {\em de trein van Gent}, {\em el tren que viene de Gent}
(lit. {\em the train that comes from Gent}). 

\subsection{Dictionary Entries}

\paragraph{Dictionary Entry Fixed Idiom}

The notation for a fixed idiom is an underscore between the parts 
of the expression, e.g. {\em naar\_huis}, {\em kant\_en\_klaar}. For the rest
the filling of the dictionary entry is as usual for the lexical category that is
assigned to the idiom. 


\paragraph{Dictionary Entry Flexible Idiom}

No difference is made in the grammatical specifications in the dictionary
between flexible idioms and flexible translation idioms. Translation idioms are
marked as being translation idioms by adding @TI@ to the comment section of the 
lemma. 

An idiom is specified in the entry of the head of the idiom. So {\em kick the
bucket} is specified in the entry of the verb {\em kick}. If such an entry does
not exist, i.e. if the non-idiomatic variant of the head does not occur, a new
entry is created which is filled analogously to entries of the same category,
but attribute {\bf mkey} is not filled. Examples of such cases are: the
verbs {\em aanbinden} in {\em de kat de bel aanbinden} and {\em nemen} (there
is no non-idiomatic {\em nemen} that takes synLOCCLOSEDPREPPPROP as
verbpattern) in {\em iets in acht nemen} have no non-idiomatic counterpart. 

The information in an entry for a flexible idiom specifies:
\begin{enumerate}
  \item The canonical syntactic structure of the idiom. This is done by 
specifying the skeys of the leaves of the idiom and an identifier (the 
idiompattern) indicating
the syntactic tree model. 
  \item The syntactic key of the idiom.
  \item The meaning key of the idiom.
\end{enumerate}

Below examples of entry parts for idiom specification are given. The first
example is for the idiom {\em de pijp uitgaan} (slightly modified for
expository reasons). Between angular brackets the skeys of the leaves of the
idiom are specified. i.e. {\em pijp} and {\em uit} (remember that this
specification of the idiom is part of the entry of {\em uitgaan}, so the skey
of that verb does not have to be repeated here). The order in which the skeys
are represented is important. The order should be the same as when the idiom is
pronounced infinitively in subordinate clause order; so one says {\em de pijp
uitgaan} when speaking about the idiom and the order of the keys is then first
{\em pijp} and then {\em uit}. Sometimes it is impossible to pronounce an idiom
infinitively in subordinate clause order (cf. {\em het geld groeit hem niet op
de rug}), in that case an abstract infinitive surbordinate clause order should,
cf. {\em het geld iemand op de rug groeien} with req: negpol, cf. below. Note
that the article {\em de} is not specified in this list. This is due to the
fact that this article is introduced syncategorematically in the
idiomformationrule. 
Words that are introduced syncategorematically are:
\begin{enumerate}
  \item  articles {\em de, het, een}, 
  \item  circumstance {\em het} ({\em het regent, het is mooi weer, het is leuk
in England}), 
  \item extraposition {\em het} ({\em hij betreurt het dat hij komt}), 
  \item existential {\em er} ({\em er zijn eenhoorns}), 
  \item quantificational {\em er} (eg. in {\em het zijn er twee}), 
  \item in idioms prepositional {\em er} (cf. {\em het bijltje er bij
neergooien}; note that the skey of {\em het} should be specified), 
  \item particles (eg. {\em aan} in the idiomatic expression {\em de buikriem
aantrekken}), 
  \item prepositions in prepositional objects ({\em kijken naar}), 
  \item the indirect object preposition {\em aan}, 
  \item the passive preposition {\em door}, 
  \item the benefactive preposition {\em voor}. 
\end{enumerate}

The corresponding words in other languages are, of course, also introduced
syncategorematically (when they exist in that language, of course). Between
angular brackets the idiompattern vpid30 is given which refers to a rule that
specifies the syntactic structure. The term s\_id\_depijpuitgaan\_BVERB
specifies the skey of the idiom; the term m\_id\_depijpuitgaan the mkey. The
last part is the meaning description of the idiom. 

\begin{verbatim}
    < $s_aN_00_pijp $s_prep_uit > [vpid30]
    $s_id_depijpuitgaan_BVERB $m_id_depijpuitgaan
    s1 "doodgaan" 
\end{verbatim}



As an example the full entry of the idiom {\em de plaat poetsen} is given. It 
is incorporated in the entry for the verb {\em poetsen} with skey
s\_aV\_00\_poets, mkey m\_aV\_0001\_poets and meaning description
"schoonmaken". 

\begin{verbatim}
poets
   :$s_aV_00_poets
   :< $plaatBNOUNkey > [ vpid1 ] 
    $plaatpoets_skey $plaatpoets_mkey
   :$m_aV_0001_poets                s1 "schoonmaken"
   :BVERB (
{req:}            [pospol, negpol, omegapol],
{env:}            [pospol, negpol, omegapol],
{conjclasses:}    [3],
{particle:}       0,
{possvoices:}     [active, passive, Dooractive],
{reflexive:}      notreflexive,
{synvps:}         [synNP
                  ],
{thetavp:}        VP120,
{adjuncts:}       [resAP],
{CaseAssigner:}   true,
{subc:}           mainverb,
{perfauxs:}       [hebaux],
{prepkey1:}       0,
{prepkey2:}       0,
{controller:}     none,
{verbraiser:}     noVR,
{IPP:}            NOIPP,
{classes:}        [ durativeclass]
);
\end{verbatim}

An idiom can have more than one meaning. This is specified as in the following 
example.

\begin{verbatim}
aan    {on | a} 
       :$aanprepkey
       : < $s_hand_BNOUN_1 $vanprepkey > [ vpid19 ]
         $s_id_aandehandvan_PREP_1 $m_id_aandehandvan_1
         {" met behulp van "},
         < $s_hand_BNOUN_1 $vanprepkey > [ vpid19 ]
         $s_id_aandehandvan_PREP_2 $m_id_aandehandvan_2
         {" iets in aanmerking nemend "}
       :$m_prep12376000 "m.b.t. fysieke verbondenheid"
\end{verbatim}

Idiom patterns have to describe the form of an idiom exactly. Specifically, the
NPs vary due to the following distinctions: 

\begin{enumerate}
  \item mass vs. count: eg. the NP {\em z'n mond} in {\em z'n mond
voorbijpraten} is different from the NP {\em z'n geduld} in {\em z'n geduld
verliezen}, because the noun {\em mond} is count and the noun {\em geduld} is
mass. 
  \item singular vs. plural: the NP {\em stukken} in {\em iets in stukken 
hakken} is different from the NP {\em acht} in  {\em iets in acht nemen}.
  \item diminutive vs. not diminutive: cf. {\em de pijp uitgaan} and {\em het 
hoekje omgaan}.
  \item the combination of singular, plural, diminutive and not diminutive: cf.
{\em iets in acht nemen}, {\em het hoekje omgaan}, {\em iets in mootjes
hakken}, {\em iets in stukken hakken} 
  \item definite article/ indefinite article/ no article: eg. {\em de plaat
poetsen}, {\em het hoekje omgaan}, {\em met behulp van}. Note 1: it is specified
at the noun whether it is a {\em de}-word or a {\em het}-word; the selection of
the article with diminutives is done by rules (cf. {\em de krant}, {\em het
krantje}), so it not necessary to make different idiom patterns for these
distinctions. Note 2: in some cases only the diminutive form of a word exists 
only in an idiom, cf. {\em ootje} in {\em iemand in het ootje nemen} or
{\em blauwtje} in {\em een blauwtje lopen}. These words should be listed as 
such as new entries, because they are irregular.
  \item propernoun: eg. {\em de pijp aan Maarten geven}
  \item presence of modifiers, etc.: cf. {\em de eerste viool spelen}, {\em een
kat in de zak kopen} ({\em in de zak} modifies {\em een kat}), {\em zoete
broodjes bakken}. 
\end{enumerate}

Of course, several other combinations than the ones given here are possible. 
Each of these results in a different idiom pattern.

Other examples:
\begin{enumerate}
  \item {\em geven} and {\em aanbinden} both have synpattern synIONP\_DONP, but
the idpattern of {\em iemand de zak geven} and {\em de kat de bel aanbinden}
have to be different, since in the first the indirect object is free and in the
latter the indirect object is a fixed part of the idiom. 
  \item the verbs in the following idioms all have synpattern synNP, yet the 
idiom patterns have to be different: {\em de kroon spannen}, {\em zijn geduld
verliezen}, {\em zijn mond houden}, {\em iemand's hart breken} (free argument 
in NP), {\em de pet aantikken} (particle) and {\em herrie schoppen}. 
\end{enumerate}

Note 1: in some cases the non-idiomatic variant of a part of the idiom, other
than the head does not exist. Examples are the nouns {\em brui} in {\em ergens
de brui aan geven} and {\em lurch} in {\em leave somebody in the lurch}. In
these cases new entries are created which are filled analogously to entries of
the same category, but an mkey is not specified. 

Note 2: Negation: if an idiom contains a negation element, then if this 
negation element can be deleted in a negatively polar context, the attribute 
req should be put to negpol; if the negation cannot be deleted in a negatively 
polar context the skey of the negation element should be specified at the idiom
description.

Note 3: the order of embedded free arguments: in {\em iemand's hart breken}
iemand is a free argument embedded in the object NP {\em iemand's hart}. 
Embedded free arguments should be ordered according to the Argument Ordering 
Convention as if it were the constituent it is embedded in (cf. Report to the 
Lexic-Project, Definitive Version). So in the above example the order of the
free argument is determined as if it were the direct object. 

Note 4: passives: some idioms cannot passivize although they consist of a verb 
and an object NP, cf. {\em kick the bucket} cannot passivize. This has to be
specified at the idiom entry, unless passivization is ruled out by the grammar;
cf. {\em z'n geduld verliezen} cannot passivize because the pronoun is bound by
the subject. 

\subsection{Tests for dictionary filling}
\label{tfdf}

Below tests and characteristics are given to determine the status of an 
expression as an idiom, a translation idiom or a fixed idiom.

\paragraph{Idioms}
An expression is an idiom if no fixed part of that expression can be 
\begin{enumerate}
  \item Topicalized with shift of focus, cf. {\em De plaat, poetste hij}.
  \item Part of a control structure, cf. {\em He instructed the piper to be
paid}.
  \item Modified, cf. {\em Hij poetste de mooie plaat}.
  \item Relativized, cf. {\em Hij poetste de plaat, die hij mooi vond}.
  \item Clefted or pseudo-clefted, cf. {\em Het was de plaat, die hij mooi
vond}.
  \item The antecedent of pronominal reference, cf. {\em Hij poetste de plaat.
Hij vond hem mooi}, where {\em hem} refers to {\em plaat}. 
\end{enumerate}

All examples are ok when read literally, but out when interpreted
idiomatically.

Note 1:
\begin{enumerate}
  \item Parts of an idiom can be subject to raising, verb-second, object to 
subject, subject to comp, topicalization without shift of focus.
  \item Parts of an idiom, that contain a free argument can be subject to 
topicalization, cleft, pseudo-cleft, wh-movement, cf. {\em Marie's hart, brak
hij}, {\em Het was Marie's hart, dat hij brak}, {\em Wiens hart heeft hij
gebroken?}. 
\end{enumerate}

Note 2:
For some expressions there may be speaker variation in the assessment of the 
status of a complex expression. For example, the expression {\em iemand een 
loer draaien} is an idiom for some speakers of Dutch, but a metaphor or a 
complex predicate for others, cf. some people find {\em de loer die hij mij
draaide zal ik hem betaald zetten} ok, while others find that sentence
ungrammatical. People that find this sentence ungrammatical consider {\em 
iemand een loer draaien} an idiom.

\paragraph{Translation idioms}
Criteria to assess the status of a complex expression as a translation idiom:
\begin{enumerate}
  \item Does a complex expression correspond to a non-similar expression in
another language. 
  \item It is not an idiom. 
\end{enumerate}

\paragraph{Fixed idioms}
Criteria to assess the status of a complex expression as a fixed idiom:
\begin{enumerate}
  \item The order of words of the expression cannot change. 
  \item No words can intervene between the words of the expression.
  \item The expression is assignable to a lexical category. 
\end{enumerate}
Note that it is always possible to treat a fixed idiom as a flexible idiom.


\section{Idiom Formation Rules}
\label{idform}

In this section the format of the idiom formation rules is given.
In section~\ref{or} rules, transformations and filters that have been added or
adapted for idioms will be discussed. 

The models of a syntactic D-tree are specified in subgrammar VERBPPROPFORMATION
in the startrules rule class, or in the case of other heads in the relevant
other subgrammars. We will only discuss idioms with a verbal head here.


Below, as an example, the listing of RULE RIDDERIV1 has been given, which works
for idioms with idpattern vpid1 (eg de plaat poetsen) and vpid25 (eg herrie
schoppen). The input model consists of a subverb specifying the idiomatic key
and the models of the free arguments of the idiom. In the example below KEY1
and T1, respectively. The output model consists of the syntactic derivation
tree that specifies the canonical S-tree of the idiom. Note that, in a literal
derivation, of the S-tree several other rules and transformations apply. For
example, rules for superdeixis have not been included. We will highlight some 
interesting properties of the output model.

Transformation TIDCLAUSETOVPPROP changes the top category from clause into
vpprop. It is prefixed by the name of subgrammar IDFORMATION. Under this name
all rules and transformations that are not part of M-grammar have to be
specified.
 
Rule RSUBSTITUTION1 is prefixed by the name of subgrammar CLAUSETOSENTENCE. It 
has as parameters level X1 and subst 2. Transformation TIDVPPROPTOCLAUSE is 
the reverse of TIDCLAUSETOVPPROP.

In BLEX/BVERB(KEY2), BLEX specifies that it is a terminal node. KEY2 is a
variable over keys. 

RIDALTNPFORMATION1 is an IDFORMATION rule that choses between more than one
alternative rule, cf below. The idpattern is a parameter of this rule. 

The rules and transformations are applied in the same manner as in M-PARSER and 
M-GENERATOR and dictionary access is also similar. 

Note that, analogously to BLEX/BVERB(KEY2), it is possible to specify more 
complex S-trees; in fact it is possible to use alternately S-tree and D-tree 
models.

In the compositional conditions the function COMPINIDDICT checks whether the
idiom is in the dictionary with KEY1 and VPID1. In the compositional actions
the function COMPGETIDDICT looks up the keys of the leaves of the idiom in the
dictionary on the basis of KEY1 and VPID1. These keys are assigned to KEY2 and
KEY3. 

In the decompositional conditions the function DECOMPINIDDICT
checks whether there is an idiom in the dictionary with keys KEY2 and KEY3
and idpattern VPID1. In the decompositional actions the function
DECOMPGETIDDICT looks up the key of the idiom in the dictionary on the basis of 
the keys KEY2 and KEY3 and the idpattern VPID1. This key is assigned to KEY1.
Since, in general, a sentence that possibly contains an idiom usually has the
idiomatic reading it gets a higher bonus with WBONUS +5. 

\begin{verbatim}
%RULE RIDDERIV1

<m1: I1::SUBVERB{SUBVERBREC1}[head/BVERB(KEY1){BVERBREC1}]
 m2: T1
>

< m:  IDFORMATION/TIDCLAUSETOVPPROP
        [CLAUSETOSENTENCE/RSUBSTITUTION1{LEVEL:X1 
                                         SUBST:2}
           [IDFORMATION/TIDVPPROPTOCLAUSE
              [VERBPPROPFORMATION/RVERBPATTERN1
                 [VERBPPROPFORMATION/RSTARTVPPROP120
                    [VERBDERIVATION/RBVERBTOSUB
                       [BLEX/BVERB(KEY2)],
                     BLEX/I2:T1,
                     BLEX/NPVAR(X1){NPVARrec1}
                    ]
                 ]
              ],
            IDFORMATION/RIDALTNPFORMATION1{VPID}
              [CNFORMATION/RCNFORMATION1
                 [IDFORMATION/RIDSUBNOUNTONOUN{VPID}
                    [NPDERIVATION/BNOUNTOSUBNOUN
                       [BLEX/BNOUN(KEY3)]
                    ]
                 ]
              ]
           ]
        ]
>
   MATCHCONDITIONS
      <
       I1: SUBVERBREC1.thetavp = vp120 
       I2: T1.CAT IN AUX_VARCATSET 
       m2: T1.CAT IN AUX_VARCATSET 
      >
COMP
 <
   C1: COMPINIDDICT(KEY1, VPID1)
   A1: <KEY2, KEY3> := COMPGETIDDICT (KEY1, VPID1);
   C2: COMPINIDDICT(KEY1, VPID25)
   A2: <KEY2, KEY3> := COMPGETIDDICT (KEY1, VPID25);
 >
DECOMP
 <
   C1: DECOMPINIDDICT(<KEY2, KEY3>, VPID1)
   A1: WBONUS +5;
       KEY1 := DECOMPGETIDDICT(<KEY2, KEY3>, VPID1);
       SUBVERBREC1 := COPYT_bverbtosubverb(BVERBREC1);
   C2: DECOMPINIDDICT(<KEY2, KEY3>, VPID25)
   A2: WBONUS +5;
       KEY1 := DECOMPGETIDDICT(<KEY2, KEY3>, VPID25);
       SUBVERBREC1 := COPYT_bverbtosubverb(BVERBREC1);
 >
&
\end{verbatim}

Below an example of an alternation rule is given.
Rule RIDALTNPFORMATION1 alternates between rules RIDNPFORMATION2 for 
detless NPs and RNPFORMATION for NPs with determiner on the basis of the
idpattern which is a parameter of the rule.
Examples are {\em op basis van} vs {\em aan de hand van}.

\begin{verbatim}
%RULE RIDALTNPFORMATION1

<m1: T1
>

< m: T2
>
PARAMETERS
<vpid:synpatternSETtype
>
<
   SUBRULE  (* 1 *)
      <T1: T3
      >
      <T2: IDFORMATION/RIDNPFORMATION2[BLEX/T3]
      >

COMP
 <
   C1: vpid * AUX_detlessNPvps <> []
   A1: @
 >
DECOMP
 <
   C1: TRUE
   A1: vpid := AUX_detlessNPvps
 >
   SUBRULE  (* 2 *)
      <T1: T3
      >
      <T2: NPFORMATION/RNPFORMATION4[BLEX/T3]
      >

COMP
 <
   C1: vpid * AUX_detNPvps <> []
   A1: @
 >
DECOMP
 <
   C1: TRUE
   A1: vpid := AUX_detNPvps
 >
>
&
\end{verbatim}


\section{Other Rules Added or Adapted for Idioms}
\label{or}

In this section the rules that were added to the grammar, especially for
idioms, apart from the idiom formation rules, will be discussed. 

In the verbpattern transformation class transformation RIdVerbpattern
has been added to let VERBPPROPs pass the
verbpattern transformations without conditions on the status of the arguments, 
such that NPs and PPs also pass and not only variables. This happens when the 
synvpefs attribute of the verbpprop contains idiomatic patterns.

In subgrammar CLAUSETOSENTENCE in rule class RC-SUBSTITUTION rules have been
added to substitute in 'idiomatic' configurations. For example, 
RIdSubstitution1 substitutes for x1 in x1 hart word
gebroken, eg. Jans/ mijn/ mijn vaders/ wiens hart word gebroken. 

In subgrammar CLAUSETOSENTENCE transformation TSubstSpeed has been added to
reduce the number of substitions in M-Parser. Substitution in possibly
idiomatic configurations is only allowed when the relevant idiom patterns are
specified in the S-tree. The idiom patterns have to be added to the relevant
sets in lsauxdomain.auxdom. 

In subgrammar VERBPPROPFORMATION the transformation TPOSSADJSPELLING1 and the
corresponding filter FPOSSADJSPELLING1 have been added. TPOSSADJSPELLING1
spells out a POSSADJ on the basis of the subject of the sentence (eg in {\em 
Jan verliest zijn geduld}). 

In subgrammar XPPROPtoCLAUSE, Tidsuperdeixisadaptation has been added to TC:
Superdeixisadaptation to let through propositional constituents with omega
value for superdeixis, when the synvpefs attribute of the clause contains an
idpattern that indicates a propositional idiom pattern. 

In the subgrammar VERBPPROPFORMATION the transformations TISIDIOM and
TISNOTIDIOM have been added. TISIDIOM lets pass VERBs that are specified to be
an idiom. TISNOTIDIOM lets pass VERBs that are specified not
to be an idiom. If TISIDIOM applies, only the startrules for 
idioms are applicable; if TISNOTIDIOM applies, only the non-idiomatic
startrules are applicable. 

Throughout M-grammar rules and transformations are sensitive to whether they 
allow idiom parts to undergo this operation. Most operations only work on 
variables. Some operations work on idiom chunks (i.e. non-variables as NP or
PP). Examples are object to subject and raising to subject.

\end{document}



