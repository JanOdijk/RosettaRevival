\documentstyle{Rosetta}
\begin{document}
   \RosTopic{Rosetta3.doc.Mrules.English}
   \RosTitle{Rosetta3 English M-rules: Derivation Subgrammars}
   \RosAuthor{Margreet Sanders}
   \RosDocNr{316}
   \RosDate{November 30, 1989}
   \RosStatus{approved}
   \RosSupersedes{-}
   \RosDistribution{Project}
   \RosClearance{Project}
   \RosKeywords{English, M-rules, Derivation, Documentation}
   \MakeRosTitle
%
%

\newpage
\section{Introduction}
This document describes the five English derivation subgrammars used in 
Rosetta3. There is a difference in the treatment of prepositions as opposed to 
the other four major categories (verb, noun, adjective and adverb). For the 
latter class, three
levels are distinguished: a lexical entry, a derivation level and an 
inflectional level. Since Rosetta3 has not found many derivation processes yet 
that can be dealt with isomorphically in all languages, the change 
from lexical entry (BVERB, BNOUN, BADJ, BADV) to the derivation level 
(SUBVERB, SUBNOUN, SUBADJ, SUBADV) is 
usually very simple and performed by only one rule. These rules are described 
in the current document. The change from SUB-X to
the full X accounting for inflection is done in the XPPROP subgrammars for 
ADJs, VERBs and ADVs, and in 
the CNformation subgrammar for NOUNs. The rules concerning this change are 
described in the relevant subgrammars. For comment on 
the treatment of Propernouns, see section 3.2.

For prepositions, no derivation or inflection is necessary, 
but for isomorphism with the 
derivation rules for adjectives and adverbs, an extra level is still needed. 
Rather than change all the M-rules already written with a PREP as lexical 
category, it was decided to introduce a SUBPREP level (with a record identical 
to that of a PREP) in the derivation grammar, and to delete this SUBPREP level 
again in the PREPP(PROP) subgrammars, leaving a simple PREP in the S-tree. 
Hence, all languages now also have a PREP-derivation subgrammar.

As was said above, inflection of the five major categories is not dealt with 
in the current document. However, there are some lexical categories 
(viz.\ certain kinds of pronouns) that have inflection but do not allow for a 
CN-level (e.g.\ because they never have the kind of modification that is 
allowed at that level). Because their inflection cannot be described together 
with the inflection of SUBNOUNs in the CNformation subgrammar, 
their inflectional 
rules have been incorporated in the derivation subgrammar for NOUNs. 
Note that pronouns never have derivation. For INDEFPROs, which do not show any 
inflection in English, the same strategy was adopted as for PREPs: to preserve 
isomorphism with other languages, a SUBINDEFPRO level has been defined (with a 
record identical to INDEFPRO), and in the next subgrammar, NPformation, this 
level is deleted again, leaving a simple INDEFPRO in the S-tree. 


\newpage
\section{Verbderivation}
This is the first grammar a BVERB has to pass when an English sentence is 
generated. The grammar is extremely simple, since English does not have any 
incorporated particles. Thus, there is only one rule, always transforming a
BVERB into a SUBVERB. No rules have been written yet to deal with any kind of 
derivation from one category to another, e.g.\ producing a verb {\em standardize
\/} from a head {\em -ize\/} and an import noun stem {\em standard\/}. The main 
problem with this kind of derivation is that it often is not productive over 
all languages Rosetta3 deals with, and thus cannot be dealt with isomorphically 
in all grammars.

\subsection{Grammar specification}
The grammar definition is to be found in file {\bf 
english:Verbderivation.mrule}, which is {\em mrules1.mrule\/}.

\begin{description}
\item[Control Expression] \mbox{}
   \begin{verbatim}
%
%SUBGRAMMAR Verbderivation

RBVerbToSub 
   \end{verbatim}
\item[Head]    BVERB    \ \ \ \ \ BASIC EXPRESSION 
\item[Export]  SUBVERB 
\item[Import] --
\end{description}


\subsection{RC\_VerbBtoSub}
As stated before, there is only one rule in this rule class.

\begin{description}
\item[Name] RBverbToSub
\item[Task] to make a subverb out of the dictionary entry BVERB, setting the 
attribute {\bf affix} at the appropriate value (currently: always {\em 
noaffix\/})
\item[File] english:RC\_VerbBtoSub.mrule (mrules2.mrule)
\item[Semantics]
\item[Example] swim $\rightarrow$ swim 
\item[Remarks]
\end{description}


\newpage
\section{NPderivation}
This is the first grammar B(PROPER)NOUNs and certain pronouns have to pass 
when an English NP or CN is 
generated. The grammar is extremely simple, since English does not have 
derivation rules for any of these basic expressions yet, except for 
propernouns. Thus, there is only 
one rule for all basic expressions except propernouns. For comment on 
the treatment of Propernouns, see section 3.2.

The pronouns that are dealt with in 
NPderivation are: BWHPRO, INDEFPRO and BPERSPRO. The rules in this subgrammar 
have not been grouped into one or more rule classes, since
they cover different kinds of cases (real 
derivation for nouns, inflection for propernouns and pronouns). 

\subsection{Grammar specification}
The grammar definition is to be found in file {\bf 
english:NPsubgrammars.mrule}, which is {\em mrules56.mrule\/}.

\begin{description}
\item[Control Expression] \mbox{}
   \begin{verbatim}
%
%SUBGRAMMAR NPDERIVATION

(  BNOUNtoSUBNOUN | RBWHPROtoWHPRO | RBPERSPROtoPERSPRO
 | RBPROPERNOUNtoPROPERNOUN | RINDEFPROtoSUBINDEFPRO 
 | RBPROPERNOUNtoSUBNOUN   )

   \end{verbatim}

\item[Head] \mbox{}\\
\begin{tabular}{ll}
BNOUN         & BASIC EXPRESSION \\
BWHPRO        & BASIC EXPRESSION \\
BPERSPRO      & BASIC EXPRESSION \\
BPROPERNOUN   & BASIC EXPRESSION \\
INDEFPRO      & BASIC EXPRESSION 
\end{tabular}

\item[Export]  \mbox{}\\
SUBNOUN \\
WHPRO \\
PERSPRO \\
PROPERNOUN \\
SUBINDEFPRO

\item[Import] --
\end{description}

\newpage
\subsection{NP derivation Rules}
The noun derivation rules are necessary to change the lexical entry BNOUN into 
a SUBNOUN. For Propernouns, no derivation is allowed and hence they have no 
SUBPROPERNOUN level. In Dutch, Propernouns do have inflection (they may form an 
inflectional genitive), but just as for 
pronouns, they do not allow for a CN-level and hence this inflection cannot 
be accounted for there (see the
Introduction to this document). Thus, a rule has been included in the
NPderivation subgrammar.
Note that in fact English has no inflection for Propernouns, but still needs a 
rule here for isomorphism with Dutch.

It is assumed that when a Propernoun seems to be modified at the CN-level, it 
has in fact become a noun. To be able to deal with modification of 
Propernouns at CN-level and with plural Propernouns (remember that plural is an 
inflectional phenomenon, and therefore dealt with in the CNformation 
subgrammar), 
a special derivation rule has been added which changes 
a BPROPERNOUN into a SUBNOUN. From then on, the form can do everything an 
ordinary noun can.
Plural propernouns are considered to be ordinary nouns in Rosetta, 
since no counter examples could be found. All Propernouns taking an 
article are also considered nouns. Those Propernouns always taking an 
article even when not modified are entered in the lexicon as BNOUN straight 
away (e.g.\ {\em the Rhine, the Hebrides\/}). In section 3 of doc.\ 306, 
{\em Rosetta3 
English Morphology: Lextree Rules\/}, this point is also discussed.

\vspace{1 cm}
\begin{description}
\item[Name] RBnounToSubnoun
\item[Task] to make a subnoun out of the dictionary entry BNOUN, setting the 
attribute {\bf affix} at the appropriate value (currently: always {\em 
noaffix\/})
\item[File] english:cnformation.mrule (mrules69.mrule)
\item[Semantics]
\item[Example] table $\rightarrow$ table
\item[Remarks]
\end{description}

\vspace{1 cm}
\begin{description}
\item[Name] RBpropernounToPropernoun
\item[Task] to make a propernoun out of the dictionary entry BPROPERNOUN. The 
attribute {\bf numbers} is always set at the default value 
{\em [singular]\/}, 
since it is now thought that plural propernouns are ordinary nouns. A special 
derivation rule from BPROPERNOUNs to SUBNOUNs has been added (see below) 
to cover for 
the plural of propernouns, and other modifications of propernouns on a 
CN-level.
\item[File] english:cnformation.mrule (mrules69.mrule)
\item[Semantics]
\item[Example] John $\rightarrow$ John (John is a nice guy)
\item[Remarks]
\end{description}

\vspace{1 cm}
\begin{description}
\item[Name] RBpropernounToSubnoun
\item[Task] to make a subnoun out of the dictionary entry BPROPERNOUN, so that 
it can form a plural, or be modified at the CN-level.
\item[File] english:cnformation.mrule (mrules69.mrule)
\item[Semantics]
\item[Example] John $\rightarrow$ John (The John I mean is very tall)
\item[Remarks]
\end{description}

\vspace{1 cm}
\begin{description}
\item[Name] RBwhproTOwhpro
\item[Task] to make a WHPRO out of the dictionary entry BWHPRO, setting the 
attribute {\bf cases} at the default value, which is the empty set {\em [ ]\/}.
\item[File] english:cnformation.mrule (mrules69.mrule)
\item[Semantics]
\item[Example] who $\rightarrow$ who (no case yet); which $\rightarrow$ which
\item[Remarks]
\end{description}

\vspace{1 cm}
\begin{description}
\item[Name] RBpersproTOperspro
\item[Task] to make a PERSPRO out of the dictionary entry BPERSPRO, setting the 
attribute {\bf persprocase} at the default value, which is {\em omegacase\/}.
\item[File] english:cnformation.mrule (mrules69.mrule)
\item[Semantics]
\item[Example] he $\rightarrow$ he (no case yet)
\item[Remarks]
\end{description}

\vspace{1 cm}
\begin{description}
\item[Name] RIndefproTOsubindefpro
\item[Task] To change the dictionary entry INDEFPRO to a different 
category, SUBINDEFPRO, which has the same record except for the key. This rule
is needed for isomorphy with Dutch, where a BINDEFPRO level exists next to an 
INDEFPRO level (for the inflectional genitive: {\em iemand - iemands\/}). In 
the subsequent NPformation subgrammar, the SUBINDEFPRO level is deleted again.
\item[File] english:cnformation.mrule (mrules69.mrule)
\item[Semantics]
\item[Example] somebody $\rightarrow$ somebody
\item[Remarks]
\end{description}

\newpage
\section{AdjDerivation}
This is the first grammar a BADJ has to pass when an English ADJPPROP is 
generated. The grammar is extremely simple, since English does not have any 
derivation rules for adjectives. Thus, there is only one rule, always 
transforming a BADJ into a SUBADJ. The derivation rule changing a SUBADJ to a 
SUBADV is dealt with in the AdvDerivation subgrammar (see below).

\subsection{Grammar specification}
The grammar definition is to be found in file {\bf 
english:Adjsubgrammars.mrule}, which 
is {\em mrules68.mrule\/}.

\begin{description}
\item[Control Expression] \mbox{}
   \begin{verbatim}
%
%SUBGRAMMAR AdjDerivation

RBADJtoSUB1
   \end{verbatim}
\item[Head]    BADJ \ \ \ \ \   BASIC EXPRESSION 
\item[Export]  SUBADJ
\item[Import] --
\end{description}


\subsection{RC\_AdjDerivation}
As stated before, there is only one rule in this rule class.

\begin{description}
\item[Name] RBADJtoSUB1
\item[Task] to make a SUBADJ out of the dictionary entry BADJ, setting the 
attribute {\bf affix} at the appropriate value (currently: always {\em 
noaffix\/})
\item[File] english:RC\_StartAdjpprop.mrule (mrules52.mrule)
\item[Semantics]
\item[Example] beautiful $\rightarrow$ beautiful
\item[Remarks]
\end{description}

\newpage
\section{AdvDerivation}
This is the first grammar a BADV has to pass when an English ADVP or ADVPPROP
 is 
generated. The grammar is extremely simple, since English still has only one 
real derivation rule for adverbs. The `normal' rule changes a
BADV into a SUBADV. The adverb derivation rule takes the BADVsuff {\em -ly\/}
and combines it with a SUBADJ to form a SUBADV.

\subsection{Grammar specification}
The grammar definition is to be found in file {\bf 
english:AdvpSubgrammars.mrule}, which is {\em mrules84.mrule\/}.

\begin{description}
\item[Control Expression] \mbox{}
   \begin{verbatim}
%
%SUBGRAMMAR AdvDerivation

( RBAdvToSubAdv | RSubadjToSubadv )
   \end{verbatim}
\item[Head]    \mbox{}\\
  \begin{tabular}{ll}
  BADV    & BASIC EXPRESSION \\
  BADVSUFF & BASIC EXPRESSION
  \end{tabular}
\item[Export]  SUBADV
\item[Import] SUBADJ \ \ \ \ FROM (AdjDerivation)
\end{description}


\subsection{RC\_AdvDerivation}
As stated before, there are two rules in this rule class.

\begin{description}
\item[Name] RBAdvToSubAdv
\item[Task] to make a SUBADV out of the dictionary entry BADV, setting the 
attribute {\bf affix} at the appropriate value, which is {\em noaffix\/}.
\item[File] english:AdvpSubgrammars.mrule (mrules84.mrule)
\item[Semantics]
\item[Example] yesterday $\rightarrow$ yesterday
\item[Remarks]
\end{description}

\vspace{1 cm}
\begin{description}
\item[Name] RSubadjToSubadv
\item[Task] to make a SUBADV out of a SUBADJ, by combining it with the 
adverbial suffix {\em -ly\/}. The 
attribute {\bf affix} is set at the appropriate value, which is 
{\em lyaffix\/}. This rule is needed for those adverbs that are not in the 
dictionary (adverbs that are should pass straight on to 
the ordinary BadvToSubadv rule).
\item[File] english:AdvpSubgrammars.mrule (mrules84.mrule)
\item[Semantics]
\item[Example] beautiful $\rightarrow$ beautifully
\item[Remarks] In case one language has an adverb (or another lexical category) 
in its dictionary, while its translation in another language is an adjective 
that has to be changed into an adverb, there is a problem with isomorphism: the 
application of the current rule demands an extra level in the derivation tree, 
and there also is an extra basic category (the suffix) in the derivation tree.
This is of course the standard problem for derivation: only if it works across 
all languages in the same way can it be dealt with isomorphically. 
The problem has not been solved yet.
\end{description}

\newpage
\section{Prepderivation}
This is the first grammar a PREP has to pass when an English PREPP or PREPPPROP 
is generated. The grammar is 
written only for reasons of isomorphism with derivation rules for adjectives 
and adverbs (PREPs have no derivation or inflection in the languages Rosetta3 
deals with), and the SUBPREP level is identical to the PREP-level. See what was 
said about this topic in the introduction to this document. There is only one 
rule, always transforming a PREP to a SUBPREP. 

\subsection{Grammar specification}
The grammar definition is to be found in file {\bf 
english:PrepppropFormation.mrule}, which is {\em mrules86.mrule\/}.

\begin{description}
\item[Control Expression] \mbox{}
   \begin{verbatim}
%
%SUBGRAMMAR Prepderivation

RPrepToSub 
   \end{verbatim}
\item[Head]    PREP \ \ \ \ \ BASIC EXPRESSION 
\item[Export]  SUBPREP
\item[Import] --
\end{description}


\subsection{RC\_PrepDerivation}
As stated before, there is only one rule in this rule class.

\begin{description}
\item[Name] RPrepToSub
\item[Task] To change the dictionary entry PREP to a different 
category SUBPREP which has the same record except for the key.
\item[File] english:RC\_PrepppropFormation.mrule (mrules86.mrule)
\item[Semantics]
\item[Example] without $\rightarrow$ without
\item[Remarks]
\end{description}

\end{document}


