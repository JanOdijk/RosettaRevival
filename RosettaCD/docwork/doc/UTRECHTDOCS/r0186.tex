\documentstyle{Rosetta}
\begin{document}
   \RosTopic{General}
   \RosTitle{Notulen Rosetta vergadering 23-3-1987}
   \RosAuthor{Harm Smit}
   \RosDocNr{0186}
   \RosDate{\today}
   \RosStatus{approved}
   \RosSupersedes{-}
   \RosDistribution{Project}
   \RosClearance{Project}
   \RosKeywords{minutes}
   \MakeRosTitle
\begin{itemize}
  \item {\bf aanwezig}: Lilian Kopinga, Ans Post, Ren\'{e} Leermakers, 
             Jeroen Medema, Joep Rous, Jan Landsbergen, Andr\'{e} Schenk, 
             Lisette Appelo, Natalia Grygierczyk, Carel Fellinger, Jan Odijk, 
             Elly van Munster, Harm Smit, Jan Stevens, Margreet Sanders,
             Chris Hazenberg, Franciska de Jong.
  \item {\bf afwezig}: -.
  \item {\bf Agenda}:
    \begin{enumerate}
       \item Opening en notulen
       \item Diverse mededelingen
       \item Besproken en/of nieuw verschenen documenten
       \item Kort overzicht van ieders werkzaamheden
       \item Rondvraag en sluiting
    \end{enumerate}
\section {Opening en notulen}
De notulen van de vorige vergadering worden met enkele kleine wijzigingen
aangenomen.
\section {Diverse mededelingen}
\begin{enumerate}
  \item {\bf Jan L.} is gestart met het in kleine groepjes overleggen over de
mogelijke SPIN-subsidies (zie vorige notulen) om zo het gedachtenproces 
op gang te brengen. Het gaat daarbij om `bredere' aktiviteiten dan alleen 
machinaal vertalen, zoals Question Answering en foneem-grafeem
omzetting voor spraakherkenning en spellingscorrectie. Als geheel moeten deze
aktiviteiten een {\em multilinguaal} taalverwerkingssysteem vormen.
Loek heeft hierover al een brief naar SPIN gestuurd en Jan L. gaat met hem op 
7 april bij dhr. Struch op bezoek.

Tevens vraagt Jan L. de aanwezigen om een geschikte naam voor een dergelijk
projekt te bedenken, waarbij de nadruk op het multilinguale 
karakter moet liggen.
  \item {\bf Jan L.} en {\bf Harm} zijn op bezoek geweest bij INK international,
waar men bezig is met het ontwikkelen van software ten behoeve van
(professionele) vertalers. Incidenteel wordt ook ander werk gedaan; te denken 
valt aan zaken als de -eventuele- opdracht van de HIS om woordenboeken te 
maken voor de VIDEO WRITER. Taalkundig gezien zijn de werkzaamheden bij INK 
niet verschrikkelijk interessant; het betreft voornamelijk werk 
aan simpele woordenboeken, een eenvoudige morfologie, en `statische' 
programmatuur ( het bepalen van de frequentie van woorden in een tekst e.d.).

Binnenkort zal door INK het nieuwe tijdschrift {\em Language Technology} op de 
markt worden gebracht. Bij hun bezoek hebben Jan en Harm kennis gemaakt met 
een redakteur van 
dit blad die plannen had om iets over ROSETTA te gaan schrijven in een van de
eerste nummers.
 
  \item {\bf Jan L.} meldt dat onlangs de heren Nooteboom en van Leeuwen 
van het IPO op bezoek zijn geweest; Jan L. en Jan O. hebben met hen gesproken. 
De heren waren ge\"{\i}nteresseerd in onze morfologische analyse. Van Leeuwen
werkt momenteel bij het IPO aan een grafeem-foneem omzetter.

  \item Bij {\bf Jan L.} zijn de heren Pennings (van het NOBIN, 
tegenwoordig NBBI) en Bekhof (Economische Zaken) op bezoek geweest om te praten
over het `meerjarenplan automatisch vertalen', dat binnenkort aan de 
minister zal worden voorgelegd. Het meerjarenplan omvat een twintigtal
projekten. Jan 
heeft getracht binnen het plan meer aandacht te vestigen op fundamenteel 
onderzoek, en een betere verdeling te krijgen van het werk aan woordenboeken. 
Dat laatste omvat nu drie projekten: \'{e}\'{e}n voor het maken van een vrij 
klein (circa 10.000 woorden) geformaliseerd woordenboek, \'{e}\'{e}n voor het 
maken van een groot woordenboek voor
menselijke gebruikers, en een statistisch onderzoek. Jan heeft voorgesteld beide
eerste projekten samen te voegen en het laatste te schrappen. Door zo de 
woordenboek-aktiviteiten te bundelen zou een woordenboek gemaakt kunnen worden,
dat groot {\em \`{e}n} geformaliseerd is.

\end{enumerate}

\section {Besproken en/of nieuw verschenen documenten}

\begin{itemize}
  \item {\bf besproken}: {\em niets.}
  \item {\bf verschenen}: {\em niets (behalve de diverse notulen).}
\end{itemize}

\section {Kort overzicht van ieders werkzaamheden}

{\em Geen toelichting.}

\section {Rondvraag en sluiting}
\begin{enumerate}
   \item {\bf Franciska} vraagt of de door de RUU aan een aantal 
lingu\"{\i}sten gestuurde brief betreffende de inventarisatie (door ZWO) 
van het lopend wetenschappelijk onderzoek niet middels \'{e}\'{e}n 
gemeenschappelijke reaktie afgehandeld kan worden. Niemand ziet hier een bezwaar
in; Franciska zal het afhandelen.
   \item {\bf Carel} maakt zich (naar aanleiding van een recent in de 
Volkskrant verschenen artikel over DLT) zorgen over de naar zijn oordeel 
geringe aandacht van de publieke media voor ROSETTA en hij vraagt zich af of 
hier niet wat aan gedaan moet worden. Jan L. deelt mede dat het beleid bij 
Philips erop gericht is onderzoek pas in de publiciteit te brengen als er 
resultaten zijn, en {\em niet} al halverwege het onderzoek. Bovendien wordt
het contact met de media verzorgd door de Philips Persdienst; het is dus niet de
bedoeling dat ons projekt zelfstandig contact legt met de pers. Jan L. zal 
hierover contact opnemen met Loek.
\end{enumerate}
\end{itemize}
\end{document}
