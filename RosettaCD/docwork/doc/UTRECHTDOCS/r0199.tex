\documentstyle{Rosetta}
\begin{document}
   \RosTopic{General}
   \RosTitle{Notulen Rosetta vergadering 27-4-87}
   \RosAuthor{Margreet Sanders}
   \RosDocNr{0199}
   \RosDate{April 27,1987}
   \RosStatus{approved}
   \RosSupersedes{-}
   \RosDistribution{Project}
   \RosClearance{Project}
   \RosKeywords{minutes}
   \MakeRosTitle

\begin{description}
\item[Aanwezig:] Lisette Appelo, Carel Fellinger, Natalia Grygierczyk,
                 Chris Hazenberg, Franciska de Jong, 
                 Lilian Kopinga, Jan Landsbergen, Ren\'{e} Leermakers, 
                 Jeroen Medema, Elly van Munster, 
                 Jan Odijk, Joep Rous, Margreet Sanders (not),
                 Andr\'{e} Schenk, 
                 Harm Smit, Jan Stevens
\item[Afwezig:]  Ans Post
\item[Agenda:]
  \begin{enumerate}
  \item Opening en notulen
  \item Mededelingen
  \item Verslag Conferenties
  \item Besproken en/of nieuw verschenen documenten
  \item Rondvraag en sluiting
  \end{enumerate}
\end{description}

\section{Opening en notulen}
De notulen van de vorige vergadering worden met enkele kleine wijzigingen 
aangenomen.

\section{Mededelingen}
\begin{enumerate}
  \item {\bf Jan L.} en Loek hebben een bezoek gebracht aan de heren De Hoog en 
Berkhof van het Nat. Lab. Geldrop om hun belangstelling voor natuurlijke 
taal\-ver\-werkingssystemen te peilen. In principe zijn er twee 
toepassingsgebieden:
\\
a. CARIN (Car Information); probleem is dat dit EUREKA-project bedoeld is om de 
veiligheid te bevorderen, en dat strookt misschien niet met bv. het spuien van 
toeristische informatie, al dan niet als vertaling van een locale omroep.\\
b. in workstations voor kantoorautomatisering (van brieven vertalen tot omgaan 
met het operating system in natuurlijke taal). Nader overleg hierover volgt 
nog. Als er interesse is, gaat het op het budget van Geldrop.
  \item {\bf Jan L.} en {\bf Harm} bezochten Gerard Kempen en Koen de Smedt 
in Nijmegen, die werken aan:
\begin{enumerate}
  \item een ESPRIT-project over dialoog-systemen
  \item auteursystemen (hulp bij teksten schrijven; van spellingscorrectie van 
namen tot informatie over grammaticale structuren voor luxe editing)
  \item een SPIN-project voor taalonderwijs (o.a. spellingshulp)
\end{enumerate}
Er zijn nog meer plannen, o.a. voor samenwerking met Tilburg (Bunt en Oc\'{e}:
mens-machine interface), met een koppeling van auteur- en Q/A-systemen. Het 
wordt dus steeds moeilijker om een `gat in de markt' te vinden voor mogelijke 
Rosetta-uitbreidingen. Kempen wil best met ons in zee in een nieuw 
SPIN-project.\\
In Nijmegen wordt gewerkt aan een {\em top 10.000} van frekwente woorden, die 
geheel geanalyseerd kunnen worden. In principe zou deze lijst als 
test-woorden\-boek voor Rosetta gebruikt kunnen worden. Basis voor de lijst 
vormden de top van Uittenbogaard, de f-gemarkeerde woorden uit Van Dale en de 
woordenschat van een 6-jarige.
  \item {\bf Jan L.} en {\bf Harm} zijn ook bij CELEX (Centre for Lexical 
Information) in Nijmegen geweest. De heren Kerkman en K\"{o}sters gaven 
informatie over de stand van zaken in hun woordenboek-analyse. In 1988 zal deze 
beschikbaar komen, met morfologische informatie, afbreekstreepjes, klemtoon 
etc. Het bestand beslaat nu 110.000 woorden. Ook voor het Engels moeten er twee 
woordenboeken worden opgenomen (Collins en Longman). Voordeel van CELEX wordt 
misschien dat hun produkten niet meer onder het auteursrecht vallen.
Na 1988 is sponsoring van het bedrijfsleven nodig, liefst voor 
grootschaliger onderzoeken. Jan L. hoopt dat niet alleen het Nat.Lab. maar ook 
Philips als geheel hieraan wil bijdragen. 
  \item {\bf Carel} doet verslag van de SPICOS-demonstratie op het IPO. De drie 
componenten in dit systeem (spraakherkenning, Q/A-database, spraaksynthese) 
zijn op zich niet uiterst geavanceerd, maar de integratie ervan in \'{e}\'{e}n 
systeem wel. De verwerking van een beperkt aantal zinnen ($\pm$ 200) loopt 
redelijk, hoewel de herkenning nog erg lang duurt (1/2 uur per zin). 
\end{enumerate}
\section{Verslag conferenties}
\begin{enumerate}
\item {\bf Franciska} doet verslag van het {\em Amsterdam-colloquium}. De sprekers 
deden te weinig moeite hun verhalen begrijpelijk te maken, zodat het geheel een 
erg vage indruk maakte. Thema's waren o.a. Discourse Representatie Theorie en 
asymmetrische quantificatie. Dit keer werd in de discussies geen rode draad 
duidelijk. Jeroen G. heeft zijn praatje 
niet hier, maar in Nijmegen gehouden. De {\em proceedings} zijn binnen 
afzienbare tijd (juni/juli) te verwachten.
\item {\bf Lisette} was op de {\em Philips A.I.-conferentie} in Arundel (Gr.Br.).
De vertegenwoordigers uit {\em Groot-Brittanni\"{e}} leken weinig op 
implementatie gericht. Het niveau waarop ze 
werken bleef vaag. Ook {\em Brussel} was meer fundamenteel bezig.
Voor {\em Hamburg} en het {\em IPO} waren er praatjes over SPICOS. Hamburg zelf 
deed degelijke research
maar had weinig theoretische diepgang. Waar in {\em Amerika} nu precies 
aan gewerkt wordt is onduidelijk gebleven. Ons eigen {\em Nat. Lab.} hield 
voordrachten over DOOM, PRISMA, Oscar (de robot uit de groep Kreuwels) en 
Boltzmann machines (groep Van Utteren).\\
Het programma was overladen (10-11 lezingen per dag), en de sfeer wat mat. Jan 
L. oppert voor volgend jaar wat meer voordrachten van groepsleiders om het 
{\em beleid} van de diverse groepen beter te profileren.
\end{enumerate}

\section{Besproken en/of nieuw verschenen documenten}
\begin{itemize}
\item {\bf besproken} in de linguistenvergadering: Joep Rous: A solution for 
the `on\-oplosbaar' problem (doc. 0189). Het voorstel om partikels voortaan in 
de segmentatie-regels te scheiden is aangenomen.
\item {\bf verschenen:}
\begin{enumerate}
\item Chris Hazenberg: Auxiliary Domain Compiler (doc. 0188). Van dit document 
verschijnt een nieuwe versie.
\item Joep Rous: Conversion Problems with B-LEX (doc. 0196). Elke keer als 
domein T veranderd wordt, kunnen de woordenboeken niet meer worden geaccesseerd 
zonder conversie. Jan L. zal het `woordenboeken-comit\'{e}' bijeenroepen om 
over dit probleem te praten.
\end{enumerate}
\end{itemize}

\section{Rondvraag en sluiting}
Er zijn geen punten voor de rondvraag, zodat de vergadering wordt gesloten.
De bespreking van doc. 0180 (Joep: Control Grammars) geschiedt in select 
gezelschap na de vergadering.

\end{document}
