


\documentstyle{Rosetta}
\begin{document}
   \RosTopic{Rosetta3.doc.linguistics.Dutch}
   \RosTitle{Dutch M-rules:subgrammar DETPformation}
   \RosAuthor{Franciska de Jong}
%Lisette Appelo}
   \RosDocNr{484}
   \RosDate{December 4, 1991}
   \RosStatus{approved}
   \RosSupersedes{}
%concept of September 4, 1989}
   \RosDistribution{Project}
   \RosClearance{Project}
   \RosKeywords{Dutch, M-rules, DETP}
   \MakeRosTitle
%
%
\input{[dejong.mrules]mrudocdef}
\input{[dejong]definitions}


\section{Introduction}

This document describes the subgrammar DETPformation for Dutch.
The subgrammar is relevant to DETPs with as a head expressions of the 
category:\\
\begin{itemize}
  \item QP
  \item DET
  \item DEMADJ
  \item NUM
  \item CARDINAL
\end{itemize}

\noindent
DETPs with an article or a possessive expression as its head are dealt with by 
NPformation4, NPformation5, NPpartitiveformation1 and RNPpartitiveformation2. 
Cf. doc:r480 (FdeJ).



\section{Subgrammar Specification}

The formation of NPs is dealt with by 
the subgrammar NPformation:\\
\begin{description}
  \item[Head] DET, NUM, DEMADJ, QP, CARDINAL
  \item[Export] DETP
  \item[Import] DETP
%
  \item[File] dutch:npsubgrammars.mrule (mrules67)
\end{description}

\section{Control Expression}
The control expression has been  defined as follows:
\begin{verbatim}

(
 (
    ( DETPFORMATION1/1  | DETPFORMATION2/2  | DETPFORMATION4/4 |
      RcardToDetp/7
    )
   . RDETPsuperdeixis/6 
 ) | ( DETPFORMATION3/3 | DETPformation3a/9 )
)

. [RDETPpartitiveformation/5]


HEAD:
<
QP               FROM (QPFORMATION)
DET              FROM (DETDERIVATION)
NUM              BASIC EXPRESSION
DEMADJ           BASIC EXPRESSION
>


\end{verbatim}
\section{Planned extension}
There are no rules written yet that account for the modification of DETPs.
Such rules are needed to deal with examples such as:\\

\begin{lxam}
&&& {\em al} die boeken\\
&&& {\em bijna} twee boeken\\
&&& de {\em ongeveer} honderd boeken\\
&&& {\em rond de} honderd boeken\\
&&& {\em om en nabij de} honderd boeken\\
\end{lxam}


\noindent
The rules should be related to the formation of partitives 
in order to account for the relation between 
{\em al die boeken} and {\em all of these books}.


\begin{mruleclass}{RC\_DETPformationI}
\begin{classdescr}
\kind in combination with RC\_DETformationII: obligatory rule class
\classtask The formation of a DETP level
\classremarks

\nofilters

\nospeedrules

\noplannedrules

\norulesnotince

\rulelist

\end{classdescr}

\begin{members}
\begin{member}
\rulename detpformation1
\ruletask making a DETP-node out of a DET
\file dutch:detprules.mrule (mrules44)
\semantics LDETPformation
\example {\em alle, elke, veel, sommige, menig(e), verschillende}, etc.
\remarks 
The value for DETP attribute $.superdeixis$ is determined by a separate 
RC\_DETPsuperdeixis. Cf. below.

\end{member}
\begin{member}
\rulename DETPformation2
\ruletask making a DETP-node out of a NUM
\file dutch:detprules.mrule (mrules44)
\semantics LDETPformation
\example ��n, twee, drie, etc. (i.e. all cardinals)
\remarks 
The value for DETP attribute $.superdeixis$ is determined by a separate 
RC\_DETPsuperdeixis. Cf. below.
\end{member}
\begin{member}
\rulename RCARDtoDETP
\ruletask making a DETP-node out of a CARDINAL
\file dutch:detprules.mrule (mrules44)
\semantics 
LDETPformation
\example 4, 16, 1991
\remarks 
The value for DETP attribute $.superdeixis$ is determined by a separate 
RC\_DETPsuperdeixis (see below).

\end{member}
\begin{member}
\rulename DETPformation4
\ruletask
Creating a DETP-node out of a DEMADJ.
\file dutch:detprules.mrule (mrules44)
\semantics LDETPformation
\example 
{\em dit, dat, deze, die}
\remarks 
\begin{enumerate}
\item 
The value for DETP attribute $.superdeixis$ is determined by a separate 
RC\_DETPsuperdeixis. Cf. below.
\item 
$dit/dat$: DETP.eorenformation = Noform)\\
  $deze/die$: DETP.eorenformation = eForm)
\end{enumerate}

\end{member}
\end{members}

\end{mruleclass}

\begin{mruleclass}{RC\_DETPsuperdeixis}
\begin{classdescr}
\kind rule class which is obligatory in case of a DETP not dominating a QP
\classtask Assigning a superdeixis value to DETPs.
\classremarks

\nofilters

\nospeedrules

\noplannedrules

\norulesnotince

\rulelist

\end{classdescr}

\begin{members}

\begin{member}
\rulename RDETPsuperdeixis
\ruletask Assign the superdeixis value of the parameter to the DETP node 
in generation and give the the parameter the value of the superdeixis attribute 
of the DETP node  in analysis.
\file dutch:detprules.mrule (mrules44)
\semantics LSuperdeixis
\example all DETPs not dominating a QP
\remarks No remarks
\end{member}
\end{members}

\end{mruleclass}


\begin{mruleclass}{RC\_DETPformationII}
\begin{classdescr}
\kind in combination with RC\_DETformationI: obligatory rule class
\classtask The formation of a DETP level for QPs
\classremarks
QPs are specified for superdeixis via RQPsuperdeixis of sg QPformation. 
Therefore RC\_DETPformationII is a separate rule class 
ordered after RC\_DETPformationI and RC\_DETPsuperdeixis.

\nofilters

\nospeedrules

\noplannedrules

\norulesnotince

\rulelist

\end{classdescr}

\begin{members}

\begin{member}
\rulename DETPformation3
\ruletask Making a DETP out of a QP, either complex or not
\file dutch:detprules.mrule (mrules44)
\semantics LDETPformation
\example {\em genoeg,  meer, meer dan ooit}
\remarks
\begin{enumerate}
\item
QP $meest$ is dealt with by DETPformation3a
\item
The determiners
{\em veel} en {\em weinig} are exluded from DETPformation via RDETPformation3
by condition DETP.$posspred$ =true; they become DETP via RDETPformation1.
For the treatment of non-determiner QP 
{\em veel} and {\em weinig}, cf. doc.r483.
\end{enumerate}

\end{member}
\begin{member}
\rulename DETPformation3a
\ruletask Making a complex DETP with an 
article out of a simple QP headed by {\em meeste}
\file dutch:detprules.mrule (mrules44)
\semantics LDETPformation
\example {\em meeste} $\rightarrow$ {\em de/het meeste}
\remarks No remarks
\end{member}
\end{members}
\end{mruleclass}


\begin{mruleclass}{RC\_PartitiveDETPformation}
\begin{classdescr}
\kind optional rule class
\classtask The formation of a complex DETP.
\classremarks

\nofilters

\nospeedrules

\noplannedrules

\norulesnotince

\rulelist

\end{classdescr}

\begin{members}


\begin{member}
\rulename RDETPpartitiveformation
\ruletask The formation of a Partitive DETP out of two DETPs 
(DET$_1$ and DET$_2$).
\file dutch:detprules.mrule (mrules44)
\semantics LDETPpartitiveformation
\example 

\begin{enumerate}
  \item {\em elk} + {\em die} $\rightarrow$ {\em elk van die}
  \item {\em een} + {\em beide} $\rightarrow$ {\em een van beide}
\end{enumerate}

\remarks
\begin{enumerate}
  \item
Partitive DETPs with a definite article or a possessive expression as DET$_2$
are dealt with by RC\_NPpartitiveformation. This rule class accounts for the 
syncategorematic introduction of articles, and of the conversing of NPs in 
postnominal posrel-position into a prenomninal determiner position.
  \item
The restrictions on the DET$_1$ and DET$_2$ have been specified
only partially. For example, there are no means yet to exclude
examples such as *{\em alle/beide van die}.
\end{enumerate}
\end{member}
\end{members}
\end{mruleclass}


\end{document}
