
\documentstyle{Rosetta}
\begin{document}
   \RosTopic{General}
   \RosTitle{Notulen linguistenbijeenkomst d.d. 15-10-1987}
   \RosAuthor{Harm Smit}
   \RosDocNr{0235}
   \RosDate{October 20, 1987}
   \RosStatus{concept}
   \RosSupersedes{-}
   \RosDistribution{Linguists, Joep Rous}
   \RosClearance{Project}
   \RosKeywords{notulen}
   \MakeRosTitle
%
%

\begin{itemize}
  \item {\bf aanwezig}: Andr\'{e} Schenk, 
             Lisette Appelo, Jan Odijk, Franciska de Jong, Margreet Sanders,
             Elly van Munster, Harm Smit.
  \item {\bf afwezig}: -
\end{itemize}
\section{Notuleren}
Wegens notuleer-moeheid stopt Lisette met notuleren; Harm werd bereid gevonden
de volgende drie vergaderingen te notuleren (waarvan dit de eerste keer is).
\section {Derivatie}
Hierover is nog niet echt nagedacht door de aanwezigen. Voorlopig blijft de
derivatie-subgrammatica nog bestaan maar we doen er niets mee. De startregels
blijven voorlopig zoals ze zijn. Voor PREP's moet iets speciaals gedaan worden.
\section{Semantiek}
Sommige attributen of attribuut-waarden hebben eigenlijk betrekking op de 
semantiek, zoals bijvoorbeeld:
\begin{itemize}
  \item thetavp
  \item polarity (met de waarden {\em req} en {\em env})
  \item `semantische' condities op o.a. passivatie
  \item aktionsart
\end{itemize}
Het syntactisch beregelen van eigenschappen van semantische aard leidt tot
ongewenste -en in de syntaxis irrelevante- ambiguiteiten. Zo moet op grond van
het bestaan van het woord {\em moer} in het idioom {\em hij snapt er geen moer 
van} (waarbij {\em moer} altijd samen met een ontkennend element voorkomt), de
expressie {\em moer} ambigu zijn in de syntaxis terwijl het verschil alleen in 
de semantiek relevant is.

Bij {\em thetavp} komen problemen voor m.b.t. zinnen als {\em hij brengt het 
boek naar Amsterdam} vs. {\em hij brengt het boek}, waarbij we met een 
drieplaatsige resp. tweeplaatsige functie te maken hebben.
Een oplossing zou zijn om voor de PP een empty toe te staan. Een andere 
oplossing zou zijn om in de syntaxis \'{e}\'{e}n functie te hebben, die in de 
semantiek op twee (een twee- resp. drieplaatsige) functies wordt afgebeeld.
Nadeel hiervan is dat de \'{e}\'{e}n op \'{e}\'{e}n relatie hierbij wordt 
losgelaten.

Jan Odijk licht in het kort een voorstel van hem m.b.t. tot deze materie toe; 
hij zal dit in een document uitgebreider toelichten. Voor Rosetta3 zal dit 
echter wellicht niet meer van belang zijn, maar misschien is het iets voor 
Rosetta4.
\section{Testwoordenboek}
Voor het testwoordenboek hebben Jan O. en Harm {\em ranges} aan de 
verschillende categorieen toegekend. Joep heeft al een consistentie-programma 
gemaakt waardoor reeds drie fouten uit het testwoordenboek zijn gehaald. 
Binnenkort kan het vullen van attributen door eenieder beginnen.
\section{Kennisoverdracht}
Franciska begint met het overzicht van haar werk.
\end{document}
