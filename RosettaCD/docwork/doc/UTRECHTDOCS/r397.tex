\documentstyle{Rosetta}
\begin{document}
   \RosTopic{Rosetta3.doc.linguistics.Dutch}
   \RosTitle{Dutch M-rules:subgrammar ADJPPROPtoADJPFORMULA}
   \RosAuthor{Franciska de Jong, Lisette Appelo}
   \RosDocNr{397}
   \RosDate{August 16, 1990}
   \RosStatus{approved}
   \RosSupersedes{concept of August 31, 1989}
   \RosDistribution{Project}
   \RosClearance{Project}
   \RosKeywords{Dutch, M-rules, AdjpFormulaFormation}
   \MakeRosTitle
%
%
\input{[dejong.mrules]mrudocdef}

\section{Introduction}
In this document the subgrammar ADJPPROPtoADJPFORMULA for Dutch is described.
The parts on temporal aspect, retrospectivity, deixis and superdeixis
have been written by 
Lisette Appelo, the other parts have been written by Franciska de Jong.

In the relevant aspects, this grammar has been attuned to the 
subgrammar XPPROPtoClause. It is not relevant to {\bf all} 
adjectival constructions, but only to the ones that end up  
either as ADJP-utterance or as NP-internal constituent, and to the ones that
eventually -as the result of the subgrammar ADJPFormulato ADJPPROP- are 
imported 
into the subgrammar XPPROPtoClause via substitution.


\section{Subgrammar Specification}

\begin{description}
  \item[Head] ADJPPROP
  \item[Export] ADJPFORMULA
  \item[Import] 
ADJPPROP, EMPTY, SENTENCE, all the
variables in
VARCATSET (expected: ADVPVAR, PREPPVAR, SENTENCEVAR).
  \item[File] dutch:adjsubgrammars.mrule (mrules48)
\end{description}

\section{Control Expression}
The control expression can be defined as follows:
\begin{verbatim}

  {RC_ADJSentenceSubstitution}

. RC_ADJPFORMULAformation

. RC_ADJAspect

. RC_ADJdeixis

. TC_ADJsuperdeixisadaptation

. {TC_ADJsuperdeixisadaptation}

. [TC_ADJdeixisadaptation]

. TC_ADJcontrol

. TC_ADJObjectok

. {RC_ADJEMPTYsubstitution}

. {TC_ADJCaseAssignment}

. {TC_ADJargreflreciprspelling} (* not activated; filters must 
                                   still be added *) 

. [TC_ADJSENTextraposition]

. [TC_ADJCOMPLextraposition]

. [TC_ADJPREPPextrapositionA]

. [TC_ADJPAcomplmovement]

. [TC_ADJPREPPextrapositionB]

. [TC_ADJTETOTEZEER]  (* plannend *)

. TC_ADJcomparativeforms

. TC_ADJsuperlativeforms

\end{verbatim}

\section{Rules and Transformations}
\begin{mruleclass}{RC\_ADJSentenceSubstitution}
\begin{classdescr}
\kind iterative rule class
\classtask to substitute sentencevars for SENTENCEs
\classremarks

\nofilters

\begin{speedrules}
\begin{member}
\rulename FADJpreSentencesubst
\ruletask To speed up analysis for testing purposes. 
The filter blocks paths for ADJPPROP 
structures 
that contain a complement SENTENCE. 
\file dutch:rcs\_adjsubst.mrule (mrules38)
\remarks
Note that this is incompatible with the
occurrence of idioms that contain propositional complements. 
As long as there are no examples found of idiomatic sentential 
structures within adjectival contexts, this rule may be 
considered to have a permanent character.
\end{member}
\end{speedrules}


\begin{plannedrules}
\item 
Substitution rules for sentencevars with a PRO-object.
Example: {\em dit boek is leuk om te lezen}
\item 
An additonal substitution rule for sentencevars 
is still to be written in order to derive
sentences such as {\em dat het alsmaar regent wordt onderhand vervelend}, 
that do not allow extraposition, nor the introduction of DEMPRO {\em dat}. 
As M-grammar cannot deal with these cases
yet, this is postponed. Cf. doc:r314 (documentation of 
subgrammar XPPROPtoCLAUSE).
\end{plannedrules}

\norulesnotince
\rulelist
\end{classdescr}

\begin{members}
\begin{member}
\rulename RADJSENTENCESUBSTITUTION1
\ruletask Substitution of non-subject SENTENCEVARs
(whether argument or modifier),
excluding omteinf-complements and omteinf-modifiers. There is a separate rule 
for omteinf-complements; omteinf-modifiers are not introduced via a VAR. For
the latter, cf. doc r376.

\file dutch:rcs\_adjsubst.mrule (mrules38)
\semantics proposition substitution
\example\mbox{}
\begin{enumerate}
  \item Ik ben blij dat het regent (complrel)
  \item Het is onduidelijk wie er gewonnen heeft (NB:ergative, so complrel too)
  \item ?? De hoewel hij al jaren ziek was niet afgestompte man (sentadvrel)
\end{enumerate}


\remarks\mbox{}
\begin{enumerate}
  \item 
Sentential complements to a modifier (e.g. {\em genoeg} are introduced
directly (NOT via
substitution) in 
other subgrammars, e.g. QPformation. 
  \item
Sentential constituents under omtemodrel and resultrel are substituted in 
subgrammars that derive degree-modifiers, e.g. QPformation.
\end{enumerate}
\end{member}
\begin{member}
\rulename RADJSENTENCESUBSTITUTION2
\ruletask Substitution of SENTENCEVAR (.infsort is not omteinf) 
in subjrel: {\em het} is introduced
in subjrel, the substituent SENTENCE in complrel.
\file dutch:rcs\_adjsubst.mrule (mrules38)
\semantics proposition substitution
\example\mbox{}
het werd vervelend dat hij zo ziek was
\remarks\mbox{}
In analysis both {\em het } and {\em 't } are accepted in subjrel; 
only {\em het} is 
generated.

\end{member}
\begin{member}
\rulename RADJSENTENCESUBSTITUTION3
\ruletask 
Substitution of SENTENCEVAR (.infsort is not omteinf) in subjrel: DEMPRO 
{\em dat} is introduced
in subjrel, the substituent SENTENCE in leftdislocrel.
\file dutch:rcs\_adjsubst.mrule (mrules38)
\semantics proposition substitution
\example\mbox{}
x1 vreemd $\rightarrow$ dat de wereld ronddraait(,) dat lijkt vreemd

\remarks\mbox{}
In analysis, both variants with, and variants without a comma preceding {\em 
dat} are accepted. In generation the comma is always introduced.
\end{member}
\begin{member}
\rulename Radjsentenceomtecompl
\ruletask  Substitution of complement SENTENCEVARs
with .infsort = omteinf.
\file dutch:rcs\_adjsubst.mrule (mrules38)
\semantics proposition substitution
\example\mbox{}
zij is bang om alleen te reizen
\remarks\mbox{}
In analysis both complement 
sentences with {\em te} as with {\em om te} are accepted.
In generation only variants with {\em te} are derived.

\end{member}
\begin{member}
\rulename Radjsentenceomtesubj
\ruletask  Substitution of SENTENCEVARs in subjrel,
with .infsort = omteinf: {\em het} is introduced
in subjrel, the substituent SENTENCE in complrel.
\file dutch:rcs\_adjsubst.mrule (mrules38)
\semantics proposition substitution
\example\mbox{}
het bleek moeilijk (om) op tijd weg te komen
\remarks\mbox{}
\begin{enumerate} 
  \item
In analysis both complement 
sentences with {\em te} as with {\em om te} are accepted.
In generation only variants with {\em te} are derived.
  \item
In analysis both {\em het } and {\em 't } are accepted in subjrel; 
only {\em het} is 
generated.
\end{enumerate}
\end{member}
\end{members}
\end{mruleclass}

\begin{mruleclass}{RC\_ADJPFORMULAformation}
\begin{classdescr}
\kind obligatory rule class
\classtask to replace the topnode ADJPPROP by ADJPFORMULA. 
\classremarks \mbox{}
The rule class has not the status of transformation class 
because of historical reasons (When it was written, the corresponding
rule class in Spanish was meaningful. It introduced the distinction between 
{\em ser} and {\em estar}).

\nofilters
\nospeedrules
\rulelist
\end{classdescr}

\begin{members}
\begin{member}
\rulename RADJPFORMULAFORMATION1
\ruletask replacement of the topcategory ADJPPROP by ADJPFORMULA. 
\file dutch:rcs\_adjsubst.mrule (mrules38)
\semantics \nosemantics
\example all adjpprops
\remarks\mbox{}
\end{member}

\end{members}

\end{mruleclass}
\begin{mruleclass}{RC\_ADJAspect}
\begin{classdescr}
\kind obligatory rule class
\classtask 
Give attribute {\em aspect} a value.
\classremarks
This class is necessary for isomorphy reasons. All ADJPPROPs have imperfective 
aspect. See also doc. R263.
\nofilters

\nospeedrules

\noplannedrules

\norulesnotince
\rulelist
\end{classdescr}
\begin{members}
\begin{member}
\rulename RADJaspectimperfective
\ruletask 
imperfective aspect relation between the interval E and a 
reference interval R.
The attributre {\em aspect} gets the  value {\em imperfective}.
\file dutch:tempadj1.mrule (mrules79)
\semantics 
imperfective aspect relation between the interval E and a 
reference interval R.
\example\mbox{}
 ADJPFORMULA(x$_{1}$ ziek)$_{omegaaspect}$ $\leftrightarrow$\\
ADJPFORMULA(x$_{1}$ ziek)$_{imperfective}$
\remarks\mbox{}

\end{member}

\end{members}

\end{mruleclass}


\begin{mruleclass}{RC\_ADJdeixis}
\begin{classdescr}
\kind obligatory rule class
\classtask To deal with deixis and superdeixis. Cf. also doc. R263.
\classremarks

The superdeixis values are also relevant 
in the case 
of a sentence with {\em graag}.


\nofilters

\nospeedrules

\noplannedrules

\norulesnotince

\rulelist
\end{classdescr}

\begin{members}
\begin{member}
\rulename RADJpresentsuperdeixis
\ruletask The relation between Rs and S is simultanuous: PRESENT.
               This rule is for a dependent ADJPFORMULA.
               The deixis value of the possible reference adverbial is checked 
               for omega.
               The ADJPFORMULA is marked for present superdeixis.
\file dutch:tempadj2.mrule (mrules80)
\semantics The `indirect' relation between Rs and S is simultanuous: PRESENT.

\example\mbox{}
\begin{enumerate}
  \item 
de (zieke) man
  \item
                            de (om 3 uur/op zaterdag zieke) man
\end{enumerate}
\remarks\mbox{}

\end{member}
\begin{member}
\rulename RADJpastsuperdeixis
\ruletask The relation between Rs and S is before: PAST.
               This rule is for a dependent ADJPFORMULA.
               The deixis 
value of the possible reference adverbial is checked for 
               omega.
               The ADJPFORMULA is marked for past superdeixis.
\file dutch:tempadj2.mrule (mrules80)
\semantics The `indirect' relation between Rs and S is before: PAST.

\example\mbox{}
\begin{enumerate}
  \item 
 ik zag de (zieke) man
  \item
ik zag de (met Kerstmis zieke) man
\end{enumerate}
\remarks\mbox{}

\end{member}

\end{members}
\end{mruleclass}

\begin{mruleclass}{TC\_ADJsuperdeixisadaptation}
\begin{classdescr}
\kind obligatory transformations; iterative transformation plus filter
\classtask  To check the superdeixis of the embedded propositional phrases and 
sentences with the superdeixis of ADJPFORMULA and adapt the superdeixis (and 
deixis) of the embedded sentences on behalf of the surface parser. 
\classremarks
For further 
explanation cf. doc. R314, section 5.8. 
Superdeixisadaptation is also necessary for 
modifiers here, because modification rules are ordered before the (super)deixis 
rules.
Cf. also doc. R263 and R320.

\begin{filters}
\begin{member}
\rulename FADJsuperdeixisadaptation2
\ruletask To guarantee the correct application of TADJSuperdeixisAdaptation2.
\file dutch:tempadj2.mrule (mrules80)
\end{member}
\end{filters}

\nospeedrules

\noplannedrules

\norulesnotince

\rulelist
\end{classdescr}

\begin{members}
\begin{member}
\rulename TADJsuperdeixisadaptation1
\ruletask Check superdeixis of a QP-, NP-, ADVP- or PREPP-modifier or a
complement sentence 
in case of temporally dependent complement 
sentences with superdeixis or 
deixis of the higher ADJPFORMULA. The superdeixis of the complement sentence or 
the QP-, NP-, ADVP- or PREPP-modifier is set omega/given a 
value present or past, and the deixis value of the complement sentence 
is given a value present or past/ 
set omega in cases necessary. (This is needed for efficiency 
reasons of the surface parser.) For temporally independent sentences this rule 
does nothing. \\
\file dutch:tempadj2.mrule (mrules80)
\semantics \nosemantics
\example\mbox{}
de ((erg$_{superdeixis=presentdeixis}$) zieke$_{superdeixis=presentdeixis}$) man
$\leftrightarrow$\\
de ((erg$_{superdeixis=omegadeixis}$) zieke$_{superdeixis=presentdeixis}$) man
\remarks\mbox{}

QP, NP, ADVP and PREPP as modifiers of an adjective should receive 
a  value for superdeixis in analysis and checked + set omega in generation, 
because the modification rules are ordered (incorrectly) before the (super)
deixis rules.

\end{member}
\begin{member}
\rulename TADJnosuperdeixisadaptation
\ruletask Default rule for superdeixisadaptation in the case there was no 
proposition substitution or some kind of degree modification or a voorobj 
modifier.
\file dutch:tempadj2.mrule (mrules80)
\semantics \nosemantics
\example
\remarks\mbox{}

\end{member}
\begin{member}
\rulename TADJsuperdeixisadaptation2
\ruletask Iterative rule to adapt the superdeixis value of sentential 
modifiers with postmodrel, postadjrel and omtemodrel.
\file dutch:tempadj2.mrule (mrules80)
\semantics \nosemantics
\example sterk als een beer, handig om mee te nemen, genoeg.
.. om mee te nemen, te... om mee te nemen.
\remarks\mbox{}
Some sentences with omtemodrel are given/checked for 
superdeixis here while this should take place in the ADVP or QP subgrammar 
because they are complements or modifiers that were raised at the moment of 
substitution/modification of their dominating node. This superdeixis value 
should be adapted back again at that moment! The superdeixis adaptation rules 
of ADVP and QP then can take care of the superdeixis value of those SENTENCEs
`legally'. See also doc. R376, sections 3, 6.5.
\end{member}

\end{members}

\end{mruleclass}

\begin{mruleclass}{TC\_ADJcontrol}
\begin{classdescr}
\kind obligatory transformation class
\classtask to deal with control, 
i.e. the interpretation and deletion of subjects of 
infinitival sentences, both in complements and in modifying sentences.

\classremarks\mbox{}
\begin{enumerate}
\item 
This TC has not been tested in all details, as not all paths are complete yet.
For example, there is no complete path for 
infinitives with a so called pro-object yet. 
\item
The distinction between obligatory control and non-obligatory control that 
plays a role in the XPPROPtoCLAUSE subgrammar is assumed to be irrelevant here.
\item

The combination of an extraposed infinitival subject 
and an infinitival complement 
to a degree modifier seems blocked: {\em 
 *Het is voor mij te moeilijk om te bereiken
 om  op tijd te komen} versus 
{\em ?Om op tijd te komen is (voor mij) te moeilijk om te bereiken}.
\end{enumerate}

\nofilters

\nospeedrules

\begin{plannedrules}
\item 
Rules for infinitival 
modifiers to {\em te} and {\em genoeg} with a non-subject antecedent. 
(These examples involve the interpretation of empty objects as well.
) At least (a merge of) the following rule(s) is needed. 
(Alternatively these rules may be incorporated into TADJcontrol2a/2b/2c.)
Note that for these modifiers the initial CONJ {\em om} is obligatory.
\begin{enumerate}
\item
TADJcontrol5a; example: deze tas leek voor mij te zwaar om mee te nemen
\item
TADJcontrol5b; example: deze tas leek  mij te zwaar om mee te nemen
\item
TADJcontrol5c; example: deze tas leek EMPTY te zwaar om mee te nemen
\end{enumerate}

\item Rule(s) for arbitrary control in case of an omtemodrel/SENTENCE
while there is no voorobj or degreemodifier and the adjective has the value 
{\em subjectiveadj} in .subcs. Probably this rule should 
replace Tadjcontrol2c. 
\item Rules for 
infinitival sentences that are linked to ADJPFORMULA by relation {\em 
leftdislocrel} - if correct examples exist.
\end{plannedrules}

\rulelist
\end{classdescr}

\begin{members}
\begin{member}
\rulename TADJCONTROL0
\ruletask to account for cases that do not involve any kind of control
\file dutch:tc\_adjcontrol.mrule (mrules40)
\semantics \nosemantics
\example all adjectival phrases without infinitival constituents
\remarks\mbox{}

\end{member}

\begin{member}
\rulename TADJCONTROL1
\ruletask to determine the interpretation of the subject of
a sentential complement, in case the antecedent is  subject of ADJPFORMULA

\file dutch:tc\_adjcontrol.mrule (mrules40)

\semantics \nosemantics

\example \mbox{}\\
\begin{enumerate}
  \item 


{\tt subjrel}/x1 + {\tt predrel}/[{\tt complrel}/[{\tt subjrel}/x1 +
om + {\tt predrel}/te verzuipen] + {\tt head}/bang]
$\rightarrow$\\ 
x1 + /[{\tt complrel}/om te verzuipen + {\tt head}/bang] 
\\
(Jan bleek bang om te verzuipen)\\

  \item

{\tt subjrel}/x1 + {\tt predrel}/[(aan) x3 + {\tt complrel}/[{\tt subjrel}/x1 +
om + {\tt predrel}/te komen] + {\tt head}/waard]
$\rightarrow$\\ 
x1 + [(aan) x3 + {\tt complrel}/om te komen + 
{\tt head}/waard] 
\\
(Jan leek het ons waard om te komen)
\end{enumerate}

\remarks\mbox{}


\end{member}
\begin{member}
\rulename TADJCONTROL2a
\ruletask to determine the interpretation of the subject of
a sentential complement, in case the antecedent is a prepositional object 
while the complement contains an empty object that is 
controlled by the subject of ADJPFORMULA. The latter control relation is 
accounted for by TADJPROOBJCONTROL1.
\file dutch:tc\_adjcontrol.mrule (mrules40)
\semantics \nosemantics
\example \mbox{}\\
{\tt subjrel}/x2 + {\tt head}/lastig + {\tt voorobjrel}/voor x1
{\tt postadjrel}/[om + {\tt subjrel}/x1 + {\tt objrel}/x2  + mee te nemen]\\
$\rightarrow$ \\
x2 + lastig + {\tt voorobjrel}/voor x1
{\tt postadjrel}/[om x2 mee te nemen]\\
(Deze tas bleek  lastig voor mij om mee te nemen)

\remarks\mbox{}
It might as well turn out that certain example sentences are
cases with  x1 as  a voorobj to the verb rather than the adjective.
\end{member}
\begin{member}
\rulename TADJCONTROL2b
\ruletask to determine the interpretation of the subject of
a sentential modifier, in case the antecedent is an indirect object 
of ADJPFORMULA,
 while the complement contains an empty object that is 
controlled by the subject of ADJPFORMULA. 
The latter control relation is 
accounted for by TADJPROOBJCONTROL1. 
\file dutch:tc\_adjcontrol.mrule (mrules40)
\semantics \nosemantics
\example\mbox{}\\ 
{\tt subjrel}/x2 + {\tt predrel}/[{\tt indobjrel}/x1  + 
{\tt head}/lastiger + 
{\tt omtemodrel}/[om {\tt subjrel}/x1 + 
{\tt predrel}/[{\tt objrel}/x2 mee te nemen]]]
$\rightarrow$\\
 x2 + [x1 
lastiger [om x2 mee te nemen ]]\\
(Die tas bleek mij lastiger om mee te nemen) \\

\remarks\mbox{} 
\begin{enumerate}
\item
It is questionable whether this kind of control really exists.
Certainly not for all adjectives. May be it 
should be 
constrained to comparative adjectives.
It might as well turn out that all example sentences are
cases with  x2 as  an indirect object to the verb rather than the adjective.
Under this alternative interpretation 
arbitrary control seems to be the more natural interpretation.
  \item 
The rules that introduce indobjs (or the transformations that 
derive them from voorobjs) to adjectives have not been 
written yet.
 \end{enumerate}

\end{member}
\begin{member}
\rulename TADJCONTROL2c
\ruletask to determine the interpretation of the subject of a sentential 
complement, in case the antecedent is an EMPTY indirect object 
of ADJPFORMULA, while the complement contains an empty object that is 
controlled by the subject of ADJPFORMULA .
The latter control relation is 
accounted for by TADJPROOBJCONTROL1.
\file dutch:tc\_adjcontrol.mrule (mrules40)
\semantics \nosemantics
\example\mbox{}\\
{\tt subjrel}/x2 + {\tt predrel}/
[x1(EMPTYVAR) + lastig +
{\tt omtemodrel}/[om + {\tt subjrel}/x1 + [{\tt objrel}/x2 mee te nemen]]]
$\rightarrow$\\
x2 + 
[x1(EMPTYVAR) + lastig  + [om {\tt objrel}/x2 mee te nemen]]\\
(Deze vraag leek  lastig om te beantwoorden)

\remarks\mbox{}
The interpretation of the EMPTY is not an existential quantifier. Either 
it is some
contextually determined object, or it has a generic interpretation.
In the former case it would be preferrable to use a special basic expression: 
e.g. DISCEMPTY. In the latter case there is also the alternative of 
BIGPRO-control (e.g. zegenBIGPRO).  
\end{member}
\begin{member}
\rulename TADJCONTROL3a
\ruletask To determine the interpretation of the subject of a
subject sentence in complrel (with dummy subject). 
The antecedent is a
voorobj in ADJPFORMULA.
\file dutch:tc\_adjcontrol.mrule (mrules40)
\semantics \nosemantics
\example\mbox{}\\
het + [
{\tt complrel}/[om subjrel/x1 te komen] +
{\tt head}/moeilijk + 
{\tt voorobjrel}/voor x1 
$\rightarrow$\\
het + [[om te komen] + moeilijk + voor x1 \\
(Het wordt voor mij moeilijk om te komen)

\remarks\mbox{}

\end{member}
\begin{member}
\rulename TADJCONTROL3b
\ruletask To determine the interpretation of the subject of a
subject sentence in complrel 
(with dummy subject). 
The antecedent 
is a non-empty indirect object 
in ADJPFORMULA.
\file dutch:tc\_adjcontrol.mrule (mrules40)
\semantics \nosemantics
\example\mbox{}
het + [{\tt indobjrel}/x1 + 
{\tt complrel}/[om subjrel/x1 te komen] +
{\tt head}/moeilijk ]
$\rightarrow$\\
het + [ x1 + [om te komen] + moeilijk] \\
(Het lijkt/valt mij mij moeilijk om te komen)

\remarks\mbox{}
Cf. the remarks to TADJcontrol2b. 

\end{member}
\begin{member}
\rulename TADJCONTROL3c
\ruletask to determine the interpretation of the subject of a subject
sentence in complrel (with dummy subject). 
The antecedent is EMPTY (or DISCEMPTY; cf. TADJCONTROL2c), an empty 
indirect object in ADJPFORMULA.
\file dutch:tc\_adjcontrol.mrule (mrules40)
\semantics \nosemantics
\example\mbox{}\\
{\tt subjrel}/het + {\tt predrel}/
[x1 (=EMPTYVAR) + 
{\tt complrel}/[om {\tt subjrel}/x1 
+ {\tt predrel}/[Jan te begrijpen]] + moeilijk] $\rightarrow$ \\

[ het +  x1 (= EMPTYVAR) +  [om Jan te begrijpen] + moeilijk]\\
(Het is moeilijk om Jan te begrijpen)
\remarks\mbox{}
The interpretation of the EMPTY is not an existential quantifier. Either 
it is some
contextually determined object, or it has a generic interpretation.
In the former case it would be preferrable to use a special basic expression: 
e.g. DISCEMPTY. In the latter case there is also the alternative of 
BIGPRO-control (e.g. zegenBIGPRO).  
\end{member}
\begin{member}
\rulename TADJCONTROL4
\ruletask to determine the  interpretation 
of the subject of infinitival modifier of {\em genoeg} and {\em te}.
The antecedent is the subject of ADJPFORMULA.
\file dutch:tc\_adjcontrol.mrule (mrules40)
\semantics \nosemantics
\example\mbox{}
\begin{enumerate}
\item
x1 + [slim + {\tt hoprel}/genoeg + 
{\tt postadjrel}/[om subjrel/x1 op tijd te komen]] $\rightarrow$\\
x1 + [slim genoeg [om op tijd te komen]]\\
(Jan lijkt  slim genoeg om op tijd  te komen)
\item
x1 + [te + mooi + {\tt postadjrel}/[om subjrel/x1 waar te zijn]] $\rightarrow$\\
x1 + [te mooi + [om waar te zijn]] \\
(Dit lijkt te mooi om waar te zijn)
\item
 x1 + [te bang (+ voor onweer) + {\tt postadjrel}/
[om subjrel/x1  op tijd te vertrekken]] $\rightarrow$\\
 x1 + [te bang (+ voor onweer) + 
[om op tijd te vertrekken ]]\\
(Jan bleek te bang (voor onweer) om op tijd te kunnen vertrekken)\\
(The prepositional object of {\em bang} need not be overt!!)
\end{enumerate}
\remarks\mbox{}
\begin{enumerate} 
  \item

Rules for cases with an arbitrary interpretation of the subject of the 
infinitival omtecomplmod, for example (1) and (2), 
will probably apply to the examples mentioned here too.
There is no syntactic feature/property by which this double interpretation 
could be excluded. Possibly selectional restrictions are responsible for a 
choice among the two interpretaions. The ambiguity should be studied in more 
detail as soon as a semantic component is available.

\begin{description}
  \item (1) Het lijkt te warm om te fietsen
  \item (2) De treinreis bleek  te kort om te kunnen slapen
\end{description}
  \item
For testing purposes a special subrule is written that 
relates the EMPTYVAR that occurs as auxiliary subject to BIGPRO.
\end{enumerate}

\end{member}

\end{members}
\end{mruleclass}

\begin{mruleclass}{TC\_ADJproobjcontrol}
\begin{classdescr}
\kind obligatory transformation class
\classtask to deal with the interpretation and deletion of non-subject arguments
of 
infinitival sentences, both in complements and in modifying sentences.

\classremarks

\nofilters

\nospeedrules

\begin{plannedrules}
\item
Default rule TADJPROOBJCONTROL0
\item
Rules for modifiers to {\em te} and {\em genoeg}. Cf. TC\_ADJcontrol.
\end{plannedrules}

\norulesnotince

\rulelist
\end{classdescr}

\begin{members}

\begin{member}
\rulename TADJPROOBJCONTROL1
\ruletask To decide the interpretation of an empty object in infinitival 
SENTENCEs
\file dutch:tc\_adjcontrol.mrule (mrules40)
\semantics \nosemantics
\example\mbox{}\\
x1  lijkt leuk om  objrel/x1 mee te nemen\\
Die foto lijkt  leuk om mee te nemen (als herinnering)
\remarks\mbox{}
\begin{enumerate}
\item 
For some cases it is still to be decided whether the infinitival sentence 
has the status of complement or modifier.
\item
The relation of the example given here and {\em Het is leuk om die foto 
mee te nemen} needs to be studied in more detail. Note that in Spanish,
only the latter contruction is possible.
\end{enumerate}
\end{member}
\end{members}

\end{mruleclass}
\begin{mruleclass}{TC\_ADJObjectok}
\begin{classdescr}
\kind obligatory rule class
\classtask 
In general: to guarantee the presence of a subject if required.
In particular: to realize a subject in case of an ergative adjective, i.e. an 
i.e. an adjective with thetaadj = adjp120.
NB. adjp010 does not seem to be relevant in Dutch.
\classremarks
\nofilters
\nospeedrules

\rulelist
\end{classdescr}

\begin{members}
\begin{member}
\rulename TObjectok0
\ruletask Default rule: lets  non-ergative adjectival phrases pass this TC.
\file dutch:tcs\_adjsurfadap.mrule (mrules42)
\semantics \nosemantics
\example
x1 leuk, x1 vervelend
\remarks\mbox{}

\end{member}
\begin{member}
\rulename TObjectOk1
\ruletask To move the object of an ergative adjective 
into subject  position. 
\file dutch:tcs\_adjsurfadap.mrule (mrules42)
\semantics \nosemantics
\example \mbox{}
\begin{enumerate}
\item
obj/x1 duidelijk $\rightarrow$ subj/x1 duidelijk\\
(de vraag werd duidelijk)\\
\item
obj/het duidelijk complrel/.. $\rightarrow$ subj/het duidelijk complrel/..\\
(het werd duidelijk dat ...)\\
\item
indobj/x2, obj/x1 bekend $\rightarrow$ subj/x1, indobj/x2, bekend\\
( de uitslag werd ons pas laat bekend )\\

\end{enumerate}
\remarks\mbox{}

\end{member}
\begin{member}
\rulename TObjectOk2
\ruletask 
To introduce ECNP, not requiring definiteness restrictions on the object,
if the adjective is
ergative, and if the object does not contain a personal pronoun. 
NB. In the case 
of a personal
pronoun object TADJObjectok1  applies.
\file dutch:tcs\_adjsurfadap.mrule (mrules42)
\semantics \nosemantics
\example\mbox{}\\
io/hem het antwoord duidelijk $\rightarrow$\\
EC io/hem het antwoord duidelijk \\
([Hij zei dat] hem het antwoord duidelijk werd)\\
\remarks\mbox{}

\end{member}
\end{members}
\end{mruleclass}

\begin{mruleclass}{RC\_ADJEMPTYsubstitution}
\begin{classdescr}
\kind optional iterative rule class
\classtask substitution of EMPTYVARs by an EMPTY (which amounts to 
deletion of  EMPTYVARs)
\classremarks
The interpretation of the rules of this RC is not uniform. In some cases 
existential quantification seems to be the correct analysis. In other
cases the context seems to provide the clue. Here universal 
quantification over a contextually given set seems to be a more 
proper analysis. Informally this will be referred to by {\em 
DISCEMPTYsubstitution}. As no formal distinction 
is made between the EMPTY's that 
figure as  
substituent in this RC, the idea of DISCEMPTYsubstitution does not play a
role in the system at present. But the mapping of the various rules of this RC
keeps track of the differences.  

\nofilters
\begin{speedrules}
\begin{member}
\rulename FADJpresubst
\ruletask to speed up analysis. It filters out all ADJPFORMULA's
with .prosubject = true.
\file dutch:rcs\_adjsubst.mrule (mrules38)
\end{member}
\end{speedrules}

\rulelist
\end{classdescr}

\begin{members}
\begin{member}
\rulename RADJEMPTYSUBST1
\ruletask to substitute EMPTYs as prepobjrel/..
\file dutch:rcs\_adjsubst.mrule (mrules38)
\semantics Existential Quantification
\example ik ben verliefd, hij is verslaafd, zij is bang
\remarks\mbox{}
\begin{enumerate}
\item
This rule does not apply to cases
such as the dutch adjective {\em geschikt}, that do not involve Existential 
Quantification. It is still to be established how
these cases should be dealt with. Two possibilities: 
\begin{enumerate} 
\item by means of what may be called 
DISCEMPTYsubstitution,  thus accounting for the
fact that the second argument of the adj is to be found in the context of
discourse. Cf. also RADJEMPTYsubstitution4 for EMPTY's with locargrel.

\item as an idiom corresponding to for example `fit for it'.
\end{enumerate}

\item The models of this rule (but not the match-
conditions) are general enough to cover also virtual examples of ADJs with
synPAPREPEMPTY as value for .adjpattern. As long as no actual examples 
are found, the rule may stay as it is now (4-MAR-1988). It could also 
be made more specific, with the ADJ-head explicitly in the model.\\

\end{enumerate}

\end{member}
\begin{member}
\rulename RADJEMPTYSUBST2
\ruletask to substitute (IO)EMPTYs as indobjrel/.. while synVOOREMPTY 
is in the valueset for .synpattern
\file dutch:rcs\_adjsubst.mrule (mrules38)
\semantics interpretation is given by the context (DISCEMPTYsubstitution)
\example No examples of adjectives with synVOOREMPTY found yet.
\remarks\mbox{}

\end{member}
\begin{member}
\rulename RADJEMPTYSUBST3
\ruletask to substitute (IO)EMPTYs as indobjrel/.. 
\file dutch:rcs\_adjsubst.mrule (mrules38)
\semantics interpretation is given by the context (DISCEMPTYsubstitution)
\example\mbox{}
\begin{enumerate}
\item
  de bekende auteur 
\item zij is verplicht te komen
\item
 het is duidelijk dat ... 
\item
 pim is die inspanning wel waard
\end{enumerate}
\remarks\mbox{}

\end{member}
\begin{member}
\rulename RADJEMPTYSUBST4
\ruletask to substitute EMPTYs (with a meaning that is different than 
the substituents for RADJEMPTYSUBST1) as locargrel/..
\file dutch:rcs\_adjsubst.mrule (mrules38)
\semantics interpretation is given by the context (DISCEMPTYsubstitution)
\example  hij is al een beetje gewend
\remarks\mbox{}

\end{member}
\end{members}
\end{mruleclass}


\begin{mruleclass}{TC\_ADJCaseAssignment}
\begin{classdescr}
\kind iterative rule classes plus filters
\classtask to assign accusative case
\classremarks \mbox{}
\begin{enumerate}
\item
In order to avoid infinite application of the iterative rule class
in analysis, the decompositional condition requires 
that the set of cases is not reduced yet.
\item
The value [Nominative] is never assigned as a surface value in 
this TC.
\end{enumerate}
\begin{filters}
\item Naming convention: FADJpost.. applies in generation, FADJpre.. applies in 
analysis.

\begin{member}
\rulename FADJPostCaseAssignment1
\ruletask to guarantee the proper application of TADJCaseAssignment1 in 
generation.

\end{member}
\begin{member}
\rulename FADJPostCaseAssignment2
\ruletask to guarantee the proper application of TADJCaseAssignment2 in 
generation.
cases.

\end{member}
\begin{member}
\rulename FADJPostCaseAssignment3
\ruletask to guarantee the proper application of TADJCaseAssignment3 in 
generation.

\end{member}
\begin{member}
\rulename FADJPreCaseAssignment1
\ruletask to guarantee the proper application of TCaseAssignment1 in 
analysis.

\end{member}
\begin{member}
\rulename FPreADJCaseAssignment2
\ruletask to guarantee the proper application of TCaseAssignment2 in 
analysis.
\end{member}
\begin{member}
\rulename FADJPreCaseAssignment3
\ruletask to guarantee the proper application of TCaseAssignment3 in 
analysis.
\end{member}

\item All filters are in file dutch:tc\_adjcaseassignment.mrule (mrules39).
\end{filters}

\nospeedrules

\noplannedrules

\norulesnotince

\rulelist
\end{classdescr}

\begin{members}
\begin{member}
\rulename TADJCaseAssignment1
\ruletask In generation: change the value of the attribute case for objs
in NPVAR, 
CNVAR and NP from [Nominative] to the surface value [Accusative]
\file dutch:tc\_adjcaseassignment.mrule (mrules39)
\semantics \nosemantics
\example hij beu $\rightarrow$ hem beu
\remarks\mbox{}

\end{member}
\begin{member}
\rulename TADJCaseAssignment2
\ruletask In generation: change the value of the attribute case in prepobjs 
NPVAR, 
CNVAR and NP from [Nominative] to the surface value [Accusative].
\file dutch:tc\_adjcaseassignment.mrule (mrules39)
\semantics \nosemantics
\example op hij verliefd $\rightarrow$ op hem verliefd 
\remarks\mbox{}

\end{member}
\begin{member}
\rulename TADJCaseAssignment3
\ruletask In generation: change the value of the attribute case for 
indobjs NPVAR, 
CNVAR and NP from [Nominative] into the surface value [Dative, Accusative].
\file dutch:tc\_adjcaseassignment.mrule (mrules39)
\semantics \nosemantics
\example wij bekend $\rightarrow$ ons bekend
\remarks\mbox{}

\end{member}
\end{members}


\end{mruleclass}
\begin{mruleclass}{TC\_ADJargreflreciprspelling}
\begin{classdescr}
\kind Presently the rule class is not activated. Cf. also planned rules.
\classtask to replace argument NPVARs by reflexive or reciprocal pronouns.

\classremarks The transformation must be transformed into a rule class for 
reasons of isomorphy.
\nofilters

\nospeedrules

\begin{plannedrules}
\item Rules for CNVAR antecedents
\item Filters to guarantee correct apllication of this TC, if in the CE 
it is incorporated as an optional iterative rule class.
Alternatively the TC can be made obligatory. In that case it should 
be extended with a default rule.
\end{plannedrules}

\norulesnotince

\rulelist
\end{classdescr}

\begin{members}
\begin{member}
\rulename TADJArgreflspelling1
\ruletask to spell out non-prepositional reflexive arguments.
\file dutch:rc\_helpadjputt.mrule (mrules50)
\semantics unclear
\example ik ben mezelf beu 
\remarks\mbox{}

\end{member}
\begin{member}
\rulename TADJArgreflspelling2
\ruletask to spell out reflexive arguments in the complement of a 
preposition.
\file dutch:rc\_helpadjputt.mrule (mrules50)
\semantics unclear
\example hij is verliefd op zichzelf
\remarks\mbox{}

\end{member}
\begin{member}
\rulename TADJARGreciprspelling1
\ruletask to spell out non-prepositional reciprocal arguments.
\file dutch:rc\_helpadjputt.mrule (mrules50)
\semantics unclear
\example zij zijn elkaar beu
\remarks\mbox{}

\end{member}
\begin{member}
\rulename TADJARGreciprspelling2
\ruletask to spell out reciprocal arguments in the complement of a 
preposition.
\file dutch:rc\_helpadjputt.mrule (mrules50)
\semantics unclear
\example zij zijn verliefd op elkaar
\remarks\mbox{}

\end{member}
\end{members}
\end{mruleclass}

\begin{mruleclass}{TC\_ADJSENTextraposition}
\begin{classdescr}
\kind optional transformation class with associated filters
\classtask to deal with extraposition of sentences
\classremarks

\begin{filters}
\item 
The associated filter FADJPOSTSENTExtrapos
must guarantee the proper application of this TC (obligatory
if possible) in generation. 


\end{filters}
\begin{speedrules}
\begin{member}
\rulename FADJPRESENTExtrapos
\ruletask to speed up analysis. It filters out all ADJPFORMULA's
with with sentential complements in postadjectival position.
\file dutch:rcs\_adjsubst.mrule (mrules41)
\end{member}
\end{speedrules}

\noplannedrules

\norulesnotince

\rulelist
\end{classdescr}
\begin{members}
\begin{member}
\rulename TADJSENTExtrapos
\ruletask Extraposition of preadjectival sentential complements.
\file dutch:tcfcs\_adjextrapos.mrule (mrules41)
\semantics \nosemantics
\example dat je komt tevreden $\rightarrow$ tevreden dat je komt
\remarks\mbox{}

\end{member}
\end{members}
\end{mruleclass}
\begin{mruleclass}{TC\_ADJCOMPLextraposition}
\begin{classdescr}
\kind optional transformation class with associated filter
\classtask to deal with extraposition of comparative complements
\classremarks

\begin{filters}
\item The obligatory associated filter FADJPOSTCOMPLExtrapos  
must guarantee the proper application of this TC (obligatory
if possible) in generation. 

\end{filters}
\begin{speedrules}
\begin{member}
\rulename FADJPRECOMPLExtrapos
\ruletask to speed up analysis. It filters out all ADJPFORMULA's
with with extraposed comparative complements.
\file dutch:rcs\_adjsubst.mrule (mrules41)
\end{member}
\end{speedrules}



\noplannedrules

\norulesnotince

\rulelist
\end{classdescr}
\begin{members}

\begin{member}
\rulename TADJCOMPLExtrapos
\ruletask To account for the extraposition of comparative-complements.
\file dutch:tcfcs\_adjextrapos.mrule (mrules41)
\semantics \nosemantics
\example (veel/bijna) minder/meer dan NPVAR lang 
$\rightarrow$
(veel/bijna) minder/meer lang dan NPVAR
\remarks\mbox{}


\end{member}
\end{members}
\end{mruleclass}
\begin{mruleclass}{TC\_ADJPREPPextrapositionA}
\begin{classdescr}
\kind optional transformation class
\classtask to deal with extraposition of PREPPs
\classremarks

\nofilters


\begin{speedrules}
\item 
A speed rule 
must guarantee the proper application of TADJPREPPextrapos2 (obligatory
if possible in analysis).
Extraposition of 
PREPPs is not obligatory, but is must apply in analysis in order
to reconstruct the canonical structure that is presupposed in the analytical 
patternrules.
\end{speedrules}

\noplannedrules

\norulesnotince

\rulelist
\end{classdescr}
\begin{members}
\begin{member}
\rulename TADJPREPPExtrapos2
\ruletask To move preadjectival prepositional modifiers (non-arguments)
to postadjectival  position not preceded by postadjectival PREPP-arguments.
\file dutch:tcfcs\_adjextrapos.mrule (mrules41)
\semantics \nosemantics
\example \mbox{}

\begin{enumerate}
  \item
door de zon verhit   
$\rightarrow$  verhit door de zon
\item
in 1988 belangrijk (genoeg) $\rightarrow$ belangrijk (genoeg) in 1988 

\end{enumerate}
\remarks\mbox{}
\end{member}
\end{members}

\end{mruleclass}



\begin{mruleclass}{TC\_ADJPAcomplmovement}
\begin{classdescr}
\kind optional rule class 
\classtask To account for the movement of postadjectival PREPPs into 
preadjectival position (erposrel).
\classremarks 
\nofilters
\nospeedrules
\begin{plannedrules}
\item
A set of movement rules for complements 
of degreemodifiers (e.g. {\em te} or {\em genoeg}); 
they should move into the position for voorobjrels, or to
erposrel. 
\end{plannedrules}
\rulelist
\end{classdescr}

\begin{members}
\begin{member}
\rulename TADJPAcomplmovement
\ruletask  To move PA-prepobjs from their canonical position into 
pre-adjectival position.

\file dutch:rcs\_adjsubst.mrule (mrules43)
\semantics \nosemantics
\example \mbox{}\\
{\tt head}/bang + {\tt paprepobjrel}/voor de os $\rightarrow$ \\
{\tt erposrel}/voor de os + {\tt head}/bang\\
 (de voor de os bange jongen)
\remarks\mbox{}
\end{member}
\begin{member}
\rulename TADJcomplmovement
\ruletask   To move prepobjs, aanobjs and locargs 
from their canonical pre-adjectival position into 
erposrel.

\file dutch:rcs\_adjsubst.mrule (mrules43)
\semantics \nosemantics
\example .., {\tt prepobjrel}/met de os, {\tt head}/bezig $\rightarrow$ 
{\tt erposrel}/met de os, .., {\tt head}/bezig\\
(de met de os zelden bezige jongen)

\remarks\mbox{}
\end{member}
\begin{member}
\rulename TADJvoorobjmovement1
\ruletask  To move voorobj-modifiers from their canonical position into 
pre-adjectival position (erposrel).
\file dutch:rcs\_adjsubst.mrule (mrules43)
\semantics \nosemantics
\example \mbox{}\\
{\tt head}/belangrijk + {\tt voorobjrel}/voor het project 
$\rightarrow$ \\
{\tt erposrel}/voor 
het project + {\tt head}/belangrijk\\
(de voor het project belangrijke beslissing)
 
\remarks\mbox{}
\end{member}

\end{members}

\end{mruleclass}
\begin{mruleclass}{TC\_ADJPREPPextrapositionB}
\begin{classdescr}
\kind optional transformation classes 
\classtask to deal with extraposition of PREPPs
\classremarks

\nofilters


\nospeedrules

\noplannedrules

\norulesnotince

\rulelist
\end{classdescr}
\begin{members}
\begin{member}
\rulename TADJPREPPExtrapos1
\ruletask To move preadjectival prepositional complements
to postadjectival 
position. No elements may occur between ADJ and (VAR)PREPP, except `hoppers' 
such as {\em genoeg}.
\file dutch:tcfcs\_adjextrapos.mrule (mrules41)
\semantics \nosemantics
\example \mbox{}
\begin{enumerate}
  \item 
{\tt prepobjrel}/met x1 + vertrouwd $\rightarrow$ vertrouwd + {\tt paprepobjrel}/met x1
  \item 
{\tt aanobjrel}/aan x1 + bekend $\rightarrow$ bekend + {\tt paprepobjrel}/aan x1
\end{enumerate}
\remarks\mbox{}

\end{member}
\begin{member}
\rulename TADJPREPPExtrapos3
\ruletask 
To move preadjectival prepositional arguments in locargrel
to postadjectival 
position.
No elements may occur between ADJ and (VAR)PREPP, except `hoppers' 
such as {\em genoeg}.
\file dutch:tcfcs\_adjextrapos.mrule (mrules41)
\semantics \nosemantics
\example \mbox{}
{\tt locargrel}/van x1 naar x2 + {\tt head}/onderweg 
$\rightarrow$ onderweg + {\tt postadjrel}/van x1 naar x2
\remarks\mbox{}

\end{member}
\end{members}
\end{mruleclass}


\begin{mruleclass}{TC\_ADJTETOTEZEER}
\begin{classdescr}
\kind obligatory rule class plus filter
\classtask to replace degree modifier {\em te} by {\em te zeer} 
\classremarks 
There are no rules implemented yet.
\nofilters
\nospeedrules
\rulelist
\end{classdescr}

\begin{members}
\begin{member}
\rulename TADJTETOTEZEER
\ruletask replacement of degree modifier {\em te} by {\em te zeer}
if it does not
immediately precede an adjective.
\file dutch:rcs\_adjsubst.mrule (mrules38)
\semantics \nosemantics
\example \mbox{}
\begin{enumerate} 
\item
hij is [te $\rightarrow$ te zeer] aan zijn familie gehecht 
                            om te kunnen emigreren 
\item 
hij is [te $\rightarrow$ te zeer] in de war om te kunnen koken
\end{enumerate}
\remarks\mbox{}
\end{member}

\end{members}

\end{mruleclass}

\begin{mruleclass}{TC\_ADJcomparativeforms}
\begin{classdescr}
\kind obligatory rule class
\classtask account for the spelling out of the attribute .form
\classremarks
Analytically, ADJPs are assigned  the default value [] for .actuseefs 
by this RC. This attribute is supposed to be irrelevant in the remaining part 
of the
analytical derivation. This is checked in RC\_startADJPPROP.
\nofilters

\nospeedrules

\begin{plannedrules}
\item  Rules that account for the spelling 
out of irregular comparative forms (erIrregSup).
\end{plannedrules}

\rulelist
\end{classdescr}
\begin{members}

\begin{member}
\rulename TADJmeerTOer
\ruletask To account for the 
morphological incorporation of the comparative expression `meer'
for adjectives with the value {\em erComp} for .comparatives.
\file dutch:tcs\_adjsurfadap.mrule
\semantics \nosemantics
\example meer lang $\rightarrow$ langer 
\remarks\mbox{}

\end{member}
\begin{member}
\rulename TADJmeerNOTTOer
\ruletask Lets adjectives with the value {\em meerComp}
for .comparatives pass pass this TC.
\file dutch:tcs\_adjsurfadap.mrule
\semantics \nosemantics
\example meer openbaar  $\rightarrow$ meer openbaar 
\remarks\mbox{}

\end{member}
\begin{member}
\rulename TADJNOerNOmeer
\ruletask Defaultrule. Lets adjectival phrases without degreemodifier {\em meer}
pass this TC.
\file dutch:tcs\_adjsurfadap.mrule
\semantics \nosemantics
\example any ADJP without comparative modifier
\remarks\mbox{}

\end{member}
\end{members}
\end{mruleclass}

\begin{mruleclass}{TC\_ADJsuperlativeforms}
\begin{classdescr}
\kind obligatory rule class
\classtask account for the spelling out of the attribute .form
\classremarks
\nofilters

\nospeedrules

\begin{plannedrules}
\item  Rules that account for the spelling 
out of irregular superlative forms (stIrregSup).
\item  Rules that account for the spelling out of {\em aller-}superlative 
forms (allerSup, allerIrregSup).
\end{plannedrules}

\rulelist
\end{classdescr}

\begin{members}

\begin{member}
\rulename TADJmeestTOst1
\ruletask To account for the 
morphological incorporation without remaing {\em het} of the superlative
expression `het meest'
for adjectives with the value {\em stSup} for .superlatives.
\file dutch:tcs\_adjsurfadap.mrule
\semantics \nosemantics
\example het meest lang $\rightarrow$ langst(e)
\remarks\mbox{}
The resulting form can not be used predicatively, nor can it be used as an ADV.

\end{member}
\begin{member}
\rulename TADJmeestTOst2
\ruletask morphological incorporation of the superlative
expression `meest' with remaining {\em het}
for adjectives with the value {\em stSup} for .superlatives.
\file dutch:tcs\_adjsurfadap.mrule
\semantics \nosemantics
\example het meest lang $\rightarrow$ het langst 
\remarks\mbox{}

\end{member}
\begin{member}
\rulename TADJmeestNOTTOst
\ruletask Lets adjectives with the value {\em meestSup}
for .superlatives pass pass this TC.
\file dutch:tcs\_adjsurfadap.mrule
\semantics \nosemantics
\example meest openbaar $\rightarrow $ meest openbaar
\remarks\mbox{}
\end{member}

\begin{member}
\rulename TADJNOstNOmeest
\ruletask Default rule. Lets adjectives phrases without 
degree modifier {\em meest}
pass this TC.
\file dutch:tcs\_adjsurfadap.mrule
\semantics \nosemantics
\example 
\remarks\mbox{}

\end{member}
\end{members}

\end{mruleclass}


\end{document}
