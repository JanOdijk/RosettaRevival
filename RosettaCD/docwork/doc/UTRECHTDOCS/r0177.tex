\documentstyle{Rosetta}
\begin{document}
   \RosTopic{General}
   \RosTitle{Tweede verslag aan Van Dale Lexicografie}
   \RosAuthor{Harm Smit}
   \RosDocNr{177}
   \RosDate{\today}
   \RosStatus{approved}
   \RosSupersedes{-}
   \RosDistribution{Project}
   \RosClearance{Project}
   \RosKeywords{dictionary, Van Dale, report}
   \MakeRosTitle

\section{Inleiding}
In dit tweede halfjaarlijkse verslag zal een overzicht gegeven worden van de 
stand van zaken betreffende het gebruik van de N-N, N-E en E-N bestanden.

In het najaar van 1986 zijn de tapes ontvangen met de bestanden N-E en E-N.
Tot nu toe is met deze bestanden weinig gewerkt, maar op het eerste gezicht
lijken de bestanden in ieder geval wel compleet te zijn.

Eind november 1986 zijn namens ROSETTA Jeroen Medema en ondergetekende op 
bezoek geweest bij Cees van Eijden. De bedoeling van dit bezoek was, de 
banden weer wat aan te halen en eens te bekijken wat wij en Van Dale zelf
zoal met de bestanden deden. Daarbij is afgesproken dat Van Dale ons de nieuwste
versie van het zgn. `produktieboek' zou sturen, terwijl wij van onze kant een
lijst zouden opstellen van dingen die in de Van Dale's ontbraken, maar die 
in veel moderne woordenboeken wel te vinden zijn en derhalve ook in de Van 
Dale's hadden kunnen worden vermeld. Deze laatste lijst is door tijdgebrek wat 
laat tot stand gekomen en we hebben besloten hem in dit verslag te incorporeren.
Tevens is bij dit bezoek het voornemen uitgesproken elkaar circa twee maal per 
jaar te treffen, om zo een regelmatig contact te garanderen. Tijdens het bezoek
bleek ook dat in de huidige werkzaamheden van beide partijen weinig overlap zat,
zodat samenwerking op dit moment niet zinvol lijkt. We kunnen
ons echter voorstellen dat dit in de toekomst wel nuttig
kan zijn, gezien het feit dat de plannen van Cees van Eijden (nl. het maken van 
een database waarbij verschillende talen d.m.v. de betekenissen aan elkaar
gekoppeld zijn) qua problematiek voor een deel overeenstemmen met onze plannen
voor het koppelen van de toekomstige ROSETTA-woordenboeken voor Nederlands en 
Engels. 

\section{Produktieboek}
Met het produktieboek hebben we voor het eerst een complete syntaxis van het
lemma van N-N gekregen. Inmiddels hadden we zelf al een syntaxis gemaakt, die
echter in een aantal details verschilt van die uit het produktieboek. Hoewel
ook onze syntaxis een aantal lemma's niet aankan (wat niet verwonderlijk is
omdat sommige lemma's aanwijsbaar `fout' zijn), hebben wij het idee dat dit 
aantal veel lager zal liggen dan bij de syntaxis uit het produktieboek het
geval zal zijn. Ook hebben we geen redenen om te veronderstellen dat de laatste
syntaxis `logischer' zou zijn, en derhalve te prefereren. Dit alles deed bij 
ons het vermoeden rijzen dat de syntaxis uit het produktieboek wel eens zou 
kunnen horen bij een bestand dat, zij het slechts in licht mate,
{\em verschilt} van het onze. We hebben het
idee, dat hetzelfde geldt voor de syntaxis van het N-E en die van het E-N.
De door ons gemaakte syntax van het N-N is bij ons bezoek aan Cees van Eijden
overhandigd, samen met een lijst van de lemma's die er niet door kwamen.

\section{Lijst met dingen die o.i. ten onrechte ontbreken}

Het betreft hier informatie die in veel recent op de markt gebrachte 
woordenboeken wel voorkomt, maar in de Van Dale ontbreekt. Uiteraard blijft 
een opsomming
als deze tamelijk willekeurig, want een volledig gespecificeerde lijst van wat 
een woordenboek naar onze mening zou moeten bevatten, zou neerkomen op een 
afspiegeling van {\em alle} in ROSETTA gebruikte attributen. Voor een
woordenboek wat bedoeld is voor normale gebruikers zou dat echter wat te veel
van het goede zijn.

\begin{enumerate}
   \item Bij werkwoorden: `verbpatterns' (= `valentie-patronen').

   Bij elk werkwoord komt een (of meer) van deze verbpatterns, waarmee 
   aangegeven wordt hoeveel argumenten een werkwoord neemt, en wat voor soort
   argumenten. 
   Het aantal argumenten kan bijv. 1 zijn (bij `slapen'), of 2 (bij `slaan'),
   etc. 
   Het soort argumenten kan bijvoorbeeld zijn:

   \begin{itemize}
      \item NP               : {\em de man} geeft {\em de vrouw} {\em een boek}.
      \item PP               : hij wacht {\em op zijn vader}.
      \item infinitief + `te': hij probeert {\em te komen}.
      \item inf. zonder `te' : hij leert {\em fietsen}.
      \item bijzin met `dat' : hij zegt {\em dat hij komt}.
      \item bijz. met vraagw.: hij vraagt {\em wie er komen}.
      \item bijz. met `of'   : hij vraagt {\em of we komen}.
   \end{itemize}

etc.

  In vele buitenlands-talige woordenboeken is een dergelijk `verbpattern' wel
aangegeven, zoals in de `Longman Dictionary of Contemporary English', in de
`Oxford Advanced Learner's Dictionary', `dtv W\"{o}rterbuch der deutschen 
Sprache', e.d. Ook in het `Basiswoordenboek Nederlands' is dergelijke 
informatie opgenomen. 
 
  \item Vergelijkbaar met `verbpatterns' bij werkwoorden zijn `nounpatterns' bij 
   zelfstandige naamwoorden. Bijv.: 

  \begin{itemize}
    \item bijzin met `dat' : de eis, {\em dat allen weg moesten}, was duidelijk.
    \item bijz. met `of'   : de vraag, {\em of hij er was}, is nooit gesteld.
    \item infinitief + `te': de poging, {\em het verhaal op te schrijven}, mislukte.
  \end{itemize}
  
etc.

  \item Vergelijkbaar met `verbpatterns' bij werkwoorden zijn `adjectivepatterns' 
   bij bijvoeglijke naamwoorden. Bijv.: 

  \begin{itemize} 
       \item infinitief + `te': hij is blij, {\em hem weer te zien}.
       \item PP               : hij is woonachtig {\em te Utrecht}.
  \end{itemize}

etc.

  \item Idiomen zouden altijd op dezelfde manier opgenomen moeten worden, nl. onder
   hun `hoofd', en altijd in dezelfde, canonieke vorm. Daarbij moeten de vaste
   delen van de vrije onderscheiden worden in de typografie. Voorbeeld:

  \begin{itemize}
     \item `Hij {\em geeft de pijp aan Maarten}', onder: `geven'.
     \item `Hij {\em naait} Maarten {\em een oor aan}', onder: `aannaaien'.
  \end{itemize}

   Aldus is het duidelijk, dat `Maarten' in de eerste zin (vast) deel uitmaakt
   van het idioom, terwijl het in de tweede uitdrukking vrij is. Het verdient 
   de voorkeur voor de vrije delen voornaamwoorden te gebruiken (hij, hem, haar,
   iem., iets, e.d.). 

  \item  Bij het toevoegen van verbpatterns kan tevens voorkomen worden dat 
   hinderlijke inconsistenties optreden zoals nu bij `verkopen'; onder `I' heeft
   dit werkwoord een subjekt, dat in de betekenissen onder `II' als direkt 
   objekt op zou treden (vgl.: `dat boek verkoopt goed' met: `hij verkoopt 
   dat boek'). Dit verschil is in de huidige generatie woordenboeken niet
   voldoende aangegeven; vgl. bijvoorbeeld `eten' waar tussen `I' en `III'
   niet hetzelfde verschil in betekenis bestaat als bij `verkopen', maar wel
   hetzelfde formele onderscheid is gemaakt tussen niet overgankelijke en 
   wel overgankelijke werkwoorden. 

  \item In N-N ontbreekt het `count-mass' onderscheid; in E-N is dit wel 
aangebracht.

  \item  Naast een `count-mass' onderscheid (zie 6) zou ook een verdere 
semantische typering gewenst zijn; te denken valt bijvoorbeeld aan 
`animate - inanimate',
`human - non-human', etc. Op die manier zou het verschil tussen een `stoker' en
een `tandenstoker' ook formeel uitgedrukt zijn (en niet, zoals nu het geval
is, alleen uit de omschrijving op te maken zijn. Overigens zou een wat 
formelere opbouw van de betekenis omschrijving (bijv. beginnend met `persoon'
bij `stoker' en met `object', of `ding' bij `tandenstoker') ook al veel 
helpen). 

  \item  Bij adverbia zou meer semantische informatie meegecodeerd kunnen 
worden. Te denken valt bijvoorbeeld aan een onderverdeling van adverbia van 
tijd, plaats, e.d. 

\section{Werkzaamheden aan het bestand}

In het afgelopen halfjaar is begonnen met het `verrijken' van het N-N bestand; 
dit betekent dat we aan de bestanden nieuwe codes toevoegen die corresponderen 
met bepaalde attributen in ROSETTA en daarachter de juiste waarde van dit 
attribuut
voor het betreffende ingangswoord. Voorlopig wordt daarbij geen informatie
weggegooid. Een deel van dit verrijken geschiedt met de hand, waarbij van 
speciale interactieve programmatuur gebruik gemaakt wordt. Indien mogelijk 
wordt echter automatisch gevuld, eveneens met speciaal hiervoor ontwikkelde
programma's. Zo is bijvoorbeeld de oorspronkelijke meervoudsaanduiding in het
bestand gebruikt voor het vullen van de waarden van het meervouds-attribuut 
voor ROSETTA. Omdat onze morfologie voor het Nederlands krachtig genoeg is om
zelf te zorgen voor het verdubbelen van medeklinkers en deleren van klinkers 
indien dat nodig is, konden woorden als `rat', `raat' en `krant' alle de
waarde `enPlural' krijgen, ofschoon ze oorspronkelijk de vermeldingen
`-ten', `raten' en `-en' hadden. Het programma vult circa 98 \% van de 
zelfstandige naamwoorden automatisch, de resterende 2 \% zullen met de hand
gedaan moeten worden (het betreft hierbij zeldzame meervoudsvormen,
omschrijvingen als `vaak mv.', fouten, e.d.).

Ook zijn recent alle lemma's van het bestand van een nummer voorzien. Verder
zijn een aantal tellingen verricht (o.a. naar het aantal lemma's, het aantal
verwijzingen, de grootte van het lemma, e.d.). Verder zijn er nieuwe bestanden
aangemaakt, zoals een bestand van alle zelfstandige naamwoorden, alle 
adjectieven, en alle werkwoorden. Op dit moment wordt gewerkt aan een retrograde
versie van het N-N bestand.

\section{Inconsistenties en fouten}

Ook nu weer werd, als neveneffect van het werken aan het bestand, een aantal
fouten getraceerd. Dit gebeurde echter min of meer toevallig; er is dus niet
sprake geweest van systematisch zoeken naar fouten. Hier volgt een opsomming:

   \begin{itemize}
      \item in een aantal gevallen is het gebruik van de tilde in voorbeelden
            niet conform de regel dat dit teken alleen {\em onverbogen} 
            ingangswoorden aanduidt. Zo staat `beetje' in betekenis 3.1 
            `$\sim$jes' i.p.v. `$\sim$s', en bij `beentje' (in bet. 3.2) `$\sim$tjes'
            i.p.v. `$\sim$s'. Andere voorbeelden zijn `kop' (in bet. 1.1), waar
            het woord `kop' uitgeschreven staat, maar daarnaast ook nog eens 
            door een tilde wordt aangegeven. Bij `oogverblindend' staat 
            `l$\sim$' i.p.v. `$\sim$'.

      \item in de meervoudsaanduidingen bleken ook fouten te zitten: 

            `oppas' heeft `-en' i.p.v. `-sen';
            `rotteknip' en `stap' hebben `-en' i.p.v. `-pen';
            `contactafdruk', `geleidingshek' en `steigerbok' hebben 
               `-en' i.p.v. `-ken'; 
            `vijgenmat', `zoetstoftablet' en `zuigtablet' hebben `-en' i.p.v. 
               `-ten';
            `gel' en `prijsverschil' hebben `-en' i.p.v. `-len';
            `tweepersoonsbed' heeft `-en' i.p.v. `-den';

            bij `pijlvergif' staat alleen `-en', wat juist is voor de variant 
            `pijlvergift', maar niet voor `pijlvergif'; iets soortgelijks is
            het geval bij `pijlgif'; bij `lijkegif' daarentegen is alleen de
            meervoudsvorm `-ten' vermeld terwijl ook daar de variant 
            `lijkegift' bestaat.

            `vijgenmand' heeft het meervoud `-ten', terwijl het `-en' moet zijn.

            Bij een groot aantal woorden is het streepje voor de meervoudsvorm
            weggevallen; zo staat er `'s' i.p.v. `-'s' bij: `fata morgana', 
            `nazi', `woonsilo', `PCB', e.a.; er staat `en' i.p.v. `-en' bij:
            `grensconflict', `lampionvrucht', `afgrond', e.a.; `men' i.p.v. 
            `-men' bij `vrijdom' en `stinkbom'; `n' i.p.v. `-n' bij 
            `geboorteakte'; `pen' bij `smeerlap'; etc.
            
      \item de werkwoorden `dubbelen' en `truten' zijn per ongeluk als 
            zelfstandig naamwoord gecategoriseerd. Ook bij `onthalzen' 
            lijkt dat het geval, hoewel hier geen verbogen vormen voor
            verleden tijd en voltooid deelwoord vermeld zijn.

      \item in de E-N werd toevallig een fout in de bij het woord vermelde
            uitspraak gevonden: `undulate$^{1}$' heeft de uitspraak van 
            `undress'.

   \end{itemize}

\end{document}
