\documentstyle{Rosetta}
\begin{document}
   \RosTopic{Rosetta3.Linguistics.Dutch}
   \RosTitle{De vulling van speciale werkwoorden in het Nederlands}
   \RosAuthor{Jan Odijk}
   \RosDocNr{R0390}
   \RosDate{September 15,1989}
   \RosStatus{concept}
   \RosSupersedes{-}
   \RosDistribution{Project}
   \RosClearance{Project}
   \RosKeywords{auxiliary verbs, special verbs, dictionaries}
   \MakeRosTitle
%
%

\def\PP{$\wp$}
\def\to{$\rightarrow$}
\section{Introductie}

Dit document documenteert  de vulling van speciale werkwoorden. De vulling
is gebaseerd op de vulling in de Van Dale NE, maar deze is in de meeste
gevallen behoorlijk gewijzigd.
Speciale werkwoorden zijn werkwoorden met {\em subc} ongelijk aan
{\em  mainverb} en werkwoorden die in Mregels genoemd worden.

Iedere sectie behandelt een werkwoord. Achtereenvolgens worden gegeven:

\begin{description}
\item[Van Dale vulling] de Van Dale NE vulling zoals gegenereerd door Harm
\item[bespreking] commentaar op de Van Dale vulling (optioneel)
\item[herziene entries] voorstel voor een andere, betere vulling
\item[idiomen] een lijst idiomen en andere bijzondere constructies gebaseerd
op het relevante werkwoord, voorzover ze in Van Dale voorkomen (dit is niet
voor alle werkwoorden volledig gedaan)
\item[feitelijk gedaan] Een indicatie van wat er feitelijk gedaan is. Dit is
meestal een subset van de {\bf herziene entry}. Aangegeven wordt welke
betekenissen
(gescheiden door een komma) bij welke entries (gescheiden door puntkomma)
in het woordenboek opgenomen zijn.
Als een betekenis weggegooid is, dan is die ook in het Engelse woordenboek
weggegooid
\end{description}

\noindent
De volgende criteria voor het opslitsen van entries in de syntaxis zijn
aangehouden:

\begin{description}
\item[andere argumentstruktuur] Als twee betekenissen een verschillende
argumentstructuur hebben, dan worden ze in de syntaxis opgesplitst.
(bv. {\em gaan} in {\em de bel gaat} (1-plaatsig) vs. {\em hij is naar huis
gegaan} (2-plaatsig).
\item[andere patterns] Als een verschil in betekenis correspondeert
met een verschil in patterns, dan wordt er alleen dan in de
syntaxis opgesplitst indien duidelijk is dat de semantiek dit niet op kan
lossen. Vaak kan de ambiguiteit opgelost worden doordat andere patterns
andere types van complementen opleveren, en dus kan de ambiguiteit uitgesteld
worden tot in de semantiek (type-systeem).\\
Voorbeeld: {\em know} in het Engels kan betekenen `weten' of `kennen'. In de
betekenis `weten' neemt het een zin of een NP als complement; in de betekenis
`kennen' alleen een NP. In dit geval dient er {\bf niet} in de syntaxis
opgesplitst te worden, aangezien het betekenisonderscheid correspondeert met
een type-verschil tussen complementen ({\em truth value} vs. {\em entity}).
\end{description}


\section{zijn}

In Van Dale worden de volgende werkwoorden `zijn' onderscheiden:

\begin{tabular}[t]{|c|c|p{.2\textwidth}|p{.3\textwidth}|}
\hline
Skey          & Mkey            & Omschrijving & Voorbeeld\\
\hline
\$s\_aV\_20\_ben
              &\$m\_aV\_2001\_ben   & s1 "bestaan" & God is. er zijn mensen\\
              &\$m\_aV\_2002\_ben   & s2 "zich bevinden" & hij is in de tuin\\
              &\$m\_aV\_2003\_ben   & s3 "gebeuren" & het was in 1984 dat..\\
              &\$m\_aV\_2004\_ben   & s4 "behoren aan" & Van wie is dat?\\
              &\$m\_aV\_2005\_ben   & s5 "leven" & hij is niet meer;
                                              hij is er geweest\\
              &\$m\_aV\_2006\_ben   & s6 "bezig zijn met" & hij is aan het zingen\\
              &\$m\_aV\_2007\_ben   & s7 "bedragen" & Dat is dan drie gulden\\
\hline
\$s\_aV\_21\_ben
              &\$m\_aV\_2101\_ben   & koppelwerkwoord & Hij is ziek/ een soldaat\\
\hline
\$s\_aV\_22\_ben
              &\$m\_aV\_2201\_ben   &s1 "hww $<$van tijd$>$" & Hij is gekomen\\
              &\$m\_aV\_2202\_ben   &s2 "$<$hww van de lijdende vorm$>$" & Hij is
                                                  gedood\\
\hline
\end{tabular}\\


Commentaar:

\begin{description}
\item[\$m\_aV\_2001\_ben, "bestaan"] In de voorbeelden {\em God is.} ok. In de voorbeelden
{\em er zijn mensen} = hww, of = "zich bevinden"\\
\item[\$m\_aV\_2002\_ben, "zich bevinden"] ok
\item[\$m\_aV\_2003\_ben, "gebeuren"] = cleft constructie, moet weg!
\item[\$m\_aV\_2004\_ben, "behoren aan"] ok
\item[\$m\_aV\_2005\_ben, "leven"] idiomen {\em  er geweest zijn} en {\em  niet
meer zijn} (laatste is ms. "bestaan"), dus weg!
\item[\$m\_aV\_2006\_ben,  "bezig zijn met"] gebeurt met gewone hww, dus weg.
\item[\$m\_aV\_2007\_ben, "bedragen"] apart werkwoord?, of misschien niet, maar hoe
is dan de interactie???
\end{description}

{\em  zijn} als koppelwerkwoord en als hulpwerkwoord worden
vervangen door het werkwoord in auxverb.dict.

Herziene entries voor {\em  zijn}

\begin{itemize}
  \item vp100 synnovpargs   main \$m\_aV\_2001\_ben "bestaan"
  \item vp120 synLOCOPENPPP main \$m\_aV\_2002\_ben "zich bevinden"
  \item vp120 synPREPNP van main \$m\_aV\_2004\_ben "behoren aan"
  \item vp120 synMEASUREphrase main \$m\_aV\_2007\_ben "bedragen"
  \item zijn zoals in auxverb
\end{itemize}


\begin{description}
\item[feitelijke gedaan] s1;s2,s4,s7;\\
\end{description}

\newpage
\section{worden}

\begin{tabular}[t]{|c|c|p{.2\textwidth}|p{.3\textwidth}|}
\hline
Skey          & Mkey            & Omschrijving & Voorbeeld\\
\hline
\$s\_aV\_00\_word & \$m\_aV\_0001\_word &"in de genoemde toestand raken"& hij
wordt ziek\\
              & \$m\_aV\_0002\_word &"de genoemde hoedanigheid krijgen"& hij
wordt minister\\
\hline
\$s\_aV\_01\_word & \$m\_aV\_0101\_word &"$<$ter aanduiding v.d. lijdende vorm$>$" &\\
\hline
\$s\_aV\_02\_word & \$m\_aV\_0201\_word &"gaan kosten"& dat wordt dan drie
gulden\\
\hline
\end{tabular}

Commentaar:
\begin{description}
\item[\$m\_aV\_0001\_word,"in de genoemde toestand raken"] Deze en de volgende zijn
de zelfde, afhankelijk van NP of AP complement
\item[\$m\_aV\_0002\_word,"de genoemde hoedanigheid krijgen"] Deze en de vorige zijn
de zelfde, afhankelijk van NP of AP complement\\
\item[\$m\_aV\_0101\_word,"$<$ter aanduiding v.d. lijdende vorm$>$"]=hww uit auxverb
\item[\$m\_aV\_0201\_word,"gaan kosten"]ok, vgl {\em  zijn=bedragen}
\end{description}

Herziene entries:

\begin{itemize}
  \item vp010 synclosedNPP/APP "in de genoemde toestand of
                                           hoedanigheid geraken"
  \item vp120 synMEASUREPHRASE "gaan kosten"
  \item hww zoals in auxverb
\end{itemize}


\begin{description}
\item[feitelijke gedaan] zoals aangegeven
\end{description}


\newpage
\section{hebben}


Van Dale vulling:\\

\begin{tabular}[t]{|c|c|p{.2\textwidth}|p{.3\textwidth}|}
\hline
Skey          & Mkey              & Omschrijving & Voorbeeld\\
\hline
\$s\_aV\_00\_heb  & \$m\_aV\_0001\_hebben & s1 "bezitten" & ik heb een boek\\
              & \$m\_aV\_0002\_hebben & s2 "toegerust zijn met & een scherp gehoor hebben\\
              & \$m\_aV\_0003\_hebben & s3 "mbt. een (verwantschaps)betrekking"&een
zoon hebben\\
              & \$m\_aV\_0004\_hebben & s4 "getroffen zijn door" &zorgen, pijn,
verdriet hebben\\
              & \$m\_aV\_0005\_hebben & s5 "in genoemde omstandigheden verkeren"&
mooi weer hebben???\\
              & \$m\_aV\_0006\_hebben & s6 "(gevoelens) koesteren"&geduld hebben,
een hekel hebben\\
              & \$m\_aV\_0007\_hebben & s7 "beschikken over"&??ze hebben brood in
huis; ze hebben de dief\\
              & \$m\_aV\_0008\_hebben & s8 "in het genot gesteld zijn van" & een
inkomen hebben, bezoek hebben\\
              & \$m\_aV\_0009\_hebben & s9 "deelachtig worden" & een boek te leen
hebben????\\
              & \$m\_aV\_0010\_hebben & s10 "$<$mbt. iets dat gedaan kan/moet worden$>$"
                                                 & iets te doen hebben, iets te
berichten\\
              & \$m\_aV\_0011\_hebben & s11 "aantreffen"& kijk wie we daar hebben\\
              & \$m\_aV\_0012\_hebben & s12 "in genoemde toestand houden"&????\\
              & \$m\_aV\_0013\_hebben & s13 "verdragen" &de tekst heeft hier een
zetfout?????; niet veel kunnen hebben; het kunnen hebben\\
              & \$m\_aV\_0014\_hebben & s14 "$<$+ aan$>$nut ondervinden van"&\\
\hline
\$s\_aV\_01\_heb  & \$m\_aV\_0101\_hebben & s1 "$<$ter aanduiding van de voltooide tijd
bij ww.$>$" & \\
\hline
\end{tabular}
\newpage
Verdere voorbeelden gegeven in de Van Dale:\\
\begin{tabular}[t]{|c|p{.6\textwidth}|p{.3\textwidth}|}
\hline
S  &  Voorbeeld                                              & Commentaar\\
s1 &  ik heb nog altijd een boek van u te leen               & =CPPP\\
   &  hij heeft een eigen huis                               & ok\\
   &  schoenen aan de voeten hebben                          & =CPPP\\
s2 &  hij heeft een scherp gehoor/ een lange adem            & sid?/id?\\
   &  hij heeft veel haar                                    &  ok\\
   &  dat schip heeft drie masten                            & ok\\
s3 &  zij heeft twee zoons, een minnaar                      & ok\\
s4 &  last (van iets)hebben; zorgen hebben; pijn, verdriet
      hebben, geduld hebben                                  & sids\\
s5 & mooi weer hebben                                        & ???\\
s6 & achting voor iemand hebben, bezwaar tegen iets hebben   & sid\\
   & geen droge draad meer aan het lijf hebben               & id?\\
   & het hart op de tong hebben                              & id\\
   & een hoed op het hoofd hebben                            & =CPPP\\
   & een onaangename lucht bij zich hebben                   & =CPPP\\
   & plezier van iets hebben                                 & sid\\
   & ergens spijt van hebben                                 & (s?)id\\
s7 & ze hebben geen brood in huis                            & =CPPP\\
   & de goedheid hebben om                                   & id\\
   & de tijd hebben (tijd hebben voor= sid?)/een leuke tijd
     hebben                                                  & id/ sid?/sid?\\
s8 & een behoorlijk inkomen hebben                           & =s2?\\
   & macht hebben                                            & sid\\
   & vrijheid hebben (= de vrijheid hebben om;veel/weinig
     vrijheid hebben??)                                      & sid\\
   & aanspraak hebben op iets(?)                             & sid\\
   & mensen hebben                                           & id\\
   & bezoek hebben                                           & ?sid\\
   & mag ik dat potlood even van je hebben=gebruiken         & ??\\
   & het woord hebben                                        & sid\\
\hline
\end{tabular}
\begin{tabular}[t]{|c|p{.6\textwidth}|p{.3\textwidth}|}
\hline
S  &  Voorbeeld                                              & Commentaar\\
s9 & een boek te leen hebben                                 & =CPPP\\
   & die les hebben we al gehad                              & sid?\\
   & die pantoffels heb ik van mijn vrouw                    & =gekregen hebben\\
   & ik moet nog een tientje van hem hebben                  & =krijgen\\
   & hij heeft een trap van een paard gehad     perf only    & =sid??\\
%   & idem: schop, zoen;                                      & \\
s10 &                                                        & \\
s11 &                                                        &\\
s12 & ze hebben de dief                                      & ??\\
    & ze hebben een stok in de hand                          & =CPPP\\
    & een paard bij de toom hebben                           & =CPPP\\
    & vat op iets hebben                                     & id\\
    & God hebbe zijn ziel                                    & id/sprw\\
s13 & de tekst heeft hier een zetfout                        & !!\\
\hline
\end{tabular}

Correcte betekenissen:

\begin{tabular}[t]{|c|p{.1\textwidth}|p{.4\textwidth}|p{.3\textwidth}|}
\hline
Theta & patterns & Betekenis             & typering\\
\hline
vp120 & synNP    & s1"arg1 bezit arg2"     & arg1=$<$human$>$\\
      &          &                       & arg2=$<$possessable concrete$>$ \\
      &          & s2"arg1 beschikt over,
                   is toegerust met arg2"& arg1=$<$?entity$>$\\
      &          &                       & arg2=$<$?abstract$>$\\
      &          & s3"mbt betrekking"    & arg1=?\\
      &          &                       & arg2=$<$relational entity$>$\\
\hline
vp120 & synLCPPP,
        synOCPPP,
        synCAPP  & s3"arg1 bevindt zich in
                    de toestand waarin
                    arg2 waar is"        & arg1=$<$animate$>$\\
      &          &                       & arg2=$<$(location) t$>$\\
\hline
vp120 & synPREPNP,
        +{\em aan}
                 & s4"iets aan iets/iemand hebben" & arg1=$<$human$>$\\
      &          &                                 & arg2=$<$entity$>$\\
\hline
\end{tabular}\\

Voorbeelden:
\begin{enumerate}
  \item hij heeft een auto
  \item hij heeft een scherp gehoor;   Het schip heeft 4 zeilen
  \item Hij heeft een oom, vriendin, zoon
  \item hij heeft een auto in de garage; een boek te leen; het koud
  \item Daar hebben we niets aan.
\end{enumerate}

Andere gevalen met {\em hebben}:

\begin{description}
  \item[Idiomen] het land hebben aan, een hekel hebben aan ; geen klagen hebben
goed praten hebben; (veel) kunnen hebben 'kunnen verdragen'; er niets van
moeten hebben 'er niet mee te maken willen hebben'; willen hebben dat 'willen
dat'; iets niet willen hebben; iets(!) met iemand hebben , het ervan
hebben; het over iemand hebben; veel van iemand hebben 'veel lijken op';
het (niet) op iemand hebben; aan iets een broertje dood hebben; het niet meer
hebben
  \item[+te]
iets te betekenen hebben,
met iemand te doen hebben(medelijden hebben),
met iemand te doen hebben (met iem. te maken hebben), iets te zeggen/berichten hebben
hij heeft duizend gulden te verteren=lit.
  \item[??] mag ik dat hebben; iets willen hebben; we zullen hem hebben(?);
wat had u gehad willen hebben?
zijn hele hebben en houden;
moety je net Freek hebben;
iets moeten hebben, waar wil je me hebben=CLPPP;
ik wou je wijzer hebben=CAPP.
iets zus of zo (gedaan) willen hebben;
zij hebben elkaar gelukkig nog;
kun je het nogal hebben?
Hoe heb ik het nu?
Hoe hebben we het nu met elkaar
heeft u nog iets? (i.e. te vragen)
dat heb je ervan;
dan heb je dat,
daar zullen we (zal je) het hebben
daar heb je het al
daar heb je de kerk al
je hebt/ men heeft = er is/ zijn
wat zullen we nu hebben?
ik heb het (dwz. gevonden)
liever hebben 'prefereren (te hebben?)'
iemand tot vrouw/man hebben
de hoeveelste hebben we?
wanneer hebben we Pasen?
hij heeft heel wat gehad
\item[CAPPP] het goed, het kwaad het eenzaam hebben; het druk hebben.
ik heb het tien over drie.  Hoe laat heeft u het ; het uit de eerste hand
hebben, ik heb het niet van mezelf.
het aan de voet hebben
\end{description}


Verdere bijzonderheden:







Verdere gevallen met {\em hebben}:
\begin{description}
  \item[het heeft er veel van dat]
  \item[ een boek te leen hebben] =1) hij heeft een boek ter uitlening (=s3);\\
=2) een boek geleend hebben = idioom.

  \item[gehad hebben] = gekregen hebben (alleen in de voltooide tijd)
hij heeft een zoen gehad; een cadeau gehad
  \item[te V hebben] (+{\em maar})= {\em moeten}:
hij heeft dit probleem maar op te lossen = CVPPP
zonder {\em maar}= s3
  \item[hebben + van] (hij heeft dit van Marie ) $\approx${\em gehad hebben}
\end{description}

\begin{description}
\item[feitelijke gedaan] s1,s5;s14\\
betekenissen s2 en s3 zijn tijdelijk uitgezet
\end{description}


\newpage
\section{laten}

Van Dale vulling:\\
\begin{tabular}[t]{|c|c|p{.2\textwidth}|p{.3\textwidth}|}
\hline
Skey          & Mkey              & Omschrijving & Voorbeeld\\
\hline
\$s\_aV\_00\_laat & \$m\_aV\_0001\_laat   & s1 "achterwege laten" & hij kan het niet
                                                             laten\\
              & \$m\_aV\_0002\_laat   & s2 "op een plaats/in een toestand houden"&
                                          \\
              & \$m\_aV\_0003\_laat   & s3 "achterlaten" & Waar heb ik het
                                                        potlood gelaten?\\
              & \$m\_aV\_0004\_laat   & s4 "ergens in bergen" & waar moet ik het
                                                             boek laten?\\
              & \$m\_aV\_0005\_laat   & s5 "toegang geven tot" & laat de kat
                                                              maar in de tuin\\
              & \$m\_aV\_0006\_laat   & s6 "toestaan, dulden"  & laat de kinderen
                                                               maar\\
              & \$m\_aV\_0007\_laat   & s7 "veroorzaken, $<$+ actief object$>$"  & de dokter laten komen\\
              & \$m\_aV\_0008\_laat   & s8 "veroorzaken, $<$+ passief object$>$" & ??\\
              & \$m\_aV\_0009\_laat   & s9 "niet inhouden" & een traantje, boertje   laten\\
              & \$m\_aV\_0010\_laat   & s10 "op een plaats/in een toestand brengen" & de lamp naar beneden laten\\
              & \$m\_aV\_0011\_laat   & s11 "opgeven" & het leven laten\\
              & \$m\_aV\_0012\_laat   & s12 "iets niet ontnemen" & iemand het leven laten\\
              & \$m\_aV\_0013\_laat   & s13 "bij zijn dood nalaten" & hij liet mij zijn huis\\
              & \$m\_aV\_0014\_laat   & s14 "afstaan" & ik laat hem de eer\\
              & \$m\_aV\_0015\_laat   & s15 "in stand houden" & --\\
\hline
\$s\_aV\_01\_laat & \$m\_aV\_0101\_laat   & s1 "$<$mbt. wenselijkheid, aansporing$>$" & Laten we opschieten\\
              & \$m\_aV\_0102\_laat   & s2 "$<$mbt. mogelijkheid$>$" & laat ie mooi zijn, verstandig is ie niet\\
              & \$m\_aV\_0103\_laat   & s3 "$<$in uitroepen$>$" & laat hij het nu nog doen ook!\\
\hline
\end{tabular}\\


Herziene entries:\\

\begin{tabular}[t]{|c|p{.1\textwidth}|p{.3\textwidth}|p{.3\textwidth}|p{.1\textwidth}}
\hline
Theta & Patterns &  Omschrijving & Voorbeeld & Types\\
\hline
vp120 & synNP       & s1 "achterwege laten"  & hij liet het roken & animate-t\\
vp120 & synLCPPP,
        synOCPPP,   & s2 "op een plaats/toestand houden" & hij liet hem in de waan; hij liet hem in het huis& animate-loc t\\
      & synDCPPP    & s5 "iemand ergens toelaten         & ik liet hem in de kamer& animate-loc t\\
      &             & s10 "naar een plaats laten"         & ik liet hem de kamer in animate-dir t\\
      & synCIS,
       synDACT,    & s7 "doen (causatief)"              & Ik liet hem zwemmen; ik liet een auto door hem maken& human-t\\
      & synAACT     & s8 "toelaten (permissief)"         & Ik liet het boek aan hem lezen& human-t\\
vp123 & synIONP\_DONP& s13 "bij zijn dood nalaten"     & ik liet hem het huis& human-concrete-human\\
      &             & s14 "afstaan"                   & ik liet hem de eer& human-entity-human\\
\hline
\end{tabular}\\



Opmerkingen:
\begin{description}
  \item[s2,s5,s10] deze worden verder door semantische typering (van het
complement) onderscheiden
  \item[s7,s8] deze worden verder door semantische typering (van het
complement) onderscheiden
  \item[s3,s4] Dit zijn m.i. specifieke interpretaties van bet. s2. S3 en s4
kunnen niet voorkomen buiten de in de voorbeeldzinnen gebruikte tijden
(voltooide tijd vs. toek. tijd)
  \item[s6] is een uitdrukking
  \item[s9] {\em een boer laten} etc. zijn semi-idiomen
  \item[s11] {\em  het leven laten} is een idioom
  \item[s12] {\em iemand het leven laten} is een idioom
  \item[s15] ?????
\end{description}

Het andere werkwoord {\em  laten} (\$s\_aV\_01\_laat) betreft gevallen waarbij het
werkwoord syncategorematisch moet woren geintroduceerd. Hiervoor hoeven geen
aparte entries te worden opgenomen, dus deze kunnen alle vervallen.

\begin{description}
  \item[idiomen]  iets blauwblauw laten, zich iets laten gezeggen, iets laten
lopen, laten steken(?), laten gebeuren(?), iets laten varen, iemand laten
zitten, zijn oog laten gaan over, zijn gedachten laten gaan over, het ergens
bij laten, het erbij laten zitten, alles bij het oude laten, iets in het midden
laten, iemand met rust laten, (veel) geld laten(?)
  \item[andere uitdrukkingen] leven en laten leven, laat staan (dat), laat
maar!, laat maar zitten
\end{description}

\begin{description}
\item[feitelijke gedaan] s1;s13,s14;s7,s8,s2,s5,s10 (in auxverb). De
oorspronkelijke opsplitsing in een permissief en een causatief {\em laten} is
ongedaan gemaakt. De correlatie met synpatterns is of niet correct, of indien
correct, kan door type-restricties verantwoord worden, zodat een opsplitsing
in de syntaxis niet noodzakelijk is.
\end{description}


\newpage
\section{kunnen}

Van Dale vulling:\\
\begin{tabular}[t]{|c|c|p{.2\textwidth}|p{.3\textwidth}|}
\hline
Skey          & Mkey              & Omschrijving & Voorbeeld\\
\hline
\$s\_aV\_20\_kan  & \$m\_aV\_2001\_kan    & s1 "mbt. bekwaamheid" & ok\\
              & \$m\_aV\_2002\_kan    & s2 "mbt. mogelijkheid inherent aan onderwerp"&??\\
\$s\_aV\_21\_kan  & \$m\_aV\_2101\_kan    & s1 "mbt. mogelijkheid zoals geschat door spreker"&ok\\
\$s\_aV\_22\_kan  & \$m\_aV\_2201\_kan    & s1 "mbt. toelating"&ok\\
              & \$m\_aV\_2202\_kan    & s2 "mbt. wens/verwensing"&=idioom\\
              & \$m\_aV\_2203\_kan    & s3 "mbt. irritatie"&=idioom\\
              & \$m\_aV\_2204\_kan    & s4 "van een bekwaamheid/mogelijkheid gebruik maken"& =1s1\\
\$s\_aV\_23\_kan  & \$m\_aV\_2301\_kan    & s1 "aanvaardbaar zijn"&=idioom\\
\hline
\end{tabular}\\



Herziene entries: alles weg, behalve wat al in auxverb.dict staat.

\begin{description}
  \item[idiomen] kwaad kunnen,  (geen) goed kunnen doen, iets met iets kunnen,
er wat van kunnen, ergens tegen kunnen, er niet bij kunnen, ergens van op aan
kunnen, er niet over uit kunnen , er mee door kunnen, maar niet op kunnen,
ergens niet uit kunnen (jurk uit stof), tegen iemand op kunnen, buiten iets
kunnen, over iets heen kunnen(?), niet meer kunnen, de pot op kunnen
  \item[andere uitdrukkingen] willen is kunnen, je kan me wat
\end{description}


\begin{description}
\item[feitelijke gedaan] zoals aangegeven
\end{description}

\newpage
\section{mogen}

Van Dale vulling:\\
\begin{tabular}[t]{|c|c|p{.2\textwidth}|p{.3\textwidth}|}
\hline
Skey          & Mkey              & Omschrijving & Voorbeeld\\
\hline
\$s\_aV\_00\_mag  & \$m\_aV\_0001\_mogen  & s1 "toestemming/recht/vrijheid hebben"&ok\\
              & \$m\_aV\_0002\_mogen  & s2 "reden hebben, moeten"& ok\\
              & \$m\_aV\_0003\_mogen  & s3 "mbt. toegeving"&=syncat. intro\\
              & \$m\_aV\_0004\_mogen  & s4 "mbt. een mogelijkheid"& =syncat. intro\\
              & \$m\_aV\_0005\_mogen  & s5 "kunnen"& ??\\
              & \$m\_aV\_0006\_mogen  & s6 "mbt. een wens"& syncat. intro\\
\$s\_aV\_01\_mag  & \$m\_aV\_0101\_mogen  & s1 "sympathiek vinden" & ok\\
\hline
\end{tabular}\\


Herziene entries: de bestaande in auxverb, plus toevoeging van s2 (plus in de semantiek een sterke semantische typering van
het type complementzinnen dat hierbij toegelaten is (stative? only), cf. {\em je mag blij zijn dat je nog leeft}, etc.

Daarnaast, het werkwoord met thetavp=vp120, synNP "sympathiek vinden" (types: arg1=human-arg2=human).

\begin{description}
  \item[idiomen] er mogen zijn/wezen
\end{description}

\begin{description}
\item[feitelijke gedaan] s2 is nog niet toegevoegd; 2s1 is toegevoegd
\end{description}


\newpage
\section{willen}
Van Dale vulling:\\
\begin{tabular}[t]{|c|c|p{.2\textwidth}|p{.3\textwidth}|}
\hline
Skey          & Mkey              & Omschrijving & Voorbeeld\\
\hline
\$s\_aV\_00\_wil  & \$m\_aV\_0001\_wil    & s1 "tot/als wil hebben"&\\
              & \$m\_aV\_0002\_wil    & s2 "de wens uitdrukken"&\\
              & \$m\_aV\_0003\_wil    & s3 "lukken"&\\
              & \$m\_aV\_0004\_wil    & s4 "beweren"&\\
\$s\_aV\_01\_wil  & \$m\_aV\_0101\_wil    & s1 "zullen"&\\
              & \$m\_aV\_0102\_wil    & s2 "mbt. een gebod, verzoek"&\\
              & \$m\_aV\_0103\_wil    & s3 "mbt. een mogelijkheid, waarschijnlijkheid"&\\
\hline
\end{tabular}\\

Herziene entry:

Zoals in auxverb.dict. , met eventueel daarbij, 1s3 (negative polar); en eventueel 2s3, als hier een systematisch vertaling voor
gevonden kan worden.


\begin{description}
\item[feitelijke gedaan] 1s3 en 2s3 moeten nog toegevoegd
\end{description}


\newpage
\section{moeten}


Van Dale entry:

\begin{tabular}[t]{|c|c|p{.2\textwidth}|p{.3\textwidth}|}
\hline
Skey          & Mkey              & Omschrijving & Voorbeeld\\
\hline
\$s\_aV\_20\_moet & \$m\_aV\_2001\_moet   & s1 "willen"&\\
              & \$m\_aV\_2002\_moet   & s2 "verplicht zijn, zich verplicht voelen"&\\
              & \$m\_aV\_2003\_moet   & s3 "behoren"&\\
              & \$m\_aV\_2004\_moet   & s4 "logisch onvermijdelijk / noodzakelijk zijn"&\\
              & \$m\_aV\_2005\_moet   & s5 "waar(schijnlijk) zijn"&\\
              & \$m\_aV\_2006\_moet   & s6 "$<$AZN$>$ behoeven"&\\
\$s\_aV\_21\_moet & \$m\_aV\_2101\_moet   & s1 "mogen, believen"&\\
\hline
\end{tabular}\\

Herziene entry: zoals in auxverb.dict; alles hierboven gewoon weg, behalve ms. 2s2, negatief polair, (ik moet die man niet).
Misschien moet een onderscheid zoals tussen s3 en s2 ivm bv Duits {\em  sollen}
vs. {\em mussen} gemaakt worden.

\begin{description}
  \item[idiomen] nodig moeten
\end{description}



\begin{description}
\item[feitelijke gedaan] zoals aangegeven
\end{description}

\newpage
Van Dale vulling:\\
\section{zullen}

\begin{tabular}[t]{|c|c|p{.2\textwidth}|p{.3\textwidth}|}
\hline
Skey          & Mkey            & Omschrijving & Voorbeeld\\
\hline
\$s\_aV\_00\_zal & \$m\_aV\_0001\_zullen & s1 "moeten"&\\
\$s\_aV\_01\_zal & \$m\_aV\_0101\_zullen & s1 "$<$ter vorming van de toekomende tijd$>$"&\\
             & \$m\_aV\_0102\_zullen & s2 "$<$van modaliteit$>$"&\\
\hline
\end{tabular}\\

herziene entry, zie auxverb.dict.

\begin{description}
\item[feitelijke gedaan] zoals aangegeven
\end{description}

\newpage
\section{zitten}

s14 van de entry in de lexico's dient te verdwijnen.
De rest dient gewoon goed ingevuld
te worden.
Voorstel ter correcte vulling van de rest:

Van Dale vulling, met commentaar van de lexico's:\\
\begin{tabular}[t]{|c|c|p{.2\textwidth}|p{.3\textwidth}|}
\hline
Skey          & Mkey              & Omschrijving & Voorbeeld\\
\hline
\$s\_aV\_00\_zit
  & \$m\_aV\_0001\_zit    & s1 "gezeten zijn"& {??'hij zit'}\\
              & \$m\_aV\_0002\_zit    & s2 "zich met een doel ergens bevinden" & {??'hij zit aan de koffie'
                                                                             \to vp120 synLOCOPENPREPPPROP}\\
              & \$m\_aV\_0003\_zit    & s3 "een functie bekleden" & {??'hij zit in het bestuur' *** s2}\\
              & \$m\_aV\_0004\_zit    & s4 "geruime tijd ergens vertoeven" & {??\#zie s2}\\
              & \$m\_aV\_0005\_zit    & s5 "wonen" & {?? *** s2}\\
              & \$m\_aV\_0006\_zit    & s6 "verblijven" & {??\#zie s2}\\
              & \$m\_aV\_0007\_zit    & s7 "zich bevinden in de genoemde toestand" &
                        {??id? 'in de problemen zitten','met problemen zitten','op zware lasten zitten'
zitten op,met,in \to vp120 synPREPNP}\\
              & \$m\_aV\_0008\_zit    & s8 "mbt. een volharden in, gelaten worden op een plaat" &
                           {??id, alleen in combinatie met blijven/laten}\\
              & \$m\_aV\_0009\_zit    & s9 "mbt. zaken, zich bevinden" & {??'in sla zit veel vitamine c' ***  s2} \\
              & \$m\_aV\_0010\_zit    & s10 "mbt. kleding" & {??'die jas zit goed', verplichte modificatie}\\
              & \$m\_aV\_0011\_zit    & s11 "bevestigd zijn" & {??id}\\
\hline
\end{tabular}\\

\begin{tabular}[t]{|c|c|p{.2\textwidth}|p{.3\textwidth}|}
\hline
Skey          & Mkey              & Omschrijving & Voorbeeld\\
\hline
              & \$m\_aV\_0012\_zit    & s12 "gevuld, bedekt zijn met" & {??vol en onder zitten *** s7}\\
              & \$m\_aV\_0013\_zit    & s13 "treffen"& {??id}\\
              & \$m\_aV\_0014\_zit    & s14 "$<$met onbep. w.$>$bezig zijn met" & {??=progaux}\\
              & \$m\_aV\_0015\_zit    & s15 "$<$met `op'$>$lid zijn van, beoefenen" & {?? \to vp120 synPREPNP}\\
              & \$m\_aV\_0016\_zit    & s16 "gevangen gehouden worden" & {??'hij zit al vier jaar'}\\
\hline
\end{tabular}

\begin{tabular}[t]{|c|c|p{.2\textwidth}|p{.2\textwidth}|p{.1\textwidth}|}
\hline
Theta & Patterns &  Omschrijving & Voorbeeld & Types\\
\hline
vp100 & synnovpargs & s1 "gezeten zijn" & {'hij zit'} & animate\\
      &             & s16 "gevangen gehouden worden" & {'hij zit al vier jaar'}& human\\
vp120 & synLOPP,    & s2 "zich bevinden" & er zit een vogel in de tuin & concrete-loc\\
      & synOAPP?    & s10 "mbt. kleding" & {'die jas zit goed', verplichte modificatie}& cloth-quality\\
\hline
\end{tabular}\\

Opmerkingen:
\begin{description}
  \item[s2] idioom
  \item[s3] idioom
  \item[s4] = s2
  \item[s5, s6] = s2
  \item[s7] idiomen?
  \item[s8] idiomen
  \item[s9] = s2
  \item[s11] = s2 of idioom.
  \item[s12] = idioom
  \item[s13] = idioom
  \item[s14] = zie auxverb
  \item[s15] = idioom
\end{description}


\begin{description}
  \item[Idiomen] aan de koffie zitten, in iets zitten (human-organization(s3)), (?)in de problemen zitten, blijven zitten
(school),
(?)op zware lasten zitten, vol zitten met, onder iets zitten, op iets zitten (club, sport)

  \item[andere uitdrukkingen] die zit;
\end{description}

\begin{description}
\item[feitelijke gedaan] s1,s16;s2; s10 moet nog toegevoegd worden (hier is
geen goed pattern voor)
\end{description}


\newpage
\section{gaan}

De directionele bepaling toelatende betekenissen van gaan moeten onder een
ww gebracht worden; de skey hiervan dient gelijk gesteld te worden aan
de in de Mregels genoemde skey.


Voorstel voor de correcte vulling van de rest:

Van Dale vulling, met commentaar van de lexico's:\\
\begin{tabular}[t]{|c|c|p{.2\textwidth}|p{.3\textwidth}|}
\hline
Skey          & Mkey              & Omschrijving & Voorbeeld\\
\hline
\$s\_aV\_00\_ga   & \$m\_aV\_0001\_ga     & s1 "zich verplaatsen"&{??'hij gaat (naar huis)'}\\
              & \$m\_aV\_0002\_ga     & s2 "vertrekken, weggaan"& {??'hoe laat gaat de trein' \to vp010 synNP}\\
              & \$m\_aV\_0003\_ga     & s3 "zich begeven"    & {??\#zie s1}\\
              & \$m\_aV\_0004\_ga     & s4 "$<$+ onbep. wijs$>$beginnen te" & {??'het ging regenen' \to vp010 synCLOSEDINFSENT}\\
              & \$m\_aV\_0005\_ga     & s5 "in beweging zijn, functioneren"& {??'de bel gaat' *** s2}\\
              & \$m\_aV\_0006\_ga     & s6 "losraken" & {??'die kurk gaat van de fles'?}\\
              & \$m\_aV\_0007\_ga     & s7 "plaatshebben" & {??'de zaken gaan slecht *** s2, verplichte modificatie}\\
              & \$m\_aV\_0008\_ga     & s8 "in een bep. toestand raken" & {??id 'zich laten gaan' 'verloren gaan'}\\
              & \$m\_aV\_0009\_ga     & s9 "lopen" & {??'hij ging langs de straat'?}\\
              & \$m\_aV\_0010\_ga     & s10 "verdwijnen" & {??id 'daar gaat je goede naam'}\\
              & \$m\_aV\_0011\_ga     & s11 "mbt. kleding" & {??\#'hij gaat in uniform'}\\
              & \$m\_aV\_0012\_ga     & s12 "haalbaar/redelijk zijn" & {??id'dat zal niet gaan'}\\
              & \$m\_aV\_0013\_ga     & s13 "begrepen zijn in" & {??'er gaan drie flessen in een doos'
                                                  \to vp012 synDONP\_LOCOPENPREPPPROP},
                                     {??'er gaat 20 liter in de tank' -$>$ to hold
                                      'er gaan 25 mensen in'        -$>$ to seat}\\
\hline
\end{tabular}\\
\begin{tabular}[t]{|c|c|p{.2\textwidth}|p{.3\textwidth}|}
\hline
Skey          & Mkey              & Omschrijving & Voorbeeld\\
\hline
              & \$m\_aV\_0014\_ga     & s14 "$<$+ over$>$beheren" & {??'gaan over' \to vp120 synPREPNP}\\
              & \$m\_aV\_0015\_ga     & s15 "$<$+ over$>$tot onderwerp hebben" & {??'gaan over' *** s14}\\
:\$s\_aV\_01\_ga  & \$m\_aV\_0101\_ga     & s1 "gesteld zijn" & {??id, 'het gaat goed met de man'?}\\
              & \$m\_aV\_0102\_ga     & s2 "geschieden" & {??\#zie s7 vorig lemma}\\
              & \$m\_aV\_0103\_ga     & s3 "mbt. beweging" & {??\#}\\
              & \$m\_aV\_0104\_ga     & s4 "$<$+ om$>$te doen zijn"  & {??id, 'het gaat om mensenlevens'}\\
\hline
\end{tabular}


\begin{tabular}[t]{|c|c|p{.2\textwidth}|p{.2\textwidth}|p{.1\textwidth}|}
\hline
Theta & Patterns &  Omschrijving & Voorbeeld & Types\\
\hline
vp012 & synDONP\_DOPPP,& s1 "zich verplaatsen" &'hij gaat (naar huis)' &  concrete-path\\
      & synDONP\_LOPPP & s2 "vertrekken, weggaan"& 'hoe laat gaat de trein' & concrete-time\\
      &                & s13 "begrepen zijn in, passen in" & er gaan 25 mensen in & concrete-loc\\
vp012 & synOISENT      & s4 "$<$+ onbep. wijs$>$beginnen te" & {'het ging regenen'}& concrete,c-t\\
vp010 & synNP          & s5 "rinkelen" & 'de bel, zoemer, telefoon gaat', ms. sid? & ?\\
      &                & s12 "meevallen, redelijk,zijn" &  het gaat wel & c\\
      &                & s12'"lukken" & het gaat heus wel&t\\
vp012 & synNP\_Adv     & s7 "er voor staan" & 'de zaken gaan slecht & -animate,c-quality\\
      &                & 2s2 "geschieden" & het ging gauw&c-Manner\\
vp012 & synNP\_PNP     & s14 "$<$+ over$>$beheren" & hij gaat over het magazijn & human-entity\\
      &                & s15 "$<$+ over$>$tot onderwerp hebben" & het gesprek gaat over abortus& infocontainer-entity\\
vp010 & synPNP         & s4 "$<$+ om$>$te doen zijn"  & {??id, 'het gaat om mensenlevens'}& ??\\
\hline
\end{tabular}\\

Opmerkingen:

\begin{description}
  \item[s2] cf. voorstel vps of Dutch, + sem typering; hoe laat = om hoe laat.
  \item[s3] = s1
  \item[s4] niet closed, e.g. *Er ging gedanst worden; vgl. voorstel wheather-it.
  \item[s5] betekenis omschrijving veranderd
  \item[s6] dit is misschien een vertaalidioom {\em van iets af gaan}
  \item[s7] pattern bestaat nog niet, ms. zoals middle verb; betekenis omschrijving aangepast
  \item[s8]=idioom
  \item[s9] onzin; {\em in de pas gaan} is geen NL
  \item[s10] onzin
  \item[s11] ms. zoals klinken, etc., sterk getypeerd (mannerofcloths)???
  \item[s13] ???
  \item[2s1]??
  \item[2s2]??
\end{description}

\begin{description}
  \item[idiomen] (?)van iets af gaan (de kurk ging van de fles), verloren gaan
\end{description}


\begin{description}
\item[feitelijke gedaan] s1;s2,s13,s4;s5;s14,s15;\\
              nog niet toegevoegd:  s7,s12, 2s4 (pattern probleem)\\
Voor bet. 1s4 dient een pattern toegevoegd te worden (synDONP\_OPENINFSENT).
\end{description}


\newpage
\section{hoeven}

Staat niet in de lexico dictionaries.

\begin{description}
\item[idiomen] (?)geen betoog hoeven
\end{description}



\end{document}


ROSETTA.sty
\typeout{Document Style 'Rosetta'. Version 0.4 - released  24-DEC-1987}
% 24-DEC-1987:  Date of copyright notice changed
\def\@ptsize{1}
\@namedef{ds@10pt}{\def\@ptsize{0}}
\@namedef{ds@12pt}{\def\@ptsize{2}}
\@twosidetrue
\@mparswitchtrue
\def\ds@draft{\overfullrule 5pt}
\@options
\input art1\@ptsize.sty\relax


\def\labelenumi{\arabic{enumi}.}
\def\theenumi{\arabic{enumi}}
\def\labelenumii{(\alph{enumii})}
\def\theenumii{\alph{enumii}}
\def\p@enumii{\theenumi}
\def\labelenumiii{\roman{enumiii}.}
\def\theenumiii{\roman{enumiii}}
\def\p@enumiii{\theenumi(\theenumii)}
\def\labelenumiv{\Alph{enumiv}.}
\def\theenumiv{\Alph{enumiv}}
\def\p@enumiv{\p@enumiii\theenumiii}
\def\labelitemi{$\bullet$}
\def\labelitemii{\bf --}
\def\labelitemiii{$\ast$}
\def\labelitemiv{$\cdot$}
\def\verse{
   \let\\=\@centercr
   \list{}{\itemsep\z@ \itemindent -1.5em\listparindent \itemindent
      \rightmargin\leftmargin\advance\leftmargin 1.5em}
   \item[]}
\let\endverse\endlist
\def\quotation{
   \list{}{\listparindent 1.5em
      \itemindent\listparindent
      \rightmargin\leftmargin \parsep 0pt plus 1pt}\item[]}
\let\endquotation=\endlist
\def\quote{
   \list{}{\rightmargin\leftmargin}\item[]}
\let\endquote=\endlist
\def\descriptionlabel#1{\hspace\labelsep \bf #1}
\def\description{
   \list{}{\labelwidth\z@ \itemindent-\leftmargin
      \let\makelabel\descriptionlabel}}
\let\enddescription\endlist


\def\@begintheorem#1#2{\it \trivlist \item[\hskip \labelsep{\bf #1\ #2}]}
\def\@endtheorem{\endtrivlist}
\def\theequation{\arabic{equation}}
\def\titlepage{
   \@restonecolfalse
   \if@twocolumn\@restonecoltrue\onecolumn
   \else \newpage
   \fi
   \thispagestyle{empty}\c@page\z@}
\def\endtitlepage{\if@restonecol\twocolumn \else \newpage \fi}
\arraycolsep 5pt \tabcolsep 6pt \arrayrulewidth .4pt \doublerulesep 2pt
\tabbingsep \labelsep
\skip\@mpfootins = \skip\footins
\fboxsep = 3pt \fboxrule = .4pt


\newcounter{part}
\newcounter {section}
\newcounter {subsection}[section]
\newcounter {subsubsection}[subsection]
\newcounter {paragraph}[subsubsection]
\newcounter {subparagraph}[paragraph]
\def\thepart{\Roman{part}} \def\thesection {\arabic{section}}
\def\thesubsection {\thesection.\arabic{subsection}}
\def\thesubsubsection {\thesubsection .\arabic{subsubsection}}
\def\theparagraph {\thesubsubsection.\arabic{paragraph}}
\def\thesubparagraph {\theparagraph.\arabic{subparagraph}}


\def\@pnumwidth{1.55em}
\def\@tocrmarg {2.55em}
\def\@dotsep{4.5}
\setcounter{tocdepth}{3}
\def\tableofcontents{\section*{Contents\markboth{}{}}
\@starttoc{toc}}
\def\l@part#1#2{
   \addpenalty{-\@highpenalty}
   \addvspace{2.25em plus 1pt}
   \begingroup
      \@tempdima 3em \parindent \z@ \rightskip \@pnumwidth \parfillskip
      -\@pnumwidth {\large \bf \leavevmode #1\hfil \hbox to\@pnumwidth{\hss #2}}
      \par \nobreak
   \endgroup}
\def\l@section#1#2{
   \addpenalty{-\@highpenalty}
   \addvspace{1.0em plus 1pt}
   \@tempdima 1.5em
   \begingroup
      \parindent \z@ \rightskip \@pnumwidth
      \parfillskip -\@pnumwidth
      \bf \leavevmode #1\hfil \hbox to\@pnumwidth{\hss #2}
      \par
   \endgroup}
\def\l@subsection{\@dottedtocline{2}{1.5em}{2.3em}}
\def\l@subsubsection{\@dottedtocline{3}{3.8em}{3.2em}}
\def\l@paragraph{\@dottedtocline{4}{7.0em}{4.1em}}
\def\l@subparagraph{\@dottedtocline{5}{10em}{5em}}
\def\listoffigures{
   \section*{List of Figures\markboth{}{}}
   \@starttoc{lof}}
   \def\l@figure{\@dottedtocline{1}{1.5em}{2.3em}}
   \def\listoftables{\section*{List of Tables\markboth{}{}}
   \@starttoc{lot}}
\let\l@table\l@figure


\def\thebibliography#1{
   \addcontentsline{toc}
   {section}{References}\section*{References\markboth{}{}}
   \list{[\arabic{enumi}]}
        {\settowidth\labelwidth{[#1]}\leftmargin\labelwidth
         \advance\leftmargin\labelsep\usecounter{enumi}}}
\let\endthebibliography=\endlist


\newif\if@restonecol
\def\theindex{
   \@restonecoltrue\if@twocolumn\@restonecolfalse\fi
   \columnseprule \z@
   \columnsep 35pt\twocolumn[\section*{Index}]
   \markboth{}{}
   \thispagestyle{plain}\parindent\z@
   \parskip\z@ plus .3pt\relax
   \let\item\@idxitem}
\def\@idxitem{\par\hangindent 40pt}
\def\subitem{\par\hangindent 40pt \hspace*{20pt}}
\def\subsubitem{\par\hangindent 40pt \hspace*{30pt}}
\def\endtheindex{\if@restonecol\onecolumn\else\clearpage\fi}
\def\indexspace{\par \vskip 10pt plus 5pt minus 3pt\relax}


\def\footnoterule{
   \kern-1\p@
   \hrule width .4\columnwidth
   \kern .6\p@}
\long\def\@makefntext#1{
   \@setpar{\@@par\@tempdima \hsize
   \advance\@tempdima-10pt\parshape \@ne 10pt \@tempdima}\par
   \parindent 1em\noindent \hbox to \z@{\hss$^{\@thefnmark}$}#1}


\setcounter{topnumber}{2}
\def\topfraction{.7}
\setcounter{bottomnumber}{1}
\def\bottomfraction{.3}
\setcounter{totalnumber}{3}
\def\textfraction{.2}
\def\floatpagefraction{.5}
\setcounter{dbltopnumber}{2}
\def\dbltopfraction{.7}
\def\dblfloatpagefraction{.5}
\long\def\@makecaption#1#2{
   \vskip 10pt
   \setbox\@tempboxa\hbox{#1: #2}
   \ifdim \wd\@tempboxa >\hsize \unhbox\@tempboxa\par
   \else \hbox to\hsize{\hfil\box\@tempboxa\hfil}
   \fi}
\newcounter{figure}
\def\thefigure{\@arabic\c@figure}
\def\fps@figure{tbp}
\def\ftype@figure{1}
\def\ext@figure{lof}
\def\fnum@figure{Figure \thefigure}
\def\figure{\@float{figure}}
\let\endfigure\end@float
\@namedef{figure*}{\@dblfloat{figure}}
\@namedef{endfigure*}{\end@dblfloat}
\newcounter{table}
\def\thetable{\@arabic\c@table}
\def\fps@table{tbp}
\def\ftype@table{2}
\def\ext@table{lot}
\def\fnum@table{Table \thetable}
\def\table{\@float{table}}
\let\endtable\end@float
\@namedef{table*}{\@dblfloat{table}}
\@namedef{endtable*}{\end@dblfloat}


\def\maketitle{
   \par
   \begingroup
      \def\thefootnote{\fnsymbol{footnote}}
      \def\@makefnmark{\hbox to 0pt{$^{\@thefnmark}$\hss}}
      \if@twocolumn \twocolumn[\@maketitle]
      \else \newpage \global\@topnum\z@ \@maketitle
      \fi
      \thispagestyle{plain}
      \@thanks
   \endgroup
   \setcounter{footnote}{0}
   \let\maketitle\relax
   \let\@maketitle\relax
   \gdef\@thanks{}
   \gdef\@author{}
   \gdef\@title{}
   \let\thanks\relax}
\def\@maketitle{
   \newpage
   \null
   \vskip 2em
   \begin{center}{\LARGE \@title \par}
      \vskip 1.5em
      {\large \lineskip .5em \begin{tabular}[t]{c}\@author \end{tabular}\par}
      \vskip 1em {\large \@date}
   \end{center}
   \par
   \vskip 1.5em}
\def\abstract{
   \if@twocolumn \section*{Abstract}
   \else
      \small
      \begin{center} {\bf Abstract\vspace{-.5em}\vspace{0pt}} \end{center}
      \quotation
   \fi}
\def\endabstract{\if@twocolumn\else\endquotation\fi}


\mark{{}{}}
\if@twoside
   \def\ps@headings{
      \def\@oddfoot{Rosetta Doc. \@RosDocNr\hfil \@RosDate}
      \def\@evenfoot{Rosetta Doc. \@RosDocNr\hfil \@RosDate}
      \def\@evenhead{\rm\thepage\hfil \sl \rightmark}
      \def\@oddhead{\hbox{}\sl \leftmark \hfil\rm\thepage}
      \def\sectionmark##1{\markboth {}{}}
      \def\subsectionmark##1{}}
\else
   \def\ps@headings{
      \def\@oddfoot{Rosetta Doc. \@RosDocNr\hfil \@RosDate}
      \def\@evenfoot{Rosetta Doc. \@RosDocNr\hfil \@RosDate}
      \def\@oddhead{\hbox{}\sl \rightmark \hfil \rm\thepage}
      \def\sectionmark##1{\markboth {}{}}
      \def\subsectionmark##1{}}
\fi
\def\ps@myheadings{
   \def\@oddhead{\hbox{}\sl\@rhead \hfil \rm\thepage}
   \def\@oddfoot{}
   \def\@evenhead{\rm \thepage\hfil\sl\@lhead\hbox{}}
   \def\@evenfoot{}
   \def\sectionmark##1{}
   \def\subsectionmark##1{}}


\def\today{
   \ifcase\month\or January\or February\or March\or April\or May\or June\or
      July\or August\or September\or October\or November\or December
   \fi
   \space\number\day, \number\year}


\ps@plain \pagenumbering{arabic} \onecolumn \if@twoside\else\raggedbottom\fi




% the Rosetta title page
\newcommand{\MakeRosTitle}{
   \begin{titlepage}
      \begin{large}
     \begin{figure}[t]
        \begin{picture}(405,100)(0,0)
           \put(0,100){\line(1,0){404}}
           \put(0,75){Project {\bf Rosetta}}
           \put(93.5,75){:}
           \put(108,75){Machine Translation}
           \put(0,50){Topic}
           \put(93.5,50){:}
           \put(108,50){\@RosTopic}
           \put(0,30){\line(1,0){404}}
        \end{picture}
     \end{figure}
     \bigskip
     \bigskip
     \begin{list}{-}{\setlength{\leftmargin}{3.0cm}
             \setlength{\labelwidth}{2.7cm}
             \setlength{\topsep}{2cm}}
        \item [{\rm Title \hfill :}] {{\bf \@RosTitle}}
        \item [{\rm Author \hfill :}] {\@RosAuthor}
        \bigskip
        \bigskip
        \bigskip
        \item [{\rm Doc.Nr. \hfill :}] {\@RosDocNr}
        \item [{\rm Date \hfill :}] {\@RosDate}
        \item [{\rm Status \hfill :}] {\@RosStatus}
        \item [{\rm Supersedes \hfill :}] {\@RosSupersedes}
        \item [{\rm Distribution \hfill :}] {\@RosDistribution}
        \item [{\rm Clearance \hfill :}] {\@RosClearance}
        \item [{\rm Keywords \hfill :}] {\@RosKeywords}
     \end{list}
      \end{large}
      \title{\@RosTitle}
      \begin{figure}[b]
     \begin{picture}(404,64)(0,0)
        \put(0,64){\line(1,0){404}}
        \put(0,-4){\line(1,0){404}}
        \put(0,59){\line(1,0){42}}
        \begin{small}
        \put(3,48){\sf PHILIPS}
        \end{small}
        \put(0,23){\line(0,1){36}}
        \put(42,23){\line(0,1){36}}
        \put(21,23){\oval(42,42)[bl]}
        \put(21,23){\oval(42,42)[br]}
        \put(21,23){\circle{40}}
        \put(4,33){\line(1,0){10}}
        \put(9,28){\line(0,1){10}}
        \put(9,36){\line(1,0){6}}
        \put(12,33){\line(0,1){6}}
        \put(29,13){\line(1,0){10}}
        \put(34,8){\line(0,1){10}}
        \put(28,10){\line(1,0){6}}
        \put(31,7){\line(0,1){6}}

        \put(1,21){\line(1,0){0.5}}
        \put(1.5,21.3){\line(1,0){0.5}}
        \put(2,21.6){\line(1,0){0.5}}
        \put(2.5,21.9){\line(1,0){0.5}}
        \put(3,22.1){\line(1,0){0.5}}
        \put(3.5,22.3){\line(1,0){0.5}}
        \put(4,22.5){\line(1,0){0.5}}
        \put(4.5,22.7){\line(1,0){0.5}}
        \put(5,22.8){\line(1,0){0.5}}
        \put(5.5,22.9){\line(1,0){0.5}}
        \put(6,23){\line(1,0){0.5}}
        \put(6.5,22.9){\line(1,0){0.5}}
        \put(7,22.8){\line(1,0){0.5}}
        \put(7.5,22.7){\line(1,0){0.5}}
        \put(8,22.5){\line(1,0){0.5}}
        \put(8.5,22.3){\line(1,0){0.5}}
        \put(9,22.1){\line(1,0){0.5}}
        \put(9.5,21.9){\line(1,0){0.5}}
        \put(10,21.6){\line(1,0){0.5}}
        \put(10.5,21.3){\line(1,0){0.5}}

        \put(1,23){\line(1,0){0.5}}
        \put(1.5,23.3){\line(1,0){0.5}}
        \put(2,23.6){\line(1,0){0.5}}
        \put(2.5,23.9){\line(1,0){0.5}}
        \put(3,24.1){\line(1,0){0.5}}
        \put(3.5,24.3){\line(1,0){0.5}}
        \put(4,24.5){\line(1,0){0.5}}
        \put(4.5,24.7){\line(1,0){0.5}}
        \put(5,24.8){\line(1,0){0.5}}
        \put(5.5,24.9){\line(1,0){0.5}}
        \put(6,25){\line(1,0){0.5}}
        \put(6.5,24.9){\line(1,0){0.5}}
        \put(7,24.8){\line(1,0){0.5}}
        \put(7.5,24.7){\line(1,0){0.5}}
        \put(8,24.5){\line(1,0){0.5}}
        \put(8.5,24.3){\line(1,0){0.5}}
        \put(9,24.1){\line(1,0){0.5}}
        \put(9.5,23.9){\line(1,0){0.5}}
        \put(10,23.6){\line(1,0){0.5}}
        \put(10.5,23.3){\line(1,0){0.5}}

        \put(1,25){\line(1,0){0.5}}
        \put(1.5,25.3){\line(1,0){0.5}}
        \put(2,25.6){\line(1,0){0.5}}
        \put(2.5,25.9){\line(1,0){0.5}}
        \put(3,26.1){\line(1,0){0.5}}
        \put(3.5,26.3){\line(1,0){0.5}}
        \put(4,26.5){\line(1,0){0.5}}
        \put(4.5,26.7){\line(1,0){0.5}}
        \put(5,26.8){\line(1,0){0.5}}
        \put(5.5,26.9){\line(1,0){0.5}}
        \put(6,27){\line(1,0){0.5}}
        \put(6.5,26.9){\line(1,0){0.5}}
        \put(7,26.8){\line(1,0){0.5}}
        \put(7.5,26.7){\line(1,0){0.5}}
        \put(8,26.5){\line(1,0){0.5}}
        \put(8.5,26.3){\line(1,0){0.5}}
        \put(9,26.1){\line(1,0){0.5}}
        \put(9.5,25.9){\line(1,0){0.5}}
        \put(10,25.6){\line(1,0){0.5}}
        \put(10.5,25.3){\line(1,0){0.5}}

        \put(11,21){\line(1,0){0.5}}
        \put(11.5,20.7){\line(1,0){0.5}}
        \put(12,20.4){\line(1,0){0.5}}
        \put(12.5,20.1){\line(1,0){0.5}}
        \put(13,19.9){\line(1,0){0.5}}
        \put(13.5,19.7){\line(1,0){0.5}}
        \put(14,19.5){\line(1,0){0.5}}
        \put(14.5,19.3){\line(1,0){0.5}}
        \put(15,19.2){\line(1,0){0.5}}
        \put(15.5,19.1){\line(1,0){0.5}}
        \put(16,19){\line(1,0){0.5}}
        \put(16.5,19.1){\line(1,0){0.5}}
        \put(17,19.2){\line(1,0){0.5}}
        \put(17.5,19.3){\line(1,0){0.5}}
        \put(18,19.5){\line(1,0){0.5}}
        \put(18.5,19.7){\line(1,0){0.5}}
        \put(19,19.9){\line(1,0){0.5}}
        \put(19.5,20.1){\line(1,0){0.5}}
        \put(20,20.4){\line(1,0){0.5}}
        \put(20.5,20.7){\line(1,0){0.5}}

        \put(11,23){\line(1,0){0.5}}
        \put(11.5,22.7){\line(1,0){0.5}}
        \put(12,22.4){\line(1,0){0.5}}
        \put(12.5,22.1){\line(1,0){0.5}}
        \put(13,21.9){\line(1,0){0.5}}
        \put(13.5,21.7){\line(1,0){0.5}}
        \put(14,21.5){\line(1,0){0.5}}
        \put(14.5,21.3){\line(1,0){0.5}}
        \put(15,21.2){\line(1,0){0.5}}
        \put(15.5,21.1){\line(1,0){0.5}}
        \put(16,21){\line(1,0){0.5}}
        \put(16.5,21.1){\line(1,0){0.5}}
        \put(17,21.2){\line(1,0){0.5}}
        \put(17.5,21.3){\line(1,0){0.5}}
        \put(18,21.5){\line(1,0){0.5}}
        \put(18.5,21.7){\line(1,0){0.5}}
        \put(19,21.9){\line(1,0){0.5}}
        \put(19.5,22.1){\line(1,0){0.5}}
        \put(20,22.4){\line(1,0){0.5}}
        \put(20.5,22.7){\line(1,0){0.5}}

        \put(11,25){\line(1,0){0.5}}
        \put(11.5,24.7){\line(1,0){0.5}}
        \put(12,24.4){\line(1,0){0.5}}
        \put(12.5,24.1){\line(1,0){0.5}}
        \put(13,23.9){\line(1,0){0.5}}
        \put(13.5,23.7){\line(1,0){0.5}}
        \put(14,23.5){\line(1,0){0.5}}
        \put(14.5,23.3){\line(1,0){0.5}}
        \put(15,23.2){\line(1,0){0.5}}
        \put(15.5,23.1){\line(1,0){0.5}}
        \put(16,23){\line(1,0){0.5}}
        \put(16.5,23.1){\line(1,0){0.5}}
        \put(17,23.2){\line(1,0){0.5}}
        \put(17.5,23.3){\line(1,0){0.5}}
        \put(18,23.5){\line(1,0){0.5}}
        \put(18.5,23.7){\line(1,0){0.5}}
        \put(19,23.9){\line(1,0){0.5}}
        \put(19.5,24.1){\line(1,0){0.5}}
        \put(20,24.4){\line(1,0){0.5}}
        \put(20.5,24.7){\line(1,0){0.5}}

        \put(21,21){\line(1,0){0.5}}
        \put(21.5,21.3){\line(1,0){0.5}}
        \put(22,21.6){\line(1,0){0.5}}
        \put(22.5,21.9){\line(1,0){0.5}}
        \put(23,22.1){\line(1,0){0.5}}
        \put(23.5,22.3){\line(1,0){0.5}}
        \put(24,22.5){\line(1,0){0.5}}
        \put(24.5,22.7){\line(1,0){0.5}}
        \put(25,22.8){\line(1,0){0.5}}
        \put(25.5,23.9){\line(1,0){0.5}}
        \put(26,23){\line(1,0){0.5}}
        \put(26.5,22.9){\line(1,0){0.5}}
        \put(27,22.8){\line(1,0){0.5}}
        \put(27.5,22.7){\line(1,0){0.5}}
        \put(28,22.5){\line(1,0){0.5}}
        \put(28.5,22.3){\line(1,0){0.5}}
        \put(29,22.1){\line(1,0){0.5}}
        \put(29.5,21.9){\line(1,0){0.5}}
        \put(30,21.6){\line(1,0){0.5}}
        \put(30.5,21.3){\line(1,0){0.5}}

        \put(21,23){\line(1,0){0.5}}
        \put(21.5,23.3){\line(1,0){0.5}}
        \put(22,23.6){\line(1,0){0.5}}
        \put(22.5,23.9){\line(1,0){0.5}}
        \put(23,24.1){\line(1,0){0.5}}
        \put(23.5,24.3){\line(1,0){0.5}}
        \put(24,24.5){\line(1,0){0.5}}
        \put(24.5,24.7){\line(1,0){0.5}}
        \put(25,24.8){\line(1,0){0.5}}
        \put(25.5,24.9){\line(1,0){0.5}}
        \put(26,25){\line(1,0){0.5}}
        \put(26.5,24.9){\line(1,0){0.5}}
        \put(27,24.8){\line(1,0){0.5}}
        \put(27.5,24.7){\line(1,0){0.5}}
        \put(28,24.5){\line(1,0){0.5}}
        \put(28.5,24.3){\line(1,0){0.5}}
        \put(29,24.1){\line(1,0){0.5}}
        \put(29.5,23.9){\line(1,0){0.5}}
        \put(30,23.6){\line(1,0){0.5}}
        \put(30.5,23.3){\line(1,0){0.5}}

        \put(21,25){\line(1,0){0.5}}
        \put(21.5,25.3){\line(1,0){0.5}}
        \put(22,25.6){\line(1,0){0.5}}
        \put(22.5,25.9){\line(1,0){0.5}}
        \put(23,26.1){\line(1,0){0.5}}
        \put(23.5,26.3){\line(1,0){0.5}}
        \put(24,26.5){\line(1,0){0.5}}
        \put(24.5,26.7){\line(1,0){0.5}}
        \put(25,26.8){\line(1,0){0.5}}
        \put(25.5,26.9){\line(1,0){0.5}}
        \put(26,27){\line(1,0){0.5}}
        \put(26.5,26.9){\line(1,0){0.5}}
        \put(27,26.8){\line(1,0){0.5}}
        \put(27.5,26.7){\line(1,0){0.5}}
        \put(28,26.5){\line(1,0){0.5}}
        \put(28.5,26.3){\line(1,0){0.5}}
        \put(29,26.1){\line(1,0){0.5}}
        \put(29.5,25.9){\line(1,0){0.5}}
        \put(30,25.6){\line(1,0){0.5}}
        \put(30.5,25.3){\line(1,0){0.5}}

        \put(31,21){\line(1,0){0.5}}
        \put(31.5,20.7){\line(1,0){0.5}}
        \put(32,20.4){\line(1,0){0.5}}
        \put(32.5,20.1){\line(1,0){0.5}}
        \put(33,19.9){\line(1,0){0.5}}
        \put(33.5,19.7){\line(1,0){0.5}}
        \put(34,19.5){\line(1,0){0.5}}
        \put(34.5,19.3){\line(1,0){0.5}}
        \put(35,19.2){\line(1,0){0.5}}
        \put(35.5,19.1){\line(1,0){0.5}}
        \put(36,19){\line(1,0){0.5}}
        \put(36.5,19.1){\line(1,0){0.5}}
        \put(37,19.2){\line(1,0){0.5}}
        \put(37.5,19.3){\line(1,0){0.5}}
        \put(38,19.5){\line(1,0){0.5}}
        \put(38.5,19.7){\line(1,0){0.5}}
        \put(39,19.9){\line(1,0){0.5}}
        \put(39.5,20.1){\line(1,0){0.5}}
        \put(40,20.4){\line(1,0){0.5}}
        \put(40.5,20.7){\line(1,0){0.5}}

        \put(31,23){\line(1,0){0.5}}
        \put(31.5,22.7){\line(1,0){0.5}}
        \put(32,22.4){\line(1,0){0.5}}
        \put(32.5,22.1){\line(1,0){0.5}}
        \put(33,21.9){\line(1,0){0.5}}
        \put(33.5,21.7){\line(1,0){0.5}}
        \put(34,21.5){\line(1,0){0.5}}
        \put(34.5,21.3){\line(1,0){0.5}}
        \put(35,21.2){\line(1,0){0.5}}
        \put(35.5,21.1){\line(1,0){0.5}}
        \put(36,21){\line(1,0){0.5}}
        \put(36.5,21.1){\line(1,0){0.5}}
        \put(37,21.2){\line(1,0){0.5}}
        \put(37.5,21.3){\line(1,0){0.5}}
        \put(38,21.5){\line(1,0){0.5}}
        \put(38.5,21.7){\line(1,0){0.5}}
        \put(39,21.9){\line(1,0){0.5}}
        \put(39.5,22.1){\line(1,0){0.5}}
        \put(40,22.4){\line(1,0){0.5}}
        \put(40.5,22.7){\line(1,0){0.5}}

        \put(31,25){\line(1,0){0.5}}
        \put(31.5,24.7){\line(1,0){0.5}}
        \put(32,24.4){\line(1,0){0.5}}
        \put(32.5,24.1){\line(1,0){0.5}}
        \put(33,23.9){\line(1,0){0.5}}
        \put(33.5,23.7){\line(1,0){0.5}}
        \put(34,23.5){\line(1,0){0.5}}
        \put(34.5,23.3){\line(1,0){0.5}}
        \put(35,23.2){\line(1,0){0.5}}
        \put(35.5,23.1){\line(1,0){0.5}}
        \put(36,23){\line(1,0){0.5}}
        \put(36.5,23.1){\line(1,0){0.5}}
        \put(37,23.2){\line(1,0){0.5}}
        \put(37.5,23.3){\line(1,0){0.5}}
        \put(38,23.5){\line(1,0){0.5}}
        \put(38.5,23.7){\line(1,0){0.5}}
        \put(39,23.9){\line(1,0){0.5}}
        \put(39.5,24.1){\line(1,0){0.5}}
        \put(40,24.4){\line(1,0){0.5}}
        \put(40.5,24.7){\line(1,0){0.5}}
        \begin{large}
           \put(60,45){Philips Research Laboratories}
           \put(60,30){\copyright\ 1988 Nederlandse Philips Bedrijven B.V.}
        \end{large}
     \end{picture}
      \end{figure}
      \newpage
      \pagenumbering{roman}
      \tableofcontents
      \newpage
      \pagenumbering{arabic}
   \end{titlepage}
}
\title{}
\topmargin 0pt
\oddsidemargin 36pt
\evensidemargin 36pt
\textheight 600pt
\textwidth 405pt
\pagestyle{headings}
\newcommand{\@RosTopic}{General}
\newcommand{\@RosTitle}{-}
\newcommand{\@RosAuthor}{-}
\newcommand{\@RosDocNr}{}
\newcommand{\@RosDate}{}
\newcommand{\@RosStatus}{informal}
\newcommand{\@RosSupersedes}{-}
\newcommand{\@RosDistribution}{Project}
\newcommand{\@RosClearance}{Project}
\newcommand{\@RosKeywords}{}
\newcommand{\RosTopic}[1]{\renewcommand{\@RosTopic}{#1}}
\newcommand{\RosTitle}[1]{\renewcommand{\@RosTitle}{#1}}
\newcommand{\RosAuthor}[1]{\renewcommand{\@RosAuthor}{#1}}
\newcommand{\RosDocNr}[1]{\renewcommand{\@RosDocNr}{#1 (RWR-102-RO-90#1-RO)}}
\newcommand{\RosDate}[1]{\renewcommand{\@RosDate}{#1}}
\newcommand{\RosStatus}[1]{\renewcommand{\@RosStatus}{#1}}
\newcommand{\RosSupersedes}[1]{\renewcommand{\@RosSupersedes}{#1}}
\newcommand{\RosDistribution}[1]{\renewcommand{\@RosDistribution}{#1}}
\newcommand{\RosClearance}[1]{\renewcommand{\@RosClearance}{#1}}
\newcommand{\RosKeywords}[1]{\renewcommand{\@RosKeywords}{#1}}

