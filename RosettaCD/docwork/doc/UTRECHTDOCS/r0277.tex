
\documentstyle{Rosetta}
\begin{document}
   \RosTopic{General}
   \RosTitle{Notulen Rosetta vergadering 22-08-1988}
   \RosAuthor{Margreet Sanders}
   \RosDocNr{277}
   \RosDate{September 26, 1988}
   \RosStatus{approved}
   \RosSupersedes{-}
   \RosDistribution{Project}
   \RosClearance{Project}
   \RosKeywords{minutes}
   \MakeRosTitle



\begin{description}
\item[Aanwezig:] Lisette Appelo, Franciska de Jong, Jan Landsbergen,
                 Ren\'{e} Leermakers, Jan Odijk, Joep Rous, 
                 Margreet Sanders (not),
                 Andr\'{e} Schenk, Harm Smit
\item[Afwezig:] Elly van Munster
\item[Agenda:]\mbox{}
  \begin{enumerate}
  \item Opening en Notulen
  \item Mededelingen
  \item Bezoek JEIDA
  \item Computer configuratie
  \item Stand van zaken
  \item Rondvraag en Sluiting
  \end{enumerate}
\end{description}

\section{Opening en notulen}
Omdat een aantal mensen de notulen van 7 juli niet gehad schijnt te hebben, 
wordt goedkeuring ervan uitgesteld tot de volgende vergadering. Degenen die ze 
niet hebben, moeten er bij Fred om gaan vragen.

\section{Mededelingen}
\begin{enumerate}
  \item Woordenboekvulster: Petra de Wit zal pas per 1 oktober aangesteld 
worden, en wel als student-assistent op een 5/10 contract voor 1 jaar. Dat zij 
in de praktijk minder maanden werkt maar wel 10/10de, 
maakt niets uit voor het contract. Een 
aanstelling waarbij ze niet verplicht wordt om als student ingeschreven te 
staan lijkt niet mogelijk. 

Er wordt nog gewerkt aan een contract voor een tweede woordenboek-medewerker.
  \item Bespreking met WSF etc: Jan L.\ heeft op 8 juli vergaderd met Bosma, de 
Hoog, Berkhof en Fellinger over contacten van Rosetta met WSF en met TDS in het 
algemeen. De Hoog is nog steeds positief over natuurlijke taal software, maar 
dan wel in een algemeen kader van software-ontwikkeling, en niet als 
kostenplaats in het WSF-project. Zoals al 
eerder werd gemeld ziet hij het meeste toekomst voor interactief vertalen van 
elektronische post en voor multilinguale (standaard) tekst-generatie. 

Eind augustus of zo spoedig mogelijk daarna volgt een nieuw conclaaf van Carel 
en Jan L.\ namens Rosetta en van twee mensen uit Geldrop (waaronder iemand uit 
het ECHO-project die verstand heeft van de eisen van mensen in een 
kantooromgeving) om een concreet plan voor ontwikkeling van dergelijke 
software op te stellen. Dit zal dan weer door de vier oorspronkelijke heren 
besproken worden.
  \item Copy-cards van de R.U.U.: Jan L. wil weten hoe het systeem van 
copy-cards voor de linguisten is geregeld, om te kunnen voorkomen 
dat er elke keer kleine rekeningen naar Bosma gestuurd worden door de 
universiteit.
  \item LSP-symposium: Margreet vertelt heel kort de twee belangrijkste punten 
uit het symposium over Language for Special Purposes, waar zij 3 -- 5 augustus 
was: 
\begin{description}
\item
Een medewerker van Digital (Ron Verheijen, gepromoveerd taalkundige) komt 
op 5 september naar het Lab.\ om een demonstratie te krijgen van Rosetta, 
deels uit priv\'{e} interesse, deels zakelijk (DEC geeft subsidie aan 
software-ontwikkelaars die speciaal op (nieuwe) DEC-apparatuur werken). 
Mis\-schien kan dit bezoek onze blik op de arbeidsmarkt voor taalkundigen weer 
verruimen...
\item Mw.\ van Willigen van de nieuwe studierichting Vertaalwetenschap in 
Utrecht heeft gevraagd of Rosetta een van haar inleidende colleges voor 
tweedejaars studenten over vertaalsystemen wil komen opluisteren. Dit is in 
principe toegezegd.
\end{description}
\end{enumerate}


\section{Bezoek JEIDA}
Op woensdag 31 augustus van 9 tot 12 uur zullen twaalf Japanners van de 
Japanese Electronic Industrial Development Association (waarin ook alle 
organisaties die aan automatisch vertalen werken zijn vertegenwoordigd) een 
bezoek aan Rosetta brengen, helaas echter zonder de heer Nagao. Het programma 
dat zij in WB5 zullen volgen ziet er voorlopig als volgt uit: 
\begin{description}
\item 9.00 -- 9.30 welkomstwoord door Bosma
\item 9.30 -- 11.00 introduktie van Rosetta door Jan L.; \\
      speciale introduktie van de Rosetta-software door Ren\'{e};\\
      demo van Rosetta op aantal terminals met begeleidende uitleg
\item 11.00 -- 12.00 verhaal van de Japanners zelf
\end{description}
Jan L., Jan O.\ en Ren\'{e} zullen v\'{o}\'{o}r vrijdag 26 augustus een 
interessante voorbeeldzin bedenken. Aangezien er toch al veel 
Rosetta-medewerkers afwezig zullen zijn wegens vakantie en/of conferenties, 
worden de 
wel aanwezigen verzocht ook inderdaad acte de pr\'{e}sence te geven.


\section{Computer configuratie}
In een gesprek van Jan L.\ en Joep met Arnold Dissel en dhr.\ Joosen van de 
afdeling Informatieverwerking is besloten dat Rosetta per eind september bij 
wijze van proef zal overschakelen op de nieuwe VAX 8820, die het centrale 
NatLab cluster half september binnen hoopt te krijgen. Deze is veel krachtiger 
dan onze huidige 11/785. Conversie naar het nieuwe systeem (inclusief het 
overstappen naar de nieuwe release 5.0) zou maximaal drie dagen moeten duren.
Als de proef niet goed uitvalt, gaan we terug naar onze oude configuratie; als 
het wel bevalt (en wij de hoofdgebruiker van de machine zijn), kunnen we op 
de 8820 blijven en gaan onze huidige machines weg.


\section{Stand van zaken}
Sinds de vorige vergadering begin juli is er wat betreft {\bf software} met 
succes ge\-werkt aan een nieuwe woordenboek-compiler, en aan maatregelen voor 
semi-idiomen en id~iomen (zowel met S-trees als met derivatiebomen) en bonussen 
in allerlei soorten regels (dit laatste alleen nog syntactisch mogelijk; er kan 
nog niets mee gedaan worden). Er wordt nog gewerkt aan een goede omgeving voor 
de woordenboek-vulster en aan robuustheidsmaatregelen.

In de {\bf lingware} zijn regels geschreven en meestal ook getest voor idiomen 
en semi-idiomen in 
alle talen, voor PREPPs in het Engels en Spaans, voor ADVPPROPs (Ned.) en 
PREPPPROPs (Eng.), Spaanse 
NPs inclusief DETPs, Engelse transfer voor NPs, possessieve en partitieve 
NPs (Ned.\ en Sp.), er zijn Spaanse moodregels en er is een functie voor 
(Ned.)
polariteit geschreven. Jan O.\ zal deze laatste op de volgende 
projectvergadering nader toelichten. Verder is er hard gewerkt aan het vullen 
van het Engelse en Spaanse BVERB.dict, en aan het testen van geschreven regels. 
Voor alle duidelijkheid: er kan nu ook naar het Spaans vertaald worden!

Het programma dat uit het Van Dale woordenboek een Rosetta3 woordenboek maakt 
werkt ook: er is op deze manier een BNOUN.dict gemaakt. Er wordt opgemerkt dat 
er ook weer een halfjaarlijks verslag naar Van Dale moet. Overigens zijn de 
plannen van Van Dale/Eurotra en het NBBI (dhr.\ Akkermans) voor Spaanse 
woor\-denboeken nog steeds erg vaag; er lijkt vooralsnog weinig haast bij om al 
te gaan praten over het `lexicologisch concept' (zie notulen 30 juni).


\section{Rondvraag en sluiting}

{\bf Lisette} vraagt of we de gekraakte stagiaire-kamer al weer terug hebben. 
Jan L.\ meldt dat er een herverdeling van kamers komt, waarin voor Rosetta in 
1.5 stagiairekamer is voorzien, maar niet noodzakelijk dezelfde als die we 
eerst hadden.

{\bf Joep} herinnert aan de afspraak om het domein per 15 augustus te
 bevriezen. Aangezien het vullen van de woordenboeken toch iets wordt 
uitgesteld, is de nieuwe afspraak om het Nederlandse domein vanaf 1 oktober te 
bevriezen voor wat betreft de basiscategorie\"{e}n. Aan de typen van de 
betreffende attributen mag dan niets 
meer veranderd worden, en aan de attribuut-waarden eigenlijk ook niet.


\end{document}
