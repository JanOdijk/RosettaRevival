\documentstyle{Rosetta}
\begin{document}
   \RosTopic{General}
   \RosTitle{Notulen Groepsvergadering 26-3-1990}
   \RosAuthor{Petra de Wit}
   \RosDocNr{433}
   \RosDate{25-4-1990}
   \RosStatus{approved}
   \RosSupersedes{-}
   \RosDistribution{Project}
   \RosClearance{Project}
   \RosKeywords{minutes}
   \MakeRosTitle



\hyphenation{ant-woordt}
\hyphenation{ge-splitste}
\hyphenation{woorden-boek}
\begin{itemize}
  \item {\bf aanwezig}: Andr\'{e} Schenk, Jan Landsbergen, Lisette Appelo,
                     Franciska de Jong, Petra de Wit, Elly van Munster, 
                     Elena Pinillos, Joep Rous, Jan Odijk, Harm Smit,
                     Ren\'{e} Leermakers, Frank Uittenbogaard.

  \item {\bf afwezig}: Josien Willems, Harold Leurs 
  \item {\bf Agenda}:
    \begin{enumerate}
       \item Opening en notulen
       \item Reizen
       \item Voortgang CRE:
         \begin{enumerate} 
         \item Demo Brievenvertaler
         \item Conjugator
         \item Localros 
  	 \item Rosetta 3D
         \end{enumerate}
       \item Rondvraag
    \end{enumerate}
\end{itemize}

\section {Opening en notulen}
De notulen van de vorige vergadering zijn helaas nog niet vermenigvuldigd en 
zullen op de volgende groepsvergadering besproken worden.


\section {Reizen}

De volgende mensen gaan via Philips op reis:
\begin{itemize}
   \item ACL/Pittsburgh : Lisette
   \item Coling         : Jan O.
   \item ECAI           : Frank (indien het programma interessant voor ons is)
\end{itemize}
Er is dus nog \'{e}\'{e}n Philips-Europa reis over.\\

\noindent De volgende mensen gaan via de STT op reis:
\begin{itemize}
   \item ACL/Pittsburgh : Andr\'{e}
   \item Leuven         : Petra, Elena
\end{itemize}

\noindent Franciska vraagt of het ook in komende jaren mogelijk is om via de 
STT reizen aan te vragen. Jan L. antwoordt dat dit in principe kan, maar wil 
er liever geen gewoonte van maken.

\section {CRE}

\subsection {Demo Brievenvertaler}

De nieuwe demo is nagenoeg af ; enkele mogelijke aanpassingen zullen nog 
bekeken worden. Op de vraag van Harm of het mogelijk is nieuwe 
tekst toe te voegen, antwoordt Frank dat dit het geval is.

\subsection {Conjugator}

Lisette meldt dat er reeds een systeem draait op basis van Rosetta 3D. De 
Spaanse morfologie is ernstig vertraagd door met name integratieproblemen 
en problemen met de grootte van suffix.seg. De werkwoorden zijn inmiddels 
gevuld en dienen nog getest te worden.

\subsection {Localros} 
De software is nagenoeg klaar. De skeys en mkeys uit de grote VD dienen 
nog toegevoegd te worden. Ook moet er nog getest worden om eventuele 
fouten uit de aanmaakprogramma's en lexico bestanden op te sporen en te 
verbeteren.

\subsection {Rosetta 3D} 

\begin{itemize}
\item Jan O. meldt dat hij aan woordenboeken, samenstellingen, co\H{o}rdinatie en 
data gewerkt heeft. 
\item Ren\'{e} zoekt een week om alle s-regels te grabben voor optimalisering. 
Jan O. en Lisette merken op dat een eerdere uitdunning niet veel heeft 
opgeleverd. 
Besloten wordt dat Ren\'{e} op 2 april begint met grabben. Localros en Conjugator 
kunnen of de nieuwe versie van de s-parser kopi\H{e}ren of de oude compiler 
blijven gebruiken.
\item Jan L. meldt dat bij het testen 
van 70 woorden er 16 problemen gaven. Verder vraagt hij zich af hoe het met de 
[count,mass] problematiek staat. Franciska merkt op dat
er verschillende mogelijkheden zijn om dit op de lossen: splitsen van [mass,
count] entries in het woordenboek en/of een aanpassing van de regels in de 
grammatica. Ook kunnen bepaalde betekenissen in het woordenboek uitgezet worden.
Mocht er voor een grammatica oplossing gekozen worden, dan zal er toch enig 
woordenboekwerk verricht moeten worden. Een idee is om dit parallel in alle 
drie de talen te doen. Op korte termijn zal er een geschikte oplossing gekozen 
worden.
\item Andr\'{e} en Jan O. hebben uit het engelse verb woordenboek rare strings
 en 
nummertjes verwijderd of vervangen en de combinatie "be+adjectief" vervangen 
door het desbetreffende adjectief.
\item Harm meldt dat het programma voor de Engelse adjectieven bijna af is. Bij 
gesplitste lemma's moet het samenklappen nog voorkomen worden. Verschillende 
entries met dezelfde mkey geven problemen. Lisette vraagt of dit niet 
verbeterd kan worden. Harm antwoordt dat dit gezien de structuur van het 
programma zeker een maand zou kosten.
\item Jan O. meldt dat hij aan adjuncts heeft gewerkt, dat de testwoordenboek 
integratie nagenoeg af is en dat bijna alle crashes in het Engels en Spaans 
verholpen zijn. Lisette merkt op dat er nog een crash in de Engelse en Spaanse 
tijdregels zit.
\item Andr\'{e} vraagt of er iets gedaan kan worden aan de integratie batch. 
Er moet over het algemeen lang gewacht worden voordat een integratie aan de 
beurt is. Joep meldt dat er 
inmiddels voor onafhankelijke dictintegraties een aparte ROS\$DI\_batch is. Het 
commando hiervoor is dictintegrate. Misschien kunnen integraties versneld 
worden door CMS uit te zetten. Jan O. merkt op dat ook geheugenproblemen 
vertraging opleverden en dat met de nieuwe disk die vandaag ge\H{i}nstalleerd 
is integraties misschien sneller gaan. Besloten wordt het geheel nog even aan 
te zien. Joep merkt op dat domein-integraties het beste alleen voor het weekend 
opgestart kunnen worden.
\end{itemize}


\section {Rondvraag}

Frank Stoots heeft ons verzocht de code \underline{today} niet meer in \LaTeX
documenten te gebruiken.\\

\noindent Jan L.. meldt dat de volgende vergadering op \underline{donderdag 12 
april} om \underline{13.30 uur} plaats zal vinden 
\end{document}
