\documentstyle{Rosetta}
\begin{document}
   \RosTopic{Rosetta3.dictionaries}
   \RosTitle{Rosetta3 Dictionary keys}
   \RosAuthor{Harm Smit, Joep Rous}
   \RosDocNr{0279}
   \RosDate{\today}
   \RosStatus{concept}
   \RosSupersedes{-}
   \RosDistribution{Project}
   \RosClearance{Project}
   \RosKeywords{Dutch, dictionaries, keys, documentation}
   \MakeRosTitle

\section{Introduction}

This document describes the format of m-keys, s-keys and f-keys. Also,
limitations are given for the length of some parts of the lemmas of 
dictionaries in `testwoordenboek'-format.

\section{Definition of the format of keys}

Three types of keys can be used in the dictionaries now: m-key, s-key and f-key.
They all have to be made after the following format:

\begin{tabular}{lcl} 
        &    &   \\
KEYname & :: & keytype $[$ ''{\tt {\bf \_}}'' idtype $]$ ''{\tt {\bf \_}}'' 
               string ''{\tt {\bf \_}}'' category  $[$ ''{\tt {\bf \_}}'' 
               integer $]$   \\
keytype & :: & ''{\tt s}'' $\mid$ ''{\tt f}'' $\mid$ ''{\tt m}'' \\
idtype  & :: & ''{\tt id}'' $\mid$ ''{\tt sid}'' \\
string  & :: & stem    \\
category & :: & ''{\tt PREP}'' $\mid$ ''{\tt BPERSPRO}'' $\mid$ 
                ''{\tt BINDEFPRO}'' $\mid$  ... \\
integer & :: & 1..maxint \\  
        &    &   \\
\end{tabular}

The `keytype' denotes whether the key is a {\em f-key, s-key} or a {\em m-key}.
The two values for `idtype' refer to {\em idioms} and {\em semi-idioms}.
The `stem' is the stem of the actual word. `category' refers to the 
category.
`Integer' can be used for ambiguous words, like :

\begin{tabular}{ll}
   & {\tt s{\bf \_}ezel{\bf \_}BNOUN{\bf \_}1} \\
\end{tabular}

which refers to the animal, 
and:

\begin{tabular}{ll}
   & {\tt s{\bf \_}ezel{\bf \_}BNOUN{\bf \_}2} \\
\end{tabular}

which refers to a person
(note that these two words must have two different lemmas because they have
different values for the attribute `human').

Keys that are language-dependent, like s-keys and f-keys, can get the string 
and category as given in the specific language. For the language-independent 
m-keys, a more ``semantic'' string can be chosen, like:


\begin{tabular}{ll}
   & {\tt s{\bf \_}FirstPersSgMasc{\bf \_}BPERSPRO} \\
\end{tabular}

refering to the masculine first person singular.
Note that strings may contain characters like `\"{e}', `\c{c}', etc, and 
apostrophes, bars, points. 
Capitals should be avoided because the compiler is case-insensitive. Thus,
{\tt s{\bf \_}Job} and {\tt s{\bf \_}job} are equal. Spaces are not allowed.
In doubtful cases, consult Joep Rous.


\section{Automatically assigned keys}

For all open categories the Van Dale dictionaries are used. The program that
yields the Rosetta3-lemmas based on the entries of Van Dale will assign
keys automatically. In order to prevent problems with identical key-names,
all automatically generated keys will have an extra suffix in their format:
they will have category names like `{\tt aN}', `{\tt aV}' etc. Their
format looks like:

\begin{tabular}{lcl} 
        &    &   \\
KEYname & :: & keytype ''{\tt {\bf \_}a}'' category ''{\tt {\bf \_}}'' integer  
               ''{\tt {\bf \_}}'' string $[$ ''{\tt {\bf \_}}''  capital $]$\\
keytype & :: & ''{\tt s}'' $\mid$ ''{\tt f}'' $\mid$ ''{\tt m}'' \\
category & :: & {\tt N} $\mid$ {\tt V} $\mid$ {\tt A} $\mid$ {\tt AV} \\
integer & :: & digit digit $\mid$ digit digit digit digit \\
string  & :: & stem    \\
capital & :: & ''{\tt C}'' $\mid$ ''{\tt CC}'' \\
        &    &   \\
\end{tabular}

\begin{tabular}{ll}
Examples:  & {\tt s{\bf\_}aN{\bf\_}00{\bf\_}jaarwisseling} \\
           & {\tt s{\bf\_}aN{\bf\_}01{\bf\_}Job{\bf\_}C} \\
           & {\tt s{\bf\_}aN{\bf\_}01{\bf\_}JAC{\bf\_}CC} \\
           & {\tt m{\bf\_}aN{\bf\_}0101{\bf\_}jacht} \\
\end{tabular}

The `categories' are: N is noun, V is verb, A is adjective {\em and} adverbs (all
with Van Dale-codes in the range 20-29), AV is adverbs (in the range 50-59).

The m-keys have an `integer' of {\em four} digits. The last two digits 
correspond to the meaning number, the second digit corresponds 
to a {\em sub entry} (which are numbered by romans in the Van Dale N-E) 
and the first corresponds to {\em numbered
entry-words}. For most of the words, the first two digits will be {\em zero}, 
which indicates that there is no roman number nor a numbered headword in the 
Van Dale.

The `integer' consists of two digits for s-keys and f-keys, and in this case
they correspond to roman numbers or numbered headwords.

`Capital' says that a word begins with a capital (like: `Job') or that all
letters of a word are capitals (like: `JAC'). This is necesarry because
key-names are not case-sensitive.

\section{Files}

The m-keys are in the file {\em general:mkeydef.kdf}.
The f-keys and s-keys are in the files {\em dutch:skeydef.kdf} for Dutch,
{\em english:skeydef.kdf} for English, and {\em spanish:skeydef.kdf} for 
Spanish. Note that these keys are language dependent.

When the files with keys have been changed, the (relevant) dictionaries should
be evaluated.

\section{Limitations}

Some parts of the lemma are subject to limitations:

\begin{enumerate}
  \item The stem of a word has a maximum length of 40 characters.
 
        This is long enough to capture the longest word in the N-E, which is
        34 characters long.

  \item The meaning description has a maximum length of 50 characters.

  \item The s-keys, m-keys and f-keys have a maximum length of 50 characters.

        This is 10 characters longer than the stem. The extra length is 
        necessary for the pre- and suffixes, like `{\tt s{\bf\_}}'.
\end{enumerate}

\end{document}
