\documentstyle{Rosetta}
\begin{document}
   \RosTopic{General}
   \RosTitle{Derde halfjaarlijkse verslag aan Van Dale Lexicografie}
   \RosAuthor{H.E. Smit, J.P. Medema}
   \RosDocNr{234}
   \RosDate{October 19, 1987}
   \RosStatus{informal}
   \RosSupersedes{-}
   \RosDistribution{Project}
   \RosClearance{Project}
   \RosKeywords{dictionary, Van Dale report}
   \MakeRosTitle
%
%
\hyphenation{woord-ver-ta-ler woor-den-boe-ken zelf-stan-di-ge on-der-ver-de-
ling meer-vouds-vor-ming naam-woor-den at-tri-bu-ten}

\section{Inleiding}
In dit derde halfjaarlijkse verslag zal een overzicht gegeven worden van de 
stand van zaken betreffende het gebruik van de N-N, N-E en E-N bestanden.

Zoals in de volgende secties wordt aangegeven, blijkt de omzetting van de Van 
Dale bestanden in ROSETTA-woordenboeken nog problematischer te zijn dan was 
verwacht. Vrijwel alle syntactische en een deel van de morfologische informatie 
die voor de ROSETTA-woordenboeken nodig is, zal handmatig moeten worden 
ingevoerd (zie sectie 2). Verder is er een probleem met de de koppeling van de 
bestanden N-E en E-N (zie sectie 5). Het ziet er naar uit dat we voor ROSETTA3
met een vrij klein woordenboek zullen moeten volstaan.

\section{N-N}
In het afgelopen halfjaar is een intern document over de N-N geschreven. 
Dit document geeft een overzicht van de soort informatie die in het bestand
te vinden is en de manier waarop deze gestruktureerd is. 
Het bevat o.a. ook de informatie die in de vorige halfjaarlijkse verslagen 
te vinden was.

Verder is een voorstudie gemaakt naar de mogelijkheden die het bestand ons biedt
voor het automatisch vullen van attributen in de 
RO\-SET\-TA-woor\-den\-boe\-ken. Dit bleek tegen te vallen: waarschijnlijk 
kunnen slechts enkele morfologische attributen automatisch 
gevuld worden (zoals o.a. attributen voor meervoudsvorming van 
en grammaticaal geslacht van zelfstandige naamwoorden, 
vervoeging van werkwoorden, etc.). 
Daarbij moet opgemerkt worden dat het automatisch vullen ten gevolge van fouten
en inconsistenties nooit voor 100 \% zal lukken; er blijft altijd een 
restgroep over die handmatig gedaan zullen moeten worden.

Andere morfologische attributen, en vrijwel alle syntactische attributen 
zullen echter {\em geheel} handmatig gevuld moeten worden. 

Naast het programma dat de meervoudsvorming van zelfstandige naamwoorden uit
de N-N omzet in de juiste attribuutwaarde voor ROSETTA (zie vorige 
halfjaarlijkse verslag), is
een voorstudie gemaakt naar een programma dat op dezelfde manier de informatie
bij werkwoorden (m.b.t. de vervoeging) converteert. Het programma zelf moet nog 
geschreven worden.

Met het handmatig vullen van de woordenboeken is ook een begin gemaakt: er is 
een indeling gemaakt van de adverbia naar type, en er is gewerkt aan de 
verbpatterns van werkwoorden.

Ook in het laatste halfjaar zijn nog wat fouten en inconsistenties aan het licht
gekomen, zoals:
\begin{itemize}

   \item enkele woorden hebben een verkeerde categorie gekregen:
         zo heeft het werkwoord {\em verorberen} code `21', die voor 
         adjectieven gereserveerd is. 

         Het werkwoord {\em heupwiegen}
         heeft code `30' gekregen, wat zou betekenen dat het werkwoord niet
         vervoegd kan worden. Na deze code volgen echter de verleden tijd en
         het voltooid deelwoord van dit woord.

         Het werkwoord {\em scheuken} is vervoegd met `zich', maar heeft 
         de code `32' i.p.v. `33'.
         
   \item bij enkele woorden was het categorie-nummer in de romeinse 
         onderverdeling (GR\#-code) niet consistent met het nummer 
         achter de GI\#-code; dit was o.a. het geval bij: 
         {\em boze} en {\em averechts}.

   \item in sommige lemma's is de romeinse nummering incorrect; dit is het 
         geval bij {\em exerceren}, {\em gans}, {\em stijf}, {\em verhongeren},
         {\em beheersen}, en {\em rook}.

   \item sommige woorden komen niet voor in de N-N, terwijl dat eigenlijk
         wel het geval had moeten zijn (ze staan dan ook wel in de N-E):
         de zelfstandige naamwoorden {\em bel} en {\em bacterie}, en het 
         werkwoord {\em huishouden}.

   \item het woord {\em minstbedeelden} is fout gespeld; er staat:
         {\em minstbebedeelden}.

\end{itemize}

Verder zijn nog enkele tellingen verricht:
\begin{itemize}
  \item er is geteld hoe vaak een code in de {\em syntax} ``gepasseerd'' werd 
        als het hele bestand doorlopen werd. We weten nu bijvoorbeeld dat de 
        code TI\# 3528 keer voorkomt, waarvan in 3524 gevallen achter een 
        GI\#-code in een {\em normaal} lemma, en 4 keer achter een 
        GI\#-code in een {\em romeins} lemma.

  \item er is geteld hoeveel betekenissen er voorkomen in de lemma's, waarbij
        per aantal betekenissen het aantal lemma's gegeven is. Dit is gedaan
        voor de open categorie\"{e}n. Tevens is een uitdraai gemaakt van woorden
        die {\em nul} betekenissen hadden, met name met het idee dat het 
        hierbij ging om woorden die alleen deel uit maken van idiomen, zoals
        {\em lurven}. Dit laatste bleek wat tegen te vallen: er zaten ook veel
        woorden bij die o.i. {\em wel} een zelfstandige
        betekenis hadden, en daarnaast
        vonden we ook veel woorden {\em niet} (die dus ten onrechte een 
        zelfstandige betekenis hebben gekregen).

\end{itemize}

\section{N-E}

De {\em syntax} voor het lemma van de N-E is gemaakt en loopt goed; er zal
slechts een tiental lemma's veranderd moeten worden omdat deze niet aan de
syntax voldoen. Vier hiervan zijn zichtbaar fout, d.w.z. leveren merkwaardige
lemma's in de gedrukte van Dale. Het betreft:
\begin{itemize}
  \item {\em kiessysteem} - dit woord heeft in de N-E {\em geen} vertaling. Het
                        wordt echter meteen gevolgd door een tweede lemma met
                        {\em kiessysteem} als ingangswoord, dit keer {\em wel}
                        met vertaling.
  \item {\em kindertoeslag} - idem.
  \item {\em knieboog} - idem.
  \item {\em zetten} - bij dit woord staat voor de DI-code op de tape het 
                       karakter `\^{o}'.
\end{itemize}

Ook voor de N-E zijn tellingen verricht; zo is onder meer vastgesteld dat er
93502 lemma's zijn.

\section{E-N}

Er is gewerkt aan de syntax van het lemma voor de E-N, waarbij bleek dat er
relatief {\em veel} fouten en inconsistenties in het bestand zaten. Het
zal derhalve nog wel enige tijd duren voordat de syntax over het hele bestand
zal lopen.

Wel is op basis van de ingangswoorden van de E-N 
een retrograde woordenlijst voor het Engels gemaakt.

\section{Woordvertaler}

De eerste concrete toepassing van woordenboeken in ROSETTA zal de 
{\em woordvertaler} zijn; dit is een {\em mode} van ROSETTA waarin alleen losse 
woorden vertaald worden, maar wel in de juiste verbuiging; m.a.w. als
het Nederlandse woord {\em tafels} ingetypt wordt, vertaalt het systeem dit met
{\em tables}. Omdat de woordvertaler slechts weinig attributen nodig heeft,
en dit voornamelijk morfologische attributen betreft, kan een optimaal gebruik
van de woordenschat van de Van Dale bestanden 
worden gebruikt (bij het vertalen van 
{\em zinnen} zal dit niet zo zijn; daarvoor zijn alle ROSETTA
attributen nodig en omdat het merendeel daarvan met de hand gevuld 
moeten worden, zal slechts een beperkte hoeveelheid woorden gebruikt 
kunnen worden).

Een probleem bij het gebruik van de Van Dale bestanden voor de woordvertaler
is het feit dat de N-E en E-N bestanden eigenlijk {\em gemerged} zouden moeten 
worden. Dit is echter moeilijk om meerdere redenen: 
\begin{enumerate}
   \item voor beide bestanden geldt dat de gegeven {\em vertaling} vaak niet
         te vinden is in het andere bestand. Zo wordt het Nederlandse woord
         {\em jaarbalans} vertaald in het Engelse {\em annual balance sheet},
         wat als zodanig {\em niet} in de E-N te vinden is. Een
         (wellicht niet representatieve) steekproef leverde een verontrustend
         laag percentage woorden op die {\em wel} gevonden werden: 
         slechts circa 30 \%.
   \item verder staat de morfologische informatie voor Nederlandse woorden 
         alleen in de N-N, en die voor het Engelse woorden alleen (en slechts 
         summier) in de E-N. Dit heeft tot gevolg dat we de Nederlandse 
         vertalingen van de Engelse woorden uit de E-N {\em niet direkt} aan het
         N-N bestand kunnen toevoegen (dit is zou eventueel wel mogelijk zijn 
         als het alleen samenstellingen betrof, waarvan het laatste lid
         {\em wel} in de N-N zou staan en we ook met zekerheid zouden kunnen 
         zeggen wat het laatste lid was).
\end{enumerate}

De N-N en N-E bestanden komen qua ingangswoorden gelukkig wel in hoge mate 
overeen, alhoewel het aantal betekenissen bij sommige woorden verschilt in 
beide bestanden. Omdat de morfologische attributen meestal per lemma (en
niet per betekenis) gespecificeerd zijn, kunnen 
we de N-E als basis voor de woordvertaler gebruiken, en
de morfologische attributen voor het Nederlands uit de N-N halen.
De morfologische attributen voor het Engels moeten dan wel grotendeels {\em 
default} gevuld worden (dit lijkt voor het Engels duidelijk minder problemen 
op te leveren dan voor het Nederlands: zo hebben bijvoorbeeld de Engelse 
zelfstandige naamwoorden een {\em default} waarde voor de meervoudsvorming, 
nl. {\em \mbox{-s}}, dit in tegenstelling tot de Nederlandse 
zelfstandige naamwoorden, die zowel {\em \mbox{-en}} als {\em \mbox{-s}}
hebben). 
Daarbij zal de morfologische informatie uit de E-N met de hand toegevoegd
worden, waarbij natuurlijk nooit met 100 \% gegarandeerd is dat daarmee 
{\em alle} uitzonderingen in het bestand van Engelse vertalingen ``gedekt''
zijn. 

Het zal duidelijk zijn dat het gebruik van alleen de N-E als basis voor een 
tweetalig woordenboek voor de woordvertaler
eigenlijk wel wat mager is, maar door de te beperkte 
koppeling tussen de N-E en E-N en het gebrek aan mankracht
zal het helaas niet beter kunnen.

\section{Database}

Er is in het afgelopen halfjaar ook gewerkt aan een {\em database} voor
de ROSETTA-woordenboeken.
De database is voorlopig gebaseerd op de N-E (toegespitst op gebruik als 
woordenboek voor de woordvertaler). In een later stadium zal de database 
uitgebreid worden voor een algemener gebruik binnen ROSETTA.


\end{document}
