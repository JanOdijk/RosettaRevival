\documentstyle{Rosetta}

\begin{document}

%Frontpage
   \RosTopic{Rosetta3.Lexicon}
   \RosTitle{Retrograde vocabulary Van Dale N~--~N}
   \RosAuthor{Jeroen Medema, Harm Smit}
   \RosDocNr{173}
   \RosDate{87/02/16}
   \RosStatus{Informal}
   \RosSupersedes{-}
   \RosDistribution{Harm Smit, Jeroen Medema}
   \RosClearance{Project}
   \RosKeywords{Dictionary, Retrograde vocabulary, N~--~N}

   \MakeRosTitle

%Text
\section{Introduction}
 The following pages contain the vocabulary of the N~--~N (= Dutch -- Dutch) 
 dictionary of 
 Van Dale (\cite{st:groot}). The ordering of the words is retrograde (which 
 means that they are not ordered with respect to the first characters of a 
 word -as in normal dictionaries- but with respect to the last ones). 
 This list contains all the main entries as given in the 
 N~--~N. It consist of exactly 86550 entries (count it yourself).\\ \\
 It was made to be used by linguists and dictionary programmers when a set 
 of words with a special ending is needed (especially for the morphological 
 part). The project already had disposal of a retrograde wordlist of the 
 `Grote Van Dale' (\cite{ni:retro}), but needed this list because it 
 contains exactly the same words as the N~--~N dictionary 
 (which will be the base of the dictionaries in the Rosetta system). 
\section{Vocabulary order}
 The following constraints apply to the ordering:
 \begin{enumerate}
  \item characters with an accent (as e.g. \^{e}, \~{n}, \aa, \"{o}) are 
  considered to be equal to the same character without the accent.
  \item words containing the string ``(-)'' are considered to be equal to the 
  same word without that string.
  \item words containing {\em other} characters than letters, ``-'', ``~'', and 
  ``.'' are considered to be equal to the same word without those characters.
  \item the ordering is done by smallest first; that is: first ``~'', then 
  ``-'', next ``.'', followed by the letters.
  \item the upper and lower case form of a letter are considered to be equal.
 \end {enumerate}
 The words are split up in twenty-eight parts: for every letter one part and 
 two additional for the ``-'' and ``.'' (no word ends with a space). Each part 
 starts at a new page.
\section{Realisation}
 The proces of creating the retrograde vocabulary was divided in several steps:
 \begin{enumerate}
 \item 26 files are created that contain entries (per letter) without all the 
  information (like meanings and example sentences) which was stored in the 
  original files (see \cite{sm:descr}); this file contains one headword per 
  line.
 \item 26 new files were created; in these files each headword was preceded by 
  its reverse form (which, however, was without any accent or character other
  than letters, ``-'', ``~'', and ``.''.
 \item these files were merged and, consequently, sorted by using the system 
  routine sort; this could be done because every line started 
  with the reversed headword and didn't contain any special characters.
 \item a new file was created which no longer contained the reversed words;
  thus, the result was a file containg the same information as the twenty-six
  mentioned in 1; however, the words were now placed in retrograde order.
 \item from this file the printfile was created with the following constraints:
  \begin{itemize}
   \item every page should consist of four columns and sixty rows.
   \item if a word has a length greater than 19 (which is the pagewidth -80-
    divided by the 
    number of columns minus 1 for the separation of the columns) it has to be 
    split into two or more consecutive (that is, in one column) printed strings.
   \item if a word (because of its length) does not fit anymore in the column
    a new column is to be used (the same applies for pages).
   \item the order of reading is column first, row second, page last.
   \item the headwords (as stored in the file) are converted from a coded form 
    to printable strings more or less as in \cite{st:groot} (for coding details 
    see \cite{sm:descr}).
  \end{itemize}
 \end{enumerate}
\begin{thebibliography}{AAA99a}
 \bibitem[SME87]{sm:descr} Harm Smit \& Jeroen Medema, {\em Description Van 
  Dale dictionary N~--~N}, {\bf Rosetta report 174}, Philips Research labs, 
  1987
 \bibitem[NIE69]{ni:retro} E.R. Nieuwborg, {\em Retrograde woordenboek van de 
  Nederlandse taal}, Universit\'{e} Catholique de Louvain, 1969
 \bibitem[STE84]{st:groot} P.G.J. van Sterkenburg, ed. {\em Groot woordenboek 
  hedendaags Nederlands}, Van Dale Lexicografie, 1984
\end{thebibliography}
\end{document}
