\documentstyle{Rosetta}
 
\begin{document}
   \RosTopic{General}
   \RosTitle{Using Van Dale For Rosetta}
   \RosAuthor{J.P. Medema, H.E. Smit}
   \RosDocNr{R0247}
   \RosDate{April 11, 1988}
   \RosStatus{Informal}
   \RosSupersedes{-}
   \RosDistribution{Project}
   \RosClearance{Philips}
   \RosKeywords{Mapping, Van Dale, Rosetta Dictionaries}
   \MakeRosTitle
%
%
\setlength{\parskip}{2mm}
\setlength{\parindent}{0mm}
\hyphenation{Engels Ne-der-lands}
 
\section{Introduction}
 
\subsection{General}
 
The dictionaries of Rosetta will be developed using the Van Dale dictionaries 
as much as possible. Therefore, in this document we present 
possibilities and problems of using
these files for this purpose. 
 
The report is divided into four chapters of which the introduction is one.
In the introduction --besides a general overview-- also a `solution' of an 
earlier mentioned problem with dictionaries in Rosetta3 is given.
 
Chapter two describes the relation between the several Van Dale dictionaries
in relationship with the use of them creating the Rosetta dictionaries. It also
describes several programs made (or to be made) for formalization of the Van 
Dale dictionaries.
 
In chapter three a more or less detailed description is given of how to convert 
the Van Dale dictionaries to Rosetta dictionaries (as far as possible --if
possible at all). A formal description of the Rosetta dictionaries is given and 
the constructs used in the Van Dale dictionaries are explained.
 
The attributes of the Rosetta dictionaries and a proposal to fill them on the 
basis of the Van Dale dictionaries is given in the last chapter, chapter four. 
As much as possible is described how to fill the Dutch attributes (whether 
automatically or not).
 
\subsection{The Conversion Problem}
 
The conversion problem (as described in \cite{ro:conv}) is (in short) the 
following:
the structure of the Rosetta dictionaries cannot change without making 
conversion programs {\em by hand}. This is an effort which will return every 
time the structure changes. This change occurs too often to do it by hand.
 
It (the problem) is thought to be solved using a commercially available 
database management 
system instead of a self-implemented database. This approach has several 
advantages of which the next is the most important. Most of those 
database systems do have a program which can convert tables (automatically)
from one format into another. This solves the problem in the aforementioned 
report. 
 
Furthermore, there are DBMSs which can check several 
database requirements. These requirements can be of a tuple, intertuple, and 
intertable level. This means that several check programs (for checking the 
requirements) are not to be written but only to be specified.
 
\newpage
\section{The Use Of Van Dale Dictionaries}
 
\subsection{In General}
 
A detailed description of the N-N dictionary of Van Dale can be found in 
document \cite{sm:descr}. The other dictionaries have not been described in a
document, but have been investigated too. For the N-E, a {\em syntax} has 
been written.
 
The N-N gives (of course) a vocabulary that fits for Dutch. The N-E has been 
based 
mainly on the same vocabulary; both files share the majority of their entry 
words. The E-N has been based on an English vocabulary. This means that 
the E-N is
{\em not} the reverse of the N-E. So, the N-E gives English words that fit as 
translation for the set of words that are characteristic for the Dutch language
(and for the Dutch culture, way of living etc.), which does not imply that 
these words form a good vocabulary for English. The same holds for the 
translations in the E-N: these words will not form a good vocabulary for Dutch
(Examples: words like ``tea-things'', ``tea-shop'' and ``tea-lead'' can be found
in the E-N, but their translations, resp. ``theeboel'', ``theewinkel'' and 
``theelood'', which are correct Dutch words, cannot be found in the N-E or 
N-N. It seems that only 40 \% of the translations\footnote{This 
is a very rough estimate, however.} in the E-N can be found 
in the N-E ).
 
The ideal situation would be to {\em merge} both files, but this is extremely 
difficult, because all `new' Dutch words should have the right attributes, 
like {\em article} (for nouns) and 
{\em conjugation} (for verbs). Also, one should try to avoid that 
certain words occur in the dictionary {\em twice}.\footnote{Say, we have the 
word 
{\em bank} in the Dutch vocabulary --from the cross-section of N-N and N-E--, 
and get the same string from the E-N as translation. How do we know 
whether or not this word
is a `new' meaning? And if not, to which of the meanings of ``bank'' that we 
already had does it belong? 
Of course, evaluating several hundreds of thousands translations
`by hand' in such a way is {\em not} possible!}
 
Another possibility is to use {\em two separate} dictionaries: the N-E for 
translating Dutch into English and the E-N for translating English into Dutch
(which implies that there is no real IL at dictionary-level). Now, there is 
no merging problem, but there are still problems to give all attributes of the
translations their values. For Dutch this will be impossible: some Dutch 
attributes have {\em several} frequent values. 
The attribute {\em pluralforms} for nouns for instance has two frequent
values: {\em en-plural} and {\em s-plural}, which do not correspond to
certain stem-endings and thus cannot be filled automatically. 
Therefore, all Dutch nouns that are given as translations 
in the E-N should be evaluated in order to get a proper filling.
In the limited time of Rosetta3, this will not be possible.
 
A {\em weaker} solution would be to take only {\em one} 
of the bilingual dictionaries,
which will result in an unnatural vocabulary for one of the languages. We
do not have the problem of merging the original 90.000 words with the 
translations (although there might be some overlap among the translations 
themselves), but we still have to assign the right values to the attributes 
of the words that are give as translation in one of the two languages. 
 
Fortunately, English is (compared to Dutch) very regular
in the attributes that are given in the Van Dale dictionaries (like 
{\em pluralform, conjugation, article,} etc.).
Therefore it seems possible to use the N-E as base for the Rosetta 
dictionaries.
The English translations will form the English dictionary (which, of course,
will be not as good an English vocabulary as the words of the E-N), and the 
(morphological) attributes can be assigned automatically, which means that the 
`new' English vocabulary will have at least the same amount of attributes 
as the E-N.
The E-N will also serve as source of information, because we can retrieve 
information about irregular morphological forms from it.
The N-N contains information about various attributes, like: 
{\em article, pluralform,} etc. of nouns, 
{\em conjugationclass, particles, reflexives,} etc. of verbs, etc.
Because the N-N and N-E share most of the words, this information can be 
transferred to the N-E (probably, not the N-E itself but the intersection 
of N-E and N-N will be the base for the Rosetta dictionary).
 
In Rosetta, dictionaries should contain very detailed information which is not
in all cases present in the Van Dale dictionaries. Information that cannot be 
found in the Van Dale dictionaries is for instance: 
information about verbpatterns, nounpatterns, the type of adverbs, etc.
Of course this information should be added to the dictionaries, which has to
be done by hand. It will be clear that this cannot be done for the complete
N-E vocabulary with the limited manpower in our project.
A complete overview of what has to be done for each attribute of nouns, 
verbs, adjectives and adverbs can be found in chapter four.
 
\subsection{Mapping N-N And N-E}
The two Van Dale dictionaries used for the Dutch Rosetta dictionaries are 
of course not
equal in content. This holds for some parts of the entries (like
translation --~in N-E~--, morphological information --~in N-N) but, 
unfortunately, 
this is also the case at another level. These differences occur at varying 
places:
 
\begin{description}
  \item [{\bf Different Sets Of Entries}]\hspace{1cm}\\ 
        The set of entries of the two mentioned dictionaries are not the same.
        Even worse, N-E $\setminus$ ( N-E $\cap$ N-N ) $\neq \emptyset$ 
        {\em and}   N-N $\setminus$ ( N-N $\cap$ N-E )
        $\neq \emptyset$. This means that only the words in the overlap can be 
        processed automatically.\footnote{This is the case because words 
        in the N-E which aren't present in the N-N don't have morphological 
        information, while words in the N-N which aren't present in the N-E 
        don't have a translation.}
  \item [{\bf Joined Entries}]\hspace{1cm}\\
        Some words (like ``jus'') had several entries in the N-N because of the
        different pronunciation of the various meanings. This difference is
        not present in the N-E where the two words form one entry. What has to 
        be done with these words is either separate them in the N-E or join
        them in the N-N (of which the latter seems to be the easiest).
  \item [{\bf Different Roman Subdivision}]\hspace{1cm}\\
        An entry can have a roman subdivision in the N-E that differs from the 
        one in the N-N. There are words (like ``ka'') which have a subdivision 
        in the N-N while not in the N-E. The opposite also occurs (see 
        ``kakken''). These words can be used --~as given in the N-E~-- if the 
        morphological information is specified at the top of the entry (behind 
        the first G-code) and not in a roman subdivision {\em and} if the word 
        has at 
        least one translation.\footnote{Which implies that not every roman 
        subdivision has translations.}
  \item [{\bf Different Meaning Subdivision}]\hspace{1cm}\\
        An entry can have a meaning subdivision in the N-E that differs from 
        the one in the N-N. There are words (like ``kolokwint'') which have more
        subdivisions in the N-N than in the N-E. The opposite also occurs (cf. 
        ``kogel''). These words can be used (as given in the N-E) if the 
        morphological information specified at the top of the entry (behind 
        the GI-code) doesn't affect one of the meanings (like it does in ``in 
        bet. 3 g.mv.'' -- in meaning 3 no plural) {\em and} if it has at least 
        one translation.
\end{description}
 
All these `inconsistencies' have as result that the total amount of words will
reduce. At the time this report was made, it wasn't known yet how many words
would be `equal' (that is, usable). However, a simple check program showed that 
over 20\% of the words (of the letters ``p'', ``s'', ``v'') were different in 
one of the ways mentioned above. These differences were mainly due to the 
different sets of entries.
 
\subsection{Formalizing Dictionaries}
 
In this section the programs that will help filling the morphological 
attributes are described. These attributes are the only ones which can be 
filled automatically. Therefore, two kinds of programs are made:
 
\begin{itemize}
  \item Programs that {\em restructure} the information of the Van Dale; these
        programs rewrite the information in codes that can easily be mapped 
        on attributes of the Rosetta dictionaries. These programs also
        {\em check} the information of the Van Dale as far as possible.
        Their names start with ``Set'' because they set certain attributes.
  \item Programs that check on {\em comment}; these programs are used for
        categories that do not have information that can be restructured.
        In fact these programs yield lists of entry-word that are irregular
        in some respect. Their names start with ``Get'' because they retrieve
        information.
\end{itemize}
 
Both kinds of programs will be explained in more detail in the next sections.
 
\subsubsection{Restructuring}
 
Restructuring of information is only useful when the majority of the words
of a category have this information. This is the case with information
about the pluralforms of nouns and the conjugation of verbs in the N-N.
Other information, like the kind of article that a noun can have, is already
coded in the N-N and therefore does not need to be restructured.
 
The following programs rewrite morphological information of N-N
in terms of (codes that map directly on) Rosetta attributes:
 
\begin{itemize}
  \item {\bf SetPlural} - This program rewrites the pluralforms of 
        nouns 
        in the corresponding values of the Rosetta attribute {\em pluralforms}.
        In some cases, pluralforms
        hold for only one of the meanings of the entry-word; this is also 
        detected by the program. 
  \item {\bf SetConj} - This program rewrites the information of verbs: it 
        yields the stem of a verb, and fills the attributes {\em conjclasses, 
        particle, reflexivity, perfauxs}. 
\end{itemize}
 
Both programs also yield a list of entry-words with {\em faults};
these lists contains also (grammatical) {\em comment}
which can be useful for the filling of other attributes (SetPlural
for instance, yields a list that contains many --often irregular--
diminutiveforms).\footnote{With respect to 
this kind of information, the programs can be compared with those of the 
type that check on {\em comment}.}
 
Adjectives and adverbs lack systematic information about inflection; for these 
categories a program will be written that checks on comment only.
 
\subsubsection{Comment}
 
For Dutch, the adjectives and adverbs of the N-N are checked; verbs and 
nouns do not need extra check-programs because lists of words with all kinds of 
comment are yielded --as side-effect-- by the programs that restructure the 
information (see 2.3.1).
 
For English, all open categories of the E-N are checked on grammatical 
comment only. Checks are made on G-codes (all G-codes that are followed by 
{\em more} information than merely a category-number) and on the appearance of 
CI-, CR- and CV-codes. Of course, the resulting lists contain {\em raw} 
material and need to be adapted.
 
\subsubsection{Results}
 
The original information of Van Dale has been split up in files per category
(see \cite{sm:descr}). 
The restructuring-programs of the last section take category-files as input
and yield new files. In principle, this results in 
{\em two} files: one with the new lemma's, and one with
a list of entry-words that couldn't be restructured because of errors, missing
information, etc. 
It seems best to rectify these errors in (the copy of) the original files,
so that these lemma's can be handled by all (following) programs. By doing 
this, entry-words that have a wrong category-number (like ``aanmoedigen'', 
``boze'', ``doling'', etc.) will come in the right category-file. This 
feedback-mechanism will --after repeated application-- result in files 
without errors.
 
Thus, we get the following {\em stages} in our files:
\begin{enumerate}
  \item the copies of the original files (and files with rectifications,
        based on the error-files under 4),
  \item the rectified versions of (the copies of) the original files,
  \item the files per category
  \item the `restructured' category files (with separate file with errors)
\end{enumerate}
 
and the following programs:
\begin{itemize}
  \item {\em Rectify}. Input: original files, files with rectifications 
        (derived from error-files under 4); output: files under 2.
  \item {\em Categories}. Input: files under 2; output: files under 3.
  \item {\em Set....} or {\em Get....}. Input: files under 3; output: 
        files under 4 and error-files (which may lead to new 
        rectification-files).
\end{itemize}
 
Of course, errors may appear at other levels than those mentioned. This can 
also lead to adaptations in the rectification-files.
 
\newpage
\section{From Van Dale To Rosetta Dictionaries}
 
This section deals with the conversion of the Van Dale files to Rosetta 
dictionaries. 
It does not treat the conversion in detail (i.e., of the attributes; see 
section 2.3) but it shows how the global constructs used in the Van Dale 
dictionaries will 
be mapped on the Rosetta dictionary structure. First, a detailed structure 
definition is given of the Rosetta dictionaries. Next, the constructs used in 
the Van Dale dictionaries are treated. Finally, the mapping between both
dictionaries is discussed, which is illustrated by various examples.
 
\subsection{The Rosetta Dictionaries}
 
In this section the Rosetta dictionaries are described. Of 
course, more detailed information can be found in other reports 
(like \cite{la:r2},
\cite{fe:r2}, \cite{la:r3},and \cite{os:r3}). 
 
At the moment, the Rosetta translation system is being developed for two 
different functions: for the translation of {\em phrases}
(i.e. sentences, or parts of sentences), and for the translation of
{\em (single) words}. For the latter, not all attributes that are
specified for the Rosetta dictionaries are needed.
The information that can be found in the N-E or N-N is mainly 
{\em morphologically}, and 
corresponds to a high extent with what we need for the translation of words.
Thus, we can use the N-E (with the aforementioned additions) without much extra
work for the translation of words. For this purpose, all {\em syntactic} 
attributes can be filled with default values, because they will never be
evaluated.
 
To derive a Rosetta dictionary that is filled good enough for the translation 
of words from the Van Dale files, the following activities are required:
\begin{itemize}
  \item A database structure is needed. This has been designed and made 
        (see below). 
  \item We need a program to fill the database with the information of 
        N-E and N-N.
  \item Some (large) programs are needed to restructure the information
        in the N-N for Dutch verbs and Dutch nouns. 
        Other (smaller) programs 
        are needed for Dutch adjectives and adverbs and English nouns, 
        verbs and adjectives; in all these cases the N-N and E-N resp. contain
        information for a limited group of entry-words, namely the ones that 
        are irregular. All programs are described in section 2.3.
\end{itemize}
 
Of course the same activities are required as a first step towards the 
ultimate Rosetta dictionaries.
 
The Van Dale files will {\em not} be used to fill the closed categories of the
Rosetta dictionaries because the words of these categories are not filled in a 
very consistent way by Van Dale (see, for instance, \cite{od:cert}).
 
The Rosetta dictionaries are `monolingual' dictionaries; this means that they
should describe only the meaning of words (in IL). During the translation of 
words, a dictionary is only used in the first half of a translation. The other 
half is done by means of a dictionary (of the target language) that is 
organized in the same way but which is used the other way around to obtain the
word(s) in the target language.
 
Within the Rosetta dictionaries there is a difference between the theoretical 
model\footnote{This one is simple to understand.} and the practical 
model.\footnote{This one yields a quicker program.} These two models are 
described in the following two sections.
After that, some possible improvements on the Rosetta dictionaries are given.
In the last sub-section the completeness of the dictionaries is discussed.
 
\subsubsection{In Theory}
 
The Rosetta dictionaries consist of four dictionaries (relevant here) per 
language (at the moment, Rosetta is developped for three languages). 
The dictionaries are structured in tables which are defined as follows:
 
{\bf Database Definition}
\begin{verbatim}
Rosetta Dict = (  M-Dict  = {( stem   : String,
                               fkey   : Natural,
                               Phon   : PhonType,
                               rest   : RestType1 )} ,
      
                  S-Dict  = {( fkey   : Natural,
                               skey   : Natural,
                               rest   : RestType2 )} ,
      
                  B-Lex   = {( skey   : Natural,
                               cat    : SynCatInfo )} ,
          
                  Il-Dict = {( skey   : Natural,
                               desamb : String,
                               mkey   : Natural,
                               rest   : RestType3)}
               )
\end{verbatim}
{\bf Database Keys}
\begin{verbatim}      
      Key (M-Dict)  = {(stem,fkey,phon)}
      Key (S-Dict)  = {(skey)}
      Key (B-Lex)   = {(skey)}
      Key (Il-Dict) = {(skey,mkey),(skey,desamb)}
\end{verbatim}
\newpage
{\bf Entity Relationship Diagram}
\begin{verbatim}      
                       MDict
                         |n 
                         |
                         |
                         |                   m,n IN {1,2,3,...}
                         |
                         |
                         |m
         IlDict-n----1-SDict-1----1-BLex
\end{verbatim}      
 
There is a relation between the keys of the various tables which is depicted by 
their names. The dictionaries all have a distinct purpose. M-Dict is used for 
conversion from string to number during the segmentation phase. That 
number is used in the next phases. Here (in the segmentation phase), the 
phonetic information is used. In S-Dict the splitting of the various verbs 
(including 
the particle) is arranged. B-Lex contains the syntactic and morphological 
category information for every S-Dict entry. In Il-Dict each of the meanings of
an S-Dict entry has a separate entry. 
 
The key-to-key mapping within one table is described in the following table 
(where an arrow denotes the direction of the mapping and an asterix 
that there can be a mapping on more than one key --which, in this case, means 
that more ambiguities can be a result): 
 
\begin{center}
 \begin{tabular}{|l||c|c|c|}\hline
               & MDict                      
                       & SDict                          
                                & IlDict                               \\ \hline
 In analysis   & str $\rightarrow$ fkey     
                       & fkey $\rightarrow$ skey$^{\bf \ast}$ 
                                & skey $\rightarrow$ mkey$^{\bf \ast}$ \\
 In generation & str$^{\bf \ast} \leftarrow$ fkey 
                       & fkey $\leftarrow$ skey  
                                & skey$^{\bf \ast} \leftarrow$ mkey    \\ \hline
\end{tabular}
\end{center}
 
\subsubsection{In Practice}
 
The difference between the theoretical and the practical model is the use 
of the technique of entry reduction by means of table skipping. This is, in 
short, the skipping of intermediate tables when possible. This is 
possible when 
the intermediate tables have a 1-to-1 mapping of certain keys. The technique has
as a result that the total number of entries reduces. It has two main 
advantages: 
the searches on the database are quicker and the storage of it is smaller. It 
has, however, also impact on the database constraints, the programs using the 
database and the key value ranges:
 
\begin{itemize}
   \item There are more (and other) database constraints needed to handle more 
         relations and to guarantee the completeness of the database.
   \item Programs will become more elaborate, because of the `exception' 
         handling of not finding a key. 
   \item Key value ranges must be disjunct for all relevant tables.
\end{itemize}
      
In the Rosetta dictionaries, the technique is used for words which do not
need a division in S-Dict. For these words the fkey is used in both B-Lex and 
Il-Dict, instead of an skey. The key values of skeys and fkeys must be 
disconjunct, otherwise searching in B-Lex or Il-Dict can give wrong results. 
Searching for fkeys in S-Dict which are unsuccessful, imply (using the database 
constraints) the use of the fkey value as skey value in B-Lex and Il-Dict. 
The current programs are built using this technique.
 
The two models can, happily enough, be mapped on each other. As can be seen in 
the following examples, words without a division in S-Dict are present there in 
the first model while they are missing in the second. For each of these missing 
words the value of their fkey is used.
 
\begin{verbatim}
           M-Dict           S-Dict                 B-Lex
      
       I   "x" -> 12        12 -> 26               12 -> ....
           "y" -> 26        26 -> 12               26 -> ....
                            26 -> 27               27 -> ....
      
      II   "x" -> 12                               12     -> ....
           "y" -> 26        26 -> 100012           100012 -> ....
                            26 -> 100027           100027 -> ....
\end{verbatim}
 
The database definition of the practical model is the same as in the theoretical
model except 
for two name changes: `skey' is renamed to `key' in both B-Lex and Il-Dict. This
also applies for the keys of the same tables. The entity relationship diagram,
however, has changed radically (because of the S-Dict skipping). 
 
\begin{verbatim}
                       MDict
                      /n |n \n
                     /   |   \
                    /    |    \
                   /     |     \             m,n IN {0,1,2,...}
                  /      |      \            p   IN {0,1}
                 /       |       \
                /        |        \
               /         |         \
              /          |          \
             /m          |m          \p
         IlDict-n----p-SDict-p----p-BLex
\end{verbatim}
 
Furthermore, there are some extra requirements (because m,n, and p can 
be zero)
to maintain the completeness of the database ($\oplus$ stands for an exclusive 
or):
 
\begin{itemize}
  \item The entryword is either present in S-Dict or in both B-Lex and Il-Dict:
 
        $ \forall_{x \in {\rm MDict}} [
             \exists_{y \in {\rm SDict}} [ x.{\rm fkey} = y.{\rm fkey} ] \oplus
             \exists_{y \in {\rm BLex}, z \in {\rm IlDict}}
                    [ x.{\rm fkey} = y.{\rm key} = z.{\rm key} ] ]
        $
  \item An entry in S-Dict (sub-word) has both a translation and syntactic  
        information:
 
        $ \forall_{x \in {\rm SDict}} [
             \exists_{y \in {\rm IlDict}} [ x.{\rm skey} = y.{\rm key} ] \wedge
             \exists_{y \in {\rm BLex}}  [ x.{\rm skey} = y.{\rm key} ] ]
        $
  \item Translations and syntactic information are either for a word or for a 
        sub-word:
 
        $ \forall_{x \in {\rm BLex} \vee x \in {\rm TrLex}} [
             \exists_{y \in {\rm MDict}} [ x.{\rm key} = y.{\rm fkey} ] \oplus
             \exists_{y \in {\rm SDict}} [ x.{\rm key} = y.{\rm skey} ] ]
        $\\
\end{itemize}
 
\subsubsection{Future Improvements}
 
In this section some ideas about the structure of the Rosetta dictionaries 
are dealt with. These ideas were found during the investigation for and the 
writing
of this report. The first idea deals with the deletion of B-Lex while the 
second 
gives a way to reduce the size of the new S-Dict. Both changes have impact on 
the formalism as well.
 
\begin{description}
  \item [{\bf No B-Lex}]\hspace{1cm}\\
        Because of the fact that B-Lex and S-Dict have the same key (in the
        theory model) these tables can be combined. This has several advantages:
 
        \begin{itemize}
            \item The theoretical and the practical model will be one again.
                  This means that the database requirements will be far more 
                  simple.
            \item The extra search needed in the practice model is not needed 
                  any more because every fkey can be found in S-Dict. This
                  means that programs will be quicker.
            \item There are less tables. Because of that, it will be easier to
                  write programs, maintain the tables, etc.
        \end{itemize}
 
  \item [{\bf Reduce Size S-Dict}]\hspace{1cm}\\
        If the previous suggestion is followed, then the size of S-Dict will
        be less than the size of S-Dict and B-Lex before the change. It is,
        however, possible to reduce the total size even more by introducing
        a new table which contains the different syntactic records (which, in
        earlier days, were all -with duplications- in B-Lex). 
 
        This table, RecLex, will then hold no
        equal syntactic category information records. This will reduce the 
        size considerably.\footnote{This is the case because the size of 
        these records is  
        quite large and certain records are rather frequent.} The programs 
        will, in contrast to the suggested model, perhaps be slower because 
        of one more database access (the one in RecLex) which, however, will be 
        (partly) compensated by quicker searches because of the reduced size.
 
\newpage
{\bf Database Definition}
\begin{verbatim}
     Rosetta Dicts = (  M-Dict  = {( stem   : String,
                                     fkey   : Natural,
                                     phon   : PhonType,
                                     rest   : RestType1 )} ,
           
                        S-Dict  = {( fkey   : Natural,
                                     skey   : Natural,
                                     rkey   : Natural,
                                     rest   : RestType2)} ,
           
                        RecLex  = {( rkey   : Natural,
                                     cat    : SynCatInfo )} ,
                
                        Il-Dict = {( skey   : Natural,
                                     desamb : String,
                                     mkey   : Natural,
                                     rest   : RestType3)}
                     )
\end{verbatim}         
{\bf Database Keys}
\begin{verbatim}
     Key (M-Dict)  = {(stem,fkey,phon)}
     Key (S-Dict)  = {(skey)}
     Key (RecLex)  = {(rkey),(cat)}
     Key (Il-Dict) = {(skey,mkey),(skey,desamb)}
\end{verbatim}           
{\bf Entity Relationship Diagram}
\begin{verbatim}           
     M-Dict--1-------n--S-Dict--n------1--RecLex
               fkey      |        rkey
                         n
                         |  s
                         |  k
                         |  e                 m,n IN {0,1,2,...}
                         |  y
                         m
                         |
                       Il-Dict
\end{verbatim} 
     
        The suggested RecLex can also be added in the current practical model.
        There, it could be either a table `behind' B-Lex (making B-Lex a table
        with pairs of keys as tuples) containing only the different records, 
        or it could be a change on B-Lex where B-Lex would adopt the function of
        RecLex and S-Dict would be extended with an extra key `rkey' which 
        gives the syntactic category record (in B-Lex).
\end{description}
 
\subsubsection{Completeness}
The completeness of the tables is necessary in the three mentioned models 
(this can be derived using both the entity relationship diagrams and the 
requirements). However, for some special cases (like `empties') this 
completeness is violated. 
 
A solution to this problem would be to use special keys for those cases and 
to modify the 
requirements by adding exceptions for the appropriate keys. This would, however,
disturb the compactness of the model because of the need for more separate 
requirements. Furthermore, these requirements would be more elaborate.
 
Another solution would be to add so-called dummy entries to maintain the
completeness of the tables. This approach has several advantages:
\begin{itemize}
   \item The model is still clear and compact.
   \item The dictionaries are easier to use (because of less exception handling)
         in the robust mode of Rosetta. This, because then every database entry          does have a representation.
\end{itemize}
 
\subsection{Van Dale Constructs}
 
In the Van Dale dictionary several constructs are used to organize the 
information of an entry. The description of the entry (in detail) is done
in \cite{sm:descr}.\footnote{There, only the entry of the N-N dictionary is 
described.} The
constructs are described here in a more general way to give a global idea of
the possible organization of an entry:
 
\begin{itemize}
  \item {\bf Roman sub-division}\\
        The entry has been divided in two or more so-called roman sub-entries.
        This is done because of the sub-category difference. It is achieved by 
        means of the GR-code and more than one RR-code. For example, the Dutch 
        verb ``aankondigen'' is both reflexive and intransitive. For this reason
        the entry has been split in two parts,  
        one part for each sub-category.
  \item {\bf Meaning Ambiguity}\\
        The entry (or roman sub-entry) has been divided in more meanings. 
        These are marked by NN-codes. E.g. the Dutch noun ``briljant'' is 
        divided for the  meanings of ``diamond'' and (as used in printing) of 
        ``half non-pareil''.
  \item {\bf Spelling Variant}\\
        The spelling variants (behind a VV-code which is not followed
        by a GV-code or behind a SI-code) only differ in the spelling 
        (or perhaps a phonetic attribute) with respect to the entry. 
        Syntactical behaviour, meaning and translation should be exactly the 
        same. E.g. the Dutch (well..., more or less) noun ``jioe-jitsoe'' has 
        two spelling variants: ``djioe-djitsoe'' and ``jiu-jitsu''. 
  \item {\bf Form Variant}\\
        The form variant (behind VV-code but {\em with} GV-code) does not only 
        differ in spelling (and phonetic attributes) but also has a different  
        syntactic behaviour (e.g. a different gender). There are two kinds of
        form variants: the first, having the same meaning and translation;
        the second, with {\em only} the same (English) translation. An example
        of both classes: 
        \begin{enumerate}
           \item ``spekkie'' (more or less the Dutch equivalent of 
                 ``marshmallow'') has a form variant ``spek'' with
                 a different plural form (``-s'' and ``-en'' respectively).
           \item ``schenker'' (``donor'', masculine) has a form variant 
                 ``schenkster'' (``donor'', feminine). Both forms have, however,
                 different meanings and grammatical information but the same  
                 English translation.
        \end{enumerate}
  \item {\bf More Entries With Same String}\\
        The two (or more) entries have the same string but a different category.
        An example are the Dutch words ``briljant'' (noun, for meaning see 
        above) and ``briljant'' (adjective, ``brilliant''). 
  \item {\bf Syntactic Differences}\\
        A word can have syntactic differences between several meanings.
        In the Van Dale dictionaries this is marked quite badly. E.g. the Dutch
        noun ``aardworm'' has three meanings from which one (which depicts the 
        class of animals) does not have a plural form (the other two have).
        It is marked at the top of the entry behind the GI-code by 
        ``in bet. 3 g.mv.'' (in meaning 3 no plural).
\end{itemize}
 
\subsection{Mapping Van Dale On Rosetta Dictionaries}
 
\subsubsection{Method}
 
The mapping of the two systems (models, dictionaries, etc.) has some
interesting aspects with respect to the Rosetta dictionaries. These are the
result of the special structure of the dictionaries: information is given in
layers (tables). The following two items are special within the Rosetta 
dictionaries while they cannot be found (at least not directly) in the Van Dale.
 
\begin{itemize}
  \item {\bf Phonetic Attribute Ambiguity}\\
        This ambiguity occurs when, for example, a noun can be inflected by a 
        rule that generates two forms. E.g. the Dutch noun ``spons'' has two 
        plural forms, ``sponsen'' and ``sponzen'', although its set of plural
        formings has only one element. This is due to the phonetic attribute 
        `change'.
  \item {\bf Verb With Particle}\\
        Particles are separated from their verb. For example, the Dutch verbs
        ``aankondigen'' and ``afkondigen'' will not be found in M-Dict. Only
        a verb with stem ``kondig'' will be present. Which particle can be used 
        (in this case {\em must} be used) can be found using S-Dict. There, the 
        key associated with ``kondig'' will give two keys: one as reference to 
        the syntactic category record with obliged use of ``aan'' and the
        other to such a record with ``af'' (which are the correct particles).
\end{itemize}
 
These two items form, together with the previous mentioned Van Dale constructs,
the list of items which the dictionary generator should be able to handle. So, 
for each of the items a solution (implementation) is given for the correct 
(possible) conversion. 
 
\begin{itemize}
  \item {\bf Meaning Ambiguity}\\
        This is solved in the step from S-Dict to Il-Dict. Here, with one S-Dict
        entry there can be more Il-Dict entries with the same skey.
  \item {\bf Roman sub-division}\\
        For each roman sub-entry there is an entry in S-Dict with the same fkey
        as the entry has in M-Dict.
  \item {\bf Spelling Variant}\\
        Spellings variants are all stored in M-Dict and all mapped to the same
        S-Dict entry (having equal fkeys).
  \item {\bf Form Variant}\\
        Both entry and form variant occur in M-Dict, S-Dict and Il-Dict 
        separately. In Il-Dict they `share' the desambiguation strings.
        However, for both kinds of form variants there is a different solution 
        concerning the mkeys:
        \begin{description}
           \item [Same meaning] entry and form variant share the mkey(s).
           \item [Different meaning] entry and form variant have a disjunct
                 set of mkeys which, however, are mapped on the same word (in
                 English, that is).
        \end{description}
  \item {\bf More Entries With Same String}\\
        Words with the same string are treated as one roman entry (per word
        a sub-entry).
  \item {\bf Syntactic Differences}\\
        Such a word is treated as a roman entry (with the syntactic 
        differences in different sub-entries).
  \item {\bf Phonetic Attribute Ambiguity}\\
        These words are doubled\footnote{Or quadrupled in case of double 
        ambiguity -- this really happens: see ``gannef''.} in M-Dict where only 
        the phonetic parts of the entry differ.
  \item {\bf Verb With Particle}\\
        Verbs with the same stem are only present once in M-Dict. In S-Dict 
        these words will be split for having a different syntactic category 
        information record (viz. a different particle). 
\end{itemize}
 
In the following example dictionary the principles as described above are
`implemented'. Some overlap of principles is visible in certain words but this 
is avoided as much as possible.
 
\subsubsection{Example Dictionary}
 
Below, an example dictionary which treats at least one example of each different
construct in the Van Dale dictionary is given. Also, the special Rosetta 
dictionary features are treated in these examples.\vspace{3mm}
 
\begin{tabular}{|l||c|c|c|c|c|c|c|c|c||}    \hline
  Van Dale entry   &a&c&d&e&f&m&p&r&s     \\ \hline
  aankondigen      & & & & & & &x& &      \\
  aardworm         &x& & & & & & & &x     \\
  afkondigen       &x& & & & & &x&x&      \\
  briljant$^1$     &x& & & & &x& & &      \\
  briljant$^2$     & & & & & &x& & &      \\
  djioe-djitsoe    & & &x& & & & & &      \\
  jioe-jitsoe      & & &x& & & & & &      \\
  ju-jitsu         & & &x& & & & & &      \\
  pannespons       & &x& & & & & & &      \\
  schenker         &x& & & &x& & & &      \\ 
  spekkie          & & & &x& & & & &      \\ \hline
\end{tabular}
%\hspace{.2cm}
\shortstack{\vspace{-15mm}
\begin{tabular}{l@{\hspace{1mm}=\hspace{1mm}}l}
   a & ambiguous within sub-entry\\
   c & phonchange can have both values\\
   d & diff. string with same meaning\\
   e & form-variant with equal meaning\\
   f & form-variant with diff. meaning\\
   m & more entries with the same string\\
   p & verb with particle\\
   r & roman sub-division of entry\\
   s & syntax diff. between meanings
\end{tabular}
}
 
\vspace{3mm}
The table above is converted into the database. This datebase is made by 
applying the 
earlier mentioned transformation rules. In the first table (M-Dict) 
a don't-care is represented as an ``x''. True and false are represented by 
``T'' and ``F'' respectively. 
 
\begin{verbatim}
M-Dict
    
    stem               fkey              SC (sjwa-change)
 
    "aardworm"         F-aardworm        xx
    "briljant"         F-briljant        xx
    "djioe-djitsoe"    F-jioe-jitsoe     xx      {spell. var. jioe-jitsoe}
    "jioe-jitsoe"      F-jioe-jitsoe     xx
    "ju-jitsu"         F-jioe-jitsoe     xx      {spell. var. jioe-jitsoe}
    "kondig"           F-kondig          xx      {af- and aan-kondigen}
    "pannespons"       F-pannespons      xF      {sponsen}
    "pannespons"       F-pannespons      xT      {sponzen}
    "schenker"         F-schenker        xx
    "schenkster"       F-schenkster      xx      {form variant diff. mean.}
    "spek"             F-spek            xx      {form variant equal mean.}
    "spekkie"          F-spekkie         xx
\end{verbatim}
\begin{verbatim}
 
S-Dict
    
   fkey             skey
    
   F-aardworm       S-aardworm1      {1 and 2: syntax diff. -- see B-Lex}
   F-aardworm       S-aardworm2
   F-briljant       S-briljant1      {1 and 2: more entries -- see B-Lex}
   F-briljant       S-briljant2
   F-jioe-jitsoe    S-jioe-jitsoe 
   F-kondig         S-aankondig1     {1 and 2: roman - see B-Lex}
   F-kondig         S-aankondig2
   F-kondig         S-afkondig
   F-pannespons     S-pannespons
   F-schenker       S-schenker
   F-schenkster     S-schenkster
   F-spek           S-spek
   F-spekkie        S-spekkie
\end{verbatim}
\begin{verbatim}
 
B-Lex
    
   skey
    
   S-aardworm1                     { EnPlural }
   S-aardworm2                     { NoPlural }
   S-aankondig1                    { <ov.ww.> }
   S-aankondig2                    { <wk.ww.> }
   S-afkondig
   S-briljant1                     { <zn.>}
   S-briljant2                     { <bn.>}
   S-jioe-jitsoe
   S-pannespons
   S-schenker
   S-schenkster
   S-spek                          { EnPlural }
   S-spekkie                       { SPlural  }
\end{verbatim}
\begin{verbatim}
 
Il-Dict
    
   skey            desamb                         mkey
    
   S-aardworm1    "regenworm"                    M-01
   S-aardworm1    "mens"                         M-02
   S-aardworm2    "<dierk.> klasse"              M-03
   S-aankondig1   "officieel bekendmaken"        M-04
   S-aankondig1   "te kennen geven"              M-05
   S-aankondig1   "inluiden"                     M-06
   S-aankondig1   "meedelen"                     M-07
   S-aankondig1   "het verschijnen bekendmaken"  M-08
   S-aankondig1   "<kaartspel>"                  M-09
   S-aankondig2   "zich openbaren"               M-10
   S-afkondig     ""                             M-11
   S-briljant1    "diamant"                      M-12
   S-briljant1    "<druk.; vero.>"               M-13
   S-briljant2    ""                             M-14
   S-jioe-jitsoe  ""                             M-15
   S-pannespons   ""                             M-16
   S-schenker     "gever"                        M-17
   S-schenker     "mbt. dranken"                 M-18
   S-schenkster   "gever"                        M-19
   S-schenkster   "mbt. dranken"                 M-20
   S-spek         ""                             M-21
   S-spekkie      ""                             M-21
\end{verbatim}
 
\subsubsection{Interesting Combinations}
 
Some interesting combinations of entries with different qualities (like
roman subdivision, meaning ambiguity, form variant(s), particles, etc.)
are given here. Since it would take a lot of time to find all possible 
combinations (whether they exist or not) only {\em some} are given.
 
\begin{itemize}
  \item The entries ``vorst$^1$'', ``vorst$^2$'',
        and ``vorstin'' together form an interesting combination. The first is
        a noun without a form variant and with a roman subdivision. The second
        entry is a noun with a form variant (``vorstin'') but without a roman 
        subdivision. The third entry already exists as a form variant but
        also has a meaning ambiguity. The table scheme looks as follows:\newpage
        \begin{verbatim}
             M-Dict         S-Dict         Il-Dict
 
                            S-vorst1I      M1
             "vorst"        S-vorst1II     M2
                            S-vorst2       M3
             "vorstin"      S-vorstin      M4
                                           M5
             
        \end{verbatim}
  \item The entry ``acanthus'' is 
        a noun with a spelling variant {\em and} a form variant. 
        The table scheme looks as follows:\\
        \begin{verbatim}
             M-Dict         S-Dict         Il-Dict
 
             "acanthus"     S-acant(h)us
             "acantus"                     M1
             "akant"        S-akant
        \end{verbatim}
  \item The entry ``chic'' is 
        a noun with a spelling variant {\em and} roman subdivision. 
        The table scheme looks as follows:\\
        \begin{verbatim}
             M-Dict         S-Dict         Il-Dict
 
             "chic"         S-I            M-1
             "sjiek"        S-II           M-2
        \end{verbatim}
  \item The entries ``corrector$^1$'' and ``corrector$^2$'',
        also form an interesting combination. The first is
        a noun with a spelling variant. The second entry is a noun with a 
        spelling variant (``korrektor'') and a form variant (``correctrice'') 
        that has a spelling variant too (``korrektrice'').\footnote{Remarkable 
        is that both entries are in reverse order in the N-E.}
        The table scheme looks as follows:\\
        \begin{verbatim}
             M-Dict         S-Dict         Il-Dict
 
             "corrector"    S-cor1         M-1
             "korrektor"    S-cor2         
             "correctrice"  S-cor3         M-2
             "korrektrice"  
        \end{verbatim}
\end{itemize}
 
\subsection{Structure English Dictionary}
Within the described method, the N-E dictionary of the Van Dale is used 
as source for the {\em English} vocabulary. This will result in a rather 
unnatural 
dictionary, and when used in the reverse way (for translation of English words
in Dutch), there is no desambiguation.\footnote{The use of the generated 
dictionaries in the reverse way never was an objective.} Also, it is possible 
that ambiguities 
exist longer than necessary. As can be seen in the English Rosetta dictionaries 
derived from the example Dutch dictionary, the amount of mkeys and skeys is 
quite large in comparison with the Dutch dictionary.
 
\begin{verbatim}
IlDict                 S-Dict             M-Dict
 
   M-01  ""  S-01         S-01  F-07         F-01  "announce"
   M-02  ""  S-19         S-02  F-17         F-02  "bid"
   M-02  ""  S-02         S-03  F-07         F-03  "brilliant"
   M-03  ""  S-03         S-04  F-01         F-04  "cupbearer"
   M-04  ""  S-04         S-05  F-10         F-05  "diamond"
   M-05  ""  S-05         S-06  F-01         F-06  "donor"
   M-05  ""  S-20         S-07  F-01         F-07  "earthworm"
   M-06  ""  S-06         S-08  F-01         F-08  "half non-pareil"
   M-06  ""  S-21         S-09  F-02         F-09  "herald"
   M-07  ""  S-07         S-10  F-15         F-10  "indicate"
   M-08  ""  S-08         S-11  F-14         F-11  "jiujitsu"
   M-09  ""  S-09         S-12  F-05         F-12  "jujitsu"
   M-10  ""  S-10         S-13  F-08         F-13  "marshmallow"
   M-11  ""  S-11         S-14  F-03         F-14  "proclaim"
   M-12  ""  S-12         S-15  F-11         F-15  "reveal"
   M-13  ""  S-13         S-16  F-16         F-16  "scourer"
   M-14  ""  S-14         S-17  F-06         F-17  "worm"
   M-15  ""  S-15         S-18  F-04
   M-15  ""  S-22         S-19  F-17
   M-16  ""  S-16         S-20  F-15
   M-17  ""  S-17         S-21  F-09
   M-18  ""  S-18         S-22  F-12
   M-19  ""  S-17         S-23  F-13
   M-20  ""  S-18
   M-21  ""  S-23
\end{verbatim}
 
\newpage
\section{Filling Dutch Attributes}
 
Under the present circumstances, the attributes of the open categories `bverb',
`bnoun', `badj' and `badv' cannot 
--like those of the closed categories-- be filled by hand, because there are 
too many words belonging to these categories. Therefore, all possibilities to
fill attributes in a `clever' way are mentioned in this overview. 
 
In each of the next sections, all attributes of one of the open categories are 
listed. Note that the suggested default values (in the `records') 
have a preliminary status and may change in the future. 
 
\subsection{The attributes of BNOUN}
 
The domain of BNOUN:
 
\begin{verbatim}
BNOUNrecord      =
                  <
                   req:          polarityEFFSETtype:[pospol, 
                                                     negpol, omegapol]
                   env:          polarityEFFSETtype:[pospol, 
                                                     negpol, omegapol]
 
                   pluralforms:  pluralformSETtype:[enPlural, sPlural]
                   genders:      genderSETtype:[omegagender]  
 
                   class:        timeadvclasstype:omegaTimeAdvClass
                   deixis:       deixistype:omegadeixis
                   aspect:       aspecttype:omegaAspect
                   retro:        retrotype:false
 
                   sexes:        sexSETtype:[]                        
                   subcs:        nounsubcSETtype:[othernoun]
                   temporal:     temporaltype:false
                   possgeni:     possgenitype:false
                   animate:      animatetype:Omegaanimate
                   human:        humantype:Omegahuman                 
                   posscomas:    posscomaSETtype:[count]
                   thetanp:      thetanptype:omegathetanp
                   nounpatterns: synpatternSETtype:[]
                   prepkey:      keytype:0
                   personal:     personaltype:true
 
                   KEY              
                  >
\end{verbatim}
 
\begin{itemize}
  \item {\bf req:} words that have a non-default value can be found in 
        linguistic literature.
 
  \item {\bf env:} words that have a non-default value can be found in 
        linguistic literature.
 
  \item {\bf pluralforms:} the majority of the nouns is filled correctly by the 
        program {\em SetPlural}. Approximately 0.1 \% will need evaluation; 
        most of these are irrelevant with respect to the attribute {\em 
        pluralforms} (like `vaak mv.' etc), some are irregular diminutives 
        (which should be used for the `dimforms' attribute, which is not yet 
        in the domain). Also many FONmarkers get their value in this 
        program.\footnote{Almost all {\em relevant} cases can be found (the 
        attribute is not relevant for words like ``dikkerd'' that would get the 
        same inflected forms for the value {\em sjwa equals false}.}
 
  \item {\bf genders:} can be filled automatically by using the category 
        numbers of the Van Dale.
 
  \item {\bf class:} has to be done by hand; papers of trainees (\cite{po:tp},
        \cite{ko:ta}) contain interesting material.
        Maybe a pre-selection can be made on strings like ``maat'' or 
        ``maataanduiding'' or ``eenheid'', etc.
 
  \item {\bf deixis:} has to be done by hand; papers of trainees (\cite{po:tp},
        \cite{ko:ta}) contain interesting material.
        Maybe a pre-selection can be made on strings like ``maat'' or 
        ``maataanduiding'' or ``eenheid'', etc.
 
  \item {\bf aspect:} has to be done by hand; papers of trainees (\cite{po:tp},
        \cite{ko:ta}) contain interesting material.
        Maybe a pre-selection can be made on strings like ``maat'' or 
        ``maataanduiding'' or ``eenheid'', etc.
 
  \item {\bf retro:} has to be done by hand; papers of trainees (\cite{po:tp},
        \cite{ko:ta}) contain interesting material.
        Maybe a pre-selection can be made on strings like ``maat'' or 
        ``maataanduiding'' or ``eenheid'', etc.
 
  \item {\bf sexes:} maybe a quite good level of filling can be achieved by 
        using several parts of information of the Van Dale:
\begin{enumerate}
     \item the category numbers `08' and `09', that are used for neuter words 
           that denote female or male persons, like ``joch'', 
           ``jochie'', ``animeermeisje'', etc.
     \item words with BB-codes beginning with ``man'' or ``vrouw'' can be 
           evaluated; the same holds for BB-codes (that begin with) with 
           ``pers'', etc. See also the attribute {\em human}.
     \item the synonyms of the words mentioned in 1. and 2. can be evaluated.
     \item of course, `de'-words often have a value for sexes that corresponds 
           with the value for {\em genders}. Then, we only need to check the
           value of {\em animate} to see whether or not they are indeed living 
           creatures.
           Note that the `het'-words have been done in 1 (see above).
\end{enumerate}
  \item {\bf subcs:} has to be done by hand; the special values can (probably) 
        be found in a thesaurus or by checking special features in the N-N,
        like (the strings):
        ``titel'', ``beroep'', ``eenheid'', ``maat'', etc.
  \item {\bf temporal:} has to be done by hand; papers of trainees (\cite{po:tp},
        \cite{ko:ta}) may help.
        Maybe a pre-selection can be made on ``maat'' or ``maataanduiding'' or 
        ``eenheid'', etc.
  \item {\bf possgeni:} has to be done by hand. The non-default value seems to 
        be limited to {\em few} words, which probably can be extracted from a
        thesaurus.
  \item {\bf animate:} maybe some words can be found by searching strings
        ``dier'', ``vogel'', etc. Of course there is a relation with the 
        attributes {\em human} and {\em sexes}, etc. (both are subsets 
        of {\em animate}). 
  \item {\bf human:} maybe some words can be found by searching strings
        ``pers'', ``iem.'', etc. See also {\em sexes}.
  \item {\bf posscomas:} the `mass'-value corresponds to a high 
        degree to the feature `g.mv.' in the Van Dale. Also, words that do not 
        have a feature for plural at all in the Van Dale are candidate for the 
        value `mass'.
  \item {\bf thetanp:} has to be done by hand, but probably a lot of nouns with 
        interesting values can be derived from verbs (compare: ``pogen'' vs. 
        ``poging'').
  \item {\bf nounpatterns}: has to be done by hand. A lot of nouns with 
        interesting values can be derived from verbs, (compare: ``pogen'' vs. 
        ``poging'').
  \item {\bf prepkey:} has to be done by hand, but a lot of nouns with 
        interesting values can be derived from verbs. Also, the 
        `combination category' of the N-N may help to find nouns that can 
        be combined with prepositions.
  \item {\bf personal:} words like``winter'' have a non-default value, because 
        of 
        constructions like ``het is winter''. It is not clear whether or not 
        these words can be found and how many of them exist.
\end{itemize}
 
\subsection{The attributes of BVERB}
 
The domain of BVERB:
 
\begin{verbatim}
BVERBrecord    =
                  <
                    req:          polarityEFFSETtype:[pospol, 
                                                      negpol, omegapol]
                    env:          polarityEFFSETtype:[pospol, 
                                                      negpol, omegapol]
 
                    conjclasses:  conjclassSETtype:[3]  
                    particle:     keytype:0  
 
                    possvoices:   VoiceSETtype:[active, passive, 
                                  DoorActive]     
                    reflexivity:  reflexivetype:notreflexive
                    synvps:       SynpatternSETtype:[]
                    thetavp:      Thetavptype:omegathetavp
                    CaseAssigner: CaseAssignertype:true
 
                    subc:         verbsubcSETtype:[mainverb]
                    perfauxs:     perfauxSETtype:[hebaux]   
 
                    prepkey1:     keytype:0
                    prepkey2:     keytype:0 
 
                    controller:   controllertype:none
                    verbraiser:   verbraisertype:noVR               
                    IPP:          IPPtype:NOIPP                        
 
                    classes:      classSETtype:[durative]
 
                    KEY
                  >
\end{verbatim}
 
\begin{itemize}
  \item {\bf req:} there do not seem to be many verbs with negation; a program 
        that checked examples in the Van Dale delivered  very few verbs that 
        cannot occur without negation (it {\em did} deliver quite a few 
        interesting idioms, however). Probably, investigation of linguistic 
        literature will yield more verbs.
         
  \item {\bf env:} most verbs will have default value. Verbs with interesting 
        non-default values can probably be found in literature.
 
  \item {\bf conjclasses:} is done automatically by the program {\em SetConj}.
         Also, almost all FONmarkers can be set by this program (in fact the 
         only words that possibly do not get a proper value for FONmarker, 
         are verbs that cannot be inflected; therefore the chance that a 
         wrong value leads to consequences is very small). 
 
  \item {\bf particle:} is done automatically by program {\em SetConj} that 
        also fills {\em conjclasses}.
 
  \item {\bf possvoices:} normally, reflexives and verbs that have `isaux' get 
        the value `[active]'; non-reflexives with `hebaux' get 
        `[active, passive, DoorActive]'. The verbs ``lezen'', ``voelen'', 
        ``zien'', ``horen'', ``proeven'', ``ruiken'' get the value 
        `[AanActive]'. Also verbs that {\em don't} have a past participle, 
        like ``zullen'', cannot have a passive. Maybe the distinction 
        `transitive' vs. `intransitive' in the N-N can be useful. 
        Note that some verbs get a wrong value (when filled as stated here); 
        ``bezitten'' gets a wrong value, so that wrong (or at least strange) 
        sentences like ``een boek werd door hem bezeten'' 
        can be analyzed. Sometimes (for example for ``kosten'') a wrong value 
        does not yield 
        problems because the grammar more or less `corrects' it.
 
  \item {\bf reflexivity:} can be filled by program {\em SetConj}.
        Also, a program has been made that checks examples on the string 
        ``zich'' (but {\em not (yet)} on ``me'', ``je'', ``ons'' etc.). 
        Also, \cite{an:s84} contains a list.
 
  \item {\bf synvps:} has (probably) to be done by hand. Some preparations 
        have been made to fill all attributes that express the `verbpattern': 
        programs have been made that select verbs with example-sentences with 
        specific complements, like `te' + inf., `dat'-sentences, etc.
 
  \item {\bf thetavp:} has (probably) to be done by hand. Some preparations 
        have been made to fill all attributes that express the `verbpattern': 
        programs have been made that select verbs with example-sentences with 
        specific complements, like `te' + inf., `dat'-sentences, etc.
 
  \item {\bf CaseAssigner:} verbs with `hebaux' are case-assigners, verbs with
        `isaux' are {\em not}. Some exceptions exist: ``vergeten'', ``volgen'', 
        ``nalopen''.
 
  \item {\bf subc:} a (small) list exists of non-default words.
 
  \item {\bf perfauxs:} is done by program {\em SetConj}, that analyses
        the string with the conjugation in the N-N that contains 
        ``i.'', ``h.'', etc.
 
  \item {\bf prepkey1:} by hand; the `combination category' of N-N may 
        give a indication.
 
  \item {\bf prepkey2:} by hand; the `combination category' of N-N may 
        give a indication.
 
  \item {\bf controller:} only relevant for verbs that have a bare infinitive
        as complement (which is a limited number; the verbs can be found in the 
        linguistic literature), and verbs that have {\em ``te''} + infinitive 
        (most of which have been extracted from the Van Dale by a special 
        program followed by some searching for synonyms).
 
  \item {\bf verbraiser:} only relevant for a subset of the verbs that have a 
        non-default value for the attribute `controller'.
 
  \item {\bf IPP (infinitivum pro participium):} only relevant for a subset 
        of the verbs that have a non-default value for the attribute 
        `verbraiser' (all verbs that have a bare infinitive and some verbs 
        that have {\em `te'} + infinitive are interesting).
 
  \item {\bf classes:} has to be done by hand.
\end{itemize}
 
\subsection{The attributes of BADJ}
 
The domain of BADJ:
 
\begin{verbatim}
BADJrecord       =
                   <
                    req:          polarityEFFSETtype:[pospol, 
                                                      negpol, omegapol]
                    env:          polarityEFFSETtype:[pospol, 
                                                      negpol, omegapol]
 
                    class:        timeadvclasstype:omegaTimeAdvClass
                    deixis:       deixistype:omegadeixis
                    aspect:       aspecttype:omegaAspect
                    retro:        retrotype:false
 
                    uses:         adjuseSETtype:[attributive, predicative,
                                  nominalised] 
                    eFormation:   eFormationtype:true  
                    sFormation:   sFormationtype:true 
                    eNominalised: eNominalisedtype:true  
                    comparatives: comparativeSETtype:[erComp]  
                    superlatives: superlativeSETtype:[stSup]  
 
                    temporal:     temporaltype:false
                    subcs:        adjsubcSETtype:[otheradj]
                    reflexivity:  reflexivetype:notreflexive
                    thetaadj:     thetaadjtype:omegathetaadj
                    adjpatterns:  synpatternSETtype:[]
                    prepkey:      keytype:0
                    possadv:      possadvtype:true
                    KEY
                   >
\end{verbatim}
 
\begin{itemize}
\item {\bf req:} words that have a non-default value can be found in 
        linguistic literature.
 
\item {\bf env:} words that have a non-default value can be found in 
        linguistic literature.
 
\item {\bf Qstatus:} list can be made based on several parts of the adverb list
        made by A. Post.
 
\item {\bf class:} only for `temporal' adj.; \cite{po:tp},
        \cite{ko:ta} may be interesting.
 
\item {\bf deixis:} only for `temporal' adj.; \cite{po:tp},
        \cite{ko:ta}
        may be interesting.
 
\item {\bf aspect:} only for `temporal' adj.; \cite{po:tp},
        \cite{ko:ta}
        may be interesting.
 
\item {\bf retro:} only for `temporal' adj.; \cite{po:tp},
        \cite{ko:ta}
        may be interesting.
 
\item {\bf uses:} many can be found in the N-N.
 
\item {\bf eFormation:} literature; retrograde list; some can be found in N-N
      (behind GI-code).
 
\item {\bf sFormation:} literature; retrograde list; some can be found in N-N.
 
\item {\bf eNominalised:} literature; retrograde list; some can be found in N-N.
 
\item {\bf comparatives:} irregular and some regular forms in N-N.
 
\item {\bf superlatives:} compare `comparatives'.
 
\item {\bf temporal:} has to be done by hand probably. See also {\em class}, 
      {\em deixis}, etc. 
 
\item {\bf subcs:} has to be done by hand.
 
\item {\bf reflexivity:} probably only one: ``{\em zich} bewust van''; 
      a program (especially made to detect interesting cases) did not find 
      others. 
 
\item {\bf thetaadj:} by hand; maybe some (half-)automatically (compare BVERB).
 
\item {\bf adjpatterns:} by hand; maybe some (half-)automatically (comp. BVERB).
 
\item {\bf prepkey:} by hand; maybe some (half-)automatically (compare BVERB).
 
\item {\bf possadv:} probably automatically (for instance, `22' behind `GI').
\end{itemize}
 
\subsection{The attributes of BADV}
 
The domain of BADV:
 
\begin{verbatim}
BADVrecord       =
                  <
                   req:          polarityEFFSETtype: [pospol, 
                                                      negpol, omegapol]
                   env:          polarityEFFSETtype: [pospol, 
                                                      negpol, omegapol]
 
                   comparatives: comparativeSETtype: [erComp]  
                   superlatives: superlativeSETtype: [stSup]  
 
                   subcs:        ADVsubcSETtype: [VPAdv]
                   Qstatus:      Qstatustype: false
 
                   class:        timeadvclasstype: omegaTimeAdvClass
                   deixis:       deixistype: omegadeixis
                   aspect:       aspecttype: omegaAspect
                   retro:        retrotype: false
 
                   advmood:      xpmoodtype: declxpmood
                   thetaadv:     thetaadvtype: omegathetaadv
                   advpatterns:  synpatternSETtype: []
                   prepkey:      keytype: 0
                   temporal:     temporaltype: false
                   possnietnp:   possnietnptype: false
                   thanas:       thanascompltype: omegacompl
 
                   KEY
                  >
\end{verbatim}
 
\begin{itemize}
\item {\bf req:} words that have a non-default value can be found in 
        linguistic literature (words like ``nauwelijks'' have `pospol').
 
\item {\bf env:} maybe exceptions can be found in literature.
 
\item {\bf  comparatives:} compare BADJ.
 
\item {\bf  superlatives:} compare BADJ.
 
\item {\bf  subcs:} can be found in adverb list made by A. Post.
 
\item {\bf  class:} compare BADJ.
 
\item {\bf  deixis:} compare BADJ.
 
\item {\bf  aspect:} compare BADJ.
 
\item {\bf  retro:} compare BADJ.
 
\item {\bf  advmood:} literature (``waarom'', ``hoe'', etc).
 
\item {\bf  thetaadv:} probably by hand.
 
\item {\bf  advpatterns:} probably by hand.
 
\item {\bf  prepkey:} probably by hand. Combination category in N-N may be 
      useful.
 
\item {\bf  temporal:} probably by hand.
 
\item {\bf  possnietnp:} list can probably be derived from literature.
 
\item {\bf  thanas:} (small) list can be found in literature.
 
\end{itemize}
 
\newpage
\begin{thebibliography}{AAA99a}
 \bibitem[ANS84]{an:s84} G. Geerts, W. Hasereyn, J. de Rooij, M.C. van den 
  Toorn, eds., {\bf Algemene Nederlandse Spraakkunst}, {\em Wolters-Noordhoff},
  1984.
 \bibitem[FEL86]{fe:r2} C. Fellinger, {\bf On dictionaries in Rosetta2},
  {\em Philips Research Labs}, Rosetta report 20, 1986.
 \bibitem[MAR84]{ma:groot} W. Martin, ed., {\bf Groot woordenboek 
  Engels -- Nederlands}, {\em Van Dale Lexicografie}, 1984.
 \bibitem[MTO86]{mt:groot} W. Martin, G.A.J. Tops, eds., {\bf Groot woordenboek 
  Nederlands -- Engels}, {\em Van Dale Lexicografie}, 1986.
 \bibitem[KOP87]{ko:ta} L. Kopinga, {\bf Time Adverbials}, {\em
  Free University of Amsterdam}, Senior thesis, 1987.
 \bibitem[LAN85]{la:r2} S.P.J. Landsbergen, {\bf Dictionaries in Rosetta2},
  {\em Philips Research Labs}, Rosetta report 15, 1985.
 \bibitem[LAN86]{la:r3} S.P.J. Landsbergen, {\bf Dictionaries in Rosetta3},
  {\em Philips Research Labs}, Rosetta report 125, 1986.
 \bibitem[ODI86]{od:cert} J.E. Odijk, {\bf (Certain) Pronouns in Rosetta3},
  {\em Philips Research Labs}, Rosetta report 148, 1986.
 \bibitem[OSC87]{os:r3} J.E. Odijk \& A.Y. Schenk, {\bf Dictionaries in 
  Rosetta3}, {\em Philips Research Labs}, Rosetta report 201, 1987.
 \bibitem[POS87]{po:tp} A. Post, {\bf Temporal Predication}, {\em
  University of Utrecht}, Senior thesis, 1987.
 \bibitem[ROU87]{ro:conv} J.H.G. Rous, {\bf Conversion problems with B-LEX},
  {\em Philips Research Labs}, Rosetta report 196, 1987.
 \bibitem[SME87]{sm:descr} H.E. Smit \& J.P. Medema, {\bf Description Van 
  Dale dictionary N~--~N}, {\em Philips Research Labs}, Rosetta report 174,
  1987.
 \bibitem[STE84]{st:groot} P.G.J. van Sterkenburg, ed., {\bf Groot woordenboek 
  hedendaags Nederlands}, {\em Van Dale Lexicografie}, 1984.
\end{thebibliography}
 
\end{document}
