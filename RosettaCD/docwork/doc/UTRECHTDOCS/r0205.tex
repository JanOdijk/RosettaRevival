
\documentstyle{Rosetta}
\begin{document}
   \RosTopic{Formalism}
   \RosTitle{Dictionaries in Rosetta3 Revisited}
   \RosAuthor{Jan Odijk, Andr\'{e} Schenk}
   \RosDocNr{0205}
   \RosDate{\today}
   \RosStatus{concept}
   \RosSupersedes{-}
   \RosDistribution{Project}
   \RosClearance{Project}
   \RosKeywords{dictionaries, idioms}
   \MakeRosTitle
%
%

\section{Introduction}
\paragraph{Intro}
This document discusses a variant of the proposals in document R0194 (Andr\'{e}
Schenk - Idioms and the Dictionary) and document R0201 (Jan Odijk and Andr\'{e}
Schenk - Dictionaries in Rosetta3) in as far as they concern the syntactic 
parts of the grammar and B-LEX. As will be shown below the proposal is very 
much in the spirit of DOC R0108 (Agnes Mijnhout - Verbpatterns in Rosetta3), in 
which it is argued that a distinction is necessary between thetavps and synvps 
since thetavps specify semantic properties and synvps specify syntactic 
properties.

\paragraph{Dictionaries}
As indicated in R0194 a new definition of the dictionaries was necessary in
order to solve some problems with idioms. Problems and solutions were given in
DOC R0201. Yet, in that proposal there were certain drawbacks related to a
mismatch between the treatment of idioms and of simple basic expressions. 

\paragraph{Idioms}
The strongest theory on idioms predicts that are no special syntactic rules for
idioms, i.e. idioms are not visible in the grammar apart from the lexicon. The
lexicon lists all phenomena which constitute exceptions to rules. The form of
an idiom cannot be captured by rules, since the lexical elements in an idiom
are idiosyncratic. Under the above theory that is exactly the outcome, the
idiom problem has been reduced to a dictionary problem. The proposal forwarded
here attempts to define M-grammars and the lexicons in such a way that this is
possible. 

\paragraph{Preview Sectioning}
In the next sections the problem will be formulated, an informal description of 
the solution will be given; then the solution will be formalized and in the 
last chapter certain potential problems will be given with some potential 
solutions.

\section{Definition of the Problem}
\paragraph{Intro}
In order to sketch the problem a comparison will be made between the analysis 
and the generation of the canonical form of simple and complex basic 
expressions as discussed in R0201. The comparison will be stated in general and 
thus informally.

\paragraph{Simple Basic Expressions}
The canonical form of simple basic expressions is generated as follows (note
that the term simple basic expression is somewhat misleading, simple basic
expressions also comprise for example verb particle combinations). The
derivation tree associated to the name of the basic expression is looked up.
For simple expressions this D-tree is rather trivial. It equals the name of the
basic expression. A terminal expression is formed on the basis of syntactic
information drawn from slex associated to the name of this basic expression
(skey). The syntactic information comprises the canonical syntactic
environment the expression can appear in, in the form of thetavps and synvps
(cf. DOC R0108, Agnes Mijnhout - Verbpatterns in Rosetta3) and the particle and
reflexive attributes. The start and the verbpattern rules then build the
syntactic environment of the verb (together with the zichspelling and the
particle spelling rules) on the basis of information specified at the verb. 

\paragraph{Complex Basic Expressions}
The canonical form of complex basic expressions (idioms) is generated in the
lexicon as described in R0194 and R0201. The derivation tree associated to the
name of the basic expression is looked up. BE-GRAMMAR generates a complex
S-tree from the non-trivial derivation tree, which corresponds to the name of
the basic expression, as specified in DERIV-TREES. 

\paragraph{Comparison}
If we compare the two methods we can see that the results (i.e. after the 
verbpattern rules) are the same for idioms and literals, in the sense that in
both cases there are S-trees containing lexical material and VARiables. Yet,
the way in which these S-trees are built is different, i.e. idiomatic canonical
S-trees are built in the lexicon and literal canonical S-trees are built in the 
grammar while using the same rules. 

\paragraph{Discussion}
This has some nasty consequences which are listed below. Most of the problems 
are more or less related to each other.

\paragraph{Problems}
\begin{enumerate}
  \item Certain attributes have to be doubled. In particular the thetavps and
the synvps attribute have to be doubled for idioms. The reason for this is that
within ID-GRAMMAR these attributes have to have other values than in literal
grammar. For example, thetavps indicates the number of free arguments a verb
can take. In ID-GRAMMAR the number of arguments for {\em kick} in {\em kick the
bucket} is 2, but in literal grammar it is one. 
  \item Because certain attribute values are different the head of an idiom can 
never be the same as the literal equivalent. For example, {\em spill} in {\em 
spill the beans} can never be the same as literal {\em spill}, while in fact 
their syntactic behaviour is the same in both cases.
  \item The fact that there is a difference between idiom heads and their
literal counterparts makes it necessary that in morphology when fkeys are 
mapped onto skeys literal and idiomatic paths are created (in analysis). To 
reduce the number of paths a so-called idiom technique is necessary.
  \item In R0201 it was necessary to refer to ILDICT in syntaxis which is in 
contradiction to the idea of modularity (i.e. there is a morphological, a 
syntactic and a semantic component and in the different components there is no 
reference to another component).
  \item It is impossible to get at the strongest theory on idioms, since idioms 
are visible in syntax.
\end{enumerate}

\section{Informal Description of the Solution}
\paragraph{Assumptions}
The proposal made here is based on the following assumptions:
\begin{enumerate}
  \item A terminal of an idiomatic S-tree behaves syntactically and 
morphologically in exactly the same way as one of its literal counterparts. For 
example, {\em spill} and {\em bean} in the idiom {\em spill the beans} have the 
same lexical properties as (one of) their literal equivalents.
  \item Non-terminal idiom parts behave in the same way as similar expressions. 
If an idiom part is a free argument then it behaves in the same way as other 
free arguments. If an idiom part is only a syntactic argument it behaves in the 
same way as other purely syntactic arguments. So, the NP {\em the beans} in the 
idiom {\em spill the beans} behaves in exactly the same way as the NP {\em it} 
in {\em it rains} or {\em it is clear that}, w.r.t. the relevant operations.
This is indicated by the category. 
\end{enumerate} 

\paragraph{Conclusion}
Under the above assumptions it should not have to be necessary to 
differentiate between literal and idiomatic paths until dictionary look-up, 
were it not for idthetavps and idpatterns (and some other attributes, cf. the 
problems section).

\paragraph{Two Possibilities}
Basically there are two ways to overcome this dilemma. The first is to try and
treat idioms in the same way as literals in the Start and Verbpattern rules, 
but as argued in doc R0194 there are several objections to such a method. The 
second is to try and treat literals in the same way as idioms.

\paragraph{Proposal}
In R0194 a lexicon DERIVTREES was defined that defined the syntactic properties 
of an idiom. In R0201 the definition of DERIVTREES had been extended such that 
DERIVTREES defined the syntactic properties of any basic expression (idioms 
or simple basic expressions). The D-tree for simple expressions was trivial in 
the sense that it was simply the name of the basic expression. DERIVTREES under 
the proposal forwarded here is extended such that the D-tree for simple
expressions is non-trivial either (at least in most cases); it contains
particle spelling, zich spelling rules, if necessary, and start and verbpattern
rules. Application of BE-GRAMMAR to DERIVTREES then renders a complex S-tree
for both idioms and non-idioms. 

\paragraph{BE-GRAMMAR}
BE-GRAMMAR consists of at least the following rules:

C: particlespelling . C: startverbrules . \{C: verbpatternrules\} . \{C: 
propositionsubstitution\} . C: controlrules . C: zichspellingrules .
\{C: nowhshiftsubstitution\}

\paragraph{Comment}
Particle spelling, startverb, verbpattern and zich spelling rules are 
applicable for both idioms and non-idioms. The other rules applicable for 
idioms only.

\paragraph{Examples Result of BE-GRAMMAR}
If we look at the results of BE-GRAMMAR for the sentence {\em John spilled the 
beans} in its non-idiomatic (figure~\ref{NID}) and idiomatic (figure~\ref{ID}) 
reading.

\begin{figure}[htb]
\par
\begingroup
\def\par{\leavevmode\endgraf}
\obeylines\obeyspaces
{\obeyspaces\global\let =\ }
                          S

                   V$_{1}$         VP

                              VERB      V$_{2}$
                               spill
\endgroup
\caption{Example of a non-idiomatic canonical S-tree}
\label{NID}
\end{figure}

\begin{figure}[htb]
\par
\begingroup
\def\par{\leavevmode\endgraf}
\obeylines\obeyspaces
{\obeyspaces\global\let =\ }
                          S

                   V$_{1}$         VP

                              VERB      NP
                               spill
                                        the beans
\endgroup
\caption{Example of an idiomatic canonical S-tree}
\label{ID}
\end{figure}

\paragraph{Characteristics}
{\em Spill} has identical properties in both figures and that {\em bean} is the
same as literal {\em bean}. There are V's which are placeholder variables for
VARiables. 

\paragraph{Placeholder substitution rules}
The V's are replaced by VARiables in the placeholder substitution rules that
apply before other syntactic rules in the grammar (cf. below). It has
implicitly been assumed that the attribute thetavps is not necessary any
longer. The number and the position of the free arguments is indicated by the
V's in the S-tree\footnote{In fact for the part of the grammar under discussion
the verbpattern attribute is superfluous as well; however, this attribute is
necessary in S-parser, so it has to be retained}. So, in M-GRAMMAR the start,
verbpattern, zich spelling and particle spelling do not exist any longer. A new
type of rule is necessary i.e. the placeholder substitution rules. These are
similar to other substitution rules, i.e. from left to right V's in the
canonical S-tree are replaced by VARiables. 

\section{Definitions}
The definitions are the same as in R0201 with, possibly, the following
exception. 

The set of basic expressions B is as follows.

\begin{tabbing}
aaaaaa \= aaaaaaaaaa \= aaaaaaaaaaa \= aaaaaaaaaa \= aaaaaaaaaa \= \kill
B =    \> \{ t $\mid \exists$ d, skey $\wedge$ \\
       \> d = BE-PARSER(t) $\wedge$\\
       \> $\langle$ skey,d $\rangle$  $\in$ DERIVTREES \\
       \> \}
\end{tabbing}

\section{Example}
\paragraph{Intro}
To illustrate, in this section an example is presented. The sentence is {\em 
John spilled the beans}.

\subsection{Example Lexicons and Dictionaries}

\begin{tabbing}
\= aa \= aaaaaaaaaaaaaaaaaaaaaaaaaaaaaaaaaaaaaaaaaaaaaaaaaaaaaaaaaaaaaaaaaa\kill
\> MDICT =\\
\> \> \{(spill,98), (the,752), (bean,110), (John,1334)\}\\ \\
\> MLEX =\\
\> \> \{\\
\> \> (98, BVERB,\{... regularconj ...\}),\\
\> \> (752, ART, \{ \}),\\
\> \> (110, BNOUN, \{... regplur ...\}),\\
\> \> (1334, BPROPERNOUN, \{... regulargenitive ...\})\\
\> \> \}\\ \\
\> SDICT =\\
\> \> \{(98, 98.1), (752, 752.1), (110, 110.1), (1334, 1334.1)\}\\ \\
\> SLEX =\\
\> \> \{\\
\> \> (98.1,BVERB,\{... trans, 2-place, +passive,..\}),\\
\> \> (752.1,ART,\{... def, ...\}),\\
\> \> (110.1,BNOUN,\{... count, ...\}),\\
\> \> (1334.1,BPROPERNOUN,\{... firstname ...\})\\
\> \> \}\\ \\
\> DERIVTREES =\\
\> \> \{\\
\> \> (98.1.1, TVP2a [RSTARTCL2 [98.1,V$_{1}$,V$_{3}$]]),\\
\> \> (98.1.7, RARGSUBST1,3 [TVP2a [RSTARTCL2 [98.1,V$_{1}$,V$_{3}$]], 
RNP1 [110.1]]),\\
\> \> (110.1.1, 110.1),\\
\> \> (1334.1.1, 1334.1)\\
\> \> \}\\ \\
\> ILDICT =\\
\> \> \{(98.1.1, 33), (98.1.7, 38), (110.1.1, 220), (1334.1.1, 267)\}\\ \\
\> ILLEX =\\
\> \> \{\\
\> \> (33, 2-pl-function,\{sort:activity,arg1:animate,arg2:concrete,...\}),\\
\> \> (38, 1-pl-function,\{sort:achievement,arg1:animate,meaning descr: 
'divulge a secret',...\}),\\
\> \> (220, constant,\{sort:...\}),\\
\> \> (267, constant,\{sort:human\})\\
\> \> \}
\end{tabbing}

\subsection{Example Derivation}
\paragraph{Intro}
The analytical derivation of {\em John spilled the beans} is given globally.

\paragraph{Input string} 
John spilled the beans

\paragraph{After segmentation} 
1334,98,SFKed,752,110,SFKs

\paragraph{LEX-Tree} 
PROPERNOUN[.1334.1.],VERB[.98.1.],ART[.752.1.],NOUN[.110.1.]

\paragraph{After the surface parser}
\begin{verbatim}
                      SENTENCE
 
                    NP         VP

                          VERB       NP

                                 ART    NOUN

                 1334.1   98.1  752.1   110.1
\end{verbatim}
\paragraph{After the Placeholder substitution rules}
The substitution rules are iterative and do not apply obligatory. The following 
structures will be checked whether they are basic expressions.

\paragraph{NP VERB NP}
fails, no D-trees associated to 98.1 can generate two NPs

\paragraph{V VERB NP}
succeeds, one D-tree (98.1.7) associated to 98.1 in DERIVTREE equals the
D-tree to generate this structure 

\paragraph{NP VERB V}
fails, no D-trees associated to 98.1 can generate an NP in subject position

\paragraph{V VERB V}
succeeds, one D-tree (98.1.1) associated to 98.1 in DERIVTREE equals the
D-tree to generate this structure 

\paragraph{Comment}
In the successful cases the name of the basic expression is incorporated into 
the D-tree.

\section{Problems}
\paragraph{Intro}
There is a class of potential problems. The characteristics of this class are
that it does not follow from the syntactic representation whether a syntactic 
operation is applicable. It should be emphasized that in general it is
impossible to specify such a distinction in the representation, since this 
would lead to differences between idiomatic and non-idiomatic paths. 

\paragraph{Example}
The canonical case is passive. Literal {\em kick} and {\em spill} can
passivize, but while {\em spill the beans} can passivize, {\em kick the
bucket} cannot. For passivization in English a transitive verb and an overt 
object are necessary. {\em Kick the bucket} meets that description.

\paragraph{Discussion}
There are cases similar to the above. In general one can ask whether problems 
like the above are due to idiom theory or to a wrong (or even absent) theory on 
the phenomenon under discussion. If we take passive as an example for several 
cases it is clear why passivization is not allowed, but in some cases this is 
unknown (cf. $^{\ast}${\em er wordt door de bloemen gebloeid}).

\paragraph{Solutions}
The best solution possible is to create a correct theory on the phenomenon 
under discussion. Since this seems to be impracticable another solution is to 
temporarily double the entries in the dictionary for the relevant cases, and to 
delete these entries when theory has improved. For example, this implies that 
for {\em kick} there are two entries in slex, one for the idiomatic and one for 
the literal case, and for {\em spill} there is only one entry in slex.

\paragraph{Efficiency}
Dictionaries under this approach will be larger since D-trees will have to be 
specified in the lexicon for simple expressions (though an efficient 
implementation is conceivable). Dictionary access is necessary 
for simple expressions as well (but for fewer paths). 

\end{document}
