   \documentstyle{Rosetta}
%   \makeindex
   \begin{document}
      \RosTopic{Rosetta3.Doc.Linguistics.English}
      \RosTitle{Verbpatterns of English}
      \RosAuthor{Margreet Sanders, Petra de Wit}
      \RosDocNr{248}
      \RosDate{19 September 1991}
      \RosStatus{concept}
      \RosSupersedes{February 26, 1988}
      \RosDistribution{Project}
      \RosClearance{Project}
      \RosKeywords{verbpatterns, syntax}
      \MakeRosTitle
   %
   %

\def\csr{\mbox{}\\\begin{tabular}[t]{llll}}
\def\endcsr{\end{tabular}}
\def\examples{\mbox{}\begin{enumerate}\vspace{1ex}}
\def\endexamples{\end{enumerate}}
\def\example{\vspace{-1ex}\item}
\def\vpattern{\begin{description}}
\def\endvpattern{\end{description}\vspace{2ex}}
\def\verbpattern#1{\section{#1}}
\def\theta#1{\subsection{#1}}
\def\exv#1{#1\index{#1}}
\def\new#1{#1$^{*}$}
\def\norule{No rule written yet}
\def\noV{No example Verb found}
\def\noS{No example sentence found yet}
\def\VR{Verb Raising Structure}
\def\thetadescr{\begin{description}}
\def\endthetadescr{\end{description}}
\def\ruleitem{\item[Rule]}
\def\csritem{\item[Canonical Surface Representations]}
\def\remarksitem{\item[Remarks]}
\def\evitem{\item[Example Verbs]}
\def\esitem{\item[Example Sentences]}

\section{Introduction}

This document lists all known verbpatterns of English in alphabetic order.
Examples are supplied together with some additional information. 
The list is based on the 
list created by Agnes Mijnhout (doc.\ R0168) and has been extended by the 
authors, often inspired by Dutch verbpatterns. The revised edition of this 
document also contains all the verbpatterns that were discovered during the 
filling of the large dictionaries. This document can function as a document 
for reference. 

For each verbpattern the following additional information is supplied:
\begin{description}
  \item[Rule] The name of the M-rule treating this verbpattern, with the number 
of the subrule (if present). Verbpatterns mentioned in this document that do 
not exist yet in the English domain are marked with an asterisk. 
  \item[Canonical Surface Representations] An indication of how this 
verbpattern is realized in structures that are the output of M-generator.
(Hence an indication of what kind of structures the surface parser must be
able to deal with). {\em  Canonical\/} is supposed to mean here that no
special rules such as {\em  shift rules\/}
have applied yet. 
  \item[Remarks]    Any remarks that seem appropriate, including doubts about 
the existence of the pattern.
\end{description}

Furthermore, for each {\em  thetavp\/} compatible with the verbpattern
being discussed {\bf example verbs} and {\bf example sentences} have been
supplied in special subsections. All these examples are taken from the current 
Rosetta dictionaries. An index of all example verbs is added at the 
end of the document.

Idiom patterns are not dealt with in the current document, nor are the patterns 
that are used specifically for adjectives. These latter are:\\
synFOREMPTY, synIOEMPTY\_DONP, synIOEMPTY\_THATSENT, synIOEMPTY
\_QSENT, 
synLOCEMPTY, synLOCPREPP, synnoadjpargs, 
synPREPOPENTOSENT, syn\-OPEN\-TOINF\-SENT\-PROOBJ, synPREPEMPTY, synTONP, 
synTONP\_DONP, synTONP\_THATSENT, synTONP\_QSENT.\\
The patterns synVERBPPROP and synCLAUSE are used for adverbs only and are not 
dealt with here either. The pattern synBE is used for the verb {\em be\/}, but 
is not a real verbpattern. Note that English verbpatterns do not distinguish 
between {\em to\/}, {\em for\/} and other preps. They all receive PREPNP.\\


A main problem for English at the moment is that no strict rules were used to 
call something an indirect object or a direct object. So, the patterns 
IONP\_... and DONP\_... were used without too much thought. The result is that 
some patterns have IONP\_... in English, while they have DONP\_... in Dutch. 
Because direct objects are passed on from IL as the second argument in the 
startrules, and indirect 
objects as the third, there now is a translation problem when going from the 
Dutch to the English dictionary. This could be solved by either giving the 
English verbs a thetavp {\em vp132\/}, or by giving them another synpattern. 
This will have to be checked thoroughly! Perhaps in case of prepositional 
objects, such mapping errors occur as well.
For a further discussion of problems 
concerning verbpatterns, see section 3 of doc.\ 240 on Dutch verbpatterns. The 
remarks on optionality, semi-arguments, directionals, adverbs and psych verbs
made in that document hold for English as well. 

Some verbpatterns have become superflous after the set of arguments was devided 
into (real) arguments and adjuncts. This is indicated under remarks in the 
description of the verb patterns.\\

Of the suggestions for 
amelioration, none has been described in any detail yet, so this document
cannot refer to them. See however also doc.\ 325, {\em Guide to determining 
Verbpatterns\/}, by Jan Odijk. The remark made at the end of 
section 3 of doc.\ 240 holds for English as well: 

{\bf It is absolutely necessary to revise certain patterns before large-scale 
dictionary filling is going to be done!!}



\newpage
\verbpattern{[synASIFSENT] }
\begin{vpattern}
\ruleitem TVerbpattern1,6
\csritem \mbox{}\\
     \begin{csr}
      extraposrel/SENTENCE\{as if\} 
     \end{csr}
\remarksitem
\end{vpattern}

\theta{vp010}

\begin{thetadescr}
\evitem  look\index{look}
\esitem
     \begin{examples}
        \example It looks as if she will win
     \end{examples}
\end{thetadescr}


\theta{vp120}

\begin{thetadescr}
\evitem  act\index{act}
\esitem
     \begin{examples}
        \example He acted as if he had gone mad
     \end{examples}
\end{thetadescr}


\newpage
\verbpattern{[synCLOSEDADJPPROP] }
\begin{vpattern}
\ruleitem TVerbpattern1,4a
\csritem \mbox{}\\
     \begin{csr}
                & predrel/ADJP\\
      objrel/NP & predrel/ADJP 
     \end{csr}
\remarksitem
  
\end{vpattern}

\theta{vp010}

\begin{thetadescr}
\evitem  remain\index{remain}, seem$^{1}$\index{seem$^{1}$}
\esitem
     \begin{examples}
        \example He remained calm\\
        \example He seems ill\\
                 Note that there probably are two verbs `seem': one with one 
                 argument
                 and one with two arguments (seem to sb). The latter case is 
                 covered below, in synCLOSEDADJPPROP\_EMPTY.
     \end{examples}
\end{thetadescr}


\theta{vp120}

\begin{thetadescr}
\evitem  consider\index{consider}
\esitem
     \begin{examples}
        \example I consider him foolish
     \end{examples}
\end{thetadescr}


\newpage
\verbpattern{[synCLOSEDADJPPROP\_EMPTY] }
\begin{vpattern}
\ruleitem TVerbpattern5,7
\csritem \mbox{}\\
     \begin{csr}
      predrel/ADJP
     \end{csr}
\remarksitem
  
\end{vpattern}

\theta{vp012}

\begin{thetadescr}
\evitem  seem$^{2}$ to\index{seem$^{2}$ to}
\esitem
     \begin{examples}
        \example He seems ill\\
                 Note that there probably are two verbs `seem': one with one 
                 argument
                 and one with two arguments (seem to sb). The former case is 
                 covered above, in synCLOSEDADJPPROP.
     \end{examples}
\end{thetadescr}


\theta{vp123}

\begin{thetadescr}
\evitem ? 
\esitem
     \begin{examples}
        \example ?
     \end{examples}
\end{thetadescr}


\newpage
\verbpattern{[synCLOSEDADJPPROP\_PREPNP] }
\begin{vpattern}
\ruleitem TVerbpattern8,5
\csritem \mbox{}\\
     \begin{csr}
      predrel/ADJP & prepobjrel/PREPP
     \end{csr}
\remarksitem For seem$^{2}$ going with an XPPROP, the pattern assigned is 
always XPPROP\_PREPNP. However, seem$^{2}$ going with a SENTENCE has received 
synPREPNP\_...SENT, because this seems to reflect the surface order directly.
Perhaps some more consistent strategy should be used here.
\end{vpattern}

\theta{vp012}

\begin{thetadescr}
\evitem  seem$^{2}$ to\index{seem$^{2}$ to}
\esitem
     \begin{examples}
        \example He seemed ill to me
     \end{examples}
\end{thetadescr}


\theta{vp123}

\begin{thetadescr}
\evitem  ?
\esitem
     \begin{examples}
        \example ?
     \end{examples}
\end{thetadescr}


\newpage
\verbpattern{[synCLOSEDGERUND] }
\begin{vpattern}
\ruleitem TVerbpattern1,16a
\csritem \mbox{}\\
     \begin{csr}
      complrel/SENTENCE\{accing\}
     \end{csr}
\remarksitem
\end{vpattern}

\theta{vp010}

\begin{thetadescr}
\evitem ? 
\esitem
     \begin{examples}
        \example ?
     \end{examples}
\end{thetadescr}


\theta{vp120}

\begin{thetadescr}
\evitem  watch\index{watch}, have\index{have}
\esitem
     \begin{examples}
        \example He watched the sun setting
        \example I had them laughing at my jokes
     \end{examples}
\end{thetadescr}


\newpage
\verbpattern{[synCLOSEDINFSENT] }
\begin{vpattern}
\ruleitem Tverbpattern1,11a
\csritem \mbox{}\\
     \begin{csr}
      pruning\\
      complrel/SENTENCE\{inf\}
     \end{csr}
\remarksitem
\end{vpattern}

\theta{vp010}

\begin{thetadescr}
\evitem  can$^{1}$\index{can$^{1}$}
\esitem
     \begin{examples}
        \example He can come (epistemic modal)
     \end{examples}
\end{thetadescr}


\theta{vp120}

\begin{thetadescr}
\evitem  see\index{see}, have\index{have}
\esitem
     \begin{examples}
        \example He saw the man leave
        \example I had him do it
     \end{examples}
\end{thetadescr}


\newpage
\verbpattern{[synCLOSEDNPPROP] }
\begin{vpattern}
\ruleitem TVerbpattern1,3a
\csritem \mbox{}\\
     \begin{csr}
                 & predrel/NP\\
       objrel/NP & predrel/NP
     \end{csr}
\remarksitem
\end{vpattern}

\theta{vp010}

\begin{thetadescr}
\evitem  become\index{become}, seem$^{1}$\index{seem$^{1}$}
\esitem
     \begin{examples}
        \example She became a poet
        \example He seemed a fool
     \end{examples}
\end{thetadescr}


\theta{vp120}

\begin{thetadescr}
\evitem  consider\index{consider}
\esitem
     \begin{examples}
        \example They considered him a fool
     \end{examples}
\end{thetadescr}


\newpage
\verbpattern{[synCLOSEDNPPROP\_EMPTY] }
\begin{vpattern}
\ruleitem TVerbpattern5,8
\csritem \mbox{}\\
     \begin{csr}
       predrel/NP
     \end{csr}
\remarksitem
\end{vpattern}

\theta{vp012}

\begin{thetadescr}
\evitem  seem$^{2}$ to\index{seem$^{2}$ to}
\esitem
     \begin{examples}
        \example He seemed a fool
     \end{examples}
\end{thetadescr}


\theta{vp123}

\begin{thetadescr}
\evitem ?
\esitem
     \begin{examples}
        \example ?
     \end{examples}
\end{thetadescr}


\newpage
\verbpattern{[synCLOSEDNPPROP\_PREPNP] }
\begin{vpattern}
\ruleitem TVerbpattern8,6
\csritem \mbox{}\\
     \begin{csr}
       predrel/NP & prepobjrel/PREPP
     \end{csr}
\remarksitem
\end{vpattern}

\theta{vp012}
\begin{thetadescr}
\evitem  seem$^{2}$ to\index{seem$^{2}$ to}
\esitem
     \begin{examples}
        \example He seemed a fool to me
     \end{examples}
\end{thetadescr}


\theta{vp123}

\begin{thetadescr}
\evitem ?
\esitem
     \begin{examples}
        \example ?
     \end{examples}
\end{thetadescr}


\newpage
\verbpattern{[synCLOSEDTOSENT] }
\begin{vpattern}
\ruleitem TVerbpattern1,12a
\csritem \mbox{}\\
     \begin{csr}
      complrel/SENTENCE\{toinf\}
     \end{csr}
\remarksitem
\end{vpattern}

\theta{vp010}

\begin{thetadescr}
\evitem  seem$^{1}$\index{seem$^{1}$}, appear\index{appear}
\esitem
     \begin{examples}
        \example He seems to be your friend
        \example He appears to want to leave
     \end{examples}
\end{thetadescr}


\theta{vp120}

\begin{thetadescr}
\evitem expect\index{expect}, want\index{want}, consider\index{consider}, trust\index
{trust}
\esitem
     \begin{examples}
        \example I expect you to come
        \example I wanted them to cooperate
        \example I consider him to be a fool
        \example You cannot trust him to do anything right
     \end{examples}
\end{thetadescr}


\newpage
\verbpattern{[synCLOSEDVERBPPROP] }
\begin{vpattern}
\ruleitem TVerbpattern1,10a
\csritem \mbox{}\\
     \begin{csr}
                & predrel/VERBP\\
      objrel/NP & predrel/VERBP
     \end{csr}
\remarksitem
\end{vpattern}

\theta{vp010}

\begin{thetadescr}
\evitem get\index{get}
\esitem
     \begin{examples}
        \example  He got trapped
     \end{examples}
\end{thetadescr}


\theta{vp120}

\begin{thetadescr}
\evitem  have\index{have}, see\index{see}
\esitem
     \begin{examples}
        \example  He had a house built
        \example  I had him looked after
        \example  I had the problem taken care of
        \example  ?I saw the man knocked down
     \end{examples}
\end{thetadescr}


\newpage
\verbpattern{[synDIRCLOSEDPREPPPROP] }
\begin{vpattern}
\ruleitem TVerbpattern1,8a1-8a2
\csritem \mbox{}\\
     \begin{csr}
      (1) objrel/NP & dirargrel/PREPP\{dir\}\\
      (2) objrel/NP & dirargrel/ADVP\{dir\}
     \end{csr}
\remarksitem
\end{vpattern}

\theta{vp010}

\begin{thetadescr}
\evitem ?
\esitem
     \begin{examples}
        \example  ?
     \end{examples}
\end{thetadescr}


\theta{vp120}

\begin{thetadescr}
\evitem ?see\index{see}, ?bring\index{bring}
\esitem
     \begin{examples}
        \example ?We saw them to the door
        \example ?We brought them home
     \end{examples}
\end{thetadescr}


\newpage
\verbpattern{[synDIROPENPREPPPROP]}
\begin{vpattern}
\ruleitem TVerbpattern1,8b1-8b2
\csritem \mbox{}\\
     \begin{csr}
     (1) dirargrel/PREPP\{dir\}\\
     (2) dirargrel/ADVP\{dir\}
     \end{csr}
\remarksitem
\end{vpattern}

\theta{vp120}

\begin{thetadescr}
\evitem  jump\index{jump}, swim\index{swim}
\esitem
     \begin{examples}
        \example  He jumped off the fence
        \example  ?He swam southwards
     \end{examples}
\end{thetadescr}


\newpage
\verbpattern{[synDONP\_DIROPENPREPPPROP]}
\begin{vpattern}
\ruleitem TVerbpattern5,5a/b1-5a/b2
\csritem \mbox{}\\
     \begin{csr}
     (1)           & dirargrel/PREPP\{dir\}\\
     (2)           & dirargrel/ADVP\{diradv\}\\
     (1) objrel/NP & dirargrel/PREPP\{dir\}\\
     (2) objrel/NP & dirargrel/ADVP\{diradv\}
     \end{csr}
\remarksitem
\end{vpattern}

\theta{vp012}

\begin{thetadescr}
\evitem  ?perhaps ergative {\em go\/} \index{go}
\esitem
     \begin{examples}
        \example ?He went home
     \end{examples}
\end{thetadescr}


\theta{vp123}

\begin{thetadescr}
\evitem  drive\index{drive}, send\index{send}
\esitem
     \begin{examples}
        \example  He drove the car into the garage
        \example  ?I sent the boy home
     \end{examples}
\end{thetadescr}


\newpage
\verbpattern{[synDONP\_EMPTY]}
\begin{vpattern}
\ruleitem TVerbpattern5,6
\csritem \mbox{}\\
     \begin{csr}
      objrel/NP 
     \end{csr}
\remarksitem In case of a choice between synEMPTY\_DONP (where EMPTY stands for 
IONP) and synDONP\_EMPTY (where EMPTY stands for PREPNP), the latter pattern 
should be chosen, to avoid needless ambiguities. 
\end{vpattern}

\theta{vp012}

\begin{thetadescr}
\evitem  ?
\esitem
     \begin{examples}
        \example ?
     \end{examples}
\end{thetadescr}


\theta{vp123}

\begin{thetadescr}
\evitem  tell to\index{tell to}, suggest to\index{suggest to}
\esitem
     \begin{examples}
        \example He told the whole story
        \example He suggested the following solution
     \end{examples}
\end{thetadescr}


\newpage
\verbpattern{[synDONP\_LOCOPENPREPPPROP] }
\begin{vpattern}
\ruleitem TVerbpattern5,4a/b1-4a/b2
\csritem \mbox{}\\
     \begin{csr}
       (1)        & locargrel/PREPP\{loc\}\\
       (2)        & locargrel/ADVP\{locadv\}\\
       (1) objrel/NP & locargrel/PREPP\{loc\}\\
       (2) objrel/NP & locargrel/ADVP\{locadv\}
     \end{csr}
\remarksitem
\end{vpattern}

\theta{vp012}

\begin{thetadescr}
\evitem  ?
\esitem
     \begin{examples}
        \example ?
     \end{examples}
\end{thetadescr}


\theta{vp123}

\begin{thetadescr}
\evitem  put\index{put}, ?meet\index{meet}
\esitem
     \begin{examples}
        \example I put the book on the table
        \example ?The cars met head-on
     \end{examples}
\end{thetadescr}


\newpage
\verbpattern{[synDONP\_OPENADJPPROP] }
\begin{vpattern}
\ruleitem TVerbpattern5,3   
\csritem \mbox{}\\
     \begin{csr}
               & predrel/ADJP\\
     objrel/NP & predrel/ADJP
     \end{csr}
\remarksitem
\end{vpattern}

\theta{vp012}

\begin{thetadescr}
\evitem ?
\esitem
     \begin{examples}
        \example ?
     \end{examples}
\end{thetadescr}


\theta{vp123}

\begin{thetadescr}
\evitem paint\index{paint}, turn\index{turn}
\esitem
     \begin{examples}
        \example He paints the door green
        \example He paints the door  (ADJP contains an EMPTY)
        \example The heat turned the grass brown
     \end{examples}
\end{thetadescr}


\newpage
\verbpattern{[synDONP\_OPENGERUND] }
\begin{vpattern}
\ruleitem \norule
\csritem \mbox{}\\
     \begin{csr}
               & complrel/SENTENCE\{open-ing\} \\
     objrel/NP & complrel/SENTENCE\{open-ing\} 
     \end{csr}
\remarksitem
\end{vpattern}

\theta{vp012}

\begin{thetadescr}
\evitem ?
\esitem
     \begin{examples}
        \example ?
     \end{examples}
\end{thetadescr}


\theta{vp123}

\begin{thetadescr}
\evitem catch\index{catch}
\esitem
     \begin{examples}
        \example I caught him lying
     \end{examples}
\end{thetadescr}


\newpage
\verbpattern{[synDONP\_OPENNPPROP] }
\begin{vpattern}
\ruleitem TVerbpattern5,2   
\csritem \mbox{}\\
     \begin{csr}
                & predrel/NP\\
      objrel/NP & predrel/NP
     \end{csr}

\remarksitem
\end{vpattern}

\theta{vp012}

\begin{thetadescr}
\evitem  ?
\esitem
     \begin{examples}
        \example   ?
     \end{examples}
\end{thetadescr}


\theta{vp123}

\begin{thetadescr}
\evitem  elect\index{elect}, call\index{call}
\esitem
     \begin{examples}
        \example They elected him President
        \example They called the boy Peter
     \end{examples}
\end{thetadescr}


\newpage
\verbpattern{[synDONP\_OPENTOSENT] }
\begin{vpattern}
\ruleitem TVerbpattern5,1   
\csritem \mbox{}\\
     \begin{csr}
       objrel/NP & complrel/SENTENCE\{toinf,decl\} 
     \end{csr}
\remarksitem
\end{vpattern}

\theta{vp012}

\begin{thetadescr}
\evitem ?
\esitem
     \begin{examples}
        \example  ?
     \end{examples}
\end{thetadescr}


\theta{vp123}

\begin{thetadescr}
\evitem  force\index{force}, name\index{name}
\esitem
     \begin{examples}
        \example We forced them to cooperate
        \example The President named him to be Secretary of State (OALD)
     \end{examples}
\end{thetadescr}


\newpage
\verbpattern{[synDONP\_OTHEROPENPREPPPROP] }
\begin{vpattern}
\ruleitem TVerbpattern5,10
\csritem \mbox{}\\
     \begin{csr}
       objrel/NP & predrel/PREPP\{other\} 
     \end{csr}
\remarksitem
\end{vpattern}

\theta{vp012}

\begin{thetadescr}
\evitem ?
\esitem
     \begin{examples}
        \example  ?
     \end{examples}
\end{thetadescr}


\theta{vp123}

\begin{thetadescr}
\evitem  tear\index{tear}
\esitem
     \begin{examples}
        \example He tore it to pieces
     \end{examples}
\end{thetadescr}


\newpage
\verbpattern{[synDONP\_PREPNP] }
\begin{vpattern}
\ruleitem TVerbpattern8,3
\csritem \mbox{}\\
     \begin{csr}
               & prepobjrel/PREPP[..objrel/NP..]\\
     objrel/NP & prepobjrel/PREPP[..objrel/NP..]
     \end{csr}
\remarksitem
This verbpattern is to be used only for arguments that have no obvious relation 
between each other. If a relation is apparent, the pattern is classified as 
DONP\_PREPOPENNPPROP (see there).
\end{vpattern}

\theta{vp012}

\begin{thetadescr}
\evitem ?turn into\index{turn into}
\esitem
     \begin{examples}
        \example ?He turned into a frog
     \end{examples}
\end{thetadescr}


\theta{vp123}

\begin{thetadescr}
\evitem give to\index{give to}
\esitem
     \begin{examples}
        \example I gave the book to John
     \end{examples}
\end{thetadescr}


\newpage
\verbpattern{[synDONP\_PREPOPENADJPPROP] }
\begin{vpattern}
\ruleitem TVerbpattern8,2
\csritem \mbox{}\\
     \begin{csr}
                & prepobjrel/PREPP[..objrel/NP..]\\
      objrel/NP & prepobjrel/PREPP[..objrel/NP..]
     \end{csr}
\remarksitem
\end{vpattern}

\theta{vp012}

\begin{thetadescr}
\evitem ?
\esitem
     \begin{examples}
        \example ?
     \end{examples}
\end{thetadescr}


\theta{vp123}

\begin{thetadescr}
\evitem change to\index{change to}, regard as\index{regard as}
\esitem
     \begin{examples}
        \example  We changed the main colour (from red) to brown
        \example  We regard her as clever.
     \end{examples}
\end{thetadescr}


\newpage
\verbpattern{[synDONP\_PREPOPENGERUND] }
\begin{vpattern}
\ruleitem TVerbpattern8,4
\csritem \mbox{}\\
     \begin{csr}
                & prepobjrel/PREPP[..objrel/NP..]\\
      objrel/NP & prepobjrel/PREPP[..objrel/NP..]
     \end{csr}
\remarksitem
\end{vpattern}

\theta{vp012}

\begin{thetadescr}
\evitem ?
\esitem
     \begin{examples}
        \example ?
     \end{examples}
\end{thetadescr}


\theta{vp123}

\begin{thetadescr}
\evitem talk out of\index{talk out of}, keep from\index{keep from}, prevent 
from\index{prevent from}, regard as\index{regard as}
\esitem
     \begin{examples}
        \example  They talked him out of jumping from the top of the building
        \example  Can't you keep him from forgetting?
        \example  You cannot prevent me from going there!
        \example  I regard him as being without principles.
     \end{examples}
\end{thetadescr}


\newpage
\verbpattern{[synDONP\_PREPOPENNPPROP] }
\begin{vpattern}
\ruleitem TVerbpattern8,1
\csritem \mbox{}\\
     \begin{csr}
                & prepobjrel/PREPP[..objrel/NP..]\\
      objrel/NP & prepobjrel/PREPP[..objrel/NP..]
     \end{csr}
\remarksitem See what was said under [synDONP\_PREPNP]
\end{vpattern}

\theta{vp012}

\begin{thetadescr}
\evitem ?turn into\index{turn into}
\esitem
     \begin{examples}
        \example ?He turned into a frog (unless this is a closed NPPROP, and
                 {\em turn\/} takes only one argument)
     \end{examples}
\end{thetadescr}


\theta{vp123}

\begin{thetadescr}
\evitem turn into\index{turn into}, 
name as\index{name as}, mistake for\index{mistake for}, 
name after\index{name after},
 charge on\index{charge on}, allow on\index{allow on}, ask for
\index{ask for}, want for\index{want for}, regard as\index{regard as}
\esitem
     \begin{examples}
        \example  We turned our backyard into a workshop
        \example  The President named him as Secretary of State
        \example  I mistook you for somebody else
        \example  They named him after his father
        \example  They charge a tax on imported bottles of wine
        \example  The bank allows 5\% (interest) on money kept with it
        \example  How much are you asking / do you want for that painting?
        \example  They regard her as a friend
     \end{examples}
\end{thetadescr}


\newpage
\verbpattern{[synDONP\_PREPOTHEROPENPREPPPROP] }
\begin{vpattern}
\ruleitem TVerbpattern8,8
\csritem \mbox{}\\
     \begin{csr}
       objrel/NP & prepobjrel/PREPP[..objrel/PREPP..] 
     \end{csr}
\remarksitem
\end{vpattern}

\theta{vp012}

\begin{thetadescr}
\evitem ?
\esitem
     \begin{examples}
        \example  ?
     \end{examples}
\end{thetadescr}


\theta{vp123}

\begin{thetadescr}
\evitem  regard as\index{regard as}
\esitem
     \begin{examples}
        \example I regard him as without principles
     \end{examples}
\end{thetadescr}


\newpage
\verbpattern{[synDONP\_PREPQSENT] }
\begin{vpattern}
\ruleitem \norule
\csritem \mbox{}\\
     \begin{csr}
       objrel/NP & prepobjrel/PREPP[..objrel/SENTENCE\{q\}..]
     \end{csr}
\remarksitem
\end{vpattern}

\theta{vp012}

\begin{thetadescr}
\evitem ?
\esitem
     \begin{examples}
        \example  ?
     \end{examples}
\end{thetadescr}


\theta{vp123}

\begin{thetadescr}
\evitem  instruct in\index{instruct in}
\esitem
     \begin{examples}
        \example They instructed him in how to do it
     \end{examples}
\end{thetadescr}


\newpage
\verbpattern{[synDONP\_PROSENT] }
\begin{vpattern}
\ruleitem \norule
\csritem \mbox{}\\
     \begin{csr}
       objrel/NP 
     \end{csr}
\remarksitem
\end{vpattern}

\theta{vp012}

\begin{thetadescr}
\evitem ?
\esitem
     \begin{examples}
        \example  ?
     \end{examples}
\end{thetadescr}


\theta{vp123}

\begin{thetadescr}
\evitem  force\index{force}, ?stop\index{stop}
\esitem
     \begin{examples}
        \example We forced them (Wij dwongen hen ertoe)
        \example ?She stopped me (Zij belette het mij)
     \end{examples}
\end{thetadescr}


\newpage
\verbpattern{[synDONP\_QSENT] }
\begin{vpattern}
\ruleitem \norule
\csritem \mbox{}\\
     \begin{csr}
       objrel/NP & complrel/SENTENCE\{q\} 
     \end{csr}
\remarksitem
\end{vpattern}

\theta{vp012}

\begin{thetadescr}
\evitem ?
\esitem
     \begin{examples}
        \example  ?
     \end{examples}
\end{thetadescr}


\theta{vp123}

\begin{thetadescr}
\evitem  instruct\index{instruct}
\esitem
     \begin{examples}
        \example They instructed him how to do it
     \end{examples}
\end{thetadescr}


\newpage
\verbpattern{[synDONP\_THATSENT] }
\begin{vpattern}
\ruleitem \norule
\csritem \mbox{}\\
     \begin{csr}
       objrel/NP & complrel/SENTENCE\{0-that\} \\
       objrel/NP & extraposrel/SENTENCE\{that\} 
     \end{csr}
\remarksitem
\end{vpattern}

\theta{vp012}

\begin{thetadescr}
\evitem ?
\esitem
     \begin{examples}
        \example  ?
     \end{examples}
\end{thetadescr}


\theta{vp123}

\begin{thetadescr}
\evitem  convince\index{convince}
\esitem
     \begin{examples}
        \example He convinced her that they couldn't afford it
     \end{examples}
\end{thetadescr}


\newpage
\verbpattern{[synEMPTY] }
\begin{vpattern}
\ruleitem TVerbpattern1,2   
\csritem \mbox{}\\
     \begin{csr}
     -- 
     \end{csr}
\remarksitem
\end{vpattern}

\theta{vp010}

\begin{thetadescr}
\evitem ?
\esitem
     \begin{examples}
        \example  ?
     \end{examples}
\end{thetadescr}


\theta{vp120}

\begin{thetadescr}
\evitem eat\index{eat}
\esitem
     \begin{examples}
        \example John eats
     \end{examples}
\end{thetadescr}


\newpage
\verbpattern{[synEMPTY\_CLOSEDTOSENT] }
\begin{vpattern}
\ruleitem \norule
\csritem \mbox{}\\
     \begin{csr}
       complrel/SENTENCE\{toinf\} 
     \end{csr}
\remarksitem The EMPTY in this pattern comes from a PREPNP. Perhaps the order 
of the two arguments is not correct.
\end{vpattern}

\theta{vp012}

\begin{thetadescr}
\evitem ?
\esitem
     \begin{examples}
        \example  ?
     \end{examples}
\end{thetadescr}


\theta{vp123}

\begin{thetadescr}
\evitem  report to\index{report to}
\esitem
     \begin{examples}
        \example He reported a star to have appeared
     \end{examples}
\end{thetadescr}


\newpage
\verbpattern{[synEMPTY\_DONP] }
\begin{vpattern}
\ruleitem TVerbpattern4,2   
\csritem \mbox{}\\
     \begin{csr}
     objrel/NP
     \end{csr}
\remarksitem This verbpattern might well not exist!! Cf. synDONP\_EMPTY. A verb 
that does not take a prepositional argument and hence cannot have 
synDONP\_
EMPTY is {\em allow\/}. Perhaps that is the only kind of verb needing 
this pattern.
\end{vpattern}

\theta{vp012}

\begin{thetadescr}
\evitem ?
\esitem
     \begin{examples}
        \example ?
     \end{examples}
\end{thetadescr}


\theta{vp123}

\begin{thetadescr}
\evitem ?allow\index{allow}
\esitem
     \begin{examples}
        \example ?I'll allow it this time
     \end{examples}
\end{thetadescr}


\newpage
\verbpattern{[synEMPTY\_MEASUREPHRASE] }
\begin{vpattern}
\ruleitem TVerbpattern4,10  
\csritem \mbox{}\\
     \begin{csr}
      predrel/NP
     \end{csr}
\remarksitem
\end{vpattern}

\theta{vp123}

\begin{thetadescr}
\evitem cost\index{cost}
\esitem
     \begin{examples}
        \example  This costs \$6
     \end{examples}
\end{thetadescr}


\newpage
\verbpattern{[synEMPTY\_OPENGERUND] }
\begin{vpattern}
\ruleitem TVerbpattern4,14
\csritem \mbox{}\\
     \begin{csr}
       complrel/SENTENCE\{open-ing\}
     \end{csr}
\remarksitem
\end{vpattern}

\theta{vp012}

\begin{thetadescr}
\evitem ?
\esitem
     \begin{examples}
        \example ?
     \end{examples}
\end{thetadescr}


\theta{vp123}

\begin{thetadescr}
\evitem allow\index{allow}
\esitem
     \begin{examples}
        \example  ?I allow smoking
     \end{examples}
\end{thetadescr}


\newpage
\verbpattern{[synEMPTY\_OPENTOSENT] }
\begin{vpattern}
\ruleitem TVerbpattern4,8   
\csritem \mbox{}\\
     \begin{csr}
       complrel/SENTENCE\{decl,toinf\}
     \end{csr}
\remarksitem
\end{vpattern}

\theta{vp012}

\begin{thetadescr}
\evitem ?
\esitem
     \begin{examples}
        \example ?
     \end{examples}
\end{thetadescr}


\theta{vp123}

\begin{thetadescr}
\evitem promise\index{promise}
\esitem
     \begin{examples}
        \example  I promised to come
     \end{examples}
\end{thetadescr}


\newpage
\verbpattern{[synEMPTY\_PREPNP] }
\begin{vpattern}
\ruleitem TVerbpattern10,1
\csritem \mbox{}\\
     \begin{csr}
       prepobjrel/PREPP
     \end{csr}
\remarksitem At the moment, there are two problems with this pattern. Firstly, 
it is wrongly used for verbs that should have synEMPTY\_PREP2NP (see below). 
This can be remedied easily, by checking the dictionaries on verbs that have a 
non-zero prepkey2. The pattern rule must be adapted so that it uses prepkey1 
for verbs with 
pattern synEMPTY\_PREPNP (or perhaps it is simpler if that pattern is dealt 
with by 
TVerbpattern8 or TVerbpattern9, depending on whether the EMPTY is direct or 
indirect object; see the following remark) and prepkey2 
for verbs with synEMPTY\_PREP2NP. 

The other problem is more difficult and concerns the order 
of the arguments in transfer: if the EMPTY is some sort of indirect object, it 
should be the third argument passed on by IL (see the startrules), 
but if it is a direct object, 
it will be the second argument. Whether this means that the pattern should be 
split up, or that the thetavp of the verbs differ between vp123 and vp132 is 
not clear to me at all. The mapping with Dutch is of course important, but has 
not always been checked.
\end{vpattern}

\theta{vp012}

\begin{thetadescr}
\evitem ?
\esitem
     \begin{examples}
        \example ?
     \end{examples}
\end{thetadescr}


\theta{vp123}

\begin{thetadescr}
\evitem refer to\index{refer to}, ask for\index{ask for}
\esitem
     \begin{examples}
        \example  We refer to the manager
        \example  We asked for a break
     \end{examples}
\end{thetadescr}


\newpage
\verbpattern{[synEMPTY\_PREP2NP] }
\begin{vpattern}
\ruleitem No rule has been written for this specific pattern, although 
TVerbpattern10 is the rule that will have to deal with it. It must simply 
change the name of the pattern in the CA-pairs. See the remarks made in the 
previous rule.
\csritem \mbox{}\\
     \begin{csr}
       prepobjrel/PREPP
     \end{csr}
\remarksitem
\end{vpattern}

\theta{vp012}

\begin{thetadescr}
\evitem ?
\esitem
     \begin{examples}
        \example ?
     \end{examples}
\end{thetadescr}


\theta{vp123}

\begin{thetadescr}
\evitem talk to about\index{talk to about}
\esitem
     \begin{examples}
        \example  I talked about a new car
     \end{examples}
\end{thetadescr}


\newpage
\verbpattern{[synEMPTY\_PREPOPENGERUND] }
\begin{vpattern}
\ruleitem \norule
\csritem \mbox{}\\
     \begin{csr}
       prepobjrel/PREPP[..objrel/NP\{open-ing\}..]
     \end{csr}
\remarksitem
\end{vpattern}

\theta{vp012}

\begin{thetadescr}
\evitem ?
\esitem
     \begin{examples}
        \example ?
     \end{examples}
\end{thetadescr}


\theta{vp123}

\begin{thetadescr}
\evitem advise against\index{advise against}
\esitem
     \begin{examples}
        \example  He advised against doing that.
     \end{examples}
\end{thetadescr}


\newpage
\verbpattern{[synEMPTY\_PROSENT] }
\begin{vpattern}
\ruleitem TVerbpattern4,13
\csritem \mbox{}\\
     \begin{csr}
       --
     \end{csr}
\remarksitem
\end{vpattern}

\theta{vp012}

\begin{thetadescr}
\evitem ?
\esitem
     \begin{examples}
        \example ?
     \end{examples}
\end{thetadescr}


\theta{vp123}

\begin{thetadescr}
\evitem ask\index{ask}, offer\index{offer}
\esitem
     \begin{examples}
        \example  I'll ask.
        \example You know that we offered.
     \end{examples}
\end{thetadescr}


\newpage
\verbpattern{[synEMPTY\_QSENT] }
\begin{vpattern}
\ruleitem TVerbpattern4,11
\csritem \mbox{}\\
     \begin{csr}
     complrel/SENTENCE\{q\}\\
     extraposrel/SENTENCE\{q\}
     \end{csr}
\remarksitem
\end{vpattern}

\theta{vp012}

\begin{thetadescr}
\evitem ?
\esitem
     \begin{examples}
        \example  ?
     \end{examples}
\end{thetadescr}

\theta{vp123}

\begin{thetadescr}
\evitem ask\index{ask}
\esitem
     \begin{examples}
        \example I asked yesterday why he wanted to leave so early
        \example I asked if he wanted tea
     \end{examples}
\end{thetadescr}


\newpage
\verbpattern{[synEMPTY\_THATSENT] }
\begin{vpattern}
\ruleitem TVerbpattern4,4
\csritem \mbox{}\\
     \begin{csr}
     extraposrel/SENTENCE\{that\}\\
     complrel/SENTENCE\{0-that\}
     \end{csr}
\remarksitem
\end{vpattern}

\theta{vp012}

\begin{thetadescr}
\evitem ? The verb seem$^{2}$ cannot be used with an empty, because than it 
overlaps with seem$^{1}$.
\esitem
     \begin{examples}
        \example  ?
     \end{examples}
\end{thetadescr}

\theta{vp123}

\begin{thetadescr}
\evitem promise\index{promise}
\esitem
     \begin{examples}
        \example We promised yesterday that we would be home early
        \example We promised we would stay
     \end{examples}
\end{thetadescr}


\newpage
\verbpattern{[synFORTOSENT] }
\begin{vpattern}
\ruleitem TVerbpattern1,13
\csritem \mbox{}\\
     \begin{csr}
      complrel/SENTENCE\{fortoinf\}
     \end{csr}
\remarksitem
\end{vpattern}

\theta{vp010}

\begin{thetadescr}
\evitem ?
\esitem
     \begin{examples}
        \example  ?
     \end{examples}
\end{thetadescr}


\theta{vp120}

\begin{thetadescr}
\evitem prefer\index{prefer}
\esitem
     \begin{examples}
        \example I prefer for John to go
     \end{examples}
\end{thetadescr}


\newpage
\verbpattern{[synFRONTSOPROSENT] }
\begin{vpattern}
\ruleitem \norule
\csritem \mbox{}\\
     \begin{csr}
      complrel/PROSENT\{so\}
     \end{csr}
\remarksitem
\end{vpattern}

\theta{vp010}

\begin{thetadescr}
\evitem ?
\esitem
     \begin{examples}
        \example  ?
     \end{examples}
\end{thetadescr}


\theta{vp120}

\begin{thetadescr}
\evitem indicate\index{indicate}
\esitem
     \begin{examples}
        \example So he indicated.
     \end{examples}
\end{thetadescr}


\newpage
\verbpattern{[synIONP\_DONP] }
\begin{vpattern}
\ruleitem TVerbpattern4,1
\csritem \mbox{}\\
     \begin{csr}
      indobjrel/NP & objrel/NP
     \end{csr}
\remarksitem
\end{vpattern}

\theta{vp012}

\begin{thetadescr}
\evitem ?
\esitem
     \begin{examples}
        \example ?
     \end{examples}
\end{thetadescr}


\theta{vp123}

\begin{thetadescr}
\evitem give\index{give}, tell\index{tell}, charge\index{charge}
\esitem
     \begin{examples}
        \example I gave him the book
        \example I told him a lie
        \example How much do you charge me (for that)?
     \end{examples}
\end{thetadescr}


\newpage
\verbpattern{[synIONP\_EMPTY] }
\begin{vpattern}
\ruleitem \norule
\csritem \mbox{}\\
     \begin{csr}
      --
     \end{csr}
\remarksitem Only found for one verb, which is vp132
\end{vpattern}

\theta{vp012}

\begin{thetadescr}
\evitem ?
\esitem
     \begin{examples}
        \example  ?
     \end{examples}
\end{thetadescr}


\theta{vp123}

\begin{thetadescr}
\evitem ?
\esitem
     \begin{examples}
        \example ?
     \end{examples}
\end{thetadescr}

\theta{vp132}

\begin{thetadescr}
\evitem envy\index{envy}
\esitem
     \begin{examples}
        \example I envy you
     \end{examples}
\end{thetadescr}


\newpage
\verbpattern{[synIONP\_MEASUREPHRASE] }
\begin{vpattern}
\ruleitem TVerbpattern4,9
\csritem \mbox{}\\
     \begin{csr}
      indobjrel/NP & predrel/NP
     \end{csr}
\remarksitem
\end{vpattern}

\theta{vp012}

\begin{thetadescr}
\evitem ?
\esitem
     \begin{examples}
        \example ?
     \end{examples}
\end{thetadescr}


\theta{vp123}

\begin{thetadescr}
\evitem cost\index{cost}
\esitem
     \begin{examples}
        \example This cost me \$6
     \end{examples}
\end{thetadescr}


\newpage
\verbpattern{ [synIONP\_OPENINFSENT] }
\begin{vpattern}
\ruleitem \norule
\csritem \mbox{}\\
     \begin{csr}
      indobjrel/NP & complrel/SENTENCE\{openinf\}
     \end{csr}
\remarksitem
\end{vpattern}

\theta{vp012}

\begin{thetadescr}
\evitem ?
\esitem
     \begin{examples}
        \example ?
     \end{examples}
\end{thetadescr}


\theta{vp123}

\begin{thetadescr}
\evitem help\index{help}
\esitem
     \begin{examples}
        \example I helped him clean the windows
     \end{examples}
\end{thetadescr}


\newpage
\verbpattern{ [synIONP\_OPENNPPROP] }
\begin{vpattern}
\ruleitem \norule
\csritem \mbox{}\\
     \begin{csr}
      indobjrel/NP & predrel/NP
     \end{csr}
\remarksitem
\end{vpattern}

\theta{vp012}

\begin{thetadescr}
\evitem ?
\esitem
     \begin{examples}
        \example ?
     \end{examples}
\end{thetadescr}


\theta{vp123}

\begin{thetadescr}
\evitem make\index{make}
\esitem
     \begin{examples}
        \example I'll make you a good wife, dear John
     \end{examples}
\end{thetadescr}


\newpage
\verbpattern{[synIONP\_OPENTOSENT] }
\begin{vpattern}
\ruleitem TVerbpattern4,7
\csritem \mbox{}\\
     \begin{csr}
      indobjrel/NP & complrel/SENTENCE\{toinf\}\\
      indobjrel/NP & extraposrel/SENTENCE\{toinf\}
     \end{csr}
\remarksitem This verbpattern is for both subject and object controllers. See 
the remarks on the order of the arguments in the introduction of this document.
\end{vpattern}

\theta{vp012}

\begin{thetadescr}
\evitem ?
\esitem
     \begin{examples}
        \example ?
     \end{examples}
\end{thetadescr}


\theta{vp123}

\begin{thetadescr}
\evitem promise\index{promise}, persuade\index{persuade}, allow\index{allow}, 
motion\index{motion}
\esitem
     \begin{examples}
        \example John persuaded Bill to see a doctor
        \example John promised Bill to see a doctor
        \example I allowed them to stay up late
        \example He motioned me to sit down
     \end{examples}
\end{thetadescr}


\newpage
\verbpattern{[synIONP\_PREPCLOSEDADJPPROP] }
\begin{vpattern}
\ruleitem TVerbpattern9,1
\csritem \mbox{}\\
     \begin{csr}
      indobjrel/NP & prepobjrel/PREPP[.. predrel/ADJP..]
     \end{csr}
\remarksitem
\end{vpattern}

\theta{vp012}

\begin{thetadescr}
\evitem ?strike as\index{strike as}
\esitem
     \begin{examples}
        \example ?He struck me as pompous
     \end{examples}
\end{thetadescr}


\theta{vp123}

\begin{thetadescr}
\evitem ?
\esitem
     \begin{examples}
        \example ?
     \end{examples}
\end{thetadescr}


\newpage
\verbpattern{[synIONP\_PREPNP] }
\begin{vpattern}
\ruleitem TVerbpattern9,2
\csritem \mbox{}\\
     \begin{csr}
      indobjrel/NP & prepobjrel/PREPP[...objrel/NP...]
     \end{csr}
\remarksitem Perhaps this pattern should not exist, and the examples be 
classified as DONP\_PREPNP. Mapping of that pattern to one of another language 
should not cause a problem with argument matching, however.
\end{vpattern}

\theta{vp012}

\begin{thetadescr}
\evitem ?
\esitem
     \begin{examples}
        \example ?
     \end{examples}
\end{thetadescr}


\theta{vp123}

\begin{thetadescr}
\evitem ask about\index{ask about}, tell about\index{tell about}, 
?provide with\index{provide with}
\esitem
     \begin{examples}
        \example Why don't you ask him about it?
        \example Why don't you tell me about it?
        \example ?I provided them with the relevant documents
     \end{examples}
\end{thetadescr}


\newpage
\verbpattern{[synIONP\_PREPOPENGERUND] }
\begin{vpattern}
\ruleitem \norule
\csritem \mbox{}\\
     \begin{csr}
      indobjrel/NP & prepobjrel/PREPP[...objrel/NP\{open-ing\}...]
     \end{csr}
\remarksitem 
\end{vpattern}

\theta{vp012}

\begin{thetadescr}
\evitem ?
\esitem
     \begin{examples}
        \example ?
     \end{examples}
\end{thetadescr}


\theta{vp123}

\begin{thetadescr}
\evitem reproach for\index{reproach for}
\esitem
     \begin{examples}
        \example I reproached him for being lazy
     \end{examples}
\end{thetadescr}


\newpage
\verbpattern{[synIONP\_PROSENT] }
\begin{vpattern}
\ruleitem TVerbpattern4,12
\csritem \mbox{}\\
     \begin{csr}
      indobjrel/NP 
     \end{csr}
\remarksitem
\end{vpattern}

\theta{vp012}

\begin{thetadescr}
\evitem ?
\esitem
     \begin{examples}
        \example ?
     \end{examples}
\end{thetadescr}


\theta{vp123}

\begin{thetadescr}
\evitem ask\index{ask}, tell\index{tell}
\esitem
     \begin{examples}
        \example I'll ask him (Ik zal het hem vragen)
        \example I told him (?translation)
     \end{examples}
\end{thetadescr}


\newpage
\verbpattern{[synIONP\_QSENT] }
\begin{vpattern}
\ruleitem TVerbpattern4,6
\csritem \mbox{}\\
     \begin{csr}
      indobjrel/NP & complrel/SENTENCE\{q\}
     \end{csr}
\remarksitem
\end{vpattern}

\theta{vp012}

\begin{thetadescr}
\evitem ?
\esitem
     \begin{examples}
        \example ?
     \end{examples}
\end{thetadescr}


\theta{vp123}

\begin{thetadescr}
\evitem tell\index{tell}, ask\index{ask}
\esitem
     \begin{examples}
        \example Tell me where to go
        \example I asked him if he wanted tea
     \end{examples}
\end{thetadescr}


\newpage
\verbpattern{[synIONP\_SOPROSENT] }
\begin{vpattern}
\ruleitem TVerbpattern4,5
\csritem \mbox{}\\
     \begin{csr}
      indobjrel/NP & complrel/PROSENT\{so\}
     \end{csr}
\remarksitem
\end{vpattern}

\theta{vp012}

\begin{thetadescr}
\evitem ?
\esitem
     \begin{examples}
        \example ?
     \end{examples}
\end{thetadescr}


\theta{vp123}

\begin{thetadescr}
\evitem tell\index{tell}
\esitem
     \begin{examples}
        \example I told him so
     \end{examples}
\end{thetadescr}


\newpage
\verbpattern{[synIONP\_THATSENT] }
\begin{vpattern}
\ruleitem TVerbpattern4,3
\csritem \mbox{}\\
     \begin{csr}
      indobjrel/NP & extraposrel/SENTENCE\{that\}
     \end{csr}
\remarksitem
\end{vpattern}

\theta{vp012}

\begin{thetadescr}
\evitem ?
\esitem
     \begin{examples}
        \example ?
     \end{examples}
\end{thetadescr}


\theta{vp123}

\begin{thetadescr}
\evitem tell\index{tell}
\esitem
     \begin{examples}
        \example He told me that he would be back early
     \end{examples}
\end{thetadescr}


\newpage
\verbpattern{[synITTHATSENT] }
\begin{vpattern}
\ruleitem TVerbpattern2
\csritem \mbox{}\\
     \begin{csr}
      objrel/NP & extraposrel/SENTENCE\{that\}
     \end{csr}
\remarksitem The example given here is the only example found so far.
Note that this verbpattern causes a factive reading for an otherwise 
non-factive verb.
\end{vpattern}

\theta{vp010}

\begin{thetadescr}
\evitem ?
\esitem
     \begin{examples}
        \example ?
     \end{examples}
\end{thetadescr}


\theta{vp120}

\begin{thetadescr}
\evitem believe\index{believe}
\esitem
     \begin{examples}
        \example I can't believe it that I've won the lottery\\
     \end{examples}
\end{thetadescr}


\newpage
\verbpattern{[synLOCCLOSEDPREPPPROP] }
\begin{vpattern}
\ruleitem TVerbpattern1,7a1-7a2   
\csritem \mbox{}\\
     \begin{csr}
        (1)          & locargrel/PREPP\{loc\}\\
        (2)          & locargrel/ADVP\{locadv\}\\
        (1) objrel/NP & locargrel/PREPP\{loc\}\\
        (2) objrel/NP & locargrel/ADVP\{locadv\}
     \end{csr}
\remarksitem
\end{vpattern}

\theta{vp010}

\begin{thetadescr}
\evitem ?
\esitem
     \begin{examples}
        \example ?
     \end{examples}
\end{thetadescr}


\theta{vp120}

\begin{thetadescr}
\evitem ?get\index{get}, ?wish\index{wish}
\esitem
     \begin{examples}
        \example ?I got the two sides round the table
        \example ?I got them home
        \example ?I wished her on the moon
        \example ?I wished her (far) away
     \end{examples}
\end{thetadescr}


\newpage
\verbpattern{[synLOCOPENPREPPPROP] }
\begin{vpattern}
\ruleitem TVerbpattern1,7b1-7b2
\csritem \mbox{}\\
     \begin{csr}
      locargrel/PREPP\{loc\}\\
      locargrel/ADVP\{locadv\}
     \end{csr}
\remarksitem
\end{vpattern}

\theta{vp120}

\begin{thetadescr}
\evitem stand\index{stand}, remain\index{remain}, live\index{live}
\esitem
     \begin{examples}
        \example He stood behind the door
        \example He remained in bed
        \example ?He lives southwards
     \end{examples}
\end{thetadescr}


\newpage
\verbpattern{[synMEASUREPHRASE]}
\begin{vpattern}
\ruleitem TVerbpattern1,5
\csritem \mbox{}\\
     \begin{csr}
     predrel/NP
     \end{csr}
\remarksitem The rule assumes that canonical Measurephrases are OPENNPPROPs.

\end{vpattern}

\theta{vp120}

\begin{thetadescr}
\evitem weigh\index{weigh}
\esitem
     \begin{examples}
        \example This weighs a stone
     \end{examples}
\end{thetadescr}


\newpage
\verbpattern{[synNOTPROSENT]}
\begin{vpattern}
\ruleitem \norule
\csritem \mbox{}\\
     \begin{csr}
       Not determined yet
     \end{csr}
\remarksitem It has not been decided how to deal with this pattern, because its 
mapping to other languages seems hard.
\end{vpattern}



\theta{vp010}

\begin{thetadescr}
\evitem ?
\esitem
     \begin{examples}
        \example ?
     \end{examples}
\end{thetadescr}


\theta{vp120}

\begin{thetadescr}
\evitem think\index{think}
\esitem
     \begin{examples}
        \example (Did she bring her umbrella?) I think not
     \end{examples}
\end{thetadescr}


\newpage
\verbpattern{[synNoVpArgs]}
\begin{vpattern}
\ruleitem TVerbpattern0     
\csritem \mbox{}\\
     \begin{csr}
     --
     \end{csr}
\remarksitem
\end{vpattern}

\theta{vp000}

\begin{thetadescr}
\evitem rain\index{rain} (snow, thaw, pour, teem, pelt, drizzle, mizzle)
\esitem
     \begin{examples}
        \example It's raining
     \end{examples}
\end{thetadescr}



\theta{vp100}

\begin{thetadescr}
\evitem sleep\index{sleep}
\esitem
     \begin{examples}
        \example John is sleeping
     \end{examples}
\end{thetadescr}


\newpage
\verbpattern{[synNP] }
\begin{vpattern}
\ruleitem TVerbpattern1,1   
\csritem \mbox{}\\
     \begin{csr}
     objrel/NP\\
     --
     \end{csr}
\remarksitem
\end{vpattern}

\theta{vp010}

\begin{thetadescr}
\evitem  melt\index{melt}
\esitem
     \begin{examples}
        \example The ice melts
     \end{examples}
\end{thetadescr}


\theta{vp120}

\begin{thetadescr}
\evitem beat\index{beat}, regret\index{regret}
\esitem
     \begin{examples}
        \example Mary beats John
        \example I regret his coming home late
     \end{examples}
\end{thetadescr}


\newpage
\verbpattern{[synOPENADJPPROP] }
\begin{vpattern}
\ruleitem TVerbpattern1,4b
\csritem \mbox{}\\
     \begin{csr}
     predrel/ADJP
     \end{csr}
\remarksitem
\end{vpattern}

\theta{vp120}

\begin{thetadescr}
\evitem ?smell\index{smell}
\esitem
     \begin{examples}
        \example ?This book smells old\\
                 (Unless this is an ergative - or a middle ... )
     \end{examples}
\end{thetadescr}


\newpage
\verbpattern{[synOPENGERUND]}
\begin{vpattern}
\ruleitem TVerbpattern1,16b
\csritem \mbox{}\\
     \begin{csr}
     complrel/SENTENCE\{opening\}
     \end{csr}
\remarksitem
\end{vpattern}

\theta{vp120}

\begin{thetadescr}
\evitem burst out\index{burst out}
\esitem
     \begin{examples}
        \example He burst out singing
     \end{examples}
\end{thetadescr}


\newpage
\verbpattern{[synOPENGERUND\_PREPNP]}
\begin{vpattern}
\ruleitem \norule
\csritem \mbox{}\\
     \begin{csr}
     complrel/SENTENCE\{opening\} & prepobjrel/PREPP
     \end{csr}
\remarksitem
\end{vpattern}

\theta{vp012}

\begin{thetadescr}
\evitem ?
\esitem
     \begin{examples}
        \example ?
     \end{examples}
\end{thetadescr}

\theta{vp123}

\begin{thetadescr}
\evitem leave to\index{leave to}
\esitem
     \begin{examples}
        \example I'll leave cleaning up to you
     \end{examples}
\end{thetadescr}


\newpage
\verbpattern{[synOPENINFSENT]}
\begin{vpattern}
\ruleitem TVerbpattern1,11b
\csritem \mbox{}\\
     \begin{csr}
      pruning 
     \end{csr}
\remarksitem
\end{vpattern}

\theta{vp120}

\begin{thetadescr}
\evitem can$^{2}$\index{can$^{2}$} (ability)
\esitem
     \begin{examples}
        \example He can come
     \end{examples}
\end{thetadescr}


\newpage
\verbpattern{[synOPENNPPROP]}
\begin{vpattern}
\ruleitem TVerbpattern1,3b  
\csritem \mbox{}\\
     \begin{csr}
     predrel/NP
     \end{csr}
\remarksitem
\end{vpattern}

\theta{vp120}

\begin{thetadescr}
\evitem ?act\index{act}
\esitem
     \begin{examples}
        \example ?He is always acting the experienced man
     \end{examples}
\end{thetadescr}


\newpage
\verbpattern{[synOPENTOSENT]}
\begin{vpattern}
\ruleitem TVerbpattern1,12b
\csritem \mbox{}\\
     \begin{csr}
     complrel/SENTENCE\{toinf\}\\
     extraposrel/SENTENCE\{toinf\}
     \end{csr}
\remarksitem
\end{vpattern}

\theta{vp120}

\begin{thetadescr}
\evitem decide\index{decide}, prefer\index{prefer}, get\index{get}
\esitem
     \begin{examples}
        \example We decided to leave
        \example I prefer to go
        \example You're getting to be a bad influence on my children
     \end{examples}
\end{thetadescr}


\newpage
\verbpattern{[synOPENVERBPPROP]}
\begin{vpattern}
\ruleitem TVerbpattern1,10b
\csritem \mbox{}\\
     \begin{csr}
     ?
     \end{csr}
\remarksitem
\end{vpattern}

\theta{vp120}

\begin{thetadescr}
\evitem ?
\esitem
     \begin{examples}
        \example ?
     \end{examples}
\end{thetadescr}


\newpage
\verbpattern{[synOTHERCLOSEDPREPPPROP]}
\begin{vpattern}
\ruleitem TVerbpattern1,9a1-9a2
\csritem \mbox{}\\
     \begin{csr}
      (1)         & predrel/PREPP\{other\}\\
      (2)         & predrel/ADVP\{other\}\\
      (1) objrel/NP & predrel/PREPP\{other\}\\
      (2) objrel/NP & predrel/ADVP\{other\}
     \end{csr}
\remarksitem
\end{vpattern}

\theta{vp010}

\begin{thetadescr}
\evitem  ?seem$^{1}$\index{seem$^{1}$}
\esitem
     \begin{examples}
        \example  ?She seemed with child
     \end{examples}
\end{thetadescr}


\theta{vp120}

\begin{thetadescr}
\evitem ?
\esitem
     \begin{examples}
        \example ?
     \end{examples}
\end{thetadescr}


\newpage
\verbpattern{[synOTHERCLOSEDPREPPPROP\_EMPTY]}
\begin{vpattern}
\ruleitem TVerbpattern5,9
\csritem \mbox{}\\
     \begin{csr}
       predrel/PREPP\{other\} 
     \end{csr}
\remarksitem
\end{vpattern}

\theta{vp012}

\begin{thetadescr}
\evitem  ?seem$^{2}$ to\index{seem$^{2}$ to}
\esitem
     \begin{examples}
        \example  ?She seemed with child
     \end{examples}
\end{thetadescr}


\theta{vp123}

\begin{thetadescr}
\evitem ?
\esitem
     \begin{examples}
        \example ?
     \end{examples}
\end{thetadescr}


\newpage
\verbpattern{[synOTHERCLOSEDPREPPPROP\_PREPNP]}
\begin{vpattern}
\ruleitem TVerbpattern8,7
\csritem \mbox{}\\
     \begin{csr}
      (1) predrel/PREPP\{other\} & prepobjrel/PREPP\\
      (2) predrel/ADVP\{other\} & prepobjrel/PREPP
     \end{csr}
\remarksitem
\end{vpattern}

\theta{vp012}

\begin{thetadescr}
\evitem  ?seem$^{2}$ to\index{seem$^{2}$ to}
\esitem
     \begin{examples}
        \example  ?She seemed with child to me
     \end{examples}
\end{thetadescr}


\theta{vp123}

\begin{thetadescr}
\evitem ?
\esitem
     \begin{examples}
        \example ?
     \end{examples}
\end{thetadescr}


\newpage
\verbpattern{[synOTHEROPENPREPPPROP]}
\begin{vpattern}
\ruleitem TVerbpattern1,9b1-9b2
\csritem \mbox{}\\
     \begin{csr}
      (1)         & predrel/PREPP\{other\}\\
      (2)         & predrel/ADVP\{other\}\\
      (1) objrel/NP & predrel/PREPP\{other\}\\
      (2) objrel/NP & predrel/ADVP\{other\}
     \end{csr}
\remarksitem
\end{vpattern}

\theta{vp120}

\begin{thetadescr}
\evitem ?act\index{act}
\esitem
     \begin{examples}
        \example ?He acted crazily
     \end{examples}
\end{thetadescr}


\newpage
\verbpattern{[synPREPCLOSEDADJPPROP] }
\begin{vpattern}
\ruleitem TVerbpattern3,5
\csritem \mbox{}\\
     \begin{csr}
    objrel/NP & prepobjrel/PREPP[...predrel/ADJP...]
     \end{csr}
\remarksitem If this rule is used, a special rule must be written for the 
preposition {\em as\/} to put it back into its canonical place. (The rule was 
originally for `regard as', which is now considered a 3-place verb.)
\end{vpattern}

\theta{vp010}

\begin{thetadescr}
\evitem ?
\esitem
     \begin{examples}
        \example ?
     \end{examples}
\end{thetadescr}

\theta{vp120}

\begin{thetadescr}
\evitem ?
\esitem
     \begin{examples}
        \example ?
     \end{examples}
\end{thetadescr}


\newpage
\verbpattern{[synPREPCLOSEDGERUND] }
\begin{vpattern}
\ruleitem TVerbpattern3,3a
\csritem \mbox{}\\
     \begin{csr}
       prepobjrel/PREPP[...objrel/SENTENCE\{accing\}...]
     \end{csr}
\remarksitem In case this rule is used for the preposition {\em as\/}, a special 
rule must be written to put it back into its canonical place.
\end{vpattern}

\theta{vp010}

\begin{thetadescr}
\evitem ?
\esitem
     \begin{examples}
        \example ?
     \end{examples}
\end{thetadescr}


\theta{vp120}

\begin{thetadescr}
\evitem count on\index{count on}
\esitem
     \begin{examples}
        \example You can't count on the weather being fine
     \end{examples}
\end{thetadescr}


\newpage
\verbpattern{[synPREPCLOSEDNPPROP]}
\begin{vpattern}
\ruleitem TVerbpattern3,2a
\csritem \mbox{}\\
     \begin{csr}
                & prepobjrel/PREPP[..objrel/NP]\\
      objrel/NP & prepobjrel/PREPP[..objrel/NP]
     \end{csr}
\remarksitem
\end{vpattern}

\theta{vp010}

\begin{thetadescr}
\evitem ?
\esitem
     \begin{examples}
        \example ?
     \end{examples}
\end{thetadescr}

\theta{vp120}

\begin{thetadescr}
\evitem ?
\esitem
     \begin{examples}
        \example ?
     \end{examples}
\end{thetadescr}


\newpage
\verbpattern{[synPREPCLOSEDTOSENT]}
\begin{vpattern}
\ruleitem TVerbpattern3,4
\csritem \mbox{}\\
     \begin{csr}
     prepobjrel/PREPP[..objrel/SENTENCE\{toinf\}..]
     \end{csr}
\remarksitem
\end{vpattern}

\theta{vp010}

\begin{thetadescr}
\evitem ?
\esitem
     \begin{examples}
        \example ?
     \end{examples}
\end{thetadescr}


\theta{vp120}

\begin{thetadescr}
\evitem count on\index{count on}, wait for\index{wait for}
\esitem
     \begin{examples}
        \example I count on you to help
        \example I waited for the President to speak
     \end{examples}
\end{thetadescr}


\newpage
\verbpattern{[synPREPMEASUREPHRASE]}
\begin{vpattern}
\ruleitem \norule
\csritem \mbox{}\\
     \begin{csr}
     prepobjrel/PREPP[..objrel/OPENNPPROP..]
     \end{csr}
\remarksitem
\end{vpattern}


\theta{vp120}

\begin{thetadescr}
\evitem amount to\index{amount to}
\esitem
     \begin{examples}
        \example This amounts to \$6
     \end{examples}
\end{thetadescr}


\newpage
\verbpattern{[synPREPNP]}
\begin{vpattern}
\ruleitem TVerbpattern3,1   
\csritem \mbox{}\\
     \begin{csr}
      prepobjrel/PREPP[..objrel/NP..]
     \end{csr}
\remarksitem
\end{vpattern}

\theta{vp010}

\begin{thetadescr}
\evitem ?
\esitem
     \begin{examples}
        \example ?
     \end{examples}
\end{thetadescr}


\theta{vp120}

\begin{thetadescr}
\evitem count on\index{count on}
\esitem
     \begin{examples}
        \example I count on John
        \example I count on his finishing the job
     \end{examples}
\end{thetadescr}


\newpage
\verbpattern{[synPREPNP\_CLOSEDTOSENT]}
\begin{vpattern}
\ruleitem TVerbpattern6,5
\csritem \mbox{}\\
     \begin{csr}
      prepobjrel/PREPP & complrel/SENTENCE\{toinf\}
     \end{csr}
\remarksitem See the remark under synCLOSEDADJPPROP\_PREPNP for the order of 
the two arguments.
\end{vpattern}

\theta{vp012}

\begin{thetadescr}
\evitem seem$^{2}$ to\index{seem$^{2}$ to}
\esitem
     \begin{examples}
        \example  He seemed to her to be ill
     \end{examples}
\end{thetadescr}


\theta{vp123}

\begin{thetadescr}
\evitem ?
\esitem
     \begin{examples}
        \example ?
     \end{examples}
\end{thetadescr}


\newpage
\verbpattern{[synPREPNP\_EMPTY]}
\begin{vpattern}
\ruleitem TVerbpattern6,4
\csritem \mbox{}\\
     \begin{csr}
      prepobjrel/PREPP 
     \end{csr}
\remarksitem 
\end{vpattern}

\theta{vp012}

\begin{thetadescr}
\evitem ?
\esitem
     \begin{examples}
        \example  ?
     \end{examples}
\end{thetadescr}


\theta{vp123}

\begin{thetadescr}
\evitem talk to about\index{talk to about}
\esitem
     \begin{examples}
        \example I talked to my father
     \end{examples}
\end{thetadescr}


\newpage
\verbpattern{[synPREPNP\_ITOPENTOSENT]}
\begin{vpattern}
\ruleitem \norule
\csritem \mbox{}\\
     \begin{csr}
      objrel/NP\{it\} & prepobjrel/PREPP & extraposrel/SENTENCE\{toinf\}
     \end{csr}
\remarksitem The pattern can be used to indicate that this infinite sentence 
has obligatory extraposition. Note that the extraposition rule cannot deal with 
infinite sentences yet, nor does it introduce {\em it\/}!
\end{vpattern}

\theta{vp012}

\begin{thetadescr}
\evitem ?
\esitem
     \begin{examples}
        \example ?
     \end{examples}
\end{thetadescr}


\theta{vp123}

\begin{thetadescr}
\evitem leave to\index{leave to}
\esitem
     \begin{examples}
        \example I'll leave it to you to clean up the mess
     \end{examples}
\end{thetadescr}


\newpage
\verbpattern{[synPREPNP\_OPENTOSENT]}
\begin{vpattern}
\ruleitem TVerbpattern6,3
\csritem \mbox{}\\
     \begin{csr}
      prepobjrel/PREPP & complrel/SENTENCE\{toinf\}
     \end{csr}
\remarksitem
\end{vpattern}

\theta{vp012}

\begin{thetadescr}
\evitem ?
\esitem
     \begin{examples}
        \example  ?
     \end{examples}
\end{thetadescr}


\theta{vp123}

\begin{thetadescr}
\evitem ?sign to\index{sign to}
\esitem
     \begin{examples}
        \example ?The policeman signed to me to stop\\
        Cf.\ a version with {\em for\/} me to stop: this could be a 
        PREPCLOSEDTOSENT, or even a FORTOSENT...
     \end{examples}
\end{thetadescr}


\newpage
\verbpattern{[synPREPNP\_PREPOPENGERUND]}
\begin{vpattern}
\ruleitem \norule
\csritem \mbox{}\\
     \begin{csr}
      prepobjrel/PREPP & prepobjrel/PREPP[..objrel/NP\{opening\}..]
     \end{csr}
\remarksitem 
\end{vpattern}

\theta{vp012}

\begin{thetadescr}
\evitem ?
\esitem
     \begin{examples}
        \example  ?
     \end{examples}
\end{thetadescr}


\theta{vp123}

\begin{thetadescr}
\evitem agree with on\index{agree with on}
\esitem
     \begin{examples}
        \example They agreed with him (up)on selling the house first
     \end{examples}
\end{thetadescr}


\newpage
\verbpattern{[synPREPNP\_PREPNP]}
\begin{vpattern}
\ruleitem TVerbpattern7,1
\csritem \mbox{}\\
     \begin{csr}
     prepobjrel/PREPP & prepobjrel/PREPP
     \end{csr}
\remarksitem It is assumed that the `indirect object' is always the first 
argument, and is combined with prepkey1 of the verb. Perhaps a rule should be 
written to allow thetapattern 132 for verbs that seem to have their
preparguments in a reversed order.\\
It is also assumed that 
`to' is just an ordinary preposition, allowing a prepobjrel (no special 
toobjrel). \end{vpattern}

\theta{vp012}

\begin{thetadescr}
\evitem ?
\esitem
     \begin{examples}
        \example  ?
     \end{examples}
\end{thetadescr}


\theta{vp123}

\begin{thetadescr}
\evitem talk to about\index{talk to about}, talk with about\index{talk with 
about}
\esitem
     \begin{examples}
        \example I talked to my father about a new car
        \example I talked with my friends about our new house 
     \end{examples}
\end{thetadescr}


\newpage
\verbpattern{[synPREPNP\_QSENT]}
\begin{vpattern}
\ruleitem TVerbpattern6,2
\csritem \mbox{}\\
     \begin{csr}
     prepobjrel/PREPP & complrel/SENTENCE\{q\}\\
     prepobjrel/PREPP & extraposrel/SENTENCE\{q\}
     \end{csr}
\remarksitem
\end{vpattern}

\theta{vp012}

\begin{thetadescr}
\evitem ?
\esitem
     \begin{examples}
        \example  ?
     \end{examples}
\end{thetadescr}


\theta{vp123}

\begin{thetadescr}
\evitem inquire of\index{inquire of}
\esitem
     \begin{examples}
        \example I inquired of him what he wanted
     \end{examples}
\end{thetadescr}


\newpage
\verbpattern{[synPREPNP\_THATSENT]}
\begin{vpattern}
\ruleitem TVerbpattern6,1
\csritem \mbox{}\\
     \begin{csr}
     prepobjrel/PREPP & complrel/SENTENCE\{0-that\}\\
     prepobjrel/PREPP & extraposrel/SENTENCE\{that\}
     \end{csr}
\remarksitem For the order of the arguments, see the remarks under 
synCLOSED\-ADJP\-PROP\_PREPNP.
\end{vpattern}

\theta{vp012}

\begin{thetadescr}
\evitem seem$^{2}$ to\index{seem$^{2}$ to}
\esitem
     \begin{examples}
        \example It seems to me that he is ill
     \end{examples}
\end{thetadescr}


\theta{vp123}

\begin{thetadescr}
\evitem mention to\index{mention to}
\esitem
     \begin{examples}
        \example He mentioned to me that he would be back early
     \end{examples}
\end{thetadescr}


\newpage
\verbpattern{[synPREPOPENGERUND]}
\begin{vpattern}
\ruleitem TVerbpattern3,3b   
\csritem \mbox{}\\
     \begin{csr}
     prepobjrel/PREPP[..objrel/NP\{opening\}..]
     \end{csr}
\remarksitem 
\end{vpattern}

\theta{vp120}

\begin{thetadescr}
\evitem hesitate about\index{hesitate about}, succeed in\index{succeed in}
\esitem
     \begin{examples}
        \example He is still hesitating about joining that expedition.
        \example He succeeded in running the marathon in 3 hours flat.
     \end{examples}
\end{thetadescr}


\newpage
\verbpattern{[synPREPOPENNPPROP]}
\begin{vpattern}
\ruleitem TVerbpattern3,2b   
\csritem \mbox{}\\
     \begin{csr}
     prepobjrel/PREPP[..objrel/NP..]
     \end{csr}
\remarksitem 
\end{vpattern}

\theta{vp120}

\begin{thetadescr}
\evitem come across as\index{come across as}, serve as\index{serve as}
\esitem
     \begin{examples}
        \example He comes across as a nice person\\
                 (unless this is an ergative)
        \example She served as a scapegoat
     \end{examples}
\end{thetadescr}


\newpage
\verbpattern{[synPREPOTHERCLOSEDPREPPPROP]}
\begin{vpattern}
\ruleitem TVerbpattern3,7
\csritem \mbox{}\\
     \begin{csr}
     objrel/PREPP & prepobjrel/PREPP[ ..objrel/PREPP..]
     \end{csr}
\remarksitem 
\end{vpattern}

\theta{vp010}

\begin{thetadescr}
\evitem ?
\esitem
     \begin{examples}
        \example ?
     \end{examples}
\end{thetadescr}


\theta{vp120}

\begin{thetadescr}
\evitem ?
\esitem
     \begin{examples}
        \example ?
     \end{examples}
\end{thetadescr}


\newpage
\verbpattern{[synPREPQSENT]}
\begin{vpattern}
\ruleitem TVerbpattern3,6
\csritem \mbox{}\\
     \begin{csr}
      prepobjrel/PREPP[..objrel/SENTENCE\{q\}..]\\
      prepobjrel/PREPP & extraposrel/SENTENCE\{q\}
     \end{csr}
\remarksitem Note that the extraposition rules only work for verbs having the 
pattern synITTHATSENT now! Extraposition of other SENTENCEs from a PREPP is not 
accounted for yet.
\end{vpattern}

\theta{vp010}

\begin{thetadescr}
\evitem ?
\esitem
     \begin{examples}
        \example  ?
     \end{examples}
\end{thetadescr}


\theta{vp120}

\begin{thetadescr}
\evitem ?talk about\index{talk about}, hesitate about\index{hesitate about}, 
hesitate over\index{hesitate over}
\esitem
     \begin{examples}
        \example We all talked about why he would have murdered her
        \example We all talked about whether we should leave early\\
        (Unless of course `talk' is always a three place verb and needs an 
         EMPTY here; the pattern synEMPTY\_PREPQSENT does not exist yet!)
        \example He hesitated about what to do next
        \example He is still hesitating over whether to join the expedition
     \end{examples}
\end{thetadescr}


\newpage
\verbpattern{[synPREPTHATSENT]}
\begin{vpattern}
\ruleitem TVerbpattern3,8
\csritem \mbox{}\\
     \begin{csr}
     prepobjrel/PREPP & ..extraposrel/SENTENCE\{that\}
     \end{csr}
\remarksitem 
\end{vpattern}

\theta{vp120}

\begin{thetadescr}
\evitem count on\index{count on}
\esitem
     \begin{examples}
        \example I count on it that you'll be there
     \end{examples}
\end{thetadescr}


\newpage
\verbpattern{[synPROSENT] }
\begin{vpattern}
\ruleitem TVerbpattern1,17
\csritem \mbox{}\\
     \begin{csr}
      --
     \end{csr}
\remarksitem
\end{vpattern}


\theta{vp120}

\begin{thetadescr}
\evitem wonder\index{wonder}, ?refuse\index{refuse}, know\index{know}, 
hear\index{hear}, understand\index{understand}
\esitem
     \begin{examples}
        \example I wonder
        \example He refused  (although Dutch also has no argument there: 
                 Hij weigerde )
        \example She knew all along
        \example We've just heard (NOT in Longman)
        \example We understand, dear.
     \end{examples}
\end{thetadescr}


\newpage
\verbpattern{[synQSENT]}
\begin{vpattern}
\ruleitem TVerbpattern1,15
\csritem \mbox{}\\
     \begin{csr}
      complrel/SENTENCE\{q\}\\
      extraposrel/SENTENCE\{q\}
     \end{csr}
\remarksitem
\end{vpattern}

\theta{vp010}

\begin{thetadescr}
\evitem ?
\esitem
     \begin{examples}
        \example ?
     \end{examples}
\end{thetadescr}


\theta{vp120}

\begin{thetadescr}
\evitem wonder\index{wonder}, hesitate\index{hesitate}
\esitem
     \begin{examples}
        \example I wonder where to go
        \example I wonder whether I should leave
        \example He hesitated what to do next
     \end{examples}
\end{thetadescr}


\newpage
\verbpattern{[synSOPROSENT]}
\begin{vpattern}
\ruleitem TVerbpattern1,18
\csritem \mbox{}\\
     \begin{csr}
       complrel/PROSENT\{so\}
     \end{csr}
\remarksitem
\end{vpattern}


\theta{vp120}

\begin{thetadescr}
\evitem think\index{think}
\esitem
     \begin{examples}
        \example I think so
     \end{examples}
\end{thetadescr}


\newpage
\verbpattern{[synSOPROSENT\_EMPTY]}
\begin{vpattern}
\ruleitem \norule
\csritem \mbox{}\\
     \begin{csr}
       complrel/PROSENT\{so\}
     \end{csr}
\remarksitem
\end{vpattern}

\theta{vp012}

\begin{thetadescr}
\evitem ?
\esitem
     \begin{examples}
        \example  ?
     \end{examples}
\end{thetadescr}


\theta{vp123}

\begin{thetadescr}
\evitem say to\index{say to}
\esitem
     \begin{examples}
        \example I said so
     \end{examples}
\end{thetadescr}


\newpage
\verbpattern{[synSOPROSENT\_PREPNP]}
\begin{vpattern}
\ruleitem \norule
\csritem \mbox{}\\
     \begin{csr}
       complrel/PROSENT\{so\} & prepobjrel/PREPP
     \end{csr}
\remarksitem
\end{vpattern}

\theta{vp012}

\begin{thetadescr}
\evitem ?
\esitem
     \begin{examples}
        \example  ?
     \end{examples}
\end{thetadescr}


\theta{vp123}

\begin{thetadescr}
\evitem say to\index{say to}
\esitem
     \begin{examples}
        \example I said so to my father
     \end{examples}
\end{thetadescr}


\newpage
\verbpattern{[synTHATSENT]}
\begin{vpattern}
\ruleitem TVerbpattern1,14
\csritem \mbox{}\\
     \begin{csr}
     complrel/SENTENCE\{0-that\}\\
     extraposrel/SENTENCE\{that\}
     \end{csr}
\remarksitem
\end{vpattern}

\theta{vp010}

\begin{thetadescr}
\evitem seem$^{1}$\index{seem$^{1}$}
\esitem
     \begin{examples}
        \example  It seems that John comes
     \end{examples}
\end{thetadescr}

\theta{vp120}

\begin{thetadescr}
\evitem know\index{know}, hold\index{hold}
\esitem
     \begin{examples}
        \example  I know that he will come
        \example  I hold that he is a fool
     \end{examples}
\end{thetadescr}


\newpage
\verbpattern{[synTHATSENT\_EMPTY]}
\begin{vpattern}
\ruleitem \norule
\csritem \mbox{}\\
     \begin{csr}
     extraposrel/SENTENCE\{that\}
     \end{csr}
\remarksitem It has not been accounted for yet that extraposition is obligatory. 
Perhaps this pattern should be analysed differently. Also note that in the 
current  pattern {\em say\/} is vp123, while in the next pattern, 
synTHATSENT\_LOCOPENPREPPPROP, it would be vp012.
\end{vpattern}

\theta{vp012}

\begin{thetadescr}
\evitem ?
\esitem
     \begin{examples}
        \example  ?
     \end{examples}
\end{thetadescr}

\theta{vp123}

\begin{thetadescr}
\evitem say\index{say}
\esitem
     \begin{examples}
        \example  The note said that I was to leave at six.
     \end{examples}
\end{thetadescr}


\newpage
\verbpattern{[synTHATSENT\_LOCOPENPREPPPROP]}
\begin{vpattern}
\ruleitem TVerbpattern5,11
\csritem \mbox{}\\
     \begin{csr}
     predrel/PREPP & extraposrel/SENTENCE\{that\}
     \end{csr}
\remarksitem The control rules can probably not deal with this structure, nor 
has it been accounted for yet that extraposition is obligatory. Perhaps this 
pattern should be analysed differently. See the remarks on the previous rule.
\end{vpattern}

\theta{vp012}

\begin{thetadescr}
\evitem say\index{say}
\esitem
     \begin{examples}
        \example  It said in the paper that Honecker resigned.
     \end{examples}
\end{thetadescr}

\theta{vp123}

\begin{thetadescr}
\evitem ?
\esitem
     \begin{examples}
        \example  ?
     \end{examples}
\end{thetadescr}


\newpage
% \input{doc:r0248.idx}

\begin{theindex}
\item act  3, 71, 77
\item advise against 41 
\item agree with on  88
\item allow  35, 37, 52
\item allow on  27 
\item amount to  82
\item appear  3, 12
\item ask  42, 43, 56, 57
\item ask about 54
\item ask for  27, 39
\item beat  66
\item become  9 
\item believe  60 
\item bring  14 
\item burst out  68
\item call  21 
\item can$^{1}$  8 
\item can$^{2}$  70
\item catch  20 
\item change to  25
\item charge  47 
\item charge on  27 
\item come across as  93
\item consider  4, 9
\item convince  32 
\item cost  36, 49
\item count on  79, 81, 83, 96
\item decide  72
\item drive  16 
\item eat  33 
\item elect  21 
\item envy  48 
\item expect  12 
\item force  22, 30
\item get  13, 61, 72
\item give 47
\item give to  24
\item go 16
\item have 7, 8, 13
\item hear  97
\item help  50 
\item hesitate  98
\item hesitate about  92, 95
\item hesitate over  95
\item hold  12, 102
\item indicate  46 
\item inquire of  90
\item instruct  31 
\item instruct in  29
\item jump  15 
\item keep from  26
\item know  97, 102
\item leave to  69, 86
\item live  62 
\item make  51 
\item meet  18 
\item melt  66
\item mention to  91
\item mistake for  27 
\item motion  52 
\item name  22 
\item name after  27 
\item name as  27 
\item offer 42
\item paint  19 
\item persuade  52 
\item prefer  45, 72
\item prevent from  26
\item promise  38, 44, 52
\item provide with 54 
\item put  18 
\item rain  65
\item refer to  39 
\item refuse  97
\item regard as  25, 26, 27, 28
\item regret  66
\item remain  62 
\item report to 34 
\item reproach for  55 
\item say  103, 104
\item say to 100, 101
\item see  8, 13, 14
\item seem$^{1}$  4, 9, 12, 74, 102
\item seem$^{2}$ to 5, 6, 10, 11, 75, 76, 84, 91
\item send  16 
\item serve as  93
\item sign to  87
\item sleep  65
\item smell  67
\item stand  62 
\item stop  30 
\item strike as  53 
\item succeed in  92
\item suggest to 17 
\item swim  15 
\item talk out of  26
\item talk to about  40, 85, 89
\item talk with about  89
\item talk about 95
\item tear  23 
\item tell  47, 56, 57, 58, 59
\item tell about 54
\item tell to 17
\item think 64, 99
\item trust  12 
\item turn  19 
\item turn into  24, 27 
\item turn out  4 
\item understand  97
\item want  12 
\item want for  27 
\item wait for  81
\item watch 7
\item weigh  63 
\item wish  61 
\item wonder  97, 98

\end{theindex}

\end{document}

