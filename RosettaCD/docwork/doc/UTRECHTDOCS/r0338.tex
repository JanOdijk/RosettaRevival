\documentstyle{Rosetta}
\begin{document}
   \RosTopic{General}
   \RosTitle{Notulen Rosetta vergadering 22-05-1989}
   \RosAuthor{Andr\'{e} Schenk}
   \RosDocNr{R0338}
   \RosDate{June 28, 1989}
   \RosStatus{approved}
   \RosSupersedes{-}
   \RosDistribution{Project}
   \RosClearance{Project}
   \RosKeywords{minutes}
   \MakeRosTitle
%
%
\begin{description}
\item[Aanwezig:] Harm Smit, Andr\'{e} Schenk,
                 Lisette Appelo, Ren\'{e} Leermakers, 
                 Jan Odijk, Jan Landsbergen, Elly van Munster. 

\item[Afwezig:]  Margreet Sanders, Joep Rous, Franciska de Jong.

\item[Agenda:]\mbox{}
  \begin{enumerate}
  \item Notulen.
  \item Mededelingen.
  \item Documentatie.
  \item WSF.
  \item Woordenboeken.
  \item Rosetta4 e.d.
  \item Leesclub.
  \item Rondvraag.
  \end{enumerate}
\end{description}

\section{Notulen.}
De notulen van de vorige keer worden met enkele kleine wijzigingen goedgekeurd.

\section{Mededelingen.}
\begin{itemize}
  \item De aanvraag voor de PC en de CD-ROMspeler is goedgekeurd. Zij zijn
besteld. Voordat er nieuwe CD-ROMs besteld worden, moet eerst vastgesteld
worden of het nuttig is voor Rosetta. 
  \item Christian Rohrer, de organisator van de MT Summit in M\"{u}nchen, heeft 
bij Jan L. aangedrongen op een Rosettademonstratie op die conferentie. Jan wil 
eerst kijken of er geen technische problemen zijn en dan beslissen of er een 
demonstratie gegeven wordt.
\end{itemize}


\section{Documentatie.}
\begin{itemize}
   \item Er is (nog) geen bevroren versie van Rosetta3.
   \item Fouten niet verbeteren, alleen markeren, zodat we een correcte 
demonstratieversie hebben. In de documentatie wel de fouten opnemen en aangeven
dat er wijzigingen/verbeteringen zijn aangebracht in een later systeem.
   \item Er moet een lijst bijgehouden worden die aangeeft welke stukken 
interessant zijn voor iedereen, zodat die bespoken kunnen worden.
\end{itemize}

\section{WSF.}
\begin{itemize}
  \item Fraunhofer heeft een voorstel gedaan voor een marktonderzoek voor DM 
100.000. De resultaten van dat onderzoek geven echter niet de door ons gewenste
informatie. Er is een brief gestuurd met daarin een lijst met concrete vragen, 
die beantwoord moeten worden door het onderezoek. Deze week komt er antwoord of 
Fraunhofer dit voorstel accepteert of niet.
  \item Jan L. heeft met Jaap de Hoog gesproken en hem gevraagd of hij, 
vooruitlopend op het marktonderzoek \'{e}\'{e}n of meer mensen ter beschikking 
wil stellen voor de conversie naar C. In principe wilde hij dat wel, maar hij 
heeft nu al moeite met het werven van mensen, dus in de praktijk zal het 
moeilijk zijn. Jaap de Hoog suggereerde ook dat we moesten kijken of er tools 
waren voor een automatische conversie van Pascal naar C.
\end{itemize}


\section{Woordenboeken.}
Harm deelde een stuk uit getiteld {\em Overzicht van het woordenboekwerk aan de 
lexico-bestanden D en E}. Hier moet aan toegevoegd worden dat voor D alleen de 
eerste betekenis van een woord betrouwbaar en volledig gevuld is.

Er zal voor de volgende projectvergadering een werkplan voor het vullen van de 
restanten gemaakt worden met een taakverdeling. Prioriteit hebben: (i) het
vullen van de eerste betekenis van de Engelse verbs, (ii) het mergen van de
testwoordenboeken met de lexico-woordenboeken. 

\section{Rosetta4 e.d.}
\begin{itemize}
  \item Jan L. moedigt iedereen aan om de semantiek voor bestaande regels te 
schrijven. Hieruit zou kunnen blijken dat sommige regels identiteit als 
semantiek hebben. Dit werk moet gezien worden als startpunt voor andere
applicaties. 
  \item Jan O. schrijft een stukje over het Wielinga corpus.
  \item Jan L. en Andr\'{e} hebben een aantal brieven die de Tape Activated
Typewriter aan zou moeten kunnen. Het zou aardig zijn als wij kunnen zeggen
dat wij deze brieven ook kunnen vertalen. Dit lijkt ook zo te zijn.
\end{itemize}

\section{Leesclub.}
De eerste bijeenkomst van het leesclubje Unificatie grammatica's is 22-06-89
(t/m blz. 36), de tweede 29-06-89 (t/m blz. 50).

\section{Rondvraag.}
Lisette vraagt of er een vacature is voor een informaticus. Jan L. zegt dat dat 
zo is.

Jan L. sluit de vergadering.
\end{document}
