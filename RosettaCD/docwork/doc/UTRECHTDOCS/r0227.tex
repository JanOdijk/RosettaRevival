\documentstyle{Rosetta}
\begin{document}
   \RosTopic{General}
   \RosTitle{Description Plural-Form Restructuring}
   \RosAuthor{J.P. Medema, H.E. Smit}
   \RosDocNr{227}
   \RosDate{\today}
   \RosStatus{concept}
   \RosSupersedes{-}
   \RosDistribution{Software, Harm Smit}
   \RosClearance{Project}
   \RosKeywords{Dutch, Dictionary, Morphology, Plural}
   \MakeRosTitle

\newcommand{\kw}[1]{{\bf #1}\ }
\newlength{\Indentation}
\newlength{\Indentsize}
\setlength{\Indentsize}{1.3em}
\newcommand{\Tab}[1]
   {\addtolength{\Indentation}{#1\Indentsize}\\ \hspace*{\Indentation}}
\newcommand{\Tabs}[1]
   {\setlength{\Indentation}{#1\Indentsize}\hspace*{\Indentation}}
\setlength{\parskip}{2mm}
\setlength{\parindent}{0mm}

\section{Introduction}

\subsection{Contents}

This report gives a description of how so-called plural codes (as defined in
\cite{HS:morph}, page 29) are added automatically to a large human-made and 
human-oriented dictionary (\cite{VD:NN}). This dictionary is a monolingual Dutch
dictionary. The plural codes define, in context with the morphology, how nouns 
can be inflected. As a side effect of the program also phonetic information is 
obtained. 

The report also gives an idea which problems arise doing the restructuring 
automatically (the transformation from rather unformalized to formalized data).
It'll show how human-made dictionaries are filled and how they should be filled
for a better use with automated language systems.

\subsection{Structure}

The report is divided into 6 chapters of which this introduction is the first.
Chapter two describes (informally) the input of the restructuring program while
the output of the program is described (informally) in chapter three. The 
formalization of both the input and the output is given in chapter four. 
Chapter five decribes the algorithm which was used to restructure the data.
Some final remarks (including statistics and conclusions) are given in chapter 
six.

\subsection{Motivation}

The information needed for plural codes is already part of the input 
dictionary. It can be found in grammatical information which is given in
human-readable instead of a formal form (i.e. in a string instead of a 
code). These strings have to be parsed to result in the needed code. Grammatical
information can consist (for nouns) of plural information --this we use for 
this program--, diminutive information, and `garbage' --for us it is-- which 
mostly is comment like ``vaak mv.'' (often in plural form) and ``in bet. 3 
g.mv.'' (in meaning 3 no plural forming).

The restructuring of information (transforming strings to codes) has been done 
to be 
used in the dictionaries of Rosetta (though any other automated language system
could use the information as well). It is one of the first steps of formalizing 
the human-oriented dictionaries of Van Dale. This, to make it easier 
transforming the text-file-oriented dictionaries to the database-oriented 
dictionaries of Rosetta (for a detailed description of this activity, see
\cite{MSM:use}).

\newpage
\section{Input Description}

\subsection{Entry Form}

A global description of the entries of the used dictionary is given in 
\cite{SME:descr}. The knowledge given there is not quite a prerequisite for 
reading this section but could be rather useful for understanding certain 
aspects of it. However, the entries given in this section have, for clarity 
reasons, been stripped of all codes which aren't needed for the restructuring. 
This results in an entry syntax as in figure 1.

\begin{figure}[ht]
 \begin{center}
  \begin{verbatim}
                          II ( GR GI { GI }+
                             | GI { { VV }+ GV }
                             )
  \end{verbatim}
  \caption{Syntax of the input entry}
 \end{center}
\end{figure}

As can be seen, the GI-code is used on two different places in the entry.
The semantics of this code in the two places are not the same. In a `roman' 
subdivision (i.e. behind an GR-code) it'll give information only for one 
subdivision of the entry. Otherwise, it gives information for the entire
entry. The II-code gives the entry-word. The GR-code is used to mark that the 
entry is a roman entry; the given information applies for the whole entry. A 
form variant is given with the VV-code whilst the grammatical information for 
this form variant is given with the following GV-code.

The list below describes the information given with the codes:

\begin{description}
  \item [\verb+II+] \hspace{3mm}
             Each entryword is given (when,
             of course, needed) with a mark for the stress before the sylable 
             which is stressed. It can also contain special characters needed
             for the representation of, for example, ``\"{o}'', ``\~{n}'',
             ``\'{e}'', etc. These characters are all specified in 
             \cite{JM:retro}.
  \item [\verb+GI+] \hspace{3mm} 
             Gives the grammatical code of the (sub-)entry. It can also give 
             grammatical information which, in case of a roman subdivision,
             should be different for each sub-entry. The grammatical info can
             also contain the stress-marker and the special characters which
             the \verb+II+-code can contain.
  \item [\verb+GR+] \hspace{3mm}
             An entryword that has sub-entries with the same main category code
             (here always noun, i.e. 1) but different sub-categories (e.g., 
             masculine and/or feminine versus neuter noun) is marked by this 
             code. The main category of the entryword is always given. Further, 
             grammatical information which applies for all the sub-entries is 
             given here too.
  \item [\verb+VV+] \hspace{3mm} 
             This code preceeds a form variant of the entryword (a string). 
             It can also contain the stress-marker and the special characters 
             which the \verb+II+-code can contain.
  \item [\verb+GV+] \hspace{3mm} 
             The grammatical code for the form variant(s) is given with this
             code. Further, it gives grammatical information for the preceeding 
             form variant(s) (with, possibly, a stress-marker and special 
             characters). 
\end{description}

\subsection{Structure Grammatical Information}

\subsubsection{Partitioning}

The grammatical information can be found behind three codes: GI, GR, and GV.
If present\footnote{So it {\em can} be a void.}, it has the following structure
(which follows the grammatical code): first a semi-colon, then, 
separated by either comma's or semi-colons, the subsequent grammatical 
information parts.\footnote{Which can be more plural forms, diminutive form(s),
and comment (`garbage').}

According to our view, it should have been better when comment was always 
placed behind the second semi-colon. The subsequent plural forms and diminutive 
forms would have been before that semi-colon and separated by comma's. This 
would have made automatic parsing an easier task. For it is not known now 
whether parsing grammatical information or comment, comment parsing results in 
quite an amount of errors. This is not the case for human (Dutch) users of 
the dictionary who know enough of the language to distinguish grammatical 
information from syntactical/semantical comment.

\subsubsection{Part Description}

Of the three kinds of grammatical information parts (plural, diminutive, 
comment) only the plural part is described here. This plural information 
consists of a string which is (a part of) the word in plural form. This 
information can be given in three ways:

\begin{itemize}
  \item The complete word in plural form. See, for example, ``aas'' (``azen''),
        ``beer'' (``beren'').
  \item The ending of the plural form preceeded by a hyphen where the hyphen 
        represents a {\em part} of the entry word. This is the case at, e.g.,
        the words ``abri{\em ko}os'' (``-{\em ko}zen'') and ``kool{\em mijn}''
        (``-{\em mijn}en'').
  \item The ending of the plural form preceeded by a hyphen where the hyphen 
        represents the {\em whole} entry word. Cf. ``garde'' (``-s'',``-n''),
        ``jaagpaard'' (``-en''), etc.
\end{itemize}

For a human (Dutch) user of the dictionary this is clear (at least, I hope). 
He/she knows how to use the hyphen: the plural form of ``abrikoos'' is {\em not}
equal to ``abrikooskozen'', though it could be as it was interpreted as
defined above (in the third item). So, again, the user of the dictionary (whether 
born or manufactured) {\em needs} language knowledge.

\subsection{Example Dictionary}

The example dictionary given below is not a complete subset of all the cases 
that occur in the input dictionary. However, it gives an overview of the 
global structuring of the information throughout the dictionary:

\begin{verbatim}
   II#   aas           II#   abri+;koos      II#   +;garde
   GR#   1             GR#   1; -kozen       GR#   1
   GI#   13; g.mv.     GI#   14              GI#   14; -s
   GI#   15; azen      GI#   11              GI#   11; -n
\end{verbatim}

\begin{verbatim}
   II#   +;jaagpaard                         II#   +;jongen
   GI#   13; -en                             GI#   11; -s, jongetje
\end{verbatim}

\begin{verbatim}
   II#   +;junior                            II#   +;houtspaander
   GI#   11; -oren, -ores; meestal mv.       GI#   11; -s
                                             VV#   +;houtspaan
                                             GV#   14; -spanen
\end{verbatim}

As can be seen, the grammatical information is easy to understand; at least 
when one can understand Dutch so\footnote{This is true by now.} {\em definitely 
not} when one is a computer program. 

\newpage
\section{Output Description}

\subsection{Entry Form}

The output entry has the input entry as skeleton. As can be seen in figure 2, 
the input syntax has been extended with some extra codes for the morphological 
and the phonetic information. Within an entry that has no roman subdivision 
the G-code(s) are followed by an M-code. In entries with roman subdivision 
either the GR-code is followed by an M(R)-code or all the GI-codes are followed 
by M(I)-code. In both kinds of entries, every M-code {\em can} be followed by 
an FO-code. 

\begin{figure}[ht]
 \begin{center}
  \begin{verbatim}
                 II ( GR ( MR [ FO ] GI { GI }+
                         | GI MI [ FO ] { GI MI [ FO ] }+
                         )
                    | GI MI [ FO ] { { VV }+ GV MV [ FO ] }
                    )
  \end{verbatim}
  \caption{Syntax of the output entry}
 \end{center}
\end{figure}

All codes equal to the input codes have kept their meaning.\footnote{It is
{\em restructuring} of information.} The new codes are explained below:

\begin{description}
  \item [\verb+MI+] \hspace{3mm}
                    The grammatical information is coded here. The line consists
                    of a number of morphological codes (always two digits) as 
                    defined in section ?? separated by comma's. The number of
                    codes is equal to the number of grammatical information 
                    parts in the GI-code before the MI-Code. 
  \item [\verb+MR+] \hspace{3mm}
                    Is the same as MI-code but then applied for an GR-code.
  \item [\verb+MV+] \hspace{3mm}
                    Is the same as MI-code but then applied for an GV-code.
  \item [\verb+FO+] \hspace{3mm}
                    This code contains the phonetic information found as a side
                    effect. The form is quite simple. It consists of two 
                    characters of which the first denotates the `change' and 
                    the second the `sjwa'. There are three possible values each
                    character can have. A slash (``/'') means that the phonetic 
                    attribute has the value `false', a zero (``0'') corresponds
                    with the value `true', and a one (``1'') means that the 
                    attribute can have both values (cf. ``spons'' for change).
                    \\ Formally it is not possible to have both values for 
                    sjwa (a word either has it or not) but the program can 
                    give it for words with more plurals (cf. ``aanhangsel'')
                    for, e.g., an S-plural will never result in a sjwa\footnote{
                    In this case, from the combination of the entryword and the pluralstring 
                    nothing can be concluded for the phonetic information; for 
                    example, ``hotel'' and ``schotel'' both have an 
                   pluralstring equal to ``-s'' but only the latter has a sjwa.}
                    while an En-Plural can.
                    \\ If the FO-code has the value ``//'', it'll be absent in 
                    the output entry (the ``//'' value is the default value).
\end{description}

\subsection{Morphological Codes}

Most codes in the output are codes as defined in \cite{HS:morph}. For 
clarity reasons they're given here too (with examples).

\begin{tabular}{|r|l|l|} \hline
 Code & Code Description & Example's\\ \hline
   1  & enPlural         & boeken, appelen, koloni\"{e}n\\
   2  & sPlural          & etalages, appels, boompjes\\
   3  & aTOaaPlural      & daken, baden, verdragen\\
   4  & aTOeePlural      & steden\\
   5  & eTOeePlural      & bevel, gebrek, weg\\
   6  & eiTOeePlural     & waarheden, heren\\
   7  & iTOeePlural      & leden, schepen, smeden\\
   8  & oTOooPlural      & goden, geboden, motoren\\
   9  & erenPlural       & goederen, lammeren, liederen\\
  10  & ienPlural        & vlooien, koeien\\
  11  & denPlural        & roeden, treden\\
  12  & nenPlural        & lendenen, redenen\\
  13  & ieAccentPlural   & knie\"{e}n, antipathie\"{e}n\\
  14  & luiPlural        & werklui\\
  15  & liedenPlural     & brandweerlieden\\
  16  & LatPlural        & schemata, matrices, spectra, bases\\
  17  & enIrregPlural    & bamboezen\\
  18  & sIrregPlural     & vlaas, egaas\\
  19  & LatIrregPlural   & tempora, casus\\
  20  & NoPlural         & verdriet, heelal\\
  21  & OnlyPlural       & hersens, onkosten, inkomsten\\ \hline
\end{tabular}

The other codes which occur in the output are 98 and 99. The code 98 has as 
meaning that the grammatical information was absent (which, probably, means that 
the entryword has no plural form at all; this, eventually, will result in a 
code 20 in the Rosetta dictionary). Code 99 has as meaning that the part of the
information could not be parsed by the current parser. Often, this is caused by
an input `error'.

\subsection{Example Dictionary}

The dictionary as given in paragraph ?? should (and in fact does) result in the 
following output dictionary:

\begin{verbatim}
   II#   aas           II#   abri+;koos      II#   +;garde
   GR#   1             GR#   1; -kozen       GR#   1
   GI#   13; g.mv.     MR#   01              GI#   14; -s
   MI#   20            FO#   0/              MI#   02
   GI#   15; azen      GI#   14              GI#   11; -n
   MI#   01            GI#   11              MI#   01
   FO#   0/
\end{verbatim}

\begin{verbatim}
   II#   +;jaagpaard                         II#   +;jongen
   GI#   13; -en                             GI#   11; -s, jongetje
   MI#   01                                  MI#   02,99
\end{verbatim}

\begin{verbatim}
   II#   +;junior                            II#   +;houtspaander
   GI#   11; -oren, -ores; meestal mv.       GI#   11; -s
   MI#   08,16,99                            MI#   02
                                             VV#   +;houtspaan
                                             GV#   14; -spanen
                                             MI#   01
\end{verbatim}

Note that ``jongetje'' and ``meestal mv.'' cannot be parsed as pluralform and 
therefore result the value 99. Both strings, however, are for human (Dutch) 
users interesting information sources (the first string is the diminutive form 
while the latter is information on how the word is to be used).

\newpage
\section{Formalization}

\subsection{Description Treatable Parts}

In this paragraph the parts which can be treated by the program will be 
described formally. The treating is like a multiple filter system: the word 
and its plural form will be tested by the first layer; if they don't pass the 
filter (i.e. they fulfil the requirements for that kind of plural) then the 
kind of plural form has been found; otherwise the next filter is tried; at the 
end (when no more filters exist) the erroneous parts remain.

First, a number of sets is defined to make it easier to describe the words and
grammatical information parts. The set L denotates all the letters; K1 
denotates all vowels; M2 denotates a subset of all consonants; and M3 denotates 
all consonants.

\begin{tabular}{lclcl}
L   & = & \{ a, b, ..., y, z \} & & \\
K1  & = & \{ a, e, i, o, u \} & & \\
M2  & = & \{ b, d, f, g, k, l, m, n, p, r, s, t \} & & \\
M3  & = & M2 $\cup$ \{ c, h, j, q, v, w, x, y, z \} & = & L $\setminus$ K1
\end{tabular}

The following table describes the first ten filters of the program. The zeroth
column gives the number of the filter; in the first a kind of variable 
declaration is done; the second describes the form the entry word must have; 
the third column describes the form of the grammatical information part; the 
fourth gives the plural code the word is assigned to if it passes the test;
the fifth gives the value of the phonetic attributes; the sixth column 
gives examples which fulfil the test (plain) and which don't (italic).

\begin{tabular}{|r|l|l|l|r|r|l|} \hline
   N & Domain(s)
       & Entry
         & Plural
           & Cd.
             & Ph.
               & Example(s) \\ \hline \hline
   1 & $\alpha,\beta \in$ L*
       & $\beta\alpha$
         & -s, -$\alpha$s, $\beta\alpha$s,
           & 02
             & //
               & broer $\rightarrow$ -s \\
     &
       &
         & -'s, -$\alpha$'s, $\beta\alpha$'s
           & 
             & //
               & vla $\rightarrow  $ -'s \\ \hline
   2 & $\alpha \in$ L*
       & $\alpha$
         & g.mv., g.mv, g.{\leavevmode\hbox{\tt\char`\ }}mv.
           & 20
             & //
               & aanblik $\rightarrow$ g.mv. \\ \hline
   3 & $\alpha \in$ L*
       & $\alpha$
         & mv., mv, g.enk.
           & 21
             & //
               & dii $\rightarrow$ mv. \\ \hline
   4 & $\alpha,\beta,\gamma \in$ L*
       & $\beta\alpha$ $\wedge$
         & -n, -$\alpha$n, $\beta\alpha$n,
           & 01
             & //
               & zieke $\rightarrow$ -n \\
     & $\delta \in$ M3
       & $\neg$ ($\gamma\delta\varepsilon\zeta$)
         & -en, -$\alpha$en, $\beta\alpha$en
           & 
             & //
               & ligfiets $\rightarrow$ -en\\
     & $\varepsilon \in$ K1
       &
         &
           & 
             & //
               & koolmijn $\rightarrow$ mijnen \\
     & $\zeta \in$ M2
       &
         &
           & 
             & 
               & {\em dak $\rightarrow$ -en} \\ \hline
   5 & $\alpha \in$ L*
       & $\alpha$ee
         & -\"{e}n
           & 01
             & //
               & slee $\rightarrow$ -\"{e}n \\
     & 
       & 
         & 
           & 
             & //
               & wee $\rightarrow$ -\"{e}n \\ \hline
   6 & $\alpha \in$ K1
       & $\beta\alpha\gamma$
         & -$\gamma$en, -$\alpha\gamma\gamma$en, 
           & 01
             & //
               & tak $\rightarrow$ -ken \\
     & $\beta \in$ L*
       &
         & $\beta\alpha\gamma\gamma$en
           & 
             & //
               & val $\rightarrow$ -len \\
     & $\gamma \in$ M2 $\cup$ \{ z \}
       &
         &
           & 
             & //
               & kam $\rightarrow$ -men\\ \hline
   7 & $\alpha,\beta \in$ L*
       & $\beta\alpha\gamma\gamma\delta$
         & -$\gamma\delta$en, -$\alpha\gamma\delta$en, 
           & 01
             & //
               & taak $\rightarrow$ taken \\
     & $\gamma \in$ K1
       &
         & $\beta\alpha\gamma\delta$en
           & 
             & //
               & beet $\rightarrow$ beten \\
     & $\delta \in$ M2
       &
         &
           & 
             & //
               & aap $\rightarrow$ apen \\ \hline
   8 & $\alpha,\beta \in$ L*
       & $\beta\alpha$ie
         & -i\"{e}n, -$\alpha$i\"{e}n, 
           & 01
             & //
               & kolonie $\rightarrow$ -ni\"{e}n \\ 
     &
       &
         & $\beta\alpha$i\"{e}n
           & 
             & //
               & orgie $\rightarrow$ orgi\"{e}n\\ \hline
   9 & $\alpha,\beta,\gamma \in$ L*
       & $\beta\alpha$f $\wedge$
         & -ven, -$\alpha$ven,
           & 01
             & 0/
               & korf $\rightarrow$ korven \\ 
     & $\delta \in$ M3
       & $\neg$ ($\gamma\delta\varepsilon$f)
         & $\beta\alpha$ven
           & 
             & 0/
               & lijf $\rightarrow$ lijven \\
     & $\varepsilon \in$ K1
       &
         &
           & 
             & 
               & {\em staf $\rightarrow$ staven} \\ \hline
  10 & $\alpha,\beta,\gamma \in$ L*
       & $\beta\alpha$s $\wedge$
         & -zen, -$\alpha$zen,
           & 01
             & 0/
               & accijns $\rightarrow$ -zen \\ 
     & $\delta \in$ M3
       & $\neg$ ($\gamma\delta\varepsilon$s)
         & $\beta\alpha$zen
           & 
             & 0/
               & kruis $\rightarrow$ kruizen\\
     & $\varepsilon \in$ K1
       &
         &
           & 
             & 
               & {\em glas $\rightarrow$ glazen} \\ \hline
  11 & $\alpha,\beta \in$ L*
       & $\beta\alpha\gamma\gamma$f
         & -$\gamma$ven, -$\alpha\gamma$ven, 
           & 01
             & 0/
               & raaf $\rightarrow$ raven \\
     & $\gamma \in$ K1
       &
         & $\beta\alpha\gamma$ven
           & 
             & 0/
               & zeef $\rightarrow$ zeven \\ \hline
  12 & $\alpha,\beta \in$ L*
       & $\beta\alpha\gamma\gamma$s
         & -$\gamma$zen, -$\alpha\gamma$zen, 
           & 01
             & 0/
               & doos $\rightarrow$ dozen \\
     & $\gamma \in$ K1
       &
         & $\beta\alpha\gamma$zen
           & 
             & 0/
               & baas $\rightarrow$ bazen \\ \hline
  13 & $\alpha,\beta \in$ L*
       & $\beta\alpha$ie
         & -\"{e}n, -$\alpha$ie\"{e}n,
           & 13
             & //
               & knie $\rightarrow$ -\"{e}n\\ 
     &
       &
         & $\beta\alpha$ie\"{e}n
           & 
             & //
               & orgie $\rightarrow$ -\"{e}n \\ \hline
\end{tabular}

\newpage
\section{Algorithm}

\subsection{Main Program}

\Tabs{0} i := 1
\Tab{0}  \kw{while} \kw{still} Nouns \kw{do}
\Tab{1}   ReadNoun$_i$
\Tab{0}   j := 1
\Tab{0}   \kw{while} \kw{still} GCodes$_i$ \kw{do}
\Tab{1}    \kw{if} GCode$_{i,j}$ \kw{is} Treatable \kw{then}
\Tab{1}     k := 1
\Tab{0}     \kw{repeat}
\Tab{1}      TreatGCodePart$_{i,j,k}$
\Tab{0}      k := k + 1
\Tab{-1}    \kw{until} GCode$_{i,j}$ \kw{is} Treated
\Tab{-1}   \kw{else}
\Tab{1}     No plural is defined
\Tab{-1}   \kw{end} \kw{if}
\Tab{0}    j := j + 1
\Tab{-1}  \kw{end} \kw{while}
\Tab{0}   WriteNoun$_i$
\Tab{0}   i := i + 1
\Tab{-1} \kw{end} \kw{while}

\subsection{Treat Grammatical Information Part}

\Tabs{0} \kw{proc} TreatGCodePart$_{i,j,k}$ \kw{is}
\Tab{0}  \kw{begin}
\Tab{1}   \kw{if} ConditionSplural$_{i,j,k}$ \kw{is} TRUE \kw{then}
\Tab{1}    Append SPlural to MI Code 
\Tab{-1}  \kw{elsif} ConditionEnPlural$_{i,j,k}$ \kw{is} TRUE \kw{then}
\Tab{1}    Append EnPlural to MI Code 
\Tab{-1}  \kw{elsif} Condition\ldots$_{i,j,k}$ \kw{is} TRUE \kw{then}
\Tab{1}    \vdots
\Tab{-1}  \kw{end if}
\Tab{-1} \kw {end} \kw{proc} 

\subsection{Condition S-Plural}

\Tabs{0} \kw{func} ConditionSPlural$_{i,j,k}$ : Boolean \kw{is}
\Tab{0}  \kw{begin}
\Tab{1}   \kw{if} $\neg$ (Headword$_i$End and GCodePart$_{i,j,k}$End Matches) 
          \kw{then}
\Tab{1}    b := FALSE
\Tab{-1}  \kw{else}
\Tab{1}    b := AlphaBetaOK$_{i,j,k}$
\Tab{-1}  \kw{end if}
\Tab{-1} \kw{end} \kw{func}

\subsection{Compare On $\alpha, \beta$}

\Tabs{0} \kw{func} AlphaBetaOK$_{i,j,k}$ : Boolean \kw{is}
\Tab{1}   \kw{var} b : Boolean
\Tab{-1} \kw{begin}
\Tab{1}   \kw{if} Not checked part GCodePart$_{i,j,k}$ \kw{is} `-' \kw{then}
\Tab{1}    b := TRUE
\Tab{-1}  \kw{elsif} First Character of GCodePart$_{i,j,k}$ \kw{is} `-' \kw{then}
\Tab{1}    b := (AlphaPart$_i$ matches AlphaGCodePart$_{i,j,k}$)
\Tab{-1}  \kw{else}
\Tab{1}    b := (BetaAlphaPart$_i$ matches BetaAlphaGCodePart$_{i,j,k}$)
\Tab{-1}  \kw{end if}
\Tab{0}   \kw{return} b
\Tab{-1} \kw{end} \kw{func}

\newpage
\section{Final Remarks}

\begin{itemize}
 \item All codes can now be generated;
 \item This is done by means of the list of exceptions found using the program 
       as described before;
 \item On an estimated number of 100,000 GCodeParts circa 1,500 of these will 
       not been `understood' by the program; among these are:
   \begin{itemize}
    \item strings as ``meestal mv.'',  ``in bet. 12 -en'', etcetera;
    \item irregular plurals;
    \item data errors.
   \end{itemize}
\end{itemize}

\begin{thebibliography}{AAA99a}
 \bibitem[MED87]{JM:retro} J.P. Medema, {\em System To Create Retrograde 
  Dictionaries}, {\bf Rosetta report 190}, Philips Research labs, 1987
 \bibitem[MSM87]{MSM:use} J.P. Medema \& H.E. Smit, {\em Using Van
  Dale For Rosetta}, {\bf Rosetta report 227}, Philips Research labs, 
  1987
 \bibitem[SME87]{SME:descr} H.E. Smit \& J.P. Medema, {\em Description Van 
  Dale dictionary N~--~N}, {\bf Rosetta report 174}, Philips Research labs, 
  1987
 \bibitem[SMI86]{HS:morph} H.E. Smit, {\em Rosetta3 Dutch morphology,
  inflection (comments)}, {\bf Rosetta report 134}, Philips Research labs, 
  1986
 \bibitem[STE84]{VD:NN} P.G.J. van Sterkenburg, ed., {\em Groot woordenboek 
  hedendaags Nederlands}, Van Dale Lexicografie, 1984
\end{thebibliography}

\end{document}
