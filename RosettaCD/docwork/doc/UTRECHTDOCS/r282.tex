
\documentstyle{Rosetta}
\begin{document}
   \RosTopic{General}
   \RosTitle{Notulen Linguistenvergadering 27-09-88}
   \RosAuthor{Margreet Sanders}
   \RosDocNr{0282}
   \RosDate{September 28, 1988}
   \RosStatus{informal}
   \RosSupersedes{-}
   \RosDistribution{Linguists, Joep Rous}
   \RosClearance{Project}
   \RosKeywords{minutes, RelShift, modalen}
   \MakeRosTitle
%
%
\begin{description}
\item[Aanwezig:] Lisette Appelo, Franciska de Jong, Elly van Munster,
                 Jan Odijk, Margreet Sanders (not),
                 Andr\'{e} Schenk, Harm Smit
\item[Agenda:]\mbox{}
  \begin{enumerate}
  \item RelShift
  \item Modalen
  \item Huiswerk
  \end{enumerate}
\end{description}

\section{RelShift}
Tot nu toe bestond er alleen een Wh-Shift transformatieklasse. Het was de 
bedoeling dat daarnaast een regelklasse RelShift geschreven zou worden waarin 
de index van de VAR die gerelativiseerd gaat worden wordt doorgegeven, en 
waarin 
tevens de shift zelf wordt uitgevoerd. Aangezien de shiftregels voor wh-shift 
en relativisatie praktisch hetzelfde zijn, en de wh-shift regel bovendien enorm 
lang is, heeft Jan O.\ besloten RelShift op te delen in twee stappen: in de 
optionele regelklasse {\em RelMarking\/} worden nu de 
relevante NPVARs (de index daarvan 
is uit analyse bekend) gemarkeerd met de waarde {\em 
relativexpmood\/} (CNVARs hoeven niet gemarkeerd te worden, omdat 
zij altijd zullen relativiseren), en in de transformatieklasse {\em WhShift\/}
wordt de eigenlijke shift uitgevoerd (NB: deze TC heeft dus zijn oude naam 
gehouden, ook al is de shift nu niet meer alleen voor Wh-elementen!). Een 
bijkomend voordeel van deze aanpak is dat isomorfie met de ADJP-grammatica 
makkelijker is: daar hoeft nu alleen nog een (optionele) regelklasse RelMarking 
voor NPs te worden geschreven, en geen echte RelShift.

\section{Modalen}
Het ongenummerde document over modalen dat Jan O.\ op de linguistenvergadering 
van 20 september had uitgedeeld wordt besproken. Er worden wat wijzigingen 
aangebracht in het schema op p.\ 1, de problemen van p.\ 2 (en nog enige nieuwe 
problemen) worden besproken, en er wordt een commissie ingesteld bestaande uit 
Lisette en Jan O.\ die de voorgestelde oplossingen nader zullen bekijken op hun 
consequenties en vervolgens op papier zullen zetten. Dit papier volgt nog.

\section{Huiswerk}
Voor de volgende vergadering moet iedereen twee lijstjes maken:
\begin{enumerate}
\item een lijstje met interessante voorbeeldzinnen die het systeem aankan 
(of aan hoort te 
kunnen) volgens de huidige regels. Deze zinnen kunnen gebruikt worden bij 
demonstraties en als basis voor een testzinnenbank
\item een lijstje met de regels die nog geschreven moeten worden, en de 
prioriteit ervan.
\end{enumerate}
Beide lijstjes zullen op de volgende vergadering worden besproken.
\end{document}

