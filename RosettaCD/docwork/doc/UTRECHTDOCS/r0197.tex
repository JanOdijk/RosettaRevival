\documentstyle{Rosetta}
\begin{document}
   \RosTopic{General}
   \RosTitle{Notulen Linguisten bijeenkomsten 14-4-1987 en 22-4-1987}
   \RosAuthor{Harm Smit}
   \RosDocNr{0197}
   \RosDate{\today}
   \RosStatus{approved}
   \RosSupersedes{-}
   \RosDistribution{Linguists, Joep Rous}
   \RosClearance{Project}
   \RosKeywords{notulen}
   \MakeRosTitle
\begin{itemize}
  \item {\bf aanwezig}: Andr\'{e} Schenk, Jan Odijk, Elly van Munster, 
             Harm Smit, Margreet Sanders.
  \item {\bf afwezig}: Franciska de Jong, Lisette Appelo.
  \item {\bf opmerking}: de `onoplosbaar' kwestie is in een 
aparte bijeenkomst met Joep besproken; hier waren bovengenoemde personen, 
met uitzondering van Elly bij aanwezig. Het resultaat van deze bespreking is 
als laatste punt van deze notulen opgenomen.
\end{itemize}

\section {notulen vorige bijeenkomst}

Bij de onder 2 genoemde attribuut waarden zijn niet alle correct; wie 
ermee werkt 
moet ze in het domein naslaan. Bijv.: {\em er-case} moet {\em Rcase} zijn, etc.

\section {M-rule editor en editor voor surface-regels}

In de M-rule editor kon voor het Nederlands altijd al $<$GOLD$>$ H + NP.rec 
gebruikt worden, waarbij alle attributen van het record verschenen; dit 
werkt nu -behalve voor het Nederlands- ook voor het Spaans en het Engels.
Er is nu ook een editor voor surface regels (zowel voor Spaans, Engels als 
Nederlands). Informatie over deze twee editors bij Jan O.

\section {Adverbia-regels}

Er zijn drie soorten adverbia bijgekomen:
\begin{itemize}
   \item {\bf instrumentalis} (`met' + {\em noun}), zoals `met een hamer';
   \item {\bf `samen met'}-gevallen (`samen met' + {\em noun (bijv. met waarde 
          +animate)}, zoals `samen met zijn broer');
   \item {\bf causalis}, zoals `daarom', `waarom', `om die redenen', etc.
\end{itemize}
Er moet bekeken worden hoe deze gevallen gedaan worden; Jan O. heeft voor alle 
drie soorten al een aantal regels geschreven. Bij de eerste soort wordt daarbij
in de VERBPPROP een PP (als geheel) gezet van de vorm `met {\em x}'. De 
variabele {\em x} wordt pas later gevuld. 

Zij die met deze materie te maken krijgen, dienen hierover met Jan O. te 
overleggen.

\section {Schrijfwijze van {\em er} in combinatie met preposities e.d.}

De manier van schrijven van {\em er} (en {\em hier}, {\em daar}, etc.)
in combinatie met preposities e.d. is niet 
duidelijk: wanneer moet het los geschreven en wanneer aan elkaar?

Tijdens de vergadering blijkt dat er ook onder de aanwezigen niet echt een 
duidelijke voorkeur bestaat; zelfs over het verdere (syntactische) gedrag van 
{\em er} ontstaat discussie.

In principe wordt het volgende besloten (behalve voor {\em er} geldt dit ook 
voor {\em hier}, {\em daar}, etc.):
  \begin{itemize}
    \item in {\em generatie}: als {\em er} los geschreven {\bf kan} worden 
wordt geen GLUE ge\"{\i}ntroduceerd. Alleen als er tussen {\em er} en de 
prepositie {\bf niets} staat en ook {\bf niets kan} staan (buiten eventuele 
modificeerders van de prepositie, zoals `verder') worden {\em er} en de 
prepositie aan elkaar geschreven. Voorbeeld: `de huizen van {\em hiernaast}'.
    \item in {\em analyse}: hier wordt in meer situaties een GLUE toegestaan 
dan in de generatie geproduceerd. Jan O. zal dit in de M-grammatica's 
beregelen. 
  \end{itemize}

\section {Wijziging in planning door uitstel testen Nederlandse Morfologie}

Doordat er ernstige problemen waren met de efficientie bij het testen van de 
Nederlandse morfologie, is door Joep en Harm 
besloten de reguliere expressie (die aangeeft welke affix-combinaties de
Nederlands lextree-regels toestaan) versneld toe te voegen. Zolang deze 
reguliere expressie er niet is kan een deel van de regels {\em niet} getest 
worden (omdat de morfologie uit het geheugen loopt) en een ander deel slechts 
onder moeizame omstandigheden (nl. met een zeer traag werkende morfologie). 
Harm verwacht daarom dat een aanpassing van de planning noodzakelijk is (d.w.z.
een 
verschuiving van de test-aktiviteiten met 2 - 3 weken). Ondertussen zullen de 
reeds gevonden fouten verbeterd worden.

\section {Bespreking van het `onoplosbaar'-voorstel van Joep op 22-4}

Niemand heeft bezwaren tegen de oplossing. Geconstateerd wordt dat met de 
gegeven oplossing wel `opgelost' (als {\em gepartikelde} variant van een gewoon 
voltooid deelwoord, vgl. voorstel Joep, laatste lextree-regel sectie 2) en 
`onoplosbaar' (waarbij het partikel gewoon als prefix wordt behandeld) 
mooi behandeld kunnen worden, 
maar dat woorden als `onopgelost' problematisch blijven: hoewel hiervoor ook 
een oplossing bedacht kan worden, resulteert dit in forse bomen, omdat de 
inflectie ({\em ge-} en {\em -t}) pas op VERB-nivo is gerealiseerd (tesamen met 
het partikel), maar er vervolgens nog derivatie moet plaatsvinden ({\em on-}),
zodat `VERB' veranderd moet worden in `SUBADJ'. Dergelijke boomstrukturen 
lijken minder gewenst. Omdat het er verder ook op lijkt dat dergelijke 
combinaties van 
{\em on-} met een VERB niet semantisch regelmatig zijn, ziet het er naar uit 
dat we `opgelost' beter als aparte entry van de categorie BADJ op kunnen 
nemen in het woordenboek. Het affix {\em on-} kan dan wel het {\em BADJ}
`opgelost' prefigeren, maar niet het {\em VERB} `opgelost'.

Harm zal t.z.t., als de derivatie-regels worden geschreven, nog eens over dit 
probleem nadenken.

\end{document}
