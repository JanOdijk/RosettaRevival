% one-substitution: die naast de trein - the one next to the train
%                                        those next to the train

\documentstyle{Rosetta}
\begin{document}
   \RosTopic{Rosetta3.doc.linguistics.english}
   \RosTitle{English M-rules:subgrammar CNFormation}
   \RosAuthor{Franciska de Jong}
   \RosDocNr{414}
   \RosDate{\today}
   \RosStatus{concept}
   \RosSupersedes{-}
   \RosDistribution{Project}
   \RosClearance{Project}
   \RosKeywords{English, M-rules, CNFormation}
   \MakeRosTitle
%
%
\input{[dejong.mrules]mrudocdef}

\section{Introduction}
In this document the subgrammar CNformation for English is described.
In general the task of this subgrammar 
is to handle the phenomena involved in the proces of NPformation that pertain 
syntactically and/or semantically to the level CN.
It is not relevant to {\bf all} 
NPs, but only to those 
that are headed by CN. (This holds for NPs containing either a full NOUN or a 
EN (empty head).)
It does not apply to NPs without a CN head (NPs 
headed by singular proper names or a pronoun.)
The subgrammar is not supposed to be isomorphic to any other subgrammar 
of English. 

\section{Subgrammar Specification}

\begin{description}
  \item[Head] SUBNOUN, EN
  \item[Export] CN
  \item[Import] 
OPENADJPPROP,  DETP, OPENPREPPPROP, 
NP, PROPERNOUN, SENTENCE, NPVAR, SENTENCEVAR

\item[File] english:cnformation.mrule (mrules69)
\end{description}

\section{Control Expression}
The control expression can be defined as follows:
\begin{verbatim}

    [( RSUBNOUNTONOUN1/1 | RSUBNOUNTONOUN2/2)]

   .( RCNFORMATION1/3 | RCNFORMATION2/4 | RCNFORMATION3/5 
      | RCNFORMATION4/6  )

   . (RCNPresentSuperdeixis/100   | RCNPastSuperdeixis/101)

   . [RNOUNargmod1/110 | RNOUNargmod2/111]

   . [RCNmodbareNP/23]

   . [RCNspecProperName/24]

   .{ RCNMODADJP/7       | RCNMODNUM/8    
      | RCNMODPOSS1/9   | RCNMODPOSS2/10 | RCNMODPOSS3/11 
      | RCNMODPP/12 | RCNMODRELSENT1/13 
      | RCNMODANTEREL1/22
    }            

\end{verbatim}

\section{Rules and Transformations}
\begin{mruleclass}{RC\_subnounTOnoun}
\begin{classdescr}
\kind obligatory rule class
\classtask
\begin{enumerate}
  \item 
Introduction of the syntactic level NOUN
  \item
Determination of the value for .number
\end{enumerate}
\classremarks
In Rosetta, the morphological aspect of number (inflection),
as well as the translation
of number for CNs with a non-empty head
are dealt with by this RC.
That is, the two rules that assign values for .number 
are each mapped onto  a different IL-rule, 
namely one 
for singular count nouns (LSUBNOUNTONOUN1), the other both for singular
mass nouns and plural nouns  (LSUBNOUNTONOUN2).

Note that for CNs with an empty head (EN)
there is no morphology and hence no number assignment 
on the the CN head. (The record of category EN contains only the attribute 
.key). The CN itself {\bf is} 
specified for number. The translation
of EN number is dealt with by the mapping of the rules of RC\_CNformation. 

For further comment on the treatment of number cf. doc:R412 (to appear)
and the remarks to RC\_CNformation.
\nofilters
\nospeedrules

\noplannedrules

\norulesnotince

\rulelist

\end{classdescr}

\begin{members}
\begin{member}
\rulename RSUBNOUNtoNOUN1
\ruletask \begin{enumerate}
  \item 
introducing the syntactic level NOUN
  \item
assigning singular number to 
count NOUNs
\end{enumerate}
\file dutch:cnformation.mrule (mrules69)
\semantics \nosemantics
\example garden, table, book, bloke, hour
\remarks\mbox{}

\end{member}
\begin{member}
\rulename RSUBNOUNtoNOUN2
\ruletask \mbox{}\\
\begin{enumerate} 
\item introducing the syntactic level NOUN
\item
assigning plural number to count nouns 
\item assigning singular number to mass nouns 
\end{enumerate}
\file dutch:cnformation.mrule (mrules69)
\semantics \nosemantics
\example gardens, hours, police,  toys, bread
\remarks\mbox{}
\end{member}
\end{members}
\end{mruleclass}
\begin{mruleclass}{RC\_CNformation}
\begin{classdescr}
\kind obligatory rule class
\classtask
 \begin{enumerate}
  \item 
Introduction of  the syntactic level CN.
  \item
Account for the number of CNs headed by EN.
\end{enumerate}

\classremarks\mbox{}
\begin{enumerate}
  \item 

As a consequence of the present approach, the translation of 
number is distributed over two rule classes, pertaining to different
syntactic levels: RC\_subnountonoun and RC\_CNformation.
Morphology (inflection) has played a decisive role
in this decision. As
 there is no translational relation between full nouns and EN, 
it does not
affect the translational performance.

Alternatively we might choose for an approach that translates uniformly on the
level of CNformation. This would require that RCNformation is split up and 
mapped onto ILrules in a way comparable to the present rules RsubnounTOnoun1 
and RsubnounTOnoun2. The latter two should then be replaced by one rule
(or transformation) that -generatively spoken- makes all 
forms available for 
a certain noun. I.e. 
a temporal ambiguity would be introduced: one path for
each possible values for .number.

This alternative would at least have the advantage of 
a uniform level for the translation of number.
Future research is needed in order to decide whether this alternative is 
to be preferred. Especially ellipsis on the basis of contextual information
should be studied in more detail.
 
 \item Two ENs are distinguished: one for count interpretations and one for 
mass interpretations. This distinction is a.o. motivated for Dutch. Cf. doc. 
R409.
  \item
Among the three rules introducing a CN node to dominate an empty head (EN) 
the same semantic distinction has been made as for the rules of RC\_
subnounTOnoun.
They are mapped on two ILrules: 
CNformation3 is  mapped onto LCNformationcountsing, the other two
onto 
LCNformationmassplur.

\end{enumerate}

\nofilters

\nospeedrules

\noplannedrules

\norulesnotince

\rulelist

\end{classdescr}

\begin{members}
\begin{member}
\rulename CNformation1
\ruletask Introducing the syntactic level CN in case of a non-empty head NOUN.
\file dutch:cnformation.mrule (mrules69)
\semantics \nosemantics
\example all full nouns
\remarks\mbox{}
\end{member}
\begin{member}
\rulename CNformation2
\ruletask
Introducing a CN node (singular, mass) for a mass EN (empty noun).
\file dutch:cnformation.mrule (mrules69)
\semantics LCNformationmassplur
\example much EN, ....
\remarks\mbox{}
\end{member}
\begin{member}
\rulename CNformation3
\ruletask Introducing a CN node (singular, count) for a count EN (empty noun).
\file dutch:cnformation.mrule (mrules69)
\semantics LCNformationcountsing
\example one EN, ...
\end{member}
\begin{member}
\rulename CNformation4
\ruletask Introducing a CN node (plural, count) for a count EN (empty noun).
\file dutch:cnformation.mrule (mrules69)
\semantics LCNformationmassplur
\example many EN, both EN, three yellow EN, some EN

\end{member}
\end{members}
\end{mruleclass}
\begin{mruleclass}{RC\_CNSuperdeixis}
\begin{classdescr}
\kind obligatory rule class
\classtask Deal with superdeixis
\classremarks

\nofilters

\nospeedrules

\noplannedrules

\norulesnotince

\rulelist

\end{classdescr}

\begin{members}
   
\begin{member}
\rulename RCNPresentsuperdeixis
\ruletask In generation: set value for superdeixis at presentdeixis.\\
               In analysis: set value for superdeixis at omegadeixis.    
\file dutch:cnformation.mrule (mrules69)
\semantics \nosemantics
\example all CNs
\remarks\mbox{}

\end{member}
\begin{member}
\rulename RCNPastsuperdeixis
\ruletask In generation: set value for superdeixis at pastdeixis.\\
               In analysis: set value for superdeixis at omegadeixis.    
\file dutch:cnformation.mrule (mrules69)
\semantics \nosemantics
\example all CNs 
\remarks\mbox{}

\end{member}
\end{members}
\end{mruleclass}
\begin{mruleclass}{RC\_NOUNargmod}
\begin{classdescr}
\kind optional rule class
\classtask To introduce variables for complements to nouns.
\classremarks\mbox{}\\
\begin{enumerate}
  \item 
The application of this rule is constrained by  conditions on the attributes 
{\em 
thetanp} and {\em nounpatterns}. Optional complements are not introduced as 
EMPTYs. Instead it is assumed that the application of 
noun-modification is optional.
  \item
In its present state the RC is provisional: it accounts for 
 varinsertion (only NPVAR), and   caseassignment.
\end{enumerate}
\nofilters

\nospeedrules

\begin{plannedrules}
\item
Rules for nouns with double complements. E.g. {\em request to the government 
for help}.
\end{plannedrules}
\norulesnotince

\rulelist

\end{classdescr}

\begin{members}
   
\begin{member}
\rulename RNOUNargmod1
\ruletask To introduce a variable for a prepositional argument.
\file dutch:npsubgrammars.mrule (mrules67)
\semantics modification
\example (the) answer to NP
\remarks\mbox{}

\end{member}
   
\begin{member}
\rulename RNOUNargmod2
\ruletask To introduce a variable for a sentential complement.
\file dutch:npsubgrammars.mrule (mrules67)
\semantics modification
\example  (tha) fact that .. ; (the) question whether ...
\remarks\mbox{}

\end{member}
\end{members}
\end{mruleclass}

\begin{mruleclass}{RC\_CNmodbareNP}
\begin{classdescr}
\kind optional rule class
\classtask To specify the content of the head NOUN of the CN.
\classremarks

\nofilters

\nospeedrules

\noplannedrules

\norulesnotince

\rulelist

\end{classdescr}

\begin{members}


\begin{member}
\rulename RCNmodbareNP
\ruletask To specify the content of the head NOUN of the CN, by
introducing an NP preceded by {\em of} in postnominal position .
\file dutch:cnformation.mrule (mrules69)
\semantics \nosemantics
\example\mbox{}
\begin{enumerate}
  \item 
bottle (CN) + milk (NP) $\rightarrow$ bottle of milk 
  \item
sack (CN) + potatoes (NP) $\rightarrow$ sack of potatoes
\end{enumerate}
\remarks\mbox{}
Example such as {\em (three) ounces of cheese } 
are presently analysed by RCNmodbareNP 
as well. 
This should be prohibited (with a restriction on the valueset for 
.(act)subcs) as soon as there are rules for the formation of a 
DETP out of a measure NP. 
\end{member}
\end{members}
\end{mruleclass}
\begin{mruleclass}{RC\_CNspecPN}
\begin{classdescr}
\kind optional rule class
\classtask To add a specifying propername to a nominal head.
\classremarks

\nofilters

\nospeedrules

\noplannedrules

\norulesnotince

\rulelist

\end{classdescr}

\begin{members}


\begin{member}
\rulename RCNspecProperName
\ruletask To add a specifying propername to a nominal head.
\file dutch:cnformation.mrule (mrules69)
\semantics \nosemantics
\example \mbox{}
\begin{enumerate}
  \item 
project (CN) + Rosetta (proper name) $\rightarrow$ 
(the ) project Rosetta
  \item
sister (CN) + Margreet (proper name) $\rightarrow$ 
(my) sister Margreet
\end{enumerate}
\remarks\mbox{}
This rule should not be confused with the rule for NP-appositions.
\end{member}
\end{members}

\end{mruleclass}

\begin{mruleclass}{RC\_CNmodification}
\begin{classdescr}
\kind recursive rule class
\classtask Introduction of (restrictive) modifiers to the head of the CN.
\classremarks\mbox{}
This RC contains a rather heterogeneous set of modification rules. 
More detailed documentation on the the subset of modpossrules 
(which deals with possessive modification)
can be found in doc. R413 
(to appear).

This document also dicusses the fact that there is a difference between English 
and Dutch with respect to the interpretation of the two possesive constructions 
available.
In Dutch,
the NPs to which RCNmodposs3  applies (in combination with RNPformation5)
are semantically equivalent
to those derived by a combination of CNmodposs1 and NPformation4; e.g. 
{\em mijn vaders boek},  and its equivalent 
{\em het boek van mijn vader}.
For English, the two counterparts {\em the book of my father} and 
{\em my father's book} should not be considered to be equivalent. 
Cf. doc:R413.
In the present mapping on IL the latter fact 
has not been properly accounted for yet.
\nofilters

\nospeedrules

\noplannedrules

\norulesnotince

\rulelist

\end{classdescr}

\begin{members}


\begin{member}
\rulename RCNmodADJP
\ruletask Modification of a CN by an ADJP. (This involves the 
substitution  of the CNVAR of an ADJPPROP.)
\file dutch:cnformation.mrule (mrules69)
\semantics Substitution of the argument variable of the
ADJPPROP by a CN.
\example x1 nice + book $\rightarrow$ nice book
\remarks\mbox{} 
\end{member}
% vertalingen testen
\begin{member}
\rulename CNmodNUM
\ruletask making a {\em definite} CN out of a CN ('omegadef') and a NUM.
\file dutch:cnformation.mrule (mrules69)
\semantics \nosemantics
\example (those) three  (Spanish) ladies; (his) five easy (dumb) pieces; 
(these) two EN.
\remarks\mbox{}
The value for the CN attribute .definite is made {\em definite} by 
this rule. Definite CNs are excluded from the formation of determinerless NPs.
Hence it excluded that on NP-level the numeral is not preceded by a determiner.
NPs with an initial numeral are derived via NPformation1 and are not
considered modifiers. 

\end{member}
\begin{member}
\rulename CNmodposs1
\ruletask
Introduction of postnominal possessive 'of'-modifiers.
\file dutch:cnformation.mrule (mrules69)
\semantics modification
\example
\begin{enumerate}
  \item
 company + Jan's father $\rightarrow$ company of Jan's father
  \item 
 book + I $\rightarrow$ book of mine
  \item
 book + everyone $\rightarrow$ book of everyone
  \item 
 book + John $\rightarrow$ book of John
\end{enumerate}
\remarks\mbox{}
\end{member}
\begin{member}
\rulename CNmodposs2
\ruletask
Introduction of possessive {\em of}-modifiers preceded by an empty CN head.
\file dutch:cnformation.mrule (mrules69)
\semantics modification
\remarks\mbox{}
\example  
\begin{enumerate}
\item
EN + my father $\rightarrow$ EN of my father
\item
EN + I $\rightarrow$ EN of mine
\item
EN + everyone $\rightarrow$ EN of everyone
\item
 EN + John $\rightarrow$ EN of John
\end{enumerate}

\end{member}
\begin{member}
\rulename CNmodposs3
\ruletask
Introduction of possessive modifiers that end up in prenominal position
via NPformation.
\file dutch:cnformation.mrule (mrules69)
\semantics modification
\remarks\mbox{}
The mapping on IL should be considered in more detail. Cf. the remarks to
this RC.
\example\
\begin{enumerate}
  \item 
book + Jan $\rightarrow$ book Jan
(via NPformation $\rightarrow$ Jan's book)
\item 
company + my mother $\rightarrow$ company my mother
(via NPformation $\rightarrow$ my mother's company)
  \item
book + everyone $\rightarrow$ book everyone
(via NPformation $\rightarrow$ everyone's book)
\item 
book + who $\rightarrow$ boek who
(via NPformation $\rightarrow$ whose company)
\end{enumerate}

\end{member}
\begin{member}
\rulename CNmodPP
\ruletask Modification of a CN by a PREPP. (This involves the
substitution  of the CNVAR of a PREPPPROP.)
\file dutch:cnformation.mrule (mrules69)
\semantics Substitution of the argument variable of the PREPPPROP by a CN.
\example (tall) man {\em with the hat}, (those) EN {\em next to the train}
\remarks\mbox{}

\end{member}

\begin{member}

\rulename RCNmodRELSENT1
\ruletask To modify a CN by an relative sentence. This involves the 
substitution of a CNVAR in sentences by CN.
\file dutch:cnmodification.mrule (mrules109)
\semantics restrictive modification
\example \mbox{}
\begin{enumerate}
% vertalingen testen
\item  the city that I hate
\item  the city of which I dream 
\item  the place where I live
\item  the city I live in 
\item  ???the man of who I dream 
\end{enumerate}
\remarks\mbox{}
\begin{enumerate}
  \item 
A comma is introduced following the modifying sentence
  \item 
There is a special rule for non-restrictive sentential modification:
RnonCNmodrelsent1. This rule generates two comma's: one preceding and one 
following the sentential modifier.
\end{enumerate}
\end{member}
\begin{member}
\rulename CNmodanterel1
\ruletask To modify a CN by an anterelative sentence containing
a present participle.
\file dutch:cnformation.mrule (mrules69)
\semantics modification
\example
child + [x1 walking] $\rightarrow$ [walking child]
\remarks\mbox{}
\end{member}
\end{members}
\end{mruleclass}
\end{document}
