


\documentstyle{Rosetta}
\begin{document}
   \RosTopic{General}
   \RosTitle{Rosetta in WSF}
   \RosAuthor{Jan Landsbergen, Andr\'{e} Schenk}
   \RosDocNr{0294}
   \RosDate{\today}
   \RosStatus{informal}
   \RosSupersedes{-}
   \RosDistribution{Project}
   \RosClearance{Project}
   \RosKeywords{}
   \MakeRosTitle
%
%

\section{Introduction}

In September/October 1988 there have been a number of meetings of a special 
task force, consisting of 
Jaap Berkhoff (TDS/PCG), Carel 
Fellinger (Nat Lab/PCG), Jan Landsbergen (Nat Lab), Rob de Vogel
(PCG) and Andr\'{e} Schenk (Nat Lab). The first of these 
meetings was also joined by Annedore Paeseler (PFH) and Dermot Scallon (PCG). 

The goal of the meetings 
was to define applications of the research in the Rosetta project that 
would fit in with the WSF plans of TDS Predevelopment. This report contains our
conclusions.
 
\bigskip

We selected two applications, the 
electronic
mail translator and the letter generator, to be described in section 2. 
Both applications offer a user the possibility to generate
texts in a language that he does not master, without help from a human 
translator.
This is an ambitious goal. For unrestricted text it
is only feasible if the translation takes place 
in an interactive fashion, where the system may ask the user for additional 
information.


\bigskip

{\bf Quality of translations}

We will indicate the expected quality of translations by means of the terms 
{\bf good}, {\bf fair} and {\bf poor}.

\begin{enumerate}

\item {\bf good}: 
comparable with quality of human translation. Correct syntax, good 
style, very few errors.

\item {\bf fair}: the contents of  the message are conveyed correctly, 
but the form may 
be weak. The sentences are usually syntactically 
correct, but may be 
formulated in a clumsy way. Sometimes the translation is too literal.


\item {\bf poor}: many of the translated sentences are unintelligible or have 
a meaning which is different from the original. What is conveyed is not much 
more than the topic of the text.



\end{enumerate}

The required quality is different for different applications. Some examples:

\begin{enumerate}

\item For business letters a good quality is required.

\item For informal communication, e.g. between colleagues, a fair quality is 
sufficient.

\item For scanning of texts, with the goal to get a fast impression of the 
contents, a poor translation is acceptable. 

\end{enumerate} 

In appendix A the difference between good and fair translations
is illustrated by means of examples.

\bigskip

{\bf Interactive translation}

For translation two kinds of knowledge are needed:
knowledge of language and knowledge of the world. If a system has sufficient 
knowledge of language, but insufficient knowledge of the world may lead to 
incorrect translations, because in case of ambiguities it may make the wrong 
decision.
The aim of the Rosetta project is to develop an interactive system: if a 
sentence contains an ambiguity the system cannot solve, the user will be asked 
to select the intended interpretation. The most frequently occurring 
ambiguities will be lexical ambiguities, caused by the fact that words often 
have more than one meaning.
In that case the system will describe the various meanings to the user - in his 
own language - and the user may select the correct one. 

For some types of users a high frequency of disambiguation interactions may be 
irritating. Therefore it may be useful to try and make an educated guess about 
this frequency here.
The frequency of interactions will depend on the subject domain of the texts.
If this domain is limited, words will be less ambiguous in the first place and 
in addition the system can be provided with information about the restricted 
domain. 
If we define 
the interaction degree as the number of interactions per word 
in the text,
we can make the following estimates. 

For an extremely limited domain (e.g. translation of weather 
bulletins) the remaining degree of interaction may be 0.

For an unrestricted domain, e.g. arbitrary letters, the degree of interaction 
will be approximately 0.5, i.e. about one interaction per content word.

For a moderately limited domain, e.g. tourism, the degree of interaction 
may be something like 0.1 or 0.2. 

These figures are given under the assumption that there will be an interaction
for all ambiguities the system cannot solve with reasonable
certainty. If the user accepts occasional errors, 
he may indicate that he wants less or no 
interactions. 


\section{Description of applications}

\subsection{Brief description}

Two applications have been selected: the
electronic mail translator and the letter generator. 

\begin{itemize}


\item {\bf e-mail translator } 

The e-mail translator is an interactive translation system, tuned to the 
translation of informal messages from the user's language into a foreign 
language. It translates free text and yields a fair translation quality,
which is quite acceptable 
for this type of informal communication.


The degree of disambiguation interaction is high for unrestricted text, but can 
be reduced if the user indicates the subject domain(s) of his message.


This application fits in well with the PICA plans in WSF. PICA is
a communication device of which one of the important goals is to
enhance the reachability of people.
It is already envisaged to extend the current possibility of the answer-phone
with a facility to send short typed messages. E-mail translation would be an 
extension of this facility and give the opportunity of communication 
between people speaking different languages. 

The e-mail translator can be extended to  a 
`conversation translator', a
multilingual communication system, 
which allows a user to carry a 
`conversation' with a person which speaks a different 
language. (The conversation is in typewritten form, but this 
can be seen as a first step towards speech-to-speech 
translation)

Even between people who are able to communicate - with difficulty - in 
a common foreign language
(e.g. Philips English),
it will be useful to use this system as a fall-back during oral communication,
e.g. for  confirmation of appointments.

\bigskip


\item {\bf Letter generator}

A letter generator (more precisely: a foreign language text generator) 
serves primarily for producing semi-standard letters or other 
documents in a foreign language. It is also useful for generation of 
monolingual documents.

Directed by the system, the user composes a letter in his own language, by
choosing from the possibilities offered to him by a kind of menu system. 
He may also insert free 
phrases at well-defined places in the text. The system may interrogate the
user about ambiguities in these 
phrases, in his own language. 
When
the letter is ready, its translation is generated without further
interaction of the user.
So, in this way 
letters and documents in a foreign 
language can be generated by a person who does not know that language.
In appendix B an example scenario for the letter generator is given.

Because
 the system has an explicit 
representation of the linguistic structure of the standard part of the 
text, it is possible to 
allow a large amount of freedom and still generate a translation of good
quality.


\bigskip

Standard letters are already  frequently used in offices, usually for one 
language, but also for foreign languages. Up till now these letters have 
a very rigid format, cf. the examples in appendix C. 
With the linguistic techniques developed in the Rosetta project 
 the flexibility and quality of monolingual 
standard letters can be 
considerably improved. Cf. appendix D for examples.
This may be a very useful spin-off, but the main motivation for 
using these linguistic techniques is in behalf of the generation of 
foreign language letters. 
For purely monolingual use the Rosetta-based 
approach may be a bit costly; at first sight it seems that
 an acceptable (although lower) flexibility and quality
could be achieved by simpler - ad hoc - means.



\end{itemize}


\subsection{Standard letters}

From the point of view of the letter generator, 
we can distinguish four classes of letters, with an increasing need
for linguistic knowledge. 

\begin{enumerate}

\item Letters that consist completely of a standard text. So, everything 
except the sender and the addressee is fixed. 


\item Parameterized standard letters. In these letters there is a variation 
which can be described by a finite number of parameters. The user has to choose 
between a number of parameter values for determining the final text. 

Examples of parameters: 

- the choice between two or more explicit words, phrases or 
full sentences (Current standard letters 
often consist of a list of standard sentences, of which the relevant ones
have to be marked). The choice may also have consequences for other parts of 
the text.

- specification of a number or a name (not to be translated)
at a well-defined place in the text, e.g.
the number of beds, in a letter on hotel reservation.

- parameters which do not  refer to actual parts of the text, but which do have 
consequences for the text, for example 
specification of the sex of the addressee or the sender, which may have 
consequences for e.g. the participles in Spanish.
Another example might be a parameter that influences the general style of 
the letter, e.g.
a parameter for indicating whether the letter is formal or informal, personal 
or impersonal, in the `I' or the `we' form, etc.


\item Letters with parameters as in 2), but also with open 
slots where more or less 
free phrases 
can be inserted, which are translated by the system (with interactive 
disambiguation, if necessary). 
Somehow it is indicated what variety the system 
allows in these free phrases, explicitly or implicitly, by the
context or by a suggestive example. 


\item Letters which consist completely of free text, to be 
translated by the system (interactively).

\end{enumerate}

This enumeration might suggest that the letter generators of class 2 or 3 
are only intermediate steps towards the final goal of free text translation in 
class 4, which coincides more or less with the e-mail translator. 
This suggestion is not correct, although there is certainly a growth path 
from letter generator to 
e-mail translator, especially for the 
linguistic work. 
In many circumstances semi-standard letters 
will be preferred to free letters, 
because they can be generated more conveniently and efficiently. Furthermore the
letter generator is able to compose letters of good quality, needed for 
business letters and other formal documents. In the foreseeable future this 
quality cannot be attained by automatic translation of 
completely free texts, for which only a fair quality 
is possible. 

So the goal of the letter generator activity
is a system of class 3, with the right mixture 
of standard and free parts.

\bigskip

\subsection{Generation of letter generators}

Let us call a particular instance of a letter generator a {\em template} (with 
parameters and open slots to be specified by the end user). A large variety of 
templates will be needed, 
since each office may require its own type of standard
letters. 

Two approaches are possible. Either the templates themselves 
are Philips products or Philips sells  a {\em 
template composer}, a tool that enables a specialized user to make
new templates.

In the latter case there are two kinds of users: 
 the {\em template builder} (similar to the {\em case builder} in ECHO), 
who uses the template 
composer for making new templates and there is
the end user (the {\em case worker} in ECHO terms), 
who uses the templates for generating letters.
 
The template builder 
should know the source and the target language,
but we cannot expect him to know the internal details of the letter generator,
e.g. the grammar rules of the system. As far as we can foresee 
now, it will 
not be possible to make a template composer in the next two years.
One of the things that should be done 
first is writing sufficiently 
powerful grammars tuned to formulations in business 
letters, so that the translations desired by the template builder 
can indeed be generated by these grammars. Therefore we propose to start with
making templates ourselves and in parallel do preparatory work 
for the template 
composer.


\subsection{Other writer support}

A letter generator offers a special kind of writer support. One may wonder 
whether the Rosetta work can also be used for other kinds of support, such as 
spelling and syntax checking. To a certain extent this is already the case.
The free phrases inserted by the end user into standard letters, will be 
analyzed by the letter generator 
in order to make their translation possible. This
implies that the system functions as a spelling and syntax checker for
these phrases. For the e-mail translator this holds for the complete text.

In principle the morphological and syntactic components of Rosetta could also 
be used for spelling  and syntax checking of other texts, not as a 
side-effect 
of translation but as a goal in itself. However, this is less straightforward
than it may seem, for the following reasons.

\begin{enumerate}

\item The morphological components of Rosetta have been written 
with a different goal and are therefore more complex than necessary for 
spelling checking. The morphological rules yield a complete 
morphological analysis of each word and even more than one if the word can be 
analyzed in more than one way. 
For spelling checking purposes simpler rules could be used,
or no rules at all, but an explicit list of word forms.  
Using the full Rosetta power would give a slightly better functionality, but at 
a considerable cost in terms of time and space.



\item Similar remarks can be made with regard to the use of Rosetta's syntactic 
components for syntax checking. For syntax checking a broad coverage of the
grammars is more 
important than the deep analysis needed for translation. However, the first 
part of our syntactic analysis, the surface parser, could in principle be used 
for this purpose, if its syntax would be extended.

\item What is usually wanted, is not merely spelling and syntax checking, but 
also {\em correction} (or: suggestions for correction). We have no experience 
with this, so this would require new research.


\end{enumerate}


\section{Time schedules}


We present here two possible time schedules: one for the development of
the e-mail translator alone and one for the development of the letter 
generator, 
possibly followed by the e-mail translator.

The letter generator has the advantage that useful results can be provided
on a relatively short term, say 2 years. Even a linguistically modest
letter generator can be useful. The coverage and the 
flexibility of such a system can be gradually extended. 

The e-mail translator is a more ambitious and challenging application, 
which requires more time to arrive at an acceptable
functionality.
On the other hand, the contribution from 
research can be larger in this case, because the work links
up much better with the research activities. 


\begin{itemize}


\item E-mail translator 


1989 - 1991/1992

Joint project TDS/Nat Lab on e-mail translator.

Results: 

after 3 years: translators for Dutch-English and English-Dutch,

after 4 years: translators for Dutch-Spanish, Spanish-Dutch, English-Spanish,
Spanish-English.

Activities for other languages (German, French, Italian, Swedish) can be added,
afterwards or in parallel. 

\item Letter generator 


1989 - 1990

\begin{enumerate}

\item Joint project TDS/Nat Lab on letter generator.

Result: a prototype letter generator for one language pair. 


\item In parallel on-going research
on translation, further development of grammars, tuned to e-mail 
translation and template composer. 

\end{enumerate}

1991 - ... 

Further development of letter generator and template composer by TDS.

Possibly, a joint project TDS/Nat Lab on e-mail translator.
The results will 
become available about one year later than in the first scenario.

\end{itemize}





\section{HAMT versus MAHT}


Both the letter generator and the e-mail translator are systems for human-aided
machine translation (HAMT), intended for users who are not 
professional translators. 
It should be noted that we
could have chosen the applications from a different 
area, called machine-aided human translation (MAHT). 
There the goal is to make 
the translation process more efficient by offering tools to 
the professional translator. These tools can range from terminology banks to 
full-fledged translation systems. In the latter case the 
system gives - fully automatically - 
a rough translation of the text, which has to be `post-edited'
 by the 
human translator. Most translation systems developed thus far are 
intended to be used in
such a post-edit environment. They are usually tuned to a special kind of 
technical texts,
e.g. computer documentation, and have large dictionaries for the terminology in 
these texts. 

This application area has the advantage that there is a well-defined market.
However, we had two reasons for not selecting MAHT:

(i) Competition. There are already  commercial 
systems available for helping professional translators.
So Philips would enter this market with
a serious time-lag.
In the selected areas there is no competition yet. 

(ii) Research strength. The Rosetta 
research thus far has been focused on the linguistic aspects of translation,
which are more important for interactive HAMT than for MAHT with post-editing.  
The current MAHT systems usually have large terminology banks, but simple 
grammars.  There certainly is a 
growth path from HAMT to MAHT; in the reverse direction this is much more 
difficult, because the transition from a simple grammar to a sophisticated 
grammar would
require a complete revision of the system, including the dictionaries.


\section{Appendices}

Appendix A. Examples of good and fair translations.

Appendix B. Scenario for the letter generator (demonstration).

Appendix C. Various types of standard letters.

Appendix D. The necessity of linguistic power in standard document generation.



\end{document}

