
\documentstyle{Rosetta}
\begin{document}
   \RosTopic{General}
   \RosTitle{Notulen Linguistenvergadering 20-09-88}
   \RosAuthor{Margreet Sanders}
   \RosDocNr{0280}
   \RosDate{September 27, 1988}
   \RosStatus{informal}
   \RosSupersedes{-}
   \RosDistribution{Linguists, Joep Rous}
   \RosClearance{Project}
   \RosKeywords{minutes, Mrule notation, transfer}
   \MakeRosTitle
%
%
\begin{description}
\item[Aanwezig:] Lisette Appelo, Franciska de Jong, Elly van Munster,
                 Jan Odijk, Margreet Sanders (not),
                 Andr\'{e} Schenk, Harm Smit
\item[Agenda:]\mbox{}
  \begin{enumerate}
  \item Polariteit
  \item Mregel-notatie
  \item Transfer
  \item W.v.t.t.k.
  \end{enumerate}
\end{description}

\section{Polariteit}
Jan Odijk heeft een functie geschreven voor het beregelen van polariteit. Deze 
functie heeft onder andere tot gevolg dat voor alle categorie\"{e}n boven het 
lexicale niveau de attributen {\bf req} en {\bf env} overbodig zijn geworden. 
In Mregels hoeft hier dus niet meer naar verwezen of op getest te worden. Op de 
volgende projectvergadering, maandag{\bf morgen} 27 september, zal Jan O.\ deze 
functie toelichten.

\section{Mregel-notatie}
Er zijn een aantal nieuwigheden. De twee voornaamste daarvan zijn:
\begin{enumerate}
\item  In de condities kan een expressie van de vorm 
((NOT b1) OR b2), met b1 en b2 Booleaanse expressies, \\
voortaan worden geschreven als 

b1 $\rightarrow$ b2.\\
Dit heeft als voordelen dat de notatie doorzichtiger wordt en dat er meer kan 
(NOT kon niet bij alle booleaanse expressies); verder is het zo 
ge\"{i}mplementeerd dat b2 alleen ge\"{e}valueerd wordt als b1 
inderdaad TRUE oplevert, zodat b2 dus voor gevallen dat b1 FALSE is niet pers\'{e} 
hoeft te bestaan.
\item In alle CA-paren moet de string {\em AUX\_...key\/} vervangen 
worden door de 
string {\em KEY\_...key\/}. In de modellen mag {\em AUX\_...key\/} nog blijven 
staan maar ook daar moet het op den duur 
worden vervangen. Bij gebruik van 
keynamen die nog niet in de file {\bf skeydef.kdf} van de betreffende taal 
staan gedefinieerd geeft het systeem een waarschuwing.
\end{enumerate}

\section{Transfer}
De strategie die door Jan O.\ wordt gevolgd komt erop neer dat elke taal zijn 
regels \'{e}\'{e}n op \'{e}\'{e}n op een unieke ILregel afbeeldt, en 
daarnaast eventueel in generatie ook nog op een andere regel, met verlaging van 
de bonus. Deze bonus wordt genoteerd direct achter de ILregelnaam, in de 
volgende vorm: {\bf WBONUS-1} voor een weak bonus, dus een vertaling die niet 
fout is maar soms wel minder mooi, en {\bf SBONUS-1} voor een foute vertaling 
die soms als noodoplossing geaccepteerd moet worden.

Verder kan in transfer nu ook een integer als parameter worden meegegeven. Dit 
geval doet zich voor 1) bij het onthouden van de index van het antecedent in 
recipro-substitutie en 2) (in generatie) bij het markeren van de variabele die 
geshift moet worden omdat hij gaat relativiseren. Normaal worden de waarden van 
een parameter opgesomd in transfer, behalve bij de integer LEVEL, waar het 
systeem zelf voor de goede transfer zorgt. Voor andere integer parameters is nu 
de variabele {\bf X1} ingevoerd, die zowel bij de betreffende Mregel als bij de 
ILregel meegegeven moet worden als parameter. Een transferregel ziet er dan als 
volgt uit: \\
RelkaarSubst\{(*LEVEL*) antecedent=X1\} $\leftrightarrow$
 LReciproSubst\{(*LEVEL*) antecedent=X1\}. \\
Mochten er meer van zulke parameters nodig zijn in een 
regel, dan kan ook nog gebruik gemaakt worden van {\bf X2} en {\bf X3}.


\section{W.v.t.t.k.}
\begin{description}
\item [Keydefinitions:] In document 279 van Harm en Joep staat de nieuwe notatie 
voor skeys, mkeys en fkeys. Op de volgende linguistenvergadering kan hierop 
commentaar worden geleverd.
\item [Modals:] Jan O.\ deelt een (ongenummerd) document uit met een voorstel 
voor de vertaling van modals. Dit moet op de volgende vergadering uitgebreid 
worden besproken. Eventule achtergrond en eerdere voorstellen zijn ook te 
vinden in doc.\ 162 (Miscellaneous Topics) en doc.\ 263 (...Temporal 
Expressions).
\item [Voortgang:] Op de volgende project-vergadering moet ook weer een 
voortgangsverslag worden gegeven. Een ieder wordt aangeraden om eens te 
inventariseren hoe~veel en welke regels er nog geschreven moeten (kunnen) 
worden voor 1 november, en hoeveel er nog getest en/of verbeterd moet worden.
\end{description}

\end{document}

