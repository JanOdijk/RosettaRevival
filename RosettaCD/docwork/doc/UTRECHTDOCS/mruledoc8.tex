\documentstyle{Rosetta}
\begin{document}
   \RosTopic{Rosetta3.doc.Mrules.English}
   \RosTitle{Rosetta3 English M-rules: PREPPFORMULAtoPROP}
   \RosAuthor{Margreet Sanders}
   \RosDocNr{382}
   \RosDate{November 15, 1989}
   \RosStatus{approved}
   \RosSupersedes{-}
   \RosDistribution{Project}
   \RosClearance{Project}
   \RosKeywords{English, documentation, Mrules, PREPPFORMULAtoPROP}
   \MakeRosTitle
%
%

\section{Introduction}
As all main category grammars, the English PREPPPROP grammar is divided in 
three parts. First, {\bf PREPPPROPformation} forms the PROP-structure. Then, {
\bf PREPPPROPtoFORMULA} turns this PROP into an intermediate structure, called 
PREPPFORMULA
(comparable to the CLAUSE in the sentence grammar). Finally, {\bf 
PREPPFORMULAtoPROP} makes an open or closed PROP-structure, comparable to the 
SENTENCE level in the sentence grammar. Prior to PREPPPROPformation, there is 
the {\bf PrepDerivation} grammar, which was discussed in doc.\ 316, {\em 
Rosetta3 English M-rules: Derivation Subgrammars\/}. The final open or closed 
PREPPPROP is usually input to the Proposition Substitution Rules in the 
XPPROPtoCLAUSE subgrammar. A `bare' PREPPPROP may also form an expression on 
its own with help of the copula {\em be\/}. In that case, it leaves the 
PREPPPROP grammars after PREPPPROPformation, and 
is input to the ClauseFormation Rules of the XPPROPtoCLAUSE subgrammar. 
There is no rule to make a complete Utterance of a closed PREPPPROP.

The current document describes the contents of the third PREPPPROP subgrammar, 
PREPPFORMULAtoPROP (the first subgrammar was discussed in doc.\ 380, {\em 
Rosetta3 English M-rules: PrepppropFormation\/}, and the second one 
in doc.\ 381, {\em Rosetta3 English Mrules: PREPPPROPtoFORMULA\/}). 
The subgrammar consists of 
a number of rule classes and transformation classes. A rule class in its turn
consists of a number of rules and a transformation class of a number of 
transformations. The relative ordering of the rules and transformations in the
(sub)grammar is indicated by a {\em control expression}. A summary of this
control expression (i.e.\ a listing of the ordering of the rule classes, 
without explicit mentioning of the rules themselves) is also included here, 
and the initial (= head), import and export categories are given. 

In the section on the rules and transformations, only the rule names are given, 
but not the exact rule formulation. What is attempted 
is to provide a detailed overview of the workings of the subgrammar, and 
how the different rule classes achieve this,
together with some comments on the problems still to be solved, the reasons 
behind certain choices, and perhaps possible alternatives. For every rule, an 
example is given. If it is uncertain whether the example is correct (either 
because it may not be an example of the phenomenon in question, or because it 
may not be correct English), it is preceded by a question mark. Note that all 
explanation of rules and transformations is given from a generative viewpoint
only, unless explicitly stated otherwise. Often, the information given in this 
document is based strongly on the comment already present in the documentation 
of the rules themselves. Discrepancies between what is stated here and what is 
said in the rule itself are usually caused by the fact that the rule file has 
not  been updated, although insights have changed. The semantics of the rules 
has been left unspecified in the current documentation, since it is not at all 
clear.

Whenever the current implementation differs widely from the strategy that was 
devised in the definition phase of Rosetta3 (as laid down for English in docs.\ 
150, {\em Subgrammars of English\/}, 153, {\em Rule and Transformation Classes 
of English\/}, and 155, {\em Rule and Transformation Classes common to all 
languages\/}, all written by Jan Odijk), this will be indicated explicitly in 
the current document. Conditions on crucial orderings of rule classes will be 
repeated here, even if they do not differ from the original strategy, to make 
the document as self-contained as possible.

Finally note that the rules described in this document have NOT been tested 
properly. English analysis is not possible yet (there is no Surface Parser), and 
English generation has only been tested in as far as the construction was the 
translation of a Dutch sentence to be tested.

\newpage
\section{PREPPFORMULAtoPROP}
The exact contents of the PREPPFORMULAtoPROP subgrammar are mainly determined 
by the requirements of isomorphy with the CLAUSEtoSENTENCE subgrammar. Thus, 
many rule classes just provide some sort of 
`default' value for the category to be used when it is substituted into the 
higher sentence. The subgrammar is only partially isomorphic with the 
CLAUSEtoSENTENCE subgrammar: for some optional rule classes present there 
(viz.\ RelMarking, PosNegVar and ConjSent), no counterparts exist in the 
present subgrammar. This means that only if no rules from these classes are 
present in the derivation tree, can an isomorphic derivation be found in the 
current subgrammar.

In doc.\ 150, it was assumed that the current subgrammar would also have rules 
for reciprocal and reflexive spelling. However, since the antecedent of these 
elements will usually be in the higher sentence, it seems doubtful whether 
these rules are useful here. They have not been written. For the rest, the 
contents of the subgrammar more or less agree with the definition in doc.\ 150.

\section{Grammar Specification}
The grammar definition can be found in the file which also contains all the 
rules of this subgrammar, {\bf PreppFormulaToProp.mrule}, 
which is {\em mrules88.mrule\/}.

\begin{verbatim}
%SUBGRAMMAR PreppformulaTOprop


   ( TC_pppPROstatus)
.  { RC_pppSubstitution }
.  ( TC_pppAspectneutralization )
.  ( RC_pppMood )
.  ( RC_pppPunc )

\end{verbatim}

\begin{description}
  \item[Head]  PREPPFORMULA  \ \ \ \ FROM (PREPPPROPtoFORMULA)
  \item[Export] OPENPREPPPROP, CLOSEDPREPPPROP
  \item[Import] NP, PREPP, ADVP
\end{description}

\newpage
\section{Rules and Transformations}

\subsection{TC\_pppPROstatus}
\begin{description}
\item[Kind] Obligatory Transformation Class
\item[Task] To assign a value to the attribute {\bf PROsubject}. The default 
value is {\em false\/}. In generation, there is a free choice between the two 
rules, and both an OPENPREPPPROP (PROsubject = {\em true\/}) and a 
CLOSEDPREPPPROP (PROsubject = {\em false\/}) can be made. If a subject 
substitution rule is applied (see below), only the version with PROsubject = {
\em false\/} is allowed; hence, the transformation class is ordered crucially 
before the substitution rules.

\vspace{1 cm}
\begin{description}
\item[Name] TpppNoPROsubj
\item[Task] Vacuous rule, leaving the {\bf PROsubject} attribute of the 
PREPPFORMULA at its default value, which is {\em false\/}.
\item[File] english:PreppformulaToProp.mrule (mrules88.mrule)
\item[Semantics] --
\item[Example] [x1 in x2]$_{PROsubject=false}$ $\rightarrow$ 
[x1 in x2]$_{PROsubject=false}$ (She seemed in her early 
twenties) (with {\em seem\/} as one-place verb)
\item[Remarks] 
\end{description}

\vspace{1 cm}
\begin{description}
\item[Name] TpppPROsubj
\item[Task] To set the {\bf PROsubject} attribute of the 
PREPPFORMULA at the value {\em true\/}.
\item[File] english:PreppformulaToProp.mrule (mrules88.mrule)
\item[Semantics] --
\item[Example] [x1 in x2]$_{PROsubject=false}$ $\rightarrow$ 
[x1 in x2]$_{PROsubject=true}$ (She seemed in her early 
twenties) (with {\em seem\/} as two-place verb)
\item[Remarks] 
\end{description}

\end{description}


\newpage
\subsection{RC\_pppSubstitution}
\begin{description}
\item[Kind] Iterative Rule Class
\item[Task] To substitute non-sentential expressions for their variable, in the 
order dictated by analysis and consistent with the substitution order 
conditions.

The rules use the system parameter LEVEL to check whether the index of the 
variable is consistent with (the level of) the variable that is to be 
substituted for according to the rule parameter.

For more information on the rule class, see the documentation to the comparable 
class in the CLAUSEtoSENTENCE subgrammar (doc.\ 370, p.\ 14). The same 
constraints hold, except that reflexives etc.\ have not yet been excluded from 
the current rules explicitly (needed to speed up analysis). 
Also, the substorder constraint is only mentioned 
for subject substitution, since it is not needed for substitution in the PREPP 
(there only is one argument there anyway).

In the NP substitution rules, the value of the attribute {\bf generic} is set 
to {\em omegageneric\/}, because the Surface Parser cannot decide on genericity.
Also, the value of the attribute {\bf superdeixis} is set to {\em 
omegadeixis\/}, for the same reason.

There still is no way to limit the workings of the substitution rules in this 
grammar to those cases where it is really needed, i.e.\ where there will not be 
any substitution in the higher clause. Thus, many ambiguities arise caused by
substitution having applied in different cycles. See section 5 of doc.\ 368, 
{\em Scope in Rosetta3\/}, for a discussion of this problem.

\vspace{1 cm}
\begin{description}
\item[Name] RpppSubjSubst
\item[Task] To substitute a non-generic NP for its subjrel VAR
\item[File] english:PreppformulaToProp.mrule (mrules88.mrule)
\item[Semantics]
\item[Example] [x1 against the analysis] + the man $\rightarrow$ [the man 
against the analysis] (The man seemed against the analysis)
\item[Remarks] This rule was added because in Spanish, some constructions never 
get a subject in the main clause (there only is something like {\em It seemed 
that the man was against the analysis\/}).

The rule is practically a copy of an older version of the subject substitution 
rule in the 
sentence grammar. Hence, it is restricted to non-generic NPs. Once generic NPs 
can be made, it should be decided whether the rule can simply be parametrized 
for genericity, or whether an extra rule (containing more restrictions on the 
model) is needed. 
\end{description}

\vspace{1 cm}
\begin{description}
\item[Name] RpppNPSubst
\item[Task] To substitute a non-generic NP for its objrel VAR
\item[File] english:PreppformulaToProp.mrule (mrules88.mrule)
\item[Semantics]
\item[Example] [x1 behind x2] + the door $\rightarrow$ [x1 behind the door]
\item[Remarks]
\end{description}

\vspace{1 cm}
\begin{description}
\item[Name] RpppPREPPSubst
\item[Task] To substitute a PREPP for its objrel VAR
\item[File] english:PreppformulaToProp.mrule (mrules88.mrule)
\item[Semantics]
\item[Example] [x1 from x2] + behind the door $\rightarrow$ [x1 from behind 
the door]
\item[Remarks]
\end{description}

\vspace{1 cm}
\begin{description}
\item[Name] RpppADVPSubst
\item[Task] To substitute an ADVP for its objrel VAR
\item[File] english:PreppformulaToProp.mrule (mrules88.mrule)
\item[Semantics]
\item[Example] ? [x1 until x2] + tomorrow $\rightarrow$ [x1 until tomorrow]
\item[Remarks] Perhaps this is the wrong example, since temporal expressions 
are considered PREPPs, and not PREPPPROPs. If so, there may be no need for the 
current rule.
\end{description}

\end{description}


\newpage
\subsection{TC\_pppAspectNeutralization}
\begin{description}
\item[Kind] Obligatory Transformation Class
\item[Task] To reset the attribute {\bf aspect} of the PREPPFORMULA, which was  
set at the value {\em imperfective\/} in the Aspect Rule Class (see previous 
subgrammar), to the 
value {\em omegaaspect\/}. This is needed because the Surface Parser cannot 
decide the correct value in analysis, while the Aspect Rule Class is an 
obligatory class in the sentence grammar.

\vspace{1 cm}
\begin{description}
\item[Name] TpppAspectNeutralization
\item[Task] see above
\item[File] english:PreppformulaToProp.mrule (mrules88.mrule)
\item[Semantics] --
\item[Example] [x1 behind x2]$_{imperfective}$ $\rightarrow$ [x1 behind 
x2]$_{omegaaspect}$ (He hid behind the door)
\item[Remarks]
\end{description}

\end{description}

\newpage
\subsection{RC\_pppMood}
\begin{description}
\item[Kind] Obligatory Rule Class
\item[Task] To turn a PREPPFORMULA into a closed or open PREPPPROP, depending 
on the value of the attribute {\bf PROsubject}. This rule is the 
`XPPROP-version' of the Mood Rules in the sentence grammar, but there are no 
new attributes going with the new topnode in the current subgramamar.

\vspace{1 cm}
\begin{description}
\item[Name] ROpenPPPMood
\item[Task] To turn a PREPPFORMULA into an OPENPREPPPROP
\item[File] english:PreppformulaToProp.mrule (mrules88.mrule)
\item[Semantics] 
\item[Example] $_{PREPPFORMULA}$[x1 out] $\rightarrow$ $_{OPENPREPPPROP}$[x1 
out]
\item[Remarks]
\end{description}

\vspace{1 cm}
\begin{description}
\item[Name] RClosedPPPMood
\item[Task] To turn a PREPPFORMULA into a CLOSEDPREPPPROP
\item[File] english:PreppformulaToProp.mrule (mrules88.mrule)
\item[Semantics] 
\item[Example] $_{PREPPFORMULA}$[the ball out] $\rightarrow$ 
$_{CLOSEDPREPPPROP}$[the ball out] (The ball seemed out)
\item[Remarks] 
\end{description}

\end{description}

\newpage
\subsection{RC\_pppPunc}
\begin{description}
\item[Kind] Obligatory Rule Class
\item[Task] Vacuous rule, to serve as counterpart in the isomorphic scheme for 
the punctuation rules in the sentence subgrammar. 

\vspace{1 cm}
\begin{description}
\item[Name] RpppNoPunc
\item[Task] see above
\item[File] english:PreppformulaToProp.mrule (mrules88.mrule)
\item[Semantics] 
\item[Example] the ball out
\item[Remarks]
\end{description}

\end{description}


\end{document}

