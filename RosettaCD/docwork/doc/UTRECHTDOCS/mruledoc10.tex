\documentstyle{Rosetta}
\begin{document}
   \RosTopic{Rosetta3.doc.Mrules.English}
   \RosTitle{Rosetta3 English M-rules: ADVPPROPformation}
   \RosAuthor{Margreet Sanders}
   \RosDocNr{386}
   \RosDate{\today}
   \RosStatus{concept}
   \RosSupersedes{-}
   \RosDistribution{Project}
   \RosClearance{Project}
   \RosKeywords{English, documentation, Mrules, ADVPPROPformation}
   \MakeRosTitle
%
%

\section{Introduction}
As all main category grammars, the English ADVPPROP grammar is divided in 
three parts. First, {\bf ADVPPROPformation} forms the PROP-structure. Then, {
\bf ADVP\-PROPtoFORMULA} turns this PROP into an intermediate structure, called 
ADVPFORMULA
(comparable to the CLAUSE in the sentence grammar). Finally, {\bf 
ADVPFORMULAtoPROP} makes an open or closed PROP-structure, comparable to the 
SENTENCE level in the sentence grammar. Prior to ADVPPROPformation, there is 
the {\bf AdvDerivation} grammar, which was discussed in doc.\ 316, {\em 
Rosetta3 English M-rules: Derivation Subgrammars\/}. The final open or closed 
ADVPPROP usually is input to the Proposition Substitution Rules in the 
XPPROPtoCLAUSE subgrammar. An OPENADVPPROP may also be used as VP-modifying 
adverbial (see RC\_VPAdv in XPPROPtoCLAUSE). Theoretically, a `bare' ADVPPROP 
might form an expression on 
its own with help of the copula {\em be\/}. In that case, it would leave the 
ADVPPROP grammars after ADVPPROPformation, and 
be input to the ClauseFormation Rules of the XPPROPtoCLAUSE subgrammar. 
However, no example of a predicatively used adverb has been found yet, and the 
relevant clause formation rule has not been written. In English, there
also is no rule to make a complete Utterance of a closed ADVPPPROP.

The current document describes the contents of the first ADVPPROP subgrammar, 
ADVPPROPformation (the other two subgrammars are discussed in docs 387 and 
388). The subgrammar consists of 
a number of rule classes and transformation classes. A rule class in its turn
consists of a number of rules and a transformation class of a number of 
transformations. The relative ordering of the rules and transformations in the
(sub)grammar is indicated by a {\em control expression}. A summary of this
control expression (i.e.\ a listing of the ordering of the rule classes, 
without explicit mentioning of the rules themselves) is also included here, 
and the initial (= head), import and export categories are given. 
Conditions on crucial orderings of rule classes, if they exist in the current 
subgrammar, are mentioned explicitly.

In the section on the rules and transformations, only the rule names are given, 
but not the exact rule formulation. What is attempted 
is to provide a detailed overview of the workings of the subgrammar, and 
how the different rule classes achieve this,
together with some comments on the problems still to be solved, the reasons 
behind certain choices, and perhaps possible alternatives. For every rule, an 
example is given, if one can be found. If 
it is uncertain whether the example is correct (either 
because it may not be an example of the phenomenon in question, or because it 
may not be correct English), it is preceded by a question mark. Note that all 
explanation of rules and transformations is given from a generative viewpoint
only, unless explicitly stated otherwise. Often, the information given in this 
document is based strongly on the comment already present in the documentation 
of the rules themselves. Discrepancies between what is stated here and what is 
said in the rule itself are usually caused by the fact that the rule file has 
not  been updated, although insights have changed. The semantics of the rules 
has been left unspecified in the current documentation, since it is not at all 
clear. Note that most of the present document is an adapted copy of the 
document on the isomorphic PREPPPROPformation subgrammar.

Finally note that the rules described in this document have NOT been tested 
properly. English analysis is not possible yet (there is no Surface Parser), and 
English generation has only been tested in as far as the construction was the 
translation of a Dutch sentence to be tested.

\newpage
\section{AdvppropFormation}
The exact contents of the AdvppropFormation subgrammar are mainly determined 
by the requirements of at least partial
isomorphy with the other XPPROPformation subgrammars. In theory, translation 
from adjectives to adverbs must be possible too; especially for Spanish, this 
seems necessary in the translation of Dutch constructions like {\em Het klinkt 
mooi\/} (claiming {\em mooi\/} as an adjective) to a Spanish version with 
obligatorily has an adverb. (Perhaps English has the same problem: {\em It 
sounds beautiful\/} is correct, but {\em It sounds beautifully\/} seems 
possible too.) However, since adverbs are derived from adjectives, they have an 
extra level in the derivation tree (SUBADJ $\rightarrow$ SUBADV), and hence, 
isomorphy is not achieved yet. For English, the requirements of isomorphy have 
focussed on the PREPPPROPformation subgrammar, since it may be necessary to 
translate PREPs in ADVs (again with the same problem of the extra level, at 
least if the BADV is not in the dictionary).
Thus, some rule classes have only one rule, providing some sort of 
`default' value, just as in the PREPPPROPformation subgrammar. 
Transformations are hardly needed, since the surface structure 
and many attribute values will be determined by the sentence the PROP is 
substituted in. 

Since the ADVPPROP may be pruned in the sentence grammar, just leaving an
ADVP, the attributes of the ADVP are filled (for the most part, they are a 
copy of the ADV attributes). Cf.\ the VERBP-level in the sentence grammars, 
which has no such status and is not filled in the VERB startrules.

In doc.\ 150, {\em Subgrammars of English\/}, where the plans for
the general lay-out of the translation system were presented, the 
ADVPPROPformation grammar was not explained in any detail. Instead, reference 
was made to an earlier proposal, in doc.\ 103 by Franciska 
de Jong, {\em The organisation of the ADVPsubgrammar\/}. Since in that document
no clear distinction is made between the ADVP subgrammar and the 
ADVPPROP formation
subgrammar, I will not compare the original plans and the current 
implementation. Instead, similarities and discrepancies between the current 
subgrammar and the PREPPPROPformation subgrammar have been indicated.
For more information on the decision to 
separate ADVP and ADVPPROP subgrammars, and the exact working of the insertion 
of an ADVPPROP into a CLAUSE, see the document on the Treatment of Adverbs by 
Jan Odijk (to appear).

No rules have been written for the {\em very - very much\/} alternation,
described in doc.\ 103, nor are there any other rules for modification of the 
adverb. It is not impossible that more rules have to be added in future.

\newpage
\section{Subgrammar Specification}
The subgrammar definition can be found in the file which also contains all the 
rules of all ADV subgrammars, {\bf AdvpSubgrammars.mrule}, 
which is {\em mrules84.mrule\/}.

\begin{verbatim}
%SUBGRAMMAR AdvppropFormation


   ( RC_StartAdvppRules )
.  ( TC_AdvPatterns )
.  { RC_AdvTempVar }
.  { RC_AdvSentAdvVar }
.  { RC_AdvLocAdvVar }
.  ( RC_AdvVoiceRules )

\end{verbatim}

\begin{description}
  \item[Head]  SUBADV  \ \ \ \ FROM (AdvDerivation)
  \item[Export] ADVPPROP
  \item[Import] NPVAR, CNVAR, ADJPPROPVAR, ADVPPROPVAR, NPPROPVAR, 
PREPPPROPVAR, VERBPPROPVAR, SENTENCEVAR, EMPTYVAR, PROSENTVAR, 
ADVPVAR, PREPPVAR, CLAUSEVAR
\end{description}

\newpage
\section{Rules and Transformations}

\subsection{RC\_StartAdvppRules}
\begin{description}
\item[Kind] Obligatory Rule Class
\item[Task] To provide a SUBADV with its correct number of argument variables 
and build an ADVP and ADVPPROP around it. 

In the rules, the Aktionsarts of the ADVPPROP are set to {\em [stative]\/};
no separate transformation class is needed to to determine this `default' 
value.

\vspace{1 cm}
\begin{description}
\item[Name] RStartAdvpprop100
\item[Task] To provide a PROP structure for a SUBADV that takes one subject 
argument, and no arguments in the ADVP (i.e.\ an ordinary adverb)
\item[File] english:AdvpSubgrammars.mrule (mrules84.mrule)
\item[Semantics]
\item[Example] softly + x1 $\rightarrow$ x1 softly (He called her softly)
\item[Remarks]
\end{description}

\vspace{1 cm}
\begin{description}
\item[Name] RStartAdvpprop010
\item[Task] To build the PROP structure for a SUBADV which does not take a
subject argument, but which does take an argument in the ADVP (esp.\ sentence 
modifying adverbs).
\item[File] english:AdvpSubgrammars.mrule (mrules84.mrule)
\item[Semantics]
\item[Example] perhaps + x1 $\rightarrow$ [perhaps x1] (Perhaps he would call 
her)
\item[Remarks] This rule is needed as an alternative to RStartAdvpprop100, 
since only the ADVP argument can be specified in the pattern rules (see below). 
The subject argument remains unspecified in the current subgrammar.
\end{description}

\vspace{1 cm}
\begin{description}
\item[Name] RStartAdvpprop120
\item[Task] To provide a PROP structure for a SUBADV that takes one subject 
argument, and one argument in the ADVP (socalled {\em agvpadv\/}s: subject 
oriented adverbs). 
\item[File] english:AdvpSubgrammars.mrule (mrules84.mrule)
\item[Semantics]
\item[Example] enthousiastically + x1 + x2 $\rightarrow$ x1 enthousiastically 
x2 (He kissed her enthousiastically)
\item[Remarks]
\end{description}

\end{description}

\newpage
\subsection{TC\_AdvPatterns}
\begin{description}
\item[Kind] Obligatory Transformation Class
\item[Task] To check the category of the argument variable and the relation 
it bears in the ADVP against the advpatterns specified for the ADV. The {\bf 
advpatternefs} attributes of the ADVP and the ADVPPROP are set at the value 
actually chosen from the advpatterns of the ADV.

\vspace{1 cm}
\begin{description}
\item[Name] TADVPattern0
\item[Task] To let ADVs that have no ADVP arguments pass this transformation 
class
\item[File] english:AdvpSubgrammars.mrule (mrules84.mrule)
\item[Semantics]
\item[Example] x1 softly (He talked to him softly)
\item[Remarks]
\end{description}

\vspace{1 cm}
\begin{description}
\item[Name] TADVPattern1
\item[Task] To specify the relation name (which must always be {\em complrel\/})
and the category of the argument in the ADVP 
\item[File] english:AdvpSubgrammars.mrule (mrules84.mrule)
\item[Semantics]
\item[Examples] ordered by subrule: \\
x1 enthousiastically argrel/VAR $\rightarrow$ x1 enthousiastically 
complrel/CLAUSEVAR
 (He talked to her enthousiastically)\\
x1 ?? argrel/VAR $\rightarrow$ x1 ?? complrel/openinfSENTENCEVAR (??)\\
x1 ?? argrel/VAR $\rightarrow$ x1 ?? complrel/closedinfSENTENCEVAR (??)\\ 
x1 ?? argrel/VAR $\rightarrow$ x1 ?? complrel/VERBPPROPVAR  (??)
\item[Remarks] Of the four subrules, only the first is currently needed. The 
fourth, taking a VERBPPROP, may be an alternative for the first subrule, in 
case it is decided to move RC\_VPadv out of the XPPROPtoCLAUSE grammar and into 
the VERBPPROP grammar (the argument of the ADV inside the ADVP is the clause 
that is modified). Subrules 2 and 3 are a copy of Dutch subrules and seem 
useless in English.
\end{description}

\end{description}

\newpage
\subsection{RC\_AdvTempVar}
\begin{description}
\item[Kind] Iterative Rule Class
\item[Task] To introduce variables for time adverbials. The variable may be for 
a sentence, a prepp or an advp. The rule class is iterative, but only one rule 
is written here. Since temporal adverbs are needed only for XPPROPs that will 
form a sentence on their own (using the copula {\em be\/}), while ADVPPROPs 
do not seem to be used that way, even this one rule may be superfluous.

\vspace{1 cm}
\begin{description}
\item[Name] RAdvRefvarinsertion
\item[Task] To introduce a variable for a referential time adverbial that is 
not retrospective.
\item[File] english:AdvpSubgrammars.mrule (mrules84.mrule)
\item[Semantics]
\item[Example] x1 ?? + refVAR $\rightarrow$ x1 ?? refVAR 
\item[Remarks]
\end{description}

\newpage
\subsection{RC\_AdvSentAdvVar}
\begin{description}
\item[Kind] Iterative Rule Class
\item[Task] To introduce variables for adverbial subordinate sentences in 
different positions and sentence or causal adverbials in initial position. The 
conjunction may also be a preposition. No rules have been written yet for 
abstract conjunctions.

The rules are in an iterative class, but have been written in such a way that 
only one order of application is possible. This to prevent unnecessary 
ambiguities in analysis. Also, the IL strategy in mapping the different 
conjsent rules is to preserve the surface order used in the source language as 
much as possible, because it may be of importance for pronominal reference.

This rule class may be completely superfluous in the current subgrammar, but 
has been added anyway for certainty.

\vspace{1 cm}
\begin{description}
\item[Name] RAdvConjsentVar
\item[Task] To introduce a variable for an adverbial subordinate sentence (or 
sentential PREPP) in initial (leftdislocrel) position.
\item[File] english:AdvpSubgrammars.mrule (mrules84.mrule)
\item[Semantics]
\item[Example] x1 ?? x2 + advSENTENCEVAR $\rightarrow$ advSENTENCEVAR x1 ?? x2 
\item[Remarks] No comma is introduced at the end of the adverbial sentence, 
although it is probably obligatory. This will have to be added.
\end{description}

\vspace{1 cm}
\begin{description}
\item[Name] RAdvFinalConjsentVar
\item[Task] To introduce a variable for an adverbial subordinate sentence (or 
sentential PREPP) in final (postsentadvrel) position.
\item[File] english:AdvpSubgrammars.mrule (mrules84.mrule)
\item[Semantics]
\item[Example] 
x1 ?? x2 + advSENTENCEVAR $\rightarrow$ x1 ?? x2 advSENTENCEVAR 
\item[Remarks] 
\end{description}

\vspace{1 cm}
\begin{description}
\item[Name] RAdvSentadvVar
\item[Task] To introduce a variable for a causal or sentence adverbial or a 
causal prepp in 
initial position. No rules have been written to account for any other position 
of the adverbial, or to relate its position to that of an adverbial or temporal 
sentence also present in the clause.
\item[File] english:AdvpSubgrammars.mrule (mrules84.mrule)
\item[Semantics]
\item[Example] 
?? x1 + sentADVVAR $\rightarrow$ sentADVVAR ?? x1 
\item[Remarks] No comma is introduced at the end of the adverbial, 
although it is probably obligatory. This will have to be added.
\end{description}

\end{description}

\newpage
\subsection{RC\_AdvLocAdvVar}
\begin{description}
\item[Kind] Iterative Rule Class
\item[Task] To introduce variables for non-argument locatives (ADVP or PREPP) 
at a fixed position (locadvrel) following the ADVP. No rules have been 
written to account for any other position 
of the locative, or to relate its position to that of an adverbial or temporal 
sentence also present in the clause.

This rule class may be completely superfluous in the current subgrammar, but 
has been added anyway for certainty.

\vspace{1 cm}
\begin{description}
\item[Name] RAdvLocAdvVar
\item[Task] To introduce a variable for a non-argument locative ADVP 
\item[File] english:AdvpSubgrammars.mrule (mrules84.mrule)
\item[Semantics]
\item[Example] x1 ?? x2 + locADVVAR $\rightarrow$ x1 ?? x2 locADVVAR
\item[Remarks]
\end{description}

\vspace{1 cm}
\begin{description}
\item[Name] RAdvLocPreppVar
\item[Task] To introduce a variable for a non-argument locative PREPP
\item[File] english:AdvpSubgrammars.mrule (mrules84.mrule)
\item[Semantics]
\item[Example] x1 ?? x2 + locADVVAR $\rightarrow$ x1 ?? x2 locADVVAR
\item[Remarks]
\end{description}

\end{description}

\newpage
\subsection{RC\_AdvVoiceRules}
\begin{description}
\item[Kind] Obligatory Rule Class
\item[Task] To provide a rule at this place in the derivation that can function 
as an counterpart in the isomorphic scheme for the voice rules in the sentence 
grammar. The rule itself is vacuous: it does nothing whatsoever.

\vspace{1 cm}
\begin{description}
\item[Name] RAdvppropVoice
\item[Task] see above
\item[File] english:AdvpSubgrammars.mrule (mrules84.mrule)
\item[Semantics]
\item[Example] x1 well
\item[Remarks]
\end{description}

\end{description}


\end{document}

