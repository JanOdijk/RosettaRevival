
\documentstyle{Rosetta}
\begin{document}
   \RosTopic{linguistics}
   \RosTitle{The NP-subgrammar and the NPProp-subgrammar}
   \RosAuthor{Franciska de Jong}
   \RosDocNr{0157}
   \RosDate{\today}
   \RosStatus{concept}
   \RosSupersedes{117, except the appendices}
   \RosDistribution{linguists}
   \RosClearance{Project}
   \RosKeywords{NP, NPProp}
   \MakeRosTitle
%
%   \input{R0000_Chap1}
%   \input{R0000_App1}
%
%
\section{Introduction}
This paper contains the conclusions with respect to the NP-subgrammar
and the NPProp-subgrammar 
as they were drawn on the basis of a discussion of document 117
(Title: 
The subgrammars specific to 
nominal constituents). Also some extensions are included in this version of the
proposal. Comment and motivation is incorporated as far as doc. 117 does not
provide any insight, either because an extensive discussion is not given there,
or because the conclusions to be presented here differ from those in 117.
The DETP-subgrammar is described in a separate document.\\ \\ 
Among the expressions commonly referred to as NP a two-fold distinction will
be made. Though the distinction is expressed by means of the categories names
NP (Noun Phrase) and NPProp, the two types of expression do not 
relate to each other in the way assumed for the other XP and XPProp categories.
VERBPs for example are not generated independently of the 
generation of
a VERBPProp-level, whereas NPs can be generated indepently by 
a separate subgrammar, called NP-subgrammar.

NPs thus generated belong to the phrases 
that occur in argument-positions or as complements. The nominal expressions 
that occur in predicative position are generated as NPProps by the NPProp-
subgrammar, in a way that is 
comparable with 
other XPProp-categories. That is, just like {\em leuk} in 'Dit boek is leuk'.
{\em soldaat} in 'Jan is soldaat' is derived as a complex expression containing
a subject-var.\\ \\
Given the enormous overlap between the two sets of expressions that
can occur in either function, the ouput of the NP-subgrammar could be 
taken as  possible input for the NPProp-subgrammar. But the overlap
is not total, and the predicative function of NPProps favours an analysis
along the lines of the other XPProp-categories. Also it should not be excluded
exclude a priori that an isomorphic analysis can be given for 
expressions 
that intuitively are to be regarded as translations of each other, even 
though they belong to different syntactic categories.
\footnote{A special issue is 
the question whether Rosetta3 should provide the possibility
for an isomorphic treatment of ADJPProps and the predicative use of NPs 
e.g. for  {\em vervelend} and {\em een klier}.
As it would require more
insight into the contribution of determiners to the semantics of NPProps than
is available at the moment and as 
it is not clear yet whether we need such isomorphic treatments from a 
translational point of view, such an isomorphic treatment will not be pursued
on short term.}
 Therefore 
it is decided that the grammar is not to account for the overlap
between the two sets.  It is to be expected however, that 
a considerable amount of rules will
be comparable and perhaps even copy-able. 
\\ \\
Special conditions are required to block the generation of certain strings as a
NP or as a NPProp. Some examples. The determinerless singular count NP will
only be generated within the NPProp-subgrammar but also within the
NP-subgrammar. Consequently the subgrammars that take NPs as import should
block the use of e.g. {\em koning} as argument by means of conditions on the
relevant rules. On the other hand, in order to exclude the generation of
NPProps with {\em elke}, {\em sommige} and {\em de meeste} as a determiner),
the introduction of  the subject var should be constrained by means of
conditions on a special attribute (Cf. Domain T). Also I will assume that
nominal constituents with a leftperipheric adjunct, such as {\em ook de
voorzitter}, {\em alleen een meisje}, etc., will only occur in
argument-position, so here too appropiate conditions are required to block the
formation of NPProp out of a thus modified NP. (Cf. appendix A of doc. 117
 for an an
inventarisation of the kind of functions that each NP-type can fulfil.
Cf. also appendix B of doc. 117 for an overview of
the various NP-classes that can function as a subject in a generic sentence
(generic NP).)\\ \\ 
NB. The overlap between the NP-subgrammar and the NPProp-subgrammar would
increase if head-modification were adopted. The adoption of head-modification
in case of predicative NPs would require that all rules introducing modifiers
and specifiers to nominal heads apply before the apllication of the
NPProp-startrules. This is not identical to the claim that the output of the
NP-subgrammar can serve as input for the NPProp-subgrammar. Predicative NPs are
of a different semantic type than argument NPs. (The former can be argued to
denote sets of entities, while the latter are usually assumed to denote sets of
sets of entities.) Therefore if the M-rules of Rosetta3 are supposed to have a
compositional interpretation, it is preferrable that both NP types be generated
by a different set of M-rules. 

As it is still to be decided whether we will adopt head-modification or not for
the XPProp-categories the exact ordering of the rule classes within the
NPProp-subgrammar will not be given here. It will be assumed that there are two
alternatives. One is a subgrammar with RC:nppropstartrules as the initial rule
class. The rules for modification and determination are ordered after this
initial rule class are supposed to add syntactic structure, e.g. in the form of
a CN-node and/or a NP-node. The other is a subgrammar with TC: CN-formation
as the initial rule class and with RC:nppropstartrules ordered after
NP-modification. The relevant section will be confined to a listing of all the
relevant rule classes. 
\\ \\ 
In general the two subgrammars are designed according to the ideas set forth in
doc. 41, in particular with respect to the distinction of a CN-level within the
NP-subgrammar. (CN is an abbreviation for Common Noun; it can be used to refer
to the head noun of an NP, as well as to its non-maximal projections. Both
syntactic levels refer to <e,t>-type expressions.). Another proposal in doc. 41
concerns the assumption of the category DETP (Determiner Phrase). The syntactic
structures discussed in doc. 41 will be repeated here schematically: (1) and
(2). Note that NP-internal modifiers, e.g. appositions (bijstellingen), are
ignored in these trees.
\newpage
\begin{verbatim}
(1)            NP

     det               head

     DETP               CN

                (mod)  (mod)   head

                ...     ...    NOUN 


 (2)                  NP

           det               head

           DETP               CN

   (mod)  (mod)   head

    DETP     of   DETP

\end{verbatim}
Note that according to the XPProp-concept NP nodes might be introduced by the
NPProp-subgrammar. In that case it has a subject-var as a sister. while both
the var and the NP are dominated by a NPProp-node: (3). 

\begin{verbatim}
  (3)
               NPProp

          subj        pred

          VAR          NP

                    .........

\end{verbatim}
In the control expressions, the conventions and conditions discussed in doc.
101 (Jan L.) are acknowledged. The filter transformations incorporated in some
of the control expressions to follow are not discussed separately in the
comment sections.

\section{The organisation of the NP-subgrammar}
\subsection{The control expression}
\begin{verbatim}

DUTCH

import:  (BNOUN/PERSPRO/EN/ ...) , AdjPProp, DETP, CLAUSE, PPProp, 
         NP, SENT, QP, AdvPProp, VAR
export:  NP

NB. The import categories that can occur as lexical head in the 
derivation (HEADCATs, cf. doc. 101, Jan L.) are put between 
parentheses.

control expression: 
TC:CN-formation.{RC: mod1rules}.[RC: modpossrules].{RC: argmodrules}.
RC: NP-formation.[TC: emptyheaddeletion].FTi.{RC: mod2rules}.
[TC:detnumshift].FTk.[TC:QPCNshift].{TC:complextraprules}.FTj.
[RC: pre-np-modrules].[RC:Substitution]


ENGLISH

import:  (BNOUN/PERSPRO/EN/ ...), AdjPProp, DETP, PPProp, NP, SENT, 
         CLAUSE, QP, AdvPProp, VAR
export:  NP

control expression:
TC:CN-formation.{RC: mod1rules}.[RC: modpossrules].{RC: argmodrules}.
RC: NP-formation.[TC: emptyheaddeletion].FTi.{RC: mod2rules}.
[TC:QPCNshift].{TC:complextraprules}.FTj.[TC: detadjshift].FTm.
[RC: pre-np-modrules].[RC:Substitution]


SPANISH

import:  (BNOUN/PERSPRO/EN/...), AdjPProp, DETP, PPProp, NP, SENT, 
         CLAUSE, QP, AdvPProp, VAR
export:  NP

control expression:
TC:CN-formation.{RC: mod1rules}.[RC: modpossrules].
{RC: argmodrules}.RC: NP-formation.[TC: emptyheaddeletion].FTi.
{RC: mod2rules}.{TC:complextraprules}.FTj.[RC: pre-np-modrules].
[RC:Substitution] 
\end{verbatim}

\subsection{General Comment on the NP-subgrammar}

For NP-internal modification, there are three RCs: the RC:mod1rules, for cases
like {\em fat cheese} 
and {\em yellow car}; the RC:modpossrules, for modifiers that
express a possessive relation as in {\em John's car}; and the RC:argmodrules,
for modifiers that function as arguments to a noun that can be related a to
verbal expression, e.g. {\em de weigering van Jan}, or that impose restrictions
on their modifiers that resemble the restrictions that follow from the
(syntactic) verbpatterns, e.g. {\em de behoefte aan rust}, {\em de vraag wie er
komt}. Note that the modifiers might contain embedded NPVARs. In order to
not to complicate the clausal substitution rules these VARs should be replaced 
by full NPs within the NP-subgrammar. See below sub m.
The RCs mod1rules, modpossrules and argmodrules 
are the first three rule classes following the
obligatory TC:CN-formationrules.\\ \\
There is no straightforward way to label argument relations within the NP.
Therefore I will assume NP-internal arguments to express the relation argmod,
They are to be introduced by RC:argmodrules. This ruleclass deals not only with
logical arguments but also with WH-sentential postnominal modifiers and cases
like {\em students of physics} (vs. {\em students with red hair}), or in
general cases where the noun puts some selectional restrictions on functional
words, such as prepositions, complementizers, etc. Differences in behaviour of
modifiers to deverbal nouns (e.g. with respect to {\em one}-substitution, with
respect to the selection of complementizers, or with respect to the required
translation of prepositions) must be dealt with 
by suitable conditions on the rules.
As no other category can express the relation 'argmod', these condition
should probably make reference to this relation name.\\ \\ 
Actually we might as well decide to consider the three modifying RCs to
constitute one large rule class. The ordering of the rule classes is arbirtrary
and there is no fixed surface order in which the expressions introduced by each
of these rule classes occur. However, a decision with respect to this
alternative requires a more thorough insight in the actual rules and the
complexity of their conditions. Therefore I propose to postpone this issue to
the rule-writing-stage.\\ \\ 
In general attributes on NP-specifiers tend to characterize the NP as a whole
as well. For example, a WH-determiner makes an NP interrogative, and the
(in)/(a)-definiteness of a determiner or numeral is decisive for the
corresponding attribute for the entire NP as well. Therefore all rules of the
NP-subgrammar that introduce expressions specified for (one of) these
attributes should percolate upwardly the appropiate attribute value in order to
guarantee that the NP is specified for these attributes too.\\ \\ 
A last remark concerns the analysis of NPs that lack an overt nominal head. In
order to prevent the need to formulate a huge number of category-changing rules
it will be assumed that NP's without a nominal head, such as 
{\em deze drie}, {\em de
te laat gewaarschuwden}, {\em de gele}, {\em de derde}, 
{\em die gerepareerde}, {\em die van
Jan}, {\em meer}, {\em veel}, etc., are derived by means of a special basic syntactic
category called EN (short for 'empty noun'). 
This EN is supposed to dominate an abstract
basic expression. 

EN can function as the carrier of the kind of information that is supposed to
be represented in the record of the head in other cases as well. In fact we
need two different ENs: one occurring with [+count]-specifiers, another
occurring with [-count]-specifiers. Unless there is a [+count, -def]
determiner, such as {\em three} and {\em vele}, ECNs  will be deleted after
NP-formation. ENs that are not deleted play a role in the treatment of
{\em one}-substitution and N-bar-deletion, cf. the doctoral thesis of Angeliek van
Hout. Note that the 'occurrence' of an EN is restricted: some determiners or
prenominal strings disallow that a nominal head is absent. E.g. {\em de gele}
vs. {\em *de drie}.

\subsection{Comments on the various rule and transformation classes}
\begin{verbatim}
a. TC: CN-formation
import: (BNOUN, PERSPRO, EN), ...
export: CN
\end{verbatim}
This obligatory trnsformation 
class introduces the node CN (short for common noun).
\begin{verbatim}

b. RC: mod1rules
import: (CN), APProp, NP, CLAUSE, DETP, PPProp, SENT, QP, ADVPProp, 
        VAR, ... (specification partly based upon the construction 
        list in doc. 81)
export: CN
\end{verbatim}
This rule class accounts for the modification of the head of the NP. It inserts
non-argument and non-possessive modifiers. The introduction modifiers is
optional and it is a recursive rule class. Note however that not all modifying
categories listed in the import for this rule class can occur more than once. 

The mod1rules are also meant to deal with the introduction of numerals
(syntactic category DETP, see below) for cases like {\em de drie boeken}, 
{\em
alle honderd deelnemers}, {\em mijn vele beperkingen}, etc. (Check:
categorystatus adopted for {\em vele} in the domain definition.) The conditions
on the rules should prohibit the introduction of a numeral if a previous
application of the rule has already resulted in the insertion of a numeral.
\\ \\ 
Dependent on the syntax of the specific language, some of the modifier
categories introduced by this rule class may occur before or after the head.
Some divergencies: in Dutch clausal modifiers have both options, whereas they
are bound to postnominal positions in the other languages; in Dutch and English
(most) adjectival modifiers are prenominal, whereas in Spanish they are mostly
postnominal.
\\ \\ 
The category NP is mentioned in the import specification partly in view of
modified NPs with construction type indication 2.1.2.X (cf. doc. 81), including
the pseudo partitive NPs such as {\em een emmer bramen}, {\em een meter stof}
(where {\em bramen} and {\em stof} are considered modifiers), and the English
and Spanish counterparts that require a the occurrence of {\em of} and {\em de}
respectively. It is assumed that these particles are introduced
syncategorematically. (Note: NPs functioning as a modifiers to a
pseudo-partitive NP are an example of a modifying category that occurs only
once.)\\ \\ 
Compounds in English and Spanish (e.g. keukentafel -kitchen table) are supposed
to be treated as fixed idioms of the category BNOUN.\\ \\ 
Given the huge number of syntactic categories that can be imported as modifier
by this rule class, it might turn out to be more transparant to split it up in
several rule classes each introducing a specific subset of the import set
mentioned above. 
\begin{verbatim}

c. RC: modpossrules
import: (CN), NP (with a relation name a proper name, or a 
        PERSPRO as a head), VAR
export: CN 
\end{verbatim}
This rule class, too, accounts for the introduction of modifying expressions,
but here a special kind is involved, namely modifiers expressing a possessive
relation. In order to allow an isomorphic treatment, both possessive genitives
(which occur as a DETP) as well as postnominal {\em van/of/de}-PPs are
introduced by this rule-class. For a thorough discussion of how the isomorphy
is realized, cf. the document on the DETP-subgrammar.
\begin{verbatim}

d. RC:argmodrules 
import: (CN), PP, CLAUSE
export: CN
\end{verbatim}
The RC:argmodrules introduce 
modifiers that function as arguments to a noun that can be related a to
verbal expression, e.g. {\em de weigering van Jan}, or that impose restrictions
on their modifiers that resemble the restrictions that follow from the
(syntactic) verbpatterns, e.g. {\em de behoefte aan rust}, {\em de vraag wie er
komt}.
\\ \\
There is no straightforward way to label argument relations within the NP.
Therefore I will assume NP-internal arguments to express the relation argmod,
The ruleclass deals not only with
logical arguments but also with WH-sentential postnominal modifiers and cases
like {\em students of physics} (vs. {\em students with red hair}), or in
general cases where the noun puts some selectional restrictions on functional
words, such as prepositions, complementizers, etc. Differences in behaviour of
modifiers to deverbal nouns (e.g. with respect to {\em one}-substitution, with
respect to the selection of complementizers, or with respect to the required
translation of prepositions) must be dealt with 
by suitable conditions on the rules.
As no other category can express the relation {\em argmod}, these condition
should probably make reference to this relation name.
\begin{verbatim}

e. RC: NP-formation
import: (CN), DETP
export: NP
\end{verbatim}
This rule class effectuates the transition from CNs., i.e. \(<e,t>\)-type
expressions into NPs, i.e. \(<<e,t>,t>\)-type 
expressions. In some constructions
this implies the insertion of a DETP. In case the CN contains DETP (i.e. a
numeral), the introduction of a (definite) determiner is even obligatory. (Cf.
doc. 41, p. 12, the condition on rule RNP2) But there are also several cases
where no determiner occurs in surface syntax: proper names, personal and
demonstrative pronouns, bare plurals and mass nouns, etc. Moreover, in view of
the analysis of possessive determiners (cf. doc. 158), it is assumed here that
the definite article occurring in non-partitive NPs is realized
syncategorematically. Possibly a syncategorematic introduction will turn out to
offer a satisfactory approach to some other translational problems as well. For
example, it is to be expected that the criteria which make an NP suitable for a
generic interpretation differ in each of the Rosetta-languages. Cf. 
Appendix B of doc. 117.
\begin{verbatim}

f. TC: emptyheaddeletion
import: (NP)
export: NP
\end{verbatim}
The sole purpose of this transformation class is to delete all ENs (empty
nominal heads). In Dutch the deletion does not apply to ENs
preceded by a [+count, -def]-determiner. This exception is needed
in order to account for the occurrence of quantitative {\em er}. The
[plus-minus count]-distinction is 
of importance here because of the fact that for
example {\em meer} and {\em veel} are ambiguous between a count-reading and a 
mass-reading. In both readings they allow an EN, but only in the former
reading do they allow an EN in combination 
with {\em er}. This ambiguity also explains why {\em meer} and {\em veel} do not always
allow a comparative as {\em dan drie}: (4) vs. (5). 
Note also that {\em meer} preceding an EN
can be
complemented by {\em dan} plus a full NP only if {\em meer} is used in the 
[-count]-reading, that is when it blocks the occurrence of {\em er[+q]}: (6).
\begin{verbatim}

 (4) a.  Hij geeft er meer/veel gekocht
     b.  Hij heeft meer/veel gekocht 

 (5) a.  Hij heeft er meer gekocht dan drie
     c. *Hij heeft meer gekocht dan drie

 (6) a. *Hij heeft er[+q] meer gekocht dan drie boeken 
     d.  Hij heeft meer gekocht dan drie boeken 

\end{verbatim}
The ordering of this transformation class 
as it is given here does not seem crucial. It
might as well be ordered later or earlier.
\begin{verbatim}

g. RC: mod2rules 
import: (NP), NP, CLAUSE, SENT
export: NP 
\end{verbatim}
This ruleclass deals with the 
introduction of appositions (=bijstelling) and non-restrictive 
modification.
\begin{verbatim}

h. TC: detnumshift (Dutch only)
import: (NP)
export: NP
\end{verbatim}
This transformation class (specific to Dutch) deals with the shift of
specifiers in the environment {\em alle de NUM N}, with as a result an NP of
the shape {\em alle NUM de N}. The import NP has to contain two DETPs, one
CN-modifier (a numeral), and a DETP with detrel that is modified by {\em al} or
{\em alle}. Example: {\em alle de drie N} $\rightarrow$ {\em alle drie de N}.
\begin{verbatim}

i. TC: QPCNshift (Dutch and Eglish only)
import: (NP)
export: NP
\end{verbatim}
This optional transformation class shifts the order beteen a QP-determiner and
the CN-string in order to derive {\em books enough}, {\em spannende boeken zat}.
The occurrence of NPs that are affected by this RC 
is restricted. They do not
occur in leftmost position:
\begin{verbatim}
   *Boeken zat zijn er te koop  vs. Er zijn boeken zat te koop
   *Books enoughare for sale    vs. John has books enough
\end{verbatim}
Therefore the Dutch clausal substitution rules 
should block their substitution for shiftrel VARs. For English the condition 
should be slightly different because of the fact that the subject in leftmost 
position does not have the relation shiftrel.
\begin{verbatim}

j. TC: complextraposrules
import: (NP)
export: NP
\end{verbatim}
This transformation class is meant to deal with the obligatory extraposition of
complements that may occur within DETPs dominating a QP (cf. the makedetprules
of the DETP-subgrammar below). Note that the sentence initial direct object NP
in {\em Meer boeken dan Jan kan ik niet dragen} contains a discontinuous
complementred DETP, viz. {\em meer ... dan Jan}. Only after the introduction of
this DETP by means of the NP-formation rules the extraposition of the modifier
{\em dan Jan} to {\em meer} can the movement of {\em dan Jan} be dealt with.
Note that the postnominal position need not be the ultimate slot for this kind
of modifiers. If an NP with such a discontinuous DETP does not occur in
sentence initial position, the complements such as {\em dan Jan} might be
extraposed again into a sentence final position: {\em Ik kan meer boeken dragen
dan Jan}. This latter movement is supposed to be dealt with by a transformation
class of the CLAUSE-to-SENTENCE subgrammar. 
\begin{verbatim}

k. TC: detadjshift (English only)
import: (NP)
export: NP
\end{verbatim}
This obligatory transformation class (specific to English) deals with the shift
of determiner {\em a(n)} and certain prenominal adjectives, such as {\em half}
and those modified by degree-modifier {\em too}. 
\begin{verbatim}
Examples:  a too big apple  -->  too big an apple
           a half apple     -->  half an apple
\end{verbatim}
A reason for incorporating this TC into the NP-subgrammar and not in the
DETP-subgrammar is that in order not to complicate TC:complextraprules it
should apply after the extraposition of complements to degree-modifier {\em
too}. 
\begin{verbatim}

l. RC: pre-np-modrules
import: (NP), ADVP
export: NP
\end{verbatim}
This ruleclass introduces leftperipheric adverbials that modify the NP 
as a whole. For example: {\em ook de voorzitter}, {\em  alleen een meisje}.
\begin{verbatim}

m. RC: substitution
import: (NP)
export: NP
\end{verbatim}
In order to prohibit that the clausal substitution rules have to look 
deep inside NPs, the NP-VARs embedded in NPs (e.g. in PP-modifiers) are to be 
substituted at a rather early stage. Therefore the NP-subgrammar contains 
RC:substitution. 
\section{The organisation of the NPProp-subgrammar}
As indicated in the introduction there is a huge amount of overlap between the
NP-subgrammar and the NPProp-subgrammar. The exact extension of the overlap is
dependent on the question whether the NPProp-subgrammar incorporates
head-modification or not. This issue also affects the question what is to be
listed as import for the most characteristic rule class of the
NPProp-subgrammar, RC: nppropstartrules. Without head-modification it will be a
BNOUN plus a number of VARs (mostly one, but probably zero in case the BNOUN is
a weather noun such as {\em winter}).  But in case modification is to apply
before the startrules, complete NPs instead of BNOUNs might be (part of) the
import. (Remember these import NPs would not be the ones resulting from the
NP-subgrammar, but the ones that generated by the RCs to be ordered before the
startrules. Cf. the introduction.) Also the import categories for the other
RCs and TCs vary accordingly. Pending a decision on the issue of 
head-modification
this
document will specifiy the rule classes of the NPProp-subgramnmar only
partially. 

In addition to the rule classes effectuating determination/modification, the
transformation classes and the RC:substitution, the
NPProp-subgrammar control-expression should contain the following two 
rule/transformation 
classes:
\begin{verbatim}
RC: nppropstartrules
TC: nppatternrules
\end{verbatim}
The RC:nppropstartrules introduces a NPProp-node by combining a BNOUN/NP and a
VAR. The transformation class TC: nppatternrules specifies the syntactic
relation BNOUN/NP and VAR to the NPProp-head. \\ \\
The conditions on the modification/determination rules should guarantee that
NPProps will never dominate substrings with {\em sommige} or {\em most} as
DETP. Probably it should also be prohibited that measure NPs, such as {\em drie
kilo kaas} be analysed as NPProp. 
\end{document}

