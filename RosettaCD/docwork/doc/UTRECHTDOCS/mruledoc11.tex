\documentstyle{Rosetta}
\begin{document}
   \RosTopic{Rosetta3.doc.Mrules.English}
   \RosTitle{Rosetta3 English M-rules: ADVPPROPtoFORMULA}
   \RosAuthor{Margreet Sanders}
   \RosDocNr{387}
   \RosDate{\today}
   \RosStatus{concept}
   \RosSupersedes{-}
   \RosDistribution{Project}
   \RosClearance{Project}
   \RosKeywords{English, documentation, Mrules, ADVPPROPtoFORMULA}
   \MakeRosTitle
%
%

\section{Introduction}
As all main category grammars, the English ADVPPROP grammar is divided in 
three parts. First, {\bf ADVPPROPformation} forms the PROP-structure. Then, {
\bf ADVP\-PROPtoFORMULA} turns this PROP into an intermediate structure, called 
ADVPFORMULA
(comparable to the CLAUSE in the sentence grammar). Finally, {\bf 
ADVPFORMULAtoPROP} makes an open or closed PROP-structure, comparable to the 
SENTENCE level in the sentence grammar. Prior to ADVPPROPformation, there is 
the {\bf AdvDerivation} grammar, which was discussed in doc.\ 316, {\em 
Rosetta3 English M-rules: Derivation Subgrammars\/}. The final open or closed 
ADVPPROP is usually input to the Proposition Substitution Rules in the 
XPPROPtoCLAUSE subgrammar. It may also be used in RC\_AdvVar, 
for subject-oriented 
verb modifying adverbs. Theoretically, a `bare' ADVPPROP may also form an 
expression on 
its own with help of the copula {\em be\/}. In that case, it would leave the 
ADVPPROP grammars after ADVPPROPformation (and hence not pass the subgrammar 
described in this document), and 
be input to the ClauseFormation Rules of the XPPROPtoCLAUSE subgrammar. 
However, no example of such a predicatively used adverb could be found, 
and the relevant clause formation rule has not been written.
In English, there is no rule to make a complete Utterance of a closed ADVPPROP.

The current document describes the contents of the second ADVPPROP subgrammar, 
ADVPPROPtoFORMULA (the first subgrammar was discussed in doc.\ 386, {\em 
Rosetta3 English M-rules: ADVPPROPformation\/}, and the last one is discussed 
in doc.\ 388). The subgrammar consists of 
a number of rule classes and transformation classes. A rule class in its turn
consists of a number of rules and a transformation class of a number of 
transformations. The relative ordering of the rules and transformations in the
(sub)grammar is indicated by a {\em control expression}. A summary of this
control expression (i.e.\ a listing of the ordering of the rule classes, 
without explicit mentioning of the rules themselves) is also included here, 
and the initial (= head), import and export categories are given. Conditions on 
crucial orderings of rule classes, if they exist in the current subgrammar, are 
mentioned explicitly.

In the section on the rules and transformations, only the rule names are given, 
but not the exact rule formulation. What is attempted 
is to provide a detailed overview of the workings of the subgrammar, and 
how the different rule classes achieve this,
together with some comments on the problems still to be solved, the reasons 
behind certain choices, and perhaps possible alternatives. For every rule, an 
example is given, if one could be found. 
If it is uncertain whether the example is correct (either 
because it may not be an example of the phenomenon in question, or because it 
may not be correct English), it is preceded by a question mark. Note that all 
explanation of rules and transformations is given from a generative viewpoint
only, unless explicitly stated otherwise. Often, the information given in this 
document is based strongly on the comment already present in the documentation 
of the rules themselves. Discrepancies between what is stated here and what is 
said in the rule itself are usually caused by the fact that the rule file has 
not  been updated, although insights have changed. Note that the current 
document is an adapted copy of doc.\ 381 on the PREPPPROPtoFORMULA subgrammar
(the two subgrammars are very similar). The semantics of the rules 
has been left unspecified in the current documentation, since it is not at all 
clear.

Finally note that the rules described in this document have NOT been tested 
properly. English analysis is not possible yet (there is no Surface Parser), and 
English generation has only been tested in as far as the construction was the 
translation of a Dutch sentence to be tested.

\newpage
\section{ADVPPROPtoFORMULA}
The exact contents of the ADVPPROPtoFORMULA subgrammar are mainly determined 
by the requirements of at least partial isomorphy with the XPPROPtoCLAUSE and  
PREPPPROPtoFORMULA subgrammars (see doc.\ 386 on the latter subgrammar for a 
more extensive discussion of the isomorphy relations). 

In doc.\ 150, {\em Subgrammars of English\/}, it was assumed that the rules of 
the current subgrammar would mirror all important rule classes of the 
XPPROPtoCLAUSE subgrammar, including Proposition Substitution, Control, and 
EMPTY substitution. Since there are no sentential variables in 
the ADVPpropFormation grammar (see the comment in the pattern rules there), and 
no EMPTY arguments, these classes do not seem necessary anymore.

As in the sentence grammar, the FormulaFormation rules have been separated into 
a number of rule classes, each performing its own task (viz.\ aspect and 
superdeixis assignment). 

Most rule classes have only one rule, providing some sort of 
`default' value. There are no transformations at all. It is not impossible that 
extra rules will appear to be necessary when the treatment of adverbs in 
Rosetta3 has been worked out in more detail.


\section{Subgrammar Specification}
The subgrammar definition can be found in the file which also contains all the 
rules of this subgrammar, {\bf AdvpSubgrammars.mrule}, which is 
{\em mrules84.mrule\/}.

\begin{verbatim}
%SUBGRAMMAR AdvppropTOformula


   ( RC_AdvFormulaFormation )
.  ( RC_AdvAspect )
.  ( RC_AdvSuperdeixis )

\end{verbatim}

\begin{description}
  \item[Head]  ADVPPROP  \ \ \ \ FROM (ADVPPROPformation)
  \item[Export] ADVPFORMULA
  \item[Import] --
\end{description}

\newpage
\section{Rules and Transformations}

\subsection{RC\_AdvFormulaFormation}
\begin{description}
\item[Kind] Obligatory Rule Class
\item[Task] To turn an ADVPPROP into an ADVPFORMULA (which has exactly the same 
record as an ADVPPROP). This rule is needed only to form a counterpart in the 
isomorphic scheme for the ClauseFormation rules of the XPPROPtoCLAUSE 
subgrammar.

\vspace{1 cm}
\begin{description}
\item[Name] RAdvToFormula
\item[Task] See above
\item[File] english:AdvpSubgrammars.mrule (mrules84.mrule)
\item[Semantics]
\item[Example] $_{ADVPPROP}$[x1 enthousiastically x2] $\rightarrow$ 
$_{ADVPFORMULA}$[x1 enthousiastically x2] (She greeted him enthousiastically)
\item[Remarks] 
\end{description}

\end{description}

\newpage
\subsection{RC\_AdvAspect}
\begin{description}
\item[Kind] Obligatory Rule Class
\item[Task] To spell out the aspect of the ADVPFORMULA, i.e.\ the aspect 
relation between the interval E and a reference interval R. It is assumed that 
this relation is always {\em imperfective\/} here; there may not be any 
temporal adverbial present. It is doubtful whether the rule is needed for 
anything but isomorphy (see the Introduction to this document).

\vspace{1 cm}
\begin{description}
\item[Name] RAdvppAspectImperf
\item[Task] see above
\item[File] english:AdvpSubgrammars.mrule (mrules84.mrule)
\item[Semantics]
\item[Example] [x1 enthousiastically x2]$_{omegaaspect}$ $\rightarrow$ 
[x1 enthousiastically x2]$_{imperfective}$ 
\item[Remarks]
\end{description}

\end{description}

\newpage
\subsection{RC\_AdvSuperdeixis}
\begin{description}
\item[Kind] Obligatory Rule Class
\item[Task] To provide the ADVPFORMULA with a value for the attribute {\bf 
superdeixis}. This rule is needed because of isomorphy reasons. It uses a 
parameter, {\em superpar\/}, to determine whether a present or past superdeixis 
value should be assigned.

If rules are added to introduce propositions or sentences in the Advpprop 
subgrammar, it may be necessary to add superdeixis adaptation transformations 
following the current rule class. These should reset the value of the 
superdeixis attribute to something the Surface Parser can cope with.

\vspace{1 cm}
\begin{description}
\item[Name] RAdvppSuperdeixis
\item[Task] see above
\item[File] english:AdvpSubgrammars.mrule (mrules84.mrule)
\item[Semantics]
\item[Example] [x1 enthousiastically x2]$_{omegasuperdeixis}$ $\rightarrow$ 
[x1 enthousiastically x2]$_{pastsuperdeixis}$ (He greeted her enthousiastically)
\item[Remarks]
\end{description}

\end{description}


\end{document}

