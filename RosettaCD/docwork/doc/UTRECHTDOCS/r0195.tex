\documentstyle{Rosetta}
\begin{document}
   \RosTopic{General}
   \RosTitle{Notulen Rosetta vergadering 13-4-1987}
   \RosAuthor{Harm Smit}
   \RosDocNr{0195}
   \RosDate{\today}
   \RosStatus{approved}
   \RosSupersedes{-}
   \RosDistribution{Project}
   \RosClearance{Project}
   \RosKeywords{minutes}
   \MakeRosTitle
\begin{itemize}
  \item {\bf aanwezig}: Lilian Kopinga, Ans Post, Ren\'{e} Leermakers, 
             Jeroen Medema, Joep Rous, Jan Landsbergen, Andr\'{e} Schenk, 
             Natalia Grygierczyk, Carel Fellinger, Jan Odijk, 
             Elly van Munster, Harm Smit, Jan Stevens, Margreet Sanders,
             Chris Hazenberg.
  \item {\bf afwezig}: Franciska de Jong, Lisette Appelo.
  \item {\bf Agenda}:
    \begin{enumerate}
       \item Opening en notulen
       \item Diverse mededelingen
       \item Verslag ACL conferentie
       \item Verslag GLOW conferentie
       \item Verslag bezoek BSO
       \item Verslag bijeenkomst WG Computerlingu\"{\i}stiek
       \item Besproken en/of nieuw verschenen documenten
       \item Rondvraag en sluiting
    \end{enumerate}
  \item {SPANAM}: de bespreking van SPANAM is uitgesteld tot 11 mei.
\end{itemize}

\section {Opening en notulen}

De notulen van de vorige vergadering worden met enkele kleine wijzigingen
aangenomen.

\section {Diverse mededelingen}

\begin{enumerate}
  \item {\bf Jan L.} is op 7 april met Loek bij SPIN op bezoek geweest. SPIN 
signaleerde de volgende `problemen' m.b.t. subsidie voor een taalprojekt bij 
ons:
   \begin{itemize}
      \item Het rijk subsidieert al andere {\em vertaal}projekten,
      \item Er zijn al twee andere SPIN-projekten over taal, nl. een al lopend 
projekt over {\em spraaksynthese}, en een projekt in aanvraag van Bunt 
(Tilburg) in samenwerking met Oc\'{e} over {\em mens-machine interface},
      \item Philips krijgt al veel SPIN-gelden (o.a. voor het PRISMA-projekt).
   \end{itemize}
Het is derhalve van belang dat we ons bij de aanvraag concentreren op o.a.: 
binnen het projekt 
niet alleen bezig zijn met {\em vertalen}, het zoeken naar een `gat in de 
markt' naast andere SPIN-projekten of het projekt presenteren als aanvulling 
daarop, binnen het projekt samen te werken met andere (liefst kleine) bedrijven
(bijvoorbeeld een uitgever).
  \item {\bf Jan L.} meldt dat er op 6 mei 3 medewerkers van IBM Nederland op 
bezoek zullen komen. Deze mensen werken bij het National Language Services 
Center aan aktiviteiten m.b.t. het Nederlands en komen op bezoek om informatie 
in te winnen over ons projekt.
  \item {\bf Jan L.} meldt dat er op 5 mei in Luik een dag wordt georganiseerd 
over automatisch vertalen. Hieraan werken o.a. Nagao en Melby mee. Het 
programma ligt ter inzage bij Jan L.
  \item {\bf Jan L.} meldt dat SRI International (SRI=Stanford Research 
Institute) in Cambridge (GB) heeft gevraagd of wij mee wilden doen aan een 
ESPRIT-projekt over {\em mens-machine interface}. Ook Siemens is bij dit 
projekt betrokken. Jan heeft hierover met Loek gepraat; waarschijnlijk doen we 
niet mee.
\end{enumerate}

\section {Verslag ACL conferentie}

Jan L., Carel en Lisette hebben de ACL in Kopenhagen bezocht; het was een goed 
georganiseerde conferentie met lezingen van een redelijk niveau. Er waren circa 
170 deelnemers, en 50 papers (geselecteerd uit 91 inzendingen) in 2 secties. 
Jan en Carel doen verslag van de door hen gevolgde programma onderdelen. 
Lisette zal mogelijk in een volgende vergadering over de door haar bijgewoonde 
lezingen vertellen. De proceedings komen nog, het programma ligt bij Jan L. 
ter inzage.

\section {Verslag GLOW conferentie}

Elly heeft de GLOW in Veneti\"{e} bezocht en vertelt hierover. 
Er waren vijf belangrijke thema's:
\begin{enumerate}
  \item Binary Branching
  \item Positie van het subject in de zin
  \item V-Raising
  \item Reanalysis
  \item Logical Form
\end{enumerate}
Het ligt in de bedoeling 
dat er van deze GLOW proceedings verschijnen, maar dat kan nog wel even op zich 
laten wachten.

\section {Verslag bezoek BSO}

Jan L., Lisette, Ren\'{e}, Elly en Harm hebben op 9 april 
een bezoek gebracht aan het DLT-projekt van 
BSO; Ren\'{e} vertelt hierover. Er waren drie zittingen; de eerste was een 
lezing van Witkam over de huidige opzet (gebaseerd op `dependency grammar') in 
vergelijking met de situatie bij het bezoek van BSO aan ons. Verder was er een 
syntactische demonstratie (van de ATN-parser die zinnen in `Simplified 
English' aankon) en een semantische demonstratie (waarbij het bepalen van 
de waarschijnlijkheid van combinaties van woorden gedemonstreerd werd). De 
hardware bestaat bij DLT momenteel uit 6 Suns, er wordt geprogrammeerd in 
PROLOG en C.

\section {Verslag bijeenkomst WG Computerlingu\"{\i}stiek}

Harm heeft de vergadering van de WG Computerlingu\"{\i}stiek op 10 april 
bezocht; er waren lezingen over ASCOT, het desambigueer projekt in Delft en het 
TRANSIT-projekt (Nederlands-Turks) aan de KUN. Met name de lezing over ASCOT 
was interessant; dit projekt is bijna af en is vergelijkbaar met onze 
morfologie (zonder derivatie): er is een versie van het Longman bestand en er 
zijn morfologische procedures om woordvormen te kunnen herleiden tot de vorm in 
het woordenboek. Harm heeft met Hetty van Zutphen van ASCOT afgesproken er 
binnenkort op bezoek te gaan. Op de aansluitende vergadering is voornamelijk 
gesproken over plannen van ST in de toekomst `thematische 
onderzoeksprogramma's' te starten (gedacht wordt aan gecombineerd onderzoek op 
verschillende plaatsen door meerdere -niet per s\'{e} beginnende- onderzoekers, 
waarbij ook andere geldbronnen (PSYCHON, Max Planck-instituut) betrokken kunnen 
zijn), en over het voorstel van het bestuur het aantal WG's te verminderen. 
Over dit laatste punt was de WG niet erg te spreken.

\section {Besproken en/of nieuw verschenen documenten}

\begin{itemize}
  \item {\bf besproken}: {\em niets}
  \item {\bf verschenen}: 
    \begin{enumerate}
       \item Joep Rous: A solution for the `onoplosbaar' problem (no. 0189). 
             Dit document wordt besproken op de lingu\"{\i}stenvergadering 
             van a.s. dinsdag.
       \item Jeroen Medema, Carel Fellinger: Key Definitions Under VMS/RBS (no. 
             0190). Dit document bevat twee fouten, nl.:
           \begin{enumerate}
              \item `sholog' (key F19) wordt niet direkt uitgevoerd; dit is 
                     gedaan met het oog op de mogelijkheid een nummer
                     als parameter (-1, -2, ...) mee te kunnen geven om
                     vorige versies te bekijken.
                     Het is ook mogelijk de sholog van iemand anders
                     te bekijken door de naam van die persoon mee te geven als 
                     parameter. De syntax van het sholog commando is als volgt:
                     SHOLOG [$<naam>$] [$<nummer>$].
              \item het selecteren van de release zit niet in de group-login; 
                    dit moet  \`{o}f in de persoonlijke login-file staan,       
                    \`{o}f `met de hand' gegeven worden.
              \end{enumerate}
            Nog een opmerking: bij de commando's GRAB, FREE, SAVE, WORK, MODIFY
            en INSPECT kan een `?' gegeven worden {\em na} de opdracht. Dit kan 
            i.p.v. de filenaam, dus zowel `$<component>$:?' als `?'. 
            Op het scherm 
            verschijnt dan informatie m.b.t. de files die de gegeven actie 
            kunnen ondergaan. Bijvoorbeeld: `GRAB $<$component$>$:?'
            vertelt welke files door wie gegrabbed zijn en welke files  
            gegrabbed kunnen worden in de gegeven component.
    \end{enumerate}
\end{itemize}

\section {Rondvraag en sluiting}
{\em Niets van belang}
\end{document}
