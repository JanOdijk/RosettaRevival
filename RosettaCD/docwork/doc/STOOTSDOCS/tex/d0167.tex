   \documentstyle{Rosetta}
   \begin{document}
      \RosTopic{formalism}
      \RosTitle{Compiler Generator}
      \RosAuthor{Ren\'{e} Leermakers}
      \RosDocNr{0167}
      \RosDate{17-12-86}
      \RosStatus{concept}
      \RosSupersedes{-}
      \RosDistribution{Software Group}
      \RosClearance{Project}
      \RosKeywords{compiler, M-rules}
      \MakeRosTitle
\section{Introduction}
This document describes a compiler generator, i.e. a program that converts
an attribute grammar that defines some language, into a compiler for this
language. The ideas for this generator developed out of the fear and awe that
crept into me, as I contemplated the task of building a compiler for Rosetta
Mrules.
The main advantage of a compiler generator is a superior maintainability
of the compilers generated by it, in addition to a complete documentation.
The main advantage of the present generator, compared to others, is probably
the amount of freedom in writing the basis grammar. The idea is that in this
way syntax can more easily be oriented towards semantics.

In section 2, I will describe the attribute grammar. It is a grammar
in the spirit of the Rosetta Surface Grammar, enriched with
inherited attributes and a new kind of synthesized attributes. Parameters are
used as in the surface grammar but there are now two kinds of them.
Section 3 is devoted to the subject of attribute evaluation, and section 4
contains some conclusions and comments. In a subsequent document
I will present the current syntax for writing the attribute grammar and the
actual implementation of the compiler generator.
\section{Attribute Grammar}
\subsection{Some Remarks}
The present attribute grammar could be called a procedural attribute grammar,
as the concepts of a procedures and processes play a dominant role, rather than
that of a function.
Note that a procedure can be seen as a state space transformer. As
a state is a function from the set of identifiers to their values, a
transformation in state space is a function from a function space to itself.
Thus, rather than a filthy concept, a procedure is a specification of
higher-order functions.

Given two arbitrary sets of identifiers $A$ and $B$, I will write
\begin{quote}
 $pr : A \rightarrow B$
\end{quote}
to mean that pr is a state transformer that leaves invariant all values of the
identifiers that are not in B, and that changes the values of the identifiers in
B in terms of the values of the identifiers in A before the transformation.

Attributes are thought of as identifiers inside the processes that evaluate
them. The specification of such a process for each node in a syntax tree,
is part of the definition of attribute evaluation, and thus of the compiler
generator, rather than being an implementation of it.

\subsection{Basis Grammar}
The basis of the attribute grammar is a simple rewrite system
$ G = <N,T,P> $, where $N$ is the set of non-terminals of which $UTT$
(utterance) is the start-symbol, $T$ the set of terminals, and
$P$ the set of production rules $P_{n}=$ $ <n,R_{n},S_{n}>$. A production rule
is commonly written as $n \rightarrow R_{n}$ with neglect of $S_{n}$, which we
call the set of categorial symbols.
In $P_{n}$, $n \in N$, and for each $n \in N$ there is exactly
one $P_{n} \in P$, and hence one $R_{n}$. $R_{n}$ is a regular expression of
categorial symbols. Each symbol $s \in S_{n}$ occurs once and only once in it.
To each symbol $s \in S_{n}$ corresponds a category, i.e. a terminal or a
non-terminal, which we denote by $\hat{s}$. The regular expression $R_{n}$ is
said to define $n$. $UTT$ does not correspond to any symbol, in order to avoid
recursive definitions of $UTT$:
\begin{quote}
$\exists_{n} s \in S_{n} \rightarrow \hat{s} \in N \cup T /\{UTT\}$
\end{quote}

Apart from the one-to-one correspondence between production
rules and non-terminals, the rewrite system G is not restricted. Hence, any
context-free, possibly ambiguous, language can be defined by a suitable G.

A sentence is part of the language if at least one syntax tree exists, with
the terminals of the sentence as its leaves.
A syntax tree t is defined as
\begin{itemize}
\item $t = <n,s,t_{1},..,t_{n}>$, where
   \begin{enumerate}
   \item the fields of t are often referred to as $t.n$, $t.s$, $t.t_{i}$
   \item $s$ is a symbol $s \in S_{n}$ and ($n = \hat{s}$ or $n = UTT$)
   \item $k>0$ iff $n \in N$ (i.e. $t$ is a leaf if $t.n \in T$)
   \item $t_{i}$ are syntax trees.
   \item $t_{1}.s ... t_{n}.s$ in L($R_{t.n}$),
   the language of symbol sequences defined by the regular expression $R_{t.n}$
   \end{enumerate}
\end{itemize}

\subsection{Surface Attributes and Parameters}
To each production rule $P_{n}$ corresponds a set of surface parameters
$SP_{n}$ and to each $n \in N \cup T$ a set of surface attributes $SA_{n}$.
To each symbol $s \in S_{n}$ for some n, also corresponds a set of
surface attributes $SA_{s}$, which is a copy of $SA_{\hat{s}}$.
Surface attributes are synthesized only.

To each symbol $s \in S_{n}$ corresponds a procedure
\begin{quote}
$pre_{n,s} : SP_{n} \cup SA_{s} \rightarrow \{status\}$,
\end{quote}
where $status$ is a predefined identifier with boolean values.

To each symbol $s \in S_{n}$ also corresponds a procedure $act_{n,s}$ that
transforms the values of the elements of $SP_{n}$, which transformation
depends solely on the values of the parameters in $SP_{n}$ before the
transformation and on the values of the elements of $SA_{s}$:
\begin{quote}
$act_{n,s} : SP_{n} \cup SA_{s} \rightarrow SP_{n}$.
\end{quote}

To each production rule $P_{n}$ and hence to each $n \in N$, as there is a
one-to-one correspondence between the two, correspond a status changing
procedure $pre_{n}$, a parameter initialization procedure $Sinit_{n}$,
and a procedure $act_{n}$ that gives all values of $AS_{n}$ a value:
\begin{quote}
$pre_{n} : SP_{n} \rightarrow \{status\}$\\
$Sinit_{n} : \emptyset \rightarrow SP_{n}$\\
$act_{n} : SP_{n} \rightarrow SA_{n}$.
\end{quote}

\subsection{Procedural Attributes and Parameters}
To each production rule $P_{n}$ corresponds a set of procedural parameters
$PP_{n}$ and to each $n \in N \cup T$ two sets of procedural attributes
$PI_{n}$ and $PS_{n}$, which are of the inherited and synthesized types,
respectively. The sets $PI_{n}$ and $PS_{n}$ are disjunct, i.e. a
procedural attribute is either inherited or synthesized. The set $PI_{UTT}$ is
always empty.
Again, to each categorial symbol $s$
also correspond sets of attributes $PI_{s}$ and $PS_{s}$, which are copies of
$PI_{\hat{s}}$ and $PS_{\hat{s}}$, respectively.

To each symbol $s \in S_{n}$ also correspond two procedures
$inh_{n,s}$ and $syn_{n,s}$:
\begin{quote}
$inh_{n,s} : PP_{n} \rightarrow PI_{s}$ \\
$syn_{n,s} : PP_{n} \cup PS_{s} \cup SA_{s} \rightarrow PP_{n}$.
\end{quote}
Note that the procedure $syn_{n,s}$ is very similar to $act_{n,s}$, and changes
the value of procedural parameters, whereas $inh_{n,s}$ assigns values to all
attributes in $PI_{s}$.

To each production rule $P_{n}$ and hence to each $n \in N$, correspond
two procedures $Pinit_{n}$ and $syn_{n}$. The procedure $Pinit_{n}$
initializes the parameters $PP_{n}$ in terms of the elements of $PI_{n}$ and
$SA_{n}$
whereas $syn_{n}$ assigns values to all elements of $PS_{n}$ in
terms of the elements of $PP_{n}$:
\begin{quote}
$Pinit_{n} : PI_{n} \cup SA_{n} \rightarrow PP_{n}$ \\
$syn_{n} : PP_{n} \rightarrow PS_{n}$
\end{quote}
\section{Atribute Evaluation}
In this section we will specify the order in which the attributes are to be
evaluated for a given syntax tree. As was mentioned before, this will be done
by defining a process for each tree node. The concurrent execution of all such
processes defines the correct attribute evaluation.

To each node in the syntax tree we assign a process, with communication
channels to and from the processes of its neighbouring nodes, i.e. to its
father (if present) and its sons (if present).
These channels are named by the relevant categorial symbol.
These names are unique as long as $S_{n} \cap S_{m} = \emptyset$ if $n \neq m$.
The order of evaluation is such that for each node in a
syntax tree firstly the surface attributes are evaluated, then the inherited
procedural attributes, and lastly the synthesized procedural attributes.

The dependencies of the various attribute evaluations is such that the
attribute evaluation for a given tree takes place in three passes across the
tree, starting at the leaves of the syntax tree, going up to the top, down to
the leaves again, and once more up to the top.

It would be straightforward to increase the number of passes arbitrarily, but
in practise three passes is often enough. In fact, I think it might even be
wise to force the writer of the attribute grammar to write a three-pass
grammar. For one, it makes the grammar as a whole more readable, whereas also
the efficiency of the attribute evaluation is increased.

A process at a node $t$ with $t.n = n$ looks like
\begin{tabbing}
~~~~\=~~~~\=~~~~\=\\
\> BEGIN\\
\> IF \#sons(t) $ \geq 1$ THEN\\
\> \> $Sinit_{n}$($\cal{SP}$$_{n}$);\\
\> \> FOR i=1 TO \#sons(t)\\
\> \> \> DO\\
\> \> \>    $s:=t.t_{i}.s$;\\
\> \> \>    $s ? SA_{s}$;\\
\> \> \>    $pre_{n,s}(SA_{s},SP_{n}$,$status$);\\
\> \> \>    $act_{n,s}(SA_{s}$,$\cal{SP}$$_{n}$)\\
\> \> \> OD;\\
\> \> $pre_{n}(SP_{n}$,$status$);\\
\> \> $act_{n}(SP_{n}$,$\cal{SA}$$_{n}$)\\
\> ELSE $terminalattributes$(t,$\cal{SA}$$_{n}$)\\
\> FI;\\
\> $t.s ! SA_{n}$;\\
\> $t.s ? PI_{n}$;\\
\> $Pinit_{n}(SA_{n},PI_{n}$,$\cal{PP}$$_{n}$);\\
\> FOR i=1 TO \#sons(t)\\
\> \> DO \\
\> \>    $s:=t.t_{i}.s$;\\
\> \>    $inh_{n,s}(PP_{n}$,$\cal{PI}$$_{s}$);\\
\> \>    $s ! PI_{s}$;\\
\> \>    $s ? PS_{s}$;\\
\> \>    $syn_{n,s}(PP_{n},PS_{s},SA_{s}$,$\cal{PP}$$_{n}$)\\
\> \> OD;\\
\> $syn_{n}(PP_{n}$,$\cal{PS}$$_{n}$);\\
\> $t.s ! PS_{n}$\\
\> END;\\
\end{tabbing}
The out parameters of procedures have been written with calligraphic letters.
The procedure $terminalattributes$ gives values to the surface attributes of
terminals. The exclamation mark signals the sending of a message through the
channel mentioned before it, and the question mark signals the waiting for a
message on the mentioned channel. For convenience, we have assumed that the
process at the top of the tree has a parent process doing nothing but accepting
the messages from the UTT process containing synthesized attributes and
sending an empty message to the UTT process to give $PI_{UTT} (= \emptyset$) a
"value".

In the above process definition, whenever the variable $status$
becomes false, the attribute evaluation is assumed to stop. In this way,
syntax trees can be rejected during the first pass, the surface attribute
evaluation. Surface attributes can thus be used to single out one syntax tree
in cases where a sentence leads to more than one parse. It is assumed that
only one syntax tree survived the first pass. Of course, this tree could still
be rejected 'by hand' by introducing some boolean attribute in $PS_{UTT}$,
signalling the correctness.

In the current implementation (in pascal), surface attributes are calculated
in the scanner (terminal attributes) and during the syntax tree construction,
and procedural attributes are evaluated by procedures rather than processes
(hence their name), with communication channels implemented as in and out
parameters of these procedures.

\section{Discussion}
We have presented a specific kind of attribute grammar, to be used to specify
a compiler. The writing of the grammar comes down to:
\begin{itemize}
\item Define the basis grammar, i.e. the sets N, T, and the production rules,
      in particular its regular expressions.
\item Define, together with their types, the sets of attributes $SA,PI,PS$,
      and the parameter sets $SP,PP$.
\item Define the procedures $Sinit,Pinit,pre,act,inh,syn$.
\end{itemize}
To keep a grammar like this understandable, it is wise to separate
clearly between syntax and semantics. Preferably the borderline between the
two should be between the evaluation of surface attributes and procedural
attributes. Thus, surface attribute evaluation is primarily meant for syntactic
purposes, which is also the rationale for its power to filter out syntax trees.
\end{document}


ROSETTA.sty
\typeout{Document Style 'Rosetta'. Version 0.2 - released  SEP-1986}
\def\@ptsize{1}
\@namedef{ds@10pt}{\def\@ptsize{0}}
\@namedef{ds@12pt}{\def\@ptsize{2}}
\@twosidetrue
\@mparswitchtrue
\def\ds@draft{\overfullrule 5pt}
\@options
\input art1\@ptsize.sty\relax


\def\labelenumi{\arabic{enumi}.}
\def\theenumi{\arabic{enumi}}
\def\labelenumii{(\alph{enumii})}
\def\theenumii{\alph{enumii}}
\def\p@enumii{\theenumi}
\def\labelenumiii{\roman{enumiii}.}
\def\theenumiii{\roman{enumiii}}
\def\p@enumiii{\theenumi(\theenumii)}
\def\labelenumiv{\Alph{enumiv}.}
\def\theenumiv{\Alph{enumiv}}
\def\p@enumiv{\p@enumiii\theenumiii}
\def\labelitemi{$\bullet$}
\def\labelitemii{\bf --}
\def\labelitemiii{$\ast$}
\def\labelitemiv{$\cdot$}
\def\verse{
   \let\\=\@centercr
   \list{}{\itemsep\z@ \itemindent -1.5em\listparindent \itemindent
      \rightmargin\leftmargin\advance\leftmargin 1.5em}
   \item[]}
\let\endverse\endlist
\def\quotation{
   \list{}{\listparindent 1.5em
      \itemindent\listparindent
      \rightmargin\leftmargin \parsep 0pt plus 1pt}\item[]}
\let\endquotation=\endlist
\def\quote{
   \list{}{\rightmargin\leftmargin}\item[]}
\let\endquote=\endlist
\def\descriptionlabel#1{\hspace\labelsep \bf #1}
\def\description{
   \list{}{\labelwidth\z@ \itemindent-\leftmargin
      \let\makelabel\descriptionlabel}}
\let\enddescription\endlist


\def\@begintheorem#1#2{\it \trivlist \item[\hskip \labelsep{\bf #1\ #2}]}
\def\@endtheorem{\endtrivlist}
\def\theequation{\arabic{equation}}
\def\titlepage{
   \@restonecolfalse
   \if@twocolumn\@restonecoltrue\onecolumn
   \else \newpage
   \fi
   \thispagestyle{empty}\c@page\z@}
\def\endtitlepage{\if@restonecol\twocolumn \else \newpage \fi}
\arraycolsep 5pt \tabcolsep 6pt \arrayrulewidth .4pt \doublerulesep 2pt
\tabbingsep \labelsep
\skip\@mpfootins = \skip\footins
\fboxsep = 3pt \fboxrule = .4pt


\newcounter{part}
\newcounter {section}
\newcounter {subsection}[section]
\newcounter {subsubsection}[subsection]
\newcounter {paragraph}[subsubsection]
\newcounter {subparagraph}[paragraph]
\def\thepart{\Roman{part}} \def\thesection {\arabic{section}}
\def\thesubsection {\thesection.\arabic{subsection}}
\def\thesubsubsection {\thesubsection .\arabic{subsubsection}}
\def\theparagraph {\thesubsubsection.\arabic{paragraph}}
\def\thesubparagraph {\theparagraph.\arabic{subparagraph}}


\def\@pnumwidth{1.55em}
\def\@tocrmarg {2.55em}
\def\@dotsep{4.5}
\setcounter{tocdepth}{3}
\def\tableofcontents{\section*{Contents\markboth{}{}}
\@starttoc{toc}}
\def\l@part#1#2{
   \addpenalty{-\@highpenalty}
   \addvspace{2.25em plus 1pt}
   \begingroup
      \@tempdima 3em \parindent \z@ \rightskip \@pnumwidth \parfillskip
      -\@pnumwidth {\large \bf \leavevmode #1\hfil \hbox to\@pnumwidth{\hss #2}}
      \par \nobreak
   \endgroup}
\def\l@section#1#2{
   \addpenalty{-\@highpenalty}
   \addvspace{1.0em plus 1pt}
   \@tempdima 1.5em
   \begingroup
      \parindent \z@ \rightskip \@pnumwidth
      \parfillskip -\@pnumwidth
      \bf \leavevmode #1\hfil \hbox to\@pnumwidth{\hss #2}
      \par
   \endgroup}
\def\l@subsection{\@dottedtocline{2}{1.5em}{2.3em}}
\def\l@subsubsection{\@dottedtocline{3}{3.8em}{3.2em}}
\def\l@paragraph{\@dottedtocline{4}{7.0em}{4.1em}}
\def\l@subparagraph{\@dottedtocline{5}{10em}{5em}}
\def\listoffigures{
   \section*{List of Figures\markboth{}{}}
   \@starttoc{lof}}
   \def\l@figure{\@dottedtocline{1}{1.5em}{2.3em}}
   \def\listoftables{\section*{List of Tables\markboth{}{}}
   \@starttoc{lot}}
\let\l@table\l@figure


\def\thebibliography#1{
   \addcontentsline{toc}
   {section}{References}\section*{References\markboth{}{}}
   \list{[\arabic{enumi}]}
        {\settowidth\labelwidth{[#1]}\leftmargin\labelwidth
         \advance\leftmargin\labelsep\usecounter{enumi}}}
\let\endthebibliography=\endlist


\newif\if@restonecol
\def\theindex{
   \@restonecoltrue\if@twocolumn\@restonecolfalse\fi
   \columnseprule \z@
   \columnsep 35pt\twocolumn[\section*{Index}]
   \markboth{}{}
   \thispagestyle{plain}\parindent\z@
   \parskip\z@ plus .3pt\relax
   \let\item\@idxitem}
\def\@idxitem{\par\hangindent 40pt}
\def\subitem{\par\hangindent 40pt \hspace*{20pt}}
\def\subsubitem{\par\hangindent 40pt \hspace*{30pt}}
\def\endtheindex{\if@restonecol\onecolumn\else\clearpage\fi}
\def\indexspace{\par \vskip 10pt plus 5pt minus 3pt\relax}


\def\footnoterule{
   \kern-1\p@
   \hrule width .4\columnwidth
   \kern .6\p@}
\long\def\@makefntext#1{
   \@setpar{\@@par\@tempdima \hsize
   \advance\@tempdima-10pt\parshape \@ne 10pt \@tempdima}\par
   \parindent 1em\noindent \hbox to \z@{\hss$^{\@thefnmark}$}#1}


\setcounter{topnumber}{2}
\def\topfraction{.7}
\setcounter{bottomnumber}{1}
\def\bottomfraction{.3}
\setcounter{totalnumber}{3}
\def\textfraction{.2}
\def\floatpagefraction{.5}
\setcounter{dbltopnumber}{2}
\def\dbltopfraction{.7}
\def\dblfloatpagefraction{.5}
\long\def\@makecaption#1#2{
   \vskip 10pt
   \setbox\@tempboxa\hbox{#1: #2}
   \ifdim \wd\@tempboxa >\hsize \unhbox\@tempboxa\par
   \else \hbox to\hsize{\hfil\box\@tempboxa\hfil}
   \fi}
\newcounter{figure}
\def\thefigure{\@arabic\c@figure}
\def\fps@figure{tbp}
\def\ftype@figure{1}
\def\ext@figure{lof}
\def\fnum@figure{Figure \thefigure}
\def\figure{\@float{figure}}
\let\endfigure\end@float
\@namedef{figure*}{\@dblfloat{figure}}
\@namedef{endfigure*}{\end@dblfloat}
\newcounter{table}
\def\thetable{\@arabic\c@table}
\def\fps@table{tbp}
\def\ftype@table{2}
\def\ext@table{lot}
\def\fnum@table{Table \thetable}
\def\table{\@float{table}}
\let\endtable\end@float
\@namedef{table*}{\@dblfloat{table}}
\@namedef{endtable*}{\end@dblfloat}


\def\maketitle{
   \par
   \begingroup
      \def\thefootnote{\fnsymbol{footnote}}
      \def\@makefnmark{\hbox to 0pt{$^{\@thefnmark}$\hss}}
      \if@twocolumn \twocolumn[\@maketitle]
      \else \newpage \global\@topnum\z@ \@maketitle
      \fi
      \thispagestyle{plain}
      \@thanks
   \endgroup
   \setcounter{footnote}{0}
   \let\maketitle\relax
   \let\@maketitle\relax
   \gdef\@thanks{}
   \gdef\@author{}
   \gdef\@title{}
   \let\thanks\relax}
\def\@maketitle{
   \newpage
   \null
   \vskip 2em
   \begin{center}{\LARGE \@title \par}
      \vskip 1.5em
      {\large \lineskip .5em \begin{tabular}[t]{c}\@author \end{tabular}\par}
      \vskip 1em {\large \@date}
   \end{center}
   \par
   \vskip 1.5em}
\def\abstract{
   \if@twocolumn \section*{Abstract}
   \else
      \small
      \begin{center} {\bf Abstract\vspace{-.5em}\vspace{0pt}} \end{center}
      \quotation
   \fi}
\def\endabstract{\if@twocolumn\else\endquotation\fi}


\mark{{}{}}
\if@twoside
   \def\ps@headings{
      \def\@oddfoot{Rosetta Doc. \@RosDocNr\hfil \@RosDate}
      \def\@evenfoot{Rosetta Doc. \@RosDocNr\hfil \@RosDate}
      \def\@evenhead{\rm\thepage\hfil \sl \rightmark}
      \def\@oddhead{\hbox{}\sl \leftmark \hfil\rm\thepage}
      \def\sectionmark##1{\markboth {}{}}
      \def\subsectionmark##1{}}
\else
   \def\ps@headings{
      \def\@oddfoot{Rosetta Doc. \@RosDocNr\hfil \@RosDate}
      \def\@evenfoot{Rosetta Doc. \@RosDocNr\hfil \@RosDate}
      \def\@oddhead{\hbox{}\sl \rightmark \hfil \rm\thepage}
      \def\sectionmark##1{\markboth {}{}}
      \def\subsectionmark##1{}}
\fi
\def\ps@myheadings{
   \def\@oddhead{\hbox{}\sl\@rhead \hfil \rm\thepage}
   \def\@oddfoot{}
   \def\@evenhead{\rm \thepage\hfil\sl\@lhead\hbox{}}
   \def\@evenfoot{}
   \def\sectionmark##1{}
   \def\subsectionmark##1{}}


\def\today{
   \ifcase\month\or January\or February\or March\or April\or May\or June\or
      July\or August\or September\or October\or November\or December
   \fi
   \space\number\day, \number\year}


\ps@plain \pagenumbering{arabic} \onecolumn \if@twoside\else\raggedbottom\fi




% the Rosetta title page
\newcommand{\MakeRosTitle}{
   \begin{titlepage}
      \begin{large}
     \begin{figure}[t]
        \begin{picture}(405,100)(0,0)
           \put(0,100){\line(1,0){404}}
           \put(0,75){Project {\bf Rosetta}}
           \put(93.5,75){:}
           \put(108,75){Machine Translation}
           \put(0,50){Topic}
           \put(93.5,50){:}
           \put(108,50){\@RosTopic}
           \put(0,30){\line(1,0){404}}
        \end{picture}
     \end{figure}
     \bigskip
     \bigskip
     \begin{list}{-}{\setlength{\leftmargin}{3.0cm}
             \setlength{\labelwidth}{2.7cm}
             \setlength{\topsep}{2cm}}
        \item [{\rm Title \hfill :}] {{\bf \@RosTitle}}
        \item [{\rm Author \hfill :}] {\@RosAuthor}
        \bigskip
        \bigskip
        \bigskip
        \item [{\rm Doc.Nr. \hfill :}] {\@RosDocNr}
        \item [{\rm Date \hfill :}] {\@RosDate}
        \item [{\rm Status \hfill :}] {\@RosStatus}
        \item [{\rm Supersedes \hfill :}] {\@RosSupersedes}
        \item [{\rm Distribution \hfill :}] {\@RosDistribution}
        \item [{\rm Clearance \hfill :}] {\@RosClearance}
        \item [{\rm Keywords \hfill :}] {\@RosKeywords}
     \end{list}
      \end{large}
      \title{\@RosTitle}
      \begin{figure}[b]
     \begin{picture}(404,64)(0,0)
        \put(0,64){\line(1,0){404}}
        \put(0,-4){\line(1,0){404}}
        \put(0,59){\line(1,0){42}}
        \begin{small}
        \put(3,48){\sf PHILIPS}
        \end{small}
        \put(0,23){\line(0,1){36}}
        \put(42,23){\line(0,1){36}}
        \put(21,23){\oval(42,42)[bl]}
        \put(21,23){\oval(42,42)[br]}
        \put(21,23){\circle{40}}
        \put(4,33){\line(1,0){10}}
        \put(9,28){\line(0,1){10}}
        \put(9,36){\line(1,0){6}}
        \put(12,33){\line(0,1){6}}
        \put(29,13){\line(1,0){10}}
        \put(34,8){\line(0,1){10}}
        \put(28,10){\line(1,0){6}}
        \put(31,7){\line(0,1){6}}

        \put(1,21){\line(1,0){0.5}}
        \put(1.5,21.3){\line(1,0){0.5}}
        \put(2,21.6){\line(1,0){0.5}}
        \put(2.5,21.9){\line(1,0){0.5}}
        \put(3,22.1){\line(1,0){0.5}}
        \put(3.5,22.3){\line(1,0){0.5}}
        \put(4,22.5){\line(1,0){0.5}}
        \put(4.5,22.7){\line(1,0){0.5}}
        \put(5,22.8){\line(1,0){0.5}}
        \put(5.5,22.9){\line(1,0){0.5}}
        \put(6,23){\line(1,0){0.5}}
        \put(6.5,22.9){\line(1,0){0.5}}
        \put(7,22.8){\line(1,0){0.5}}
        \put(7.5,22.7){\line(1,0){0.5}}
        \put(8,22.5){\line(1,0){0.5}}
        \put(8.5,22.3){\line(1,0){0.5}}
        \put(9,22.1){\line(1,0){0.5}}
        \put(9.5,21.9){\line(1,0){0.5}}
        \put(10,21.6){\line(1,0){0.5}}
        \put(10.5,21.3){\line(1,0){0.5}}

        \put(1,23){\line(1,0){0.5}}
        \put(1.5,23.3){\line(1,0){0.5}}
        \put(2,23.6){\line(1,0){0.5}}
        \put(2.5,23.9){\line(1,0){0.5}}
        \put(3,24.1){\line(1,0){0.5}}
        \put(3.5,24.3){\line(1,0){0.5}}
        \put(4,24.5){\line(1,0){0.5}}
        \put(4.5,24.7){\line(1,0){0.5}}
        \put(5,24.8){\line(1,0){0.5}}
        \put(5.5,24.9){\line(1,0){0.5}}
        \put(6,25){\line(1,0){0.5}}
        \put(6.5,24.9){\line(1,0){0.5}}
        \put(7,24.8){\line(1,0){0.5}}
        \put(7.5,24.7){\line(1,0){0.5}}
        \put(8,24.5){\line(1,0){0.5}}
        \put(8.5,24.3){\line(1,0){0.5}}
        \put(9,24.1){\line(1,0){0.5}}
        \put(9.5,23.9){\line(1,0){0.5}}
        \put(10,23.6){\line(1,0){0.5}}
        \put(10.5,23.3){\line(1,0){0.5}}

        \put(1,25){\line(1,0){0.5}}
        \put(1.5,25.3){\line(1,0){0.5}}
        \put(2,25.6){\line(1,0){0.5}}
        \put(2.5,25.9){\line(1,0){0.5}}
        \put(3,26.1){\line(1,0){0.5}}
        \put(3.5,26.3){\line(1,0){0.5}}
        \put(4,26.5){\line(1,0){0.5}}
        \put(4.5,26.7){\line(1,0){0.5}}
        \put(5,26.8){\line(1,0){0.5}}
        \put(5.5,26.9){\line(1,0){0.5}}
        \put(6,27){\line(1,0){0.5}}
        \put(6.5,26.9){\line(1,0){0.5}}
        \put(7,26.8){\line(1,0){0.5}}
        \put(7.5,26.7){\line(1,0){0.5}}
        \put(8,26.5){\line(1,0){0.5}}
        \put(8.5,26.3){\line(1,0){0.5}}
        \put(9,26.1){\line(1,0){0.5}}
        \put(9.5,25.9){\line(1,0){0.5}}
        \put(10,25.6){\line(1,0){0.5}}
        \put(10.5,25.3){\line(1,0){0.5}}

        \put(11,21){\line(1,0){0.5}}
        \put(11.5,20.7){\line(1,0){0.5}}
        \put(12,20.4){\line(1,0){0.5}}
        \put(12.5,20.1){\line(1,0){0.5}}
        \put(13,19.9){\line(1,0){0.5}}
        \put(13.5,19.7){\line(1,0){0.5}}
        \put(14,19.5){\line(1,0){0.5}}
        \put(14.5,19.3){\line(1,0){0.5}}
        \put(15,19.2){\line(1,0){0.5}}
        \put(15.5,19.1){\line(1,0){0.5}}
        \put(16,19){\line(1,0){0.5}}
        \put(16.5,19.1){\line(1,0){0.5}}
        \put(17,19.2){\line(1,0){0.5}}
        \put(17.5,19.3){\line(1,0){0.5}}
        \put(18,19.5){\line(1,0){0.5}}
        \put(18.5,19.7){\line(1,0){0.5}}
        \put(19,19.9){\line(1,0){0.5}}
        \put(19.5,20.1){\line(1,0){0.5}}
        \put(20,20.4){\line(1,0){0.5}}
        \put(20.5,20.7){\line(1,0){0.5}}

        \put(11,23){\line(1,0){0.5}}
        \put(11.5,22.7){\line(1,0){0.5}}
        \put(12,22.4){\line(1,0){0.5}}
        \put(12.5,22.1){\line(1,0){0.5}}
        \put(13,21.9){\line(1,0){0.5}}
        \put(13.5,21.7){\line(1,0){0.5}}
        \put(14,21.5){\line(1,0){0.5}}
        \put(14.5,21.3){\line(1,0){0.5}}
        \put(15,21.2){\line(1,0){0.5}}
        \put(15.5,21.1){\line(1,0){0.5}}
        \put(16,21){\line(1,0){0.5}}
        \put(16.5,21.1){\line(1,0){0.5}}
        \put(17,21.2){\line(1,0){0.5}}
        \put(17.5,21.3){\line(1,0){0.5}}
        \put(18,21.5){\line(1,0){0.5}}
        \put(18.5,21.7){\line(1,0){0.5}}
        \put(19,21.9){\line(1,0){0.5}}
        \put(19.5,22.1){\line(1,0){0.5}}
        \put(20,22.4){\line(1,0){0.5}}
        \put(20.5,22.7){\line(1,0){0.5}}

        \put(11,25){\line(1,0){0.5}}
        \put(11.5,24.7){\line(1,0){0.5}}
        \put(12,24.4){\line(1,0){0.5}}
        \put(12.5,24.1){\line(1,0){0.5}}
        \put(13,23.9){\line(1,0){0.5}}
        \put(13.5,23.7){\line(1,0){0.5}}
        \put(14,23.5){\line(1,0){0.5}}
        \put(14.5,23.3){\line(1,0){0.5}}
        \put(15,23.2){\line(1,0){0.5}}
        \put(15.5,23.1){\line(1,0){0.5}}
        \put(16,23){\line(1,0){0.5}}
        \put(16.5,23.1){\line(1,0){0.5}}
        \put(17,23.2){\line(1,0){0.5}}
        \put(17.5,23.3){\line(1,0){0.5}}
        \put(18,23.5){\line(1,0){0.5}}
        \put(18.5,23.7){\line(1,0){0.5}}
        \put(19,23.9){\line(1,0){0.5}}
        \put(19.5,24.1){\line(1,0){0.5}}
        \put(20,24.4){\line(1,0){0.5}}
        \put(20.5,24.7){\line(1,0){0.5}}

        \put(21,21){\line(1,0){0.5}}
        \put(21.5,21.3){\line(1,0){0.5}}
        \put(22,21.6){\line(1,0){0.5}}
        \put(22.5,21.9){\line(1,0){0.5}}
        \put(23,22.1){\line(1,0){0.5}}
        \put(23.5,22.3){\line(1,0){0.5}}
        \put(24,22.5){\line(1,0){0.5}}
        \put(24.5,22.7){\line(1,0){0.5}}
        \put(25,22.8){\line(1,0){0.5}}
        \put(25.5,23.9){\line(1,0){0.5}}
        \put(26,23){\line(1,0){0.5}}
        \put(26.5,22.9){\line(1,0){0.5}}
        \put(27,22.8){\line(1,0){0.5}}
        \put(27.5,22.7){\line(1,0){0.5}}
        \put(28,22.5){\line(1,0){0.5}}
        \put(28.5,22.3){\line(1,0){0.5}}
        \put(29,22.1){\line(1,0){0.5}}
        \put(29.5,21.9){\line(1,0){0.5}}
        \put(30,21.6){\line(1,0){0.5}}
        \put(30.5,21.3){\line(1,0){0.5}}

        \put(21,23){\line(1,0){0.5}}
        \put(21.5,23.3){\line(1,0){0.5}}
        \put(22,23.6){\line(1,0){0.5}}
        \put(22.5,23.9){\line(1,0){0.5}}
        \put(23,24.1){\line(1,0){0.5}}
        \put(23.5,24.3){\line(1,0){0.5}}
        \put(24,24.5){\line(1,0){0.5}}
        \put(24.5,24.7){\line(1,0){0.5}}
        \put(25,24.8){\line(1,0){0.5}}
        \put(25.5,24.9){\line(1,0){0.5}}
        \put(26,25){\line(1,0){0.5}}
        \put(26.5,24.9){\line(1,0){0.5}}
        \put(27,24.8){\line(1,0){0.5}}
        \put(27.5,24.7){\line(1,0){0.5}}
        \put(28,24.5){\line(1,0){0.5}}
        \put(28.5,24.3){\line(1,0){0.5}}
        \put(29,24.1){\line(1,0){0.5}}
        \put(29.5,23.9){\line(1,0){0.5}}
        \put(30,23.6){\line(1,0){0.5}}
        \put(30.5,23.3){\line(1,0){0.5}}

        \put(21,25){\line(1,0){0.5}}
        \put(21.5,25.3){\line(1,0){0.5}}
        \put(22,25.6){\line(1,0){0.5}}
        \put(22.5,25.9){\line(1,0){0.5}}
        \put(23,26.1){\line(1,0){0.5}}
        \put(23.5,26.3){\line(1,0){0.5}}
        \put(24,26.5){\line(1,0){0.5}}
        \put(24.5,26.7){\line(1,0){0.5}}
        \put(25,26.8){\line(1,0){0.5}}
        \put(25.5,26.9){\line(1,0){0.5}}
        \put(26,27){\line(1,0){0.5}}
        \put(26.5,26.9){\line(1,0){0.5}}
        \put(27,26.8){\line(1,0){0.5}}
        \put(27.5,26.7){\line(1,0){0.5}}
        \put(28,26.5){\line(1,0){0.5}}
        \put(28.5,26.3){\line(1,0){0.5}}
        \put(29,26.1){\line(1,0){0.5}}
        \put(29.5,25.9){\line(1,0){0.5}}
        \put(30,25.6){\line(1,0){0.5}}
        \put(30.5,25.3){\line(1,0){0.5}}

        \put(31,21){\line(1,0){0.5}}
        \put(31.5,20.7){\line(1,0){0.5}}
        \put(32,20.4){\line(1,0){0.5}}
        \put(32.5,20.1){\line(1,0){0.5}}
        \put(33,19.9){\line(1,0){0.5}}
        \put(33.5,19.7){\line(1,0){0.5}}
        \put(34,19.5){\line(1,0){0.5}}
        \put(34.5,19.3){\line(1,0){0.5}}
        \put(35,19.2){\line(1,0){0.5}}
        \put(35.5,19.1){\line(1,0){0.5}}
        \put(36,19){\line(1,0){0.5}}
        \put(36.5,19.1){\line(1,0){0.5}}
        \put(37,19.2){\line(1,0){0.5}}
        \put(37.5,19.3){\line(1,0){0.5}}
        \put(38,19.5){\line(1,0){0.5}}
        \put(38.5,19.7){\line(1,0){0.5}}
        \put(39,19.9){\line(1,0){0.5}}
        \put(39.5,20.1){\line(1,0){0.5}}
        \put(40,20.4){\line(1,0){0.5}}
        \put(40.5,20.7){\line(1,0){0.5}}

        \put(31,23){\line(1,0){0.5}}
        \put(31.5,22.7){\line(1,0){0.5}}
        \put(32,22.4){\line(1,0){0.5}}
        \put(32.5,22.1){\line(1,0){0.5}}
        \put(33,21.9){\line(1,0){0.5}}
        \put(33.5,21.7){\line(1,0){0.5}}
        \put(34,21.5){\line(1,0){0.5}}
        \put(34.5,21.3){\line(1,0){0.5}}
        \put(35,21.2){\line(1,0){0.5}}
        \put(35.5,21.1){\line(1,0){0.5}}
        \put(36,21){\line(1,0){0.5}}
        \put(36.5,21.1){\line(1,0){0.5}}
        \put(37,21.2){\line(1,0){0.5}}
        \put(37.5,21.3){\line(1,0){0.5}}
        \put(38,21.5){\line(1,0){0.5}}
        \put(38.5,21.7){\line(1,0){0.5}}
        \put(39,21.9){\line(1,0){0.5}}
        \put(39.5,22.1){\line(1,0){0.5}}
        \put(40,22.4){\line(1,0){0.5}}
        \put(40.5,22.7){\line(1,0){0.5}}

        \put(31,25){\line(1,0){0.5}}
        \put(31.5,24.7){\line(1,0){0.5}}
        \put(32,24.4){\line(1,0){0.5}}
        \put(32.5,24.1){\line(1,0){0.5}}
        \put(33,23.9){\line(1,0){0.5}}
        \put(33.5,23.7){\line(1,0){0.5}}
        \put(34,23.5){\line(1,0){0.5}}
        \put(34.5,23.3){\line(1,0){0.5}}
        \put(35,23.2){\line(1,0){0.5}}
        \put(35.5,23.1){\line(1,0){0.5}}
        \put(36,23){\line(1,0){0.5}}
        \put(36.5,23.1){\line(1,0){0.5}}
        \put(37,23.2){\line(1,0){0.5}}
        \put(37.5,23.3){\line(1,0){0.5}}
        \put(38,23.5){\line(1,0){0.5}}
        \put(38.5,23.7){\line(1,0){0.5}}
        \put(39,23.9){\line(1,0){0.5}}
        \put(39.5,24.1){\line(1,0){0.5}}
        \put(40,24.4){\line(1,0){0.5}}
        \put(40.5,24.7){\line(1,0){0.5}}
        \begin{large}
           \put(60,45){Philips Research Laboratories}
           \put(60,30){\copyright\ 1986 Nederlandse Philips Bedrijven B.V.}
        \end{large}
     \end{picture}
      \end{figure}
      \newpage
      \pagenumbering{roman}
      \tableofcontents
      \newpage
      \pagenumbering{arabic}
   \end{titlepage}
}
\title{}
\topmargin 0pt
\oddsidemargin 36pt
\evensidemargin 36pt
\textheight 600pt
\textwidth 405pt
\pagestyle{headings}
\newcommand{\@RosTopic}{General}
\newcommand{\@RosTitle}{-}
\newcommand{\@RosAuthor}{-}
\newcommand{\@RosDocNr}{}
\newcommand{\@RosDate}{\today}
\newcommand{\@RosStatus}{informal}
\newcommand{\@RosSupersedes}{-}
\newcommand{\@RosDistribution}{Project}
\newcommand{\@RosClearance}{Project}
\newcommand{\@RosKeywords}{}
\newcommand{\RosTopic}[1]{\renewcommand{\@RosTopic}{#1}}
\newcommand{\RosTitle}[1]{\renewcommand{\@RosTitle}{#1}}
\newcommand{\RosAuthor}[1]{\renewcommand{\@RosAuthor}{#1}}
\newcommand{\RosDocNr}[1]{\renewcommand{\@RosDocNr}{#1}}
\newcommand{\RosDate}[1]{\renewcommand{\@RosDate}{#1}}
\newcommand{\RosStatus}[1]{\renewcommand{\@RosStatus}{#1}}
\newcommand{\RosSupersedes}[1]{\renewcommand{\@RosSupersedes}{#1}}
\newcommand{\RosDistribution}[1]{\renewcommand{\@RosDistribution}{#1}}
\newcommand{\RosClearance}[1]{\renewcommand{\@RosClearance}{#1}}
\newcommand{\RosKeywords}[1]{\renewcommand{\@RosKeywords}{#1}}

