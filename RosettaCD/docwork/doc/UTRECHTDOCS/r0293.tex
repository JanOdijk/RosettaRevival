
\documentstyle{Rosetta}
\begin{document}
   \RosTopic{Rosetta3.Linguistics.Minutes}
   \RosTitle{Notulen Linguistenvergadering 01-12-88}
   \RosAuthor{Andr\'{e} Schenk}
   \RosDocNr{0293}
   \RosDate{December 8, 1988}
   \RosStatus{approved}
   \RosSupersedes{-}
   \RosDistribution{Project}
   \RosClearance{Project}
   \RosKeywords{minutes}
   \MakeRosTitle
%
%
\begin{description}
\item[Aanwezig:] Lisette Appelo, Franciska de Jong, Elly van Munster,
                 Jan Odijk, Margreet Sanders ,
                 Andr\'{e} Schenk
\item[Afwezig:] Harm Smit


\item[Agenda:]\mbox{}
  \begin{enumerate}
  \item Notulen
  \item Empty Nouns
  \item Toegestaan, etc.
  \item Problemen met adverbia als graag en misschien
  \end{enumerate}
\end{description}

\section{Notulen}
De notulen van de vorige vergaderingen werden met enkele kleine wijzigingen
goedgekeurd. 

\section{Empty Nouns}
Empty nouns worden in kleinere kring besproken. Verslag volgt op de volgende 
vergadering.

\section{Toegestaan, etc.}
Een probleem is dat als {\em allowed} vp210 heeft, andere modalen ook vp210
moeten hebben. Het is natuurlijker om {\em toegestaan} uitzonderlijk te maken. 
Deze krijgt dan adjp021; {\em allowed, can/may, mogen} krijgen dan vp120.

Een soortgelijk probleem zie je bij {\em graag, like, gustar, bevallen}, alleen 
is de zaak hier wat ingewikkelder. Er moet nog over nagedacht worden. Een 
suggestie is om twee {\em gustar}s op te nemen (voor {\em bevallen} en {\em 
graag}); {\em bevallen} zal dan ook twee keer opgenomen moeten worden.

\section{Problemen met adverbia als graag en misschien}
Zinnen als {\em hij zwemt misschien} en {\em hij zwemt graag} gaan goed. Een
zin als {\em hij koopt graag ossen} ging echter verkeerd, vanwege het feit dat
de casus van ossen default gemaakt werd terwijl hij later in de derivatie geen
default mocht zijn. In het algemeen is het zo dat attribuut waarden e.d. die
ontdaan zijn in analyse hersteld moeten in de context van adverbia als {\em
graag} en {\em misschien}. Het is nog niet duidelijk hoeveel hiervoor veranderd
moet worden in het Nederlands. Het Engels is misschien toevallig goed. 

\end{document}


