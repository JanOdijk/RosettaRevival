
\documentstyle{Rosetta}
\begin{document}
   \RosTopic{General}
   \RosTitle{Notulen Rosetta vergadering 5-10-89}
   \RosAuthor{Margreet Sanders}
   \RosDocNr{411}
   \RosDate{November 29, 1989}
   \RosStatus{approved}
   \RosSupersedes{-}
   \RosDistribution{Project}
   \RosClearance{Project}
   \RosKeywords{minutes, jaarplan}
   \MakeRosTitle
%
%
\begin{description}
\item[Aanwezig:] Lisette Appelo, Franciska de Jong, 
                 Jan Landsbergen, Ren\'{e} Leermakers, 
                 Elly van Munster, 
                 Jan Odijk, Elena Pinillos, Joep Rous, Margreet Sanders,
                 Andr\'{e} Schenk, Harm Smit
\item[Afwezig:]  ---
\item[Agenda:]\mbox{}
  \begin{enumerate}
  \item Opening en notulen
  \item TROPICS en onze toekomst
  \item Diversen Extern
  \item Diversen Intern
  \item Jaarplan
  \item Conferentieverslagen
  \item Rondvraag en sluiting
  \end{enumerate}
\end{description}

\section{Opening en notulen}
De notulen van de vorige twee vergaderingen (7-9 en 11-9) worden 
aangenomen zonder inhoudelijke wijzigingen.

\section{TROPICS en onze toekomst}
TDS heeft zijn interesse in parallelle systemen voorlopig verloren. Hiermee is 
niet alle grond onder de voeten van TROPICS weggeslagen: men heeft nu wel 
belangstelling voor gedistribueerde netwerken van PC's, met als concrete 
applicatie het "Ideale Archief". Dit zal onder andere een retrieval systeem van 
teksten in een ongestructureerd bestand moeten omvatten, op basis van queries 
met woordstammen. De groep Landsbergen zou op vier gebieden kunnen bijdragen 
aan dit project:\\
1) het leveren van een `stemmings'algoritme voor het Nederlands  (en evt.\ het 
Spaans), naar analogie van het algoritme dat IJsbrand-Jan Aalbersberg voor 
het vinden van Engelse woordstammen heeft. Misschien valt er ook nog wat te 
verbeteren aan het Engels (mits met behoud van snelheid).\\
2) het leveren van een vertaalwoordenboek voor de gebruikte woordstammen. Dit 
is voor ons eigen werk echter niet interessant.\\
3) het formuleren van een uitgebreidere natuurlijke taalinterface voor de 
querytaal, zodat niet alleen kale woorden kunnen worden opgevraagd, maar ook 
ingewikkelder constructies (bv.\ met een negatie).\\
4) het leveren van een vertaalsysteem voor de documenten uit het archief.
Dit is echter een nogal omvangrijke doelstelling.\\
Voorlopig heeft Jan L.\ een bijdrage van Rosetta van 2 x een half manjaar 
toegezegd voor de periode '90-'91 (vnl.\ voor het stemmingsalgoritme), en een 
wat grotere bijdrage voor '92-'93 (voor uitbreiden van de interface).

Carasso vindt dat we onze lange-termijn doelstelling niet van de korte-termijn 
belangstelling van TDS moeten laten 
afhangen; mocht TDS niet ge\"{i}nteresseerd blijken in interactief vertalen, dan 
wil hij zich wel sterk maken om dit onderwerp tot in 1992 als Basic Research op 
te voeren. Wel ziet hij graag wat concrete spin-off op kortere termijn.

De implementatie van een parallelle versie van Rosetta is dus wat betreft 
TRO\-PICS voorlopig van de baan, vooral omdat de EEG de plannen voor natuurlijke 
taalverwerking toch al zwak vond. Of CAP ge\"{i}nteresseerd is om buiten het 
verband van de EEG om toch met Philips samen te werken is nog niet bekend.

\section{Diversen Extern}
\begin{enumerate}
  \item Er zouden 10 {\bf Japanners} op bezoek komen, maar dat gaat niet door.
  \item Het Philips {\bf persbericht} over Rosetta is in de maak. Franciska heeft 
een verbeterde versie gemaakt van de inhoudelijke kant. Misschien kan het 
bericht verschijnen naar aanleiding van de oratie van Jan L.\ op 1 november.
Overigens zal Jan van 16-20 okt.\ vrij nemen om zijn oratie te schrijven.
  \item Het {\bf marktonderzoek} door M+W Test is afgerond. De resultaten zullen op 
26 oktober in Aken worden gepresenteerd, waarna op 23 november een presentatie 
aan de stuurgroep zal volgen.
  \item Jan L.\ is naar de {\bf SPICOS-III} vergadering in Brussel geweest. 
Siemens en Philips willen hun samenwerking wel voortzetten, maar het is nog niet 
duidelijk op welk aspect het accent moet komen te liggen. Waarschijnlijk wordt 
dat de mens-machine interface, met misschien een opening voor Rosetta als het 
een multilinguale interface betreft. Siemens zal de experts van de 
verschillende disciplines in M\"{u}nchen bijeenroepen, waarna in december een 
globaal plan af moet zijn.
  \item Theo Janssen heeft vergaande idee\"{e}n over het {\bf boek}, maar die 
zitten nu in verhuisdozen. Hij verwacht dat ze er in de herfstvakantie weer uit 
komen. Jan L.\ heeft benadrukt dat Rosetta nog {\em niet\/} definitief heeft 
toegezegd mee te werken.
  \item Jan L.\ is naar P.\ C.\ Uit den Bogaerd op de T.U.E.\ geweest, die daar 
een {\bf corpus} heeft van zakelijke teksten. Het corpus bevat 60.000 
woorden = tokens (6000 lemma's), in brokken van $\pm$ 100 woorden.  Het is 
onderverdeeld in
woorden afkomstig van twee nader gedefinieerde groepen: a) `lagere 
bedrijfsonderdelen' (notulen van vergaderingen van kwaliteitskringen), en b) 
`hogere regionen' (externe en formele en informele interne publicaties). Er is 
een codering in aangebracht op de manier van het Eindhovens corpus. Aangezien 
het een beperkt taalgebruik betreft, en niet zo'n grote omvang heeft, lijkt dit 
corpus niet bijzonder geschikt voor Rosetta. Wel is het waarschijnlijk niet te 
duur in aanschaf.
  \item Binnen Philips moeten alle {\bf documenten} een uniek {\bf nummer} gaan 
krijgen. De codering voor alle Rosetta-documenten (ook brieven!) is als volgt:

\begin{center} RWR-102-RO-89001-MS.
\end{center}

\begin{tabular}{ll}
RW     & : Code voor Research Lab Waalre\\
R      & : Code voor soort document: R = rapport etc., B = brief\\
102    & : Groepsnummer\\
RO     & : Code voor (eerste) auteur. Documenten die via het projectenbureau\\
       & \ \ van Frank Stoots en Fred Robert gaan, krijgen van hun altijd \\
       & \ \ RO als auteurscode. Op andere documenten moet men hier zijn \\
       & \ \ eigen initialen (twee) invullen.\\
89001  & : Documentnummer, beginnend met de laatste twee cijfers van het \\
       & \ \ jaartal. Rosetta-documenten die via het projectenbureau gaan, \\
       & \ \ krijgen daar een nummer toegewezen; voor andere documenten en \\
       & \ \ brieven is via Margot een nummer aan te vragen.\\
MS     & : Initialen van typist (voor Rosetta-documenten dus van (eerste) \\
       & \ \ auteur, voor brieven etc.\ meestal van Margot Franken (MF))
\end{tabular}

  \item {\bf Sunnyvale} is opgeheven als Research 
Laboratorium, en als ontwikkellab toege\-voegd aan Signetics.

\end{enumerate}

\section{Diversen Intern}
\begin{enumerate}
  \item {\bf Elena} is gevraagd om drie maanden langer te blijven (dus tot 1 
april), en zich vanaf 1 november een dag per week in te werken in het 
Rosetta-systeem. Mocht Elly een andere baan vinden, dan kan Elena het werk van 
Elly overnemen. Elly zal een inwerkprogramma regelen.
  \item Andr\'{e} en Jan O.\ willen (wanneer Jan O.\ geen cursus heeft 
tenminste) de {\bf Engelse Surface Parser} gaan schrijven. Margreet is 
(nog?) aanwezig voor het geval er iets niet duidelijk is. Er wordt eerst een 
paar weken proefgedraaid, om te kijken of het zo lukt. Het is niet de bedoeling 
dat bij het testen ook fouten in de M-parser op grote schaal verbeterd worden.
  \item De bespreking op 14 september van doc.\ 401 van Joep Rous over de 
{\bf semanti\-sche 
component} leverde twee concrete aktiepunten op: \\
Jan O.\ schrijft een stuk over semantische types (doc.\ 353) en zal proberen te 
bekijken of dit ook voor voorzetsels een oplossing is, en\\
Lisette maakt een case study van Aktionsart en temporele adverbia, en bekijkt 
aan de hand daarvan of er transformaties nodig lijken te zijn, en of
M-structures bomen moeten zijn of alleen feature bundles.
  \item Aangezien de projectvergaderingen nu ook weer groepsvergadering zijn
(het afscheid van de niet-Rosetta groepsleden heeft op 2 oktober 
plaatsgevonden), moeten de {\bf notulen} van de groepsvergadering voortaan weer 
via Frank Stoots en Fred Robert.
  \item De {\bf volgende bijeenkomsten} zijn op 6 en 20 november 
(groepsvergaderingen), en op dinsdag 10 oktober om 13.30 uur (bespreken van 
doc.\ 353 van Jan O.\ over semantische types). Jan O.\ deelt alvast een blad 
met wijzigen op dat laatste stuk uit.
  \item Wat betreft besproken en/of nieuw verschenen {\bf documenten}: de
documenten over de TAT-brieven hoeven niet te worden besproken.
Voor de bespreking van de linguistische documentatie is een plan in wording.
\end{enumerate}

\section{Jaarplan}
Het Jaarplan dat Jan L.\ heeft verspreid wordt besproken. Naast een aantal 
detail-opmerkingen zijn de volgende twee punten naar voren gekomen:\\
1) Jan L.\ zal duidelijker omschrijven wat de algemene taak van de groep is\\
2) In  sectie 4 worden een aantal zaken toegevoegd: een studie naar de 
mogelijkheden van een statistische aanpak, conversie naar Unix-Pascal, 
en meer aandacht voor een specifieke `applicatie-interface'.\\
Het eventueel schrijven van het boek hoort niet in dit programma thuis: er is 
nog niets over besloten, en het is ook nog niet bekend bij de directie dat daar 
plannen voor bestaan.

\section{Conferentieverslagen}
Franciska geeft een verslag van de studiedag {\bf Language Technology and Human 
Communication} in Tilburg op 29 september. Zij vermeldt o.a.\ dat IBM een groot 
project in Stuttgart schijnt te hebben dat tot doel heeft een logische 
representatie voor Natuurlijke Taal te vinden.\\
Margreet vertelt over de {\bf Eurospeech}-conferentie (European Conference on 
Speech Communication and Technology) in Parijs van 26-28 september. Ook 
spraakbewerkers hebben dringend behoeft aan grammatica's, vooral voor het 
automatisch genereren van acceptabele intonatie. 

\section{Rondvraag en sluiting}
{\bf  Andr\'{e} } vraagt of er al iets bekend is over het mogen volgen van de 
CSO-cursus door niet-Philips medewerkers.  Jan L.\ antwoordt dat een besluit 
hierover niet op korte termijn verwacht moet worden.\\
{\bf Joep} vraagt hoe het zit met onze vacature. Antwoord: die is er nog! Een 
ieder wordt gevraagd in zijn omgeving naar goede kandidaten te zoeken.\\
{\bf Franciska} merkt 
op dat haar document over het TAT-corpus een verkeerd nummer heeft gekregen: 
het moet zijn {\em 0408\/}, en niet {\em 0402\/}. \\
{\bf Jan O.\ } meldt dat aan zijn document {\em 
R0406\/} over de TAT-problemen nog een stuk ontbreekt; dit zal alsnog worden 
verspreid. \\
{\bf Margreet} vraagt of er al afspraken kunnen worden gemaakt voor de 
PROLOG-cursus. Er worden twee groepen samengesteld; groep I (Elly, Joep, Harm, 
Franciska en Andr\'{e}) zal op maandag 23 oktober om 13.30 uur beginnen, en 
groep II (Bart de Greef, Ren\'{e}, Jan O., Lisette, Margreet en Elena) begint op 
donderdag 19 oktober om 9.00 uur, beide met het eerste echt inhoudelijke 
hoofdstuk (= Hoofdstuk 2).
\\[3ex]
Hiermee wordt de vergadering gesloten.

\end{document}
