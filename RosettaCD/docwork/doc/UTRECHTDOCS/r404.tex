
\documentstyle{Rosetta}
\begin{document}
   \RosTopic{Rosetta3.doc.linguistics.english}
   \RosTitle{English M-rules:subgrammar ADJPPROPFormation}
   \RosAuthor{Franciska de Jong, Lisette Appelo}
   \RosDocNr{0404}
   \RosDate{\today}
   \RosStatus{concept}
   \RosSupersedes{-}
   \RosDistribution{Project}
   \RosClearance{Project}
   \RosKeywords{English, M-rules, AdjpPropFormation}
   \MakeRosTitle
%
%
\input{[dejong.mrules]mrudocdef}

\section{Introduction}
In this document the subgrammar ADJPPROPformation for English is described.
The parts on temporal adverbs and aktionsart calculation have been written by 
Lisette Appelo, the other parts have been written by Franciska de Jong.

This grammar is supposed to be isomorphic to the 
subgrammar VERBPPROPformation. It is relevant to {\bf all} 
adjectival constructions, 
whether they end up as ADJP-utterance, as NP-internal constituent, or as
clausal constituent.


\section{Subgrammar Specification}

\begin{description}
  \item[Head] SUBADJ
  \item[Export] ADJPPROP
  \item[Import] SUBADJ, EMPTYVAR, CNVAR, NPVAR, SENTENCEVAR, PREPP, PREPPVAR,
ADVP, ADVPVAR, QP, NP, ASP, SENTENCE

  \item[File] english:adjsubgrammars.mrule (mrules68)
\end{description}

\section{Control Expression}
The control expression can be defined as follows:
\begin{verbatim}

( RSTARTADJPPROP000/1 | RSTARTADJPPROP100/2 | RSTARTADJPPROP120/3 |
  RSTARTADJPPROP123/4 | RSTARTADJPPROP012/5 | RSTARTADJPPROP010/6)

. (TADJPATTERN0/7 | TADJPATTERN11/8 | TADJPATTERN12/9
  | TADJPATTERN14/10 | TADJPATTERN15/11 | TADJPATTERN16/12
  | TADJPATTERN18a/13 )
(*
   | TADJPATTERN17/18 |  TADJPATTERN18b/20 
   | TADJPATTERN13/14
   | TADJPATTERN15a/15 
   | TADJPATTERN21/21 | TADJPATTERN22/22 | TADJPATTERN23/23 | TADJPATTERN24/24
   | TADJPATTERNe21/81 | TADJPATTERNe22/82 | TADJPATTERN25/83 
   | TADJPATTERN26/84 )

. [RADJFOROBJMOD/25]
*)

. [RADJMOD1/26 (* | RADJTOINFMOD1/27 *) ]

. [RADJDEGREEMOD1/28 | RADJDEGREEMOD2a/29 | RADJDEGREEMOD2b/30 ] 

. { RADJRefVARinsertion/101 | RADJDurVARinsertion/102 |
    RADJRetroVARinsertion/103 }

. { RadjConjsentVar/135 | RadjSentadvVar/136 | RadjFinalConjsentVar/31 }

. { RadjLocpreppVar/133 | RadjLocadvVar/134 }

. RADJVOICEdefault/35 

.TADJAktStative1/104 

For testing purposes some of the rules are not activated in the present 
system. 

\section{Rules and Transformations}
\begin{mruleclass}{RC\_STARTADJPPROP}
\begin{classdescr}
\kind obligatory rule class
\classtask The formation of the syntactic level ADJPPROP

\nofilters

\nospeedrules

\noplannedrules

\norulesnotince

\begin{comments}
\end{comments}
\end{classdescr}

\begin{members}
\begin{member}
\rulename RSTARTADJPPROP000
\ruletask formation of an ADJPPROP-level, no semantic arguments
\file english:rc\_startadjpprop.mrule (mrules52)
\semantics The formation of a propositon.
\example  it is cold, it became rainy 
\remarks

\end{member}
\begin{member}
\rulename startadjpprop100
\ruletask Adjppropformation with introduction of one argument variable
\file english:rc\_startadjpprop.mrule (mrules52)
\semantics the formation of a propositon/predication
\example x1 blond, x1 ill, x1 probable
\remarks

\end{member}
\begin{member}
\rulename RSTARTADJPPROP120
\ruletask Formation of ADJPPROP, with introduction of two argument 
variables
\file english:rc\_startadjpprop.mrule (mrules52)
\semantics the formation of a propositon/predication
\example x1 fond x2, x1 worried about x2

\remarks\mbox{}
The difference between prepositional objects and so-
called `oorzakelijke voorwerpen' is dealt with in the adjpatternrules, 
under reference to the attribute adjpattern(ef)s. 
For the latter category ADJPPROPrec1.adjpattternefs = 
                                         [synNP, synHETOPENOMTESENT, 
                                          synTHATSENT, synHETTHATSENT]

\end{member}
\begin{member}
\rulename RSTARTADJPPROP123 
\ruletask Formation of ADJPPROP with introduction of three argument 
variables
\file english:rc\_startadjpprop.mrule (mrules52)
\semantics the formation of a propositon/predication
\example\mbox{}
\begin{enumerate}
  \item 
 x1 worth x2 x3 (to be worth something to someone)
  \item
 x1 obliged x2 x3 (to be obliged to someone to do something)
  \item
 x1 grateful x2 x3 (to be grateful to someone for something)
\end{enumerate}

\remarks

\end{member}
\begin{member}
\rulename RSTARTADJPPROP010

\ruletask formation of an ergative ADJPPROP structure with introduction 
of one argument variable. Meant for adjectives that allow raising to subject.
\file english:rc\_startadjpprop.mrule (mrules52)
\semantics the formation of a propositon/predication
\example x1 likely
\remarks

\end{member}
\begin{member}
\rulename RSTARTADJPPROP012

\ruletask formation of an ergative ADJPPROP structure with introduction 
of two argument variables. 
\file english:rc\_startadjpprop.mrule (mrules52)
\semantics the formation of a propositon/predication
\example x1 clear (to x2)

\remarks

\end{member}
\end{members}


\end{mruleclass}

\newpage
\begin{mruleclass}{TC\_adjpatterns}
\begin{classdescr}
\kind obligatory transformation class
\classtask spelling out the synpatterns. 
\nofilters

\nospeedrules

\begin{plannedrules}
\item a rule that spells out the value synPREPOPENGERUND ({\em x1 sick of x2})
\end{plannedrules}


\norulesnotince

\classremarks
NB. The difference between aanobjs and prepobjs is discussed 
in 
the section on RC\_degreemod.
\end{classdescr}

\begin{members}
\begin{member}
\rulename TADJPATTERN0
\ruletask Spelling out the synpattern value synnoadjpargs
\file english:tc\_adjpattern1.mrule (mrules53)
\semantics \nosemantics
\example Cf. document nr. 374 (Adjpatterns of Dutch)
\remarks

\end{member}
\begin{member}
\rulename TADJPATTERN11
\ruletask Spelling out the following synpattern values:
\begin{enumerate}
  \item synMEASUREPHRASE
  \item synNP
  \item synLOCEMPTY
  \item synLOCPREPP
  \item synPATHPREPP
\end{enumerate}

\file english:tc\_adjpattern1.mrule (mrules53)
\semantics \nosemantics
\example Cf. document nr. 374 (Adjpatterns of Dutch)
\remarks

\end{member}
\begin{member}
\rulename TADJPATTERN12a
\ruletask Spelling out the synpattern value synPREPNP
\file english:tc\_adjpattern1.mrule (mrules53)
\semantics \nosemantics
\example Cf. document nr. 374 (Adjpatterns of Dutch)
\remarks  A test for prepobj-status is discussed in 
the section on RC\_degreemod.
\end{member}
\begin{member}
\rulename TADJPATTERN12b
\ruletask Spelling out the synpattern value synPOSTADJPREPNP


\file english:tc\_adjpattern1.mrule (mrules53)
\semantics \nosemantics
\example Cf. document nr. 374 (Adjpatterns of Dutch)
\remarks

\end{member}
\begin{member}
\rulename TADJPATTERN13
\ruletask Spelling out the following synpattern values:
\begin{enumerate}
  \item synHETOPENOMTESENT
  \item synHETTHATSENT
\end{enumerate}
\file english:tc\_adjpattern1.mrule (mrules53)
\semantics \nosemantics
\example Cf. document nr. 374 (Adjpatterns of Dutch)
\remarks

\end{member}
\begin{member}
\rulename TADJPATTERN14
\ruletask To spell out the synpattern value synPREPEMPTY
\file english:tc\_adjpattern1.mrule (mrules53)
\semantics \nosemantics
\example Cf. document nr. 374 (Adjpatterns of Dutch)
\remarks

\end{member}
\begin{member}
\rulename TADJPATTERN15a
\ruletask To spell out the following synpattern values:
\begin{enumerate}
  \item  synPREPOPENOMTESENT
  \item  synPREPQSENT
  \item  synPREPTHATSENT
\end{enumerate}
\file english:tc\_adjpattern1.mrule (mrules53)
\semantics \nosemantics
\example Cf. document nr. 374 (Adjpatterns of Dutch)
\remarks

\end{member}
\begin{member}
\rulename TADJPATTERN15b
\ruletask To spell out the following synpattern values:
\begin{enumerate}
  \item synPAPREPOPENOMTESENT
  \item synPAPREPTHATSENT
  \item synPAPREPQSENT
\end{enumerate}
\file english:tc\_adjpattern1.mrule (mrules53)
\semantics \nosemantics
\example Cf. document nr. 374 (Adjpatterns of Dutch)
\remarks

\end{member}
\begin{member}
\rulename TADJPATTERN16
\ruletask To spell out the following synpattern values:
\begin{enumerate}
  \item synOPENTESENT
  \item synTHATSENT
  \item synQSENT
\end{enumerate}

\file english:tc\_adjpattern1.mrule (mrules53)
\semantics \nosemantics
\example Cf. document nr. 374 (Adjpatterns of Dutch)
\remarks\mbox{}
\begin{enumerate}
\item NOTA BENE: the meaning of the adjective may  vary with the specific 
synpattern, for example: {\em bang} in {\em ik ben bang te verliezen}
versus  {\em bang} in {\em ik ben er bang voor om te verliezen}.
\\

\end{enumerate}

\end{member}
\begin{member}
\rulename TADJPATTERN17
\ruletask To spell out the synpattern value synOPENOMTESENTPROOBJ
\file english:tc\_adjpattern1.mrule (mrules53)
\semantics \nosemantics
\example 
Dit boek is niet bedoeld/(MEANT) om in te schrijven\\
Dit mes is geschikt/(MEANT) om mee te snijden\\
De kaas is niet (MEANT) (??om) te snijden\\
\remarks\mbox{}
\begin{enumerate}
\item Cf. also  document nr. 374 (Adjpatterns of Dutch)
\item There is no complete path for these cases yet. 
The phenomenon - complementation by means of a sentence with a PRO-object- 
needs to be 
studied still in some more detail. 
\item 
At least the following 
two alternative treatments for the examples mentioned above are 
conceivable:
\begin{itemize}
  \item There is an additional empty argument (EMPTY,  or ALL, or whatever).
The pro-subject (BIGPRO) of the {\em om te}-sentence is deleted 
under identity with EMPTY/ALL. 
(This would require adjp123 as the value for .thetaadj (and given the implicit 
naming convention, a different name for the patternrule).) 
  \item
The pro-subject is deleted without identity conditions to be met.
\end{itemize}
Pending a principled choice the present elaboration is compatible 
with the second alternative.
\item It is not decided yet whether phrases such as 
{\em geschikt voor Jan om mee te nemen} are to be analysed by this 
transformation too, or rather by RADJVOOROBJMOD or RADJMOD.

\item Translation aspects: The dutch sentence 
{\em dit boek is niet om in te schrijven} should be 
translated into a passive construction of English: {\em this book is not meant 
to be
written in}. Sometimes a translation with a {\em for}-phrase is preferable:
{\em dit bier is niet om te drinken} translates into {\em this beer is not for 
drinking}. 

\end{enumerate}

\end{member}
\begin{member}
\rulename TADJPATTERN18a
\ruletask To spell out the synpattern synVOORNP
\file english:tc\_adjpattern1.mrule (mrules53)
\semantics \nosemantics
\example Cf. document nr. 374 (Adjpatterns of Dutch)
\remarks
\end{member}
\begin{member}
\rulename TADJPTTERN18b
\ruletask To spell out the synpattern synVOOREMPTY 
(in case of empty "belanghebbende voorwerpen")
\file english:tc\_adjpattern1.mrule (mrules53)
\semantics \nosemantics
\example Cf. document nr. 374 (Adjpatterns of Dutch)
\remarks

\end{member}
\begin{member}
\rulename TADJPATTERN21
\ruletask To spell out the following synpattern values:
\begin{enumerate}
  \item synAANNP\_DONP
  \item synAANNP\_OPENTESENT 
\end{enumerate}
\file english:tc\_adjpattern2.mrule (mrules34)
\semantics \nosemantics
\example Cf. document nr. 374 (Adjpatterns of Dutch)
\remarks

\end{member}
\begin{member}
\rulename TADJPATTERN22
\ruletask To spell out the following synpattern values:
\begin{enumerate}
  \item synIONP\_DONP
  \item synIOEMPTY\_DONP 
  \item synIONP\_OPENTESENT
\end{enumerate}
\file english:tc\_adjpattern2.mrule (mrules34)
\semantics \nosemantics
\example Cf. document nr. 374 (Adjpatterns of Dutch)

\end{member}
\begin{member}
\rulename TADJPATTERN23
\ruletask To spell out the  synpattern value synAANNP\_HETOPENTESENT
\file english:tc\_adjpattern2.mrule (mrules34)
\semantics \nosemantics
\example Cf. document nr. 374 (Adjpatterns of Dutch)
\remarks\mbox{}
\begin{enumerate}
\item In its present form this rule may be collapsed with TADJPATERN25.

\end{enumerate}

\end{member}
\begin{member}
\rulename TADJPATTERN24
\ruletask To spell out the following synpattern values:
\begin{enumerate}
  \item synIONP\_HETOPENOMTESENT 
  \item synIOEMPTY\_HETOPENOMTESENT  
\end{enumerate}
\file english:tc\_adjpattern2.mrule (mrules34)
\semantics \nosemantics
\example Cf. document nr. 374 (Adjpatterns of Dutch)
\remarks\mbox{}
\begin{enumerate}
\item In its present form this rule may be collapsed with TADJPATERN26

\end{enumerate}

\end{member}

\begin{member}
\rulename TADJPATTERNe21
\ruletask To spell out the following synpattern values:
\begin{enumerate}
  \item synAANNP\_DONP
  \item synAANNP\_QSENT
  \item synAANNP\_THATSENT
\end{enumerate}
\file english:rc\_startadjpprop.mrule (mrules52)
\semantics \nosemantics
\example Cf. document nr. 374 (Adjpatterns of Dutch)
\remarks

\end{member}
\begin{member}
\rulename TADJPATTERNe22
\ruletask To spell out the following synpattern values:
\begin{enumerate}
  \item synIONP\_DONP
  \item synIOEMPTY\_DONP
  \item synIONP\_QSENT
  \item synIONP\_THATSENT
  \item synIOEMPTY\_QSENT
  \item synIOEMPTY\_THATSENT
\end{enumerate}
\file english:rc\_startadjpprop.mrule (mrules52)
\semantics \nosemantics
\example Cf. document nr. 374 (Adjpatterns of Dutch)
\remarks
\end{member}
\begin{member}
\rulename TADJPATTERN25
\ruletask To spell out the following synpattern values:
\begin{enumerate}
  \item synAANNP\_HETTHATSENT
  \item synAANNP\_HETQSENT 
\end{enumerate}
\file english:tc\_adjpattern2.mrule (mrules34)
\semantics \nosemantics
\example Cf. document nr. 374 (Adjpatterns of Dutch)
\remarks\mbox{}
\begin{enumerate}
\item In its present form this rule may be collapsed with TADJPATERN23.

\end{enumerate}

\end{member}
\begin{member}
\rulename TADJPATTERN26
\ruletask To spell out the following synpattern values:
\begin{enumerate}
  \item synIONP\_HETTHATSENT
  \item  synIONP\_HETQSENT
  \item synIOEMPTY\_HETTHATSENT
  \item  synIOEMPTY\_HETQSENT
\end{enumerate}
\file english:tc\_adjpattern2.mrule (mrules34)
\semantics \nosemantics
\example Cf. document nr. 374 (Adjpatterns of Dutch)
\remarks\mbox{} 
\begin{enumerate}
\item In its present form this rule may be collapsed with TADJPATERN24.

\end{enumerate}


\end{member}

\end{members}

\end{mruleclass}
\newpage
\begin{mruleclass}{RC\_ADJVOOROBJMOD}
\begin{classdescr}
\kind optional rule class
\classtask Introduction of a {\em voor}-modifier to adjectives with the value 
{\em subjectiveadj} in .subcs. 
\nofilters

\nospeedrules

\noplannedrules

\norulesnotince
\classremarks
It is not clear yet whether cases like {\em deze tas is handig om mee te 
nemen} (with an infinitival modifier) 
should be dealt with by this rule class too, for it might be argued that
in this sentence an {\em ervoor}-PREPP is omitted or that is a paraphrase of 
{\em het is handig om deze tas mee te nemen}. In the former case it is to 
be treated by RC\_ADJMOD. In the latter case a transformation is needed to 
relate it to sentences with a dummy  {\em het} and a sentential 
extraposed subject.
\end{classdescr}

\begin{members}
\begin{member}
\rulename RADJVOOROBJMOD
\ruletask 
Introduction of a {\em voor}-modifier to adjectives with the value {
\em subjectiveadj} in .subcs. 
\file english:rcs\_adjmod.mrule (mrules31)
\semantics modification 
\example\mbox{}
\begin{enumerate}
  \item 
Dit boek is leuk {\em voor kinderen}
  \item
Rollers zijn handig {\em voor onervaren schilders}
  \item
Dat het niet regent is slecht {\em voor het gras}
\end{enumerate}
\remarks\mbox{}
\begin{enumerate}
\item The modifiers introduced by means of this rule are quite argument-like.
The reason that they are not treated as real arguments and hence not introduced
by RC\_startadjpprop/TC\_Adjpattern is that they are always optional and that 
they do not behave as prepobjs while {\em voor} cannot be omitted.
(NB.  A test for prepobj-status is discussed in 
the section on RC\_degreemod.)
\item
Probably it is possible to define severe dictionary constraints
on the set of adjpatterns for adjectives with .sucbs = subjectiveadj.
\end{enumerate}
\end{member}
\end{members}
\end{mruleclass}

\newpage
\begin{mruleclass}{RC\_ADJMOD}
\begin{classdescr}
\kind optional rule class
\classtask Introduction of a modifiers to adjectives other than degree-
modifiers and voorobj-modifiers.
\nofilters

\nospeedrules

\noplannedrules

\norulesnotince

\classremarks
RC\_ADJMOD in combination with those rules 
of RC\_ADJDEGREEMOD that introduce a complex
modifier with an infinitival complement may give rise to undesired ambiguities.
\end{classdescr}

\begin{members}

\begin{member}
\rulename Rmod1
\ruletask  Introduction of comparative modifiers.
\file english:rcs\_adjmod.mrule (mrules31)
\semantics modification 
\example\mbox{}
\begin{enumerate}
\item sterk als een leeuw
\item
gelig alsof .......
\end{enumerate}
\remarks\mbox{}
\begin{enumerate}
\item
The {\em als}-comparisons never occur as prenominal modifier, therefore 
they are related to the ADJP node by postmodrel.
\item problems:\\ It is not clear which adverbial modifiers that co-occur with
ADJPs should be considered sentential constituents and which should be taken as 
belonging to the ADJP proper.
\item 
Alternative: idiom-treatment. 

\end{enumerate}

\end{member}
\begin{member}
\rulename Romtemod1
\ruletask  Introduction of {\em om te}-modifiers
\file english:rcs\_adjmod.mrule (mrules31)
\semantics modification 

\example\mbox{}
\begin{enumerate}
  \item leuk om te zien
  \item ongezond  om te eten
\end{enumerate}
\remarks
The content of the dutch examples such as (i) and (ii)
that are dealt with
by RADJOMTEMOD1, 
can be  translated only 
into Spanish structures that do not involve modification, namely the structures that
correspond to the dutch examples (iii) and (iv) 
(dummy {\em het} and extraposed sentential 
subject argument).

(i) zij is leuk om te ontmoeten \\ 
(ii) dit is handig om mee te nemen\\ 
(iiiDU) het is leuk om haar te ontmoeten\\
(iiiSP) es divertido recontrar la\\
(ivDU) het is handig om dit mee te nemen\\
(ivSP) es comodo llevar se lo\\

This could be accounted for straightforwardly if the two  constructions
of dutch were related (that is, if parallel derivations would exist). 
Whether they can and should be related is still to be investigated. 
At least the following questions need to be answered: 
\begin{enumerate}
\item
Do the two constructions 
have identical meanings? (Note that {\em leuk om te zien is perhaps an 
idiomatic structure with a {\em leuk} that is not identical to the 
{\em leuk} the construction with the dummy {\em het}; in Spanish it 
translates into 
{\em attractiva}.) 
\item
Do the struture with non-dummy subject really involve modification?
Note that from {\em x1 is ADJ om ....} does not always impl {\em x1 is ADJ}.

\end{enumerate}
Mapping of the two constructions -in case 
the differences beteen (i/ii) and (iii/iv) appear to be non-semantic- 
would require (a set of) transformation(s) that takes care of 
at least the whole range of adaptations that are implemeted for stranding:
met, tot $\rightarrow$ mee, toe; PREP dit $\rightarrow$ hierPREP 
(e.g. dit is handig om mee te snijden $\rightarrow$ 
het is handig om hiermee te snijden).

\end{member}
\end{members}
\end{mruleclass}

\newpage
\begin{mruleclass}{RC\_ADJDEGREEMOD}
\begin{classdescr}
\kind optional rule class
\classtask Introduction of degree-
modifiers.

\classremarks
Degree modifiers differentiate between prepobjs and aanobjs. Degreemodifiers 
always 
follow prepositional 
objects, while with prepobjs, the ordering is arbitrary.
Compare: {\em Hij is zeer op ons gesteld} and {\em Hij is op ons zeer gesteld}, 
versus {\em De vraag is aan ons zeer duidelijk} versus {\em *De vraag is zeer 
aan ons duidelijk.}
This fact may be used as a test for prepobj-status, and consequently, as a tool
to determine the value of .adjpatterns correctly.

For example, the adjective {\em gewend} superficially  
behaves as an adjective with an 
optional aanobj as indirect object: {\em aan} can be deleted. However, the fact 
that the object can be preceded by a degreemodifier 
{\em Hij is zeer aan ons gewend}) suggests that it should 
be taken as a prepobj.  
So {\em gewend} is an example of an adjective that allows both
a prepositional object (synpattern: synPREPNP), and
a so-called "oorzakelijk voorwerp" (synpattern synNP).
(NB. There are no other adjectives found yet that would necessitate 
the distinction of a synpattern value synAANNP.)
\nofilters

\nospeedrules

\noplannedrules

\norulesnotince

\begin{comments}
\end{comments}
\end{classdescr}

\begin{members}
\begin{member}
\rulename RADJDegreemod1
\ruletask Insertion of a degree-modifier without infinitival complement.
\file english:rcs\_adjmod.mrule (mrules31)
\semantics modification
\example te groot, zeer mooi, hoe lang, minder leuk, iets gelig
\remarks\mbox{}
\begin{enumerate}
\item problems: {\em te} en {\em te zeer} are probably to be treated
as synonyms. That is, {\em te zeer} is probably not
a modified {\em te}.

\end{enumerate}

\end{member}
\begin{member}
\rulename RADJDegreemod2a
\ruletask Insertion of a degree-modifier of catgeory QP 
that contains a sentential om-te-modifier with relation omtemodrel
\file english:rcs\_adjmod.mrule (mrules31)
\semantics modification 
\example
verliefd genoeg om niet serieus te nemen
\remarks\mbox{}
\begin{enumerate}
\item 
 The superdeixis adaptation for the sentential complements is done
on the QP-level.\\
\end{enumerate}

\end{member}
\begin{member}
\rulename RADJDegreemod2b
\ruletask Insertion of a degree-modifier of catgeory ADVP
that contains a sententitial om-te-modifier with relation omtemodrel
\file english:rcs\_adjmod.mrule (mrules31)
\semantics modification
\example\mbox{}
 \begin{enumerate}
  \item 
te zwaar om mee te nemen
  \item
te (zeer) verliefd om serieus te nemen
\end{enumerate}
\remarks\mbox{}
\begin{enumerate}
\item {\em te} en {\em te zeer} shoul probably be treated
as synonyms. That is, {\em te zeer} is probably not
a modified {\em te}.

\item 
 The superdeixis adaptation for the complement sentences is done
on the QP-level.\\
\end{enumerate}

\end{member}
\end{members}

\end{mruleclass}

\newpage
\begin{mruleclass}{RC\_ADJTempVARinsertion}
\begin{classdescr}
\kind iterative rule class
\classtask to introduce variables for temporal adverbial phrases
\nofilters

\nospeedrules

\noplannedrules

\norulesnotince

\begin{comments}
\end{comments}
\end{classdescr}

\begin{members}
\begin{member}
\rulename RADJrefvarinsertion
\ruletask Introduction of variables for referential time adverbials that 
are not retrospective. These time adverbials are supposed to be of category 
ADVP, PREPP or SENTENCE. They will be substituted in the substitution rules.
\file english:tempadj1.mrule (mrules79)
\semantics \nosemantics
\example
  \remarks\mbox{}
\begin{enumerate}
\item  Only one rule for non retrospective referential time adverbials 
will be applied. Complex time adverbials such as e.g. {\em morgen om 3 uur} can 
form one constituent, but also appear disjunctive as e.g. in {\em morgen komt 
hij om drie uur}. 
\end{enumerate}

\end{member}
\begin{member}
\rulename RADJdurvarinsertion
\ruletask Introduction of variables for durative time adverbials.
These time adverbials are supposed to be of category 
ADVP, PREPP or SENTENCE. They will be substituted in the substitution rules.
\file english:tempadj1.mrule (mrules79)
\semantics \nosemantics
\example
  \remarks\mbox{}

\end{member}
\begin{member}
\rulename RADJretrovarinsertion
\ruletask Introduction of variables for referential time adverbials that 
are retrospective. These time adverbials are supposed to be of category 
ADVP, PREPP or SENTENCE. They will be substituted in the substitution rules.
\file english:tempadj1.mrule (mrules79)
\semantics \nosemantics
\example
  \remarks\mbox{}

\end{member}
\end{members}

\end{mruleclass}
\newpage
\begin{mruleclass}{RC\_ADJcausconjadvvar}
\begin{classdescr}
\kind recursive rule class
\classtask Introduction of causvars and conjvars
\nofilters

\nospeedrules

\noplannedrules

\norulesnotince
\classremarks 
It is not clear which causitive and locative PREPPs belong to the ADJP
and which belong to the sentence. 
Compare: 
\begin{enumerate}
  \item 
   van de warmte ben ik slaperig  vs. van de warmte slaperige mensen 
  \item
  in de trein ben ik bang vs. * in de trein bange mensen\
\end{enumerate}
\end{classdescr}
\begin{members}
\begin{member}
\rulename RADJcauspreppvar 
\ruletask Introduction of VARs for causative prepositional modifiers 
\file english:RC\_ADJcauslocmodvar.mrule (mrules35)
\semantics modification
\example\mbox{}
\begin{enumerate}
  \item 
door de herrie bang 
  \item
van de warmte slaperig
  \item 
door de regen verlopen (geraakt)
\end{enumerate}
\remarks\mbox{}

\end{member}
\begin{member}
\rulename RADJcausadvpvar
\ruletask Introduction of VARs for locative adverbial modifiers 
\file english:RC\_ADJcauslocmodvar.mrule (mrules35)
\semantics modification
\example hierdoor bang 
\remarks\mbox{}

\end{member}
\begin{member}
\rulename Radjppconjsvar
\ruletask Introduction of a variable for a sentential adverbial.
\file english:rc\_advvar.mrule (mrules51)
\semantics 
\example
\remarks\mbox{}

\end{member}
\begin{member}
\rulename Radjppconjsvar2
\ruletask Introduction of a variable for a non-temporal 
sentential adverbial in 
leftdislocrel.
\file english:rc\_advvar.mrule (mrules51)
\semantics 
\example
\remarks\mbox{}

\end{member}
\begin{member}
\rulename Radjppconjsvar3
\ruletask Introduction of a variable for a 
non-temporal sentential adverbial in postsentadvrel
\file english:rc\_advvar.mrule (mrules51)
\semantics
\example
\remarks\mbox{}

\end{member}
\end{members}


\end{mruleclass}
\newpage
\begin{mruleclass}{RC\_ADJlocadvvar}
\begin{classdescr}
\kind recursive rule class
\classtask Introduction of variables for locative modifiers
\nofilters

\nospeedrules

\noplannedrules

\norulesnotince
\classremarks 
It is not clear which cuasitive and locative PREPPs belong to the ADJP
and which belong to the sentence. 
Compare: 
\begin{enumerate}
  \item 
   van de warmte ben ik slaperig  vs. van de warmte slaperige mensen 
  \item
  in de trein ben ik bang vs. * in de trein bange mensen\
\end{enumerate}
\end{classdescr}

\begin{members}
\begin{member}
\rulename RADJlocpreppvar 
\ruletask Introduction of VARs for locative prepositional modifiers 
\file english:RC\_ADJcauslocmodvar.mrule (mrules35)
\semantics modification
\example in de trein bang 
\remarks\mbox{}

\end{member}
\begin{member}
\rulename RADJlocadvpvar
\ruletask Introduction of VARs for locative adverbial modifiers 
\file english:RC\_ADJcauslocmodvar.mrule (mrules35)
\semantics modification
\example overal bang 
\remarks\mbox{}

\end{member}
\end{members}

\end{mruleclass}
\newpage
\begin{mruleclass}{RC\_ADJVOICE}
\begin{classdescr}
\kind obligatory rule class.
\classtask This rule class exists for reason of isomorphy only.

\nofilters

\nospeedrules

\noplannedrules

\norulesnotince

\begin{comments}
\end{comments}
\end{classdescr}

\begin{members}

\begin{member}
\rulename RADJVOICEdefault
\ruletask This rule exists for reason of isomorphy only.
\file english:rc\_adjvoice-tc\_qphop.mrule (mrules37)
\semantics 
\example all adjectival structures
\remarks
\end{member}
\end{members}

\end{mruleclass}
\newpage
\begin{mruleclass}{TC\_ADJAktionsartcalculation }
\begin{classdescr}
\kind obligatory transformation class
\classtask Give attribute {\em aktionsarts} a value

\nofilters

\nospeedrules

\noplannedrules

\norulesnotince

\begin{comments}
\end{comments}
\end{classdescr}

\begin{members}

\begin{member}
\rulename tADJaktstative1
\ruletask Set the value for the attribute {\em aktionsarts} to stative, 
the aktionsart for all adjectives.
\file english:tempadj1.mrule (mrules79)
\semantics \nosemantics
\example all adjectives
\remarks\mbox{}

\end{member}
\end{members}

\end{mruleclass}

\begin{mruleclass}{TC\_ADJQPHopping}
\begin{classdescr}
\kind Presently this is an optional transformation class.
\classtask To account for the movement of QPs into postadjectival position.

\nofilters

\nospeedrules

\noplannedrules

\norulesnotince

\classremarks
The presently optional transformation class should be revised in order to 
guarantee application in case a constituent occurs in postadjrel, as in 
{\em zeer dol op vis}.

\end{classdescr}

\begin{members}

\begin{member}
\rulename TADJQPhopping1
\ruletask To move bare QPs with attribute .hop = true into postadjectival 
position.
\file english:rc\_adjvoice-tc\_qphop.mrule (mrules37)
\semantics \nosemantics
\example mooi genoeg, lang zat
\remarks\mbox{}


\end{member}
\begin{member}
\rulename TADJQPhopping2
\ruletask To move Qs out of complex 
QPs with attribute .hop = true into postadjectival position.
\file english:rc\_adjvoice-tc\_qphop.mrule (mrules37)
\semantics \nosemantics
\example bijna genoeg lang $\rightarrow$ bijna lang genoeg
\remarks\mbox{}

\end{member}
\end{members}

\end{mruleclass}
\end{document}
