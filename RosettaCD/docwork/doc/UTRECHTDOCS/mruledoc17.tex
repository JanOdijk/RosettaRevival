\documentstyle{Rosetta}
\begin{document}
   \RosTopic{Rosetta3.doc.Mrules.English}
   \RosTitle{Rosetta3 English M-rules: IdentPropFormation}
   \RosAuthor{Margreet Sanders}
   \RosDocNr{396}
   \RosDate{\today}
   \RosStatus{concept}
   \RosSupersedes{-}
   \RosDistribution{Project}
   \RosClearance{Project}
   \RosKeywords{English, documentation, Mrules, Identificational Propositions}
   \MakeRosTitle
%
%

\section{IdentPropFormation}
The current document describes the contents of 
the subgrammar that forms Identificational Propositions, {\bf 
IdentPropFormation}.
This subgrammar is an alternative to two other subgrammars that form 
propositions around nominal constituents, viz.\ {\bf ExistPropFormation} and {
\bf NPPROPformation}. These two grammars are described in docs.\ 395, {\em 
Rosetta3 English Mrules: ExistPropFormation\/}, and 392, {\em Rosetta3 English 
Mrules: NPPROPformation\/}, respectively. 

The export of all three grammars is 
an NPPROP, to be used as head of the XPPROPtoCLAUSE subgrammar, where the 
copula {\em be\/} is introduced and the NPPROP-node is pruned again. 
The difference between the grammars is to be 
found in their heads and import: in {\bf IdentPropFormation}, the head is an 
NPVAR (or a SENTENCEVAR in special occasions), which is combined with a 
specific subject NP (a member of the set {\em it, this, that, they, these, 
those\/}) in the startrules (or, again in special occasions, with an NPVAR). 
In {\bf ExistPropFormation}, the head of the 
subgrammar is an indefinite NPVAR, which is turned into an NPPROP all 
on its own. A dummy subject {\em There\/} will be inserted later, 
in the SubjOK rules of the sentence grammar. 
In {\bf NPPROPformation\/}, the head of the subgrammar is a 
full NP, which combines with any subject variable.

The three subgrammars forming nominal propositions are not isomorphic with any 
other subgrammars, and contain only the minimum amount of rules necessary 
to make the 
NPPROP an acceptable input to the XPPROPtoCLAUSE subgrammar. Next to the 
startrules, there only are a few rule classes to provide the various variables 
needed. In doc.\ 150, {\em Subgrammars of English\/}, in which the general 
lay-out of the Rosetta3 system as devised in the definition phase of the 
project was presented, it was assumed that the copula {\em be\/} would be 
the head of the IdentPropFormation subgrammar; this was necessary 
because the proposition was to be input to the VERBPPROPtoCLAUSE subgrammar
(no general XPPROPtoCLAUSE subgrammar was planned at that time). Also, it was 
thought that the subgrammar would contain PreVPmodrules. However, these rules 
have not even been written for the VERBPPROPformation subgrammar, and they are 
not included here either.

As any subgrammar, {\bf IdentPropFormation} consists of 
a number of rule classes. A rule class in its turn
consists of a number of rules. The relative ordering of the rules in the
(sub)grammar is indicated by a {\em control expression}. A summary of this
control expression (i.e.\ a listing of the ordering of the rule classes, 
without explicit mentioning of the rules themselves) is also included here, 
and the initial (= head) and export categories are given. 

In the section on the rules, only the rule names are given, 
but not the exact rule formulation. What is attempted 
is to provide a detailed overview of the workings of the subgrammar, and 
how the different rule classes achieve this. For every rule, an 
example is given. If it is uncertain whether the example is correct (either 
because it may not be an example of the phenomenon in question, or because it 
may not be correct English), it is preceded by a question mark. Note that all 
explanation of rules and transformations is given from a generative viewpoint
only, unless explicitly stated otherwise. Often, the information given in this 
document is based strongly on the comment already present in the documentation 
of the rules themselves. Discrepancies between what is stated here and what is 
said in the rule itself are usually caused by the fact that the rule file has 
not  been updated, although insights have changed. The semantics of the rules 
has been left unspecified in the current documentation, since it is not at all 
clear.

Finally note that the rules described in this document have NOT been tested 
properly. English analysis is not possible yet (there is no Surface Parser), and 
English generation has only been tested in as far as the construction was the 
translation of a Dutch sentence to be tested.

\newpage
\section{Subgrammar Specification}
The subgrammar definition can be found in the file which also contains all the 
rules of this subgrammar, {\bf IdentPropFormation.mrule}, which is 
{\em mrules98.mrule\/}.

\begin{verbatim}
%SUBGRAMMAR IdentPropFormation


   ( RC_StartIdentificational )
.  { RC_IdentTempAdvVar }
.  { RC_IdentSentAdvVar }
.  { RC_IdentLocAdvVar }

\end{verbatim}

\begin{description}
  \item[Head]  NPVAR \ \ \ \ BASIC EXPRESSION\\
               SENTENCEVAR \ \ \ \ BASIC EXPRESSION
  \item[Export] NPPROP
  \item[Import] NP, NPVAR, PREPPVAR, ADVPVAR, SENTENCEVAR
\end{description}

\newpage
\section{Rules and Transformations}

\subsection{RC\_StartIdentificational}
\begin{description}
\item[Kind] Obligatory Rule Class
\item[Task] To build an NPPROP around an NPVAR and its subject argument, 
putting the NPVAR in {\em idrel\/} (there is no predicate). 
Also, the Aktionsarts of the NPPROP are 
set to {\em [stative]\/}; no separate transformation classs is needed to 
determine this standard value. The rest of the NPPROP attributes remains at 
default value. If the subject is a full NP, its {\bf
superdeixis} value is reset to {\em omegadeixis\/}, and the attribute {\bf 
generic} is reset to {\em omegageneric\/}.

\vspace{1 cm}
\begin{description}
\item[Name] RDemproIdentSg
\item[Task] To make an NPPROP for an NPVAR taking as subject one of the NPs {
\em it, this, that\/}. The NPVAR may not be the word {\em what\/} (see 
RDemproWhIdentSg). In case the NPVAR has a PERSPRO as its head, it may be
plural; otherwise, it must be singular.
\item[File] english:IdentPropFormation.mrule (mrules98.mrule)
\item[Semantics]
\item[Example] x1 + this $\rightarrow$ this x1 (This is my brother, This is us, 
This is who)
\item[Remarks] Wh-shift will take place in the sentence grammar.
\end{description}

\vspace{1 cm}
\begin{description}
\item[Name] RDemproIdentPl
\item[Task] To make an NPPROP for an NPVAR taking as subject one of the NPs {
\em it, this, that\/}. The NPVAR may not be the word {\em what\/} (see 
RDemproWhIdentPl). The NPVAR must be plural (and not a PERSPRO; see the 
previous rule); thus, the singular subject is replaced by a plural form, {
\em they, these, those\/}. The replacement of the singular subject by a plural 
one is needed because Dutch only has singular basic expressions: {\em het, dit, 
dat\/}. These cannot be called plural, because they never behave as such with 
other words than the verb {\em zijn\/}:  $^*${\em Wat bewijzen dat?\/}. Mapping 
of the English plural forms to the Dutch singular ones might cause 
ambiguities.
\item[File] english:IdentPropFormation.mrule (mrules98.mrule)
\item[Semantics]
\item[Example] x1 + this $\rightarrow$ these x1 (These are my brothers, These 
are who(pl) )
\item[Remarks] Wh-shift will take place in the sentence grammar. For comment on 
the ambiguity of {\em They\/}, see below under RPersproIdent.
\end{description}

\vspace{1 cm}
\begin{description}
\item[Name] RDemproWhIdentSg
\item[Task] To make an NPPROP for an NPVAR heading the word {\em what\/} and 
taking as subject one of the NPs {\em it, this, that\/}. 
\item[File] english:IdentPropFormation.mrule (mrules98.mrule)
\item[Semantics]
\item[Example] x1 + this $\rightarrow$ this x1 (This is what)
\item[Remarks] Wh-shift will take place in the sentence grammar.\\
This rule might have been collapsed with the ordinary singular rule. However, 
because the plural WhIdent rule cannot be collapsed (see below), it was decided 
to keep all WhIdent rules separate, even in IL.
\end{description}

\vspace{1 cm}
\begin{description}
\item[Name] RDemproWhIdentPl
\item[Task] To make an NPPROP for an NPVAR heading the word {\em what\/} and 
taking as subject one of the NPs {\em they, these, those\/}. However, since the 
basic expressions provided by translation are the singular forms {\em it, this, 
that\/} (see above, in RDemproIdentPl), the plural forms must be explicitly 
created here.
\item[File] english:IdentPropFormation.mrule (mrules98.mrule)
\item[Semantics]
\item[Example] x1 + that $\rightarrow$ those x1 (Those are what)
\item[Remarks] Wh-shift will take place in the sentence grammar.\\
This rule cannot be collapsed with the ordinary plural rule, because both the 
NPVAR heading {\em what\/} and the subject NP are singular, and still it has to 
be remembered somehow that a plural has to be formed. This is done now by 
having a specific plural rule, and mapping it only to other WhIdentPlural rules 
in other languages.
\end{description}

\vspace{1 cm}
\begin{description}
\item[Name] RPersproIdent
\item[Task] To make an NPPROP for a definite NPVAR taking as subject 
an NPVAR heading a PERSPRO. 
\item[File] english:IdentPropFormation.mrule (mrules98.mrule)
\item[Semantics]
\item[Example] [x1]$_{def}$ + [x2]$_{perspro}$ $\rightarrow$ x2 x1 (He is my 
brother, They are the new champions)
\item[Remarks] Sentences with {\em They\/} are ambiguous: \\
{\em They are the new champions\/} $\rightarrow$ \\
1) {\em Zij zijn de nieuwe kampioenen\/} (by the present rule)\\
2) {\em Het zijn de nieuwe kampioenen\/} (by RDemproIdentPl).\\
This still seems to be needed to be able to deal with cases like {\em They are 
my
shoes, not yours\/}, translating to {\em Het zijn mijn schoenen, niet de jouwe 
}, and not to {\em Zij zijn mijn schoenen...\/}. Perhaps this ambiguity may be 
solved by using the attribute {\em animate\/}: {\em They\/} + animate NP = 
PERSPRO, {\em They\/} + inanimate NP = DEMPRO.
\end{description}

\vspace{1 cm}
\begin{description}
\item[Name] RSentIdent
\item[Task] To make an NPPROP for a subordinate SENTENCEVAR taking as subject 
a definite NPVAR.
\item[File] english:IdentPropFormation.mrule (mrules98.mrule)
\item[Semantics]
\item[Example] [x1]$_{S}$ + [x2]$_{defNP}$ $\rightarrow$ x2 x1 (The problem is 
that he has taken the map with him)
\item[Remarks] 
\end{description}

\end{description}

\newpage
\subsection{RC\_IdentTempAdvVar}
\begin{description}
\item[Kind] Iterative Rule Class
\item[Task] To introduce a variable for a time adverbial. The variable may be 
for a sentence, a prepp or an advp. The rule class is iterative, but only one 
rule has been written (no retrospective or durational temporals are allowed as 
yet), and this one rule may be applied only once. Perhaps other rules are 
needed too: {\em This was my hideaway during the war; ? That has always been 
our main motive}.

\vspace{1 cm}
\begin{description}
\item[Name] RIdentRefVarInsert
\item[Task] To introduce a variable for a referential time adverbial that is 
not retrospective.
\item[File] english:IdentPropFormation.mrule (mrules98.mrule)
\item[Semantics]
\item[Example] This x1 + refVAR $\rightarrow$ This x1 refVAR (This is my 
brother)
\item[Remarks]
\end{description}

\end{description}

\newpage
\subsection{RC\_IdentSentAdvvar}
\begin{description}
\item[Kind] Iterative Rule Class
\item[Task] To introduce variables for adverbial subordinate sentences in 
different positions, and sentence or causal adverbials in initial position. The 
conjunction may also be a preposition. No rules have been written yet for 
abstract conjunctions.

The rules are in an iterative class, but have been written in such a way that 
only one order of application is possible. This to prevent unnecessary 
ambiguities in analysis. Also, the IL strategy in mapping the different 
conjsent rules is to preserve the surface order used in the source language as 
much as possible, because it may be of importance for pronominal reference.

\vspace{1 cm}
\begin{description}
\item[Name] RIdentConjsentVar
\item[Task] To introduce a variable for an adverbial subordinate sentence (or 
sentential PREPP) in initial (leftdislocrel) position.
\item[File] english:IdentPropFormation.mrule (mrules98.mrule)
\item[Semantics]
\item[Example] \mbox{}\\
x1 x2 + advSENTENCEVAR $\rightarrow$ advSENTENCEVAR x1 x2\\
(Although we do not live here yet, we are your new neighbours)\\
x1 x2 + advPREPPVAR $\rightarrow$ advPREPPVAR x1 x2 candidate\\
(Without wanting to make you nervous, you are our best candidate)
\item[Remarks] No comma is introduced at the end of the adverbial sentence, 
although it is probably obligatory. This will have to be added.
\end{description}

\vspace{1 cm}
\begin{description}
\item[Name] RIdentFinalConjsentVar
\item[Task] To introduce a variable for an adverbial subordinate sentence (or 
sentential PREPP) in final (postsentadvrel) position.
\item[File] english:IdentPropFormation.mrule (mrules98.mrule)
\item[Semantics]
\item[Example] \mbox{}\\
x1 x2 + advSENTENCEVAR $\rightarrow$ x1 x2 advSENTENCEVAR \\
(We are your new neighbours, although we do not live here yet)\\
x1 x2 + advPREPPVAR $\rightarrow$ x1 x2 advPREPPVAR \\
(You are our best candidate, without considering all the details of your 
application)
\item[Remarks] 
\end{description}

\vspace{1 cm}
\begin{description}
\item[Name] RIdentSentadvVar
\item[Task] To introduce a variable for a causal or sentence adverbial or a 
causal prepp in 
initial position. No rules have been written to account for any other position 
of the adverbial, or to relate its position to that of an adverbial or temporal 
sentence also present in the clause.
\item[File] english:IdentPropFormation.mrule (mrules98.mrule)
\item[Semantics]
\item[Example] \mbox{}\\
x1 x2 + sentADVVAR $\rightarrow$ sentADVVAR x1 x2\\
(Probably, he is (will be?) our new manager)\\
x1 x2 + causPREPPVAR $\rightarrow$ causPREPPVAR x1 x2\\
(For that reason, we are the champions)
\item[Remarks] No comma is introduced at the end of the adverbial, 
although it is probably obligatory. This will have to be added.
\end{description}

\end{description}

\newpage
\subsection{RC\_IdentLocAdvVar}
\begin{description}
\item[Kind] Iterative Rule Class
\item[Task] To introduce variables for non-argument locatives (ADVP or PREPP) 
at a fixed position (locadvrel) following the NP. No rules have been 
written to account for any other position 
of the locative, or to relate its position to that of an adverbial or temporal 
sentence also present in the clause.

\vspace{1 cm}
\begin{description}
\item[Name] RIdentLocAdvVar
\item[Task] To introduce a variable for a non-argument locative ADVP 
\item[File] english:IdentPropFormation.mrule (mrules98.mrule)
\item[Semantics]
\item[Example] That x1 + locADVVAR $\rightarrow$ That x1 locADVVAR
(That is our problem here)
\item[Remarks]
\end{description}

\vspace{1 cm}
\begin{description}
\item[Name] RIdentLocPreppVar
\item[Task] To introduce a variable for a non-argument locative PREPP
\item[File] english:IdentPropFormation.mrule (mrules98.mrule)
\item[Semantics]
\item[Example] That x1 + locADVVAR $\rightarrow$ That x1 locADVVAR
(That is the problem in this country)
\item[Remarks]
\end{description}

\end{description}

\end{document}

