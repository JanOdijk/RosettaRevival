
\documentstyle{Rosetta}
\begin{document}
   \RosTopic{General}
   \RosTitle{Notulen Rosetta vergadering 31-8-1987}
   \RosAuthor{Margreet Sanders}
   \RosDocNr{0219}
   \RosDate{September 28, 1987}
   \RosStatus{approved}
   \RosSupersedes{-}
   \RosDistribution{Project}
   \RosClearance{Project}
   \RosKeywords{minutes}
   \MakeRosTitle
%
%
\begin{description}
\item[Aanwezig:] Lisette Appelo,
                 Chris Hazenberg, Franciska de Jong, 
                 Jan Landsbergen, Ren\'{e} Leermakers, 
                 Jeroen Medema,  
                 Jan Odijk, Joep Rous, Margreet Sanders (not),
                 Andr\'{e} Schenk, 
                 Harm Smit
\item[Afwezig:] Carel Fellinger, Elly van Munster
\item[Agenda:]\mbox{}
  \begin{enumerate}
  \item Opening en notulen
  \item Mededelingen
  \item Nieuwe stagiair(e)s
  \item Verslag hoogtepunten cursus computerlinguistiek
  \item Discussie Rosetta - Eurotra
  \item Rosetta summerschool?
  \item Besproken en/of nieuw verschenen documenten
  \item De stand van zaken
  \item Rondvraag en sluiting
  \item Ren\'{e}: het GETA systeem
  \end{enumerate}
\end{description}

\section{Opening en notulen}
De notulen van de vorige vergadering worden ongewijzigd 
aangenomen.

\section{Mededelingen}
\begin{enumerate}
  \item Loek wil een {\em trade mark\/} aanvragen voor de naam Rosetta, zodat 
deze niet meer door anderen gebruikt kan worden voor een vertaalsysteem.
  \item Het {\em workstation-project\/} is goedgekeurd door de Raad van 
Bestuur. Een opstartgroep moet nu bepalen hoe het project vorm krijgt; 
misschien moet Jan L. hier ook aan meewerken.
  \item Data voor de {\em volgende vergaderingen\/}:
     \begin{itemize}
       \item 28 sept., met geplande voordracht van Jan L. over de Japanse 
             vertaalsystemen
       \item 12 okt., met voordracht van Joep over het Kimmo-systeem 
             (morfologie)
       \item 26 okt., met voordracht van ???
      \end{itemize}
\end{enumerate}

\section{Stagiair(e)s}
Lisette heeft een inwerkprogramma gemaakt voor Joost Zwarts en Joleen 
Schipper. Zij zullen voorlopig op k.\ 236 komen. Eventueel gaan ze na het 
vertrek van Chris naar de kamers van Joep en Ren\'{e}, aangezien dat de band 
met de groep 
duidelijk verbetert. Harm zal op korte termijn een woordenboektaak voor hen 
formuleren.

\section{Verslag hoogtepunten cursus computerlinguistiek}
Margreet was van 26 tot 28 aug.\ in Leiden op de zomercursus 
computerlinguistiek. De belangstelling was groot ($\pm$ 70 deelnemers) en vanuit 
verschillende achtergronddisciplines. Ook de lezingen (6 per dag) gingen over 
zeer 
uiteenlopende onderwerpen. Er werd niet echt constructief gediscussieerd. 

Op de eerste dag werd aan de hand van 
concrete voorbeelden verteld wat voor problemen in de computerlinguistiek spelen 
(samenwerking linguisten - informatici, organisatie van de software (Joep 
Rous), gebruik van formalismes in linguistiek en software). De tweede dag was 
interessanter, met linguistische en informatica- benadering van automatische 
ontleding (soorten parsing, SYGMART, categoriale ont\-leding etc.). De derde dag 
kwamen zeer verschillende toepassingen aan bod: Spicos, tekst-naar-spraak 
systemen, auteursomgevingen, Van Dale \`{e}n automatisch vertalen (Anneke Neyt).
Van Dale heeft binnen afzienbare tijd een CD te koop met een viertalig 
vertaalwoordenboek (N - E/D/F), waarschijnlijk inclusief een spellingschecker 
die over bepaalde woordgrenzen heen kan kijken. De prijs zal rond de 500 gulden 
liggen. Ook de gewone woordenboeken komen op CD uit.

\section{Discussie Rosetta - Eurotra}
Lisette heeft op 30 okt.\ een informatieve bijeenkomst van Rosetta en Eurotra 
georganiseerd. Aan de hand van vooraf verstuurde stukken worden vragen 
geformuleerd waarover gediscussieerd kan worden. Onderwerp van dit gesprek is 
het formalisme dat de beide systemen gebruiken. Voor Rosetta zullen Jan L.,
Lisette, Andr\'{e}, Ren\'{e} en Jan O.\ deelnemen.

\section{Summerschool Rosetta?}
Loek heeft de suggestie gedaan een internationale zomerschool MT te organiseren, 
waarbij Rosetta uitgebreid gepresenteerd wordt. De bijbehorende papers kunnen 
dan meteen in boekvorm gepubliceerd worden. Een dergelijk evenement zou op zijn 
vroegst in 1989 zin hebben. In november a.s.\ moeten we hierop nog eens 
terugkomen, met wat uitgewerktere idee\"{e}n omtrent sprekers, onderwerpen, 
beoogd publiek en evt.\ interferentie met GLOW.

\section{Besproken en/of nieuw verschenen documenten}
\begin{description}
\item [Besproken:]\mbox{}
  \begin{itemize}
  \item Margreet Sanders: Local Dutch M-rules. Documentation (211). Dit stuk is door 
de linguisten besproken en in grote lijnen goedgekeurd. Wel moet er een nieuwe 
concept-versie verschijnen, waarin de besluiten die op basis van de bespreking
genomen zijn op een rijtje gezet worden.
  \end{itemize}
\item [Verschenen:]\mbox{}
  \begin{itemize}
  \item Harm Smit, Jeroen Medema: Description Van Dale dictionary N - N (174). 
Dit document is ter kennisgeving aangenomen en wordt niet verder besproken.
  \item Jan Landsbergen, Joep Rous: The Transfer Modules Of Rosetta3 (213).
Dit stuk wordt besproken op {\em do.\ 8 oktober\/}.
  \item Joep Rous: A dictionary for testing purposes (214). In dit document is 
de `syntax' van woordenboek-entries vastgelegd. Bespreking volgt op de 
linguistenvergadering van {\em di.\ 8 september, 9.30 uur\/}. Eventueel zal 
Joep dan ook zijn `Leidse praatje' over software beheersaspecten herhalen.
  \end{itemize}
\end{description}

\section{Stand van Zaken}
\begin{description}
\item [Software:]\mbox{}
  \begin{itemize}
  \item Er is een nieuwe versie van RBS, die vooral bij het compileren van 
korte programma's sneller zal werken. Andere projecten op het Nat.\ Lab.\ 
zullen RBS zelf niet gaan gebruiken, maar het misschien wel als basis houden 
voor een nieuw (door CST = Centrum voor Software Technieken) te ontwikkelen 
beheerssysteem, dat minder groots opgezet is.
  \item De compiler generator voor M-regels is bijna af. Carel zal hem nog 
linken met de M-parser. De S-parser is af. Rond 1 oktober moet alle software 
tot en met de M-parser in werkende vorm beschikbaar zijn.
  \item Joep en Ren\'{e} zullen november gebruiken om aan Boltzmann machines te 
werken.
  \item Jeroen gaat in de loop van het komende half jaar Rosetta verlaten, en 
zal in de nieuwe groep Nijman een beheerssysteem gaan ontwerpen.
  \item Er moet nagedacht worden wat voor hulpmiddelen het testen van de M-
regels zouden vergemakkelijken (mogelijkheden om de controle-expressie snel te 
wijzigen, grotere bomen etc.). Wensen melden bij Joep/Ren\'{e}.
  \end{itemize}
\item [Lingware:]\mbox{}
  \begin{itemize}
  \item De woordvertaler N-E en E-N is in grote lijnen af. Er moeten nog een 
paar surface regels geschreven worden, om ook reflexieven en partikels 
te kunnen toestaan in analyse. 
  \item Jan O.\ is bezig zijn kennis over de clause-grammatica over te dragen. 
Jan L.\ benadrukt dat er beter eerst in de diepte gewerkt kan worden (een pad 
door een subgrammatica heen) en pas daarna in de breedte (allerlei ingewikkelde 
gevallen). De ADJP en NP subgrammatica's worden als volgende besproken.
  \item Aan het testwoordenboek wordt hard gewerkt. Als dat af is zal het 
woordenboek voor de woordvertaler gereed worden gemaakt.
  \item  Het werk aan derivatie wordt weer uitgesteld.
  \end{itemize}
\end{description}
Samenvattend hebben we weer meer vertraging opgelopen dan verwacht, maar er is 
nog geen reden voor paniek.

\section{Rondvraag en sluiting}
{\bf Lisette} vraagt hoe het met het NEHEM-verslag staat. Jan L.\ zegt dit 
waarschijnlijk niet af te krijgen voor zijn vertrek naar Japan.\\
{\bf Margreet} wil nadere informatie over de gang van zaken bij de open dag.
Over de opvolging van Loek, die half november zal vertrekken, is nog niets 
bekend.

\section{Ren\'{e}: het GETA systeem}
Na een korte pauze vervolgt Ren\'{e} de bijeenkomst met een voordracht over 
GETA. Ondanks alle onduidelijkheden blijkt wel hoezeer de Fransen hechten aan 
het bewaren van de oorspronkelijke datastructuur. Jan L.\ vraagt zich peinzend 
af of dat nu echt allemaal in de derivatie-bomen zit bij Rosetta \ldots

\end{document}
