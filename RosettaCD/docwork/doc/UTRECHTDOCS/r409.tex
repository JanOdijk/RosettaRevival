
\documentstyle{Rosetta}
\begin{document}
   \RosTopic{Rosetta3.doc.linguistics.dutch}
   \RosTitle{Dutch M-rules:subgrammar CNFormation}
   \RosAuthor{Franciska de Jong}
   \RosDocNr{409}
   \RosDate{\today}
   \RosStatus{concept}
   \RosSupersedes{-}
   \RosDistribution{Project}
   \RosClearance{Project}
   \RosKeywords{Dutch, M-rules, CNFormation}
   \MakeRosTitle
%
%
\input{[dejong.mrules]mrudocdef}

\section{Introduction}
In this document the subgrammar CNformation for Dutch is described.
In general the task of this subgrammar 
is to handle the phenomena involved in the proces of NPformation that pertain 
syntactically and/or semantically to the level CN.
It is not relevant to {\bf all} 
NPs, but only to those 
that are headed by CN. (This holds for NPs containing either a full NOUN or a 
EN (empty head).)
It does not apply to NPs without a CN head (NPs 
headed by singular proper names or a pronoun.)
The subgrammar is not supposed to be isomorphic to any other subgrammar 
of Dutch. 

\section{Subgrammar Specification}

\begin{description}
  \item[Head] SUBNOUN, EN
  \item[Export] CN
  \item[Import] 
OPENADJPPROP,  DETP, OPENPREPPPROP, 
NP, PROPERNOUN, SENTENCE, NPVAR, SENTENCEVAR

  \item[File] dutch:npsubgrammars.mrule (mrules67)
\end{description}

\section{Control Expression}
The control expression can be defined as follows:
\begin{verbatim}

    [( RSUBNOUNTONOUN1/1 | RSUBNOUNTONOUN2/2)]

   .( RCNFORMATION1/3 | RCNFORMATION2/4 | RCNFORMATION3/5 
      | RCNFORMATION4/6  )

   . (RCNPresentSuperdeixis/100   | RCNPastSuperdeixis/101)

   . [RNOUNargmod1/110 | RNOUNargmod2/111]

   . [RCNmodbareNP/23]

   . [RCNspecProperName/24]

   .{ RCNMODADJP/7       | RCNMODNUM/8    
      | RCNMODPOSS1/9   | RCNMODPOSS2/10 | RCNMODPOSS3/11 
      | RCNMODPP/12 | RCNMODRELSENT1/13 
      | RCNMODANTEREL1/22
    }            

\end{verbatim}

\section{Rules and Transformations}
\begin{mruleclass}{RC\_subnounTOnoun}
\begin{classdescr}
\kind optional rule class
\classtask
\begin{enumerate}
  \item 
Introduction of the syntactic level NOUN
  \item
Determination of the value for .number
\end{enumerate}
\classremarks
In Rosetta, the morphological aspect of number (inflection),
as well as the translation
of number for CNs with a non-empty head
are dealt with by this RC.
That is, the two rules that assign values for .number 
are each mapped onto  a different IL-rule, 
namely one 
for singular count nouns (LSUBNOUNTONOUN1), the other both for singular
mass nouns and plural nouns  (LSUBNOUNTONOUN2).

Note that for CNs with an empty head (EN)
there is no morphology and hence no number assignment 
on the the CN head. (The record of category EN contains only the attribute 
.key). The CN itself {\bf is} 
specified for number. The translation
of EN number is dealt with by the mapping of the rules of RC\_CNformation. 

For further comment on the treatment of number cf. doc:R412 (to appear)
and the remarks to RC\_CNformation.
\nofilters
\nospeedrules

\noplannedrules

\norulesnotince

\rulelist

\end{classdescr}

\begin{members}
\begin{member}
\rulename RSUBNOUNtoNOUN1
\ruletask \begin{enumerate}
  \item 
introducing the syntactic level NOUN
  \item
assigning singular number to 
count NOUNs
\end{enumerate}
\file dutch:cnformation.mrule (mrules49)
\semantics LSUBNOUNTONOUN1 (countsing)

\example tuin, tafel, boek, vent, uur
\remarks\mbox{}

\end{member}
\begin{member}
\rulename RSUBNOUNtoNOUN2
\ruletask \mbox{}\\
\begin{enumerate} 
\item introducing the syntactic level NOUN
\item
assigning plural number to count nouns  including unitnouns
that occur as plural with singular morphology (.subc = unitnoun).
\item assigning singular number to mass nouns 
\end{enumerate}
\file dutch:cnformation.mrule (mrules49)
\semantics LSUBNOUNTONOUN2 (massplur)
\example tuinen, uren, uur,  hersenen, speelgoed, brood
\remarks\mbox{}
In order to account for the plural nature of NPs with a singular unit head noun 
such as {\em drie uur} and to guarantee the mapping of these NPs on plural 
counterparts such as the English {\em three hours},
the number of these nouns must be 
translated via the il-rule Lsubnountonoun2 (massplural). 
This is realized by 
assuming that in analysis singular 
unit nouns in plural CNs are assigned the 
value plural; in generation the reverse holds.
Cf. remarks to RCNformation1.
\end{member}
\end{members}
\end{mruleclass}
\begin{mruleclass}{RC\_CNformation}
\begin{classdescr}
\kind obligatory rule class
\classtask
 \begin{enumerate}
  \item 
Introduction of  the syntactic level CN.
  \item
Account for the number of CNs headed by EN.
\end{enumerate}

\classremarks\mbox{}
\begin{enumerate}
  \item 

As a consequence of the present approach, the translation of 
number is distributed over two rule classes, pertaining to different
syntactic levels: RC\_subnountonoun and RC\_CNformation.
Morphology (inflection) has played a decisive role
in this decision. As
 there is no translational relation between full nouns and EN, 
it does not
affect the translational performance.

Alternatively we might choose for an approach that translates uniformly on the
level of CNformation. This would require that RCNformation is split up and 
mapped onto ILrules in a way comparable to the present rules RsubnounTOnoun1 
and RsubnounTOnoun2. The latter two should then be replaced by one rule
(or transformation) that generatively spoken makes all 
forms available for 
a certain noun. I.e. 
a temporal ambiguity would be introduced: one path for
each possible values for .number.

This alternative would at least have the following advantages: \mbox{}
\begin{description}
  \item 
There is a uniform level for the translation of number
  \item
The rule CNformation would not contain the deviant assignment of values for
.number in case of a singular unit noun with plural interpretation.
\end{description}
Future research is needed in order to decide whether this alternative is 
to be preferred. Especially ellipsis on the basis of contextual information
should be studied in more detail.
 
 \item Two ENs are distinguished: one for count interpretations and one for 
mass interpretations. This distinction is a.o. motivated by the different
uses of the NP {\em veel}.
It occurs in {\em Daar ligt veel} (suiker) as well as in 
{\em Daar liggen er veel} (suikerkorrels).
Only in the latter context can it be replaced by {\em vele}. 
  \item
Among the three rules introducing a CN node to dominate an empty head (EN) 
the same semantic distinction has been made as for the rules of RC\_
subnounTOnoun.
They are mapped on two ILrules: 
CNformation 3 is mapped onto LCNformationcountsing, the other two
onto 
LCNformationmassplur.

\end{enumerate}

\nofilters

\nospeedrules

\noplannedrules

\norulesnotince

\rulelist

\end{classdescr}

\begin{members}
\begin{member}
\rulename CNformation1
\ruletask Introducing the syntactic level CN.
\file dutch:cnformation.mrule (mrules49)
\semantics \nosemantics
\example
\remarks\mbox{}
\begin{enumerate}
\item Generatively this rule applies twice to singular 
nouns with unitnoun in .subcs, such as {\em uur}. (NB. the value plurunitnoun
is used for measure units that only have a regular plural, such as {\em week}.)
The introduced CN is assigned 
either [singular] as value for .numbers, or [plural].
In combination with what is said below the
latter value is needed to guarantee that only plural deteminers are 
combined with a unitnoun that has passed the plural rule RSUBNOUNtoNOUN2.
\item As a consequence of the foregoing, this 
rule is deviant as it involves number-assignment 
(rather than number-
copying) to a record that
in general is supposed to reflect lexical values. The strategy followed
here is more efficient than to
consider singular unitnouns ambiguous between a singular and a plural form.
There is a more elegant alternative to keep track of the number of the NP as a
whole, namely by defining a special path through the control expression that 
guarantees application of RSUBNOUNtoNOUN2 in case of a singular unitnoun  with 
a plural determiner. Cf. also the alternative sketched under the `remarks'
to this RC.
\item 
"Plural" singular unitnouns 
require a plural count indefinite determiner with an "exact" cardinality claim. 
Cf. 
*{\em de (vele) uur} vs. {\em de drie uur}; *{\em (vele) uur} vs. 
{\em drie uur};
Up till now there is no way to distinguish between exact numerals and vague
numerals 
such as {\em vele}. Hence the illformed examples cannot be excluded yet.
\item 
NP's with a plural unitnoun head 
with singular morphology
not accompanied by a  plural 
count exact indefinite determiner or modifier must be blocked. Cf.
*{\em (de) uur} vs. {\em (de) uren}; *{\em veel uur} vs. {\em veel uren}.
This is accounted for by filter Fpost unitNP.
 
\end{enumerate}
\end{member}
\begin{member}
\rulename CNformation2
\ruletask
Introducing a CN node (singular, mass) for a mass EN (empty noun).
\file dutch:cnformation.mrule (mrules49)
\semantics LCNformationmassplur
\example veel EN, deze EN
\remarks\mbox{}
\end{member}
\begin{member}
\rulename CNformation3
\ruletask Introducing a CN node (singular, count) for a count EN (empty noun).
\file dutch:cnformation.mrule (mrules49)
\semantics LCNformationcountsing
\example een EN, deze EN 
\end{member}
\begin{member}
\rulename CNformation4
\ruletask Introducing a CN node (plural, count) for a count EN (empty noun).
\file dutch:cnformation.mrule (mrules49)
\semantics LCNformationmassplur
\example vele EN, beide EN, drie gele EN, sommige EN

\end{member}
\end{members}
\end{mruleclass}
\begin{mruleclass}{RC\_CNSuperdeixis}
\begin{classdescr}
\kind obligatory rule class
\classtask Deal with superdeixis
\classremarks

\nofilters

\nospeedrules

\noplannedrules

\norulesnotince

\rulelist

\end{classdescr}

\begin{members}
   
\begin{member}
\rulename RCNPresentsuperdeixis
\ruletask In generation: set value for superdeixis at presentdeixis.\\
               In analysis: set value for superdeixis at omegadeixis.    
\file dutch:cnformation.mrule (mrules49)
\semantics \nosemantics
\example all CNs
\remarks\mbox{}

\end{member}
\begin{member}
\rulename RCNPastsuperdeixis
\ruletask In generation: set value for superdeixis at pastdeixis.\\
               In analysis: set value for superdeixis at omegadeixis.    
\file dutch:cnformation.mrule (mrules49)
\semantics \nosemantics
\example all CNs 
\remarks\mbox{}

\end{member}
\end{members}
\end{mruleclass}
\begin{mruleclass}{RC\_NOUNargmod}
\begin{classdescr}
\kind optional rule class
\classtask To introduce variables for complements to nouns.
\classremarks\mbox{}\\
\begin{enumerate}
  \item 
The application of this rule is constained by  conditions on the attributes 
{\em 
thetanp} and {\em nounpatterns}. Optional complements are not introduced as 
EMPTYs. Instead it is assumed that the application of 
noun-modification is optional.
  \item
In its present state the RC is provisional: it accounts for 
 varinsertion (only NPVAR), and   caseassignment.
\end{enumerate}
\nofilters

\nospeedrules

\begin{plannedrules}
\item
Rules for nouns with double complements. E.g. {\em verzoek aan de regering om 
hulp}.
\end{plannedrules}
\norulesnotince

\rulelist

\end{classdescr}

\begin{members}
   
\begin{member}
\rulename RNOUNargmod1
\ruletask To introduce a variable for a prepositional argument.
\file dutch:npsubgrammars.mrule (mrules67)
\semantics modification
\example (het) antwoord op NP
\remarks\mbox{}

\end{member}
   
\begin{member}
\rulename RNOUNargmod2
\ruletask To introduce a variable for a sentential complement.
\file dutch:npsubgrammars.mrule (mrules67)
\semantics modification
\example  (het) feit dat .. ; (de) vraag of ..
\remarks\mbox{}

\end{member}
\end{members}
\end{mruleclass}

\begin{mruleclass}{RC\_CNmodbareNP}
\begin{classdescr}
\kind optional rule class
\classtask To specify the content of the head NOUN of the CN.
\classremarks

\nofilters

\nospeedrules

\noplannedrules

\norulesnotince

\rulelist

\end{classdescr}

\begin{members}


\begin{member}
\rulename RCNmodbareNP
\ruletask To specify the content of the head NOUN of the CN, by
introducing an NP in postnominal position.
\file dutch:cnformation.mrule (mrules49)
\semantics \nosemantics
\example\mbox{}
\begin{enumerate}
  \item 
 fles (CN) + melk (NP) $\rightarrow$ fles melk
  \item
 emmer (CN) + bramen (NP) $\rightarrow$  emmer bramen
\end{enumerate}
\remarks\mbox{}
Example such as {\em (drie) kilo kaas} are presently analysed by RCNmodbareNP 
as well. 
This should be prohibited (with a restriction on the valueset for 
.(act)subcs) as soon as there are rules for the formation of a 
DETP out of a measure NP. 
\end{member}
\end{members}
\end{mruleclass}
\begin{mruleclass}{RC\_CNspecPN}
\begin{classdescr}
\kind optional rule class
\classtask To add a specifying propername to a nominal head.
\classremarks

\nofilters

\nospeedrules

\noplannedrules

\norulesnotince

\rulelist

\end{classdescr}

\begin{members}


\begin{member}
\rulename RCNspecProperName
\ruletask To add a specifying propername to a nominal head.
\file dutch:cnformation.mrule (mrules49)
\semantics \nosemantics
\example \mbox{}
\begin{enumerate}
  \item 
project (CN) + Rosetta (proper name) $\rightarrow$ 
(het) project Rosetta
  \item
zus (CN) + Margreet (proper name) $\rightarrow$ 
(mijn) zus Margreet
\end{enumerate}
\remarks\mbox{}
This rule should not be confused with the rule for NP-appositions.
\end{member}
\end{members}

\end{mruleclass}

\begin{mruleclass}{RC\_CNmodification}
\begin{classdescr}
\kind recursive rule class
\classtask Introduction of (restrictive) modifiers to the head of the CN.
\classremarks\mbox{}
This RC contains a rather heterogeneous set of modification rules. 
More detailed documentation on the the subset of modpossrules 
(which deals with possessive modification)
can be found in doc. R413 
(to appear).

\nofilters

\nospeedrules

\noplannedrules

\norulesnotince

\rulelist

\end{classdescr}

\begin{members}


\begin{member}
\rulename RCNmodADJP
\ruletask Modification of a CN by an ADJP. (This involves the 
substitution  of the CNVAR of an ADJPPROP.)
\file dutch:cnformation.mrule (mrules49)
\semantics Substitution of the argument variable of the
ADJPPROP by a CN.
\example x1 mooi + boek $\rightarrow$ mooi boek
\remarks\mbox{} 
\end{member}
\begin{member}
\rulename CNmodNUM
\ruletask making a {\em definite} CN out of a CN ('omegadef') and a NUM.
\file dutch:cnformation.mrule (mrules49)
\semantics \nosemantics
\example .. drie (Spaanse) dames, .. vijf blonde (domme) heren.
\remarks\mbox{}
\begin{enumerate}
  \item 
The value for the CN attribute .definite is made {\em definite} by 
this rule. Definite CNs are excluded from the formation of determinerless NPs.
Hence it excluded that on NP-level the numeral is not preceded by a determiner.
NPs with an initial numeral are derived via NPformation1 and are not
considered modifiers. 
  \item
The formation of  {\em *de veel boeken} should be excluded.
Therefore {\em veel} is assumed to be an indefinite BDET rather than a
NUM.
\end{enumerate}

\end{member}
\begin{member}
\rulename CNmodposs1
\ruletask
Introduction of postnominal possessive 'van'-modifiers.
\file dutch:cnformation.mrule (mrules49)
\semantics modification
\example
\begin{enumerate}
  \item 
 boek + ik $\rightarrow$ boek van mij
  \item
 bedrijf  + Jans vader $\rightarrow$ bedrijf van Jans vader
\end{enumerate}
\remarks\mbox{}

In Dutch,
the NPs to which this rule applies are semantically equivalent
to those derived by a combination of CNformation3 and NPformation4; e.g. 
{\em het boek van mijn vader}, and its equivalent 
{\em mijn vaders boek}.

\end{member}
\begin{member}
\rulename CNmodposs2
\ruletask
Introduction of possessive 'van'-modifiers preceded by an empty CN head (EN).
\file dutch:cnformation.mrule (mrules49)
\semantics modification
\example
\remarks\mbox{}
\example  EN + ik $\rightarrow$ EN van mij (via NPformation bijv. die van mij)

\end{member}
\begin{member}
\rulename CNmodposs3
\ruletask
Introduction of possessive modifiers that end up in prenominal position
via NPformation.
\file dutch:cnformation.mrule (mrules49)
\semantics modification
\remarks\mbox{}
\example\
\begin{enumerate}
  \item 
boek + Jan $\rightarrow$ boek Jan
(via NPformation $\rightarrow$ Jans boek)
\item 
bedrijf  + mijn moeder $\rightarrow$ bedrijf mijn moeder
(via NPformation $\rightarrow$ mijn moeders bedrijf)
  \item
boek + ieder  $\rightarrow$ boek ieder
(via NPformation $\rightarrow$ ieders bedrijf)
\item 
boek + wie $\rightarrow$ boek wie
(via NPformation $\rightarrow$ wiens bedrijf)
\end{enumerate}

\end{member}
\begin{member}
\rulename CNmodPP
\ruletask Modification of a CN by a PREPP. (This involves the
substitution  of the CNVAR of a PREPPPROP.)
\file dutch:cnformation.mrule (mrules49)
\semantics Substitution of the argument variable of the PREPPPROP by a CN.
\example (lange) man {\em met de hoed}, (die) EN {\em naast  de trein}
\remarks\mbox{}

\end{member}

\begin{member}

\rulename RCNmodRELSENT1
\ruletask To modify a CN by an relative sentence. This involves the 
substitution of a CNVAR in sentences by CN.
\file dutch:cnmodification.mrule (mrules109)
\semantics restrictive modification
\example \mbox{}
\begin{enumerate}
\item  de stad die ik haat/ het werk dat ik haat (was modrelsent1)
\item  de stad waarvan ik droom (was modrelsent2)
\item  de stad waar ik van droom (was modrelsent3)
\item  de stad waar ik woon (was modrelsent4)
\item  de man van wie ik droom (was modrelsent5)
\end{enumerate}
\remarks\mbox{}
\begin{enumerate}
  \item 
A comma is introduced following the modifying sentence
  \item 
There is a special rule for non-restrictive sentential modification:
RnonCNmodrelsent1. This rule generates two comma's: one preceding and one 
following the sentetial modifier.
  \item 

In addition to the last example
{\em de man waarvan ik droom} is accepted in analysis, but only the 
variant with 
{\em van wie}
will be  generated;
  \item
Subrules have taken over the role of the original modrelsent1-5. 
Cf. the examples.
\end{enumerate}
\end{member}
\begin{member}
\rulename CNmodanterel1
\ruletask To modify a CN by an anterelative sentence containing
a present participle.
\file dutch:cnformation.mrule (mrules49)
\semantics modification
\example
\begin{enumerate}
  \item 
kind + [x1 lopend ] $\rightarrow$ [lopend kind]
\item 
man  + [x1 z'n kinderen hatend] $\rightarrow$ [z'n kinderen hatend man]
\item
ossen + [x1 door Piet gekocht ] $\rightarrow$ [door Piet gekocht ossen]
\end{enumerate}
\remarks\mbox{}
The inflection on the participle is accounted for in subgrammar NPformation
(TNPAssignE).
\end{member}
\end{members}
\end{mruleclass}
\end{document}
