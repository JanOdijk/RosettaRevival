
\documentstyle{Rosetta}
\begin{document}
   \RosTopic{General}
   \RosTitle{The filling of Dutch BNOUN entries}
   \RosAuthor{Franciska de Jong}
   \RosDocNr{R284}
   \RosDate{October 11, 1988}
   \RosStatus{informal}
   \RosSupersedes{-}
   \RosDistribution{Project}
   \RosClearance{Project}
   \RosKeywords{BNOUN, constraints, attribute value assignment}
   \MakeRosTitle
%
%
\section{Introduction}

This document is primarily meant as an instructive note for the
"lexicographers" working on the entries for BNOUNs.
It may also be seen as provisional 
documentation on the choice of attributes and attribute
values for Dutch BNOUNs. 
In the text, `BNOUN' and `noun' are both meant to refer to nominal
lexical elements.

\section{The BNOUN-record in dutch:lsdomaint.dom}
In the dutch domain, the BNOUN record is specified as follows:
\begin{verbatim}

<
req:               polarityEFFSETtype:[pospol, negpol, omegapol]   
env:               polarityEFFSETtype:[pospol, negpol, omegapol]
dimforms:          dimformSETtype:[jeDim]{*morph*}
pluralforms:       pluralformSETtype:[enPlural, sPlural] {*morph* } 
genders:           genderSETtype:[omegagender]  {*morph* ++} 
class:             timeadvclasstype:omegaTimeAdvClass
deixis:            deixistype:omegadeixis
aspect:            aspecttype:omegaAspect
retro:             retrotype:false
sexes:             sexSETtype:[]                        
subcs:             nounsubcSETtype:[othernoun]
temporal:          temporaltype:false
possgeni:          possgenitype:false
animate:           animatetype:Omegaanimate
human:             humantype:Omegahuman                 
posscomas:         posscomaSETtype:[count]
thetanp:           thetanptype:omegathetanp
nounpatterns:      synpatternSETtype:[]
prepkey:           keytype:0
personal:          personaltype:true
KEY              
>
\end{verbatim}

\newpage
For each attribute an attribute type is specified, plus a default value.
Note that some attributes have a set as their (default) value while
others have a single 
value. 
Except for the  exceptional attributes .env and .req,
this difference is encoded 
in the names of the attributes: attributes with a value set as its value have 
a name that ends in an additional -s.

In the following  list the attribute value sets for each 
noun attribute is given:
\begin{verbatim}

[req]           valueset: [pospol, negpol, omegapol]   

[env]           valueset: [pospol, negpol, omegapol]   

[dimforms]      valueset: [jeDim, etjeDim, irregDim, dimletterword, noDim]

[pluralforms]   valueset:  [enPlural, sPlural, aTOaaPlural, 
                aTOeePlural, eTOeePlural, eiTOeePlural,
                iTOeePlural, oTOooPlural, erenPlural, 
                ienPlural, jenPlural {*ph*}, denPlural, nenPlural,
                ieAccentPlural, luiPlural, liedenPlural, 
                LatPlural, enIrregPlural, sIrregPlural,
                LatIrregPlural, NoPlural, OnlyPlural,
                enOSLPlural {*ph*}, enATOePlural{*ph*}, 
                hydTOhedenPlural {*ph*},
                enITOePlural{*ph*} ]

[genders]       valueset: [mascgender, femgender, neutgender, omegagender]

[class]         valueset: [duration, reference, frequential, omegaTimeAdvClass]
 
[deixis:]       valueset: [omegadeixis, presentdeixis, pastdeixis]

[aspect]        valueset: [habitual, imperfective, perfective, omegaAspect]

[retro]         valueset: [true, false]

[sexes]         valueset: [masculine, feminine]

[subcs]         valueset: [vocativenoun, professionnoun, relationnoun,
                           unitnoun, plurunitnoun, abstractnoun, othernoun]

[temporal]      valueset: [true, false]

[possgeni]      valueset: [true, false]

[animate]       valueset: [yesanimate, noanimate, omegaAnimate]

[human]         valueset: [YesHuman, NoHuman, OmegaHuman]

[posscomas]     valueset: [count, mass]

[thetanp]       valueset: [omegathetanp, thetanp0, thetanp1, thetanp2] 

[nounpatterns]  synpatternSETtype:[] (too long to enumerate here)

[prepkey]       keytype:0 (idem)

[personal]      valueset: [true, false]

[KEY]

\end{verbatim}

\section{BNOUN records as they appear on the screen of a lexicographer}

The corresponding part of the 
lexical entries as they appear on the screen consists of the same list 
of attributes, plus a value. In case the correct attribute value 
is already determined automatically this is marked. For nouns this is the
case with some morphological attributes. 
The other values that appear on the screen are identical for all nouns, but not
always is this value the default value. For example, the "screen value" for 
animate is the most frequent value `noaminate', 
whereas the default value is `omegaanimate'. 
This is to avoid unnecessary alterations. 
Here is an example entry as it appears on the screen:
\begin{verbatim}

baas                                  <False,True>   {04728}
   :$s_aN_00_baas
   :$m_aN_0001_baas                 s1 "chef, leider"
   ,$m_aN_0002_baas                 s2 "eigenaar v.e. zaak"
   ,$m_aN_0003_baas                 s3 "mbt. een huisdier"
   ,$m_aN_0004_baas                 s4 "man, jongen"
   ,$m_aN_0005_baas                 s5 "iem., zeer bedreven in iets"
   ,$m_aN_0006_baas                 s6 "kanjer"
   ,$m_aN_0007_baas                 s7 "heer des huizes"
   :BNOUN (
{req:}          [pospol, negpol, omegapol],
{env:}          [pospol, negpol, omegapol],
{dimforms:}     { * } [jeDim],
{pluralforms:}  { * } [enPlural],
{genders:}      { * } [mascgender],
{class:}        omegaTimeAdvClass,
{deixis:}       omegaDeixis,
{aspect:}       omegaAspect,
{retro:}        false,
{sexes:}        [],
{subcs:}        [othernoun],
{temporal:}     false,
{possgeni:}     false,
{animate:}      NoAnimate,
{human:}        NoHuman,
{posscomas:}    [count],
{omegathetanp:} omegaThetaNP,
{nounpatterns:} [],
{prepkey:}      0,
{personal:}     true
);

\end{verbatim}

In filling this entry, the provisional values (or value sets) 
must be replaced by the correct value.
For each attribute the possible attribute values are specified in the
declared attribute type. (This is always a set).
In filling an entry the correct value or valueset must be chosen 
from the set corresponding to the the attribute type.
Some `omegavalues' are not relevant at the lexical level. This is indicated
in the comment on the individual attributes in the next section.\\

At the top of an entry,
information is given on the intended reading of the entry.
(Except for the entry string, here `baas', the information on the first line 
can be ignored.) Some of 
the attibutes are related to each other in such a way that their
values should be compatible. For example in filling
.human and .animate it should be excluded to assign to .human the value 
`yeshuman', while to .animate 
the value `noanimate' is assigned. 
In order to guarantee consistency, 
a number of constaints of this kind will be checked 
automatically.
Uptil now the following constraints for BNOUNs have been proposed.

\begin{verbatim}
:IMPLIES ((human = yeshuman),(animate = yesanimate)
         ) 
           "yeshuman implies yesanimate"
:IMPLIES (true,(animate <> omegaanimate)
         )           
           "there are no omegaanimate bnouns"
:IMPLIES (true,(human   <> omegahuman)
         )              
           "there are no omegahuman bnouns"
:IMPLIES (true,(genders   <> [omegagender])
         )              
           "there are no bnouns with .genders = [omegagender]"
:IMPLIES ( (sexes <> []), (animate=yesanimate)
         )
          "sexes specified require animate=yesanimate"
:IMPLIES ( (class <> omegatimeadvclass),
           (temporal =true)
         )
         "nondefault specification of class requires temporal =true"
:IMPLIES ( (deixis <> omegadeixis),
           (temporal =true)
         )
         "nondefault specification  of deixis requires temporal =true"
:IMPLIES ( (aspect <> omegaaspect),
           (temporal =true)
         )
         "nondefault specification  of aspect requires temporal =true"
:IMPLIES ( (retro<> false),
           (temporal =true)
         )
         "nondefault specification  of retro requires temporal =true"
:IMPLIES ( (possgeni = true),
           (animate= yesanimate)
         )
         "possgeni=true implies animate = yesanimate"
:IMPLIES ( (animate = noanimate),
          (possgeni = false)
         )
         "nonanimate nouns do not allow genitive forms"
:IMPLIES ( (subcs * [vocativenoun, professionnoun, relationnoun] 
           <> []),  (human = yeshuman)
         )
         "vocativenouns, professionnouns, relationnouns must be 
          yeshuman"
:IMPLIES ( (human = nohuman),
           (subcs * [vocativenoun, professionnoun, relationnoun] = [])
         )
         "nohuman nouns cannot be vocativenoun, professionnoun, or 
          relationnoun"

\end{verbatim}

\section{Criteria for value assignment}
In the remainder of this instruction the criteria for 
determining the correct value for the noun attributes are given.

\newpage

\begin{description}
\item
[req]\mbox{}

valueset: [pospol, negpol, omegapol]\\

.req is a counterexample to the convention that attributes that have a set
as value have a name that ends with an -s. \\

Filling postponed.

\newpage
\item 
[env]\mbox{}

valueset: [pospol, negpol, omegapol]\\

.env is a counterexample to the convention that attributes that have a set
as value have a name that ends with an -s. \\

Filling postponed.

\newpage
\item 
[dimforms]\mbox{}

valueset: [jeDim, etjeDim, irregDim, dimletterword, noDim]\\

This attribute is already filled with correct values automatically.
\newpage
\item 
[pluralforms]\mbox{}

\begin{verbatim}
valueset:  
            [enPlural, sPlural, aTOaaPlural, 
            aTOeePlural, eTOeePlural, eiTOeePlural,
            iTOeePlural, oTOooPlural, erenPlural, 
            ienPlural, jenPlural {*ph*}, denPlural, nenPlural,
            ieAccentPlural, luiPlural, liedenPlural, 
            LatPlural, enIrregPlural, sIrregPlural,
            LatIrregPlural, NoPlural, OnlyPlural,
            enOSLPlural {*ph*}, enATOePlural{*ph*}, 
            hydTOhedenPlural {*ph*},
            enITOePlural{*ph*} 
          ]
\end{verbatim}

This attribute is already filled with correct values automatically.

\newpage
\item 
[genders]\mbox{}

valueset: [mascgender, femgender, neutgender, omegagender]\\

This attribute is already filled with correct values automatically.\\

The value omegagender should not be assigned to BNOUNs.\\

The correctness of the selected value can be checked by 
 checking
which definite article can be combined with the noun, {\em de} or {\em het}.
In the latter case the value `neutgender' must have been assigned.
In the former case `mascgender' is correct, except for a very limited set of
inherently female nouns, e.g. abstract nouns on -ing such as {\em regering}.
These female nouns must have been asigned the value `femgender'.

Note that non-neuter 
nouns that refer to animate objects may sometimes be used to
cover both the female and the male entities. In that case
the value for .gender is [mascgender]. 

The value for the attribute .gender may overlap with the value for .sexes
(see below), but not necessarily, as for example in the case of {\em meisje}.
In case of conflicting assignments the attributes .sexes is determines
which anaphoric pronoun is to be spelled out in case of control structures or 
idioms. 

\newpage
\item 
[class]\mbox{}

valueset: [duration, reference, frequential, omegaTimeAdvClass]\\

Filling postponed.\\

This attribute is meant a.o. to distinguish between nouns such as
{\em uur}, {\em maand} and {\em keer}.
The first should be assigned the value `reference', the second should be 
assigned the value `duration', and the third `frequential'.\\

The exact criteria will be specified by Lisette Appelo, who will probably also
take care of the actual filling. \\
Nondefault specification of .class (default = omegaTimeAdvClass) 
requires the value `true' for .temporal.



\newpage
\item 
[deixis:]\mbox{}

valueset: [omegadeixis, presentdeixis, pastdeixis]\\

Filling postponed.\\

The exact criteria will be specified by Lisette Appelo, who will probably also
take care of the actual filling. \\
Nondefault specification of .deixis (default = omegadeixis) requires
the value `true' for .temporal.

\newpage
\item 
[aspect]\mbox{}

valueset: [habitual, imperfective, perfective, omegaAspect]\\

Filling postponed.\\

The exact criteria will be specified by Lisette Appelo, who will probably also
take care of the actual filling. 
Nondefault specification of .aspect 
(default = omegaAspect) requires the value `true' for .temporal.

\newpage
\item 
[retro]\mbox{}

valueset: [true, false]\\

Filling postponed.\\

The exact criteria will be specified by Lisette Appelo, who will probably also
take care of the actual filling. \\

Nondefault specification of .aspect 
(default = false) requires the value `true' for .temporal.


\newpage
\item 
[sexes]\mbox{}
           
valueset: [masculine, feminine]\\

This attribute expresses the biological gender of a noun denotatum. 
In the Rosetta  syntax, reference to this attribute is
made
in order to select the proper pronoun for which the NP heading the noun
is antecedent. (The attribute value for .gender is overruled by the value of 
.sexes in case of conflicting assignments, for example for {\em meisje}.\\

If another value than the default value [] is chosen it should be guaranteed
that the value for .animate is `yesanimate'.
The value is more or less independent of the attribute value for .genders.
Note for example that {\em meisje} should have to be assigned the value 
[feminine] for .sexes, while is has the value [neutgender] for .genders.
Note that nouns that refer to animate objects may sometimes be used to
cover both the female and the male entities. This is reflected in the
language by the possibility that such a word be referred to by
a feminine pronoun. In that case
the value for .sexes to be chosen is: [feminine, masculine]. (Note that
sometimes the context influences the actual interpretation. For example, in 
{\em De directrice geeft de directeur haar agenda} the possessive pronoun {\em 
haar} seems {\bf not} to refer to {\em directeur} although in the 
lexicon [masculine, feminine] is probably the correct value for the attribute 
.sexes of {\em directeur}.)

Some examples:
\begin{itemize}
  \item 
{\em tafel}    : []
  \item
{\em man}      : [masculine]
  \item
{\em vrouwtje} : [feminine]
  \item
{\em varken}   : [feminine, masculine]
  \item
{\em minister} : [feminine, masculine]
  \item
{\em menigte}  : []
\end{itemize}

Problematic cases are words such as {\em poes}. It seems 
that {\em poes}
is ambiguous 
between a reading that may be paraphrased by  `feline pet', covering
both the female and the male entities,  and one that
may be paraphrased by `female cat'.  In the former case the assigment 
[feminine, masculine] is favoured, while the latter would require [feminine].
If Van Dale distinguishes more than one  entry, filling is easy.
If not, the translation relation should be checked in order to see whether 
splitting is required. If splitting is not necessary, the assignment 
[masculine, feminine] is not incorrect.


\newpage
\item 
[subcs]\mbox{}

valueset: [vocativenoun, professionnoun, relationnoun, unitnoun,
            plurunitnoun, abstractnoun, othernoun]\\

\begin{description}
\item[]
.subcs =  {\bf vocativenoun}  : nouns that may be used as a term of address, 
with or
without "specifying complement".\\
Examples: {\em dokter}, {\em moeder}, {\em bakker} Poot, {\em koningin} Beatrix
\item[]
.subcs =  {\bf professionnoun}  : nouns that may be used to indicate a 
profession.

In the Rosetta syntax, reference 
to this attribute value is made to distinguish between 
nouns that may, and nouns that may not be used as the head of a singular NP
without determiner 
in predicative position.

Examples: Jan is {\em bakker}; Zij is {\em hoogleraar}.
\item[]
.subcs =  {\bf relationnoun}  : nouns that may be used to indicate a relation 
between individuals. 

In the Rosetta syntax, no reference to this attribute has been made yet.
Its intended function is taken over by .possgeni (see below). 
If a noun is specified as a relationnoun, while it has the value `false' for
.possgeni it should be listed as marked.

Examples: {\em collega}, {\em broer}, {\em buurman} 
\item[]
.subcs =  {\bf abstractnoun}  : nouns that may be used to indicate an
abstract entity. (This criterion is not very transparant. In order to 
prevent that it is assigned more than will ever turn out to be useful it
is decided that nominalisations are not considered to be abstractnouns.)

In the Rosetta syntax, no reference to this attribute has been made yet.

Examples: {\em vrede}, {\em democratie}, {\em elektriciteit}, {\em tijd}
\item[]
.subcs =  {\bf unitnoun}  : cf. text below (section 4.1)
\item[]
.subcs =  {\bf plurunitnoun} : idem
\item[]
.subcs =  {\bf othernoun}  : nouns that do not meet the
characteristics of any of the other values.
\end{description}


Constraints:
\begin{itemize}
  \item
 Vocativenouns, professionnouns, relationnouns must have the 
value  `yeshuman' for .human.
  \item 
Nouns with the value `nonhuman' for .human  
cannot be vocativenoun, professionnoun, or relationnoun.
\end{itemize}

Note that theoretically a valueset containing more than
one element may give rise to extra derivations. 

{\bf INTERMEZZO}: The values {\bf plurunitnoun} and {\bf unitnoun}\\

Up to now, 
the Dutch domain distinguishes two (b)noun attributes that are relevant
for the account of phrases that contain some kind of measure phrase, viz.
.temporal and .subcs. The former attribute is boolean and is of relevance for
M-rules refering to temporal properties of (parts of) sentences. 
The latter is meant  to deal with the fact that some subclasses of nouns
exhibit various syntactic properties slightly different from the 
set of properties for the set of nouns as a whole. For NPs that behave as 
measure phrase, the 
value `unitnoun' is incorporated in the valueset of .subcs.
It has turned out that a valueset with no other `unit'-value 
is not fine grained enough.
The relevant data will be discussed below. It will 
be argued that the extra 
value `plurunitnoun' is needed to do justice to all the relevant facts.

At least the following two things are to be accounted for:
\begin{itemize}
\item the distributional peculiarities of temporal expressions
\item the fact that some nouns following a numerical determiner 
may/must  be singular. This possibility is restricted to contexts where the
NP-node of which the noun is the head demands a measure phrase. 
\end{itemize}

The attribute .temporal accounts for the distribution of temporal expressions
within sentential structures. Cf. the relevant passage underneath.
The examples under (1)-(5) illustrate the second fact. 
\begin{enumerate}
\item drie ( *vele/*enkele ) jaar geleden    vs. ?*drie maand geleden

      drie ( *vele/*enkele ) uur sneller     vs. *drie minuut sneller
\item tien ( *vele ) meter lang              vs. *tien turf/blok hoog
\item drie/een paar ( *vele ) kilo kaas      vs. *vijf emmer bramen
\item *drie onsje paling
\item drie meter naast de goal               vs. *twintig graad onder nul      
      
\end{enumerate}

To begin with, note that in the examples above some nouns are unit nouns
intrinsically, while others can be used as such under certain circumstances.
( There must be an indefinite DETP preceding the noun, among other things).
 I take nouns such as {\em blok} and {\em
emmer} as examples of the latter category and restrict the notion "unit noun"
to the former class. (Note alo that in the case of {\em *vijf emmer bramen},
{\em emmer} is considered to be the head of the phrase, while real unit nouns
function as modifier.) 
And as illustrated by (4) an intrinsic unit noun as {\em ons} does not preserve 
this property if it occurs in diminutive form. (Probably, this diminutive 
is only used colloquially.)

Secondly, not all real unit nouns may occur without
plural marking: the set of temporal 
unit nouns is not homogeneous in this respect, and also the dutch unit noun
{\em graad} requires plural marking if it is preceded by a cardinal.

By making reference to the 
the attribute value  for .subcs  that is originally introduced to 
distinguish unit nouns from non-unit nouns, i.e. {\bf unitnoun}, the
 heterogeneous behaviour of unit nouns with respect to plural marking 
cannot be accounted for. Reinterpreting the value {\bf unitnoun} such that 
it would mean something like: may occur without plural marking in 
measure phrase contexts, would incorrectly require that {\em graad}, 
{\em minuut} and {\em seconde} were to be taken as non-unit nouns. 
The simplest way of inproving the system is to retain the original value set, 
and to add one extra value for .subcs.
This avoids the risk that any reference to `unitnoun' would make the system 
inconsistent. 

I propose to distinguish from now on between:\\

\begin{description}
\item[]
.subcs =  {\bf unitnoun}  (for intrinsic unitnouns that may occur 
               without plural marking in measure phrase contexts, e.g. {\em uur})
\item[]
.subcs =  {\bf plurunitnoun}  (for intrinsic unitnouns that may not occur 
               without plural marking in measure phrase contexts, e.g. {\em maand})
\end{description}


\newpage
\item 
[temporal]\mbox{}

valueset: [true, false]\\

This attribute is meant to mark any noun that can have a temporal interpretation.
Not only temporal unit nouns, such as {\em minuut}, {\em week} or 
{\em decennium}
should be assigned the value `true', but also words such as {\em vakantie},
{\em vergadering}, {\em finale}, {\em Kerstmis}, {\em maandag},  etc. 

Of the following criteria at least one should apply:
\begin{itemize}
  \item 
 The NP can be combined with the prepositions {\em tijdens} and/or {\em gedurende}
  \item
 The NP can be followed by the adverbs  {\em geleden} and/or {\em terug} (in
the non-locative reading).
  \item
 The NP can be preceded by temporal prepositions {\em op}, {\em met}, etc. (in
the non-locative reading).
\end{itemize}

Other, perhaps surprising examples that meet one of these criteria:
\begin{description}
  \item[] {\em kabaal}
  \item[] {\em voorbereiding}
  \item[] {\em onweersbui}
\end{description}

Often a noun with more than one meaning key fulfils the criteria given here 
only for a subset of its readings. In that case the value to be assigned 
for .temporal is `true'. Whether splitting is required will be decided later on
by checking all the ambiguous nouns with .temporal = `true'.

\newpage
\item 
[possgeni]\mbox{}

valueset: [true, false]\\

This attribute is meant to indicate whether a noun can occur with 
a genitive marker.\\

Nondefault specification of .possgeni
(default = false) requires the value `yeshuman' for .human.\\

Examples:
\begin{description}
\item []
Mijn {\em zoons} voorkeur voor muziek is verbazingwekkend
\item []
Jan heeft zijn {\em moeders} auto total loss gereden
\item []
Het is zijn {\em collega's} liefste wens dat hij gebruik zou maken van de VUT.
\item []
*Mijn {\em auto's} remmen moeten nagekeken
\item []
*Mijn {\em tuins} onderhoud 
\item []
*Mijn * {\em kats} jongen

\end{description}

\newpage
\item 
[animate]\mbox{}

valueset: [yesanimate, noanimate, omegaAnimate]\\

This attribute is meant to distinguish between nouns that refer to 
animate objects and those that do not.\\

In the Rosetta syntax,  reference to this noun attribute has not been made 
in  crucial way up to now. Probably it will turn out to be of use for 
semantic purposes only. In order to gain more insight in
this attribute it has been decided to fill it according to the following
"semantic" criterion: a noun is yesanimate if and only if 
the entity it refers to can 
be frightened physically ("fysiek aan het schrikken gemaakt worden").



Examples:
\begin{itemize}
  \item yesanimate: {\em kat}, {\em kind}, {\em fabrikant}, {\em gezelschap}
  \item noanimate: {\em  vrede}, {\em steen}, {\em auto}
\end{itemize}


The value `omegaanimate'  should not be assigned to BNOUNs.\\
\newpage
\item 
[human]\mbox{}

valueset: [YesHuman, NoHuman, OmegaHuman]\\

This attribute is meant to distinguish between nouns that refer to 
humans and those that do not.\\

In the Rosetta syntax, reference to this noun attribute has been made 
to decide whether a noun can be modified by a relative sentence
with initial PREP + {\em wie}. Examples:
\begin{enumerate}
  \item De man van wie ik droom
  \item De mensen aan wie geen vragen gesteld kunnen worden
  \item De deelnemers voor wie ik bang ben
  \item *De stad van wie ik hou (vs. De stad waarvan ik hou)
  \item *De kat van wie ik hou (vs. De kat waarvan ik hou)
\end{enumerate}

The value `yeshuman' should be assigned if and only if  a noun occur as the 
antecedent of {\em wie}.

The value `omegahuman' should not be assigned to BNOUNs.\\

\newpage
\item 
[posscomas]

valueset: [count, mass]\\

This attribute is meant to indicate whether a noun can be used as
a mass noun, as a count noun, or as both.\\

A mass noun can be the head of a singular argument NP without determiner.
E.g. In de kelder staat nog (verse) {\em melk}.\\

A count noun is either an onlyplural that can be combined with a cardinal
or it is a noun that is not an onlyplural which
can be combined with the unreduced indefinite article {\em een}.
E.g. 
\begin{itemize}
  \item
Deze pop mist drie {\em ledematen}.
  \item
Zij heeft een {\em vent} in huis gehaald.
\end{itemize}


Problematic are nouns such as {\em hersenen}. They have no singular form, hence
they cannot be combined with {\em een}. It depends on their 
translations whether they are count nouns or mass nouns.\\


Some nous have has both possibilies, e.g. {\em kaas}.
\begin{itemize}
  \item
In de kelder ligt  nog (oude) {\em kaas} 
  \item
Er ligt nog een {kaas} in het magazijn.
\end{itemize}

In cases such as {\em kaas} Van Dale may 
give two paraphrases for the meaning,
one for the mass interpretation, the other for the count reading.
Probably it would be more correct to split up an entry that is both mass
and count. Experience must learn whether this should be done systematically/
automatically, therefore a decision on this matter is postponed and filling 
must be done according to the criteria given here, with [mass, count]
as a possible value.


\newpage
\item 
[thetanp]\mbox{}

valueset: [omegathetanp, thetanp0, thetanp1, thetanp2] \\

Filling postponed.\\

The values in this set do not parallel the values in the other theta-attributes.
Anticipating the incorporation of values that may account for the predicational
function of NPs the fictive value set specified has been defined.\\

\newpage
\item 
[nounpatterns]\mbox{}

Filling postponed. \\

The set that will turn out to be relevant to BNOUNs is probably a subset
of the synpatternSET.

valueset:
\newpage
\item
[prepkey]\mbox{}

This attribute is meant to indicate whether the noun can be complemented
by a prepositional object with a fixed preposition. 
This preposition is uniquely identified by its key (of type keytype).
Such a key is indicated indirectly by its name, a string preceded
by \$, as defined in the testing dictionary.\\

Examples:
\begin{itemize}
  \item 
 gehechtheid {\em aan}; .prepkey = \$aanprepkey
  \item
 overwinning {\em op}; .prepkey = \$opprepkey
\end{itemize}

Nondefault specification of .prepkey
(default = 0) would require a nondefault value for .thetanp.

\newpage
\item 
[personal]\mbox{}

valueset: [true, false]

This attribute is meant to distinguish so-called impersonal nouns. 
Nouns such as {\em zomer}, {\em weer} and {\em herfst} can (not: can only) 
be used in sentences
such as: 
\begin{itemize}
  \item 
Het is mooi weer vandaag
  \item 
Het is een barre zomer
  \item
Het wordt een mooie herfst.
  \item
Het wordt zo langzaamaan tijd om te vertrekken 
  \item
Het was crisis 
\end{itemize}
\newpage
\item 
[KEY]\mbox{}

The attribute .key is meant to uniqely identify a syntactic lexical element.\\

The attribute .key is specified in the top of a lemma by means of a string
preceded by \$s
\end{description}
\end{document}

