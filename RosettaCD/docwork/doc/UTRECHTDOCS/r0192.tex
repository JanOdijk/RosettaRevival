\documentstyle{Rosetta}
\begin{document}
   \RosTopic{General}
   \RosTitle{Key Definitions Under VMS/RBS}
   \RosAuthor{Jeroen Medema, Carel Fellinger}
   \RosDocNr{192}
   \RosDate{\today}
   \RosStatus{concept}
   \RosSupersedes{-}
   \RosDistribution{Project}
   \RosClearance{Project}
   \RosKeywords{RBS, VMS, Key}
   \MakeRosTitle
%
%
\setlength{\parindent}{0in}
{\Large Key Definitions Under VMS/RBS}\vspace{2mm}

Within the Rosetta development system (VMS and RBS), several keys have been 
defined. These keys have, until further notice, the meaning as shown in table~1.
The definitions for the keys are available only after selecting release 
rosetta3 (which, fortunately, is done in the groupslogin).\vspace{2mm}

If a definition of a key is terminated with $_e$ it means that, when depressing 
the key, the corresponding string is displayed and immediately executed. 
Pressing one of the other keys has as result that the string is displayed only. 
Some definitions contain the string $\langle comp\rangle$ which is not 
literally displayed but replaced by the currently selected component, except in 
the -so called- document commands dbuild, dshobuild, dsholog, and dintegrate. 
There, $\langle comp\rangle$ is substituted by the selected component only when 
the selected component is doc; otherwise owndoc is taken.\vspace{2mm}

\begin{center}
\begin{tabular}{|l||l|l|}\hline
 Key       
  & State = Default             
    & State = Gold \\ \hline
 PF1   
  & State := Gold$_e$
    & State := Default$_e$              \\
 F17   
  & "lbuild $\langle comp\rangle$:"    
    & "dbuild $\langle comp\rangle$:"   \\
 F18   
  & "shobuild"$_e$          
    & "dshobuild"$_e$\\
 F19   
  & "sholog"$_e$            
    & "dsholog $\langle comp\rangle$:"  \\
 F20   
  & "integrate 23:00"   
    & "dintegrate $\langle comp\rangle$:"\\
 Help  
  & "help"$_e$              
    & Gives help on the defined keys$_e$\\
 Do    
  & "modify $\langle comp\rangle$:"    
    & "inspect $\langle comp\rangle$:"  \\
 Find  
  & "recall "           
    & "recall/all"$_e$                  \\
 Insert Here
  & "grab $\langle comp\rangle$:"      
    & "work $\langle comp\rangle$:"     \\
 Remove
  & "free $\langle comp\rangle$:"      
    & "save $\langle comp\rangle$:"     \\
 Select
  & "select "           
    & "selected"$_e$                    \\
 Prev Screen
  & "cd .."$_e$            
    & "cd"$_e$                          \\
 Next Screen
  & "cd ??"$_e$             
    & "show default"$_e$                \\ \hline
\end{tabular}\vspace{1mm}

Table 1.
\end{center}

The keys in table~2 can be useful when editing command lines.\vspace{2mm}
\begin{center}
\begin{tabular}{|l|l|}\hline
Key                       & Meaning\\ \hline
Ctrl/A or F14             & Toggles between overwrite and insert mode.\\
Ctrl/E                    & Moves cursor to end of line.\\
Ctrl/H or F12             & Moves cursor to begin of line.\\
Crtl/J\hspace{1mm} or F13 & Deletes word to left of the cursor.\\
Ctrl/U                    & Deletes characters from beginning of line \\
                          & to cursor.\\ \hline
\end{tabular}\vspace{1mm}

Table 2.
\end{center}
\end{document}
