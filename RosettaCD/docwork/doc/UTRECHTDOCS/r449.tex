
\documentstyle{Rosetta}
\begin{document}
   \RosTopic{General}
   \RosTitle{Notulen groepsvergadering 19-11-1990}
   \RosAuthor{Franciska de Jong}
   \RosDocNr{449}
   \RosDate{23-11-1990}
   \RosStatus{approved}
   \RosSupersedes{-}
   \RosDistribution{Project}
   \RosClearance{Project}
   \RosKeywords{Notulen}
   \MakeRosTitle
%
%
\begin{description}
\item[Aanwezig:] Franciska de Jong, Ren\'{e} Leermakers,  
                 Jan Landsbergen (voorzitter),   
                 Jan Odijk, Elena Pinillos, 
                 Joep Rous, Andr\'{e} Schenk,
                 Harm Smit,
                 Frank Uittenbogaard, Petra de Wit
                  
                  

\item[Afwezig:]  Lisette Appelo
\item[Agenda:]\mbox{}
  \begin{enumerate}
  \item Notulen
  \item Actiepunten
  \item Diversen
  \item Activiteiten (restant)
  \item Rondvraag
  \end{enumerate}
\end{description}

\section{Notulen}
De notulen van de vorige keer worden met een enkele wijziging goedgekeurd.

\section{Actiepunten}

Over dit nieuwe agendapunt wordt het volgende afgesproken:
\begin{itemize}
  \item Tijdens de vergadering geeft de voorzitter aan wanneer een afspraak
 in de notulen als actiepunt dient te worden aangemerkt. 
  \item De lijst met actiepunten wordt tijdens elke vergadering nagelopen.
 Actiepunten die afgerond zijn worden van de lijst 
afgevoerd. De overige punten doen in de volgende ronde opnieuw mee. 
\end{itemize}

Afgeronde actiepunten:
\begin{enumerate}

  \item Er zijn geen nieuwe Amerika-reizen in 1991 te melden.
  \item Voor de inventarisatie van ons boekenbestand heeft Jan L. 
        contact opgenomen met Margot. Zij zal met Lisette in overleg treden. De 
overige groepsleden wordt verzocht een overzicht (inclusief plaatsaanduiding) 
te maken van de Rosetta-boeken die in hun bezit zijn. 
\end{enumerate}


\section{Diversen}
\begin{enumerate}
\item {\bf Groepsbudget}\\
De hoogte van ons indicatief groepsbudget is vastgesteld op 3,5 
miljoen.
\item {\bf Investeringsstop}\\
Voor het Nat. Lab. is voor 1991 een investeringsstop afgekondigd.
Er komen daarom geen nieuwe work stations. 

Het probleem van de bijzondere 
karakters kan wellicht worden opgelost door via ruiling een Sparc 
station te bemachtigen.
\item {\bf Actie Centurion} (vervolg)\\
Op 28 november vindt
het review-gesprek plaats tussen Research en de Raad van Bestuur.
Afronding van de portfolio-discussie en de zogenaamde efficiency drive
wordt pas in januari 1991 verwacht. 
Mogelijk komt na de hoofdgroepbespreking van 30 november 
informatie over het verloop van het review-gesprek beschikbaar.

\item {\bf Octrooien}(vervolg)\\
In tegenstelling tot wat in de vorige vergadering gemeld is blijkt 
het door Joep ingediende octrooi voor window-based translation 
als prioriteringsaanduiding {\em urgentie 1} te hebben gekregen. 


\item {\bf LEXIC}\\
Monique Meyer vertrekt bij LEXIC, evenals de LEXIC-programmeur.\\
Jan O. heeft uit contacten met Lexic-taalkundigen de indruk gekregen dat ze 
met de ontwikkeling van het lexicologisch concept linguistisch gezien op de goede 
weg zijn. 

\item {\bf CE}
\begin{description}
  \item{\bf text products}\\
De eerder genoemde `zero-date', 
het moment waarop besluiten worden genomen over toekomstige
 produktie, is op 7 december 1990.
Voor die datum zal er een bespreking worden gevoerd met CELEX over hun 
offerte. De linguistische component zal deel uitmaken van hetgeen er 
op 7 december wordt besproken.
  \item{\bf CD/I} \\
Jan L. heeft een gesprek gevoerd met de heer Hunziker, de man die de vreemde 
talen-cursussen voor Philips ontwikkeld heeft, over een eventuele bijdrage van 
Rosetta aan CD/I. 
Uit het gesprek is gebleken dat methode  van Hunziker 
uitgangspunt vormt voor CD/I. 
Over het nut van een computergestuurde extensie
bestaat een groot verschil van mening. De voorlopige conclusie van Jan L. is 
daarom dat 
het onwaarschijnlijk is dat er een bijdrage van Rosetta komt.

\end{description}


\item {\bf DEC}\\
DEC-medewerkers 
Cheryl Bettels en Ron Verheyen zijn op bezoek geweest voor een Rosetta-
demonstratie. Hun interesse voor vertalen ligt voornamelijk 
op het terrein van manuals en prompts.\\

Uit gesprekken met Jan L. 
is gebleken dat DEC onder subsidi\"{e}ring van onderzoek 
vooral verstaat: het leveren van hardware die door studenten gebruikt kan 
worden bij onderzoek waar DEC in is geinteresseerd.
Jan L. heeft duidelijk gemaakt dat voor veel onderzoek,  
bijvoorbeeld naar een robuuste parser, continuiteit van belang is. Die lijkt 
alleen met financi\"{e}le ondersteuning te bereiken. 


\section{Activiteiten (vervolg)}
Overzicht van de activiteiten van juni - november:
\begin{itemize}
  \item {\bf Joep}\\
Heeft een octrooi ingediend, een cursus Bedrijfskunde gevolgd,
gewerkt aan spelling checking en correctie, en aan 
de conversie.\\ 
Geschreven: een document over efficientieverbetering, een document
over spelling checking (in het bijzonder over de optimale lengte van een 
woordenlijst), en een paper voor de ACL, 

  \item {\bf Andr\'{e}}\\
Heeft de CSO-cursus gevolgd, de stagiaires begeleid, en de 
software tools ten behoeve van de woordenboeken gedefinieerd.
Werkt aan een voorstel over complexe predikaten (Actiepunt). 

\end{itemize}

\section{Rondvraag}
\begin{itemize}
  \item
Joep memoreert de mogelijkheid de verhuizing van het CST en de groepen 
Odijk en Eggenhuizen naar WAY te volgen. Jan L. zal t.z.t. hiernaar 
informeren. 
  \item 
Andr\'{e} vraagt of we het komend jaar opnieuw aan de CRE zullen meedoen, 
bijvoorbeeld met als nieuw punt: window based translation.
Jan L. meent dat we het op dit moment vooral van contacten moeten hebben. Van 
een 
CRE-bijdrage valt in dat opzicht niet veel te verwachten.
Joep vindt een CRE-bijdrage in het licht van de conversie ongewenst.

\end{itemize}



\section{Actiepunten}
\begin{itemize}
  \item Frank: Vertaling VMS-UNIX bemachtigen 
  \item Andr\'{e}: 
\begin{itemize}
  \item (vertaal)idioomregels Spaans
  \item document met procedure voor het vullen van idiomen
  \item voorstel over complexe predikaten 
  \item documentatie idioomregels
\end{itemize}
\item Jan O. :
\begin{itemize}
  \item verspreiding van het Rosetta LEXIC-document
\end{itemize}

  \item Franciska:
\begin{itemize}
  \item Overzicht stand van zaken Spaanse NP en adjectieven
  \item stuk over floaters
\end{itemize}
  \item Harm: rapport over grammar checkers lezen
  \item Ren\'{e}: idee over regel-georienteerde semantische component in eerste 
kwartaal 1991
\end{itemize}

\section{Volgende vergadering}
De volgende vergadering is op maandag 3 december, 13.30 uur in WY7.\\
\end{document}


