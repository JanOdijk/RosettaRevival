\documentstyle{Rosetta}
\begin{document}
   \RosTopic{Rosetta3.doc.linguistics.Dutch}
   \RosTitle{Dutch M-rules:subgrammar ADJPFORMULAtoADJPPROP}
   \RosAuthor{Franciska de Jong, Lisette Appelo}
   \RosDocNr{402}
   \RosDate{August 14, 1990}
   \RosStatus{approved}
   \RosSupersedes{concept of September 4, 1989}
   \RosDistribution{Project}
   \RosClearance{Project}
   \RosKeywords{Dutch, M-rules, ADJPFORMULAtoADJPPROP}
   \MakeRosTitle
%
%
\input{[dejong.mrules]mrudocdef}

\section{Introduction}

In this document the subgrammar ADJPFORMULAtoADJPPROP for Dutch is described.
The part on  aspectneutralization 
has been written by 
Lisette Appelo, the other parts have been written by Franciska de Jong.

This subgrammar has been attuned 
to the 
subgrammar ClausetoSentence.  It is not relevant to {\bf all} 
adjectival constructions, but only to the ones that end up  
either as ADJP-utterance or as NP-internal constituent, and to the ones that
are
imported 
into the subgrammar XPPROPtoClause via substitution.

\section{Subgrammar Specification}

\begin{description}
  \item[Head] ADJPFORMULA
  \item[Export] ADJP (utterance), OPENADJPPROP, CLOSEDADJPPROP
  \item[Import] EMPTY, NP, ADVP
  \item[File] dutch:adjsubgrammars.mrule (mrules48)
\end{description}

\section{Control Expression}
The control expression can be defined as follows:
\begin{verbatim}

  [TC\_ADJProStatus] 

. {RC\_ADJNEGVARintro} (not written yet)

. {RC\_ADJSubstitution}

. {TC\_ADJNEGAdaptation} (not written yet)

. TC\_ADJAspectNeutralization

. {TC\_ADJHETtoER} (not written yet

. [TC\_ADJalsIntro]

. RC\_ADJMOOD

. RC\_ADJpunc

. [RC\_ADJEMPTYsubjSUBST]      

. [RC\_HELPEMPTYsubjSUBST]   

\end{verbatim}

\section{Rules and Transformations}
\begin{mruleclass}{TC\_ADJProstatus}
\begin{classdescr}
\kind optional transformation class
\classtask to mark the ADJPFORMULA either as closed  (default), or as open
\classremarks

\nofilters

\nospeedrules

\noplannedrules

\norulesnotince

\rulelist

\end{classdescr}

\begin{members}
\begin{member}
\rulename TADJsetprosubject
\ruletask to set the value for .prosubject to {\em true}
\file dutch:rc\_helpadjputt.mrule (mrules50)
\semantics \nosemantics
\example \mbox{}
\begin{enumerate}
\item
 all adjectival phrases that are analysed as utterance 
\item
all adjectival phrases that are imported as modifier into the subgrammar
CNformation
\end{enumerate}
\remarks\mbox{}
\end{member}
\end{members}
\end{mruleclass}


\begin{mruleclass}{RC\_ADJNEGVARintro}
\begin{classdescr}
\kind recursive rule class
\classtask 
introduction of the markers NEGVAR and POSVAR
\classremarks\mbox{} 

\begin{enumerate}
  \item 
Cf. also doc.nr.  368.
  \item
No rules have been written yet. 
\end{enumerate}
\nofilters

\nospeedrules

\noplannedrules

\norulesnotince

%\rulelist 

\end{classdescr}
\end{mruleclass}
\begin{mruleclass}{RC\_ADJSubstitution}
\begin{classdescr}
\kind recursive rule class
\classtask to substitute NPs, ADVPs and PREPPs for their VARs.
\classremarks\mbox{}
\begin{enumerate}
  \item 
The substitution order conditions that are to 
assure that scope is accounted for 
correctly are not activated yet.
  \item
As in all substitution rules, the parameter LEVEL is used to check 
the index of the variable. 
\item
All rules that introduce NPs as a substituent presume the NP to be non-
generic (.generic = nogeneric). 
Proper conditions on the introduction of generic NPs are still to be formulated.
This is postponed as the NPs resulting from the subgrammar NPformation all have 
the value nogeneric (as yet). 
\item
The case attribute .cases of substituent NPs 
is assigned a value by means of the function {\em AssignCase}.
\item
The  attribute .superdeixis of substituent NPs is determined in this RC
in accordance with the general strategy described in doc:r263 and doc:r320
  (Lisette Appelo).


\end{enumerate}

\nofilters

\begin{speedrules}
\begin{member}
\rulename FADJprehetsubst
\ruletask To speed up analysis. The result of
replacement of subject {\em het} by NPVAR is filtered out in case 
of an ergative 
adjective with a HET-pattern (AUX\_hetcomplvps).
\file dutch:rc\_helpadjputt.mrule (mrules50)
\end{member}
\end{speedrules}

\begin{plannedrules}
\item Filters associated to this RC that block derivations 
for structures containing NPs while there is idiomatic context.
\item Rules for the insertion/substitution of NEGVAR and POSVAR.
(Note that these expressions are not variables. 
Cf. the document on scope, doc. nr. 368  by J. Odijk.)
\end{plannedrules}

\norulesnotince

\rulelist

\end{classdescr}

\begin{members}
\begin{member}

\rulename RADJSubstitution1
\ruletask Substitution of object NPs, indirect object NPs,
and NPs in locargrel (if any of the latter kind exists).
\file dutch:rc\_helpadjputt.mrule (mrules50)
\semantics substitution
\example het werk + objrel/x1 beu $\rightarrow$ het werk beu
\remarks\mbox{}
\end{member}

\begin{member}
\rulename RADJSubstitution2
\ruletask Substitution of NPs in argument PREPPs and in voorobj-modifiers
\file dutch:rc\_helpadjputt.mrule (mrules50)
\semantics substitution
\example \mbox{}
\begin{enumerate}
  \item 
de os + op x1 verliefd $\rightarrow$ op de os verliefd
  \item
de kinderen + voor x1 leuk$\rightarrow$ leuk voor de kinderen
\end{enumerate}
\remarks\mbox{}
Generatively, VARPREPP, the node containing NPVAR, is replaced by PREPP.
 \end{member}

\begin{member}
\rulename RADJSubstitution3
\ruletask Substitution of subject NPs and NPs that function as preadv. The 
latter because of isomorphy but  probably not relevant in this subgrammar.
\file dutch:rc\_helpadjputt.mrule (mrules50)
\semantics substitution
\example Jan + x1 vervelend $\rightarrow$ Jan vervelend\\ 
         (Jan wordt vervelend)
\remarks\mbox{}
\end{member}

\begin{member}
\rulename RADJSubstitution4
\ruletask Substitution of ADVPs with locargrel.
\file dutch:rc\_helpadjputt.mrule (mrules50)
\semantics substitution
\example hier + locargrel/x1 gewend $\rightarrow$ hier gewend
\remarks\mbox{}

\end{member}
\end{members}
\end{mruleclass}

\begin{mruleclass}{TC\_ADJNEGAdaptation}
\begin{classdescr}
\kind iterative transformation class with associated filter
\classtask to account for the morphological incorporation of a negation
(amounts to the replacement of a negation and a following expression by another 
expression)
\classremarks\mbox{} No rules have been written yet. 
\nofilters

\nospeedrules

\noplannedrules

\norulesnotince

\rulelist

\end{classdescr}
\end{mruleclass}

\begin{mruleclass}{TC\_ADJaspectneutralization}
\begin{classdescr}
\kind obligatory transformation  class
\classtask to account for the value of .aspect
\classremarks\mbox{} 

This TC makes the value of the aspect attributes compatible with the
values on surface level. 
It facilitates a certain amount of  efficiency in the surface parser. 
\nofilters

\nospeedrules

\noplannedrules

\norulesnotince

\rulelist

\end{classdescr}

\begin{members}
\begin{member}
\rulename TADJaspectneutralization
\ruletask to change the value for .aspect from {\em imperfective} into 
{\em omegaaspect}
\file dutch:tempadj2.mrule (mrules80)
\semantics \nosemantics
\example 
\remarks\mbox{} 
\end{member}
\end{members}

\end{mruleclass}

\begin{mruleclass}{TC\_ADJHETtoER}
\begin{classdescr}
\kind obligatory transformation class
\classtask 
to deal with the transition of {\em het} to {\em er} in the context of 
prepositions
\classremarks\mbox{} No rules have been written yet. They are to be copied from the
corresponding rules by Jan Odijk.
\nofilters

\nospeedrules

\noplannedrules

\norulesnotince

\rulelist

\end{classdescr}
\end{mruleclass}

\begin{mruleclass}{TC\_ADJalsIntro}
\begin{classdescr}
\kind obligatory transformation class
\classtask In analysis: to introduce {\em als} in the context 
{\em zo ... mogelijk}. 
\classremarks\mbox{} 

\begin{filters}
\begin{member}
\rulename FADJalsIntro
\ruletask To speed up analysis: no structures
without THANAS {\em als} are accepted.
\file dutch:rc\_helpadjputt.mrule (mrules50)
\end{member}

\end{filters} 

\nospeedrules

\noplannedrules

\norulesnotince

\rulelist

\end{classdescr}
\begin{members}
\begin{member}
\rulename TADJalsIntro
\ruletask 
\example analytically: zo ADJ  mogelijk  $\rightarrow$ zo ADJ als mogelijk
\file dutch:rc\_helpadjputt.mrule (mrules50)
\semantics \nosemantics
\remarks\mbox{}

\end{member}
\end{members}
\end{mruleclass}



\begin{mruleclass}{RC\_ADJMood}
\begin{classdescr}
\kind obligatory rule class
\classtask to replace the topnode ADJPFORMULA by CLOSEDADJPPROP or OPENADJPPROP
\classremarks 
In order to decide which rule out of this RC applies
no reference is made to the 
inherent properties of the ADJP in predrel.
In analysis the matter is decided by the syntactic role of the ADJP, in 
generation it is decided by IL, subsequently the syntactic derivation tree.
\nofilters

\nospeedrules

\noplannedrules

\norulesnotince

\rulelist

\end{classdescr}

\begin{members}
\begin{member}
\rulename RADJMOOD1
\ruletask replace topnode ADJPFORMULA by CLOSEDADJPPROP
\example all adjectival phrases that are imported  into the subgrammar
XPPROPtoCLAUSE as propositional argument to a verb, e.g. {\em zij vindt {\bf 
de zin lelijk}} 
\file dutch:rc\_helpadjputt.mrule (mrules50)
\semantics identity??
\remarks\mbox{}

\end{member}

\begin{member}
\rulename RADJMOOD2
\ruletask replace topnode ADJPFORMULA by OPENADJPPROP
\file dutch:rc\_helpadjputt.mrule (mrules50)
\semantics identity??
\example \mbox{}
\begin{enumerate}
\item
 all adjectival phrases that are analysed as utterance, e.g. {\em ziek}, {\em 
waarschijnlijk}
\item
all adjectival phrases that are imported as modifier into the subgrammar
CNformation, e.g {\em de {\bf zieke} man}
\item adjectival phrases which require an antecedent for their subjectvar
that is argument of the main predicate of a a clause, e.g.
{\em de taart smaakt {\bf lekker}} 
(via substitution: x1 x2 smaak $\rightarrow$ x1 [x1 lekker] smaak)
\end{enumerate}
\remarks\mbox{}

\end{member}
\end{members}
\end{mruleclass}

\begin{mruleclass}{RC\_ADJPunc}
\begin{classdescr}
\kind obligatory rule class
\classtask the rules exist for reasons of isomorphy only
\classremarks

\nofilters

\nospeedrules

\noplannedrules

\norulesnotince

\rulelist

\end{classdescr}

\begin{members}
\begin{member}
\rulename RADJpunc
\ruletask No task. Rule is only needed to guarantee
isomorphism with corresponding subgrammars.
\file dutch:rc\_helpadjputt.mrule (mrules50)
\semantics identity??
\example any adjectival constituent
\remarks\mbox{}
\end{member}
\end{members}
\end{mruleclass}

\begin{mruleclass}{RC\_ADJEMPTYsubjSUBST}
\begin{classdescr}
\kind optional rule class
\classtask  to guarantee the presence of a subject in case of an ADJP-adjunct.
\classremarks\mbox{} 
\nofilters

\nospeedrules

\noplannedrules

\norulesnotince

\rulelist

\end{classdescr}
\begin{members}
\begin{member}
\rulename RADJEMPTYsubjSUBST

\ruletask 
To substitute subject EMPTYvar in case of a free adjunct 
ADJP(PROP).

\example 
zo spoedig (als) {\bf mogelijk}, even lang als {\bf breed}
 
\file dutch:rc\_helpadjputt.mrule (mrules50)
\semantics 
\remarks\mbox{}

\end{member}
\end{members}
\end{mruleclass}



\begin{mruleclass}{RC\_HELPEMPTYsubjsubst}
\begin{classdescr}
\kind optional rule class
\classtask analysis:
to create an adjpprop-level for the testing of adjectival utterances
 by 
introducing an auxiliary subject to ADJPs.
\classremarks

\nofilters

\nospeedrules

\noplannedrules

\norulesnotince

\rulelist

\end{classdescr}

\begin{members}
\begin{member}
\rulename RHELPEMPTYsubjSUBST
\ruletask analysis: 
to create an adjpprop-level for the testing of adjectival utterances
 by 
introducing an auxiliary subject to an ADJP.
\file dutch:rc\_helpadjputt.mrule (mrules50)
\semantics predication and empty-substitution
\example any ADJP
\remarks\mbox{}
\end{member}
\end{members}
\end{mruleclass}

\end{document}
