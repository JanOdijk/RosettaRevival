
\documentstyle{Rosetta}
\begin{document}
   \RosTopic{Rosetta3.doc.linguistics.English}
   \RosTitle{English Surface Rules}
   \RosAuthor{Petra de Wit}
   \RosDocNr{472}
   \RosDate{\today}
   \RosStatus{concept}
   \RosSupersedes{-}
   \RosDistribution{Project}
   \RosClearance{Project}
   \RosKeywords{Surface Parser, Sentences, English}
   \MakeRosTitle
%
%
\section{Introduction}
This documents describes the English surface rules for sentences which can be 
found in {\em surfrules1.sur} and {\em surfrules4.sur}. 
For each rule the regular 
expression as 
well as the parameters and auxiliary functions used in this rule are 
described. Second, some special measures having to do with specific syntactic 
constructions will be explained.\\
\section{Verb Phrase}
\subsection{Regular Expression}
The surface rules VERBPrule describes all possible grammatical orderings of 
elements within the verb phrase. The regular expression looks as follows:
\begin{verbatim}
% VERBPrule

REGULAREXPRESSION:

VERBP = AUXVERBS . [VERB/1] . [NP/3] . [PART/2] . [NP/4] . [PART/2]
        . [REST] . [PREPP/11] . [SENTENCE/5] . [PREPP/6] . [PREPP/18]

AUXVERBS = [VERB/7] . [VERB/8] . [VERB/9]

REST = PREPP/10 | 
       PREPP/12 | ADVP/14 |
       PREPP/13 | ADVP/15 |
       ADJP/16 |
       NP/17 |
       PREPP/19 |
       VERBP/20 |
       PROSENT/21
\end{verbatim}
where the numbered boxes represent the following grammatical relations:\\
\newline
\begin{tabular}{|r|l|l|} \hline
\#&relation&example\\ \hline
1&head&He {\em left}\\ \hline
2&partrel&He called {\em up} the man\\ \hline
&&He called the man {\em up}\\ \hline
3&indobjrel&He gave {\em her} a book\\ \hline
4&objrel&He loathed {\em her}\\ \hline
5&complrel&He wants {\em to leave}\\ \hline
&&He knows {\em she is ill}\\ \hline
&&I don't know {\em what to do}\\ \hline
6&prepobjrel&He seems ill{\em to me}\\ \hline
7&auxrel&The bridge might {\em have} been being built\\ \hline
8&progauxrel&The bridge might have {\em been} being built\\ \hline
9&passauxrel&The bridge might have been {\em being} built\\ \hline
10&prepobjrel&They talked {\em to her} about the new car\\ \hline
11&prepobjrel&They talked to her {\em about the new car}\\ \hline
12&locargrel&He lives {\em in Eindhoven}\\ \hline
13&dirargrel&They went {\em to Eindhoven}\\ \hline
14&locargrel&He lives {\em there}\\ \hline
15&dirargrel&He went {\em there}\\ \hline
16&predrel&He painted the floor {\em green}\\ \hline
17&predrel&They called him {\em a liar}\\ \hline
18&byobjrel&He was caught {\em by the police}\\ \hline
19&predrel&He cut the onion {\em into pieces}\\ \hline
20&predrel&He had a house {\em built}\\ \hline
21&predrel&I hope {\em so}\\ \hline
\end{tabular}
\subsection{Parameters}
The Surface Grammar uses parameters in analyzing verb phrases. These 
parameters are used to percolate attribute values to a 
higher X-bar level (e.g from the verb to the verb phrase), 
to check the computed structure and values at a 
specific point in the analysis and to compute the value of some attributes in 
sepcial functions or the rule itself. 
Below a list is given of all parameters used , their 
type and their function:\\
\begin{description}
\item [ADJPpredfound      ] BOOLEAN\\
records whether an ADJP bearing the relation predrel has been found.
\item [adjunctsvar        ] adjunctSETtype\\
records the number of adjuncts a verb may select.
\item [adjunctsefsvar     ] adjunctSETtype\\
recors the number of adjuncts actually found.
\item [adjvpefsvar        ] synpatterneffSETtype\\
records the adjpatterns of an adjective phrase in predrel
\item [asifcomplfound     ] BOOLEAN\\
records whether a complement clause starting with {\em as if} has been found.
\item [auxfound           ] BOOLEAN \\ 
records whether an auxiliary has been found.
\item [auxverbfound       ] BOOLEAN \\
records whether an auxiliary verb has been found.
\item [byobjfound         ] BOOLEAN\\
records whether a by-phrase bearing the relation objrel has been found.
\item [caseassignervar    ] BOOLEAN\\
records the value caseassigner of the verb.
\item [classesvar         ] classSETtype\\
records the class of the verb bearing the relation head.
\item [complfound         ] BOOLEAN\\
records whether complement has been found.
\item [complinfvar        ] inftype\\
records the infsort of a complement sentence.
\item [complmodusvar      ] modustype\\
records the modus of a complement sentence.
\item [complmoodvar       ] moodtype\\
records the mood of a complement sentence.
\item [complprosubjectvar ] BOOLEAN\\
records whether the complement sentence contains a PRO subject.
\item [dirargfound        ] BOOLEAN\\
records whether a directional argument has been found.
\item [envvar             ] polarityeffsettype\\
??
\item [foundauxesvar      ] auxSETtype\\
records which auxiliaries have been found.
\item [headfound          ] BOOLEAN\\
records whether the VP contains a head verb.
\item [headmodusvar       ] modustype\\
records the modus of the verb bearing the relation head.
\item [indobjfound        ] BOOLEAN\\
records whether an indirect object has been found.
\item [locargfound        ] BOOLEAN\\
records whether a locative argument has been found.
\item [modusvar           ] modustype\\
records the modus of the first auxiliary found.
\item [NPpredfound        ] BOOLEAN\\
records whether an NP bearing the relation predrel has been found.
\item [objfound           ] BOOLEAN\\
records whether an object has been found.
\item [partfound          ] BOOLEAN\\
records whether a particle has been found.
\item [passauxfound       ] BOOLEAN\\
records whether a passive auxiliary has been found.
\item [particlevar        ] keytype\\
records the key of the particle
\item [possvoicesvar      ] voiceSETtype\\
records the value of the attribute possvoices of the verb bearing the relation head.
\item [predfound          ] BOOLEAN\\
records whether an XP bearing the relation predrel has been found.
\item [prepkeyvar1        ] keytype\\
records the key of the PREPP the verb bearing the relation head selects
\item [prepkeyvar2        ] keytype\\
records the key of a second PREPP the verb bearing the relation head selects
\item [prepobjfound       ] BOOLEAN\\
records whether a prepositional object has been found.
\item [prepobj2found      ] BOOLEAN\\
records whether a second prepositional object has been found.
\item [PREPPpredfound     ] BOOLEAN\\
records whether a PREPP bearing the relation predrel has been found.
\item [progauxfound       ] BOOLEAN\\
records whether a progressive auxiliary has been found.
\item [reflexivityvar     ] reflexivetype\\
records the reflexivetype of the verb bearing the relation head.
\item [reqvar             ] polarityeffsettype\\
??
\item [SENTcomplfound     ] BOOLEAN\\
records whether a sentence bearing the relation complrel has been found.
\item [strandedPPfound    ] BOOLEAN\\
records whether a stranded prepositional phrase has been found.
\item [synvpefsvar        ] synpatterneffsettype\\
records the actual patterns used in the sentence analysed.
\item [synvpsvar          ] synpatternsettype\\
records the value of the attribute synvps of the verb bearing the relation head.
\item [tensevar           ] tensetype\\
records the tense of the verb bearing the relatio head.
\item [thetavpvar         ] thetavptype\\
records the thetavp value of the verb bearing the relation head.
\item [verbnumbersvar     ] numbersettype\\
records the number of the verb bearing the relation head.
\item [verbpersonsvar     ] personSETtype\\
records the person of the verb bearing the relation head.
\item [VERBPcomplfound    ] BOOLEAN\\
record whether a verb phrase bearing the relation predrel has been found.
\item [voicevar           ] voicetype\\
records the voice of the verb.
\end{description}
\subsection{Auxiliary Functions and Procedures}
\section{Verb Negation}
\subsection{Regular Expression}
There is a special regular expression for negative verbs which looks as 
follows:\\
\begin{verbatim}
%VERBNEGrule

REGULAREXPRESSION:

VERB = (VERB/1 . GLUE/2 . NEG/3)   
\end{verbatim}
where the numbered boxes represent the following grammatical relations:\\
\newline
\begin{tabular}{|r|l|l|} \hline
\#&relation&example\\ \hline
1&head&do\\ \hline
2&gluerel&\\ \hline
3&negrel&not\\ \hline
\end{tabular}
\subsection{Parameters}
The following parameters are used :
\begin{itemize}
\item   reqvar           
\item   envvar           
\item   conjclassesvar   
\item   ingformvar       
\item   sformvar         
\item   affixvar         
\item   modusvar         
\item   tensevar         
\item   personsvar       
\item   numbersvar       
\item   particlevar      
\item   possvoicesvar    
\item   reflexivityvar   
\item   synvpsvar        
\item   thetavpvar       
\item   CaseAssignervar  
\item   oblcontrolvar    
\item   prepkey1var      
\item   prepkey2var      
\item   controllervar    
\item   classesvar       
\item   thatdelvar       
\end{itemize}
\subsection{Auxiliary Functions and Procedures}
No auxiliary functions and procedures are needed.
\section{Sentence}
\subsection{Regular Expression}
The regular expression for sentences looks as follows:
\begin{verbatim}
% SENTENCE1rule

REGULAREXPRESSION:
 
SENTENCE = [LDL] . [SH] . [CPOS] . [GLUE/33] . [NP/2 | POSSADJ/32]  . [ADVP/25] 
           . [VERB/3] . [NEG/6] . [TOCAT/31] . [ADVP/26] . [VERBP/4]
           . ADVS . EXTRA . [[PUNC/7] . SENTENCE/23] . [PUNC/5]

LDL = (SENTENCE/14 .[PUNC/7]) | (SENTENCE/15 .[PUNC/7]) |(ADVP/35.[PUNC/7])|
      NP/27 | PREPP/28 | ADVP/29 | ADJP/30

SH = NP/9 | PREPP/10 | ADVP/11 | ADJP/12 | RELPRO/13

CPOS = VERB/1 | CONJ/8 | PREP/34

ADVS = LOCADVS . TEMPADVS . SENTADVS

LOCADVS = {ADVP/16 | PREPP/17}

TEMPADVS = {ADVP/18 | PREPP/19 | NP/22}

SENTADVS = {ADVP/20 | PREPP/21 }

EXTRA = [[PUNC/7] .SENTENCE/24]
\end{verbatim}
where the numbered boxes represent the following grammatical relations:\\
\begin{tabular}{|r|l|l|} \hline
1&conjrel&{\em Do} you love me ?\\ \hline
2&subjrel&Do {\em you} love me ?\\ \hline
3&auxrel&The bridge {\em might} have been being built\\ \hline
4&predrel&He {\em told her to leave}\\ \hline
5&puncrel&I wonder{\em .}\\ \hline
6&negrel&He does {\em not} want to leave \\ \hline
7&puncrel&Unfortunately{\em ,} this rule is never used.\\ \hline
8&conjrel&He knew {\em that} you were ill\\ \hline
9&shiftrel&{\em Who} did you talk to ?\\ \hline
10&shiftrel&{\em To whom} did you talk \\ \hline
11&shiftrel&{\em Where} does he live\\ \hline
12&shiftrel&{\em How ill} is he\\ \hline
13&shiftrel&The man {\em who} came to dinner\\ \hline
14&leftdislocrel&{\em Although he had never seen her}, ..\\ \hline
15&leftdislocrel&{\em How the book will sell} depends on the author\\ \hline
16&locadvrel&I saw her {\em there} today\\ \hline
17&locadvrel&I saw her {\em in church} today\\ \hline
18&tempadvrel&I saw her {\em daily}\\ \hline
19&tempadvrel&I saw him {\em during the concert}\\ \hline
20&sentadvrel&He didn't like me {\em apparently}\\ \hline
21&sentadvrel&I met him {\em thanks to her}\\ \hline
22&tempadvrel&I saw him {\em the day before yesterday}\\ \hline
23&postsentadvrel&He was in time {\em although the train was late}\\ \hline
24&extraposrel&It is time {\em to go home}\\ \hline
25&modrel&He {\em probably} does not like her\\ \hline
26&modrel&He promised to {\em always} love her\\ \hline
27&leftdislocrel&{\em His face} I am not fond of\\ \hline
28&leftdislocrel&{\em With great skill} he hit the ball\\ \hline
29&leftdislocrel&{\em Probably, he left}\\ \hline
30&leftdislocrel&{\em Rich} I may be\\ \hline
31&torel&He wants {\em to} leave\\ \hline
32&subjrel&Let'{\em s} swim\\ \hline
33&gluerel&Let's swim\\ \hline
34&conjrel&{\em Before he left}, he drank some wine\\ \hline
\end{tabular}
\subsection{Parameters}
To correctly analyze and compute the value of different attributes the 
following parameters are used:\\
\begin{description}
\item [accsubjfound       ] BOOLEAN\\
records whether an accusative NP bearing the relation subjrel has been found. 
\item [adjunctsvar        ] adjunctSETtype\\
??
\item [adjpinshiftfound   ] BOOLEAN\\
records whether an ADJP bearing the relation shiftrel has been found.
\item [adjppredrelfound   ] BOOLEAN\\
records whether an ADJP bearing the relation predrel has been found.
\item [adverbialvar       ] BOOLEAN\\
records whether an adverbial has been found.
\item [advfound           ] BOOLEAN\\
records whether an adverb has been found.
\item [advpinshiftfound   ] BOOLEAN\\
records whether an ADVP bearing the relation shiftrel has been found.
\item [asifcomplfound     ] BOOLEAN\\
records whether a complement clause starting with {\em as if} has been found.
\item [auxfound           ] BOOLEAN\\
records whether an auxiliary has been found.
\item [classesvar         ] classSETtype\\
records the value of the attribute classes
\item [conjaspectvar      ] aspecttype\\
\item [conjclassvar       ] timeadvclasstype\\
\item [CONJconjfound      ] BOOLEAN\\
\item [conjdeixisvar      ] deixistype\\
\item [conjfound          ] BOOLEAN\\
\item [conjkeyvar         ] keytype\\
\item [conjpatternsvar    ] synpatternEFFSETtype\\
\item [conjretrovar       ] retrotype\\
\item [deixisvar          ] deixistype\\
\item [deixisvarvar       ] deixistype\\
\item [dirargrelfound     ] BOOLEAN\\
\item [envvar             ] polarityEFFSETtype\\
\item [extraposfound      ] BOOLEAN\\
\item [extraposinfsortvar ] inftype\\
\item [extraposmoodvar    ] moodtype\\
\item [extraposmodusvar   ] modustype\\
\item [extraposok         ] BOOLEAN\\
\item [extraposprosubjectvar ] prosubjecttype\\
\item [finalpuncfound     ] BOOLEAN\\
\item [finitenessvar      ] finitenesstype\\
\item [finitenessvarvar   ] finitenesstype\\
\item [gensubjfound       ] BOOLEAN\\
\item [indobjrelfound     ] BOOLEAN\\
\item [infdoinconjfound   ] BOOLEAN\\
\item [infsortreset       ] BOOLEAN\\
\item [infsortvar         ] inftype\\
\item [infsortvarvar      ] inftype\\
\item [inversfound        ] BOOLEAN\\
\item [leftdislocfound    ] BOOLEAN\\
\item [letinconjfound     ] BOOLEAN\\
\item [letusgluefound     ] BOOLEAN\\
\item [locargrelfound     ]  BOOLEAN\\
\item [modusvar           ] modustype\\
\item [modusvarvar        ] modustype\\
\item [moodvar            ] moodtype\\
\item [negfound           ] BOOLEAN\\
\item [nomsubjfound       ] BOOLEAN\\
\item [notinauxfound      ] BOOLEAN\\
\item [npinshiftfound     ] BOOLEAN\\
\item [nppredrelfound     ]  BOOLEAN\\
\item [objrelfound        ] BOOLEAN\\
\item [onlyaccsubjfound   ] BOOLEAN \\
\item [onlynomsubjfound   ] BOOLEAN \\
\item [particlekey        ] keytype\\
\item [prepkeyvar1        ] keytype\\
\item [prepkeyvar2        ] keytype\\
\item [prepobjrelfound    ] BOOLEAN\\
\item [prepobj2relfound   ] BOOLEAN\\
\item [preppinshiftfound  ] BOOLEAN\\
\item [prepppredrelfound  ] BOOLEAN\\
\item [prosubjectvar      ] BOOLEAN\\
\item [RELPROfound        ] BOOLEAN\\
\item [reqvar             ] polarityEFFSETtype\\
\item [scomplrelfound     ] BOOLEAN\\
\item [senttypevar        ] senttypetype\\
\item [shiftcases         ] casesettype\\
\item [shiftfound         ] BOOLEAN\\
\item [shiftnumbervar     ] numbertype\\
\item [shiftpersonvar     ] persontype\\
\item [shiftsexes         ] sexSETtype\\
\item [shiftxpmoodvar     ] xpmoodtype\\
\item [Sinldlfound        ] BOOLEAN\\
\item [strandedrelfound   ] BOOLEAN\\
\item [subjNPhead         ] NPheadtype\\
\item [subjnumbervar      ] numbertype\\
\item [subjpersonvar      ] persontype\\
\item [subjfound          ] BOOLEAN\\
\item [synvpefsvar        ] synpatternEFFsettype\\
\item [temporalvar        ] BOOLEAN\\
\item [tensevar           ] tensetype\\
\item [therenpfound       ] BOOLEAN\\
\item [thetavpvar         ] thetavptype\\
\item [tofound            ] BOOLEAN\\
\item [VERBconjfound      ] BOOLEAN\\
\item [verbpersonsvar     ] personSETtype\\
\item [verbnumbersvar     ] numberSETtype\\
\item [verbpfound         ] BOOLEAN\\
\item [verbprepkeyvar1    ] keytype\\
\item [verbprepkeyvar2    ] keytype\\
\item [verbsubcvar        ] verbsubctype\\
\item [verbsynvpsvar      ] synpatternSETtype\\
\item [verbthetavpvar     ] thetavptype\\
\item [voicevar           ] voicetype\\
\item [XPinldlfound       ] BOOLEAN\\
\end{description}
\subsection{auxiliary functions and procedures}
The function {\bf argspresent} checks whether the arguments that the verb 
selects are really present. First, the function searches for the argument 
within verb phrase. If the argument is not present, it may be moved out of the 
verb phrase by means of wh-movement or NP-movement. Wh-movement to the first 
local spec-CP is only allowed 
if:
\begin{itemize}
\item moodvar = whinterrogative 
\item moodvar = declarative AND senttypevar = mainclause
\item moodvar = relative AND finitenessvar = finite.\\
This accounts for the ungrammaticality of {\em *The book which to buy}
\end{itemize}      
Wh-movement out of the CP to a higher spec-CP position is only allowed if 
\begin{itemize}
  \item moodvar = declarative AND
                      (senttypevar = subordinateclause) AND
                      ((finitenessvar = finite) OR
                       (infsortvar IN [toinf, inf])
                      );
\end{itemize}
Hiddenshift is only allowed in relatives
\begin{itemize}
\item  hiddenshiftallowed := (moodvar = relative);

\item  subjshiftallowed := ((moodvar = whinterrogative) AND
                       (senttypevar = mainclause))             OR
                      ((moodvar = whinterrogative) AND
                       (senttypevar = subordinateclause) AND
                       (conjfound = FALSE))                    OR
                      ((moodvar = relative) AND 
                       (relprofound OR npinshiftfound)   AND
                       (finitenessvar = finite));

\end{itemize}
\section{Utterance}
\subsection{Regular Expression}
The Utterance Rule {\tt UTT} makes utterances out of Noun Phrases, 
Prepositional Phrases, Adjective Phrases and Sentences:
\begin{verbatim}
%UTT

REGULAREXPRESSION:

UTT = NP/1  | PREPP/3 | SENTENCE/2 | ADJP/4
\end{verbatim}
The only conditions for a sentence to become an utterance are that it should be 
finite and have sentencetype {\tt mainclause}.
\subsection{Parameters}
This rule has no parameters
\subsection{auxiliary functions and procedures}
This rule requires no auxiliary functions and procedures
\section{Special effects}
For a number of syntactic constructions, some special measures have been taken 
within the surface rules either because of economical reasons or due to 
limitations of M-parser or S-parser. The following 
subsections describe these measures per phenomena.
\subsection{Relatives}
Within Rosetta, all relative pronouns are of the category RELPRO, except for 
the relative pronoun {\em who}. This pronoun is treated as a WHPRO even in 
relative constructions since it is able to inflect for case as in {\em whom}. 
Since the category RELPRO has no attributes of its own, this case inflection 
can only be dealt with if {\em who} is a WHPRO. \\

The strange thing is that in the M-rules, only one occurence of {\em who}, 
namely the {\em masculine, singular} entry doubles as a relative pronoun. That 
is, in CNMODRELSENT the {\em masculine, singular who } is inserted regardless 
the number and the gender of the head noun:
\begin{quote}
The man who left\\
The woman who left\\
The men who left\\
The women who left
\end{quote}
At first, the solutions seemed to be easy. In the mooddetermination functions
the mood {\em relative } was also assigned to wh-phrases headed by the {\em  
masculine, singular who}. Since English hardly shows any inflectional 
morphology and M-parser (incorrectly!) didn't distinguish between number and 
gender either, at least in relatives, this seemed the most economical solution. However, 
it leaves us with one case of relatives which cannot be correctly analyzed, 
namely present tense plural relatives:
\begin{quote}
The women who swim\\
The man who swim
\end{quote}
Since the VP either receives {\em plural, [1,2,3]} or {\em singular, [1,2]} for 
number and person, {\em masculine,singular, [3] who } is never accepted as 
subject of the wh-sentence, hence the value {\em relative } for mood is never 
assigned. There now seem to be a number of options:
\begin{itemize}
  \item Still use a WHPRO in relative sentences, but add subrules in M-grammar 
to WHMODRELSENT and assign to every wh-sentence, regardless of the features of 
the wh-element, the value {\em relative } for mood. This, however, creates a lot 
of intermediate extra parses in S-parser.
  \item More principled, perhaps, the category RELPRO should be assigned the 
attribute whcase, indicating whether it can show overt case. 
  \item Relax the restrictions on number-agreement
\end{itemize}
We have chosen for the last option. This means that in case of a sentence such 
as {\em Who is swimming} the wh-element is parsed as both singular and plural.
In all other cases, this was already happening anyway due to the deflective 
nature of English inflection.
\subsection{Be}
\subsection{Extraposed Sentences}
\section{Relevant Literature}
\section{supertree}
\begin{verbatim}

-leftdislocrel --> Probably,
-shiftrel
-conjrel
-meltnegrel
-subjrel           He   |-auxrel
-adjunctrel             |-adjunctrel
-auxrel                 |-progauxrel
-adjunctrel             |-passauxrel
-negrel                 |-adjunctrel
-posrel                 |-partrel 
-torel                  |-indobjrel
-predrel/VP-------------|-objrel
-locadvrel              |-adjunctrel
-tempadvrel             |-prepobjrel/predrel/complrel/locargrel/dirargrel
-extraposrel            |-adjunctrel
-postsentadvrel         |-byobjrel


\end{verbatim}
\end{document}
