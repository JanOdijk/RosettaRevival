
\documentstyle{Rosetta}
\begin{document}
   \RosTopic{General}
   \RosTitle{Notulen linguistenbijeenkomst d.d. 29-10-1987}
   \RosAuthor{Harm Smit}
   \RosDocNr{0236}
   \RosDate{2/11/1987}
   \RosStatus{concept}
   \RosSupersedes{-}
   \RosDistribution{Linguists, Joep Rous}
   \RosClearance{Project}
   \RosKeywords{notulen}
   \MakeRosTitle
%
%

\begin{itemize}
  \item {\bf aanwezig}: Andr\'{e} Schenk, 
             Jan Odijk, Franciska de Jong, 
             Elly van Munster, Harm Smit.
  \item {\bf afwezig}: Lisette Appelo, Margreet Sanders.
\end{itemize}

\section{Testwoordenboek}

Franciska heeft de nouns van het testwoordenboek gekregen om te vullen. Lisette
zal binnenkort de verbs krijgen. Op de volgende vergadering zal besproken 
worden hoe de gesloten klassen gevuld worden. Hiervoor hebben Jan en Harm al 
eens een schema gemaakt; dit zal de volgende keer op de vergadering besproken 
worden.

\section{Domein}

Er zijn wat wijzigingen in het domein. Een nieuw attribuut bij CONJ's:
{\em adverbial} om aan te geven dat een CONJ een bijwoordelijke bijzin in kan 
leiden (voorbeeld: {\em hoewel}). Bij PREP zal het tweewaardige attribuut 
{\em preform} worden omgezet in een driewaardig attribuut, met de waarden:
{\em pre, post} en {\em both}. De meeste PREP's (zoals bijv. {\em naar})
zullen de waarde {\em both} krijgen; de PREP {\em met} krijgt de waarde 
{\em pre} en {\em mee} krijgt de waarde {\em post}.

\section {prepV en statusV}

Vormen als {\em inlopen} in een frase als {\em het bos inlopen} krijgen prepV
i.p.v. statusV; dit omdat deze vormen niet als een geheel de morfologie in 
moeten gaan (zoals bij vormen met statusV het geval is), maar als PREP + GLUE + 
V.

\section{Surface-parser}

De surface-parser is zo goed als af; er moeten nog een paar dingen netter en er 
moet nog wat aan {\em polarity} beregeld worden.

\section{Franciska uit eigen werk}

Franciska is verder gegaan met haar overzicht.

\end{document}
