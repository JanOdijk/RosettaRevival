\documentstyle{Rosetta}
\begin{document}
   \RosTopic{Rosetta3.doc.morphology}
   \RosTitle{Rosetta3 Dutch Morphology, rules}
   \RosAuthor{Harm Smit}
   \RosDocNr{331}
   \RosDate{\today}
   \RosStatus{concept}
   \RosSupersedes{-}
   \RosDistribution{Project}
   \RosClearance{Project}
   \RosKeywords{Dutch, morphology, rules, documentation}
   \MakeRosTitle

%DUTCH1
%&

\section{Rules for verbs.}

Note:
\begin{itemize}
 \item Irregular verbs have class 1 and 2 (and not --as they had in ROSETTA2-- 
       class 0).
 \item Class 0 has been used for verbs without conjugation: these verbs have 
       only {\em infinitive}-form (and forms derived of infinitives).
 \item Even class-numbers (without 0 and 14) correspond with verbs {\em without}
       "ge"-prefix in the past participle, odd class-numbers 
       correspond with verbs {\em with} "ge"-prefix.
 \item The function "copy...." copies the values of the --in the rule not 
       explicitely mentioned-- fields of the old record to those of the new one.
 \item Conjugationclass 14 has been used for verbs with extra form for the
       1st person singular.
\end{itemize}

\subsection{Rules for derivation.}
\subsubsection{Rule for particle as prefix.}

The M-grammar should not yield forms of the type PART-V $ + $ SUBJ 
({\em *afvraag ik me?}, etc.). 
In this rule this is not done (because the morphology only handles
single words). Therefore, in generation also the set [0,1] is yielded for the
attribute persons, although person$=$0 can never be combined with particles:
{\em *afvraag jij je?, *wegga jij?,} etc.

The test on `eORenForm' is necessary because the {\em order} of this rule and 
the rule `VerbeORenvorm' should be stated.

(examples: {\em afgevraagd(e(n)), afvragen(d(e(n))), afvraag, afvraagt, ...})
\begin{verbatim}
%ParticleToVerb

m1: PART{PARTrec1}
m2: VERB{VERBrec1} 
        [ head/ SUBVERB{SUBVERBrec1} 
              [ head/ BVERB{BVERBrec1}] 
        ]
m:  VERB{VERBrec2} 
        [ head/ SUBVERB{SUBVERBrec2} 
              [ partrel/ PART{PARTrec1},
                head/ SUBVERB{SUBVERBrec1} [head/ BVERB{BVERBrec1}] 
              ]
        ]

comp:
      C1: (PARTrec1.key = VERBrec1.particle)
          and (VERBrec1.status = bareV)
          and (VERBrec1.eORenForm = NoForm)
          
      A1: SUBVERBrec2     := SUBVERBrec1;
          SUBVERBrec2.lastaffix := partaffix;
          VERBrec2        := VERBrec1;
          VERBrec2.status := partV

decomp: 
      C1: (PARTrec1.key = VERBrec2.particle)
          and (VERBrec2.status = partV)
          and (SUBVERBrec2.lastaffix = partaffix)
          and (VERBrec2.eORenForm = NoForm)
      A1: VERBrec1 := VERBrec2;
          VERBrec1.status := bareV
&
\end{verbatim}
\newpage
\subsubsection{Rule for verbs without derivation.}
\begin{verbatim}
%VerbBtoSub

m1:   BVERB{BVERBrec1}
m:    SUBVERB{SUBVERBrec1} [head/ BVERB{BVERBrec1}]

comp:        
         C:  true
         A:  SUBVERBrec1 := COPYT_bverbtosubverb(BVERBrec1)

decomp:      
         C: true
         A:  @
&
\end{verbatim}
\newpage
\subsection{Inflection: weak and strong verbs}

Note: for irregular verbs see next {\em section}.

\subsubsection{Rule for present tense "0"th and 1st person singular.}

\begin{verbatim}
%VerbOttEnk1

m1:   SUBVERB{SUBVERBrec1}[mu1]
m:    VERB{VERBrec1} [head/ SUBVERB{SUBVERBrec1}[mu1]]

comp:        
         C: [3,4,5,6,7,8,9,10,11,12,13,15,16] * SUBVERBrec1.conjclasses <> []
         A: VERBrec1           := COPYT_subverbtoverb(SUBVERBrec1);
            VERBrec1.tense     := presenttense;
            VERBrec1.modus     := indicative;
            VERBrec1.number    := singular;
            VERBrec1.persons   := [1,0];
            VERBrec1.eORenForm := NoForm

decomp:      
         C: (VERBrec1.conjclasses * [3,4,5,6,7,8,9,10,11,12,13,15,16] <> [])
            and (VERBrec1.tense     = presenttense)                  
            and (VERBrec1.modus     = indicative)                    
            and (VERBrec1.number    = singular)                      
            and ([1,0] * VERBrec1.persons <> [])
            and (VERBrec1.eORenForm = NoForm) 
         A: @
&
\end{verbatim}
\newpage
\subsubsection{Rule for extra form 0th and 1st person sg. for class 14.}
This rule makes for instance ``snij'' out of the stem ``snijd'', and ``hou'' 
out of ``houd''.

It works for the following verbs: *glij, *rij, *snij, verblij, *hou 
(and maybe for: bevrij, kastij ?)

\begin{verbatim}
%VerbOttEnk1extra

m1:   SUBVERB{SUBVERBrec1}[mu1]
m2:   SFCAT{SFCATrec1}
m:    VERB{VERBrec1} [head/ SUBVERB{SUBVERBrec1}[mu1]]

comp:        
         C: (SUBVERBrec1.conjclasses * [14] <> [])
            and (SFCATrec1.key = SFKdelD)
         A: VERBrec1           := COPYT_subverbtoverb(SUBVERBrec1);
            VERBrec1.tense     := presenttense;
            VERBrec1.modus     := indicative;
            VERBrec1.number    := singular;
            VERBrec1.persons   := [0,1];
            VERBrec1.eORenForm := NoForm

decomp:      
          C: (VERBrec1.conjclasses * [14] <> []) 
            and (VERBrec1.tense     = presenttense)
            and (VERBrec1.modus     = indicative)
            and (VERBrec1.number    = singular)
            and ([0,1] * VERBrec1.persons <> [])
            and (VERBrec1.eORenForm = NoForm)
         A: SFCATrec1.key  := SFKdelD
&
\end{verbatim}
\newpage
\subsubsection{Rule for present tense 2nd, 3rd, 4th and 5th person singular.}
\begin{verbatim}
%VerbOttEnk2

m1:   SUBVERB{SUBVERBrec1}[mu1]
m2:   SFCAT{SFCATrec1}
m:    VERB{VERBrec1} [head/ SUBVERB{SUBVERBrec1}[mu1]]

comp:        
         C: (SUBVERBrec1.conjclasses * [3,4,5,6,7,8,9,10,11,12,13,15,16] <> [])
            and (SFCATrec1.key = SFKt)
         A: VERBrec1           := COPYT_subverbtoverb(SUBVERBrec1);
            VERBrec1.tense     := presenttense;
            VERBrec1.modus     := indicative;
            VERBrec1.number    := singular;
            VERBrec1.persons   := [2,3,4,5];
            VERBrec1.eORenForm := NoForm

decomp:      
          C: (VERBrec1.conjclasses * [3,4,5,6,7,8,9,10,11,12,13,15,16] <> []) 
            and (VERBrec1.tense     = presenttense)
            and (VERBrec1.modus     = indicative)
            and (VERBrec1.number    = singular)
            and ([2,3,4,5] * VERBrec1.persons <> [])
            and (VERBrec1.eORenForm = NoForm)
         A: SFCATrec1.key  := SFKt
&
\end{verbatim}
\newpage
\subsubsection{Rule for present tense plural.}
\begin{verbatim}
%VerbOttMv

m1:   SUBVERB{SUBVERBrec1}[mu1]
m2:   SFCAT{SFCATrec1}
m:    VERB{VERBrec1} [head/ SUBVERB{SUBVERBrec1}[mu1]]

comp:
         C1: (SUBVERBrec1.conjclasses * [3,4,5,6,7,8,9,10,11,12,13,15,16] <> [])
           C2: (SFCATrec1.key = SFKen)
           A2: VERBrec1          := COPYT_subverbtoverb(SUBVERBrec1);
               VERBrec1.persons  := [1,2,3]
           C2: (SFCATrec1.key = SFKt)
           A2: VERBrec1          := COPYT_subverbtoverb(SUBVERBrec1);
               VERBrec1.persons  := [4,5]
         A1: VERBrec1.tense     := presenttense;
             VERBrec1.modus     := indicative;
             VERBrec1.number    := plural;
             VERBrec1.eORenForm := NoForm

decomp:
         C1: (VERBrec1.conjclasses * [3,4,5,6,7,8,9,10,11,12,13,15,16] <> [])
             and (VERBrec1.tense     = presenttense) 
             and (VERBrec1.modus     = indicative)
             and (VERBrec1.number    = plural)
             and (VERBrec1.eORenForm = NoForm)
           C2: [1,2,3] * VERBrec1.persons <> []
           A2:  SFCATrec1.key   := SFKen
           C2: [4,5] * VERBrec1.persons <> []
           A2:  SFCATrec1.key   := SFKt
         A1: @
&
\end{verbatim}
\newpage
\subsubsection{Rule for past tense 1st, 2nd, 3rd and 4th person singular.}

Note: this rule also treats the 5th
         person singular for the classes 3, 4, 15, 16.

\begin{verbatim}
%VerbOvtEnk

m1:   SUBVERB{SUBVERBrec1}[mu1]
m2:   SFCAT{SFCATrec1}
m:    VERB{VERBrec1} [head/ SUBVERB{SUBVERBrec1}[mu1]]

comp:
         C1:  true
           C2:  (SUBVERBrec1.conjclasses * [3,4] <> []) 
                and (SFCATrec1.key = SFKdete)
           A2:  VERBrec1          := COPYT_subverbtoverb(SUBVERBrec1);
                VERBrec1.persons  := [1,2,3,4,5]
           C2:  (SUBVERBrec1.conjclasses * [5,6] <> []) 
                and (SFCATrec1.key = SFKovt1)
           A2:  VERBrec1          := COPYT_subverbtoverb(SUBVERBrec1);
                VERBrec1.persons  := [1,2,3,4]
           C2:  (SUBVERBrec1.conjclasses * [7,8] <> [])
                and (SFCATrec1.key = SFKovtvd1)
           A2:  VERBrec1          := COPYT_subverbtoverb(SUBVERBrec1);
                VERBrec1.persons  := [1,2,3,4]
           C2:  (SUBVERBrec1.conjclasses * [9,10] <> [])
                and (SFCATrec1.key = SFKovtvd2)
           A2:  VERBrec1          := COPYT_subverbtoverb(SUBVERBrec1);
                VERBrec1.persons  := [1,2,3,4]
           C2:  (SUBVERBrec1.conjclasses * [11,12] <> []) 
                and (SFCATrec1.key = SFKovt2)
           A2:  VERBrec1          := COPYT_subverbtoverb(SUBVERBrec1);
                VERBrec1.persons  := [1,2,3,4]
           C2:  (13 in SUBVERBrec1.conjclasses) 
                and (SFCATrec1.key = SFKovt3)
           A2:  VERBrec1          := COPYT_subverbtoverb(SUBVERBrec1);
                VERBrec1.persons  := [1,2,3,4]
           C2:  (SUBVERBrec1.conjclasses * [15,16] <> []) 
                and (SFCATrec1.key = SFKdete)
           A2:  VERBrec1          := COPYT_subverbtoverb(SUBVERBrec1);
                VERBrec1.persons  := [1,2,3,4,5]
         A1: VERBrec1.tense     := pasttense;
             VERBrec1.modus     := indicative;
             VERBrec1.number    := singular;
             VERBrec1.eORenForm := NoForm

decomp:
         C1: (VERBrec1.tense         = pasttense) 
             and (VERBrec1.modus     = indicative) 
             and (VERBrec1.number    = singular)
             and (VERBrec1.eORenForm = NoForm)
           C2: ([1,2,3,4,5] * VERBrec1.persons <> [])
                and (VERBrec1.conjclasses * [3,4] <> [])
           A2:  SFCATrec1.key := SFKdete
           C2: ([1,2,3,4] * VERBrec1.persons <> [])
                and (VERBrec1.conjclasses * [5,6] <> [])
           A2:  SFCATrec1.key := SFKovt1
           C2: ([1,2,3,4] * VERBrec1.persons <> [])
                and (VERBrec1.conjclasses * [7,8] <> [])
           A2:  SFCATrec1.key := SFKovtvd1
           C2: ([1,2,3,4] * VERBrec1.persons <> [])
                and (VERBrec1.conjclasses * [9,10] <> [])
           A2:  SFCATrec1.key := SFKovtvd2
           C2: ([1,2,3,4] * VERBrec1.persons <> [])
                and (VERBrec1.conjclasses * [11,12] <> [])
           A2:  SFCATrec1.key := SFKovt2
           C2: ([1,2,3,4] * VERBrec1.persons <> [])
                and (13 in VERBrec1.conjclasses)
           A2:  SFCATrec1.key := SFKovt3
           C2: ([1,2,3,4,5] * VERBrec1.persons <> [])
                and (VERBrec1.conjclasses * [15,16] <> [])
           A2:  SFCATrec1.key := SFKdete
         A1: @
&
\end{verbatim}
\newpage
\subsubsection{Extra rule for past tense singular for the 5th person sg.}

Note: this rule treats the past tense 5th person singular for the 
classes 5, 6, 7, 8, 9, 10, 11, 12, 13.

\begin{verbatim}
%VerbOvtEnkPers5extra

m1:   SUBVERB{SUBVERBrec1}[mu1]
m2:   SFCAT{SFCATrec1}
m3:   SFCAT{SFCATrec2}
m:    VERB{VERBrec1} [head/ SUBVERB{SUBVERBrec1}[mu1]]

comp:
         C1: SFCATrec2.key = SFKt
           C2:  (SUBVERBrec1.conjclasses * [5,6] <> [])
                and (SFCATrec1.key = SFKovt1)
           A2:  VERBrec1          := COPYT_subverbtoverb(SUBVERBrec1)
           C2:  (SUBVERBrec1.conjclasses * [7,8] <> [])
                and (SFCATrec1.key = SFKovtvd1)
           A2:  VERBrec1          := COPYT_subverbtoverb(SUBVERBrec1)
           C2:  (SUBVERBrec1.conjclasses * [9,10] <> [])
                and (SFCATrec1.key = SFKovtvd2)
           A2:  VERBrec1          := COPYT_subverbtoverb(SUBVERBrec1)
           C2:  (SUBVERBrec1.conjclasses * [11,12] <> []) 
                and (SFCATrec1.key = SFKovt2)
           A2:  VERBrec1          := COPYT_subverbtoverb(SUBVERBrec1)
           C2:  (13 in SUBVERBrec1.conjclasses) 
                and (SFCATrec1.key = SFKovt3)
           A2:  VERBrec1          := COPYT_subverbtoverb(SUBVERBrec1)
         A1: VERBrec1.tense     := pasttense;
             VERBrec1.modus     := indicative;
             VERBrec1.number    := singular;
             VERBrec1.persons   := [5];
             VERBrec1.eORenForm := NoForm

decomp:
         C1: (VERBrec1.tense        = pasttense) 
             and (VERBrec1.modus    = indicative)
             and (VERBrec1.number   = singular)
             and (5 in VERBrec1.persons)
             and (VERBrec1.eORenForm = NoForm)
           C2:  VERBrec1.conjclasses * [5,6]   <> []
           A2:  SFCATrec1.key := SFKovt1
           C2:  VERBrec1.conjclasses * [7,8]   <> []
           A2:  SFCATrec1.key := SFKovtvd1
           C2:  VERBrec1.conjclasses * [9,10]  <> []
           A2:  SFCATrec1.key := SFKovtvd2
           C2:  VERBrec1.conjclasses * [11,12] <> []
           A2:  SFCATrec1.key := SFKovt2
           C2:  13 in VERBrec1.conjclasses
           A2:  SFCATrec1.key := SFKovt3
         A1: SFCATrec2.key := SFKt
&
\end{verbatim}
\newpage
\subsubsection{Rule for past tense plural.}

Note: Rule for past tense 1st, 2nd and 3rd person, as well as the 5th person 
      for the classes  5, 6, 7, 8, 9, 10, 11, 12, 13. 

\begin{verbatim}
%VerbOvtMv 

m1:   SUBVERB{SUBVERBrec1}[mu1]
m2:   SFCAT{SFCATrec1}
m3:   SFCAT{SFCATrec2}
m:    VERB{VERBrec1} [head/ SUBVERB{SUBVERBrec1}[mu1]]

comp:
         C1: true
           C2:  SFCATrec2.key = SFKen
             C3:  (SUBVERBrec1.conjclasses * [3,4] <> []) 
                  and (SFCATrec1.key = SFKdete)
             A3:  VERBrec1          := COPYT_subverbtoverb(SUBVERBrec1)
             C3:  (SUBVERBrec1.conjclasses * [5,6] <> []) 
                  and (SFCATrec1.key = SFKovt1)
             A3:  VERBrec1          := COPYT_subverbtoverb(SUBVERBrec1)
             C3:  (SUBVERBrec1.conjclasses * [7,8] <> [])
                  and (SFCATrec1.key = SFKovtvd1)
             A3:  VERBrec1          := COPYT_subverbtoverb(SUBVERBrec1)
             C3:  (SUBVERBrec1.conjclasses * [9,10] <> []) 
                  and (SFCATrec1.key = SFKovtvd2)
             A3:  VERBrec1          := COPYT_subverbtoverb(SUBVERBrec1)
             C3:  (SUBVERBrec1.conjclasses * [11,12] <> [])
                  and (SFCATrec1.key = SFKovt2)
             A3:  VERBrec1          := COPYT_subverbtoverb(SUBVERBrec1)
             C3:  (13 in SUBVERBrec1.conjclasses) 
                  and (SFCATrec1.key = SFKovt3)
             A3:  VERBrec1          := COPYT_subverbtoverb(SUBVERBrec1)
             C3:  (SUBVERBrec1.conjclasses * [15,16] <> []) 
                  and (SFCATrec1.key = SFKdete)
             A3:  VERBrec1          := COPYT_subverbtoverb(SUBVERBrec1)
           A2:  VERBrec1.persons  := [1,2,3]
           C2:  SFCATrec2.key = SFKt
             C3:  (SUBVERBrec1.conjclasses * [5,6] <> []) 
                  and (SFCATrec1.key = SFKovt1)
             A3:  VERBrec1          := COPYT_subverbtoverb(SUBVERBrec1)
             C3:  (SUBVERBrec1.conjclasses * [7,8] <> []) 
                  and (SFCATrec1.key = SFKovtvd1)
             A3:  VERBrec1          := COPYT_subverbtoverb(SUBVERBrec1)
             C3:  (SUBVERBrec1.conjclasses * [9,10] <> [])
                  and (SFCATrec1.key = SFKovtvd2)
             A3:  VERBrec1          := COPYT_subverbtoverb(SUBVERBrec1)
             C3:  (SUBVERBrec1.conjclasses * [11,12] <> []) 
                  and (SFCATrec1.key = SFKovt2)
             A3:  VERBrec1          := COPYT_subverbtoverb(SUBVERBrec1)
             C3:  (13 in SUBVERBrec1.conjclasses) 
                  and (SFCATrec1.key = SFKovt3)
             A3:  VERBrec1          := COPYT_subverbtoverb(SUBVERBrec1)
           A2:  VERBrec1.persons  := [5]
         A1: VERBrec1.tense     := pasttense;
             VERBrec1.modus     := indicative;
             VERBrec1.number    := plural;
             VERBrec1.eORenForm := NoForm

decomp:
         C1: (VERBrec1.tense         = pasttense) 
             and (VERBrec1.modus     = indicative) 
             and (VERBrec1.number    = plural)     
             and (VERBrec1.eORenForm = NoForm)
           C2: [1,2,3] * VERBrec1.persons <> []
             C3:  VERBrec1.conjclasses * [3,4] <> []
             A3:  SFCATrec1.key := SFKdete
             C3:  VERBrec1.conjclasses * [5,6] <> []
             A3:  SFCATrec1.key := SFKovt1
             C3:  VERBrec1.conjclasses * [7,8] <> []
             A3:  SFCATrec1.key := SFKovtvd1
             C3:  VERBrec1.conjclasses * [9,10] <> []
             A3:  SFCATrec1.key := SFKovtvd2
             C3:  VERBrec1.conjclasses * [11,12] <> []
             A3:  SFCATrec1.key := SFKovt2
             C3:  13 in VERBrec1.conjclasses
             A3:  SFCATrec1.key := SFKovt3
             C3:  VERBrec1.conjclasses * [15,16] <> []
             A3:  SFCATrec1.key := SFKdete
           A2: SFCATrec2.key := SFKen
           C2:  5 in VERBrec1.persons 
             C3:  VERBrec1.conjclasses * [5,6] <> []
             A3:  SFCATrec1.key := SFKovt1
             C3:  VERBrec1.conjclasses * [7,8] <> []
             A3:  SFCATrec1.key := SFKovtvd1
             C3:  VERBrec1.conjclasses * [9,10] <> []
             A3:  SFCATrec1.key := SFKovtvd2
             C3:  VERBrec1.conjclasses * [11,12] <> []
             A3:  SFCATrec1.key := SFKovt2
             C3:  13 in VERBrec1.conjclasses
             A3:  SFCATrec1.key := SFKovt3
           A2:  SFCATrec2.key := SFKt
         A1:  @
&
\end{verbatim}
\newpage
\subsubsection{Extra rule for past tense plural.}

Note: This rule treats the past form for 4th person plural, as well as the
5th person for the classes  3, 4, 15, 16. 

\begin{verbatim}
%VerbOvtMvPers4en5extra

m1:   SUBVERB{SUBVERBrec1}[mu1]
m2:   SFCAT{SFCATrec1}
m:    VERB{VERBrec1} [head/ SUBVERB{SUBVERBrec1}[mu1]]

comp:
         C1:  true
           C2:  (SUBVERBrec1.conjclasses * [3,4] <> [])
                and (SFCATrec1.key = SFKdete)
           A2:  VERBrec1          := COPYT_subverbtoverb(SUBVERBrec1);
                VERBrec1.persons  := [4,5]
           C2:  (SUBVERBrec1.conjclasses * [5,6] <> [])
                and (SFCATrec1.key = SFKovt1)
           A2:  VERBrec1          := COPYT_subverbtoverb(SUBVERBrec1);
                VERBrec1.persons  := [4]
           C2:  (SUBVERBrec1.conjclasses * [7,8] <> [])
                and (SFCATrec1.key = SFKovtvd1)
           A2:  VERBrec1          := COPYT_subverbtoverb(SUBVERBrec1);
                VERBrec1.persons  := [4]
           C2:  (SUBVERBrec1.conjclasses * [9,10] <> [])
                and (SFCATrec1.key = SFKovtvd2)
           A2:  VERBrec1          := COPYT_subverbtoverb(SUBVERBrec1);
                VERBrec1.persons  := [4]
           C2:  (SUBVERBrec1.conjclasses * [11,12] <> [])
                and (SFCATrec1.key = SFKovt2)
           A2:  VERBrec1          := COPYT_subverbtoverb(SUBVERBrec1);
                VERBrec1.persons  := [4]
           C2:  (13 in SUBVERBrec1.conjclasses) 
                and (SFCATrec1.key = SFKovt3)
           A2:  VERBrec1          := COPYT_subverbtoverb(SUBVERBrec1);
                VERBrec1.persons  := [4]
           C2:  (SUBVERBrec1.conjclasses * [15,16] <> []) 
                and (SFCATrec1.key = SFKdete)
           A2:  VERBrec1          := COPYT_subverbtoverb(SUBVERBrec1);
                VERBrec1.persons  := [4,5]
         A1: VERBrec1.tense     := pasttense;
             VERBrec1.modus     := indicative;
             VERBrec1.number    := plural;
             VERBrec1.eORenForm := NoForm

decomp:
         C1: (VERBrec1.tense         = pasttense) 
             and (VERBrec1.modus     = indicative) 
             and (VERBrec1.number    = plural) 
             and (VERBrec1.eORenForm = NoForm)
           C2: ([4,5] * VERBrec1.persons <> [])
                and (VERBrec1.conjclasses * [3,4] <> [])
           A2:  SFCATrec1.key := SFKdete
           C2: (4 in VERBrec1.persons)
                and (VERBrec1.conjclasses * [5,6] <> [])
           A2:  SFCATrec1.key := SFKovt1
           C2: (4 in VERBrec1.persons)
                and (VERBrec1.conjclasses * [7,8] <> [])
           A2:  SFCATrec1.key := SFKovtvd1
           C2: (4 in VERBrec1.persons)
                and (VERBrec1.conjclasses * [9,10] <> [])
           A2:  SFCATrec1.key := SFKovtvd2
           C2: (4 in VERBrec1.persons)
                and (VERBrec1.conjclasses * [11,12] <> [])
           A2:  SFCATrec1.key := SFKovt2
           C2: (4 in VERBrec1.persons)
                and (13 in VERBrec1.conjclasses)
           A2:  SFCATrec1.key := SFKovt3
           C2: ([4,5] * VERBrec1.persons <> [])
                and (VERBrec1.conjclasses * [15,16] <> [])
           A2:  SFCATrec1.key := SFKdete
         A1:  @
&
\end{verbatim}
\newpage
\subsubsection{Rule for past participle (with "ge") for the classes  3, 5, 9.}

Note: In this rule, the past participle is formed by one prefix and one
suffix; the suffix handles ablaut or inflectional ending.

\begin{verbatim}
%VerbVd1 

m1:   PFCAT{PFCATrec1}
m2:   SUBVERB{SUBVERBrec1}[mu1]
m3:   SFCAT{SFCATrec1}
m:    VERB{VERBrec1} [head/ SUBVERB{SUBVERBrec1}[mu1]]

comp:
         C1: PFCATrec1.key = PFKge 
           C2: (3 in SUBVERBrec1.conjclasses) 
               and (SFCATrec1.key = SFKdt) 
           A2: VERBrec1  := COPYT_subverbtoverb(SUBVERBrec1)
           C2: (5 in SUBVERBrec1.conjclasses)
               and (SFCATrec1.key = SFKen) 
           A2: VERBrec1  := COPYT_subverbtoverb(SUBVERBrec1)
           C2: (9 in SUBVERBrec1.conjclasses)
               and (SFCATrec1.key = SFKovtvd2) 
           A2: VERBrec1  := COPYT_subverbtoverb(SUBVERBrec1)
         A1: VERBrec1.tense     := omegatense;
             VERBrec1.modus     := pastpart;
             VERBrec1.number    := omeganumber;
             VERBrec1.persons   := [];
             VERBrec1.eORenForm := NoForm

decomp:
         C1: (VERBrec1.tense         = omegatense) 
             and (VERBrec1.modus     = pastpart)
             and (VERBrec1.eORenForm = NoForm) 
             and (VERBrec1.persons   = [])
             and (VERBrec1.number    = omeganumber)
           C2:  3 in VERBrec1.conjclasses
           A2:  SFCATrec1.key := SFKdt
           C2:  5 in VERBrec1.conjclasses
           A2:  SFCATrec1.key := SFKen
           C2:  9 in VERBrec1.conjclasses 
           A2:  SFCATrec1.key := SFKovtvd2
         A1: PFCATrec1.key := PFKge
&
\end{verbatim}
\newpage
\subsubsection{Rule for past participle (with "ge") for the classes 7, 11, 15.}

Note: In this rule, the past participle is formed by one prefix and two
suffixes; one suffixe handles ablaut, the other "-en"-ending.

\begin{verbatim}
%VerbVd2 

m1:   PFCAT{PFCATrec1}
m2:   SUBVERB{SUBVERBrec1}[mu1]
m3:   SFCAT{SFCATrec1}
m4:   SFCAT{SFCATrec2}
m:    VERB{VERBrec1} [head/ SUBVERB{SUBVERBrec1}[mu1]]

comp:
         C1: (PFCATrec1.key = PFKge) 
             and (SFCATrec2.key = SFKen)
           C2: (7 in SUBVERBrec1.conjclasses)
               and (SFCATrec1.key = SFKovtvd1) 
           A2: VERBrec1  := COPYT_subverbtoverb(SUBVERBrec1)
           C2: (11 in SUBVERBrec1.conjclasses)
               and (SFCATrec1.key = SFKvd1) 
           A2: VERBrec1  := COPYT_subverbtoverb(SUBVERBrec1)
           C2: (15 in SUBVERBrec1.conjclasses) 
               and (SFCATrec1.key = SFKvd2) 
           A2: VERBrec1  := COPYT_subverbtoverb(SUBVERBrec1)
         A1: VERBrec1.tense     := omegatense;
             VERBrec1.modus     := pastpart;
             VERBrec1.number    := omeganumber;
             VERBrec1.persons   := [];
             VERBrec1.eORenForm := NoForm

decomp:
         C1: (VERBrec1.tense         = omegatense) 
             and (VERBrec1.modus     = pastpart)
             and (VERBrec1.eORenForm = NoForm) 
             and (VERBrec1.persons   = [])
             and (VERBrec1.number    = omeganumber)
           C2: 7 in VERBrec1.conjclasses
           A2: SFCATrec1.key := SFKovtvd1 
           C2: 11 in VERBrec1.conjclasses 
           A2: SFCATrec1.key := SFKvd1
           C2: 15 in VERBrec1.conjclasses 
           A2: SFCATrec1.key := SFKvd2 
         A1: PFCATrec1.key   := PFKge;
             SFCATrec2.key   := SFKen
&
\end{verbatim}
\newpage
\subsubsection{Rule for past participle (without "ge") for the classes  4, 6, 10.}

Note: In this rule, the past participle is formed by one suffix only; the 
suffix handles ablaut or inflectional ending.

\begin{verbatim}
%VerbVd3 

m1:   SUBVERB{SUBVERBrec1}[mu1]
m2:   SFCAT{SFCATrec1}
m:    VERB{VERBrec1} [head/ SUBVERB{SUBVERBrec1}[mu1]]

comp:
         C1: true
           C2: (4 in SUBVERBrec1.conjclasses)
               and (SFCATrec1.key = SFKdt) 
           A2: VERBrec1  := COPYT_subverbtoverb(SUBVERBrec1)
           C2: (6 in SUBVERBrec1.conjclasses)
               and (SFCATrec1.key = SFKen) 
           A2: VERBrec1  := COPYT_subverbtoverb(SUBVERBrec1)
           C2: (10 in SUBVERBrec1.conjclasses)
               and (SFCATrec1.key = SFKovtvd2) 
           A2: VERBrec1  := COPYT_subverbtoverb(SUBVERBrec1)
         A1: VERBrec1.tense     := omegatense;
             VERBrec1.modus     := pastpart;
             VERBrec1.number    := omeganumber;
             VERBrec1.persons   := [];
             VERBrec1.eORenForm := NoForm

decomp:
         C1: (VERBrec1.tense         = omegatense) 
             and (VERBrec1.modus     = pastpart)
             and (VERBrec1.number    = omeganumber)
             and (VERBrec1.persons   = []) 
             and (VERBrec1.eORenForm = NoForm) 
           C2:  4 in VERBrec1.conjclasses 
           A2:  SFCATrec1.key := SFKdt
           C2:  6 in VERBrec1.conjclasses
           A2:  SFCATrec1.key := SFKen
           C2:  10 in VERBrec1.conjclasses
           A2:  SFCATrec1.key := SFKovtvd2
         A1: @
&
\end{verbatim}
\newpage
\subsubsection{Rule for past participle (without "ge") for the classes  8, 12, 16.}

Note: In this rule, the past participle is formed by two suffixes; one suffix
handles ablaut, the other the "-en"-ending.

\begin{verbatim}
%VerbVd4 

m1:   SUBVERB{SUBVERBrec1}[mu1]
m2:   SFCAT{SFCATrec1}
m3:   SFCAT{SFCATrec2}
m:    VERB{VERBrec1} [head/ SUBVERB{SUBVERBrec1}[mu1]]

comp:
         C1: SFCATrec2.key = SFKen
           C2: (8 in SUBVERBrec1.conjclasses)
               and (SFCATrec1.key = SFKovtvd1) 
           A2: VERBrec1  := COPYT_subverbtoverb(SUBVERBrec1)
           C2: (12 in SUBVERBrec1.conjclasses) 
               and (SFCATrec1.key = SFKvd1) 
           A2: VERBrec1  := COPYT_subverbtoverb(SUBVERBrec1)
           C2: (16 in SUBVERBrec1.conjclasses) 
               and (SFCATrec1.key = SFKvd2) 
           A2: VERBrec1  := COPYT_subverbtoverb(SUBVERBrec1)
         A1: VERBrec1.tense     := omegatense;
             VERBrec1.modus     := pastpart;
             VERBrec1.number    := omeganumber;
             VERBrec1.persons   := [];
             VERBrec1.eORenForm := NoForm

decomp:
         C1: (VERBrec1.tense         = omegatense) 
             and (VERBrec1.modus     = pastpart)
             and (VERBrec1.number    = omeganumber)
             and (VERBrec1.persons   = []) 
             and (VERBrec1.eORenForm = NoForm)
           C2: 8 in VERBrec1.conjclasses 
           A2: SFCATrec1.key := SFKovtvd1 
           C2: 12 in VERBrec1.conjclasses 
           A2: SFCATrec1.key := SFKvd1
           C2: 16 in VERBrec1.conjclasses 
           A2: SFCATrec1.key := SFKvd2 
         A1: SFCATrec2.key := SFKen
&
\end{verbatim}
\newpage
\subsubsection{Rule for the infinitive.}

Note: verbs of class 0 do have an infinitive-form.

\begin{verbatim}
%VerbInf 

m1:   SUBVERB{SUBVERBrec1}[mu1]
m2:   SFCAT{SFCATrec1}
m:    VERB{VERBrec1} [head/ SUBVERB{SUBVERBrec1}[mu1]]

comp:
         C:  (SUBVERBrec1.conjclasses * [0,3,4,5,6,7,8,9,10,11,12,13,15,16] <> 
             []) and (SFCATrec1.key = SFKen)
         A:  VERBrec1           := COPYT_subverbtoverb(SUBVERBrec1);
             VERBrec1.tense     := omegatense;
             VERBrec1.modus     := infinitive;
             VERBrec1.number    := omeganumber;
             VERBrec1.persons   := [];
             VERBrec1.eORenForm := NoForm

decomp:
         C:  (VERBrec1.conjclasses * [0,3,4,5,6,7,8,9,10,11,12,13,15,16] <> [])
             and (VERBrec1.tense     = omegatense) 
             and (VERBrec1.modus     = infinitive)
             and (VERBrec1.number    = omeganumber)
             and (VERBrec1.persons   = []) 
             and (VERBrec1.eORenForm = NoForm) 
         A:  SFCATrec1.key     := SFKen
&
\end{verbatim}
\newpage
\subsubsection{Rule for present participle.}

Note: verbs of class 0 do have a present participle.

\begin{verbatim}
%VerbTd 

m1:   SUBVERB{SUBVERBrec1}[mu1]
m2:   SFCAT{SFCATrec1}
m3:   SFCAT{SFCATrec2}
m:    VERB{VERBrec1} [head/ SUBVERB{SUBVERBrec1}[mu1]]

comp:
         C:  (SUBVERBrec1.conjclasses * [0,3,4,5,6,7,8,9,10,11,12,13,15,16] <> 
             []) 
             and (SFCATrec1.key = SFKen) 
             and (SFCATrec2.key = SFKdt)
         A:  VERBrec1           := COPYT_subverbtoverb(SUBVERBrec1);
             VERBrec1.tense     := omegatense;
             VERBrec1.modus     := prespart;
             VERBrec1.number    := omeganumber;
             VERBrec1.persons   := [];
             VERBrec1.eORenForm := NoForm

decomp:
         C:  (VERBrec1.conjclasses * [0,3,4,5,6,7,8,9,10,11,12,13,15,16] <> [])
             and (VERBrec1.tense     = omegatense) 
             and (VERBrec1.modus     = prespart)
             and (VERBrec1.number    = omeganumber)
             and (VERBrec1.persons   = []) 
             and (VERBrec1.eORenForm = NoForm)
         A:  SFCATrec1.key     := SFKen;
             SFCATrec2.key     := SFKdt
&
\end{verbatim}
\newpage
\subsubsection{Rule for imperative singular.}
\begin{verbatim}
%VerbGbEnk 

m1:   SUBVERB{SUBVERBrec1}[mu1]
m:    VERB{VERBrec1} [head/ SUBVERB{SUBVERBrec1}[mu1]]

comp:
         C:  SUBVERBrec1.conjclasses * [3,4,5,6,7,8,9,10,11,12,13,15,16] <> []
         A:  VERBrec1           := COPYT_subverbtoverb(SUBVERBrec1);
             VERBrec1.tense     := omegatense;
             VERBrec1.modus     := imperative;
             VERBrec1.number    := singular;
             VERBrec1.persons   := [];
             VERBrec1.eORenForm := NoForm

decomp:
         C:  (VERBrec1.conjclasses * [3,4,5,6,7,8,9,10,11,12,13,15,16] <> [])
             and (VERBrec1.tense     = omegatense) 
             and (VERBrec1.modus     = imperative)
             and (VERBrec1.number    = singular) 
             and (VERBrec1.persons   = [])
             and (VERBrec1.eORenForm = NoForm)
         A:  @
&
\end{verbatim}
\newpage
\subsubsection{Rule for extra form imperative singular for class 14.}
This rule makes for instance ``snij'' out of the stem ``snijd'', and ``hou'' 
out of ``houd''.

It works for the following verbs: *glij, *rij, *snij, verblij, *hou 
(and maybe for: bevrij, kastij ?)

\begin{verbatim}
%VerbGbEnkExtra

m1:   SUBVERB{SUBVERBrec1}[mu1]
m2:   SFCAT{SFCATrec1}
m:    VERB{VERBrec1} [head/ SUBVERB{SUBVERBrec1}[mu1]]

comp:        
         C: (SUBVERBrec1.conjclasses * [14] <> [])
            and (SFCATrec1.key = SFKdelD)
         A: VERBrec1           := COPYT_subverbtoverb(SUBVERBrec1);
            VERBrec1.tense     := omegatense;
            VERBrec1.modus     := imperative;
            VERBrec1.number    := singular;
            VERBrec1.persons   := [];
            VERBrec1.eORenForm := NoForm

decomp:      
         C: (VERBrec1.conjclasses * [14] <> []) 
            and (VERBrec1.tense     = omegatense) 
            and (VERBrec1.modus     = imperative)
            and (VERBrec1.number    = singular) 
            and (VERBrec1.persons   = [])
            and (VERBrec1.eORenForm = NoForm)
         A: SFCATrec1.key  := SFKdelD
&
\end{verbatim}
\newpage
\subsubsection{Rule for imperative plural.}
\begin{verbatim}
%VerbGbMv 

m1:   SUBVERB{SUBVERBrec1}[mu1]
m2:   SFCAT{SFCATrec1}
m:    VERB{VERBrec1} [head/ SUBVERB{SUBVERBrec1}[mu1]]

comp:
         C: (SUBVERBrec1.conjclasses * [3,4,5,6,7,8,9,10,11,12,13,15,16] <> [])
            and (SFCATrec1.key = SFKt)
         A: VERBrec1           := COPYT_subverbtoverb(SUBVERBrec1);
            VERBrec1.tense     := omegatense;
            VERBrec1.modus     := imperative;
            VERBrec1.number    := plural;
            VERBrec1.persons   := [];
            VERBrec1.eORenForm := NoForm

decomp:
         C: (VERBrec1.conjclasses * [3,4,5,6,7,8,9,10,11,12,13,15,16] <> []) 
            and (VERBrec1.tense     = omegatense) 
            and (VERBrec1.modus     = imperative)
            and (VERBrec1.number    = plural) 
            and (VERBrec1.persons   = [])
            and (VERBrec1.eORenForm = NoForm) 
         A: SFCATrec1.key     := SFKt
&
\end{verbatim}
\newpage
\subsubsection{Rule for conjunctive.}

Note: all weak and strong verbs have present conjunctive forms only.

\begin{verbatim}
%VerbConjunctiefOtt

m1:   SUBVERB{SUBVERBrec1}[mu1]
m2:   SFCAT{SFCATrec1}
m:    VERB{VERBrec1} [head/ SUBVERB{SUBVERBrec1}[mu1]]

comp:
         C: (SUBVERBrec1.conjclasses * [3,4,5,6,7,8,9,10,11,12,13,15,16] <> [])
            and (SFCATrec1.key = SFKe)
         A: VERBrec1           := COPYT_subverbtoverb(SUBVERBrec1);
            VERBrec1.tense     := presenttense;
            VERBrec1.modus     := subjunctive;
            VERBrec1.number    := omeganumber;
            VERBrec1.persons   := [];
            VERBrec1.eORenForm := NoForm

decomp:
         C: (VERBrec1.conjclasses * [3,4,5,6,7,8,9,10,11,12,13,15,16] <> [])
            and (VERBrec1.tense     = presenttense) 
            and (VERBrec1.modus     = subjunctive)
            and (VERBrec1.number    = omeganumber)
            and (VERBrec1.persons   = [])
            and (VERBrec1.eORenForm = NoForm)
         A: SFCATrec1.key   := SFKe
&
\end{verbatim}
\newpage
\subsection{Irregular verbs.}
\subsubsection{Extra rule for the "0"th and 1st person singular present tense.}
\begin{verbatim}
%VerbZorOttExtra 

m1:   SUBVERB{SUBVERBrec1}[mu1]
m:    VERB{VERBrec1} [head/ SUBVERB{SUBVERBrec1}[mu1]]

comp:
         C:  SUBVERBrec1.conjclasses * [1,2] <> []
         A:  VERBrec1           := COPYT_subverbtoverb(SUBVERBrec1);
             VERBrec1.tense     := presenttense;
             VERBrec1.modus     := indicative;
             VERBrec1.number    := singular;
             VERBrec1.persons   := [1,0];
             VERBrec1.eORenForm := NoForm

decomp:
         C:  (VERBrec1.conjclasses * [1,2] <> [])
             and (VERBrec1.tense     = presenttense)
             and (VERBrec1.modus     = indicative)
             and (VERBrec1.number    = singular)
             and ([0,1] * VERBrec1.persons <> [])
             and (VERBrec1.eORenForm = NoForm)
         A:  @
&
\end{verbatim}
\newpage
\subsubsection{Rule for present tense.}

SFKIrrottenk0 is needed for the (extra) forms "zul je" and "kun je".

\begin{verbatim}
%VerbZorOtt 

m1:   SUBVERB{SUBVERBrec1}[mu1]
m2:   SFCAT{SFCATrec1}
m:    VERB{VERBrec1} [head/ SUBVERB{SUBVERBrec1}[mu1]]

comp:
         C1:  SUBVERBrec1.conjclasses * [1,2] <> []
           C2:  true
             C3:  SFCATrec1.key = SFKIrrottenk0
             A3:  VERBrec1          := COPYT_subverbtoverb(SUBVERBrec1);
                  VERBrec1.persons  := [0]
             C3:  SFCATrec1.key = SFKIrrottenk2
             A3:  VERBrec1          := COPYT_subverbtoverb(SUBVERBrec1);
                  VERBrec1.persons  := [2]
             C3:  SFCATrec1.key = SFKIrrottenk3
             A3:  VERBrec1          := COPYT_subverbtoverb(SUBVERBrec1);
                  VERBrec1.persons  := [3]
             C3:  SFCATrec1.key = SFKIrrott4              
             A3:  VERBrec1          := COPYT_subverbtoverb(SUBVERBrec1);
                  VERBrec1.persons  := [4]
             C3:  SFCATrec1.key = SFKIrrott5
             A3:  VERBrec1          := COPYT_subverbtoverb(SUBVERBrec1);
                  VERBrec1.persons  := [5]
           A2:  VERBrec1.number := singular
           C2:  true
             C3:  SFCATrec1.key = SFKIrrott4
             A3:  VERBrec1          := COPYT_subverbtoverb(SUBVERBrec1);
                  VERBrec1.persons  := [4]
             C3:  SFCATrec1.key = SFKIrrott5
             A3:  VERBrec1          := COPYT_subverbtoverb(SUBVERBrec1);
                  VERBrec1.persons  := [5]
             C3:  SFCATrec1.key = SFKIrrottmv
             A3:  VERBrec1          := COPYT_subverbtoverb(SUBVERBrec1);
                  VERBrec1.persons  := [1,2,3]
           A2:  VERBrec1.number   := plural
         A1: VERBrec1.tense     := presenttense;
             VERBrec1.modus     := indicative;
             VERBrec1.eORenForm := NoForm

decomp:
         C1:  (VERBrec1.conjclasses * [1,2] <> []) 
              and (VERBrec1.tense     = presenttense)
              and (VERBrec1.modus     = indicative) 
              and (VERBrec1.eORenForm = NoForm)
           C2:  VERBrec1.number = singular
             C3:  0 in VERBrec1.persons 
             A3:  SFCATrec1.key := SFKIrrottenk0
             C3:  2 in VERBrec1.persons 
             A3:  SFCATrec1.key := SFKIrrottenk2
             C3:  3 in VERBrec1.persons 
             A3:  SFCATrec1.key := SFKIrrottenk3
             C3:  4 in VERBrec1.persons 
             A3:  SFCATrec1.key := SFKIrrott4
             C3:  5 in VERBrec1.persons 
             A3:  SFCATrec1.key := SFKIrrott5
           A2:  @
           C2:  VERBrec1.number = plural
             C3: [1,2,3] * VERBrec1.persons <> []
             A3:  SFCATrec1.key := SFKIrrottmv
             C3:  4 in VERBrec1.persons 
             A3:  SFCATrec1.key := SFKIrrott4
             C3:  5 in VERBrec1.persons 
             A3:  SFCATrec1.key := SFKIrrott5
           A2:  @
         A1:  @
&
\end{verbatim}
\newpage
\subsubsection{Rule for past tense.}
\begin{verbatim}
%VerbZorOvt

m1:   SUBVERB{SUBVERBrec1}[mu1]
m2:   SFCAT{SFCATrec1}
m:    VERB{VERBrec1} [head/ SUBVERB{SUBVERBrec1}[mu1]]

comp:
         C1:  SUBVERBrec1.conjclasses * [1,2] <> []
           C2:  true
             C3:  SFCATrec1.key = SFKIrrovtenk
             A3:  VERBrec1          := COPYT_subverbtoverb(SUBVERBrec1);
                  VERBrec1.persons  := [1,2,3,4]
             C3:  SFCATrec1.key = SFKIrrovt5
             A3:  VERBrec1          := COPYT_subverbtoverb(SUBVERBrec1);
                  VERBrec1.persons  := [5]
           A2:  VERBrec1.number   := singular
           C2:  true
             C3:  SFCATrec1.key = SFKIrrovtmv
             A3:  VERBrec1          := COPYT_subverbtoverb(SUBVERBrec1);
                  VERBrec1.persons  := [1,2,3]
             C3:  SFCATrec1.key = SFKIrrovtenk
             A3:  VERBrec1          := COPYT_subverbtoverb(SUBVERBrec1);
                  VERBrec1.persons  := [4]
             C3:  SFCATrec1.key = SFKIrrovt5
             A3:  VERBrec1          := COPYT_subverbtoverb(SUBVERBrec1);
                  VERBrec1.persons  := [5]
           A2:  VERBrec1.number   := plural
         A1: VERBrec1.tense     := pasttense;
             VERBrec1.modus     := indicative;
             VERBrec1.eORenForm := NoForm

decomp:
         C1:  (VERBrec1.conjclasses * [1,2] <> [])
              and (VERBrec1.tense     = pasttense) 
              and (VERBrec1.modus     = indicative)
              and (VERBrec1.eORenForm = NoForm)
           C2:  VERBrec1.number = singular
             C3:  [1,2,3,4] * VERBrec1.persons <> []
             A3:  SFCATrec1.key  := SFKIrrovtenk
             C3:  5 in VERBrec1.persons 
             A3:  SFCATrec1.key  := SFKIrrovt5
           A2:  @
           C2:  VERBrec1.number = plural
             C3:  [1,2,3] * VERBrec1.persons <> []
             A3:  SFCATrec1.key  := SFKIrrovtmv
             C3:  4 in VERBrec1.persons 
             A3:  SFCATrec1.key  := SFKIrrovtenk
             C3:  5 in VERBrec1.persons 
             A3:  SFCATrec1.key  := SFKIrrovt5
           A2:  @
         A1:  @
&
\end{verbatim}
\newpage
\subsubsection{Rule for past participles with "ge"-prefix.}
\begin{verbatim}
%VerbZorVd1 

m1:   PFCAT{PFCATrec1}
m2:   SUBVERB{SUBVERBrec1}[mu1]
m3:   SFCAT{SFCATrec1}
m:    VERB{VERBrec1} [head/ SUBVERB{SUBVERBrec1}[mu1]]

comp:
         C:  (SUBVERBrec1.conjclasses * [1] <> [])
             and (PFCATrec1.key = PFKge) 
             and (SFCATrec1.key = SFKIrrvd)
         A:  VERBrec1           := COPYT_subverbtoverb(SUBVERBrec1);
             VERBrec1.tense     := omegatense;
             VERBrec1.modus     := pastpart;
             VERBrec1.number    := omeganumber;
             VERBrec1.persons   := [];
             VERBrec1.eORenForm := NoForm

decomp:
         C:  (VERBrec1.conjclasses * [1] <> [])
             and (VERBrec1.tense     = omegatense) 
             and (VERBrec1.modus     = pastpart)
             and (VERBrec1.number    = omeganumber)
             and (VERBrec1.persons   = [])
             and (VERBrec1.eORenForm = NoForm)
         A:  PFCATrec1.key     := PFKge;
             SFCATrec1.key     := SFKIrrvd
&
\end{verbatim}
\newpage
\subsubsection{Rule for past participles without "ge"-prefix.}
\begin{verbatim}
%VerbZorVd2 

m1:   SUBVERB{SUBVERBrec1}[mu1]
m2:   SFCAT{SFCATrec1}
m:    VERB{VERBrec1} [head/ SUBVERB{SUBVERBrec1}[mu1]]

comp:
         C:  (SUBVERBrec1.conjclasses * [2] <> [])
             and (SFCATrec1.key = SFKIrrvd)
         A:  VERBrec1           := COPYT_subverbtoverb(SUBVERBrec1);
             VERBrec1.tense     := omegatense;
             VERBrec1.modus     := pastpart;
             VERBrec1.number    := omeganumber;
             VERBrec1.persons   := [];
             VERBrec1.eORenForm := NoForm

decomp:
         C:  (VERBrec1.conjclasses * [2] <> [])
             and (VERBrec1.tense     = omegatense) 
             and (VERBrec1.modus     = pastpart)
             and (VERBrec1.number    = omeganumber)
             and (VERBrec1.persons   = [])
             and (VERBrec1.eORenForm = NoForm)
         A:  SFCATrec1.key     := SFKIrrvd
&
\end{verbatim}
\newpage
\subsubsection{Rule for infinitive.}
\begin{verbatim}
%VerbZorInf 

m1:   SUBVERB{SUBVERBrec1}[mu1]
m2:   SFCAT{SFCATrec1}
m:    VERB{VERBrec1} [head/ SUBVERB{SUBVERBrec1}[mu1]]

comp:
         C:  (SUBVERBrec1.conjclasses * [1,2] <> [])
             and (SFCATrec1.key = SFKIrrottmv)
         A:  VERBrec1           := COPYT_subverbtoverb(SUBVERBrec1);
             VERBrec1.tense     := omegatense;
             VERBrec1.modus     := infinitive;
             VERBrec1.number    := omeganumber;
             VERBrec1.persons   := [];
             VERBrec1.eORenForm := NoForm

decomp:
         C:  (VERBrec1.conjclasses * [1,2] <> [])
             and (VERBrec1.tense     = omegatense) 
             and (VERBrec1.modus     = infinitive)
             and (VERBrec1.number    = omeganumber)
             and (VERBrec1.persons   = [])
             and (VERBrec1.eORenForm = NoForm)
         A:  SFCATrec1.key     := SFKIrrottmv
&
\end{verbatim}
\newpage
\subsubsection{Rule for present participle.}
\begin{verbatim}
%VerbZorTd

m1:   SUBVERB{SUBVERBrec1}[mu1]
m2:   SFCAT{SFCATrec1}
m3:   SFCAT{SFCATrec2}
m:    VERB{VERBrec1} [head/ SUBVERB{SUBVERBrec1}[mu1]]

comp:
         C:   (SUBVERBrec1.conjclasses * [1,2] <> []) 
              and (SFCATrec1.key = SFKIrrottmv)
              and (SFCATrec2.key = SFKdt)
         A:   VERBrec1           := COPYT_subverbtoverb(SUBVERBrec1);
              VERBrec1.tense     := omegatense;
              VERBrec1.modus     := prespart;
              VERBrec1.number    := omeganumber;
              VERBrec1.persons   := [];
              VERBrec1.eORenForm := NoForm

decomp:
         C:   (VERBrec1.conjclasses * [1,2] <> [])
              and (VERBrec1.tense     = omegatense) 
              and (VERBrec1.modus     = prespart)
              and (VERBrec1.number    = omeganumber)
              and (VERBrec1.persons   = [])
              and (VERBrec1.eORenForm = NoForm)
         A:   SFCATrec1.key := SFKIrrottmv;
              SFCATrec2.key := SFKdt
&
\end{verbatim}
\newpage
\subsubsection{Rule for imperative singular.}
\begin{verbatim}
%VerbZorGbEnk 

m1:   SUBVERB{SUBVERBrec1}[mu1]
m2:   SFCAT{SFCATrec1}
m:    VERB{VERBrec1} [head/ SUBVERB{SUBVERBrec1}[mu1]]

comp:
         C:  (SUBVERBrec1.conjclasses * [1,2] <> [])
             and (SFCATrec1.key = SFKIrrgb)
         A:  VERBrec1           := COPYT_subverbtoverb(SUBVERBrec1);
             VERBrec1.tense     := omegatense;
             VERBrec1.modus     := imperative;
             VERBrec1.number    := singular;
             VERBrec1.persons   := [];
             VERBrec1.eORenForm := NoForm

decomp:
         C:  (VERBrec1.conjclasses * [1,2] <> []) 
             and (VERBrec1.tense     = omegatense) 
             and (VERBrec1.modus     = imperative) 
             and (VERBrec1.number    = singular) 
             and (VERBrec1.persons   = [])
             and (VERBrec1.eORenForm = NoForm)
         A:  SFCATrec1.key    := SFKIrrgb
&
\end{verbatim}
\newpage
\subsubsection{Rule for imperative plural.}
\begin{verbatim}
%VerbZorGbMv 

m1:   SUBVERB{SUBVERBrec1}[mu1]
m2:   SFCAT{SFCATrec1}
m3:   SFCAT{SFCATrec2}
m:    VERB{VERBrec1} [head/ SUBVERB{SUBVERBrec1}[mu1]]

comp:
         C:  (SUBVERBrec1.conjclasses * [1,2] <> [])
             and (SFCATrec1.key = SFKIrrgb)
             and (SFCATrec2.key = SFKt)
         A:  VERBrec1           := COPYT_subverbtoverb(SUBVERBrec1);
             VERBrec1.tense     := omegatense;
             VERBrec1.modus     := imperative;
             VERBrec1.number    := plural;
             VERBrec1.persons   := [];
             VERBrec1.eORenForm := NoForm

decomp:
         C:  (VERBrec1.conjclasses * [1,2] <> [])
             and (VERBrec1.tense     = omegatense) 
             and (VERBrec1.modus     = imperative)
             and (VERBrec1.number    = plural)
             and (VERBrec1.persons   = [])
             and (VERBrec1.eORenForm = NoForm)
         A:  SFCATrec1.key     := SFKIrrgb;
             SFCATrec2.key     := SFKt
&
\end{verbatim}
\newpage
\subsubsection{Rule for conjunctive.}

Note: only the verb "zijn" has a past conjunctive form: "ware". 

\begin{verbatim}
%VerbZorConjunctief

m1:   SUBVERB{SUBVERBrec1}[mu1]
m2:   SFCAT{SFCATrec1}
m:    VERB{VERBrec1} [head/ SUBVERB{SUBVERBrec1}[mu1]]

comp:
         C1:  SUBVERBrec1.conjclasses * [1,2] <> []
           C2:  SFCATrec1.key = SFKIrrconjott
           A2:  VERBrec1         := COPYT_subverbtoverb(SUBVERBrec1);
                VERBrec1.tense   := presenttense
           C2:  SFCATrec1.key = SFKIrrconjovt
           A2:  VERBrec1         := COPYT_subverbtoverb(SUBVERBrec1);
                VERBrec1.tense   := pasttense
         A1:  VERBrec1.modus     := subjunctive;
              VERBrec1.number    := omeganumber;
              VERBrec1.persons   := [];
              VERBrec1.eORenForm := NoForm

decomp:
         C1:  (VERBrec1.conjclasses * [1,2] <> [])
              and (VERBrec1.modus     = subjunctive)
              and (VERBrec1.number    = omeganumber)
              and (VERBrec1.persons   = [])
              and (VERBrec1.eORenForm = NoForm) 
           C2:  VERBrec1.tense = presenttense 
           A2:  SFCATrec1.key := SFKIrrconjott
           C2:  VERBrec1.tense = pasttense
           A2:  SFCATrec1.key := SFKIrrconjovt
         A1:  @
&
\end{verbatim}
\newpage
\subsection{Rule for the "e(n)"-form of past and present participle.}

Note: this is the only rule that makes a VERB out of another VERB; the only
attribute that changes is "eORenForm".
Because this rule is of the type ``CAT''-to-``CAT''-type and because it should 
never be applied more than once, some measures have
been taken to prevent looping or unnecessarily trying to apply the rule again:
in the ``comp''-part, there is an explicit test on the input: 
``SFCATrec1.eORenForm = NoForm'' (see: C1); in the ``decomp''-part, there is an 
explicit assignment when the rule succeeds: ``VERBrec1.eORenForm := NoForm''
(see: A1).

\begin{verbatim}
%VerbeORenvorm

m1:   VERB{VERBrec1}[mu1]
m2:   SFCAT{SFCATrec1}
m:    VERB{VERBrec1}[mu1]

comp:
         C1:  ((VERBrec1.modus = pastpart) or (VERBrec1.modus = prespart)) 
              and (VERBrec1.eORenForm = NoForm)
           C2:  SFCATrec1.key = SFKe
           A2:  VERBrec1.eORenForm  := eForm
           C2:  SFCATrec1.key = SFKen
           A2:  VERBrec1.eORenForm  := enForm
         A1:  @

decomp:
         C1:  (VERBrec1.modus = pastpart) or (VERBrec1.modus = prespart)
           C2:  VERBrec1.eORenForm = eForm
           A2:  SFCATrec1.key := SFKe
           C2:  VERBrec1.eORenForm = enForm
           A2:  SFCATrec1.key := SFKen
         A1:  VERBrec1.eORenForm := NoForm
&
\end{verbatim}


%DUTCH2
%&

\newpage
\section{Propernouns and nouns.}
\subsection{Propernouns.}
\subsubsection{Rule for propernoun without genitive out of bpropernoun.}

Propernouns have inflection only in case of genitive formation.

Cases like: "de Antillen", "de Fillipijnen", "de Verenigde Staten", 
            "de Verenigde Arabische Emiraten", etc. are problematic.
The best way to deal with these words is to treat them as normal nouns (with 
only semantic properties of a propernoun), asif they are some kind of fixed
idioms.

Cases like "de beide Duitslanden" and "de Kennedy's" will be treated as 
derivation: out of a propernoun a subnoun is made.
Attributes like for instance {\em gender} are already present 
at the bpropernoun level. Other examples: "het zonnige Itali/"{e}", 
"het romantische Duitsland".  

Propernouns do not have a sub-level.

\begin{verbatim}
%bpropernountopropernoun

m1:   BPROPERNOUN{BPROPERNOUNrec1}
m:    PROPERNOUN{PROPERNOUNrec1} [head/ BPROPERNOUN{BPROPERNOUNrec1}]

comp:    
         C:   true
         A:   PROPERNOUNrec1  := COPYT_bpropernountopropernoun(BPROPERNOUNrec1);
              PROPERNOUNrec1.geni := false  

decomp:  
         C:   PROPERNOUNrec1.geni = false  
         A:   @
&
\end{verbatim}
\newpage
\subsubsection{Rule for propernoun with genitive out of bpropernoun.}
\begin{verbatim}
%bpropertopropergenitief

m1:   BPROPERNOUN{BPROPERNOUNrec1}
m2:   SFCAT{SFCATrec1}
m:    PROPERNOUN{PROPERNOUNrec1} [head/ BPROPERNOUN{BPROPERNOUNrec1}]

comp:    
         C:   (SFCATrec1.key = SFKgens)
              and (not(OnlyPlural in BPROPERNOUNrec1.pluralforms))
              and (BPROPERNOUNrec1.possgeni = true)
         A:   PROPERNOUNrec1 := COPYT_bpropernountopropernoun(BPROPERNOUNrec1);
              PROPERNOUNrec1.number  := singular;
              PROPERNOUNrec1.geni    := true

decomp:  
         C:  (PROPERNOUNrec1.number       = singular) 
             and (PROPERNOUNrec1.possgeni = true)
             and (PROPERNOUNrec1.geni     = true)
             and (not(OnlyPlural in PROPERNOUNrec1.pluralforms))
         A:  SFCATrec1.key := SFKgens
&
\end{verbatim}
\newpage
\subsubsection{Derivation rule for propernouns with diminutive.}
Note: 
\begin{itemize}
  \item compare with rule for diminutives of nouns.
  \item there is a separate rule for genitive forms.
\end{itemize}
\begin{verbatim}
%PropernounWithDimForm

m1: BPROPERNOUN{BPROPERNOUNrec1}
m2: SFCAT{SFCATrec1}

m:  PROPERNOUN{PROPERNOUNrec1}[complrel/BPROPERNOUN{BPROPERNOUNrec1}, 
                          head/BNOUNSUFF(dimBNOUNSUFFkey){BNOUNSUFFrec1}]

comp:

C1: true
   C2: (not(noDim in BPROPERNOUNrec1.dimforms))
      C3: (SFCATrec1.key = SFKje) and (jeDim in BPROPERNOUNrec1.dimforms)
      A3: @
      C3: (SFCATrec1.key = SFKetje) and (etjeDim in BPROPERNOUNrec1.dimforms)  
      A3: @
      C3: (SFCATrec1.key = SFKonregdim) and (irregdim in 
                                                    BPROPERNOUNrec1.dimforms)
      A3: @
      C3: (SFCATrec1.key = SFKdimletterword) and  (dimletterword in
                                                    BPROPERNOUNrec1.dimforms)
      A3: @
   A2:   PROPERNOUNrec1      := COPYT_bpropernountopropernoun(BPROPERNOUNrec1)
A1: PROPERNOUNrec1.geni     := false;
    PROPERNOUNrec1.number   := singular

decomp:

C1: (PROPERNOUNrec1.geni     = false)  and
    (PROPERNOUNrec1.number   = singular)
   C2: (not(noDim in BPROPERNOUNrec1.dimforms))
      C3: jeDim in BPROPERNOUNrec1.dimforms
      A3: SFCATrec1.key := SFKje
      C3: etjeDim in BPROPERNOUNrec1.dimforms
      A3: SFCATrec1.key := SFKetje
      C3: irregdim in BPROPERNOUNrec1.dimforms
      A3: SFCATrec1.key := SFKonregdim
      C3: dimletterword in BPROPERNOUNrec1.dimforms
      A3: SFCATrec1.key := SFKdimletterword
   A2: @
A1: @
&
\end{verbatim}
\newpage
\subsubsection{Derivation rule for propernouns with diminutive.}
Note: This is a separate rule for genitive forms.
\begin{verbatim}
%PropernounWithGeniDimForm

m1: BPROPERNOUN{BPROPERNOUNrec1}
m2: SFCAT{SFCATrec1}
m3: SFCAT{SFCATrec2}

m:  PROPERNOUN{PROPERNOUNrec1}[complrel/BPROPERNOUN{BPROPERNOUNrec1}, 
                          head/BNOUNSUFF(dimBNOUNSUFFkey){BNOUNSUFFrec1}]

comp:

C1: (BPROPERNOUNrec1.possgeni = true) and (SFCATrec2.key = SFKgens)
   C2: (not(noDim in BPROPERNOUNrec1.dimforms))
      C3: (SFCATrec1.key = SFKje) and (jeDim in BPROPERNOUNrec1.dimforms)
      A3: @
      C3: (SFCATrec1.key = SFKetje) and (etjeDim in BPROPERNOUNrec1.dimforms)  
      A3: @
      C3: (SFCATrec1.key = SFKonregdim) and (irregdim in 
                                                    BPROPERNOUNrec1.dimforms)
      A3: @
      C3: (SFCATrec1.key = SFKdimletterword) and  (dimletterword in
                                                    BPROPERNOUNrec1.dimforms)
      A3: @
   A2:   PROPERNOUNrec1      := COPYT_bpropernountopropernoun(BPROPERNOUNrec1)
A1: PROPERNOUNrec1.geni     := true;
    PROPERNOUNrec1.number   := singular

decomp:

C1: (PROPERNOUNrec1.possgeni = true)  and
    (PROPERNOUNrec1.geni     = true)  and
    (PROPERNOUNrec1.number   = singular) 
   C2: (not(noDim in BPROPERNOUNrec1.dimforms))
      C3: jeDim in BPROPERNOUNrec1.dimforms
      A3: SFCATrec1.key := SFKje
      C3: etjeDim in BPROPERNOUNrec1.dimforms
      A3: SFCATrec1.key := SFKetje
      C3: irregdim in BPROPERNOUNrec1.dimforms
      A3: SFCATrec1.key := SFKonregdim
      C3: dimletterword in BPROPERNOUNrec1.dimforms
      A3: SFCATrec1.key := SFKdimletterword
   A2: @
A1: SFCATrec2.key := SFKgens
&
\end{verbatim}
\newpage
\subsection{Nouns}
\subsubsection{Rule for subnouns out of bpropernouns.}
\begin{verbatim}
%bpropernountosubnoun

m1:   BPROPERNOUN{BPROPERNOUNrec1}
m:    SUBNOUN{SUBNOUNrec1} [head/ BPROPERNOUN{BPROPERNOUNrec1}]

comp:    
         C:   true
         A:   SUBNOUNrec1.req          := [pospol, negpol, omegapol];
              SUBNOUNrec1.env          := [pospol, negpol, omegapol];
              SUBNOUNrec1.dimforms     := BPROPERNOUNrec1.dimforms;
              SUBNOUNrec1.pluralforms  := BPROPERNOUNrec1.pluralforms;
              SUBNOUNrec1.genders      := BPROPERNOUNrec1.genders;
              SUBNOUNrec1.class        := BPROPERNOUNrec1.class;
              SUBNOUNrec1.deixis       := BPROPERNOUNrec1.deixis;
              SUBNOUNrec1.aspect       := BPROPERNOUNrec1.aspect;
              SUBNOUNrec1.retro        := BPROPERNOUNrec1.retro;
              SUBNOUNrec1.sexes        := BPROPERNOUNrec1.sexes;
              SUBNOUNrec1.subcs        := [othernoun];
              SUBNOUNrec1.temporal     := BPROPERNOUNrec1.temporal;
              SUBNOUNrec1.possgeni     := false;
              SUBNOUNrec1.animate      := BPROPERNOUNrec1.animate;
              SUBNOUNrec1.human        := BPROPERNOUNrec1.human;
              SUBNOUNrec1.posscomas    := [count];
              SUBNOUNrec1.thetanp      := omegathetanp;
              SUBNOUNrec1.nounpatterns := [];
              SUBNOUNrec1.prepkey      := 0;
              SUBNOUNrec1.personal     := true;
              SUBNOUNrec1.lastaffix    := noaffix

decomp:  
         C:   (SUBNOUNrec1.dimforms     = BPROPERNOUNrec1.dimforms)    and
              (SUBNOUNrec1.pluralforms  = BPROPERNOUNrec1.pluralforms) and
              (SUBNOUNrec1.genders      = BPROPERNOUNrec1.genders)     and
              (SUBNOUNrec1.class        = BPROPERNOUNrec1.class)       and
              (SUBNOUNrec1.deixis       = BPROPERNOUNrec1.deixis)      and
              (SUBNOUNrec1.aspect       = BPROPERNOUNrec1.aspect)      and
              (SUBNOUNrec1.retro        = BPROPERNOUNrec1.retro)       and
              (SUBNOUNrec1.sexes        = BPROPERNOUNrec1.sexes)       and
              (SUBNOUNrec1.temporal     = BPROPERNOUNrec1.temporal)    and
              (SUBNOUNrec1.possgeni     = false)                       and
              (SUBNOUNrec1.animate      = BPROPERNOUNrec1.animate)     and
              (SUBNOUNrec1.human        = BPROPERNOUNrec1.human)       and
              (SUBNOUNrec1.lastaffix    = noaffix)   
         A:   @
&
\end{verbatim}
\newpage
\subsubsection{Rule for subnouns out of bnouns without derivation.}
\begin{verbatim}
%bnountosub

m1:   BNOUN{BNOUNrec1}
m:    SUBNOUN{SUBNOUNrec1} [head/ BNOUN{BNOUNrec1}]

comp:    
         C:   true
         A:   SUBNOUNrec1       := COPYT_bnountosubnoun(BNOUNrec1)

decomp:  
         C:   true
         A:   @
&
\end{verbatim}
\newpage
\subsubsection{Derivation rule for diminutives.}
Note: 
\begin{itemize}
  \item "possgeni" (and many other attributes) come from the BNOUN;
  \item "onlyPlural"-words, like "hersenen", "notulen" and "onkosten" do NOT
         have a dim-form or have IRREGULAR dimforms like "hersentje" (or is
         "hersentjes" the only possible form??), which is of importance for the
         filling of the attribute "dimforms";
  \item for "dimletterword" see suffix-rules with "SFKdimletterword";
  \item the condition in C2 of `comp' is in this rule necessary, because
         dimforms of dimforms do not exist: *bootjetje! 
         This does NOT mean that two dimforms can never occur in one word:
         "stapJESgewijsheidJES" or "jongetJESachtigheidJES"; it only means that 
         two dimunitive-suffixes cannot be adjacent.
\end{itemize}
\begin{verbatim}
%DimForm

m1: SUBNOUN{SUBNOUNrec1}[mu1]
m2: SFCAT{SFCATrec1}

m:  SUBNOUN{SUBNOUNrec2}[complrel/SUBNOUN{SUBNOUNrec1}[mu1], 
                          head/BNOUNSUFF(dimBNOUNSUFFkey){BNOUNSUFFrec1}]

comp:

C1: true
   C2: (not(SUBNOUNrec1.lastaffix = dimaffix)) and (not(noDim in 
                                                     SUBNOUNrec1.dimforms))
      C3: (SFCATrec1.key = SFKje) and (jeDim in SUBNOUNrec1.dimforms)
      A3: @
      C3: (SFCATrec1.key = SFKetje) and (etjeDim in SUBNOUNrec1.dimforms)  
      A3: @
      C3: (SFCATrec1.key = SFKonregdim) and (irregdim in SUBNOUNrec1.dimforms)
      A3: @
      C3: (SFCATrec1.key = SFKdimletterword) and  (dimletterword in
                                                         SUBNOUNrec1.dimforms)
      A3: @
   A2: SUBNOUNrec2               := SUBNOUNrec1;
       SUBNOUNrec2.lastaffix     := dimaffix
A1: SUBNOUNrec2.dimforms    := BNOUNSUFFrec1.dimforms;
    SUBNOUNrec2.pluralforms := BNOUNSUFFrec1.pluralforms;
    SUBNOUNrec2.genders     := BNOUNSUFFrec1.genders

decomp:

C1: (SUBNOUNrec2.pluralforms = BNOUNSUFFrec1.pluralforms) and
    (SUBNOUNrec2.genders     = BNOUNSUFFrec1.genders)      and
    (SUBNOUNrec2.dimforms    = BNOUNSUFFrec1.dimforms)
   C2: (SUBNOUNrec2.lastaffix = dimaffix)    and 
       (not(noDim in SUBNOUNrec1.dimforms))  and 
       (not(SUBNOUNrec1.lastaffix = dimaffix))  
      C3:jeDim in SUBNOUNrec1.dimforms
      A3: SFCATrec1.key := SFKje
      C3: etjeDim in SUBNOUNrec1.dimforms
      A3: SFCATrec1.key := SFKetje
      C3: irregdim in SUBNOUNrec1.dimforms
      A3: SFCATrec1.key := SFKonregdim
      C3: dimletterword in SUBNOUNrec1.dimforms
      A3: SFCATrec1.key := SFKdimletterword
   A2: @
A1: @
&
\end{verbatim}
\newpage
\subsubsection{ Rule for the singular form of nouns. }

Note: in case that the noun has pluralform only ("hersenen"), this rule 
      assigns the value "plural"!

\begin{verbatim}
%nounenkelvoud

m1:   SUBNOUN{SUBNOUNrec1}[mu1]
m:    NOUN{NOUNrec1} [head/ SUBNOUN{SUBNOUNrec1}[mu1]]

comp:    
         C1:  true
           C2:  not(OnlyPlural in SUBNOUNrec1.pluralforms)
           A2:  NOUNrec1        := COPYT_subnountonoun(SUBNOUNrec1);
                NOUNrec1.number := singular
           C2:  OnlyPlural in SUBNOUNrec1.pluralforms
           A2:  NOUNrec1        := COPYT_subnountonoun(SUBNOUNrec1);
                NOUNrec1.number := plural
         A1:  NOUNrec1.geni  := false

decomp:  
         C1:  NOUNrec1.geni = false
           C2:  (NOUNrec1.number = singular) 
                and (not(OnlyPlural in NOUNrec1.pluralforms))
           A2:  @
           C2:  (NOUNrec1.number = plural) 
                and (OnlyPlural in NOUNrec1.pluralforms)
           A2:  @
         A1:   @
&
\end{verbatim}
\newpage
\subsubsection{Rule for the plural form of nouns.}
\begin{verbatim}
%nounmeervoud

m1:   SUBNOUN{SUBNOUNrec1}[mu1]
m2:   SFCAT{SFCATrec1}
m:    NOUN{NOUNrec1} [head/ SUBNOUN{SUBNOUNrec1}[mu1]]

comp:    
         C1:  (not(OnlyPlural in SUBNOUNrec1.pluralforms)) 
              and (not(NoPlural in SUBNOUNrec1.pluralforms)) 
           C2:  (SFCATrec1.key = SFKen) 
                 and (enPlural in SUBNOUNrec1.pluralforms)
           A2:  NOUNrec1  := COPYT_subnountonoun(SUBNOUNrec1)
           C2:  (SFCATrec1.key = SFKmvs) 
                 and (sPlural in SUBNOUNrec1.pluralforms)
           A2:  NOUNrec1  := COPYT_subnountonoun(SUBNOUNrec1)
           C2:  (SFCATrec1.key = SFKaTOaa) 
                 and (aTOaaPlural in SUBNOUNrec1.pluralforms)
           A2:  NOUNrec1  := COPYT_subnountonoun(SUBNOUNrec1)
           C2:  (SFCATrec1.key = SFKaTOee) 
                 and (aTOeePlural in SUBNOUNrec1.pluralforms)
           A2:  NOUNrec1  := COPYT_subnountonoun(SUBNOUNrec1)
           C2:  (SFCATrec1.key = SFKeTOee)
                 and (eTOeePlural in SUBNOUNrec1.pluralforms)
           A2:  NOUNrec1  := COPYT_subnountonoun(SUBNOUNrec1)
           C2:  (SFCATrec1.key = SFKeiTOee) 
                 and (eiTOeePlural in SUBNOUNrec1.pluralforms)
           A2:  NOUNrec1  := COPYT_subnountonoun(SUBNOUNrec1)
           C2:  (SFCATrec1.key = SFKiTOee) 
                 and (iTOeePlural in SUBNOUNrec1.pluralforms)
           A2:  NOUNrec1  := COPYT_subnountonoun(SUBNOUNrec1)
           C2:  (SFCATrec1.key = SFKoTOoo) 
                 and (oTOooPlural in SUBNOUNrec1.pluralforms)
           A2:  NOUNrec1  := COPYT_subnountonoun(SUBNOUNrec1)
           C2:  (SFCATrec1.key = SFKeren) 
                 and (erenPlural in SUBNOUNrec1.pluralforms)
           A2:  NOUNrec1  := COPYT_subnountonoun(SUBNOUNrec1)
           C2:  (SFCATrec1.key = SFKien) 
                 and (ienPlural in SUBNOUNrec1.pluralforms)
           A2:  NOUNrec1  := COPYT_subnountonoun(SUBNOUNrec1)
           C2:  (SFCATrec1.key = SFKden) 
                 and (denPlural in SUBNOUNrec1.pluralforms)
           A2:  NOUNrec1  := COPYT_subnountonoun(SUBNOUNrec1)
           C2:  (SFCATrec1.key = SFKnen) 
                 and (nenPlural in SUBNOUNrec1.pluralforms)
           A2:  NOUNrec1  := COPYT_subnountonoun(SUBNOUNrec1)
           C2:  (SFCATrec1.key = SFKieAccent) 
                 and (ieAccentPlural in SUBNOUNrec1.pluralforms)
           A2:  NOUNrec1  := COPYT_subnountonoun(SUBNOUNrec1)
           C2:  (SFCATrec1.key = SFKlui) 
                 and (luiPlural in SUBNOUNrec1.pluralforms)
           A2:  NOUNrec1  := COPYT_subnountonoun(SUBNOUNrec1)
           C2:  (SFCATrec1.key = SFKlieden) 
                 and (liedenPlural in SUBNOUNrec1.pluralforms)
           A2:  NOUNrec1  := COPYT_subnountonoun(SUBNOUNrec1)
           C2:  (SFCATrec1.key = SFKLat) 
                 and (LatPlural in SUBNOUNrec1.pluralforms)
           A2:  NOUNrec1  := COPYT_subnountonoun(SUBNOUNrec1)
           C2:  (SFCATrec1.key = SFKenIrreg) 
                 and (enIrregPlural in SUBNOUNrec1.pluralforms)
           A2:  NOUNrec1  := COPYT_subnountonoun(SUBNOUNrec1)
           C2:  (SFCATrec1.key = SFKsIrreg) 
                 and (sIrregPlural in SUBNOUNrec1.pluralforms)
           A2:  NOUNrec1  := COPYT_subnountonoun(SUBNOUNrec1)
           C2:  (SFCATrec1.key = SFKLatIrreg) 
                 and (LatIrregPlural in SUBNOUNrec1.pluralforms)
           A2:  NOUNrec1  := COPYT_subnountonoun(SUBNOUNrec1)
         A1:  NOUNrec1.number     := plural;
              NOUNrec1.geni       := false

decomp:  
         C1:   (NOUNrec1.number   = plural) 
               and (NOUNrec1.geni = false) 
               and (not(OnlyPlural in NOUNrec1.pluralforms))
           C2:  enPlural in NOUNrec1.pluralforms
           A2:  SFCATrec1.key := SFKen
           C2:  sPlural in NOUNrec1.pluralforms
           A2:  SFCATrec1.key := SFKmvs
           C2:  aTOaaPlural in NOUNrec1.pluralforms
           A2:  SFCATrec1.key := SFKaTOaa
           C2:  aTOeePlural in NOUNrec1.pluralforms
           A2:  SFCATrec1.key := SFKaTOee
           C2:  eTOeePlural in NOUNrec1.pluralforms
           A2:  SFCATrec1.key := SFKeTOee
           C2:  eiTOeePlural in NOUNrec1.pluralforms
           A2:  SFCATrec1.key := SFKeiTOee
           C2:  iTOeePlural in NOUNrec1.pluralforms
           A2:  SFCATrec1.key := SFKiTOee
           C2:  oTOooPlural in NOUNrec1.pluralforms
           A2:  SFCATrec1.key := SFKoTOoo
           C2:  erenPlural in NOUNrec1.pluralforms
           A2:  SFCATrec1.key := SFKeren
           C2:  ienPlural in NOUNrec1.pluralforms
           A2:  SFCATrec1.key := SFKien
           C2:  denPlural in NOUNrec1.pluralforms
           A2:  SFCATrec1.key := SFKden
           C2:  nenPlural in NOUNrec1.pluralforms
           A2:  SFCATrec1.key := SFKnen
           C2:  ieAccentPlural in NOUNrec1.pluralforms
           A2:  SFCATrec1.key := SFKieAccent
           C2:  luiPlural in NOUNrec1.pluralforms
           A2:  SFCATrec1.key := SFKlui
           C2:  liedenPlural in NOUNrec1.pluralforms
           A2:  SFCATrec1.key := SFKlieden
           C2:  LatPlural in NOUNrec1.pluralforms
           A2:  SFCATrec1.key := SFKLat
           C2:  enIrregPlural in NOUNrec1.pluralforms
           A2:  SFCATrec1.key := SFKenIrreg
           C2:  sIrregPlural in NOUNrec1.pluralforms
           A2:  SFCATrec1.key := SFKsIrreg
           C2:  LatIrregPlural in NOUNrec1.pluralforms
           A2:  SFCATrec1.key := SFKLatIrreg
         A1:   @
&
\end{verbatim}
\newpage
\subsubsection{Rule for genitive of singular nouns.}

Note: ANS says that (besides rigid constructions like "heer des huizes") 
      genitive forms can only be made out of proper names and a certain kind of
      nouns, namely the ones that can be used accost somebody, like: "vader",
      "moeder", "dominee", "buurman", "ouder"(?). In the grammatical Compendium
      of Van Dale, the main criterium to decide whether or not a noun can get 
      a genitive-s is family-relation.
      Also diminutives can get genitive-s: "mijn zoontjes fiets", etc.
\begin{verbatim}      
%noungenitiefenkelvoud

m1:   SUBNOUN{SUBNOUNrec1}[mu1]
m2:   SFCAT{SFCATrec1}
m:    NOUN{NOUNrec1} [head/ SUBNOUN{SUBNOUNrec1}[mu1]]


comp:    
         C:   (SFCATrec1.key = SFKgens) 
              and (SUBNOUNrec1.possgeni = true)
              and (not(OnlyPlural in SUBNOUNrec1.pluralforms))
         A:   NOUNrec1            := COPYT_subnountonoun(SUBNOUNrec1);
              NOUNrec1.number     := singular;
              NOUNrec1.geni       := true

decomp:  
         C:  (NOUNrec1.number       = singular)
             and (NOUNrec1.possgeni = true)
             and (NOUNrec1.geni     = true) 
             and (not(OnlyPlural in NOUNrec1.pluralforms))
         A:  SFCATrec1.key := SFKgens
&
\end{verbatim}
\newpage
\subsubsection{Rule for genitive of plural nouns.}

Note: because only a (very) limited number of nouns can get genitive-s, not
      every plural-ending will be possible; in fact, the only plural-ending
      here seems to be "-s": "grootouders' huis", "mijn zoontjes' fietsen",
      etc. Therefore, this rule accepts "sPlural" and "sIrregPlural" ("mijn
      eegaas' namen verschilden") only.
\begin{verbatim}
%noungenitiefmeervoud

m1:   SUBNOUN{SUBNOUNrec1}[mu1]
m2:   SFCAT{SFCATrec1}
m3:   SFCAT{SFCATrec2}
m:    NOUN{NOUNrec1} [head/ SUBNOUN{SUBNOUNrec1}[mu1]]

comp:    
         C1:  (SFCATrec2.key = SFKgens) 
              and (SUBNOUNrec1.possgeni = true)
           C2:  (SFCATrec1.key = SFKmvs) 
                 and (sPlural in SUBNOUNrec1.pluralforms)
           A2:  NOUNrec1  := COPYT_subnountonoun(SUBNOUNrec1)
           C2:  (SFCATrec1.key = SFKsIrreg) 
                 and (sIrregPlural in SUBNOUNrec1.pluralforms)
           A2:  NOUNrec1  := COPYT_subnountonoun(SUBNOUNrec1)
         A1:  NOUNrec1.number     := plural;
              NOUNrec1.geni       := true

decomp:  
         C1:  (NOUNrec1.number       = plural) 
              and (NOUNrec1.geni     = true) 
              and (NOUNrec1.possgeni = true)
           C2:  sIrregPlural in NOUNrec1.pluralforms
           A2:  SFCATrec1.key := SFKsIrreg
           C2:  sPlural in NOUNrec1.pluralforms
           A2:  SFCATrec1.key := SFKmvs
         A1:  SFCATrec2.key := SFKgens
&
\end{verbatim}
\newpage
\section{Pronouns.}
\subsection{Perspro's}
\subsubsection{Rule for nominative forms of PERSPRO's.}

Note: this rule makes the "dictionary"-forms, like "ik", "jij", etc., which are
      always nominative and never reduced.
      The attribute `reduced' is compared with/ assigned from the inherent
      attribute `generalized' of BPERSPRO; the value of `generalized' is false
      for the normal dictionary forms, like "ik", "jij", "hij", etc., but it is
      true for the BPERSPRO with `generalized' meaning, like "je", "ze" (as in
      for instance: "ZE laten JE hier ook maar aanploeteren"). These are the
      only reduced forms that are in the dictionary, al the others are derived
      from the non-reduced forms in the dictionary (see rule Perspro2.
\begin{verbatim} 
%Perspro1

m1:   BPERSPRO{BPERSPROrec1}
m:    PERSPRO{PERSPROrec1} [head/ BPERSPRO{BPERSPROrec1}]

comp:    
         C:   true
         A:   PERSPROrec1              := COPYT_bpersprotoperspro(BPERSPROrec1);
              PERSPROrec1.reduced      := BPERSPROrec1.generalized;
              PERSPROrec1.persprocases := [nominative]

decomp:  
         C:   (nominative in PERSPROrec1.persprocases) 
              and (PERSPROrec1.reduced = BPERSPROrec1.generalized)
         A:   @
&
\end{verbatim}
\newpage
\subsubsection{Rule for nominative, dative and accusative forms of PERSPRO's.}

Note: this rule makes derived forms like "mij", "jou", "hem", etc.; forms can
      be nominative, or dative, or accusative. For all cases, reduced forms, 
      like "me" are made too.
\begin{verbatim}
%Perspro2

m1:   BPERSPRO{BPERSPROrec1}
m2:   SFCAT{SFCATrec1}
m:    PERSPRO{PERSPROrec1} [head/ BPERSPRO{BPERSPROrec1}]

comp:    
         C1:  true
           C2:   BPERSPROrec1.number = singular
             C3:   SFCATrec1.key = SFKsgnom
             A3:   PERSPROrec1         := COPYT_bpersprotoperspro(BPERSPROrec1);
                   PERSPROrec1.persprocases := [nominative];
                   PERSPROrec1.reduced      := false
             C3:   SFCATrec1.key = SFKsgnomred
             A3:   PERSPROrec1         := COPYT_bpersprotoperspro(BPERSPROrec1);
                   PERSPROrec1.persprocases := [nominative];
                   PERSPROrec1.reduced      := true
             C3:   SFCATrec1.key = SFKsgaccdat
             A3:   PERSPROrec1         := COPYT_bpersprotoperspro(BPERSPROrec1);
                   PERSPROrec1.persprocases := [accusative, dative];
                   PERSPROrec1.reduced      := false
             C3:   SFCATrec1.key = SFKsgaccdatred
             A3:   PERSPROrec1         := COPYT_bpersprotoperspro(BPERSPROrec1);
                   PERSPROrec1.persprocases := [accusative, dative];
                   PERSPROrec1.reduced      := true
           A2:   @
           C2:   BPERSPROrec1.number = plural
             C3:   SFCATrec1.key = SFKplnom
             A3:   PERSPROrec1         := COPYT_bpersprotoperspro(BPERSPROrec1);
                   PERSPROrec1.persprocases := [nominative];
                   PERSPROrec1.reduced      := false
             C3:   SFCATrec1.key = SFKplnomred
             A3:   PERSPROrec1         := COPYT_bpersprotoperspro(BPERSPROrec1);
                   PERSPROrec1.persprocases := [nominative];
                   PERSPROrec1.reduced      := true
             C3:   SFCATrec1.key = SFKplaccdat
             A3:   PERSPROrec1         := COPYT_bpersprotoperspro(BPERSPROrec1);
                   PERSPROrec1.persprocases := [accusative, dative];
                   PERSPROrec1.reduced      := false
             C3:   SFCATrec1.key = SFKplaccdatred
             A3:   PERSPROrec1         := COPYT_bpersprotoperspro(BPERSPROrec1);
                   PERSPROrec1.persprocases := [accusative, dative];
                   PERSPROrec1.reduced      := true
             C3:   SFCATrec1.key = SFKplacc
             A3:   PERSPROrec1         := COPYT_bpersprotoperspro(BPERSPROrec1);
                   PERSPROrec1.persprocases := [accusative];
                   PERSPROrec1.reduced      := false
             C3:   SFCATrec1.key = SFKpldat
             A3:   PERSPROrec1         := COPYT_bpersprotoperspro(BPERSPROrec1);
                   PERSPROrec1.persprocases := [dative];
                   PERSPROrec1.reduced      := false
           A2:   @
         A1:  @


decomp:  
         C1:   PERSPROrec1.number = singular
           C2:   (nominative in PERSPROrec1.persprocases) 
                 and (PERSPROrec1.reduced = false)
           A2:   SFCATrec1.key := SFKsgnom
           C2:   (nominative in PERSPROrec1.persprocases) 
                 and (PERSPROrec1.reduced = true)
           A2:   SFCATrec1.key := SFKsgnomred
           C2:   ([accusative, dative] * PERSPROrec1.persprocases <> []) 
                 and (PERSPROrec1.reduced = false)
           A2:   SFCATrec1.key := SFKsgaccdat
           C2:   ([accusative, dative] * PERSPROrec1.persprocases <> []) 
                 and (PERSPROrec1.reduced = true)
           A2:   SFCATrec1.key := SFKsgaccdatred
         A1:  @
         C1:   PERSPROrec1.number = plural
           C2:   (nominative in PERSPROrec1.persprocases) 
                 and (PERSPROrec1.reduced = false)
           A2:   SFCATrec1.key := SFKplnom
           C2:   (nominative in PERSPROrec1.persprocases) 
                 and (PERSPROrec1.reduced = true)
           A2:   SFCATrec1.key := SFKplnomred
           C2:   ([accusative, dative] * PERSPROrec1.persprocases <> []) 
                 and (PERSPROrec1.reduced = false)
           A2:   SFCATrec1.key := SFKplaccdat
           C2:   ([accusative, dative] * PERSPROrec1.persprocases <> []) 
                 and (PERSPROrec1.reduced = true)
           A2:   SFCATrec1.key := SFKplaccdatred
           C2:   (accusative in PERSPROrec1.persprocases) 
                 and (PERSPROrec1.reduced = false)
           A2:   SFCATrec1.key := SFKplacc
           C2:   (dative in PERSPROrec1.persprocases) 
                 and (PERSPROrec1.reduced = false)
           A2:   SFCATrec1.key := SFKpldat
         A1:  @
&
\end{verbatim}
\newpage
\subsection{Whpro's, Possadj's and Posspro's}
\subsubsection{Rule for POSSADJ's.}

Note: this rule makes -generatively speaking- POSSADJ's (like "mijn", "jouw", 
      "uw", etc.) out of BPERSPRO's. Reduced forms (like "m'n") are made too.
      The attribute 'eORenForm' has a special value for "onze".
      The attribute 'geni' is 'true' for genitive forms like: "mijner", 
      "jouwer", etc.
\begin{verbatim}
%PersproToPossadj

m1:   BPERSPRO{BPERSPROrec1}
m2:   SFCAT{SFCATrec1}
m:    POSSADJ{POSSADJrec1} [head/ BPERSPRO{BPERSPROrec1}]

comp:    
         C1:   true
           C2:   BPERSPROrec1.number = singular
             C3:   SFCATrec1.key = SFKsgpossadj
             A3:   POSSADJrec1.eORenForm  := NoForm;
                   POSSADJrec1.reduced    := false;
                   POSSADJrec1.geni       := false
             C3:   SFCATrec1.key = SFKsgpossadjred
             A3:   POSSADJrec1.eORenForm  := NoForm;
                   POSSADJrec1.reduced    := true;
                   POSSADJrec1.geni       := false
             C3:   SFCATrec1.key = SFKsgpossadjgen
             A3:   POSSADJrec1.eORenForm  := NoForm;
                   POSSADJrec1.reduced    := false;
                   POSSADJrec1.geni       := true
           A2:   @
           C2:   BPERSPROrec1.number = plural
             C3:   SFCATrec1.key = SFKplpossadj
             A3:   POSSADJrec1.eORenForm  := NoForm;
                   POSSADJrec1.reduced    := false;
                   POSSADJrec1.geni       := false
             C3:   SFCATrec1.key = SFKplpossadjonze
             A3:   POSSADJrec1.eORenForm  := eForm;
                   POSSADJrec1.reduced    := false;
                   POSSADJrec1.geni       := false
             C3:   SFCATrec1.key = SFKplpossadjons
             A3:   POSSADJrec1.eORenForm  := NoForm;
                   POSSADJrec1.reduced    := false;
                   POSSADJrec1.geni       := false
             C3:   SFCATrec1.key = SFKplpossadjgen
             A3:   POSSADJrec1.eORenForm  := NoForm;
                   POSSADJrec1.reduced    := false;
                   POSSADJrec1.geni       := true
           A2:   @
         A1:  POSSADJrec1.mood := declxpmood

decomp:  
         C1:   POSSADJrec1.mood = declxpmood
           C2:   BPERSPROrec1.number = singular
             C3:   (POSSADJrec1.eORenForm   = NoForm) 
                   and (POSSADJrec1.reduced = false) 
                   and (POSSADJrec1.geni    = false) 
             A3:   SFCATrec1.key := SFKsgpossadj
             C3:   (POSSADJrec1.eORenForm   = NoForm) 
                   and (POSSADJrec1.reduced = true ) 
                   and (POSSADJrec1.geni    = false) 
             A3:   SFCATrec1.key := SFKsgpossadjred
             C3:   (POSSADJrec1.eORenForm   = NoForm) 
                   and (POSSADJrec1.reduced = false) 
                   and (POSSADJrec1.geni    = true) 
             A3:   SFCATrec1.key := SFKsgpossadjgen
           A2:   @
           C2:   BPERSPROrec1.number = plural
             C3:   (POSSADJrec1.eORenForm   = NoForm) 
                   and (POSSADJrec1.reduced = false) 
                   and (POSSADJrec1.geni    = false) 
             A3:   SFCATrec1.key := SFKplpossadj
             C3:   (POSSADJrec1.eORenForm   = eForm) 
                   and (POSSADJrec1.reduced = false) 
                   and (POSSADJrec1.geni    = false) 
             A3:   SFCATrec1.key := SFKplpossadjonze
             C3:   (POSSADJrec1.eORenForm   = NoForm) 
                   and (POSSADJrec1.reduced = false) 
                   and (POSSADJrec1.geni    = false) 
             A3:   SFCATrec1.key := SFKplpossadjons
             C3:   (POSSADJrec1.eORenForm   = NoForm) 
                   and (POSSADJrec1.reduced = false) 
                   and (POSSADJrec1.geni    = true ) 
             A3:   SFCATrec1.key := SFKplpossadjgen
           A2:   @
         A1:   @
&
\end{verbatim}
\newpage
\subsubsection{Rule for POSSPRO's.}

Note: this rule makes POSSPRO's, like: "mijne", "jouwe", etc. out of BPERSPRO's.
      There is no POSSPRO form corresponding to "jullie"!
\begin{verbatim}
%PersproToPosspro

m1:   BPERSPRO{BPERSPROrec1}
m2:   SFCAT{SFCATrec1}
m:    POSSPRO{POSSPROrec1} [head/ BPERSPRO{BPERSPROrec1}]

comp:    
         C1:   true
           C2:   BPERSPROrec1.number = singular
             C3:   SFCATrec1.key = SFKsgposs
             A3:   POSSPROrec1.nForm   := false
             C3:   SFCATrec1.key = SFKsgpossnvorm
             A3:   POSSPROrec1.nForm   := true
           A2:   @
           C2:   BPERSPROrec1.number = plural
             C3:   SFCATrec1.key = SFKplposs
             A3:   POSSPROrec1.nForm   := false
             C3:   SFCATrec1.key = SFKplpossnvorm
             A3:   POSSPROrec1.nForm   := true
           A2:   @
         A1:  @

decomp:  
         C1:   BPERSPROrec1.number = singular
           C2:   POSSPROrec1.nForm = false
           A2:   SFCATrec1.key   := SFKsgposs
           C2:   POSSPROrec1.nForm = true
           A2:   SFCATrec1.key   := SFKsgpossnvorm
         A1:   @
         C1:   BPERSPROrec1.number = plural
           C2:   POSSPROrec1.nForm = false
           A2:   SFCATrec1.key   := SFKplposs
           C2:   POSSPROrec1.nForm = true
           A2:   SFCATrec1.key   := SFKplpossnvorm
         A1:   @
&
\end{verbatim}
\newpage
\subsubsection{Rule for WHPRO's out of BWHPRO's.}

\begin{verbatim} 
%BwhProToWhPro

m1:   BWHPRO{BWHPROrec1}
m:    WHPRO{WHPROrec1} [head/ BWHPRO{BWHPROrec1}]

comp:    
         C:   true
         A:   WHPROrec1   := COPYT_bwhprotowhpro(BWHPROrec1)

decomp:  
         C:   true
         A:   @
&
\end{verbatim}
\newpage
\subsubsection{Rule for POSSADJ's out of BWHPRO.}

Note: this rule makes "wiens" and "wier" out of "wie".
\begin{verbatim}
%WhproToPossadj

m1:   BWHPRO{BWHPROrec1}
m2:   SFCAT{SFCATrec1}
m:    POSSADJ{POSSADJrec1} [head/ BWHPRO{BWHPROrec1}]

comp:    
         C1:  true
           C2:  BWHPROrec1.number = singular
             C3:  (SFCATrec1.key = SFKpossadjwiens) 
                  and (BWHPROrec1.sexes = [masculine])
             A3:  @
             C3:  (SFCATrec1.key = SFKpossadjwier) 
                  and (BWHPROrec1.sexes = [feminine])
             A3:  @
           A2:  @
           C2:  BWHPROrec1.number = plural
             C3:  (SFCATrec1.key = SFKpossadjwier) 
                  and (BWHPROrec1.sexes = [masculine,feminine])
             A3:  @
           A2:  @
         A1:  POSSADJrec1.mood      := wh;
              POSSADJrec1.reduced   := false;
              POSSADJrec1.eORenForm := NoForm;
              POSSADJrec1.geni      := false


decomp:  
         C1:  (POSSADJrec1.mood          = wh) 
              and (POSSADJrec1.reduced   = false) 
              and (POSSADJrec1.eORenForm = NoForm) 
              and (POSSADJrec1.geni      = false)
           C2:   BWHPROrec1.number = singular
             C3:   BWHPROrec1.sexes = [masculine]
             A3:   SFCATrec1.key   := SFKpossadjwiens
             C3:   BWHPROrec1.sexes = [feminine]
             A3:   SFCATrec1.key   := SFKpossadjwier
           A2:   @
           C2:   BWHPROrec1.number = plural
             C3:   BWHPROrec1.sexes = [masculine,feminine]
             A3:   SFCATrec1.key   := SFKpossadjwier
           A2:   @
         A1:   @
&
\end{verbatim}
\newpage
\subsubsection{Rule for POSSADJ out of DEMPRO.}

Note: this rule makes "diens" out of "die".
\begin{verbatim}
%DemproToPossadj

m1:   DEMPRO{DEMPROrec1}
m2:   SFCAT{SFCATrec1}
m:    POSSADJ{POSSADJrec1} [head/ DEMPRO{DEMPROrec1}]

comp:    
         C:  (SFCATrec1.key = SFKpossadjdiens) 
             and (DEMPROrec1.sexes = [masculine])
         A:  POSSADJrec1.mood      := declxpmood;
             POSSADJrec1.reduced   := false;
             POSSADJrec1.eORenForm := NoForm;
             POSSADJrec1.geni      := false
             

decomp:  
         C:  (POSSADJrec1.mood          = declxpmood) 
             and (POSSADJrec1.reduced   = false) 
             and (POSSADJrec1.eORenForm = NoForm) 
             and (POSSADJrec1.geni      = false) 
             and (DEMPROrec1.sexes      = [masculine])
         A:  SFCATrec1.key := SFKpossadjdiens
&
\end{verbatim}
\newpage

\section{Det's and Indefpro's.}
\subsection{Rules for Det's.}
\subsubsection{Rule for DET out of BDET.}
\begin{verbatim}
%BDetToDet1

m1:   BDET{BDETrec1}
m:    DET{DETrec1} [head/ BDET{BDETrec1}]

comp:    
         C:   true

         A:   DETrec1 := COPYT_bdettodet(BDETrec1);
              DETrec1.eORenForm := NoForm

decomp:  
         C:   DETrec1.eORenForm = NoForm

         A:   @
&
\end{verbatim}
\newpage
\subsubsection{Rule for DET out of BDET with inflection.}

Note: this rule makes (for instance) "vele" and "velen" out of "veel".
Note: because all forms of the words of the category (B)DET are listed in the
      suffix-rules (in combination with the suffix-keys SFKedet and SFKendet),
      only two values for eFormation suffice: NoFormation (for BDET that don't
      have an e-form) and regeFormation (for those that do have one). The value 
      irregeFormation is not needed!
\begin{verbatim}
%BDetToDet2

m1:   BDET{BDETrec1}
m2:   SFCAT{SFCATrec1}
m:    DET{DETrec1} [head/ BDET{BDETrec1}]

comp:    
         C1:  true
            C2:  (SFCATrec1.key = SFKeDet) 
                  and (BDETrec1.eFormation = regeFormation)
            A2:  DETrec1 := COPYT_bdettodet(BDETrec1);
                 DETrec1.eORenForm := eForm
            C2:  (SFCATrec1.key = SFKenDet) 
                  and (BDETrec1.enFormation = true)
            A2:  DETrec1 := COPYT_bdettodet(BDETrec1);
                 DETrec1.eORenForm := enForm
         A1:  @


decomp:  

         C1:   true
           C2:   (DETrec1.eORenForm = eForm) 
                  and (DETrec1.eFormation = regeFormation)
           A2:   SFCATrec1.key := SFKeDet 
           C2:   (DETrec1.eORenForm = enForm) 
                  and (DETrec1.enFormation = true)
           A2:   SFCATrec1.key := SFKenDet 
         A1:   @
&
\end{verbatim}
\newpage
\subsection{Rules for Indefpro's.}
\subsubsection{Rule for INDEFPRO out of BINDEFPRO.}
\begin{verbatim}
%Bindefprotoindefpro1

m1:   BINDEFPRO{BINDEFPROrec1}
m:    INDEFPRO{INDEFPROrec1} [head/ BINDEFPRO{BINDEFPROrec1}]

comp:    
         C:   true

         A:   INDEFPROrec1 := COPYT_bindefprotoindefpro(BINDEFPROrec1);
              INDEFPROrec1.geni := false 

decomp:  
         C:   INDEFPROrec1.geni = false 

         A:   @
&
\end{verbatim}
\newpage
\subsubsection{Rule for INDEFPRO out of BINDEFPRO with genitive.}

Note: this rule makes (for instance) "iemands" out of "iemand".
\begin{verbatim}
%Bindefprotoindefpro2

m1:   BINDEFPRO{BINDEFPROrec1}
m2:   SFCAT{SFCATrec1}
m:    INDEFPRO{INDEFPROrec1} [head/ BINDEFPRO{BINDEFPROrec1}]

comp:    
         C:  (SFCATrec1.key = SFKgens) 
              and (BINDEFPROrec1.possgeni = true)
         A:  INDEFPROrec1 := COPYT_bindefprotoindefpro(BINDEFPROrec1);
             INDEFPROrec1.geni := true


decomp:  

         C:  (INDEFPROrec1.geni = true) 
              and (INDEFPROrec1.possgeni = true)
         A:  SFCATrec1.key := SFKgens
&
\end{verbatim}


%DUTCH3
%&

\newpage
\section{Adjectives and adverbs.}
\subsection{Adjectives.}
\subsubsection{Rule for adjectives without derivational affixes.}
\begin{verbatim}
%adjbtosub

m1:   BADJ{BADJrec1}
m:    SUBADJ{SUBADJrec1} [head/ BADJ{BADJrec1}]

comp:    
         C:   true
         A:   SUBADJrec1 := COPYT_badjtosubadj(BADJrec1)

decomp:  
         C:   true
         A:   @
&
\end{verbatim}
\newpage
\subsubsection{Rule for the positive form of adjectives.}
\begin{verbatim}
%adjposvorm

m1:   SUBADJ{SUBADJrec1}[mu1]
m:    ADJ{ADJrec1} [head/ SUBADJ{SUBADJrec1}[mu1]]

comp:    
         C:   true
         A:   ADJrec1           := COPYT_subadjtoadj(SUBADJrec1);
              ADJrec1.form      := positive;
              ADJrec1.eORenForm := NoForm

decomp:  
         C:   (ADJrec1.eORenForm = NoForm)
              and (ADJrec1.form  = positive)
         A:   @
&
\end{verbatim}
\newpage
\subsubsection{Rule for the "-s"-form of the positive of adjectives.}
\begin{verbatim}
%adjsPositive

m1:   SUBADJ{SUBADJrec1}[mu1]
m2:   SFCAT{SFCATrec1}
m:    ADJ{ADJrec1} [head/ SUBADJ{SUBADJrec1}[mu1]]

comp:    
         C:   (SFCATrec1.key = SFKadjs) 
              and (SUBADJrec1.sFormation = true)
         A:   ADJrec1           := COPYT_subadjtoadj(SUBADJrec1);
              ADJrec1.form      := sPositive;
              ADJrec1.eORenForm := NoForm

decomp:  
         C:   (ADJrec1.form           = sPositive)
              and (ADJrec1.eORenForm  = NoForm)
              and (ADJrec1.sFormation = true)
         A:   SFCATrec1.key := SFKadjs
&
\end{verbatim}
\newpage
\subsubsection{Rule for the comparative of adjectives.}
\begin{verbatim}
%adjcompvorm

m1:   SUBADJ{SUBADJrec1}[mu1]
m2:   SFCAT{SFCATrec1}
m:    ADJ{ADJrec1} [head/ SUBADJ{SUBADJrec1}[mu1]]

comp:    
         C1: true
           C2: (SFCATrec1.key = SFKer) 
               and (erComp in SUBADJrec1.comparatives)
           A2: ADJrec1  := COPYT_subadjtoadj(SUBADJrec1)
           C2: (SFCATrec1.key = SFKonreger) 
               and (erIrregComp in SUBADJrec1.comparatives)
           A2: ADJrec1  := COPYT_subadjtoadj(SUBADJrec1)
         A1:  ADJrec1.form      := comparative;
              ADJrec1.eORenForm := NoForm

decomp:  
         C1: (ADJrec1.form          = comparative) 
             and (ADJrec1.eORenForm = NoForm)
           C2:  erComp in ADJrec1.comparatives
           A2:  SFCATrec1.key := SFKer
           C2:  erIrregComp in ADJrec1.comparatives
           A2:  SFCATrec1.key := SFKonreger
         A1:   @
&
\end{verbatim}
\newpage
\subsubsection{Rule for "-s"-form of comparatives.}
\begin{verbatim}
%adjsComparative

m1:   SUBADJ{SUBADJrec1}[mu1]
m2:   SFCAT{SFCATrec1}
m3:   SFCAT{SFCATrec2}
m:    ADJ{ADJrec1} [head/ SUBADJ{SUBADJrec1}[mu1]]

comp:    
         C1: SFCATrec2.key = SFKadjs
           C2: (SFCATrec1.key = SFKer) 
               and (erComp in SUBADJrec1.comparatives)
           A2: ADJrec1  := COPYT_subadjtoadj(SUBADJrec1)
           C2: (SFCATrec1.key = SFKonreger)
               and (erIrregComp in SUBADJrec1.comparatives)
           A2: ADJrec1  := COPYT_subadjtoadj(SUBADJrec1)
         A1: ADJrec1.form      := sComparative;
             ADJrec1.eORenForm := NoForm

decomp:  
         C1:  (ADJrec1.form          = sComparative) 
              and (ADJrec1.eORenForm = NoForm)
           C2:  erComp in ADJrec1.comparatives
           A2:  SFCATrec1.key := SFKer
           C2:  erIrregComp in ADJrec1.comparatives
           A2:  SFCATrec1.key := SFKonreger
         A1:  SFCATrec2.key := SFKadjs
&
\end{verbatim}
\newpage
\subsubsection{Rule for the superlative of adjectives.}
\begin{verbatim}
%adjsupvorm

m1:   SUBADJ{SUBADJrec1}[mu1]
m2:   SFCAT{SFCATrec1}
m:    ADJ{ADJrec1} [head/ SUBADJ{SUBADJrec1}[mu1]]

comp:    
         C1: true
           C2: (SFCATrec1.key = SFKst) 
               and (stSup in SUBADJrec1.superlatives)
           A2: ADJrec1  := COPYT_subadjtoadj(SUBADJrec1)
           C2: (SFCATrec1.key = SFKonregst) 
               and (stIrregSup in SUBADJrec1.superlatives)
           A2: ADJrec1  := COPYT_subadjtoadj(SUBADJrec1)
          A: ADJrec1.form      := superlative;
             ADJrec1.eORenForm := NoForm

decomp:  
        C1: (ADJrec1.form = superlative) 
            and (ADJrec1.eORenForm = NoForm)
          C2:  stSup in ADJrec1.superlatives
          A2:  SFCATrec1.key := SFKst
          C2:  stIrregSup in ADJrec1.superlatives
          A2:  SFCATrec1.key := SFKonregst
        A1:  @
&
\end{verbatim}
\newpage
\subsubsection{Rule for the "aller"-superlative of adjectives.}
\begin{verbatim}
%adjallersupvorm

m1:   PFCAT{PFCATrec1}
m2:   SUBADJ{SUBADJrec1}[mu1]
m3:   SFCAT{SFCATrec1}
m:    ADJ{ADJrec1} [head/ SUBADJ{SUBADJrec1}[mu1]]

comp:    
         C1: PFCATrec1.key   = PFKaller
           C2: (SFCATrec1.key = SFKst) 
               and (allerSup in SUBADJrec1.superlatives)
           A2: ADJrec1  := COPYT_subadjtoadj(SUBADJrec1)
           C2: (SFCATrec1.key = SFKonregst) 
               and (allerIrregSup in SUBADJrec1.superlatives)
           A2: ADJrec1  := COPYT_subadjtoadj(SUBADJrec1)
         A: ADJrec1.form      := allerSuperlative;
            ADJrec1.eORenForm := NoForm

decomp:  
         C1:  (ADJrec1.form          = allerSuperlative) 
              and (ADJrec1.eORenForm = NoForm)
           C2:  allerSup in ADJrec1.superlatives
           A2:  SFCATrec1.key := SFKst
           C2:  allerIrregSup in ADJrec1.superlatives
           A2:  SFCATrec1.key := SFKonregst
         A1:  PFCATrec1.key := PFKaller
&
\end{verbatim}
\newpage
\subsubsection{The "-e"-form of adjectives.}

Note:  the "e"-form can be made out of: positive, 
   comparative, superlative and allersuperlative of adjectives.

Note: attributively used adjectives can get "-e"; nominalised adjectives can get
      "-e" or "-en". The inherent attribute "uses" tells us whether or not 
      adjectives can be used attributively or nominalised.

\begin{verbatim}
%adjEform

m1:   ADJ{ADJrec1}[mu1]
m2:   SFCAT{SFCATrec1}
m:    ADJ{ADJrec1}[mu1]

comp:    
         C1:  ADJrec1.eORenForm = NoForm
           C2:  (ADJrec1.form     = comparative) 
                 or (ADJrec1.form = superlative) 
                 or (ADJrec1.form = allersuperlative)
             C3: SFCATrec1.key = SFKe
               C4: (attributive in ADJrec1.uses) 
                    or (nominalised in ADJrec1.uses)
               A4:  @
             A3:  @
           A2:  @
           C2:  ADJrec1.form = positive 
             C3:  (SFCATrec1.key = SFKe)
                  and (   ((ADJrec1.eFormation = regeFormation)
                           and (attributive in ADJrec1.uses))
                      or  ((ADJrec1.eNominalised = regeNominalised)
                           and (nominalised in ADJrec1.uses))   )
             A3:  @
             C3:  (SFCATrec1.key = SFKonrege)
                  and (   ((ADJrec1.eFormation = irregeFormation)
                           and (attributive in ADJrec1.uses))
                      or  ((ADJrec1.eNominalised = irregeNominalised)
                           and (nominalised in ADJrec1.uses))   )
             A3:  @
           A2:  @
         A1:  ADJrec1.eORenForm := eForm

decomp:  
         C1: ADJrec1.eORenForm = eForm
           C2:  (ADJrec1.form     = comparative) 
                 or (ADJrec1.form = superlative) 
                 or (ADJrec1.form = allersuperlative)
             C3: (attributive in ADJrec1.uses) 
                  or (nominalised in ADJrec1.uses) 
             A3:  @
           A2:  SFCATrec1.key := SFKe
           C2:  ADJrec1.form = positive 
             C3:  ((attributive in ADJrec1.uses) 
                    and (ADJrec1.eFormation   = irregeFormation))
                  or
                  ((nominalised in ADJrec1.uses) 
                    and (ADJrec1.eNominalised = irregeNominalised))
             A3:  SFCATrec1.key := SFKonrege
             C3:  ((attributive in ADJrec1.uses) 
                   and (ADJrec1.eFormation   = regeFormation))
                  or
                  ((nominalised in ADJrec1.uses)
                   and (ADJrec1.eNominalised = regeNominalised))
             A3: SFCATrec1.key := SFKe 
           A2:  @
         A1:  ADJrec1.eORenForm := NoForm
&
\end{verbatim}
\newpage
\subsubsection{The "-en"-form of adjectives.}
\begin{verbatim}
%adjENform

m1:   ADJ{ADJrec1}[mu1]
m2:   SFCAT{SFCATrec1}
m:    ADJ{ADJrec1}[mu1]

comp: 
        C1:  ADJrec1.eORenForm = eForm
           C2:  (nominalised in ADJrec1.uses) 
                and (SFCATrec1.key = SFKen) 
             C3:  (ADJrec1.form     = positive) 
                   or (ADJrec1.form = comparative)
                   or (ADJrec1.form = superlative) 
                   or (ADJrec1.form = allersuperlative)
             A3:  @
           A2:  @
         A1:  ADJrec1.eORenForm := enForm

decomp:  
         C1: (nominalised in ADJrec1.uses) 
             and (ADJrec1.eORenForm = enForm) 
           C2: (ADJrec1.eNominalised     = regeNominalised)
                or (ADJrec1.eNominalised = irregeNominalised)
             C3: (ADJrec1.form     = positive) 
                  or (ADJrec1.form = comparative) 
                  or (ADJrec1.form = superlative) 
                  or (ADJrec1.form = allersuperlative)
             A3: @
           A2: @
         A1: SFCATrec1.key     := SFKen;
             ADJrec1.eORenForm := eForm
&
\end{verbatim}
\newpage
\subsection{Adverbs.}
\subsubsection{Rule for adverbs without derivational affixes.}
\begin{verbatim}
%badvtosubadv

m1:   BADV{BADVrec1}
m:    SUBADV{SUBADVrec1} [head/ BADV{BADVrec1}]

comp:    
         C: true
         A: SUBADVrec1 := COPYT_badvtosubadv(BADVrec1)

decomp:  
         C: true
         A: @
&
\end{verbatim}
\newpage
\subsubsection{Rule for the derivation of adverbs from adjectives.}

\begin{verbatim}
%AdvFromAdj

m1: SUBADJ{SUBADJrec1}[mu1]

m:  SUBADV{SUBADVrec1}[complrel/SUBADJ{SUBADJrec1}[mu1], 
                        head/BADVSUFF(adjadvBADVSUFFkey){BADVSUFFrec1}]

comp:

C1: true
   C2: SUBADJrec1.possadv = true
   A2: SUBADVrec1.lastaffix     := advaffix
A1: SUBADVrec1.req          := SUBADJrec1.req;  
    SUBADVrec1.env          := SUBADJrec1.env;  
    SUBADVrec1.comparatives := SUBADJrec1.comparatives;
    SUBADVrec1.superlatives := SUBADJrec1.superlatives;
    SUBADVrec1.subcs       := BADVSUFFrec1.subcs;    
    SUBADVrec1.Qstatus     := BADVSUFFrec1.Qstatus;  
    SUBADVrec1.class        := SUBADJrec1.class;
    SUBADVrec1.deixis       := SUBADJrec1.deixis;
    SUBADVrec1.aspect       := SUBADJrec1.aspect;
    SUBADVrec1.retro        := SUBADJrec1.retro;
    SUBADVrec1.mood        := BADVSUFFrec1.mood;    
    SUBADVrec1.thetaadv    := BADVSUFFrec1.thetaadv;    
    SUBADVrec1.advpatterns := BADVSUFFrec1.advpatterns;    
    SUBADVrec1.prepkey     := BADVSUFFrec1.prepkey;    
    SUBADVrec1.temporal    := BADVSUFFrec1.temporal;
    SUBADVrec1.possnietnp  := BADVSUFFrec1.possnietnp;    
    SUBADVrec1.thanas      := BADVSUFFrec1.thanas;    
    SUBADVrec1.Radvb           := false

decomp:

C1: (SUBADVrec1.req          = SUBADJrec1.req)          and
    (SUBADVrec1.env          = SUBADJrec1.env)          and
    (SUBADVrec1.comparatives = SUBADJrec1.comparatives) and
    (SUBADVrec1.superlatives = SUBADJrec1.superlatives) and
    (SUBADVrec1.subcs        = BADVSUFFrec1.subcs)        and
    (SUBADVrec1.Qstatus      = BADVSUFFrec1.Qstatus)      and
    (SUBADVrec1.class        = SUBADJrec1.class)        and
    (SUBADVrec1.deixis       = SUBADJrec1.deixis)       and
    (SUBADVrec1.aspect       = SUBADJrec1.aspect)       and
    (SUBADVrec1.retro        = SUBADJrec1.retro)        and
    (SUBADVrec1.mood         = BADVSUFFrec1.mood)         and
    (SUBADVrec1.thetaadv     = BADVSUFFrec1.thetaadv)     and
    (SUBADVrec1.advpatterns  = BADVSUFFrec1.advpatterns)  and
    (SUBADVrec1.prepkey      = BADVSUFFrec1.prepkey)      and
    (SUBADVrec1.temporal     = BADVSUFFrec1.temporal)     and
    (SUBADVrec1.possnietnp   = BADVSUFFrec1.possnietnp)   and
    (SUBADVrec1.thanas       = BADVSUFFrec1.thanas)       and
    (SUBADVrec1.Radvb        = false)  
   C2:  (SUBADJrec1.possadv    = true)   and
        (SUBADVrec1.lastaffix  = advaffix)                  
   A2: @
A2: @
&
\end{verbatim}
\newpage
\subsubsection{Rule for positive form of adverbs.}
\begin{verbatim}
%advposvorm

m1:   SUBADV{SUBADVrec1}[mu1]
m:    ADV{ADVrec1} [head/ SUBADV{SUBADVrec1}[mu1]]

comp:     
         C:   true
         A:   ADVrec1       := COPYT_subadvtoadv(SUBADVrec1);
              ADVrec1.eORenForm  := NoForm;
              ADVrec1.form       := positive

decomp:  
         C:   (ADVrec1.eORenForm = NoForm)
              and (ADVrec1.form  = positive)
         A:   @
&
\end{verbatim}
\newpage
\subsubsection{Rule for comparative forms of adverbs.}
\begin{verbatim} 
%advcompvorm

m1:   SUBADV{SUBADVrec1}[mu1]
m2:   SFCAT{SFCATrec1}
m:    ADV{ADVrec1} [head/ SUBADV{SUBADVrec1}[mu1]]

comp:    
         C1:  true
           C2:  (SFCATrec1.key = SFKer) 
                 and (erComp in SUBADVrec1.comparatives)
           A2:  ADVrec1  := COPYT_subadvtoadv(SUBADVrec1)
           C2:  (SFCATrec1.key = SFKonreger) 
                 and (erIrregComp in SUBADVrec1.comparatives) 
           A2:  ADVrec1  := COPYT_subadvtoadv(SUBADVrec1)
         A1:  ADVrec1.eORenForm  := NoForm;
              ADVrec1.form       := comparative

decomp:  
         C1:  (ADVrec1.eORenForm = NoForm) 
               and (ADVrec1.form = comparative)
           C2:  erComp in ADVrec1.comparatives
           A2:  SFCATrec1.key := SFKer
           C2:  erIrregComp in ADVrec1.comparatives
           A2:  SFCATrec1.key := SFKonreger
         A1:   @
&
\end{verbatim}
\newpage
\subsubsection{Rule for superlatives of adverbs.}
\begin{verbatim} 
%advsupvorm

m1:   SUBADV{SUBADVrec1}[mu1]
m2:   SFCAT{SFCATrec1}
m:    ADV{ADVrec1} [head/ SUBADV{SUBADVrec1}[mu1]]

comp:    
         C1:  true
           C2:  (SFCATrec1.key = SFKst) 
                 and (stSup in SUBADVrec1.superlatives)
           A2:  ADVrec1  := COPYT_subadvtoadv(SUBADVrec1)
           C2:  (SFCATrec1.key = SFKonregst)
                 and (stIrregSup in SUBADVrec1.superlatives)
           A2:  ADVrec1  := COPYT_subadvtoadv(SUBADVrec1)
         A1:  ADVrec1.eORenForm  := NoForm; 
              ADVrec1.form       := superlative

decomp:  
        C1:  (ADVrec1.eORenForm = NoForm)
              and (ADVrec1.form = superlative)
           C2:  stSup in ADVrec1.superlatives
           A2:  SFCATrec1.key := SFKst
           C2:  stIrregSup in ADVrec1.superlatives
           A2:  SFCATrec1.key := SFKonregst
         A:  @
&
\end{verbatim}
\newpage
\subsubsection{Rule for the "aller"-superlative of adverbs.}
\begin{verbatim} 
%advallersupvorm

m1:   PFCAT{PFCATrec1}
m2:   SUBADV{SUBADVrec1}[mu1]
m3:   SFCAT{SFCATrec1}
m:    ADV{ADVrec1} [head/ SUBADV{SUBADVrec1}[mu1]]

comp:    
         C1:  PFCATrec1.key = PFKaller
           C2:  (SFCATrec1.key = SFKst) 
                 and (allerSup in SUBADVrec1.superlatives)
           A2:  ADVrec1  := COPYT_subadvtoadv(SUBADVrec1)
           C2:  (SFCATrec1.key = SFKonregst) 
                 and (allerIrregSup in SUBADVrec1.superlatives)
           A2:  ADVrec1  := COPYT_subadvtoadv(SUBADVrec1)
         A1:  ADVrec1.eORenForm  := NoForm;
              ADVrec1.form       := allersuperlative

decomp:  
         C1:  (ADVrec1.eORenForm = NoForm) 
               and (ADVrec1.form = allersuperlative)
           C2:  allerSup in ADVrec1.superlatives
           A2:  SFCATrec1.key := SFKst
           C2:  allerIrregSup in ADVrec1.superlatives
           A2:  SFCATrec1.key := SFKonregst
         A1:  PFCATrec1.key := PFKaller
&
\end{verbatim}
\newpage
\subsubsection{The "-e"-form of adverbs.}

Note: adverbs have eForm in the (aller)superlative, like "het beste", "het 
      liefste" (from: graag). They never have enForm. I assume that every
      adverb that can have (aller)superlative also can have the eForm of 
      this (aller)superlative, and that this will be made regular (with
      SFKe) because the string ends in -st.

\begin{verbatim} 
%adveform

m1:   ADV{ADVrec1}[mu1]
m2:   SFCAT{SFCATrec1}
m:    ADV{ADVrec1}[mu1]

comp:    
         C1:  (ADVrec1.eORenForm = NoForm) 
              and (SFCATrec1.key = SFKe)
           C2:  (ADVrec1.form    = superlative) 
                or (ADVrec1.form = allersuperlative)
           A2:  @
         A1:  ADVrec1.eORenForm := eForm

decomp:  
         C1: ADVrec1.eORenForm = eForm
           C2:  (ADVrec1.form    = superlative) 
                or (ADVrec1.form = allersuperlative)
           A2:  @
         A1: SFCATrec1.key     := SFKe; 
             ADVrec1.eORenForm := NoForm
&
\end{verbatim} 
\end{document}
