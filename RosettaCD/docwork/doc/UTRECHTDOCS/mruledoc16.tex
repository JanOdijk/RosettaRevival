\documentstyle{Rosetta}
\begin{document}
   \RosTopic{Rosetta3.doc.Mrules.English}
   \RosTitle{Rosetta3 English M-rules: ExistPropFormation}
   \RosAuthor{Margreet Sanders}
   \RosDocNr{395}
   \RosDate{\today}
   \RosStatus{concept}
   \RosSupersedes{-}
   \RosDistribution{Project}
   \RosClearance{Project}
   \RosKeywords{English, documentation, Mrules, Existential Propositions}
   \MakeRosTitle
%
%

\section{ExistPropFormation}
The current document describes the contents of 
the subgrammar that forms Existential Propositions, {\bf ExistPropFormation}.
This subgrammar is an alternative to two other subgrammars that form 
propositions around nominal constituents, viz.\ {\bf IdentPropFormation} and {
\bf NPPROPformation}. These two grammars are described in docs.\ 396, {\em 
Rosetta3 English Mrules: IdentPropFormation\/}, and 392, {\em Rosetta3 English 
Mrules: NPPROPformation\/}, respectively. 

The export of all three grammars is 
an NPPROP, to be used as head of the XPPROPtoCLAUSE subgrammar, where the 
copula {\em be\/} is introduced and the NPPROP-node is pruned again. 
The difference between the grammars is to be 
found in their heads and import: in {\bf ExistPropFormation}, the head of the 
subgrammar is an indefinite NPVAR, which is turned into an NPPROP all 
on its own.
A dummy subject {\em There\/} will be inserted later, in TC\_ObjectOKrules of 
the sentence grammar. In {\bf IdentPropFormation}, the head is again an NPVAR 
(or a SENTENCEVAR in special occasions), but it is combined with a specific 
subject NP (a member of the set {\em it, this, that, they, these, those\/}) in 
the startrules (or, again in special occasions, with an NPVAR). 
In {\bf NPPROPformation\/}, the head of the subgrammar is a 
full NP, which combines with any subject variable.

The three subgrammars forming nominal propositions are not isomorphic with any 
other subgrammars, and contain only the minimum amount of rules necessary 
to make the 
NPPROP an acceptable input to the XPPROPtoCLAUSE subgrammar. Next to the 
startrules, there only are a few rule classes to provide the various variables 
needed. In doc.\ 150, {\em Subgrammars of English\/}, in which the general 
lay-out of the Rosetta3 system as devised in the definition phase of the 
project was presented, it was assumed that the copula {\em be\/} would be 
the head of the ExistPropFormation subgrammar; this was necessary 
because the proposition was to be input to the VERBPPROPtoCLAUSE subgrammar
(no general XPPROPtoCLAUSE subgrammar was planned at that time). Also, it was 
thought that the subgrammar would contain PreVPmodrules. However, these rules 
have not even been written for the VERBPPROPformation subgrammar, and they are 
not included here either.

As any subgrammar, {\bf ExistPropFormation} consists of 
a number of rule classes. A rule class in its turn
consists of a number of rules. The relative ordering of the rules in the
(sub)grammar is indicated by a {\em control expression}. A summary of this
control expression (i.e.\ a listing of the ordering of the rule classes, 
without explicit mentioning of the rules themselves) is also included here, 
and the initial (= head) and export categories are given. 

In the section on the rules, only the rule names are given, 
but not the exact rule formulation. What is attempted 
is to provide a detailed overview of the workings of the subgrammar, and 
how the different rule classes achieve this. For every rule, an 
example is given. If it is uncertain whether the example is correct (either 
because it may not be an example of the phenomenon in question, or because it 
may not be correct English), it is preceded by a question mark. Note that all 
explanation of rules and transformations is given from a generative viewpoint
only, unless explicitly stated otherwise. Often, the information given in this 
document is based strongly on the comment already present in the documentation 
of the rules themselves. Discrepancies between what is stated here and what is 
said in the rule itself are usually caused by the fact that the rule file has 
not  been updated, although insights have changed. The semantics of the rules 
has been left unspecified in the current documentation, since it is not at all 
clear.

Finally note that the rules described in this document have NOT been tested 
properly. English analysis is not possible yet (there is no Surface Parser), and 
English generation has only been tested in as far as the construction was the 
translation of a Dutch sentence to be tested.

\newpage
\section{Subgrammar Specification}
The subgrammar definition can be found in the file which also contains all the 
rules of this subgrammar, {\bf ExistPropFormation.mrule}, which is 
{\em mrules99.mrule\/}.

\begin{verbatim}
%SUBGRAMMAR ExistPropFormation


   ( RC_StartExistential )
.  { RC_ExistTempAdvVar }
.  { RC_ExistSentAdvVar }
.  { RC_ExistLocAdvVar }

\end{verbatim}

\begin{description}
  \item[Head]  NPVAR \ \ \ \ BASIC EXPRESSION
  \item[Export] NPPROP
  \item[Import] PREPPVAR, ADVPVAR, SENTENCEVAR
\end{description}

\newpage
\section{Rules and Transformations}

\subsection{RC\_StartExistential}
\begin{description}
\item[Kind] Obligatory Rule Class
\item[Task] To build an NPPROP around an indefinite NPVAR, making the NPVAR the 
direct object (there is no predicate). Also, the Aktionsarts of the NPPROP are 
set to {\em [stative]\/}; no separate transformation classs is needed to 
determine this standard value.

\vspace{1 cm}
\begin{description}
\item[Name] RStartExist
\item[Task] see above
\item[File] english:ExistPropFormation.mrule (mrules99.mrule)
\item[Semantics]
\item[Example] x1 $\rightarrow$ x1 ((there are) books)
\item[Remarks]
\end{description}

\end{description}

\newpage
\subsection{RC\_ExistTempAdvVar}
\begin{description}
\item[Kind] Iterative Rule Class
\item[Task] To introduce a variable for a time adverbial. The variable may be 
for a sentence, a prepp or an advp. The rule class is iterative, but only one 
rule has been written (no retrospective or durational temporals are allowed as 
yet), and this one rule may be applied only once. Perhaps other rules are 
needed too: {\em There have been wars before; There were junkies in that 
building for six years}.

\vspace{1 cm}
\begin{description}
\item[Name] RExistRefVarInsert
\item[Task] To introduce a variable for a referential time adverbial that is 
not retrospective.
\item[File] english:ExistPropFormation.mrule (mrules99.mrule)
\item[Semantics]
\item[Example] x1 + refVAR $\rightarrow$ x1 refVAR (There were six books in 
all)
\item[Remarks]
\end{description}

\end{description}

\newpage
\subsection{RC\_ExistSentAdvvar}
\begin{description}
\item[Kind] Iterative Rule Class
\item[Task] To introduce variables for adverbial subordinate sentences in 
different positions, and sentence or causal adverbials in initial position. The 
conjunction may also be a preposition. No rules have been written yet for 
abstract conjunctions.

The rules are in an iterative class, but have been written in such a way that 
only one order of application is possible. This to prevent unnecessary 
ambiguities in analysis. Also, the IL strategy in mapping the different 
conjsent rules is to preserve the surface order used in the source language as 
much as possible, because it may be of importance for pronominal reference.

\vspace{1 cm}
\begin{description}
\item[Name] RExistConjsentVar
\item[Task] To introduce a variable for an adverbial subordinate sentence (or 
sentential PREPP) in initial (leftdislocrel) position.
\item[File] english:ExistPropFormation.mrule (mrules99.mrule)
\item[Semantics]
\item[Example] \mbox{}\\
x1 + advSENTENCEVAR $\rightarrow$ advSENTENCEVAR x1 \\
(Although they do not exist anymore, there were unicorns once)\\
x1 + advPREPPVAR $\rightarrow$ advPREPPVAR x1 candidate\\
(Without wanting to make you nervous, there is a hacker trying to log in on the 
system)
\item[Remarks] No comma is introduced at the end of the adverbial sentence, 
although it is probably obligatory. This will have to be added.
\end{description}

\vspace{1 cm}
\begin{description}
\item[Name] RExistFinalConjsentVar
\item[Task] To introduce a variable for an adverbial subordinate sentence (or 
sentential PREPP) in final (postsentadvrel) position.
\item[File] english:ExistPropFormation.mrule (mrules99.mrule)
\item[Semantics]
\item[Example] \mbox{}\\
x1 + advSENTENCEVAR $\rightarrow$ x1 advSENTENCEVAR \\
(There were unicorns once, although they do not exist anymore)\\
x1 + advPREPPVAR $\rightarrow$ x1 advPREPPVAR \\
(? There is no victim, without considering all the details of the crime)
\item[Remarks] 
\end{description}

\vspace{1 cm}
\begin{description}
\item[Name] RExistSentadvVar
\item[Task] To introduce a variable for a causal or sentence adverbial or a 
causal prepp in 
initial position. No rules have been written to account for any other position 
of the adverbial, or to relate its position to that of an adverbial or temporal 
sentence also present in the clause.
\item[File] english:ExistPropFormation.mrule (mrules99.mrule)
\item[Semantics]
\item[Example] \mbox{}\\
x1 + sentADVVAR $\rightarrow$ sentADVVAR x1 \\
(Probably, there (still) are unicorns)\\
x1 + causPREPPVAR $\rightarrow$ causPREPPVAR x1 \\
(For that reason, there are no leprechauns (anymore))
\item[Remarks] No comma is introduced at the end of the adverbial, 
although it is probably obligatory. This will have to be added.
\end{description}

\end{description}

\newpage
\subsection{RC\_ExistLocAdvVar}
\begin{description}
\item[Kind] Iterative Rule Class
\item[Task] To introduce variables for non-argument locatives (ADVP or PREPP) 
at a fixed position (locadvrel) following the NP. No rules have been 
written to account for any other position 
of the locative, or to relate its position to that of an adverbial or temporal 
sentence also present in the clause.

\vspace{1 cm}
\begin{description}
\item[Name] RExistLocAdvVar
\item[Task] To introduce a variable for a non-argument locative ADVP 
\item[File] english:ExistPropFormation.mrule (mrules99.mrule)
\item[Semantics]
\item[Example] x1 + locADVVAR $\rightarrow$ x1 locADVVAR
((There are) books there)
\item[Remarks]
\end{description}

\vspace{1 cm}
\begin{description}
\item[Name] RExistLocPreppVar
\item[Task] To introduce a variable for a non-argument locative PREPP
\item[File] english:ExistPropFormation.mrule (mrules99.mrule)
\item[Semantics]
\item[Example] x1 + locADVVAR $\rightarrow$ x1 locADVVAR
((There was) some commotion at the club)
\item[Remarks]
\end{description}

\end{description}

\end{document}

