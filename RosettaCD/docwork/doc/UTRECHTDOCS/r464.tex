\documentstyle{Rosetta}
\begin{document}
   \RosTopic{General}
   \RosTitle{Verslag Rosetta Evaluatiedag 21-3-1991}
   \RosAuthor{Alle Projectleden}
   \RosDocNr{464}
   \RosDate{16-05-1991}
   \RosStatus{approved}
   \RosSupersedes{-}
   \RosDistribution{Project}
   \RosClearance{Project}
   \RosKeywords{evaluation, minutes}
   \MakeRosTitle

\hyphenation{pro-ject we-ten-schap we-ten-schap-pe-lijk on-der-zoeks-thema}

\noindent
{\bf Rosetta Evaluatiedag}
\begin{itemize}
  \item {\bf datum}: donderdag 21 maart 1991
  \item {\bf tijd}: 9.15 - 17.00 uur
  \item {\bf plaats}: Academisch Genootschap, Parklaan 93, Eindhoven
  \item {\bf aanwezig}: Andr\'{e} Schenk, Jan Landsbergen, Lisette Appelo,
                     Franciska de Jong, Petra de Wit, 
                     Elena Pinillos, Joep Rous, Jan Odijk, Harm Smit,
                     Ren\'{e} Leermakers, Frank Uittenbogaard.

  \item {\bf Agenda}:
    \begin{enumerate}
       \item  9.15 uur : Geschiedenis van het project
       \item 10.15 uur : Resultaten van het project
       \begin{itemize}
       \begin{enumerate}
       \item Linguistische aspecten
       \item Software aspecten
       \end{enumerate}
       \end{itemize}
       \item 12.00 uur : Algemene Problematiek
       \begin{itemize}
       \begin{enumerate}
       \item Complexiteit
       \item Inwerken
       \item Compositioneel vertalen
       \end{enumerate}
       \end{itemize}
       \item 15.00 uur : Technische problemen
       \begin{itemize}
       \begin{enumerate}
       \item Isomorfie, transfer, omkeerbaarheid
       \item Regelnotatie
       \end{enumerate}
       \end{itemize}
       \item 16.00 uur : Hoe Verder ?
    \end{enumerate}
\end{itemize}

\section {Geschiedenis van het project}
Inleider: Jan L.\\
Notulist: Joep\\

Jan L. bespreekt de geschiedenis van het Rosetta project. In de periode 
augustus 85 - februari 86 werden de oorspronkelijke plannen
enigszins gewijzigd, e.e.a. werd ambitieuzer aangepakt. Als reden daarvoor
werd gegeven:
\begin{itemize}
\item Men was meer geneigd dingen beter te doen dan simpele phrase vertaler
     (Rosetta2) uit te breiden.
\item Men was nog niet produkt-gericht bezig
\item Men was op zoek naar de ideale methode ( in tegenstelling tot een
      "goed is goed genoeg" houding)
\item Uitbreiding van Rosetta2 was niet goed mogelijk, er waren structurele
      problemen, bijv. t.a.v. de behandeling van scope en tijd.
\end{itemize}

Achteraf gezien was het verstandiger geweest om met een kleinere groep te
beginnen. Deze groep had het linguistisch framework moeten opzetten.
Ook
was het beter geweest als de linguistische tools beschikbaar waren geweest
(bijv. de woordenboek editor) op het moment dat het project op grote schaal 
van start ging. Om tot de
goede tools te komen hadden de informatici  Rosetta2 kunnen gebruiken
als experimenteervehikel.

\section {Resultaten van het project}
\subsection{Linguistische Aspecten}

Inleider: Lisette\\
Notulist: Franciska\\
Leesstof: Doc. 0461\\


Er zijn op z'n minst de volgende twee manieren om een project als Rosetta 
te evalueren:

\begin{enumerate}
  \item De oorspronkelijke doelstellingen vergelijken met wat er nu is.
Daartoe kan gebruik gemaakt worden van de zogenaamde contructielijst, doc.nr. 
0081. 
  \item Het huidige systeem (N-N) toepassen op het Eurotra/Flickinger corpus. 
Dit corpus is een bij Eurotra gemaakte 
Nederlandse versie van een testbank voor het Engels die is
opgezet door mensen van Hewlett-Packard Laboratories. 
\end{enumerate}


Bij doc.nr. 461, het stuk waarin Lisette verslag doet van  
de resultaten van Rosetta voor het Flickinger-corpus, zijn de volgende 
kanttekeningen te maken.

\begin{itemize}
  \item De testbank is sterk lingu\"{\i}stisch georienteerd: de testzinnen beogen 
steeds een bepaalde lingu\"{\i}stische constructie te testen. 
  \item Zinnen die bedoeld zijn om een bepaald onderwerp te testen 
kunnen vastlopen op 
een ander lingu\"{\i}stisch fenomeen.  Zo is 23\% van de niet correct geanalyseerde
zinnen toe te schrijven aan co\"{o}rdinatieverschijnselen die Rosetta niet aankan. 
\item Een groot aantal voor het Nederlands interessante constructies komt niet 
voor in de testbank, omdat hij direct uit het Engels is vertaald.
\item Fenomenen waar we systematisch aan gewerkt hebben komen over het 
algemeen met hogere correctheidsscores uit de bus.
\item Het verschil tussen het percentage zinnen waarvan we dachten dat ze 
correct geanalyseerd zouden worden (60\%) en het feitelijk percentage correcte 
analyses (51\%), 
is mogelijk 'vertekend' door een klein aantal zinnen waarvan we 
ten onrechte verwachtten dat ze niet geanalyseerd zouden kunnen worden. 
\item Fenomenen die relatief slecht scoren 
zijn o.m. comparatieven, vraagwoorden, co\"{o}rdinatie en predicate adjuncts. 
\item  Gezien de resultaten komt de opsplitsing die in de testbank wordt 
gemaakt tussen core-verschijnselen en perifere verschijnselen overeen met onze 
opvattingen daarover. 
\item Van de zinnen die door Eurotra uit de resultaten van 
het Nederlands bestand zijn gehouden omdat van tevoren vaststond dat ze buiten 
de coverage van Eurotra vielen, kan Rosetta 54\% aan. 
\item Het globale resultaat (51\% correct),  en de vergelijking van de resultaten voor 
Rosetta met die voor Eurotra volgens 
de iets afwijkende Eurotra testmethode (57\% - 35\%), stemmen volgens Jan L. niet tot ontevredenheid. 
\item Afgezien van de correctheidsscore is 
Rosetta  aanzienlijk sneller dan Eurotra. Dit verschil 
is wellicht ten dele toe te schrijven aan het gebruik van PROLOG bij Eurotra.
\end{itemize}




\subsection{Software Aspecten}
\subsubsection{Regelnotaties, regelcompilers}

Inleider: Ren\'{e} \\
Notulist: Frank\\

In het vakgebied gebeurt veel op het gebied van 
regelnotaties. Binnen het Rosetta project is de definitie van het formalisme
vrij arbitrair geweest en latere uitbreidingen vaak ad-hoc. We hadden 
er goed aan gedaan te experimenteren met alternatieve notaties, bijvoorbeeld
met behulp van Rosetta2. \\
Een belangrijke tekortkoming van de huidige regelnotatie is het feit dat
procedure aanroepen onmogelijk zijn, zodat veel code gedupliceerd moet
worden. \\
\newline
\noindent
Discussie: \\

De vraag is of je wel een krachtiger formalisme wil. Vanuit het oogpunt van 
de lingu\"{\i}st is simpelheid juist een voordeel. Daar staat tegenover dat
je voor simpelheid betaalt met performance verlies. Bovendien hoef je de 
krachtige extensies alleen te gebruiken voor de speciale gevallen. \\

Met een beter formalisme zouden de Rosetta grammatica's compacter en beter
onderhoudbaar geweest zijn. Conversie van de huidige grammatica's zou echter
een enorme klus betekenen. Bij een significante uitbreiding zou deze 
conversie desondanks gedaan moeten worden. \\

Verder is het jammer dat de regels steeds voor zowel analyse als generatie
gespecificeerd moeten worden. Omkeerbaarheid zou echter strenge restricties
eisen, waardoor de notatie weer aan kracht zou inboeten.

\subsubsection{Performance aspecten}

Inleider: Joep\\
Notulist: Frank\\

Hardware performance is de laatste 10 jaar qua snelheid en geheugen capaciteit
met een factor 10 tot 100 verbeterd. Deze tendens kan waarschijnlijk worden
doorgetrokken voor de komende 10 jaar. Aangezien echter de parseer tijden binnen
Rosetta een exponentieel karakter hebben, zullen maatregelen genomen moeten 
worden om de effici\"{e}ntie te verbeteren. \\
Het belangrijkste knelpunt blijkt het aantal surface trees te zijn. M-parser
zelf blijkt in de praktijk ongeveer lineair te zijn per surface tree.\\
\newline
\noindent
Discussie:\\

Het aantal surface trees kan verkleind worden door een deel van de interactie
voor de M-parser te doen plaatsvinden, en in sommige gevallen zelfs voor de 
surface parser. Hiermee riskeer je wel niet-relevante vragen. 
Verder zouden bepaalde uitsplitsingen pas in de M-parser
gedaan kunnen worden. Omgekeerd zou je de surface parser krachtiger kunnen
maken, zodat daar bepaalde ambigu\"{\i}teiten reeds worden opgeheven. \\
Bij dit soort oplossingen treedt echter al gauw een verschuiving van problemen
op.\\
Men kan de response naar de gebruiker versnellen door de surface trees
te ordenen naar waarschijnlijkheid en vervolgens depth first verder te gaan.
Verder kan door relatief eenvoudige maatregelen de snelheid van het systeem
nog vergroot worden.

\section {Algemene Problematiek}
\subsection {Complexiteit: Is het systeem te groot?}
Notulist: Ren\'{e}\\

Jan O. definieert modulariteit als de afgescheiden behandeling van
phenomenen die verschillend zijn en claimt dat meer modulariteit niet
leidt to betere onderhoudbaarheid of tot grotere overzichtelijkheid
van de grammatica.
Redenen van complexiteit: taal is moeilijk, vertalen maakt het nog moeilijker,
en de notatie maakt bepaalde abstracties onmogelijk, leidend tot lange regels.
Vaak zijn er tegenstrijdige argumenten m.b.t. de volgorde van regels.
Voorbeeld: superdeixis.\\

Wat is de prijs van isomorfie? Consensus is dat de grammatica niet zo heel
erg complexer wordt door isomorfie.\\

Jan O. is het met Jan L. eens dat de complexiteit gereduceerd zou kunnen worden
door minder ambitieus te zijn.\\

Ondanks complexiteit claimt Jan O. dat de huidige grammatica's zo ver kunnen 
worden uitgebreid dat ze vrijwel linguistisch compleet worden, afgezien
van enkele principi�le problemen, zoals met ellipsis en coordinatie.
Compeetheid t.o.v. het Flickinger corpus is bereikbaar.\\

Met betrekking tot de brievenvertaler is er verschil van mening over de
haalbaarheid. Sommigen (Jan O., Lisette) denken dat het aantal problemen
van het soort dat begin '90 opdoemde uit het 'corpus' eigenlijk wel mee
valt. Anderen (Joep) geloven hier niets van.\\

Oplossingen voor complexiteitsproblemen: Sommige beregelingen naar
semantische component, betere documentatie, en een herziening van de notatie.

\subsection {Inwerken}
Inleiders: Petra en Frank\\
Notulist: Lisette\\

Petra noemde twee problemen t.a.z. van het inwerken in de lingu\"{\i}stische
kant van Rosetta:
\begin{enumerate}
  \item {\bf de complexiteit van de grammatica's}\\
Het is moeilijk om kennis over de grammatica te verwerven, omdat niets
ge\"{\i}soleerd staat. Je moet goed op de hoogte zijn van de gebruikte 
analyses, terwijl historische afwijkingen vaak niet volledig gedocumenteerd 
zijn. Ook het feit dat je niet weet waarom iets niet gebeurd is, is vaak
een probleem.
  \item {\bf de toegankelijkheid van de documentatie}\\
Er is veel documentatie, maar slecht toegankelijk. De stukken die een bepaald 
aspect behandelen zijn nog het nuttigst, omdat die verbanden leggen, maar het
zou prettig zijn als daar ook overwegingen voor andere talen in stonden en als
ze bijgewerkt zouden worden.
\end{enumerate}
Het is duidelijk dat een nieuwe medewerker niet in te werken valt via de 
documentatie. Er zou eigenlijk een soort beheerssysteem voor de documentatie
moeten komen, zodanig dat die toegankelijker zou kunnen worden. Verder zou de 
ontbrekende documentatie geschreven moeten worden. Vooral de documentatie van 
het domein is een grote omissie.\\
Er zou een manier gevonden moeten worden om de meer globale kennis over de 
grammatica vast te leggen, bijv. een soort invariant voor S-trees tussen
regelklassen.\\

Er waren geen problemen met het begrijpen van de M-regelnotatie (passief),
hoewel een document met de syntaxis van de M-regelnotatie handig geweest 
zou zijn. Dit had Ren\'{e} kunnen verschaffen.\\
Er waren wel problemen met het schrijven van M-regels (actief). Er is behoefte
aan een document met aanwijzingen hoe je bepaalde zaken aanpakt. Dit zou door 
de meer ervaren lingu\"{\i}sten, bijv.\ Jan O., geschreven kunnen worden.\\

Frank had weinig problemen ondervonden met inwerken in de software-kant van 
Rosetta. Blijkbaar is die veel modulairder. Tevens merkte hij op, dat zijn
activiteiten tot nu toe redelijk beperkt waren gebleven, maar dat 
veel documentatie nog niet was geschreven.\\

Er bestaat blijkbaar een verschil in complexiteit van de lingu\"{\i}stiek
en de software in Rosetta. Naar onze mening zou de complexiteit van de 
grammatica's niet opgevangen kunnen worden door een ander formalisme, zoals 
unificatiegrammatica's. Ook daar zijn complexiteitsproblemen, maar die komen
niet altijd naar voren, omdat veel systemen nooit de omvang (coverage) van ons 
systeem bereiken.

\subsection {Compositioneel Vertalen} 
Inleider: Jan L.\\
Notulist: Andr\'{e}\\
\newline
\noindent
De discussie wordt toegespitst op collocaties en vertaalidiomen (woordenboeken).
\\
\newline
\noindent
{\bf Vertaaldiomen}
\begin{enumerate}
\item De hoeveelheid idioompatronen is beperkt. De hoeveelheid
vertaalidioompatronen is veel groter. 
\item Er kan geen vraag worden gesteld m.b.t. de betekenis van een
vertaalidioom. 
\item Zijn niet vertaalidiomatisch m.b.t. tot elke andere taal; bijv. {\em
vroeg opstaan} kan letterlijk naar het Engels vertaald worden. Een oplossing is
om strikte interlingualiteit op te geven en in het woordenboek aan te geven
naar welke taal een expressie als vertaalidioom opgevat moet worden. De 
letterlijke vertaling moet in sommige gevallen weggegooid worden.
\item Het is lastig om een volledige lijst met
vertaalidiomen te krijgen. 
\end{enumerate}
{\bf Collocaties}
\begin{enumerate}
\item Bijv.: sterke/lichte voorkeur; groot voordeel; op een bepaalde manier.\\
Niet: zware voorkeur; klein voordeel; met een bepaalde manier.
\item Het is een vertaalprobleem: zware shag/strong tabacco; sterke
drank/strong liquor. 
\item Aantal is groot. 
\item Het is lastig om een volledige lijst met collocaties te krijgen.
Misschien in contrastieve studies. 
\item Kunnen in principe met bestaande semi-idioom methode.
\end{enumerate}


\section {Technische Problemen}
\subsection{Isomorfie, transfer, omkeerbaarheid}
Notulist: Jan O.\\
Voorbeelden waar wellicht de isomorfe methode niet zo geschikt is:

\begin{description}
  \item[Beleefdheid] Voor de brieven in het CRE-corpus moesten speciale 
maatregelen genomen worden om de juiste graad van beleefdheid te verzorgen.
Hierbij was er geen sprake meer van behoud van betekenis, maar moest een
 combinatie van een bepaalde beleefdheidsgraad in een bepaald type tekst (
brieven), afhankelijk van de gebruikte persoon omgezet worden in een speciale
vorm. Dit lijkt  niet compositioneel te zijn. De bestaande idioom-methode is
hierbij niet zo geschikt, aangezien het hoofd juist variabel is. Wellicht moet
de idioommethode uitgebreid worden. 
  \item[Vertaling van {\em  men}] Hiervoor lijkt het zinvol de vertaling
talenpaar afhankelijk te maken. Het probleem is vergelijkbaar met de
vertaling van {\em  ik} naar {\em  I} waar in een zuiver interlinguale aanpak 
vanwege het Spaans de vraag {\em  mannelijk} of {\em vrouwelijk} gesteld zal 
worden. Het betreft nu echter niet zo zeer basisexpressies, maar regels.
  \item[Asymmetrie\"{e}n] Reeds nu wordt op en aantal plaatsen in de grammatika
gebruik gemaakt van bewust asymmetrische regels, met name regels die alleen in 
generatie werken maar niet in analyse. Dit gebeurt met name in die gevallen 
waar een zeer gelijkvormige vertaling niet mogelijk is, maar waar een 
omschrijving noodzakelijk is. Voorbeelden zijn: transformaties die een
subject van een ingebedde finiete zin uitspellen als het 
coreferentieel is met een argument uit de matrix zin, en regels (in het Spaans 
en Engels) die {\em  schoon} in een zin als {\em  hij boende kamer schoon}
vertalen in de (Engelse cq Spaanse variant van) een omschrijving zoals
{\em  totdat hij schoon was}. In deze gevallen wordt het isomorfisme verzwakt
tot een homomorfisme.
\item[Vertaalidiomen] Deze vereisen eigenlijk transfer. Het probleem is het 
volgende: in het geval van vertaalidiomen levert de analyse minimaal twee 
parses op: een letterlijke, compositionele en een idiomatische. Indien de 
doeltaal geen expressie mogelijkheid heeft voor de letterlijke parse in de
brontaal, dan zal dit pad doodlopen in generatie. 
Voorbeeld: `verliefd worden' krijgt zowel een letterlijke parse 
(met 2 basis-expressies `verliefd' en `worden') als een idiomatische parse
(met 1 basis-expressie `verliefd-worden'). De idiomatische parse leidt in het
Spaans tot een constructie met het woord `enamorarse'. De letterlijke parse
loopt of dood, of (nog erger) leidt tot zoiets als `*hacerse enamorado', wat 
geen Spaans is.
Eigenlijk zou in dit geval 
dus de letterlijke parse uitgesloten moeten worden indien er ook een
vertaalidioom parse is. Wellicht is het mogelijk een functie te definieren
die een syntactische derivatie boom  weggooit indien dit de letterlijke
parse is die correspondeert met  de idiomatische parse zoals gerepresenteerd
in een andere derivatieboom.

Als het systeem niet-interlinguaal is, dan kunnen de vertaal-idiomen per
talenpaar gespecificeerd worden.
Bij niet-interlingualiteit is het pas uitgefilterd worden in de M-Generator van 
de doeltaal minder erg dan bij een interlinguaal systeem (als er natuurlijk 
maar wel nog iets anders uitkomt).
\end{description}


\subsection{Regelnotatie}
Inleider: Jan O.\\
Notulist: Elena\\
Leesstof: Doc. 243 (p. 1-2), Doc 367 (p. 1-13).\\

Dit onderwerp kwam wegens tijdgebrek te vervallen. Er werd besloten deze 
discussie op een ander tijdstip alsnog te voeren.

\section {Hoe Verder ?}

Notulist: Harm\\

Jan L. werpt de vraag op wat we denken te kunnen maken als we uitgaan van de
volgende randvoorwaarden:
\begin{itemize}
  \item een periode van drie jaar,
  \item ongeveer de huidige bezetting,
  \item het moet een {\em vertaal}-applicatie zijn, {\em interactief}, en op
        {\em zinsniveau},
  \item het moet functioneel af zijn, eventuele toeters en bellen kunnen later
        nog gemaakt worden.
\end{itemize}

Jan O. is niet zo optimistisch; hij denkt dat het niet lukt om semantiek
of een typensysteem in zo'n korte periode toe te voegen, temeer omdat hij 
verwacht dat we nog wel twee jaar nodig hebben om het hele Flickinger corpus
aan te kunnen. Anderen denken dat hij te pessimistisch is.

Ook het beperken tot een bepaald {\em type} teksten lijkt problematisch:
m.b.t. het eventueel inperken tot teksten uit een bepaald domein merkt 
Lisette op dat ze hiervan eigenlijk nooit goede voorbeelden heeft gezien.

Andr\'{e} merkt op dat we met wat aanpassingen, zoals robuustheidsmaatregelen,
een systeem kunnen maken als dat van Metal maar dan beter (het zal dan wel 
een post-editing systeem worden).

Franciska oppert de mogelijkheid om de input niet vrij te laten. 
Andr\'{e} is hierover pessimistisch: volgens bepaalde onderzoeken lukt het
mensen niet zich spontaan te beperken.
Een andere mogelijkheid is iets dat meer in de lijn van een {\em language 
sensitive} of {\em syntax driven} editor ligt.

Franciska komt met het idee iets te maken voor mensen die wel Engels kennen en
naar het Engels vertalen. Anderen merken op dat zo'n applicatie misschien al
gerealiseerd zou kunnen zijn met de combinatie van woordvertaler en 
grammarchecker.

Ren\'{e} merkt op dat we misschien toch iets in de richting van educatieve 
applicaties (bijv. COO) moeten gaan doen.

Franciska brengt CDI ter sprake; Jan L. heeft echter het idee dat dit voorlopig
het verkeerde kanaal is: de mensen die met CDI bezig zijn, zijn juist niet 
ge\"{\i}nteresseerd in PC-achtige dingen.

Jan L. merkt op dat in het kader van Centurion bedenkers van een 
goed plan binnen Research de kans krijgen dit in competitie met andere plannen 
voor een panel 
te verdedigen, waarna voor de beste plannen geld beschikbaar zal komen.
Iets dergelijks schijnt bij 3M goed te functioneren.
Ook bij het GJA
schijnt een dergelijk idee gelanceerd te zijn.

Jan O. signaleert dat we tussen alle andere door ook aan Rosetta zelf moeten
blijven werken, omdat dat er anders nooit komt. Jan L. ziet wel wat in 
Rosetta als "research vehicle". Daarbij worden de volgende onderwerpen 
als mogelijk onderzoeksthema aangedragen:
\begin{itemize}
  \item co\"{o}rdinatie (dit moet gewoon);
  \item pronominale interpretatie door zinnen heen ("dichter" bij Rosetta dan
        Jeroen Groenendijk het heeft gedaan);
  \item stijl;
  \item collocaties;
  \item desambiguering van structurele ambigu\"{\i}teiten (interactief);
  \item semantische component;
  \item discourse, bijvoorbeeld corpus-gebaseerd;
  \item mogelijke beperkingen op de input voor gebruikers.
\end{itemize}

\end{document}
