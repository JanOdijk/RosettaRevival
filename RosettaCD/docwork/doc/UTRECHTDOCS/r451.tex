\documentstyle{Rosetta}
\begin{document}
   \RosTopic{General}
   \RosTitle{Notulen groepsvergadering 2-12-1990}
   \RosAuthor{Franciska de Jong}
   \RosDocNr{451}
   \RosDate{2-12-1990}
   \RosStatus{approved}
   \RosSupersedes{-}
   \RosDistribution{Project}
   \RosClearance{Project}
   \RosKeywords{Notulen}
   \MakeRosTitle
%
%
\begin{description}
\item[Aanwezig:] Franciska de Jong, Ren\'{e} Leermakers,  
                 Jan Landsbergen (voorzitter),   
                 Elena Pinillos, 
                 Andr\'{e} Schenk,
                 Harm Smit,
                 Frank Uittenbogaard, Petra de Wit
                  
                  

\item[Afwezig:]  Lisette Appelo,
                 Jan Odijk, 
                 Joep Rous

\item[Agenda:]\mbox{}
  \begin{enumerate}
  \item Notulen
  \item Actiepunten
  \item Diversen
  \item Review-gesprek Research - RvB
  \item Rondvraag
  \end{enumerate}
\end{description}

\section{Notulen}
De notulen van de vorige keer worden met enkele wijzigingen goedgekeurd.

\section{Actiepunten}

Afgeronde actiepunten:
\begin{enumerate}
  \item Frank heeft via het CFT beschikking gekregen over een lijst 
        die voor VMS-commando's de corresponderende commando's van DOS-UNIX 
        geeft. Bovendien blijkt er een boek te zijn met informatie over 
        UNIX voor gebruikers die vertrouwd zijn met VMS. Frank zal voor de 
groep een exemplaar bestellen.
  \item Franciska's overzicht van de stand van zaken Spaanse NP en adjectieven 
        zal worden rondgezonden.
\end{enumerate}

\section{Diversen}
\begin{enumerate}
\item {\bf CE text products}\\
In verband met het feit dat de resultaten van een marktonderzoek
later beschikbaar komen dan verwacht, is de zero-date vergadering die
oorspronkelijk was gepland op 7 december 1990, 
verschoven naar januari 1991. 

De efficiency drive (operatie Centurion) 
lijkt voorlopig geen negatieve gevolgen te hebben voor 
de text products-groep.
\item {\bf Badges}\\
Jan L. heeft het verzoek gekregen badges met een verlopen geldigheid
aan te melden.
Tijdens de vergadering is de geldigheidsduur van de 
badges van de aanwezige groepsleden genoteerd.

\item {\bf Reisbudget}\\
Het reisbudget voor 1991 is 30\% lager dan voor 1990. Als gevolg hiervan 
gaat de reis van Ren\'{e} naar Amerika niet door.

\item {\bf Kerstsluiting}\\
Het Nat. Lab. is gesloten van 22 december 1990 t/m 1 januari 1991.
\item {\bf Planning}
Op de vergadering van 17 december a.s. staat de planning voor het komend 
halfjaar op de agenda. Jan L. zal vooraf  met iedereen 
individueel een bespreking hebben, voor  zover de aanwezigheid van de 
betrokkenen dat mogelijk maakt.

\item {\bf Verhalen voor de groep}\\
In het kader van de wens om de groepsvergaderingen te verlevendigen
is aan Frank, Harm en Andr\'{e} gevraagd 
een verhaal voor te bereiden over respectievelijk 
de lexicon-editor, grammatica-
checkers en idiomen. 
In januari kan  de draad van 'verhalen voor de groep' 
dan weer opgepakt worden. 

\item {\bf Binnengekomen boeken}\\
Petra heet binnengekregen: {\em The Syntactic Phenomena of English} van James 
McCawley. 

\item {\bf Vergoeding OV-jaarkaart}\\
De vergoeding van de OV-jaarkaart via het STT voor vaste gasten kan 
in een kalenderjaar hoger uitvallen dan het bedrag dat belastingvrij is.
Om 
te voorkomen dat over het verschil  belasting wordt geheven 
verdient het volgens Maaike Wijnen  (STT) 
aanbeveling om in geval van belastingaangifte 
noch de reiskosten, noch de vergoeding te melden. Dit is ook de mening 
van een geraadpleegde medewerkster van de belastingtelefoon. 

\end{enumerate}

\section{Review-gesprek Research-RvB}
Uit het review-gesprek Research-RvB zijn de volgende punten van belang voor
de groep 
Carasso.

\begin{enumerate}
  \item Het zogenaamde review-boek dat een overzicht bevat van het 
research-programma en de financiering ervan is positief beoordeeld.
In het algemeen bestaat er bij Timmer c.s. een positieve houding ten aanzien 
van de research-poot. 
  \item De research-directeuren hebben het belang van de benadrukt van 
stabiliteit.  Er bestaat tussen de RvB en de research-directie 
consensus over de noodzaak om met produktdivisies 
supportcontracten voor langere periodes
af te kunnen sluiten.  Een nieuwe 'enabling technology' vergt over 
het algemeen een periode van twee tot zes jaar.

De RvB heeft opgemerkt dat de
betrokkenheid van de research bij de produktdivisies is verbeterd.

De RvB vindt dat bij de research de verplichting 
ligt de produktdivisies te houden aan de in de supportcontracten 
gemaakte afspraken. 

\item De research-directie 
heeft voorgesteld om het gat dat is gevallen door de
reorganisatie bij IS/CS (computerdivisie) op te vangen met een 
onderzoeksprogramma op het terrrein van de informatietechnologie.
In het kader daarvan zou voor 100 mensen per jaar werk verzekerd zijn. 
De RvB heeft dit voorstel geaccepteerd op voorwaarde dat het plan 
een meer gedetailleerde uitwerking krijgt en dat duidelijk wordt gemaakt welke 
produkten uit het onderzoeksprogramma zullen resulteren. 

\item De RvB vindt dat het onderzoek internationaal moet blijven. Sluiting van 
buitenlandse laboratoria lijkt op grond hiervan niet voor de hand liggend. 
\item Er komt een noodbudget van 5 miljoen om de problemen van de afgekondigde
investeringsstop op te vangen. 

\item De bezuiniging op personeel die voortvloeit uit het 
Centurion-programma (de efficiency drive dus, NIET de portfolio-discussie), 
zal voor het Nat. Lab. naar verwachting 
minder desastreus uitpakken dan het verlies van 600 arbeidsplaatsen waarvan in 
sommige geruchten is gerept. Met eerder aangekondigde bezuinigingen 
zal rekening worden gehouden. In december is hierover meer informatie 
te verwachten. 


Het is volstrekt onduidelijk of er na 1991 nog mogelijkheden zullen zijn om 
'vaste gasten' aan te stellen. Naar het oordeel van Jan L. 
is het niet uitgesloten dat de financiele ruimte 
daarvoor volledig zal verdwijnen. 
\end{enumerate}

\section{Rondvraag}
Er zijn geen vragen en/of opmerkingen.

\section{Actiepunten}
\begin{itemize}

  \item (algemeen) 
De groepsleden wordt verzocht een overzicht (inclusief plaatsaanduiding) 
te maken van de Rosetta-boeken die in hun bezit zijn. 
Bij Lisette kan informatie worden verkregen over het gewenste formaat.
  \item Andr\'{e}: 
\begin{itemize}
  \item (vertaal)idioomregels Spaans
  \item document met procedure voor het vullen van idiomen
  \item voorstel over complexe predikaten 
  \item documentatie idioomregels
\end{itemize}
\item Jan O.:
  verspreiding van het Rosetta LEXIC-document

  \item Franciska:  stuk over floaters
  \item Harm: rapport over grammar checkers lezen
  \item Ren\'{e}: idee over regel-georienteerde semantische component in eerste 
kwartaal 1991
\end{itemize}

\section{Volgende vergadering}
De volgende vergadering is op maandag 17 december, 13.30 uur in WY7.\\
\end{document}


