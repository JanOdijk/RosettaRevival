\documentstyle{Rosetta}
\begin{document}
   \RosTopic{Rosetta3.doc.linguistics.dutch}
   \RosTitle{Dutch M-rules:subgrammar ADJPPROPFormation}
   \RosAuthor{Franciska de Jong, Lisette Appelo}
   \RosDocNr{376}
   \RosDate{August 14, 1990}
   \RosStatus{approved}
   \RosSupersedes{concept of August 7, 1989}
   \RosDistribution{Project}
   \RosClearance{Project}
   \RosKeywords{Dutch, M-rules, AdjpPropFormation}
   \MakeRosTitle
%
%
\input{[dejong.mrules]mrudocdef}

\section{Introduction}
In this document the subgrammar ADJPPROPformation for Dutch is described.
The parts on temporal adverbials and aktionsart calculation have been written by 
Lisette Appelo, the other parts have been written by Franciska de Jong.

In its relevant aspects, this subgrammar has been attuned to the 
subgrammar VERBPPROPformation. It is relevant to {\bf all} 
adjectival constructions, 
whether they end up as ADJP-utterance, as NP-internal constituent, or as
clausal constituent.


\section{Ordering of ADJP-internal constituents}
In this subgrammars ADJPPROPs are built on the basis of SUBADJs. 
Intermediate levels are ADJ, ADJP and ADJPPROP. 
ADJP functions as the predicate (predrel).
It is also the substructure that may be used attributively in the 
prenominal structure of NPs. As a consequence, the amount of syntactic 
information that is present on the ADJP level differs from 
e.g. the VERBP level. Most attributes of the ADJPPROP level 
have a value that is a copy of the ADJP value.
Exceptions are {\em superdeixis}, {\em aktionsarts} and {\em PROsubjet}.

The canonical ordering of  ADJP-internal contituents, i.e. the ordering 
at the moment of introduction, is based upon the following considerations:

\begin{enumerate}
  \item In attributive ADJPs the head ADJ is always the rightmost part. 
(Exceptional, but weird as well: {\em een [{\bf witter} dan witte] was},
{\em een [zo {\bf gunstig} mogelijke] beslissing})
  \item ADJPs that do not allow an ordering with the ADJ as rightmost element 
in any syntactic context may not be used attributively. For example 
{\em dol op vis}: *de op vis dolle man.

\item Adjectives that take a prepositional object
may be characterized according to their preference to follow this prepobj 
or to precede it. This preference can be tested by 
determining in which of the following two the contexts the ADJ may occur. There 
are different (classes of) adjpatternvalues to express this.


\begin{enumerate}
  \item 
postadjectival: ... dat x1 er niet ADJ Prep lijkt (example: bang)
  \item
preadjectival: ... dat x1 er niet Prep ADJ lijkt (example: bezig)
\end{enumerate}

For simple prepobjs the two values are synPOSTADJPREPNP and synPREPNP 
respectively. 
For the more complex cases, PA in the  name of a adjpatternvalue indicates 
the possibility to take a postadjectval prepobj. e.g. synPAPREPTHATSENT.

A subset of the adjectives does not exhibit a preference, e.g. {\em gesteld},
{\em tevreden}.
\item
If the test indicates the existence of 
a preference, it is not implied that the reversed order 
will never occur in other contexts than the test context.  
Movement rules may account for the alternative orderings.

\end{enumerate}
Ordering is expressed by the use of different relation names for different
positions. 
The ordering that exists before the application of the optional movement rules 
(described in doc.nr. 397), is indicated below.
(N.B. The * before ADVREL abbreviates temp, caus, loc.) Cf. also
LSMRUQUO\_ADJPrelprec (related to procedure `initadjprelorder') in file 
dutch:lsmruquo.pas.
\mbox{}\\ 

\begin{tabular}{ll}

prehead: &  INDOBJREL OBJREL DEGREEMODREL *ADVREL NEGREL \\
         &   LOCARGREL/PREPOBJREL  COMPLREL \\

HEAD& \\

postadjhead:& HOPREL VOOROBJREL PAPREPOBJREL POSTADJREL\\
&\\
\end{tabular}


After the application of movement rules, the position with relation ERPOSREL 
preceding the position associated with INDOBJREL is 
available for any prepositional element.
Some orderings are excluded by these presumptions. For example a prepobj
generated with  paprepobjrel will never occur between a degree modifier and the 
head: *zeer voor de os bang. 
\section{The treatment of modifiers}
In this subgrammar, the elements modifying the adjectival head
of ADJPPROP are introduced. They are all of category XP, not of category XPPROP. 

Introduction at this stage has certain striking side effects. 
As the deixis rules apply not in this subgrammar but in 
ADJPPROPtoADJPFORMULA, the superdeixis value of 
the modifiers cannot be adapted to the adjectival 
head yet. Superdeixis adaptation rules are needed 
in order to account for the dependency. 
These adaptation rules apply after the deixis rules 
of subgrammar ADJPPROPtoADJPORMULA.

This ordering of introduction followed by adaptation is chosen 
in order to enable the introduction in the subgrammar which 
is common to 
all derivations of adjectival (sub)structures, viz. ADJPPROPFORMATION.
If the introduction of modifiers were ordered after the deixisrules,
the relevant rule classes would have to be "doubled" in subgrammar
XPPROPtoCLAUSE.

Note that degree modification rules for clausal structures (e.g. in {\em de 
uitspraak verbaasde ons {\bf zeer}}) are not available yet.

\section{Subgrammar Specification}

\begin{description}
  \item[Head] SUBADJ
  \item[Export] ADJPPROP
  \item[Import] PREPP, 
ADVP,  QP, NP, ASP, SENTENCE, the variables in VARCATSET
(expected: EMPTYVAR, CNVAR, NPVAR, PREPPVAR, SENTENCEVAR)
  \item[File] dutch:adjsubgrammars.mrule (mrules48)
\end{description}

\section{Control Expression}
The control expression can be defined as follows:
\begin{verbatim}
 
  RC\_STARTADJPPROP

. TC\_ADJpatterns
   
. [RC\_ADJVOOROBJMOD]

. [RC\_ADJMOD]

. [RC\_ADJDEGREEMOD]

. {RC\_ADJVARinsertion}

. RC\_ADJVOICE
 
. TC\_ADJAktionsartcalculation

. TC\_ADJZichSpelling

. [TC\_ADJQPHopping]

\end{verbatim}

\section{Rules and Transformations}
\begin{mruleclass}{RC\_STARTADJPPROP}
\begin{classdescr}
\kind obligatory rule class
\classtask The formation of the syntactic level ADJPPROP

\nofilters

\nospeedrules

\noplannedrules

\norulesnotince

\begin{comments}
\end{comments}
\end{classdescr}

\begin{members}
\begin{member}
\rulename RSTARTADJPPROPFORMATION000
\ruletask formation of an ADJPPROP-level, no semantic arguments
\file dutch:rc\_startadjpprop.mrule (mrules32)
\semantics The formation of a propositon.
\example
\mbox{}
\begin{enumerate}
  \item 
 koud $\rightarrow$ het  koud 
  \item
        regenachtig $\rightarrow$ het regenachtig 
\end{enumerate}
\remarks
Adjectives with thetaadj =adjp000  should have a separate entry for their
attributive/predicative use: {\em een regenachtige/koude dag}, de maand 
{\em de maand februari was regenachtig en koud}. 

\end{member}
\begin{member}
\rulename startadjpprop100
\ruletask Adjppropformation with introduction of one argument variable
\file dutch:rc\_startadjpprop.mrule (mrules32)
\semantics The formation of a propositon.
\example 

\mbox{}
\begin{enumerate}
  \item 
x1 + blond $\rightarrow$ x1 blond (Jane is blond)
  \item
x1 + waarschijnlijk $\rightarrow$ x1 waarschijnlijk (het is waarschijnlijk  dat 
de treinen weer rijden)
\end{enumerate}
\remarks

\end{member}
\begin{member}
\rulename RSTARTADJPPROP120
\ruletask Formation of ADJPPROP, with introduction of two argument 
variables
\file dutch:rc\_startadjpprop.mrule (mrules32)
\semantics The formation of a propositon.
\example 
\mbox{}
\begin{enumerate}
  \item 
x1 + x2 + verliefd $\rightarrow$ x1 x2 verliefd  (Jane is verliefd op Tim)
  \item
x1 + x2 + beu $\rightarrow$ x1 x2 beu (Wij zijn het gezeur beu) 
\end{enumerate}

\remarks\mbox{}
The difference between prepositional objects and so-
called `oorzakelijke voorwerpen' is dealt with in the adjpatternrules, 
under reference to the attribute adjpattern(ef)s. 
For the latter category ADJPPROPrec1.adjpattternefs = 
                                         [synNP, synHETOPENOMTESENT, 
                                          synTHATSENT, synHETTHATSENT]

\end{member}
\begin{member}
\rulename RSTARTADJPPROP123 
\ruletask Formation of ADJPPROP with introduction of three argument 
variables
\file dutch:rc\_startadjpprop.mrule (mrules32)
\semantics The formation of a propositon.
\example 

\mbox{}
\begin{enumerate}
  \item 
x1 + x2 + x3 + waard $\rightarrow$ x1 x2 x3 waard\\
(Jane is mij die kleine moeite wel waard)
  \item
x1 + x2 + x3 + verplicht $\rightarrow$ x1 x2 x3 verplicht\\
(Wij zijn het aan Jane verplicht om op tijd te komen)
\end{enumerate}

\remarks

\end{member}
\begin{member}
\rulename RSTARTADJPPROP012 
\ruletask Formation of an ergative ADJPPROP structure with introduction 
of two argument variables
\file dutch:rc\_startadjpprop.mrule (mrules32)
\semantics The formation of a propositon.
\example x1 + x2 + bekend/duidelijk $\rightarrow$ x1 x2 bekend/duidelijk\\
(De voorwaarden zijn (ons) bekend; Het werd (hem) duidelijk dat het ging 
regenen)
\remarks

\end{member}
\end{members}


\end{mruleclass}

\newpage
\begin{mruleclass}{TC\_adjpatterns}
\begin{classdescr}
\kind obligatory transformation class
\classtask Spelling out the synpatterns. 
\nofilters

\nospeedrules

\noplannedrules

\norulesnotince

\classremarks
NB. The difference between aanobjs and prepobjs is discussed 
in 
the section on RC\_degreemod.
\end{classdescr}

\begin{members}
\begin{member}
\rulename TADJPATTERN0
\ruletask Spelling out the synpattern value synnoadjpargs
\file dutch:tc\_adjpattern1.mrule (mrules33)
\semantics \nosemantics
\example Cf. document nr. 374 (Adjpatterns of Dutch)
\remarks

\end{member}
\begin{member}
\rulename TADJPATTERN11
\ruletask Spelling out the following synpattern values:
\begin{enumerate}
  \item synMEASUREPHRASE
  \item synNP
  \item synIONP
  \item synLOCEMPTY
  \item synLOCPREPP
  \item synPATHPREPP
\end{enumerate}

\file dutch:tc\_adjpattern1.mrule (mrules33)
\semantics \nosemantics
\example Cf. document nr. 374 (Adjpatterns of Dutch)
\remarks
In the present implementation,  adjectives such as 
{\em woonachtig} cannot be translated into a verb (e.g. {\em to reside})
because adjectives select a locative PREPP as complement, whereas 
verbs select a locative PREPPPROP. 

\end{member}
\begin{member}
\rulename TADJPATTERN12a
\ruletask Spelling out the synpattern value synPREPNP
\file dutch:tc\_adjpattern1.mrule (mrules33)
\semantics \nosemantics
\example Cf. document nr. 374 (Adjpatterns of Dutch)
\remarks  A test for prepobj-status is discussed in 
the section on RC\_degreemod.
\end{member}
\begin{member}
\rulename TADJPATTERN12b
\ruletask Spelling out the synpattern value synPOSTADJPREPNP


\file dutch:tc\_adjpattern1.mrule (mrules33)
\semantics \nosemantics
\example Cf. document nr. 374 (Adjpatterns of Dutch)
\remarks

\end{member}
\begin{member}
\rulename TADJPATTERN13
\ruletask Spelling out the following synpattern values:
\begin{enumerate}
  \item synHETOPENOMTESENT
  \item synHETTHATSENT
\end{enumerate}
\file dutch:tc\_adjpattern1.mrule (mrules33)
\semantics \nosemantics
\example Cf. document nr. 374 (Adjpatterns of Dutch)
\remarks

\end{member}
\begin{member}
\rulename TADJPATTERN14
\ruletask To spell out the synpattern value synPREPEMPTY
\file dutch:tc\_adjpattern1.mrule (mrules33)
\semantics \nosemantics
\example Cf. document nr. 374 (Adjpatterns of Dutch)
\remarks

\end{member}
\begin{member}
\rulename TADJPATTERN15a
\ruletask To spell out the following synpattern values:
\begin{enumerate}
  \item  synPREPOPENOMTESENT
  \item  synPREPQSENT
  \item  synPREPTHATSENT
\end{enumerate}
\file dutch:tc\_adjpattern1.mrule (mrules33)
\semantics \nosemantics
\example Cf. document nr. 374 (Adjpatterns of Dutch)
\remarks

\end{member}
\begin{member}
\rulename TADJPATTERN15b
\ruletask To spell out the following synpattern values
(NB. the prefix PA stands for PostAdjectival):
\begin{enumerate}
  \item synPAPREPOPENOMTESENT
  \item synPAPREPTHATSENT
  \item synPAPREPQSENT
\end{enumerate}
\file dutch:tc\_adjpattern1.mrule (mrules33)
\semantics \nosemantics
\example Cf. document nr. 374 (Adjpatterns of Dutch)
\remarks

\end{member}
\begin{member}
\rulename TADJPATTERN16
\ruletask To spell out the following synpattern values:
\begin{enumerate}
  \item synOPENTESENT
  \item synTHATSENT
  \item synQSENT
\end{enumerate}

\file dutch:tc\_adjpattern1.mrule (mrules33)
\semantics \nosemantics
\example Cf. document nr. 374 (Adjpatterns of Dutch)
\remarks\mbox{}
\begin{enumerate}
\item NOTA BENE: the meaning of the adjective may  vary with the specific 
synpattern, for example: {\em bang} in {\em ik ben bang te verliezen}
which expresses the awareness of a threat, 
versus  {\em bang} in {\em ik ben er bang voor om te verliezen} 
which expresses a general attitude.
\\

\end{enumerate}

\end{member}
\begin{member}
\rulename TADJPATTERN17
\ruletask To spell out the synpattern value synOPENOMTESENTPROOBJ
\file dutch:tc\_adjpattern1.mrule (mrules33)
\semantics \nosemantics
\example\mbox{}\\
\begin{enumerate}
  \item 
Dit boek is niet bedoeld/(MEANT) om in te schrijven
  \item 
Dit mes is geschikt/(MEANT) om (kaas) mee te snijden
  \item 
De kaas is niet MEANT (??om) te snijden
\end{enumerate}
\remarks\mbox{}

\begin{enumerate}
\item MEANT is supposed to be an abstract basic expression. It does not exist 
yet. 
\item Cf. also  document nr. 374 (Adjpatterns of Dutch)
\item There is no complete path for these cases yet. 
The phenomenon - complementation by means of a sentence with a PRO-object- 
needs to be 
studied still in some more detail. 
\item 
At least the following 
two alternative treatments for the examples mentioned above are 
conceivable:
\begin{itemize}
  \item There is an additional empty argument (EMPTY,  or ALL, or whatever).
The pro-subject (BIGPRO) of the {\em om te}-sentence is deleted 
under identity with EMPTY/ALL. 
(This would require adjp123 as the value for .thetaadj (and given the implicit 
naming convention, a different name for the patternrule).) 
  \item
The pro-subject is deleted without identity conditions to be met.
\end{itemize}
Pending a principled choice the present elaboration is compatible 
with the second alternative.
\item It is not decided yet whether phrases such as 
{\em geschikt voor Jan om mee te nemen}, with {\em Jan} in the voorobj
as the obligatory antecedent for the 
subjectvar of {\em om te nemen}, 
must be  analysed by this 
pattern transformation too,  or rather by (a combination of) 
RADJVOOROBJMOD and  RADJomteMOD.

\item Translation aspects: The dutch sentence 
{\em dit boek is niet om in te schrijven} should be 
translated into a passive construction of English: {\em this book is not meant 
to be
written in}. Sometimes a translation with a {\em for}-phrase is preferable:
{\em dit bier is niet om te drinken} translates into {\em this beer is not for 
drinking}. 

The second example given above requires as a Spanish translation either
{\em Este cucillo corta bien}, or {\em Este cucillo es para cortar queso}.

It is still to be invetigated whether and how 
these translations can be realized within Rosetta. 
\end{enumerate}

\end{member}
\begin{member}
\rulename TADJPATTERN18a
\ruletask To spell out the synpattern synVOORNP
\file dutch:tc\_adjpattern1.mrule (mrules33)
\semantics \nosemantics
\example Cf. document nr. 374 (Adjpatterns of Dutch)
\remarks
\end{member}
\begin{member}
\rulename TADJPTTERN18b
\ruletask To spell out the synpattern synVOOREMPTY 
(in case of empty "belanghebbende voorwerpen")
\file dutch:tc\_adjpattern1.mrule (mrules33)
\semantics \nosemantics
\example Cf. document nr. 374 (Adjpatterns of Dutch)
\remarks

\end{member}
\begin{member}
\rulename TADJPATTERN21
\ruletask To spell out the following synpattern values:
\begin{enumerate}
  \item synAANNP\_DONP
  \item synAANNP\_OPENTESENT 
  \item synAANNP\_QSENT
  \item synAANNP\_THATSENT
\end{enumerate}
\file dutch:tc\_adjpattern2.mrule (mrules34)
\semantics \nosemantics
\example Cf. document nr. 374 (Adjpatterns of Dutch)
\remarks

\end{member}
\begin{member}
\rulename TADJPATTERN22
\ruletask To spell out the following synpattern values:
\begin{enumerate}
  \item synIONP\_DONP
  \item synIOEMPTY\_DONP 
  \item synIONP\_OPENTESENT
  \item synIONP\_QSENT
  \item synIONP\_THATSENT
  \item synIOEMPTY\_QSENT
  \item synIOEMPTY\_THATSENT
\end{enumerate}
\file dutch:tc\_adjpattern2.mrule (mrules34)
\semantics \nosemantics
\example Cf. document nr. 374 (Adjpatterns of Dutch)

\end{member}
\begin{member}
\rulename TADJPATTERN23
\ruletask To spell out the  synpattern value synAANNP\_HETOPENTESENT
\file dutch:tc\_adjpattern2.mrule (mrules34)
\semantics \nosemantics
\example Cf. document nr. 374 (Adjpatterns of Dutch)
\remarks\mbox{}
\begin{enumerate}
\item In its present form this rule may be collapsed with TADJPATERN25.

\end{enumerate}

\end{member}
\begin{member}
\rulename TADJPATTERN24
\ruletask To spell out the following synpattern values:
\begin{enumerate}
  \item synIONP\_HETOPENOMTESENT 
  \item synIOEMPTY\_HETOPENOMTESENT  
\end{enumerate}
\file dutch:tc\_adjpattern2.mrule (mrules34)
\semantics \nosemantics
\example Cf. document nr. 374 (Adjpatterns of Dutch)
\remarks\mbox{}
\begin{enumerate}
\item In its present form this rule may be collapsed with TADJPATERN26

\end{enumerate}

\end{member}

\begin{member}
\rulename TADJPATTERN25
\ruletask To spell out the following synpattern values:
\begin{enumerate}
  \item synAANNP\_HETTHATSENT
  \item synAANNP\_HETQSENT 
\end{enumerate}
\file dutch:tc\_adjpattern2.mrule (mrules34)
\semantics \nosemantics
\example Cf. document nr. 374 (Adjpatterns of Dutch)
\remarks\mbox{}
\begin{enumerate}
\item In its present form this rule may be collapsed with TADJPATERN23.

\end{enumerate}

\end{member}
\begin{member}
\rulename TADJPATTERN26
\ruletask To spell out the following synpattern values:
\begin{enumerate}
  \item synIONP\_HETTHATSENT
  \item  synIONP\_HETQSENT
  \item synIOEMPTY\_HETTHATSENT
  \item  synIOEMPTY\_HETQSENT
\end{enumerate}
\file dutch:tc\_adjpattern2.mrule (mrules34)
\semantics \nosemantics
\example Cf. document nr. 374 (Adjpatterns of Dutch)
\remarks\mbox{} 
\begin{enumerate}
\item In its present form this rule may be collapsed with TADJPATERN24.

\end{enumerate}


\end{member}

\end{members}

\end{mruleclass}
\newpage
\begin{mruleclass}{RC\_ADJVOOROBJMOD}
\begin{classdescr}
\kind optional rule class
\classtask Introduction of a variable that is to 
function in a {\em voor}-modifier to adjectives with the value 
{\em voorsubjectiveadj} in .subcs. This class of modifiers must be introduced 
via 
a variable. First of all 
because {\em voor}-modifiers may function as controllers. Secondly
because the preposition cannot be translated locally: in English 
its counterpart is either {\em to} or {\em for}. 
\nofilters

\nospeedrules

\noplannedrules

\norulesnotince
\classremarks
It is not clear yet whether cases like {\em deze tas is handig om mee te 
nemen} (with an infinitival modifier) 
should be dealt with by this rule class too, for it might be argued that
in this sentence an {\em ervoor}-PREPP is omitted, or that is a paraphrase of 
{\em het is handig om deze tas mee te nemen}. In the former case it is to 
be treated by RC\_ADJMOD. In the latter case a transformation is needed to 
relate it to sentences with a dummy  {\em het} and a sentential 
extraposed subject.

NB. Since spring 1990 the domain distinguishes between the values {\em 
voorsubjectiveadj} and {\em subjectiveadj}. The latter is to be used for 
infinitival  modifiers. 
The grammar has not been adapted accordingly yet at all relevant places. 


\end{classdescr}

\begin{members}
\begin{member}
\rulename RADJVOOROBJMOD
\ruletask 
Introduction of a {\em voor}-modifier to adjectives with the value {
\em voorsubjectiveadj} in .subcs. 
\file dutch:rcs\_adjmod.mrule (mrules31)
\semantics modification 
\example\mbox{}
\begin{enumerate}
  \item x1 leuk + x2 $\rightarrow$ x1 + voorobjrel/voor x2 + leuk \\
Dit boek is leuk  voor kinderen 
  \item x1 handig + x2 $\rightarrow$ x1 + voorobjrel/voor x2 + handig\\
Rollers zijn handig voor onervaren schilders;
de voor meer ervaren schilders handige kwasten
  \item x1 slecht + x2 $\rightarrow$ x1 + voorobjrel/voor x2 + slecht\\
Dat het niet regent is slecht voor het gras
\end{enumerate}
\remarks\mbox{}
\begin{enumerate}
\item The modifiers introduced by means of this rule are quite argument-like.
The reason that they are not treated as real arguments and hence not introduced
by RC\_startadjpprop/TC\_Adjpattern is that they are always optional and that 
they do not behave as prepobjs while {\em voor} cannot be omitted.
(NB.  A test for prepobj-status is discussed in 
the section on RC\_degreemod.)
\item
Probably it is possible to define severe dictionary constraints
on the set of adjpatterns for adjectives with .sucbs = voorsubjectiveadj.
\item The spanish counterparts of the examples with a leftdislocated sentential
subject involve the occurrence of {\em el hecho} (het feit) or of a
construction with initial {\em el que}.
\end{enumerate}
\end{member}
\end{members}
\end{mruleclass}

\newpage
\begin{mruleclass}{RC\_ADJMOD}
\begin{classdescr}
\kind optional rule class
\classtask Introduction of modifiers to adjectives other than degree-
modifiers and voorobj-modifiers.
\nofilters

\nospeedrules

\noplannedrules

\norulesnotince

\classremarks
RC\_ADJMOD in combination with those rules 
of RC\_ADJDEGREEMOD that introduce a complex
modifier with an infinitival complement may give rise to undesired ambiguities.
For example {\em te vervelend om de hele dag te horen} will get an analysis 
with a complex modifier ({\em te.. om te horen}), as well as an analysis with
a simplex degreemodifier and an {\em om te}-modifier.
\end{classdescr}

\begin{members}

\begin{member}
\rulename Rmod1
\ruletask  Introduction of modifiers that express a comparison. (NOT: 
degreemodifiers such as {\em meer}.)
\file dutch:rcs\_adjmod.mrule (mrules31)
\semantics modification 
\example\mbox{}
\begin{enumerate}
\item sterk als een leeuw 
\item
gelig alsof .......
\end{enumerate}
\remarks\mbox{}

\begin{enumerate}
\item Neither of the above examples is convincing. May be this rule will turn out 
to be superfluous. 
\item
The {\em als}-comparisons never occur as prenominal modifier, therefore 
they are related to the ADJP node by postmodrel.
\item problems:\\ It is not clear which adverbial modifiers that co-occur with
ADJPs should be considered sentential constituents and which should be taken as 
belonging to the ADJP proper.
\item 
Alternative1 for example 1: idiom-treatment. \\
Alternative2 for example 1: the modifier may be analysed as 
a complement to deleted  {\em zo}.
\end{enumerate}

\end{member}
\begin{member}
\rulename Romtemod1
\ruletask  Introduction of {\em om te}-modifiers. These modifiers
must contain a proobject argument in shiftrel. 
\file dutch:rcs\_adjmod.mrule (mrules31)
\semantics modification 

\example\mbox{}
\begin{enumerate}
  \item leuk om te zien
  \item ongezond  om rauw te eten
\end{enumerate}
\remarks
The content of the dutch examples such as (i) and (ii)
that are dealt with
by RADJOMTEMOD1, 
can be  translated only 
into 
Spanish structures that do not involve modification, namely the structures that
correspond to the dutch examples (iii) and (iv) 
(dummy {\em het} and extraposed sentential 
subject without a proobject).

(i) zij is leuk om te ontmoeten \\ 
(ii) dit is handig om mee te nemen\\ 
(iiiDU) het is leuk om haar te ontmoeten\\
(iiiSP) es divertido recontrar la\\
(ivDU) het is handig om dit mee te nemen\\
(ivSP) es comodo llevar se lo\\

This could be accounted for straightforwardly if the two  constructions
of dutch were related (that is, if parallel derivations would exist). 
Whether they can and should be related is still to be investigated. 
At least the following questions need to be answered: 

\begin{enumerate}

\item
Do the two constructions 
have identical meanings? (Note that {\em leuk om te zien} is perhaps an 
idiomatic structure with a {\em leuk} that is not identical to the 
{\em leuk} the construction with the dummy {\em het}; in Spanish it 
translates into  {\em attractiva}.) 

\item
Do the struture with non-dummy subject really involve modification?
Note that {\em x1 is ADJ om ....} does not always imply {\em x1 is ADJ}.
\end{enumerate}

Mapping of the two constructions -in case 
the differences beteen (i/ii) and (iii/iv) appear to be non-semantic- 
would require (a set of) transformation(s) that takes care of 
at least the whole range of adaptations that are implemented for stranding:
met, tot $\rightarrow$ mee, toe; PREP dit $\rightarrow$ hierPREP 
(e.g. dit is handig om mee te snijden $\rightarrow$ 
het is handig om hiermee te snijden).

\end{member}
\end{members}
\end{mruleclass}

\newpage
\begin{mruleclass}{RC\_ADJDEGREEMOD}
\begin{classdescr}
\kind optional rule class
\classtask Introduction of degree-
modifiers. 

\classremarks
Degree modifiers differentiate between prepobjs and aanobjs. Degreemodifiers 
always 
follow aanobjs, while with prepobjs, the ordering is arbitrary.
Compare: {\em Hij is zeer op ons gesteld} and {\em Hij is op ons zeer gesteld}, 
versus {\em De vraag is aan ons zeer duidelijk} versus {\em *De vraag is zeer 
aan ons duidelijk.}
This fact may be used as a test for prepobj-status, and consequently, as a tool
to determine the value of .adjpatterns correctly.

For example, the adjective {\em gewend} superficially  
behaves as an adjective with an 
optional aanobj as indirect object: {\em aan} can be deleted. However, the fact 
that the object can be preceded by a degreemodifier 
{\em Hij is zeer aan ons gewend}) suggests that it should 
be taken as a prepobj.  
So {\em gewend} is an example of an adjective that allows both
a prepositional object (synpattern: synPREPNP), and
a so-called "oorzakelijk voorwerp" (synpattern synNP).
(NB. There are no other adjectives found yet that would necessitate 
the distinction of a synpattern value synAANNP.)
\nofilters

\nospeedrules

\noplannedrules

\norulesnotince

\begin{comments}
\end{comments}
\end{classdescr}

\begin{members}
\begin{member}
\rulename RADJDegreemod1
\ruletask Insertion of a degree-modifier without infinitival complement
in preadjectival position.
NB. The modifiers may be of various categories.
\file dutch:rcs\_adjmod.mrule (mrules31)
\semantics modification
\example te + groot $\rightarrow$ te groot\\
minder dan  vroeger + leuk $\rightarrow$ minder dan vroeger leuk (cf.remarks)\\
zo als mogelijk + spoedig$\rightarrow$ zo als mogelijk spoedig\\
genoeg + mooi $\rightarrow$ genoeg mooi (cf. remarks)\\
hoe + lang$\rightarrow$ hoe lang \\
iets + gelig$\rightarrow$ iets gelig\\

\remarks\mbox{}
\begin{enumerate}
\item
Extraposition of THANPcomplements ({\em als ...}, {\em dan ..}), and hopping in
case of e.g. {\em genoeg} is dealt with by TC\_ADJCOMPLextrapos and TC\_ADJhop.
\item {\em te} en {\em te zeer} are probably to be treated
as synonyms. That is, {\em te zeer} is probably not
a modified {\em te}.
Cf. doc. r397, TC\_ADJteTOtezeer. 
\end{enumerate}

\end{member}
\begin{member}
\rulename RADJDegreemod2a
\ruletask Insertion of a degree-modifier of category QP 
that contains a sentential om-te-modifier with relation omcomplrel. This 
complement is introduced in postadjectival position. Extraposition rule need 
not apply to these modifiers within the ADJ-subgrammars.
 
\file dutch:rcs\_adjmod.mrule (mrules31)
\semantics modification 
\example
\begin{enumerate}
  \item 

verliefd + [ genoeg .. om niet serieus te nemen ] $\rightarrow$
{\tt degreemodrel}/genoeg + {\tt head}/verliefd + {\tt postadjrel}/om niet serieus te nemen

  \item 
moe + [ genoeg om zich te vergissen] $\rightarrow$
{\tt degreemodrel}/genoeg + {\tt head}/moe + {\tt postadjrel}/om zich te vergissen
\end{enumerate}
\remarks\mbox{}
\begin{enumerate}
\item TC\_ADJhop accounts for the required surface order with {\em genoeg}
following the head ADJ. 
\item  Two kinds of modifers may occur as  {\em om te}-modifiers:
infinitivals modifiers with a prosubject, and modifiers with both a prosubject 
and a proobject.
\item In generation, the superdeixis value 
for the sentential complements is replaced by the value of the QP (and later on 
by TADJsuperdeixisadaptation2 it is set to omegadeixis).
In analysis the value
is set to {\em omegadeixis}.

\end{enumerate}

\end{member}
\begin{member}
\rulename RADJDegreemod2b
\ruletask Insertion of a degree-modifier of category ADVP
that contains a sententitial om-te-modifier with relation omcomplrel.
This 
complement is introduced in postadjectival position.
\file dutch:rcs\_adjmod.mrule (mrules31)
\semantics modification
\example\mbox{}
 \begin{enumerate}
  \item 
zwaar + te om mee te nemen $\rightarrow$\\
{\tt degreemodrel}/te + {\tt head}/zwaar + {\tt postadjrel}/om mee te nemen
  \item
verliefd + te om serieus te nemen $\rightarrow$\\
{\tt degreemodrel}/te (zeer) + {\tt head}/verliefd + {\tt postadjrel}/om serieus te nemen 
\end{enumerate}
\remarks\mbox{}
\begin{enumerate}
\item {\em te} en {\em te zeer} should probably be treated
as synonyms. That is, {\em te zeer} is probably not
a modified {\em te}.

\item In generation, the superdeixis value 
for the sentential complements is replaced by the value of the
 ADVP (and later on 
by TADJsuperdeixisadaptation2 it is set to omegadeixis).
In analysis the value
is set to {\em omegadeixis}.
\end{enumerate}

\end{member}
\end{members}

\end{mruleclass}

\newpage
\begin{mruleclass}{RC\_ADJVARinsertion}
\begin{classdescr}
\kind iterative rule class
\classtask to introduce variables for adverbial phrases
\classremarks
It is not clear which causative and locative PREPPs belong to the ADJP
and which belong to the sentence. 
Compare: 
\begin{enumerate}
  \item 
   van de warmte ben ik slaperig  vs. van de warmte slaperige mensen 
  \item
  in de trein ben ik bang vs. * in de trein bange mensen\
\end{enumerate}
\nofilters

\nospeedrules

\noplannedrules

\norulesnotince


\end{classdescr}

\begin{members}
\begin{member}
\rulename RADJrefvarinsertion
\ruletask Introduction of variables for referential time adverbials that 
are not retrospective. These time adverbials are supposed to be of category 
ADVP, PREPP or SENTENCE. They will be substituted in the substitution rules.
\file dutch:tempadj1.mrule (mrules79)
\semantics unclear
\example x1 ADJ $\rightarrow$ x1 refvar ADJ (Zij was gisteren ziek)
  \remarks\mbox{}
\begin{enumerate}
\item  Only one rule for non retrospective referential time adverbials 
will be applied. Complex time adverbials such as e.g. {\em morgen om 3 uur} can 
form one constituent, but also appear disjunctive as e.g. in {\em morgen komt 
hij om drie uur}. 
\end{enumerate}

\end{member}
\begin{member}
\rulename RADJdurvarinsertion
\ruletask Introduction of variables for durative time adverbials.
These time adverbials are supposed to be of category 
ADVP, PREPP or SENTENCE. They will be substituted in the substitution rules.
\file dutch:tempadj1.mrule (mrules79)
\semantics  unclear
\example x1 ADJ $\rightarrow$ x1 durvar ADJ (Zij is drie weken ziek geweest)
  \remarks\mbox{}

\end{member}
\begin{member}
\rulename RADJretrovarinsertion
\ruletask Introduction of variables for referential time adverbials that 
are retrospective. These time adverbials are supposed to be of category 
ADVP, PREPP or SENTENCE. They will be substituted in the substitution rules.
\file dutch:tempadj1.mrule (mrules79)
\semantics unclear
\example x1 ADJ $\rightarrow$ x1 retrovar ADJ (Zij was al drie weken ziek)
  \remarks\mbox{} 

\end{member}
\begin{member}
\rulename RADJcauspreppvar 
\ruletask Introduction of VARs for causative prepositional modifiers 
\file dutch:RC\_ADJcauslocmodvar.mrule (mrules35)
\semantics modification
\example\mbox{}
\begin{enumerate}
  \item 
door de herrie bang 
  \item
van de warmte slaperig
  \item 
door de regen verlopen (geraakt)
\end{enumerate}
\remarks\mbox{}

\end{member}
\begin{member}
\rulename RADJcausadvpvar
\ruletask Introduction of VARs for causative adverbial modifiers 
\file dutch:RC\_ADJcauslocmodvar.mrule (mrules35)
\semantics modification
\example daarom bang 
\remarks\mbox{}

\end{member}
\begin{member}
\rulename Radjppconjsvar
\ruletask Introduction of a variable for a non-temporal subordinate clause in 
sentadvrel.
\file dutch:rc\_advvar.mrule (mrules51)
\semantics unclear
\example Hij lijkt hoewel hij al drie is nog een baby 

\remarks\mbox{}

\end{member}
\begin{member}
\rulename Radjppconjsvar2
\ruletask Introduction of a variable for a non-temporal 
subordinate clause 
in 
leftdislocrel.
\file dutch:rc\_advvar.mrule (mrules51)
\semantics unclear
\example Hoewel hij al drie is lijkt hij nog een baby 

\remarks\mbox{}

\end{member}
\begin{member}
\rulename Radjppconjsvar3
\ruletask Introduction of a variable for a 
non-temporal subordinate clause in postsentadvrel
\file dutch:rc\_advvar.mrule (mrules51)
\semantics unclear
\example Hij lijkt nog een baby, hoewel hij al drie is 

\remarks\mbox{}

\end{member}

\begin{member}
\rulename RADJlocpreppvar 
\ruletask Introduction of VARs for locative prepositional modifiers 
\file dutch:RC\_ADJcauslocmodvar.mrule (mrules35)
\semantics modification
\example in de trein bang 
\remarks\mbox{}

\end{member}
\begin{member}
\rulename RADJlocadvpvar
\ruletask Introduction of VARs for locative adverbial modifiers 
\file dutch:RC\_ADJcauslocmodvar.mrule (mrules35)
\semantics modification
\example overal bang 
\remarks\mbox{}

\end{member}
\end{members}

\end{mruleclass}
\newpage
\begin{mruleclass}{RC\_ADJVOICE}
\begin{classdescr}
\kind obligatory rule class.
\classtask This rule class exists for reason of isomorphy only.

\nofilters

\nospeedrules

\noplannedrules

\norulesnotince

\begin{comments}
\end{comments}
\end{classdescr}

\begin{members}

\begin{member}
\rulename RADJVOICEdefault
\ruletask This rule exists for reason of isomorphy only.
\file dutch:rc\_adjvoice-tc\_qphop.mrule (mrules37)
\semantics Lactive
\example all adjectival structures
\remarks
\end{member}
\end{members}

\end{mruleclass}
\newpage
\begin{mruleclass}{TC\_ADJAktionsartcalculation }
\begin{classdescr}
\kind obligatory transformation class
\classtask Give attribute {\em aktionsarts} a value

\nofilters

\nospeedrules

\noplannedrules

\norulesnotince

\begin{comments}
\end{comments}
\end{classdescr}

\begin{members}

\begin{member}
\rulename tADJaktstative1
\ruletask Set the value for the attribute {\em aktionsarts} to stative, 
the aktionsart for all adjectives.
\file dutch:tempadj1.mrule (mrules79)
\semantics \nosemantics
\example all adjectives
\remarks\mbox{}

\end{member}
\end{members}

\end{mruleclass}
\newpage
\begin{mruleclass}{TC\_ADJzichspelling}
\begin{classdescr}
\kind obligatory transformation class
\classtask To spell out non-argument reflexive and reciprocal pronouns.
\classremarks
reciprocals other than {\em elkaar} are accepted in analysis, but not generated

\nofilters

\nospeedrules

\noplannedrules

\norulesnotince
\end{classdescr}

\begin{members}
\begin{member}
\rulename TADJZichspellingdefault
\ruletask  To let nonreflexive, nonreciprocal adjectives pass the
TC:Zichspelling. 
\file dutch:tc\_adjzichspelling.mrule (mrules36)
\semantics \nosemantics
\example x1 ziek $\rightarrow$ x1 ziek  
\remarks\mbox{}

\end{member}
\begin{member}
\rulename TADJZichspelling1
\ruletask To spell out the reflexive pronoun {\em zich} if the antecedent is 
subject NPVAR.
\file dutch:tc\_adjzichspelling.mrule (mrules36)
\semantics \nosemantics
\example 
x1 bewust van $\rightarrow$  x1 me/je/zich/ bewust van etc. 
\remarks\mbox{}

\end{member}
\begin{member}
\rulename TADJZichspelling2
\ruletask To spell out the reflexive pronoun {\em zich} if the antecedent
is an object NPVAR. 
\file dutch:tc\_adjzichspelling.mrule (mrules36)
\semantics \nosemantics
\example Possibly none. 
The NP argument of ergative reflexive adj's would have to behave as 
an object. The only adj that is considered reflexive is `(on)bewust'.
This adj cannot be considered to be ergative. 
 \remarks\mbox{}

\end{member}
\begin{member}
\rulename TADJZichspelling3
\ruletask To spell out the reflexive pronoun zich if the antecedent is a
subject CNVAR.
\file dutch:tc\_adjzichspelling.mrule (mrules36)
\semantics \nosemantics
\example x1 bewust $\rightarrow$  x1 me/je/zich/ etc. 
bewust 
\remarks\mbox{}

\end{member}
\begin{member}
\rulename TADJZichspelling4
\ruletask To spell out {\em zich} if the antecedent is an object CNVAR.
\file dutch:tc\_adjzichspelling.mrule (mrules36)
\semantics \nosemantics
\example Possibly none. Cf. TADJZichspelling2
\remarks\mbox{}

\end{member}
\begin{member}
\rulename TADJReciprocalspelling1
\ruletask To spell out the reciprocal {\em elkaar} 
if the antecedent object 
 is NPVAR.
\file dutch:tc\_adjzichspelling.mrule (mrules36)
\semantics \nosemantics
\example No examples found yet
\remarks\mbox{}

\end{member}
\begin{member}
\rulename TADJReciprocalspelling1a
\ruletask To spell out all reciprocals other than {\em elkaar} if the antecedent object 
 is NPVAR.

\file dutch:tc\_adjzichspelling.mrule (mrules36)
\semantics \nosemantics
\example No examples found yet
\remarks\mbox{}

\end{member}
\begin{member}
\rulename TADJReciprocalspelling2
\ruletask To spell out the reciprocal {\em elkaar} in case of a 
CNVAR antecedent.
\file dutch:tc\_adjzichspelling.mrule (mrules36)
\semantics \nosemantics
\example de mensen die elkaar ADJ zijn. (No examples found yet.)
\remarks\mbox{}

\end{member}
\begin{member}
\rulename TADJReciprocalspelling2a
\ruletask To spell out all reciprocals other than {\em elkaar} in case of a 
CNVAR antecedent.
\file dutch:tc\_adjzichspelling.mrule (mrules36)
\semantics \nosemantics
\example Possibly none. 
\remarks\mbox{}

\end{member}
\begin{member}
\rulename TADJReciprocalspelling3
\ruletask To spell out the reciprocal {\em elkaar} 
in prepobj if the antecedent 
subject is NPVAR.

\file dutch:tc\_adjzichspelling.mrule (mrules36)
\semantics \nosemantics
\example Possibly {\em Ze zijn bekend met elkaar} as 
translation of  {\em They are acquainted }.\\
Zij zijn aan elkaar gewaagd/verwant. (Translation into 
English is a non-transitive predicate ({\em to be well matched})
\remarks\mbox{}

\end{member}
\begin{member}
\rulename TADJReciprocalspelling3a
\ruletask To spell out all reciprocals in prepobj other than {\em elkaar} 
in prepobj if the antecedent 
subject is NPVAR.

\file dutch:tc\_adjzichspelling.mrule (mrules36)
\semantics \nosemantics
\example zij zijn aan elkander/mekaar/malkander gewaagd/verwant 

\remarks\mbox{}

\end{member}
\begin{member}
\rulename TADJReciprocalspelling4
\ruletask To spell out reciprocal {\em elkaar} in prepobj 
in case of 
a CNVAR antecedent.
\file dutch:tc\_adjzichspelling.mrule (mrules36)
\semantics \nosemantics
\example Possibly: de mensen die aan elkaar verwant zijn
\remarks\mbox{}

\end{member}
\begin{member}
\rulename TADJReciprocalspelling4a
\ruletask To spell out all reciprocals other than {\em elkaar} in prepobj 
in case of a CNVAR.
\file dutch:tc\_adjzichspelling.mrule (mrules36)
\semantics \nosemantics
\example Possibly: de mensen die aan elkander verwant zijn
\remarks \mbox{}
\end{member}
\end{members}

\end{mruleclass}

\newpage
\begin{mruleclass}{TC\_ADJQPHopping}
\begin{classdescr}
\kind Presently this is an optional transformation class.
\classtask To account for the movement of QPs into postadjectival position.

\begin{filters}
\item   The associated filters have not been written yet.
\end{filters}

\nospeedrules

\noplannedrules

\norulesnotince

\classremarks\mbox{}



\begin{enumerate}
\item The hopping is 
obligatory for QPs with .hop =true. Therefore, filters should be added to 
guarantee the application.
\item For simple QPs the hopping could also be 
effectuated by the rule introducing the  into the ADJP. However this 
alternative would be complicated in view of the complex QPs with .hop = true.
For example {\em bijna genoeg}.
  \item 
The relevant set of QPs includes those with {\em genoeg} as a head, with 
a {\em om te}-modifier in postadjectival position.\\

\end{enumerate}
\end{classdescr}

\begin{members}

\begin{member}
\rulename TADJQPhopping1
\ruletask To move bare QPs with attribute .hop = true into postadjectival 
position. 
\file dutch:rc\_adjvoice-tc\_qphop.mrule (mrules37)
\semantics \nosemantics
\example \mbox{}\\
\begin{enumerate}
  \item 
{\tt degreemodrel}/zat +  lang $\rightarrow$ lang + {\tt hoprel}/zat 
  \item
{\tt degreemodrel}/genoeg +  mooi $\rightarrow$ mooi + {\tt hoprel}/genoeg
\end{enumerate}
\remarks\mbox{}

\end{member}
\begin{member}
\rulename TADJQPhopping2
\ruletask To move Qs out of complex 
QPs with attribute .hop = true into postadjectival position.
\file dutch:rc\_adjvoice-tc\_qphop.mrule (mrules37)
\semantics \nosemantics
\example

{\tt degreemodrel}/bijna genoeg + {\tt head}/lang $\rightarrow$ \\
{\tt premodrel}/bijna + {\tt head}/lang + {\tt hoprel}/genoeg
\remarks\mbox{}

\begin{enumerate}
\item 
The superdeixis value for the elements in {\tt premodrel} is {\em omegadeixis}
 at this 
stage in generation. It need not be adapted here. The superdeixis  value for 
the 
QP in {\tt hoprel} will be adapted (in generation: set to omegadeixis) by TADJsuperdeixisadaptation1.
\item 


The transformation presumes that initially 
{\em bijna} and {\em genoeg} constitute
one QP-phrase. This is analysis is favoured over one that considers {\em bijna}
a modifier to the ADJP {\em genoeg lang} because of the parallellism with
NPs such as {\em bijna genoeg boeken}. Unlike adjectives, nouns do not 
allow {\em bijna} as an independent modifier: {\em bijna gezond} versus
{\em *bijna boeken}. 
\end{enumerate}
\end{member}
\end{members}

\end{mruleclass}
\end{document}
