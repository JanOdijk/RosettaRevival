
\documentstyle{Rosetta}
\begin{document}
   \RosTopic{Rosetta3.Linguistics.minutes}
   \RosTitle{Notulen Linguistenvergadering 13-11-1989}
   \RosAuthor{Margreet Sanders}
   \RosDocNr{417}
   \RosDate{November 14, 1989}
   \RosStatus{concept}
   \RosSupersedes{-}
   \RosDistribution{Linguists}
   \RosClearance{Project}
   \RosKeywords{minutes}
   \MakeRosTitle
%
%
\begin{description}
\item[Aanwezig:] Lisette Appelo, Franciska de Jong (wat later), 
                 Elly van Munster, Jan Odijk, Joep Rous, Margreet Sanders,
                 Andr\'{e} Schenk, Harm Smit
\item[Afwezig:] --
\item[Agenda:]\mbox{}
  \begin{enumerate}
  \item Documentatie
  \item Bonus
  \item M-regels
  \item Rosetta 3D
  \end{enumerate}
\end{description}

\section{ Documentatie}
Jan O.\ deelt een lijst uit van de reeds verschenen documenten met een voorstel 
wie ze moeten gaan bespreken. Hij heeft rekening gehouden met een eerlijke 
bespreek-belasting, voor zover mogelijk. De lijst volgt hieronder nog een keer, 
met de data waarop de documenten zullen worden besproken, voor zover reeds 
bekend (de afspraken met Margreet gaan voor, omdat ze er nog maar kort is). 
Hiermee is Jan's voorstel impliciet aanvaard.

Tevens zal Jan een lijst maken met de documenten die nog moeten 
ver\-schij\-nen.
Deze lijst wordt eveneens als bijlage toegevoegd.

Van iedere bespreking dient een klein verslagje (max.\ een half A4-tje) aan Jan 
O.\ te worden gegeven. Hij zal zorg dragen voor bundeling en vermenigvuldiging.

\section{Bonus}
De implementatie van het Bonus-systeem is volgens Joep het makkelijkst als er 
maar \'{e}\'{e}n soort bonus is (hij heeft WBONUS ge\"{i}mplementeerd), die 
verschillende ranges heeft voor strong bonus (d.w.z.\ de vertaling komt er 
alleen uit als er geen andere is) en weak bonus (d.w.z.\ de vertaling komt er 
wel uit, maar alleen als laatste). Lisette zal Joep vragen een handige waarde 
voor strong bonus te kiezen.

\section{M-regels}
Ren\'{e} heeft de implementatie van M-regels veranderd, waardoor een 
concatenatie van condities (C1 AND C2 AND C3 AND ...) nu wordt 
ge\"{i}nterpreteerd als IF C1 THEN IF C2 THEN IF C3 THEN IF .... Hiermee wordt 
het verstandig om als eerste conditie in een CA-paar een zeer restrictieve te 
nemen, waardoor de rest niet meer ge\"{e}valueerd hoeft te worden. Het schijnt 
dat de nieuwe implementatie twee maal zo snel werkt als de vorige.

\section{Rosetta 3D}
Margreet vraagt of de activiteiten waar iedereen nu mee bezig is op elkaar 
aansluiten en toewerken naar Rosetta 3D. Jan O.\ probeert alle zinnen uit de 
testbank ook in het Spaans te laten werken, en bekijkt verder welke 
constructies uit de twee TAT-brieven nog gemaakt moeten worden voor het 
Nederlands. Franciska zal de woorden die nog aan het woordenboek moeten worden 
toegevoegd in een lijst zetten. Voor de goede voortgang lijkt bespreking van 
de afstemming tussen 
de ADJPPROP- en de zinsgrammatica noodzakelijk. Hier\-over worden afspraken 
gemaakt (zie de lijst hieronder). 
Verdere bespreking wordt uitgesteld tot het 3D-groepje (Joep, 
Andr\'{e}, Jan O.\ en Franciska) heeft vergaderd. Zij zullen ook het stuk van 
Jan O.\ over Linguistische Effici\"{e}ntie (doc.\ 407) meenemen in hun 
bespreking.

\newpage
\appendix
\noindent
\section{ Bespreking van bestaande documentatie}
\small
\begin{tabular}{llllc}
\hline
 nr  & auteur(s)  & titel                         & Besprekers      & datum\\
\hline
  I. &         & Algemeen:                                       &&\\
\hline
 312 & JO      &  Rosetta3 Linguistic Documentation        & -- & \\
 365 & JO      & The Rosetta3 Subgrammars *    & iedereen       & 20/11\\
 327 & EA,JO   & Modals in Rosetta3             & iedereen       & 4/12\\
 368 & JO      & Scope in Rosetta3              & iedereen       & 8/1\\
 367 & JO      & General Remarks on writing M-rules  & iedereen+Ren\'{e} & 8/1\\
 320 &  LA     & Superdeixis in Rosetta3        & iedereen        & 15/1\\
 325 &  JO     & Guide to determining Verbpatterns   & iedereen   & 30/11\\
 263 &  EA     & Time                           & MS,JO,EvM       & \\
\hline
II.  & &        Nederlands:                         &&\\
\hline
    &  &        Lexicon:                            &&\\
\hline
313 & JO       & BVERB Attributes                   & -- & \\
\hline
    &  &         Morph:                             &&\\
\hline
329 & HS &       Morphology, inflection             & MS,EvM & \\
330 & HS &       Morphology, derivation             & MS,EvM & \\
331 & HS &       Morphology, rules                  & MS,EvM & \\
\hline
    &      &     Syntax:                            & & \\
\hline
315 & JO       & Voice in Dutch                     & iedereen     & 22/1\\
243 & JO       & Surface rules,SENTENCE   & ged. iedereen;     & 29/1\\
    &          &                          & AS, MS             & \\
240 & JO       & Verbpatterns Dutch       & ged. iedereen;   & 29/1\\
    &          &                          & HS, MS            & \\
374 & FdJ      & Adjpatterns Dutch                  & MS, EvM  & \\
307 & JO       & Rpronouns                          & AS, MS & \\
308 & JO,EA,AS & VERBPPROPformation                 & EA, AS   & \\
214 & JO,LA    & XPPROPtoCLAUSE                     & EA, AS    & \\
335 & JO,EA,AS & CLAUSEtoSENTENCE                   & EA, AS  & \\
384 & JO       & Utterance                          & HS, EvM & \\
385 & JO       & PREPPformation                     & HS, EvM & \\
319 & JO       & PREPPPROPformation                 & HS, EvM  & \\
321 & JO       & PREPPPROPtoPREPPFORMULA            & HS, EvM & \\
322 & JO       & PREPPFORMULAtoPREPPPROP             & HS, EvM & \\
376 & FdJ,LA   & ADJPPROPformation                  & JO, EA       & 21/11\\
397 & FdJ      & ADJPPROPtoADJPFORMULA              & JO, EA       & 21/11\\
402 & FdJ,LA   & ADJPformulaToadjpprop              & JO, EA       & 21/11\\
--  & FdJ     & NPsubgrammars                       & EA, MS        & 4/12 \\
409    & FdJ   & Cnformation                        & EA, MS       & 4/12\\
\hline
\end{tabular}

\newpage

\small
\noindent
\begin{tabular}{llllc}
\hline
III.  & &        Engels:                         &&\\
\hline
     & &         Morph:                         & & \\
\hline
306 & MS        & Lextree rules                 & HS,EvM  & \\
\hline
    &  &          Syntax:                       & & \\
\hline
316 & MS         & Derivation Subgrammars       & FdJ,HS & \\
310 & MS,EA,AS   & VERBPPROPformation           & JO,EvM   &         15/11\\
326 & MS         & XPPROPtoCLAUSE               & JO,EvM    &        20/11\\
370 & MS,EA       & CLAUSEtoSENTENCE            & JO,EvM     &       22/11\\
379 & MS           & Utterance                  & EA, AS      &      14/11\\
383 & MS           & PREPPformation             & EA, AS       &     14/11\\
380 & MS           & PREPPPROPformation         & EA, AS        &    14/11\\
381 & MS            & PREPPPROPtoPREPPFORMULA   & EA, AS         &   14/11\\
382 & MS            & PREPPFORMULAtoPREPPPROP   & EA, AS          &  14/11\\
391 & MS           & ADVPformation              & HS, FdJ & \\
386 & MS &         ADvPPROPformation            & HS, FdJ & \\
387 & MS           & ADVPPROPtoADVPformula      & HS, FdJ & \\
388 & MS         & ADVPformulatoprop            & HS, FdJ & \\
395 & MS         & Existpropformation           & JO, FdJ & \\
396 & MS         & Identpropformation           & JO, FdJ & \\
\hline
IV.  & &        Spaans:                         &&\\
\hline
398 & LA         & SE                            & AS,JO,MS,EvM & \\
266 & EvM        & Verbpatterns                  & AS, JO  & \\
301 & EvM,EA,AS  & VERBPPROPformation            & MS, FdJ & \\
334 & EvM,EA     & XPPROPtoCLAUSE                & MS, FdJ & \\
328 & EvM,EA,JO  & CLAUSEtoSENTENCE              & MS, FdJ & \\
336 & EvM        & Existential                   & MS, JO & \\
337 & EvM        & Identificational              & MS, JO & \\
\hline
\end{tabular}

\newpage
\noindent
\normalsize
\section{Nog te verschijnen documenten}

\subsection{General}
\begin{description} 
  \item[LSDOMAINT] JO One document will be written for all three languages.
  \item[LSAUXDOMAIN] JO One document will be written for all three languages.
  \item[LSMRUQUO] JO One document will be written for all three languages.
  \item[Transfer] JO One general document will be written
\end{description}

\subsection{Dutch}
\begin{description}
  \item[Surface parser] (pp, advp) JO
  \item[Surface parser]( np, adjp) FdJ (JO)
  \item[Morphology, Segmentation Rules] HS
  \item[Lexicons] HS
  \item[NP subgrammars] FdJ 
  \item[ADVP  subgrammars] JO
\end{description}

\subsection{English}

\begin{description}
  \item[Morphology, Segmentation] MS
  \item[NP subgrammars] FdJ
  \item[ADJP subgrammars] FdJ
\end{description}

\subsection{Spanish}

\begin{description}
  \item[Morphology] EvM (Only a LIAversion exists)
  \item[NP subgrammars] FdJ
  \item[ADJP subgrammars] (partially finished) FdJ
  \item[PREPP subgrammars] EvM, to be finished by JO
  \item[ADVP  subgrammars] EvM, to be finished by JO
\end{description}

\subsection{Interlingua}
Interlingua documentation will be written by Jan O.

\subsection{Topic-Oriented Documents}

\begin{description}
  \item[The treatment of adverbs] JO
  \item[Polarity] JO 369
  \item[Verb-Raising] JO
  \item[Idioms and Semi-idioms] AS
  \item[Control] JO 373
  \item[Negation] JO
  \item[Verbpatterns, English] MS (an updated version of doc.\ 248)
  \item[Filling of BVERB, English] MS (an updated version of doc.\ 318)
\end{description}


\subsection{Other Documents}

\begin{description} 
  \item[Document Index] JO
  \item[Inspiration Sources] JO 311
\end{description}

\end{document}


