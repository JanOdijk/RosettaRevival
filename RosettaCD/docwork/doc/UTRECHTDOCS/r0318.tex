
\documentstyle{Rosetta}
\begin{document}
   \RosTopic{Rosetta3.doc.dictionary.English}
   \RosTitle{Lexical Entry Specification: English BVERB}
   \RosAuthor{Margreet Sanders, Petra de Wit}
   \RosDocNr{318}
   \RosDate{December 20, 1991}
   \RosStatus{approved}
   \RosSupersedes{-}
   \RosDistribution{Project}
   \RosClearance{Project}
   \RosKeywords{BVERB, dictionary, lexical entry, English}
   \MakeRosTitle
%
%

\section{Introduction}

This document describes the attributes of the English basic verb (BVERB), and their 
interpretation. It also serves as a guide for dictionary filling, with a few 
examples showing how to decide on specific attribute values.
Part of the document is a copy of a similar 
document for Dutch BVERBs, written by Jan Odijk (R0310: {\em Bverb Attributes. 
Guide to dictionary Filling\/}).

 
\section{The attributes of the BVERB record}
In Appendix A, the definition of the BVERB record in the English Domain T is 
given. The attributes mentioned there are described in more detail here. The 
full set of possible attribute values (which may include values not used for 
 BVERB itself) are given in Appendix B. The attributes are:

\begin{description}
\item[conjclasses] This attribute indicates the conjugation class or classes of
the verb. There are 15 classes, some of which may co-occur:
\begin{description}
  \item [0] For the verb {\em to be}.
  \item [1] For verbs that have an irregular past tense and past participle 
that differ from each other and from the base form (e.g. \ {\em ring - rang - rung\/}).
  \item [2] For verbs that have an irregular past tense and past participle 
that have the same form (e.g. \ {\em lead - led - led\/}).
  \item [3] For verbs that have an irregular past tense and a past participle
that is identical to the base form (e.g. \ {\em put - put - put\/}).
  \item [4] For verbs that only have an irregular past participle. The only 
verb belonging to this class ({\em lade}, mentioned in doc. 115), however, 
does have a 
regular past participle next to the irregular form. Therefore this class is no 
longer necessary, unless other examples can be found.
  \item [5] For verbs that have a second irregular past tense (e.g. \ {\em 
dive - dived/dove - dived\/}).
  \item [6] For verbs that have a second irregular past participle. Originally 
this class was intended for verbs such as 
(\ {\em shave - shaved - shaved/shaven\/}), but {\em shaven} is now treated as 
an adjective.
  \item [7] For verbs that have a third irregular past participle. The sole 
member of this class is (\ {\em stride - strode - strode/stridden/
strid\/}), but personally I have never heard of the past participle {\em strid}
.
  \item [8] For verbs that have an irregular past tense, but the past participle 
 is identical to the base form (e.g. \ {\em come - came - come\/}).
  \item [9] For verbs that have an irregular past participle, but the past 
tense is identical to the base form (e.g. \ {\em beat - beat - beaten\/}).
  \item [10] For regular verbs that do not double their consonants (e.g \ {\em 
knock - knocked - knocked\/}). 
  \item [11] For regular verbs that double their consonants (e.g. \ {\em clap - 
clapped - clapped\/}).
  \item [12] For modal verbs that have both a present tense form and a past 
tense form (e.g \ {\em may - might\/}).
  \item [13] For modal verbs that do not have a past tense form (e.g. \ {\em 
must\/}).
  \item [14] For modal verbs that do not have a present tense form(e.g. \ {\em 
used to\/}).
\end{description}
 Given the 
description of these classes in the English Morphology component of Rosetta3
(see doc.\ 115, {\em Rosetta 3 English Morphology, Inflection\/} or doc.\ 423, 
{\em Rosetta3 English Morphology: Glue Rules and Segmentation Rules\/}),
the attribute can be filled using any standard dictionary. Perhaps automatic
filling is also possible.

\item[ingform] This attribute says how the ing-form of a verb is formed. There 
are five possible values:
  \begin{description}
  \item[reging] The default value, for ordinary ing-forms, also for those 
verbs that drop their final {\em -e\/} (e.g.\ {\em cycle - cycling\/})
  \item[cding] For verbs doubling their final consonant (e.g.\ {\em swim - 
swimming\/})
  \item[cdreging] For verbs that have both options, although sometimes only one 
depending on whether American or British English is concerned (e.g.\ {\em 
travelling - traveling\/})
  \item[noing] For modals (e.g.\ {\em can\/})
  \item[irring] For irregular verbs, especially those that do not drop their 
final {\em -e\/} (e.g.\ {\em singe - singeing\/})
  \end{description}

\item[sform] This attribute indicates how the third person singular of the verb
must be spelled. The attribute can have four values:
  \begin{description}
  \item[regS] For ordinary verbs (e.g.\ {\em sing - sings\/})
  \item[regEs] For verbs that end in a sibilant in pronunciation and do not end 
in an {\em -e\/} in spelling (e.g.\ {\em banish - banishes\/})
  \item[noS] For modals (e.g.\ {\em must - must\/})
  \item[irrS] For irregular verbs (e.g.\ {\em do - does\/})
  \end{description}

\item[particle] This attribute contains the key of the particle the verb
demands. If the verb does not need a particle, the attribute is set to zero. 
The filling 
cannot be done automatically, since most dictionaries do not distinguish well 
enough between particles and prepositions. Tests for discriminating the two are:
  \begin{itemize}
  \item Pronunciation: if the verb is stressed and the P is not, the P is a 
preposition. If this is the other way around, the P is a particle. (example: He 
has looked UP the word.\ vs.\ He has LOOKED at the word.)
  \item If a P can occur both before and after the direct object NP in a 
   sentence like the one below, it is a particle (if it can occur only before 
   the NP, it is a preposition):\\
  {\em They V the man P\/}\\
  {\em They V P the man\/}\\
  (example: call up sb.\ vs.\ call on sb.)
  \item If a P can occur after, but not before, a direct object Pronoun in a
   sentence like the one below, it is a particle (if it is the other way around
   , it is a preposition):\\
  {\em They V it P\/}\\
  {\em $^{*}$They V P it\/}\\
  (example: break off sth.\ vs.\ approve of sth.)
  \end{itemize}

\item[reflexivity] This attribute indicates whether the verb is reflexive,
reciprocal, or none of these.   The attribute has four possible values: 
   \begin{description}
   \item[notreflexive] The verb is neither reflexive nor reciprocal (e.g.\
     {\em  buy}). This is the default value.
   \item[dobjrefl] The verb is reflexive; It requires {\em x-self\/} to appear. 
   The reflexive functions as a direct object (e.g.\ {\em perjure oneself\/}).
   \item[indobjrefl] The verb is reflexive; It requires {\em x-self\/} to 
   appear. The reflexive functions as an indirect object (e.g.\ {\em have 
   oneself a ball\/}).
   \item[reciprocal] Originally, this attribute was used to indicate that 
the verb may delete a reciprocal pronoun ({\em each other
   \/}), when that is the direct object (e.g.\ {\em  meet\/} in {\em they met
   \/}). It is not clear now whether the reciprocal is either deleted in 
English or introduced in Dutch. 
   \end{description}

\item[subc] This attribute is used to subclassify verbs. The values are:
  \begin{description}
  \item[modalverb] Only for the verbs {\em can (could), may (might), must; 
shall/will;    need, dare (when they select a bare infinitive)\/}
  \item[beverb] For the verb {\em be\/}
  \item[haveverb] For the auxiliary verb {\em have\/}
  \item[doaux] For the auxiliary verb {\em do\/}
  \item[mainverb] For all other verbs
\end{description}

\item[thetavp] This attribute is used to indicate the number of arguments and their 
partitioning (inside VP or outside VP). The following values are distinguished:
  \begin{description}
  \item[vp000] Verbs not taking any arguments, e.g.\ {\em rain, thaw, drizzle\/}
  \item[vp010] Ergative and Raising Verbs taking one argument, e.g.\ {\em  
    appear\/}. Examples of ergative verbs in English are {\em melt} and {
\em freeze}. Randall(??) distinguishes between ergative and non-ergative 
verbs on the basis of
whether the verb can select a resultative phrase when used 
intransitively.{\em He froze the water solid} and {\em 
The water froze solid}. Since resultative clauses require an object antecedent 
within VP, these verbs can be analyzed as ergative. This, however, is not 
represented in the dictionary yet.
  \item[vp100] Intransitive Verbs, e.g.\ {\em  dance, work, sleep\/}
  \item[vp120] Dyadic verbs, a.o.\ transitive verbs, e.g.\ {\em see, build,
             try, believe\/}
  \item[vp012] Dyadic Ergatives and Raising verbs, e.g.\ {\em seem (to sb.)}
  \item[vp123] Ditransitive verbs, e.g {\em give, sell, accuse (of)\/}
  \item[vp132] This value has been added to account for ditransitive verbs 
      that switch the `normal' order of arguments (e.g.\ {\em allow\/} in {\em
      He was allowed to swim\/}). Notice that not all argument switches are 
possible yet. For instance, the value vp210 cannot be accounted for in the 
pattern rules if the verb select a sentence, since sentences cannot be 
generated in subject position in the pattern rules. In future, 
argument switches will have to be dealt with more systematically in English.
  \item[omegathetavp] a default value; of no significance for BVERB
  \item[thetavpaux] a special value for auxiliary verbs (e.g.\ {\em be\/})
  \end{description}

\item[synvps] This set attribute describes the verbpatterns a verb may take, 
i.e.\ it indicates how the internal arguments the verb takes should be realised.
Apart from idiom patterns, some 105 different patterns are distinguished so far 
for English. For more
information, see the document on English verbpatterns (doc.\ 248: 
{\em Verbpatterns of 
English\/}) or the documentation on the transformations realising the 
verbpatterns (in doc.\ 310: {\em Rosetta3 English M-rules: 
VerbppropFormation\/}). All values currently implemented have 
also been included in appendix B.

\item[adjuncts] This set attribute specifies the number of adjuncts
a verb selects as well as their syntactic realization. Although adjuncts are 
selected by the verb, i.e they cannot freely be added to all verbs as 
true adverbials can, they are always optional.
The following values are distinguished:
\begin{description}
  \item [resAP] For resultative adjective phrases, e.g. \ {\em He painted the 
house green\/}.
  \item [resPP] For resultative prepositional phrases, e.g \ {\em He devided 
the cake into two equal parts\/}.
  \item [benfactNP] For benefactive noun phrases, e.g. \ {\em He knitted me a 
sweater\/}.
  \item [benfactPP] For benefactive prepositional phrases. e.g. \ {\em He 
knitted a sweater for me\/}.
  \item [subjcomit] For subject-oriented co-arguments, e.g \ {\em He discussed 
the plan with his friends\/}.  
  \item [objcomit] For object-oriented co-arguments, e.g. \ {\em He compared 
Marx's work with Hegel's\/}.
\end{description}

\item[CaseAssigner] This Boolean attribute indicates whether the verb 
 is a Case Assigner
or not. Intransitive and transitive verbs have this attribute set to 
{\em  true\/},
Ergative verbs (1-place and 2-place) have this attribute set to {\em false\/}, 
except for those that need not move their object to subject position 
{\em It rained complaints}.
Verbs taking no arguments have this attribute set to the default value, which 
is {\em  true\/}.

\item[possvoices] This settype attribute indicates which voices the verb can
be in. The values that can appear in the set are:
  \begin{description}
  \item[Active] Active voice, e.g.\ {\em He buys the book\/}.
  \item[Passive] Passive voice, e.g.\ {\em The house was built in 1980\/}.
  \end{description}
The default setting is [active, passive].

\item[prepkey1, prepkey2] These attributes indicate the S-keys of the 
prepositions used in the prepositional object(s). 
There are two prepkey attributes, for cases such as 
{\em talk to someone about something\/}. If two prepositions are specified, 
prepkey1 goes with the `indirect object' or `goal', prepkey2 with the 
`direct object' or `theme'.

\item[oblcontrol] This attribute indicates whether or not the 
control relation which is required for the subject of an embedded infinitival 
complement of the verb is one of `obligatory control'. For more information on 
the requirements for obligatory control, see doc.\ 111 by Jan Odijk ({\em The 
interpretation of non-overt subject in Rosetta3\/}). Tests to distinguish 
obligatory control from non-obligatory control are:
\begin{itemize}
  \item [passivisation] Only non-obligatory control verbs can passivize:\\
 {\em it was decided to leave} vs. {\em it was tried to go}
  \item [antecedent deletion] Only non-obligatory control verbs can delete the 
antecedent:\\
{\em I allow smoking} vs {\em I allow to smoke}      
  \item [interpretation] In case of non-obligatory control the non-overt 
subject receives an arbitrary interpretation:\\
{\em The committee decided to leave} vs {\em The committee promised to leave}
\end{itemize}
The possible values are:
   \begin{description}
   \item[omegaoblcontrol] For verbs that do not take such an embedded 
complement
   \item[yesoblcontrol] For verbs which must have an obligatory controller for 
the embedded subject (e.g.\ {\em He tried to go\/})
   \item[nooblcontrol] For verbs which do not need an obligatory controller for 
the embedded subject. (e.g. \ {\em The committee decided to leave at six\/})
   \end{description}
  
\item[controller] This attribute indicates what the controller of the subject
of an embedded infinitival complement is for this verb. It need be filled only 
for verbs that do not have {\em omegaoblcontrol\/} as value for {\bf controller
}. The values are:
   \begin{description}
   \item[subject] The subject is the controller
   \item[object] The object is the controller
   \item[indobj] The indirect object is the controller
   \item[prepobj] The object of the prepobj PP is the controller
   \item[none] No controller restriction is specified (default value)
   \end{description}
The relation mentioned is the relation before application of object-to-subject
transformations have applied (in ergatives and raising verbs).

Tests to determine controllerhood: 
  \begin{itemize}
  \item  If a verb can replace V in the following sentences, then controller = 
 {\em subject\/}:\\
  {\em I V him to perjure myself\/}\\
  {\em I V him not to lose my temper\/}\\
  (example: promise)
  \item If a verb can replace V in the following sentences, then controller = {
\em object\/} or {\em indobj\/}, depending on the relation of {\em  him\/}:\\
  {\em I V him to perjure himself\/}\\
  {\em I V him not to lose his temper\/}\\
  (example: persuade)
  \end{itemize}
Similar tests can be constructed for controllers in prepositional objects.
Notice that sometimes the controller also has to be specified for verbs taking 
finite complements if this complement is a direct translation of an infinitival 
complement in Dutch {\em e.g denken, think}
\item[thatdel] This attribute is of importance only for verbs that take a 
sentence starting with the conjunction {\em that\/} as an argument. The values 
are:
  \begin{description}
  \item[omegadel] Default value for verbs that do not take a {\em that\/}-
sentence
  \item[yesdel] For verbs that allow deletion of {\em that\/} (e.g.\ {\em I 
know he left\/})
  \item[nodel] For verbs that do not allow {\em that\/}-deletion (e.g.\ {\em I 
feel that you are right\/})
  \end{description}
It seems that verbs that allow their complementizer to be deleted are excactly 
those verbs that select the synpattern {\em synSOPROSENT} (see talk given by 
Culicover in Utrecht at the Going Romance conference.) This correlation
, however, has not 
been implemented yet.
\item[classes]  This attribute is used to partition verbs into (semantic) classes
with respect to the type of event described. The attribute value is used in the TIME 
rules to compute {\em  Aktionsart\/}. The values are:
  \begin{description}
  \item[stativeclass] For verbs that describe a situation which can last for an
arbitrarily long time (e.g.\ {\em belong\/} in {\em This book belongs to me\/}).
In English, these verbs usually do not get an -ing form.
  \item[dynstativeclass] As for statives, but still with an -ing form (e.g.\ 
{\em lie\/} in {\em It is lying on the table\/})
  \item[movementclass] For verbs that describe a movement (e.g.\ {\em push\/} 
in {\em He pushed the car onto the hard shoulder\/})
  \item[durativeclass] For verbs that describe an action which may last for 
some time (e.g.\ {\em sleep\/} in {\em He has slept for ten hours now\/})
  \item[momentaryclass] For verbs that describe an action which is inherently
transient (e.g.\ {\em explode\/} in {\em The bomb exploded\/})
  \item[iterativeclass] As for duratives, but the action consists of repeated 
`subactions' (e.g.\ {\em knock\/} in {\em I have been knocking on your door for 
five minutes\/})
  \end{description} 
For a flow chart on how to assign these attributes, see the Lexic document.

\item[req, env] These attributes are used to indicate the 
    {\em  polarity\/} properties of the verb, i.e.\ whether it requires a 
    question, a positive or a negated sentence, or creates
    such an environment for other elements (e.g.\ for the determiner {\em any\/
}).
    The attribute has a 
    set value, but only for historical reasons. Sensible values are
  \begin{description}
   \item[{[pospol, negpol, omegapol]}] The default value, indicating that no
     relevant polarity properties hold.
   \item[{[pospol]}] The verb requires (req) or creates (env) a positively polar
     environment (McCawley(1988) mentions {\em already, rather and had better})
   \item[{[negpol]}] The verb requires (req) or creates (env) a negatively polar
     environment (e.g.\ {\em need\/} as an auxiliary has {\bf req} = negpol: 
     {\em Need you go now?\/} vs.\ $^{*}${\em I need go now\/})
  \end{description}

\item[KEY] This attribute indicates the language specific S-key of the BVERB, 
which is related to an interlingual M-key.

\end{description}

\newpage
\section{Consistency Checks on the filling of BVERBs}

The following consistency checks are currently performed:

\begin{verbatim}
< BVERB
{
: IMPLIES((thatdel <> omegadel),(synvps * LSAUXDOM_thatcomplvps <> []))
  " thatdel <> omegadel requires appropriate verbpattern"

: IMPLIES((synvps * LSAUXDOM_thatcomplvps <> []), (thatdel <> omegadel) )
  " one of the verbpatterns requires that thatdel <> omegadel "
}
: IMPLIES((prepkey1 <> 0), (synvps * LSAUXDOM_prepobjvps <> []) )
  " prepkey1 <> 0 requires appropriate verbpattern"

: IMPLIES((synvps * LSAUXDOM_prepobjvps <> []), (prepkey1 <> 0) )
  " one of the verbpatterns requires that prepkey1 is filled "

: IMPLIES((prepkey2 <> 0), (synvps * LSAUXDOM_twoprepobjvps <> []) )
  " prepkey2 <> 0 requires appropriate verbpattern"

: IMPLIES((synvps * LSAUXDOM_twoprepobjvps <> []),(prepkey2 <> 0) )
  " one of the verbpatterns requires that prepkey2 <> 0 "

: IMPLIES((synvps = [synNoVPArgs]),(thetavp IN [vp000, vp100]) )
  " the synpattern requires that thetavp = vp000 or vp100 "

: IMPLIES((thetavp IN [vp000, vp100]),(synvps = [synNoVPArgs]) )
  " thetavp= vp000,vp100 requires synvps=[synNoVpargs]"

: IMPLIES((synPREPEMPTY IN synvps), false)
  " synPREPEMPTY is not a correct pattern for verbs. Use synEMPTY!"

: IMPLIES((thetavp IN [vp000, vp100, vp010]),(possvoices = [active]) )
  " thetavp requires and allows active voice only"

: IMPLIES((thetavp IN [vp100, vp120, vp123]),(caseAssigner = true) )
  " thetavp=vp100,vp120,vp123 requires caseassigner=true"

: IMPLIES((thetavp IN [vp010, vp012]),(caseAssigner = false) )
  " thetavp=vp100,vp120,vp123 requires caseassigner=false"

: IMPLIES((reflexivity IN [dobjrefl, indobjrefl,reciprocal]),(caseAssigner = false))
  " reflexive verbs require caseassigner=true"

: IMPLIES ((thetavp = vp000), (synvps <= LSAUXDOM_vp000vps))
  " synpattern inconsistent with thetavp" 

: IMPLIES ((thetavp = vp100), (synvps <= LSAUXDOM_vp100vps))
  " synpattern inconsistent with thetavp" 

: IMPLIES ((thetavp = vp010), (synvps <= LSAUXDOM_vp010vps))
  " synpattern inconsistent with thetavp" 

: IMPLIES ((thetavp = vp120), (synvps <= LSAUXDOM_vp120vps))
  " synpattern inconsistent with thetavp" 

: IMPLIES ((thetavp = vp012), (synvps <= LSAUXDOM_vp012vps))
  " synpattern inconsistent with thetavp" 

: IMPLIES ((thetavp = vp123), (synvps <= LSAUXDOM_vp123vps))
  " synpattern inconsistent with thetavp" 

{
: IMPLIES ((particle < aboutPARTkey),false)
  " Value of Particle not a particle key"

: IMPLIES ((particle > upPARTkey),false)
  " Value of particle is not a particle key"

againPARTkey and asidePARTkey are > upPARTkey!!
}
>
\end{verbatim}
\vspace{3 ex}
The following cases are marked as being PECULIAR:

\begin{verbatim}
< BVERB
: IMPLIES((11 IN conjclasses), (ingform = cding))
         $P " verbs of conjclass [11] ususally have consonant doubling "

: IMPLIES((ingform = cding), (conjclasses * [10] = []))
         $P " consonant doubling verbs ususally have conjclass [11] "
>
\end{verbatim}

\appendix

\newpage
\section{The BVERB record}

This section contains the definition of the BVERB record in DOMAIN T.
\begin{verbatim}
BVERBrecord      =
                      <
                       req:               polarityEFFSETtype:[pospol, negpol, 
                                                        omegapol]   
                       env:               polarityEFFSETtype:[pospol, negpol, 
                                                        omegapol]   
                       conjclasses:       conjclasseSETtype:[10]
                       ingform:           ingformtype:reging
                       sform:             sformtype:regS
                       particle:          keytype:0 
                       possvoices:        VoiceSETtype:[active, passive] 
                       reflexivity:       reflexivetype:notreflexive
                       synvps:            SynpatternSETtype:[]
                       thetavp:           Thetavptype:omegathetavp
                       adjuncts:          adjunctSETtype:[]
                       CaseAssigner:      CaseAssignertype:true
                       subc:              Verbsubctype:mainverb
                       oblcontrol:        oblcontroltype:omegaoblcontrol 
                       prepkey1:          keytype:0
                       prepkey2:          keytype:0 
                       controller:        controllertype:none
                       classes:           classSETtype:[durativeclass]
                       thatdel:           thatdeltype:omegadel      
                       KEY
                      >
\end{verbatim}

\newpage
\section{Types used in the BVERB record}
\begin{verbatim}
  adjunctSETtype      = SET OF adjuncttype;
  adjuncttype         = (ResAP, ResPP, SubjComit, ObjComit, BenfactNP,
                         BenfactPP, LocAdjunct, DirAdjunct, ResNP);
  CaseAssignertype    = BOOLEAN;
  ClassSETtype        = SET OF Classtype;
  Classtype           = (stativeclass, dynstativeclass, movementclass, 
                         durativeclass, momentaryclass, iterativeclass);
  conjclasseSETtype   = SET OF conjclasstype; 
  conjclasstype       = 0..14;
  Controllertype      = (subject, object, indobj, prepobj, none);
  ingformtype         = (reging,cding,cdreging,noing,irring);
  keytype             = INTEGER;
  oblcontroltype      = (yesoblcontrol, noOblcontrol, omegaOblcontrol);
  polaritytype        = (pospol, negpol, omegapol);
  polaritySETtype     = SET OF polaritytype;       
  polarityEFFSETtype  = {EFF}SET OF polaritytype;        
  Reflexivetype       = (notreflexive, dobjrefl, indobjrefl, reciprocal);
  sformtype           = (regS,regEs,noS,irrS);
  Synpatterntype      = (synASIFSENT, synBE, 
                         synCLAUSE, synCLOSEDADJPPROP, 
                         synCLOSEDADJPPROP_EMPTY, synCLOSEDADJPPROP_PREPNP, 
                         synCLOSEDGERUND, synCLOSEDINFSENT, 
                         synCLOSEDNPPROP, synCLOSEDNPPROP_EMPTY, 
                         synCLOSEDNPPROP_PREPNP,               
                         synCLOSEDTOSENT, synCLOSEDVERBPPROP,
                         synDIRCLOSEDPREPPPROP, synDIROPENPREPPPROP,
                         synDONP_DIROPENPREPPPROP, synDONP_EMPTY,
                         synDONP_LOCOPENPREPPPROP, 
                         synDONP_OPENADJPPROP, 
                         synDONP_OPENGERUND,   
                         synDONP_OPENNPPROP, 
                         synDONP_OPENTOSENT, 
                         synDONP_OTHEROPENPREPPPROP,       
                         synDONP_PREPNP, 
                         synDONP_PREPOPENADJPPROP,         
                         synDONP_PREPOPENGERUND,           
                         synDONP_PREPOPENNPPROP, 
                         synDONP_PREPOTHEROPENPREPPPROP,   
                         synDONP_PREPQSENT,                
                         synDONP_PROSENT,                  
                         synDONP_QSENT,                    
                         synDONP_THATSENT,                 
                         synEMPTY, synEMPTY_DONP, 
                         synEMPTY_CLOSEDTOSENT,            
                         synEMPTY_MEASUREPHRASE, 
                         synEMPTY_OPENGERUND,              
                         synEMPTY_OPENTOSENT, 
                         synEMPTY_PREPOPENGERUND,          
                         synEMPTY_PROSENT,                 
                         synEMPTY_PREPNP,                  
                         synEMPTY_PREP2NP,                 
                         synEMPTY_QSENT, synEMPTY_THATSENT, 
                         synFOREMPTY,                      
                         synFORTOSENT, synFRONTSOPROSENT,  
                         synIOEMPTY_DONP, synIOEMPTY_THATSENT,
                         synIOEMPTY_QSENT,                    
                         synIONP_DONP, synIONP_EMPTY,         
                         synIONP_MEASUREPHRASE, 
                         synIONP_OPENINFSENT,                 
                         synIONP_OPENNPPROP,                  
                         synIONP_OPENTOSENT, synIONP_PREPCLOSEDADJPPROP, 
                         synIONP_PREPNP, 
                         synIONP_PREPOPENGERUND,              
                         synIONP_PROSENT, synIONP_QSENT, 
                         synIONP_SOPROSENT, 
                         synIONP_THATSENT, synITTHATSENT, 
                         synLOCCLOSEDPREPPPROP, synLOCEMPTY,  
                         synLOCOPENPREPPPROP, synLOCPREPP,    
                         synMEASUREPHRASE, synNOTPROSENT,     
                         synNoVpArgs, synNP, 
                         synnoadjpargs,                       
                         synOPENADJPPROP, synOPENGERUND, 
                         synOPENGERUND_PREPNP,                
                         synOPENINFSENT,
                         synOPENTOINFSENTPROOBJ,         
                         synOPENNPPROP, synOPENTOSENT,   
                         synOPENVERBPPROP,
                         synOTHERCLOSEDPREPPPROP, 
                         synOTHERCLOSEDPREPPPROP_EMPTY,  
                         synOTHERCLOSEDPREPPPROP_PREPNP, 
                         synOTHEROPENPREPPPROP,
                         synPREPCLOSEDADJPPROP, synPREPCLOSEDGERUND, 
                         synPREPCLOSEDNPPROP, synPREPCLOSEDTOSENT, 
                         synPREPEMPTY,                   
                         synPREPMEASUREPHRASE,           
                         synPREPNP, 
                         synPREPNP_CLOSEDTOSENT,         
                         synPREPNP_EMPTY,                
                         synPREPNP_ITOPENTOSENT,         
                         synPREPNP_OPENTOSENT, synPREPNP_PREPNP, 
                         synPREPNP_PREPOPENGERUND,       
                         synPREPNP_QSENT, synPREPNP_THATSENT, 
                         synPREPOPENGERUND,              
                         synPREPOPENNPPROP, synPREPOTHERCLOSEDPREPPPROP,
                         synPREPOPENTOSENT,              
                         synPREPQSENT, synPREPTHATSENT, synPROSENT, 
                         synQSENT, synSOPROSENT, 
                         synSOPROSENT_EMPTY,             
                         synSOPROSENT_PREPNP,            
                         synTHATSENT, 
                         synTHATSENT_EMPTY,              
                         synTHATSENT_LOCOPENPREPPPROP,   
                         synTONP, synTONP_DONP,          
                         synTONP_THATSENT, synTONP_QSENT,
                         synVERBPPROP,                   
                         synDONP_OPENINFSENT,
                         vpid1, vpid2, vpid3, vpid4, vpid5,
                         vpid6, vpid7, vpid8, vpid9, vpid10,
                         vpid11, vpid12, vpid13, vpid14, vpid15,
                         vpid16, vpid17, vpid18, vpid19, vpid20,
                         vpid21, vpid22, vpid23, vpid24, vpid25,
                         vpid26, vpid27, vpid28, vpid29, vpid30,
                         vpid31, vpid32, vpid33, vpid34, vpid35,
                         vpid36, vpid37, vpid38, vpid39, vpid40,
                         vpid41, vpid42, vpid43, vpid44, vpid45,
                         vpid46, vpid47, vpid48, vpid49, vpid50,
                         vpid51, vpid52, vpid53, vpid54, vpid55,
                         vpid56, vpid57, vpid58, vpid59, vpid60,
                         vpid61, vpid62, vpid63, vpid64, vpid65,
                         vpid66, vpid67, vpid68, vpid69, vpid70,
                         vpid71, vpid72, vpid73, vpid74, vpid75,
                         vpid76, vpid77, vpid78, vpid79, vpid80,
                         vpid81, vpid82, vpid83, vpid84, vpid85,
                         vpid86, vpid87, vpid88, vpid89, vpid90,
                         vpid91, vpid92, vpid93, vpid94, vpid95,
                         vpid96, vpid97, vpid98, vpid99, vpid100,
                         vpid101, vpid102, vpid103, vpid104, vpid105,
                         vpid106, vpid107, vpid108, vpid109, vpid110,
                         vpid111, vpid112, vpid113, vpid114, vpid115,
                         vpid116, vpid117, vpid118, vpid119, vpid120,
                         vpid121, vpid122, vpid123, vpid124, vpid125,
                         vpid126, vpid127, vpid128, vpid129, vpid130,
                         vpid131, vpid132, vpid133, vpid134, vpid135,
                         vpid136, vpid137, vpid138, vpid139, vpid140,
                         vpid141, vpid142, vpid143, vpid144, vpid145,
                         vpid146, vpid147, vpid148, vpid149, vpid150,
                         vpid151, vpid152, vpid153, vpid154, vpid155,
                         vpid156, vpid157, vpid158, vpid159, vpid160,
                         vpid161, vpid162, vpid163, vpid164, vpid165,
                         vpid166, vpid167, vpid168, vpid169, vpid170,
                         vpid171, vpid172, vpid173, vpid174, vpid175,
                         vpid176, vpid177, vpid178, vpid179, vpid180,
                         vpid181, vpid182, vpid183, vpid184, vpid185,
                         vpid186, vpid187, vpid188, vpid189, vpid190,
                         vpid191, vpid192, vpid193, vpid194, vpid195,
                         vpid196, vpid197, vpid198, vpid199, vpid200,
                         );
  SynpatternSETtype   = SET OF synpatterntype;
  SynpatternEFFSETtype= {EFF}SET OF synpatterntype;
  thatdeltype         = (yesdel, nodel, omegadel);           
  Thetavptype         = (omegathetavp, vp000, vp100, vp010, vp120, 
                         vp012, vp123, vp132, thetavpaux );  
  Verbsubctype        = (mainverb, modalverb, doaux, beverb, 
                         haveverb, notaux);        
  verbsubcSETtype     = SET OF Verbsubctype;        
  Voicetype           = (Active, Passive, Omegavoice);
  VoiceSETtype        = SET OF Voicetype;
 >
\end{verbatim}
\end{document}
