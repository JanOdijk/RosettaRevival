
\documentstyle{Rosetta}
\begin{document}
   \RosTopic{Rosetta3.doc.Mrules.English}
   \RosTitle{Rosetta3 English Mrules: CLAUSEtoSENTENCE}
   \RosAuthor{Margreet Sanders, Lisette Appelo}
   \RosDocNr{370}
   \RosDate{December 11, 1989}
   \RosStatus{approved}
   \RosSupersedes{-}
   \RosDistribution{Project}
   \RosClearance{Project}
   \RosKeywords{English, documentation, Mrules, CLAUSEtoSENTENCE}
   \MakeRosTitle
%
%

\section{Introduction}
The English sentence grammar is divided in three parts, in the same way as all
other main category grammars. First, there is a subgrammar providing the 
PROP-structure, 
called {\bf VerbppropFormation}. Then, {\bf XPPROPtoCLAUSE} turns this prop
into a clause. Finally, {\bf ClauseToSentence} makes a full sentence of this
clause. Around the Sentence grammar there are two other, very small grammars.
Prior to VerbppropFormation, there is the {\bf VerbDerivation} grammar, and
following ClauseToSentence, there is the {\bf Utterance} grammar.

The current document describes the contents of 
the third Sentence subgrammar, CLAUSEtoSENTENCE (the first sentence subgrammar 
has been described in doc.\ 310, {\em Rosetta3 English Mrules: 
Verbppropformation\/}, and the second one in doc.\ 326, {\em Rosetta3 English 
Mrules: XPPROPtoCLAUSE\/}). The subgrammar consists of 
a number of rule classes and transformation classes. A rule class in its turn
consists of a number of rules and a transformation class of a number of 
transformations. The relative ordering of the rules and transformations in the
(sub)grammar is indicated by a {\em control expression}. A summary of this
control expression (i.e.\ a listing of the ordering of the rule classes, 
without explicit mentioning of the rules themselves) is also included here, 
and the initial (= head), import and export categories are given. Basically, 
the document describes the system as it was when the document was first 
written, viz.\ in June 1989. Modifications that took place later have been 
added in footnotes.

In the section on the rules and transformations, only the rule names are given, 
but not the exact rule formulation. What is attempted 
is to provide a detailed overview of the workings of the subgrammar, and 
how the different rule classes achieve this,
together with some comments on the problems still to be solved, the reasons 
behind certain choices, and perhaps possible alternatives. For every rule, an 
example is given. If it is uncertain whether the example is correct (either 
because it may not be an example of the phenomenon in question, or because it 
may not be correct English), it is preceded by a question mark. Note that all 
explanation of rules and transformations is given from a generative viewpoint
only, unless explicitly stated otherwise. Often, the information given in this 
document is based strongly on the comment already present in the documentation 
of the rules themselves. Discrepancies between what is stated here and what is 
said in the rule itself are usually caused by the fact that the rule file has 
not  been updated, although insights have changed. The semantics of the rules 
has been left unspecified in the current documentation, since it is not at all 
clear.

Whenever the current implementation differs widely from the strategy that was 
devised in the definition phase of Rosetta3 (as laid down for English in docs.\ 
150, {\em Subgrammars of English\/}, 153, {\em Rule and Transformation Classes 
of English\/}, and 155, {\em Rule and Transformation Classes common to all 
languages\/}, all written by Jan Odijk), this will be indicated explicitly in 
the current document. Conditions on crucial orderings of rule classes will be 
repeated here, even if they do not differ from the original strategy, to make 
the document as self-contained as possible.

Finally note that the rules described in this document have NOT been tested 
properly. English analysis is not possible yet (there is no Surface Parser), and 
English generation has only been tested in as far as the construction was the 
translation of a Dutch sentence to be tested.

In principle, the authors of the rule classes have documented their own 
classes: Lisette Appelo wrote the sections on TC\_TempEMPTYdeletion, 
TC\_DeixisRetro\-Adap\-tation and TC\_FinalTempTransf, and Andr\'{e} 
Schenk checked the description of the three idiom substitution rules.
The rest of the document was written by Margreet Sanders.


\newpage
\section{CLAUSEtoSENTENCE}
In this last sentence subgrammar, all CLAUSEs are turned into a SENTENCE. This 
is done in the moodrules, which form an essential part of this subgrammar.
Also, there are rules to introduce negations and to mark relatives, and there 
are substitution rules and 
punctuation rules. Transformations take care of shifting, incorporation of 
negations, several temporal phenomena, and some final brushing up of the 
finished sentence, including inversion and {\em do\/}-deletion.

The organisation of the CLAUSEtoSENTENCE subgrammar has changed somewhat when 
compared to the general lay-out as presented in doc.\ 150. Especially,
the rules for reciprocals and reflexives that precede substitution have been 
moved to the previous subgrammar, while the second set of these rules, that 
must follow substitution, has not been provided for yet (for sentences like 
{\em They saw a picture of themselves\/}). The AdvControlRules 
have also been moved to the previous subgrammar (see TC\_ConjSentControl in 
doc.\ 326). Furthermore, all substitution rules have been collapsed to one 
large class, for both wh- and non wh-arguments. This class follows the shift 
rules. No rules have been written yet to allow for the shift of wh-elements 
from a larger constituent ({\em Whom did you see a picture of?\/}). Just as 
for embedded reflexives which can be found only after 
substitution, it is impossible in the current framework to change things in an 
NP or move something out of it before substitution, and it is unattractive to 
double the rules (in a more complicated form too) to do many 
things that have been done on the NPVAR level once again after substitution for 
NPs.

A new rule class is RC\_PosNegVar, which introduces variables for negatives and 
(emphatic) positives. For the need for this rule class, see its documentation 
below. RC\_Punc is new too, although its functions (to provide 
punctation marks) were of course foreseen. RC\_RelMarking is new too: it marks 
the variable that is to be shifted. Many new transformation classes 
appeared to be necessary. TC\_NegAdaptation takes care of incorporation of a 
negation into a following word. This is needed because words like {\em nobody\/
} are no longer considered basic expressions, but a combination of a negation 
and a positive form ({\em somebody\/}). TC\_NegAuxAdapt glues the negation to 
the auxiliary verb, so that both can be moved together ({\em Can't you watch 
your step!\/}). Three different temporal transformation 
classes were needed to provide a correct surface structure: 
TC\_TempEMPTYdeletion, TC\_DeixisRetroAdaptation, and TC\_FinalTempTransf. For 
more information on the need for
these rule classes, see docs.\ 263, {\em Documentation of the rules 
for the translation of temporal expressions in Rosetta3, part I\/}, and 320, {
\em Superdeixis in Rosetta3\/}, both by Lisette Appelo. The transformation 
class TC\_PronounOK as yet contains only one rule, but more will have to be 
added (see its documentation below). Finally, there are two 
transformation classes to delete the auxiliary {\em do\/} in cases where it is 
not needed, and there is a transformation class which prunes the VP if 
possible (needed to make analysis, esp.\ the Surface Parser, simpler). The 
subgrammar ends with TC\_CommaIncorp, which also performs a task meant to 
relieve the burden placed on the Surface Parser.

The transformation class TC\_PPorSoutofNPADJP, which was foreseen in doc.\ 150, 
is still planned. It should follow TC\_NegAdaptation, but has not been written 
yet. 

\newpage
\section{Subgrammar Specification}
The subgrammar definition can be found in file {\bf CLAUSEtoSENTENCE.mrule}, 
which is {\em mrules5.mrule\/}.

\begin{verbatim}
%SUBGRAMMAR CLAUSEtoSENTENCE


    ( TC_PROstatus: mrules22 )
.   [ RC_RelMarking: mrules85 ]
.   ( TC_WhShift: mrules21,20 )
.   [ RC_PosNegVar: mrules83 ]
.   { RC_Substitution: mrules16-15 }
.   { TC_NegAdaptation: mrules8 }
    (* no TC_PPorSoutofNPADJP yet: mrules7 *)
.   ( TC_PronounOK: mrules6 )
    (* no TC_TopicCleft yet: mrules96 *)
.   ( RC_MoodDetermination: mrules14-12 )
      . ( TC_ConjThatDel: end mrules14 )
.   [ RC_ConjSent: mrules92 ]
.   [ TC_TempEMPTYDeletion: mrules77 ]
.   [ TC_DeixisRetroAdaptation: mrules75 ]
.   [ TC_FinalTempTransf: mrules34 ]
.   ( TC_DoBeDeletion: mrules82, begin )
    (* no TC\_ShallWillSwap yet: mrules82, end *)
.   [ TC_NegAuxAdapt: mrules11, begin ]
.   ( TC_AuxToComp: mrules11, end )
    (* no TC_CanNegIncorp yet: mrules11 *)
.   ( TC_DoDeletion: mrules82, middle )
.   ( TC_VPDeletion: mrules10 )
.   ( RC_Punc: begin mrules9 )
.   [ TC_CommaIncorp: end mrules9 ]
\end{verbatim}

\begin{description}
  \item[Head]     CLAUSE \ \ \ \ \    FROM (XPPROPtoCLAUSE)
  \item[Export] SENTENCE, NP, PREPP, CLOSEDVERBPPROP
  \item[Import] ADVP, PREPP, NP, PROSENT, POS, NEG, CONJ, PREP 
\end{description}


\newpage
\section{Rules and Transformations}

\subsection{TC\_PROstatus}
\begin{description}
\item[Kind] Obligatory Transformation Class
\item[Task] To mark the {\bf PROsubject} attribute of the clause as either {\em 
false\/} (for a closed sentence) or {\em true\/} (for an open sentence). 
In generation, there is a free choice between the two rules 
for infinite clauses (finite clauses will always be closed). In the Proposition 
Substitution rules of the next cycle, a choice between the closed and the open
version will be made, depending on the variable dictated by the verbpattern
of the main verb of the higher sentence. 

In Dutch, this transformation class was moved to the previous subgrammar, since 
there the closed/open information was needed earlier. In English, the class has 
remained where it was planned.

\vspace{1 cm}
\begin{description}
\item[Name] TNoProStatus 
\item[Task] Vacuous rule, leaving the {\bf PROsubject} attribute of a clause at
its default value, {\em false\/}, indicating that the clause will be closed. 
This rule is always applicable in generation.
\item[File] english:TC\_ProStatus.mrule (mrules22.mrule)
\item[Semantics] --
\item[Example] [x2 do prevent this]$_{PROsubj=false}$ 
 (We talked about how we could prevent this).
\item[Remarks]
\end{description}

\vspace{1 cm}
\begin{description}
\item[Name] TProStatus 
\item[Task] To mark an infinite clause as open, by setting the {\bf PROsubject} 
attribute to {\em true\/}.
\item[File] english:TC\_ProStatus.mrule (mrules22.mrule)
\item[Semantics] --
\item[Example] [x2 to prevent this]$_{PROsubj=false}$ $\rightarrow$ 
[x2 to prevent this]$_{PROsubj=true}$ (We talked about how to prevent this)
\item[Remarks]
\end{description}

\end{description}

\newpage
\subsection{RC\_RelMarking}
\begin{description}
\item[Kind] Optional Rule Class
\item[Task] To mark the NPVAR (and the VARPREPP it is in, if any), the ADVPVAR 
or the PREPPVAR that is to be spelled out as a relative with the 
appropriate mood ({\em relativexpmood\/}). In order to identify which variable 
will become a relative, the rules take a parameter, {\bf indexpar}, which 
carries the index of the variable to the target language. There are four 
different rules, for the different positions the variable may be in. Note that 
these rules violate the constraint that the record of a variable remains 
unchanged throughout the derivation. Because the PreAction has 
already taken place, the substitution rules for these NPVARs (i.e. the 
relative rules in the NPgrammar) must specify how the mood attribute of the 
VAR and the NP differ. Note that most relatives are CNVARs; NPVARs are needed 
for modification with a relative subsentence of e.g.\ INDEFPROs and PERSPROs 
(which do not have a CN-level), and for nonrestrictive `appositive' relative 
clauses.

The shifting of the relative will be done in the following transformation 
class, TC\_WhShift.

The rules will also be used to mark variables that will shift because of 
topicalization. Once that is implemented, they must use the mood, relative or 
topicalized, as a parameter\footnote{This had been done by the time this 
document was approved. The parameter is called {\bf xpmoodpar}.}. 
It is assumed that topicalized NPs are unlikely; 
hence, a topicalized NPVAR must be assigned a very low bonus in these rules. 
In analysis, any application of these rules must receive a lowered bonus for 
topicalized readings anyway, to make sure that first all non-topicalized 
meanings are translated. In analysis, subjects and constituents in 
leftdislocrel are never considered 
topicalized; in generation, this parameter value must always be accepted,
but it will 
not reset the mood attribute of a subject or a constituent in leftdislocrel, 
since it must remain where it is, and not be moved by the transformations of 
TC\_WhShift.

A filter, FNoTopic, must be added to stop any structure still harbouring 
a topicmood or relativemood constituent (for a problem with PREPPVARs, see the 
footnote to RSRelmarking) that cannot have come from an embedded sentence (for 
modals). This filter is NOT a speed filter, since it checks attribute values of 
variables that will not be checked anywhere else in the derivation\footnote{The 
filter had been added by the time the document was approved.}.

\vspace{1 cm}
\begin{description}
\item[Name] RSRelMarking
\item[Task] To mark the {\bf mood} attribute of an NPVAR, PREPPVAR or 
ADVPVAR of the correct 
index (determined by the rule parameter) as {\em relativexpmood\/} (or as {\em 
topicmood\/}, once the mood parameter has been added). The 
variable must be under S.
\item[File] english:RC\_Relmarking.mrule (mrules85.mrule)
\item[Semantics] 
\item[Example] x1$_{nowh}$ did come in from x2 $\rightarrow$ 
x1$_{relativexpmood}$ did come in from x2 (The man who came in from the cold)
\item[Remarks] There is a major problem for PREPPVARs: it is assumed in the 
grammar (see e.g.\ the startrules for PREPPs) that a
(VAR)PREPP having a CN(VAR) object always has the value {\em relativexpmood\/} 
for its {\bf mood} attribute, but has the value {\em nowh\/} when it has an 
NP(VAR) object, even when that was a relative. In analysis, the surface parser 
assigns relative mood to all relative PREPPs. The substitution 
rules then remove the (VAR)PREPP, and leave a bare PREPPVAR, which still has 
{\em relativexpmood\/}. Now, the current rule must decide whether to reset 
this mood to {\em nowh\/} (correct if there was an NP(VAR)) or leave it as it 
was (correct if there was a CN(VAR)), but it has no longer access to what the
object was. Hence, both paths are possible, and all relative PREPPs become 
ambiguous. This must still be solved.
\end{description}

\vspace{1 cm}
\begin{description}
\item[Name] RVPRelMarking
\item[Task] To mark the {\bf mood} attribute of an NPVAR, PREPPVAR or ADVPVAR
 of the correct 
index (determined by the rule parameter) as {\em relativexpmood\/} (or as {\em 
topicmood\/}, once the mood parameter has been added). The 
variable must be in the VP.
\item[File] english:RC\_Relmarking.mrule (mrules85.mrule)
\item[Semantics]
\item[Example] x1 did have x2$_{nowh}$ $\rightarrow$ x1 did have 
x2$_{relativexpmood}$ (I gave him a book which he already had)
\item[Remarks] For a problem with relative PREPPs, see the remark made in the 
previous rule.
\end{description}

\vspace{1 cm}
\begin{description}
\item[Name] RPPSRelMarking
\item[Task] To mark the {\bf mood} attributes of an NPVAR (or a PREPPVAR or 
ADVPVAR, although they are unliky as object of a PREPP!) of the correct 
index (determined by the rule parameter) and of its governing (VAR)PREPP as 
{\em relativexpmood\/} (or as {\em 
topicmood\/}, once the mood parameter has been added).
 The (VAR)PREPP is directly under S\footnote{By the  
time this document had been approved, the rule had been split up in two rules, 
one for PREPPs, and one for VARPREPPs. The latter rule is called 
RVarPPSRelMarking.}.
\item[File] english:RC\_Relmarking.mrule (mrules85.mrule)
\item[Semantics]
\item[Example] Before x3$_{nowh}$ x1 must have arrived $\rightarrow$ Before 
x3$_{relativexpmood}$ x1 must have arrived (The time before which you must 
have arrived)
\item[Remarks] PREPPs directly under S are uncommon (only PREPPVARs occur 
there, in case of locative and temporal expressions, and these are covered in 
RSRelMarking), but perhaps may occur in 
cases of topicalization. It is unclear whether topicalized relatives exist in 
English.
\end{description}

\vspace{1 cm}
\begin{description}
\item[Name] RPPVPRelMarking
\item[Task] To mark the {\bf mood} attributes of an NPVAR (or a PREPPVAR or 
ADVPVAR, although they are unliky as object of a PREPP!) of the correct 
index (determined by the rule parameter) and of its governing (VAR)PREPP as 
{\em relativexpmood\/} (or as {\em 
topicmood\/}, once the mood parameter has been added).
 The (VAR)PREPP is in the VERBP\footnote{By the  
time this document had been approved, the rule had been split up in two rules, 
one for PREPPs, and one for VARPREPPs. The latter rule is called 
RVarPPVPRelMarking.}.
\item[File] english:RC\_Relmarking.mrule (mrules85.mrule)
\item[Semantics]
\item[Example] x1 must give x2 to x3$_{nowh}$ $\rightarrow$ x1 must give x2 to 
x3$_{relativexpmood}$ (The man to whom you must give this)
\item[Remarks] 
\end{description}

\end{description}

\newpage
\subsection{TC\_WhShift}
\begin{description}
\item[Kind] Obligatory Transformation Class
\item[Task] To shift any wh-element or relative (the name of the transformation 
class has gone out of date) to a special clause-initial {\em shiftrel\/} 
position. This includes shifting a shiftrel from an embedded (declarative) 
sentence. Once RC\_Relmarking also works for topicalization (see what was said 
on that rule class in the previous subsection), constituents that 
have the value {\em topicmood\/} must also be shifted\footnote{This had been 
implemented by the time this document was approved.}. They will be moved from 
shiftrel to leftdislocrel in another transformation class (see TC\_TopicCleft).

In doc.\ 150, it was assumed that the shiftrules would follow the set of 
substitution rules dealing with non-wh elements. This would make it possible to 
deal with e.g.\ wh-elements within larger NPs or PREPPs, as in {\em Whom did you 
hear stories about?\/}, where first there is substitution of the PREPP {\em 
stories about x1\/}, then x1 is shifted, and finally there is wh-substitution.
However, on closer inspection there appeared to be many problems with this 
approach, and it was decided to skip all phenomena dealing with extraction of 
elements from a larger NP or PREPP constituent for the time being. Thus, there 
is only one substitution rule class and one shift rule class, the former
following the latter.

In generation, some extra rules may be needed to account for different surface 
positions than those provided for by the ordinary rules. This would be needed 
to comply with the substititution order required because of analysis. 
An example may be the sentence {\em There 
is someone that he did not see\/} as translation of the Dutch {\em Hij zag 
iemand niet\/}, so with the negation following the object NP i.s.o.\ the 
`ordinary' {\em He did not see someone\/}. A file 
has been reserved for this purpose (TC\_ScopeShift.mrule (mrules19.mrule)), but
the rules have not been written yet. (Note that this sentence is not so 
ordinary anyway, 
since the polarity requirements only accept {\em anybody\/} in a negative 
context; therefore, extra measures would be needed to translate the Dutch 
sentence anyway).

Finally, no rule has been written yet for the shift of an object from a THANP: {
\em who(m) are you bigger than?\/}. This rule will be added. The rule for 
shifting a {\em vpadvrel\/} element (which is not a VAR!) is not written yet.

\vspace{1 cm}
\begin{description}
\item[Name] TNoWhshift
\item[Task] to let clauses in which there is nothing to shift (i.e.\ there is a 
modal or there is no wh- or relative element and no CNVAR) pass this 
transformation class
\item[File] english:TC\_WhShift1.mrule (mrules21.mrule)
\item[Semantics] --
\item[Example] x1 did leave, x1 do not know who has left, etc.
\item[Remarks] 
Although wh-NPs are not excluded from the present `default' rule, they will 
never cause a sensible analysis: if they are part of the main sentence, they 
should have been desubstituted, and if they are part of an embedded sentence, 
they must be put back there (see TComplWhShift and TExtrapWhShift).
Therefore, the rule must be changed to exclude wh-NPs here.
\end{description}

\vspace{1 cm}
\begin{description}
\item[Name] TWhShift
\item[Task] To shift wh- and relative elements from their original position to 
the shiftrel position\footnote{When this document was approved, the 
transformation also worked for topicalized elements.}. 
The different subrules work for subjrel and (ind)objrel 
NP/CNVARs, 
predrel ADJP/NP(VAR)/PREPPs, loc/dirargrel PREPP/
ADVPs (and complrel PREPPs), locadvrel PREPP/ADVPVARs, prepobj/
byobjrel VARPREPPs, and 
tempadvrel PREPP/ADVPVARs
\item[File] english:TC\_WhShift2.mrule (mrules20.mrule)
\item[Semantics] --
\item[Example] \mbox{}\\
x1 have bought x2$_{wh}$ $\rightarrow$ x2$_{wh}$ x1 have bought 
(What have you bought?)\\
x1 have voted [against x2]$_{relativexpmood}$ $\rightarrow$ [against x2]$_{
relativexpmood}$ x1 have voted (The proposal against which we had all voted)\\
x1 did go x2$_{wh}$ $\rightarrow$ x2$_{wh}$ x1 did go (Where did she go?)\\
etc.
\item[Remarks] The shifting of relatives is only allowed for VARs (except 
ADVPVARs and identificational NPVARs) and for PREPPs\footnote{The shifting of 
topicalized elements parallels that for relative elements, but is allowed 
for ADVPVARs too.}. For 
prepobjrels, see also the remark made under TStrandedWhShift below.
\end{description}

\vspace{1 cm}
\begin{description}
\item[Name] TExtrapWhShift
\item[Task] To shift wh- and relative elements from a shiftrel in an extraposed 
sentence to shiftrel in the higher clause\footnote{When this document was 
approved, the transformation also worked for topicalized elements.}.
\item[File] english:TC\_WhShift1.mrule (mrules21.mrule)
\item[Semantics]
\item[Example] x1 did [see] [whom that he hit] $\rightarrow$ whom x1 did 
[see] [that he hit] (Whom did you see that he hit?)
\item[Remarks] The rule does not work for wh-NPs. This causes a problem 
for predrel/NPs, as in {\em Who do you think that
he is?\/}, because these will not be substituted at all. However, it is 
impossible to relax the present rule to work for NPs as well, because 
then `ordinary' (not predicative) embedded sentences like in {\em What 
did you think Pete saw\/} would receive two analyses: 
there would be a free choice in the order of application of substitution 
and shift.\\
The rule does not pose any restrictions on what elements may be present or
absent between the VERBP and the extraposed SENTENCE, and in the SENTENCE
itself. Hence, a sentence like {\em Who did you say yesterday that was ill\/}
is accepted here in analysis. The rule supposes that TConjThatDel (see far
below) will make sure that this path is never presented (that rule does not
accept a sentence with a conjunction {\em that\/} and a shifted subject). 
The restrictions on what kind of sentences may occur in extraposrel (sentences 
starting with {\em that\/} and, in analysis, opentoinfs) are laid down in 
TExtrapos1 in the SentOK rules (see previous subgrammar), so if the Surface 
Parser allows other options, they will only be stopped there, but not in the 
present rule.
\end{description}

\vspace{1 cm}
\begin{description}
\item[Name] TComplWhShift
\item[Task] To shift wh- and relative elements from a shiftrel or a subjrel 
(the latter in declarative infinites, cf.\ TNoWhInfShift below) in an embedded
complement sentence to shiftrel in the higher clause\footnote{When 
this document was 
approved, the transformation also worked for topicalized elements in shiftrel.}.
\item[File] english:TC\_WhShift1.mrule (mrules21.mrule)
\item[Semantics]
\item[Example] x1 did [see [whom him hit]] x4=yesterday $\rightarrow$ whom x1 
did [see [him hit]] x4=yesterday (Whom did you see him hit yesterday?)\\
x1 did [see [whom escape]] x4=yesterday $\rightarrow$ whom x1 did [see 
[escape]] x4=yesterday
\item[Remarks] The rule does not work for wh-NPs. See the rule above for 
comment.\\
The rule excludes shifting of subjects in case the thetavp of the clause is 
vp010. For raising verbs, the subject will be moved in the ObjectOK rules (see 
TSubjToSubjRaising in the previous subgrammar).
\end{description}

\vspace{1 cm}
\begin{description}
\item[Name] TNoWhInfShift
\item[Task] To let infinitives which have a wh- or relative subject pass 
this transformation 
class. The wh-subject must not be shifted, because it will be moved in the next 
cycle, either in the ObjectOKrules (see previous subgrammar) when it is the 
complement of a subject raising verb, or in TComplWhShift (see above), after 
having received case in TExceptCaseAssign (see again previous subgrammar). Note 
that TExceptCaseAssign also has a subrule to assign case to a shifted subject, 
in case the strategy for dealing with these subjects changes.
\item[File] english:TC\_WhShift1.mrule (mrules21.mrule)
\item[Semantics]
\item[Example] x1$_{wh}$ to have done x2 ((seem) who to have done it; Who seems 
to have done it)
\item[Remarks] The rule need not be adapted to work for topicalization, 
since subjects never should be topicalized.
\end{description}

\vspace{1 cm}
\begin{description}
\item[Name] TStrandedWhShift
\item[Task] To shift the wh- or relative NP/CNVAR object from a (VAR)PREPP in 
the VERBP
to the initial shiftrel, leaving the PREP stranded in the 
VERBP\footnote{When this document was 
approved, the transformation also worked for topicalized NP/CNVARs.}.
The rule works 
for (VAR)PREPPs in prepobjrel, locargrel and dirargrel.
\item[File] english:TC\_WhShift1.mrule (mrules21.mrule)
\item[Semantics]
\item[Example] x1 do move to x2$_{wh}$ $\rightarrow$ x2$_{wh}$ x1 do move to 
(Which house did you move to?)
\item[Remarks] This rule is an alternative to the subrule in TWhShift that 
moves the entire PREPP to shiftrel. No decision has been taken yet which path 
should be preferred in generation (since no bonus system exists for generation, 
one of the two rules should only work in analysis if only one translation must 
be produced).
\end{description}

\end{description}

\newpage
\subsection{RC\_PosNegVar}
\begin{description}
\item[Kind] Optional Rule Class
\item[Task] To introduce variables for emphatic positives and negations, so 
that these may influence the substitution order conditions (see 
RC\_Substitution below), to account for scope phenomena of esp.\ negatives.
The variables inserted are not `real' variables: they are not basic expressions, 
and will not be part of the derivation tree (note that they are not in the 
VARCATset defined in the English LSDOMAINT). Hence, they also do not have the
index-attribute that ordinary variables have and thus, they cannot use the 
LEVEL parameter. For more information on the 
treatment of variables and scope, see doc.\ 368, {\em Scope in Rosetta3\/}, 
by Jan Odijk.

There are rules for every possible position of the POSVARs and NEGVARs, 
with an extra version for negations, to mark whether they will melt with the 
following word (the function {\em FirstIsAPossNietNP\/} checks whether the 
following 
word can incorporate a negation; in case of a (VAR)PREPP, the search is for 
the word following the PREP). Melting NEGVARs are placed in a special {\em 
meltnegrel\/}. The actual melting will only take place after substitution, in a 
separate transformation class, TC\_NegAdaptation.
The rules have been formulated in such a way that only one order of 
introduction of the VARs is possible. There are a number of conditions on the 
co-occurrence of posrels and (melt)negrels, but because Spanish cannot handle 
two of them in the same sentence, the rule class has been made optional 
instead of 
iterative. Note that all var-introduction rules are mapped onto each other, 
irrespective of the position of the variable in the source language (in VP or 
directly under S). The correct scope of the negation must be taken care of by 
the substorder condition.

In doc.\ 150, this rule class was not mentioned, since it was thought that 
(positives and) negations would be introduced in the sentence without there 
first being a variable. However, this does not allow a scope sensitive 
treatment of negations. Words that have an incorporated 
negation, like {\em nobody\/}, are considered a melting of two basic 
expressions, the positive form and a negation, so that the translation of {\em 
niemand\/} can be {\em not anybody\/}, and need not always be {\em nobody\/}.

When an English Surface Parser is written, a PreFilter must be added to speed 
up analysis: no POSVAR or NEGVAR may remain. No decision has been taken yet 
which rules should receive lower bonus, the rules for incorporated negations 
or the 
rules that spell out a full {\em not\/}. Perhaps, this should depend on the 
position of the negation: in subject position, a melted form seems best ({\em 
Nobody listened, $^{*}$Not anybody listened\/}), elsewhere a full {\em not\/} 
is better ({\em I did not see anybody, I saw nobody\/}).

\vspace{1 cm}
\begin{description}
\item[Name] RSentNegVar
\item[Task] To introduce a NEGVAR directly under S, for those cases where the 
NEG will not melt with the following word.
\item[File] english:RC\_PosNegVar.mrule (mrules83.mrule)
\item[Semantics]
\item[Example] \mbox{}\\
x1 probably did [hear x2] $\rightarrow$ x1 probably did NEGVAR [hear x2] \\ 
{[Go there]} $\rightarrow$ NEGVAR [go there] (Not to go there would be 
foolish)\\
x1 did x4=ever solve x2 $\rightarrow$ x1 did NEGVAR x4=ever solve x2 (They did 
not ever solve it)
\item[Remarks]
\end{description}

\vspace{1 cm}
\begin{description}
\item[Name] RSentMeltNegVar
\item[Task] To introduce a NEGVAR directly under S, for those cases where the 
NEG will melt with the following word, which is also in S.
\item[File] english:RC\_PosNegVar.mrule (mrules83.mrule)
\item[Semantics] 
\item[Example] 
x1 do lie x4=ever $\rightarrow$ x1 do lie NEGVAR x4=ever (Peter never lies)
\item[Remarks] The rule now explicitly refers to the predicate: the NEGVAR is 
introduced only preceding the predicate. This condition must be removed to 
allow for melting of {\em not\/} with a temporal adverb (tempadvrel follows 
predrel)\footnote{This had been implemented by the time the document was 
approved.}. 
It has not been decided yet at what point in the grammar the movement 
rules for temporal adverbials will be located.
\end{description}

\vspace{1 cm}
\begin{description}
\item[Name] RVPNegVar
\item[Task] To introduce a NEGVAR in the VERBP, in case it will melt with the 
following word there.
\item[File] english:RC\_PosNegVar.mrule (mrules83.mrule)
\item[Semantics]
\item[Example] x1 did [look at x2=somebody] $\rightarrow$ x1 did [look NEGVAR 
at x2=somebody] (He looked at nobody in particular)
\item[Remarks]
\end{description}

\vspace{1 cm}
\begin{description}
\item[Name] RSentPosVar
\item[Task] To introduce a POSVAR under S.
\item[File] english:RC\_PosNegVar.mrule (mrules83.mrule)
\item[Semantics]
\item[Example] x1 did [give x3 x2] $\rightarrow$ x1 did POSVAR [give x3 x2] 
(But I did give all the children some sweets at the party)
\item[Remarks] There is no basic expression connected to the English POSVAR 
(cf.\ Dutch, which has the word {\em wel\/}). The effect of introducing a 
POSVAR in English is prevention of {\em do\/}-deletion, as in the example 
above. It is assumed in the rule that POSVARs cannot occur if the VP has a 
negation in it: $^{*}${\em I did see nothing\/}.
\end{description}

\end{description}

\newpage
\subsection{RC\_Substitution}
\begin{description}
\item[Kind] Iterative Rule Class
\item[Task] To substitute expressions for their variable, in the order dictated 
by analysis and consistent with the Substitution Order Conditions (the latter 
must assure that scope is accounted for correctly).

The rules use the system parameter LEVEL, to check whether the index of the 
variable is consistent with the (level of the) variable that is to be 
substituted for according to the rule parameter.

In English, the generative Substition Order Conditions have not been 
implemented yet. They always return the value {\em true\/}. That means that 
scope is not accounted for yet in generation. For more information on the 
function of these conditions, see doc.\ 368, {\em Scope in Rosetta3\/}, by Jan 
Odijk\footnote{By the time this document was approved, the functions for 
generative substitution order had been implemented. For more information, see 
also the document on LSMRUQUO by Jan Odijk (to appear).}.

There may be a need to arrive in this rule class with different surface 
structures than those provided for by the ordinary rules, esp.\
to comply with the substititution order required because of analysis. 
See what was said above in the comment on TC\_WhShift about a rule class 
TC\_ScopeShift. When the functions checking generative substitution order 
work, no translation will be produced in case of a scope problem!

No rules have been written yet for PREPPs that have a PREP which is to be 
deleted (e.g.\ {\em He left early {\em on} that morning\/}). Also, rules for 
vpadvrel/PREPPs still must be written.

The superdeixis of the substituees (if they have one) is adjusted in the 
substitution rules:
it is checked against the (super)deixis of the clause and set to {\em omega\/} 
in generation, and copied from the CLAUSE 
(super)deixis in analysis. Rules that were written later use the functions {\em 
DeixisMax\/} and {\em AssDeixisMax\/} for this. 
The rules use the function {\em AssignCase\/} to 
pass the case of the VAR on to the elements in the substituee, if necessary.

For generic NPs, more constraints on the input model hold than for nongeneric 
ones.
However, the appropriate rules have not been formulated yet (since no generic 
NPs are made yet anyway). The current NP rules only work for non-generic ones.
Also note that the Surface Parser always assigns the value {\em omegageneric\/} 
to NPs. Therefore, in generation the current rules reset the genericity 
value of the NP to {\em omegageneric\/} (i.e.\ in analysis, the value {\em 
nogeneric\/} is assigned).

In the NP rules, several elements are excluded explicitly: reflexives, 
reciprocals, sentence-like NPs and NPs containing an incorporated negation (the 
latter is checked by means of the function {\em NegPhrase\/}).

\vspace{1 cm}
\begin{description}
\item[Name] RTempAdvSubstitution1
\item[Task] To substitute a temporal ADVP or a frequential NP for their VAR
\item[File] english:RC\_TempVar.mrule (mrules36.mrule)
\item[Semantics]
\item[Example] x1 did dance x4 + yesterday $\rightarrow$ x1 did dance yesterday
\item[Remarks] For ADVPs, the Aktionsarts of the clause are reduced on 
desubstitution, so that only one element remains in the set.\\
An extra rule will have to be written to account for substitution of temporal 
ADVPs and frequential NPs that have been shifted (wh-mood and 
topicmood)\footnote{This had been done by the time the document was approved. 
The rule is called RTemp\-Adv\-SubstShift1.}.
\end{description}

\vspace{1 cm}
\begin{description}
\item[Name] RTempAdvSubstitution2
\item[Task] To substitute a temporal PREPP or VARPREPP for its VAR
\item[File] english:RC\_Tempvar.mrule (mrules36.mrule)
\item[Semantics]
\item[Example] x1 danced x4 + For three hours $\rightarrow$ x1 danced for three 
hours
\item[Remarks]
An extra rule will have to be written to account for substitution of temporal 
(VAR)PREPPs that have been shifted (wh-mood, relatives and 
topicmood)\footnote{This had been done by the time the document was approved. 
The rule is called RTemp\-Adv\-SubstShift2.}.
\end{description}

\vspace{1 cm}
\begin{description}
\item[Name] RObjNPSubst
\item[Task] To substitute a nongeneric NP for its (ind)objrel or predrel VAR 
in the VERBP
\item[File] english:RC\_Substitution1.mrule (mrules16.mrule)
\item[Semantics]
\item[Example] x1 see x2$_{accusative}$ + he $\rightarrow$ x1 see him\\
This is x1 + John $\rightarrow$ This is John
\item[Remarks] 
\end{description}

\vspace{1 cm}
\begin{description}
\item[Name] RSubjNPSubst
\item[Task] To substitute a nongeneric NP for its subjrel or shiftrel VAR
\item[File] english:RC\_Substitution1.mrule (mrules16.mrule)
\item[Semantics]
\item[Example] x1 do see him + She $\rightarrow$ She do see him
\item[Remarks] The rule excludes substitution when the main verb (or a modal 
preceding it, except {\em will\/}) has vp010, because then substitution should 
take place in a later cycle (important for analysis).\\
The rule will be split up into two rules: one for substitution 
of subjects, and one for substitution of shifted elements (wh or 
topicalized)\footnote{This had been done by the time this document was 
approved. The new rule is called {\em RShiftNPSubst\/}. For topicalized 
elements, the {\bf mood} attribute is reset to {\em nowh\/}, because the 
surface parser cannot recognize topicalized elements.}.
\end{description}

\vspace{1 cm}
\begin{description}
\item[Name] RPrepNPSubst
\item[Task] To substitute a nongeneric NP for the object NPVAR of a PREPP in 
the VERBP. The PREPP may be in locargrel, dirargrel, complrel or 
instr\-adv\-rel.
\item[File] english:RC\_Substitution1.mrule (mrules16.mrule)
\item[Semantics]
\item[Example] x1 do jump [off x2] + the fence $\rightarrow$ x1 do jump [off 
the fence]
\item[Remarks] 
\end{description}

\vspace{1 cm}
\begin{description}
\item[Name] RVarPrepNPSubst
\item[Task] To substitute a nongeneric NP for the object NPVAR of a VARPREPP 
in the VERBP. The VARPREPP may be in prepobjrel or byobjrel.
\item[File] english:RC\_Substitution1.mrule (mrules16.mrule)
\item[Semantics]
\item[Example] x1 count [on x2] + Mary $\rightarrow$ x1 count on Mary\\
x2 be quickly turned on by x1$_{accusative}$ + she $\rightarrow$ x2 be quickly 
turned on by her
\item[Remarks] 
\end{description}

\vspace{1 cm}
\begin{description}
\item[Name] RShiftPrepNPSubst
\item[Task] To substitute a nongeneric NP for the object VAR of a shifted PREPP 
in S.
\item[File] english:RC\_Substitution1.mrule (mrules16.mrule)
\item[Semantics]
\item[Example] In x1 did you put it + which drawer $\rightarrow$ In which 
drawer did you put it
\item[Remarks]
\end{description}

\vspace{1 cm}
\begin{description}
\item[Name] RShiftVarPrepNPSubst
\item[Task] To substitute a nongeneric NP for the object VAR of a shifted 
VARPREPP in S.
\item[File] english:RC\_Substitution1.mrule (mrules16.mrule)
\item[Semantics]
\item[Example] By x1$_{accusative}$ you were hit + who $\rightarrow$ By whom 
you were hit (By whom were you hit?)
\item[Remarks]
\end{description}

\vspace{1 cm}
\begin{description}
\item[Name] RLocAdvSubst
\item[Task] To substitute a (VAR)PREPP or ADVP for its locadvrel VAR
\item[File] english:RC\_Substitution1.mrule (mrules16.mrule)
\item[Semantics]
\item[Example] x1 train x4 + indoors $\rightarrow$ x1 train indoors
\item[Remarks] An extra rule will have to be written for locatives in 
shiftrel\footnote{By the time this document was approved, the rule had 
been written: {\em RShiftLocAdvSubst\/}. It also works for topicalized 
elements, and resets their mood to {\em nowh\/}, because the surface parser 
cannot recognize topicalized elements.}.
\end{description}

\vspace{1 cm}
\begin{description}
\item[Name] RSentAdvSubst
\item[Task] To substitute a sentadv or causadv ADVP for its VAR in 
leftdislocrel, and to introduce a separating comma between the ADVP and the 
rest of the clause.
\item[File] english:RC\_Substitution1.mrule (mrules16.mrule)
\item[Semantics]
\item[Example] x4 he do smoke + probably $\rightarrow$ Probably, he do smoke
\item[Remarks] No hoprules or alternative positions for sentadvs have been 
defined yet, and the attribute {\em position\/} of the Adv is not used yet.
\end{description}

\vspace{1 cm}
\begin{description}
\item[Name] RSentPreppSubst
\item[Task] To substitute a causative PREPP for its VAR in leftdislocrel,
and to introduce a separating comma between the PREPP and the 
rest of the clause.
\item[File] english:RC\_Substitution1.mrule (mrules16.mrule)
\item[Semantics]
\item[Example] x4 he do smoke + for that reason $\rightarrow$ For that reason, 
he do smoke
\item[Remarks] No hoprules or alternative positions for sentprepps have been 
defined yet.
\end{description}

\vspace{1 cm}
\begin{description}
\item[Name] RPROSentSubst
\item[Task] To substitute an abstract PROSENT for its VAR, resulting in total 
deletion of the category.
\item[File] english:RC\_Substitution1.mrule (mrules16.mrule)
\item[Semantics]
\item[Example] \mbox{}\\
x1 do wonder x2 + PROSENT $\rightarrow$ x1 do wonder\\
x1 do x4=not know x2 + PROSENT $\rightarrow$ x1 do x4=not know 
\item[Remarks] The translation of PROSENTs (and their negation) in Dutch is 
still not quite clear. Currently, all PROSENTs are translated into {\em het 
(niet) \/}.
\end{description}

\vspace{1 cm}
\begin{description}
\item[Name] RSoPROSentSubst
\item[Task] To substitute a soPROSENT for its VAR (which is a complement in the 
VERBP)
\item[File] english:RC\_Substitution1.mrule (mrules16.mrule)
\item[Semantics]
\item[Example] x1 do x4=not think x2 + so $\rightarrow$ x1 do x4=not think so\\
x1 did tell x3 x2 $\rightarrow$ x1 did tell x3 so
\item[Remarks] \mbox{}
\begin{itemize}
\item For the moment, only the complement {\em so\/} in the VERBP can be dealt 
with. Initial {\em So\/} is not accounted for yet ({\em So I understand\/}).
\item The translation of English {\em so\/} seems to depend on the kind of verb 
it goes with: verbs of believe best translate in {\em van wel\/} ({\em I hope 
so\/} $\rightarrow$ {\em Ik hoop van wel\/}), or perhaps simply in {\em het\/} 
({\em Ik hoop het\/}), and other verbs, esp.\ verbs of 
saying, best take {\em Dat\/} + inversion ({\em I told you so\/} $\rightarrow$ 
{\em Dat zei ik je\/}, or perhaps again simply {\em het\/}: {\em Ik zei het je
\/}). However, the negation of {\em so\/} is not just {
\em niet\/} + {\em van wel\/} or {\em het\/} + {\em niet\/}: 
$^{*}${\em Ik denk niet van wel\/}, $^{*}${\em Ik zei het je niet\/}. 
Therefore,
Dutch has postponed implementing any rules for the translation of soPROSENTs, 
until it is clear how they should be dealt with.
\end{itemize}
\end{description}

\vspace{1 cm}
\begin{description}
\item[Name] RNotProSentSubst
\item[Task] To substitute a notPROSENT for its VAR, resulting in the word {\em 
not\/}.
\item[File] english:RC\_Substitution1.mrule (mrules16.mrule)
\item[Semantics]
\item[Example] x1 do think x2 + not $\rightarrow$ x1 do think not
\item[Remarks] This use of the notPROSENT should only be allowed for verbs of 
belief (indicated by having a NOTProSent in their verbpatterns). The mapping to 
Dutch seems simple: always to {\em van niet\/}: {\em I think not\/} 
$\rightarrow$ {\em Ik denk van niet\/}. It is not checked whether Spanish has a 
similar solution. The Dutch rule with {\em van niet\/} has NOT been implemented 
yet, however.
\end{description}

\vspace{1 cm}
\begin{description}
\item[Name] RPosSubst
\item[Task] To substitute an abstract element POS for its VAR. In the 
DoDeletion rules (see below), either POS or {\em do\/} will be deleted.
\item[File] english:RC\_Substitution1.mrule (mrules16.mrule)
\item[Semantics]
\item[Example] x1 do x4 love you + POS $\rightarrow$ x1 do POS love you\\
x1 have x4 seen that movie + POS $\rightarrow$ x1 have POS seen that movie
\item[Remarks] Remember that POSVAR is not a real VAR (see RC\_PosNegVar), and 
that hence this rule is not a real substitution rule but an introduction rule.
\end{description}

\vspace{1 cm}
\begin{description}
\item[Name] RSentNegSubst
\item[Task] To substitute the negation {\em not\/} for its VAR under S.
\item[File] english:RC\_Substitution1.mrule (mrules16.mrule)
\item[Semantics]
\item[Example] x1 did x4 come + not $\rightarrow$ x1 did not come
\item[Remarks] In the rule, the Aktionsarts of the CLAUSE are set to {\em 
[stative]\/} in generation. In analysis, they are set to the set of all 
possible values. Should this cause ambiguities in the AktionsArt rules, then 
perhaps the function AssignAktArts can be used to determine a more restrictive 
set.

Remember that NEGVAR is not a real VAR (see RC\_PosNegVar), and 
that hence this rule is not a real substitution rule but an introduction rule.
\end{description}

\vspace{1 cm}
\begin{description}
\item[Name] RSentMeltNegSubst
\item[Task] To substitute a negation {\em not\/} that will melt with the 
following word for its VAR under S. The actual melting will be done in 
TC\_NegAdaptation (see below).
\item[File] english:RC\_Substitution1.mrule (mrules16.mrule)
\item[Semantics]
\item[Example] x1 do x4 ever [go away] + not $\rightarrow$ x1 do not ever [go 
away]
\item[Remarks] In the rule, the Aktionsarts of the CLAUSE are set to {\em 
[stative]\/} in generation. In analysis, they are set to the 
set of all possible values. See RSentNegSubst for a remark on that.

Remember that NEGVAR is not a real VAR (see RC\_PosNegVar), and 
that hence this rule is not a real substitution rule but an introduction rule.
\end{description}

\vspace{1 cm}
\begin{description}
\item[Name] RVPMeltNegSubst
\item[Task] To substitute a negation {\em not\/} that will melt with the 
following word for its VAR in the VERBP. The actual melting will be done in 
TC\_NegAdaptation (see below).
\item[File] english:RC\_Substitution1.mrule (mrules16.mrule)
\item[Semantics]
\item[Example] x1 did [look x4 at somebody] + not $\rightarrow$ x1 did [look 
not at somebody]
\item[Remarks] In the rule, the Aktionsarts of the CLAUSE are set to {\em 
[stative]\/} in generation. In analysis, they are set to the 
set of all possible values. See RSentNegSubst for a remark on that.

Remember that NEGVAR is not a real VAR (see RC\_PosNegVar), and 
that hence this rule is not a real substitution rule but an introduction rule.
\end{description}

\vspace{1 cm}
\begin{description}
\item[Name] RIdSubstitution1
\item[Task] To substitute an NP for its determiner VAR in a higher objrel NP, 
resulting in a genitive NP determiner or a POSSADJ (of PERSPRO or WHPRO)
\item[File] english:RC\_Substitution2.mrule (mrules15.mrule)
\item[Semantics]
\item[Example] x1 break x2 heart + John/my father/she/who $\rightarrow$ x1 
break John's/my father's/her/whose heart
\item[Remarks]
\end{description}

\vspace{1 cm}
\begin{description}
\item[Name] RIdSubstitution2
\item[Task] To substitute an NP for its determiner VAR in a higher shiftrel or 
subjrel NP, resulting in a genitive NP determiner.
\item[File] english:RC\_Substitution2.mrule (mrules15.mrule)
\item[Semantics]
\item[Example] x2 heart be broken + John $\rightarrow$ John's heart be broken
\item[Remarks] The rule will be extended to cover POSSADJ
determiners as well\footnote{This had been done by the time this document was 
approved.}.
\end{description}

\vspace{1 cm}
\begin{description}
\item[Name] RIdSubstitution3
\item[Task] To substitute an NP for its determiner VAR in a higher NP which is 
the obj of a locargrel or dirargrel PREPP in the VERBP, resulting in a genitive 
NP determiner.
\item[File] english:RC\_Substitution2.mrule (mrules15.mrule)
\item[Semantics]
\item[Example] x1 lay x2 at x3 door + Mary $\rightarrow$ x1 lay x2 at Mary's 
door
\item[Remarks] The rule will be extended to cover POSSADJ
determiners as well\footnote{This had been done by the time this document was 
approved.}.
\end{description}

\vspace{1 cm}
\begin{description}
\item[Name] RxppObjNPsubst
\item[Task] To substitute a nongeneric NP for its (ind)objrel VAR under a 
predrel in the VERBP
\item[File] english:RC\_Substitution2.mrule (mrules15.mrule)
\item[Semantics]
\item[Example] ? x1 do [be [against x2] + he $\rightarrow$ x1 do [be [against 
him]
\item[Remarks]
\end{description}

\vspace{1 cm}
\begin{description}
\item[Name] RxppVarPrepNPsubst
\item[Task] To substitute a nongeneric NP for the object VAR of a VARPREPP in 
prepobjrel, toobjrel, forobjrel, or postadjrel under a predrel in the VERBP.
\item[File] english:RC\_Substitution2.mrule (mrules15.mrule)
\item[Semantics]
\item[Example] x1 do [be [fond [of x2]]] + she $\rightarrow$ x1 do [be [fond 
[of her]]]
\item[Remarks]
\end{description}

\end{description}

\newpage
\subsection{TC\_NegAdaptation}
\begin{description}
\item[Kind] Iterative Transformation Class, followed by Obligatory Filter
\item[Task] To incorporate a negation that is in {\em meltnegrel\/} into the 
following word (in fact, the 
negation and the following word are both thrown away and replaced by another 
word). No rules have been written yet to yield the word {\em no\/} (neither 
from {\em not\/} alone, nor from {\em not\/} + {\em a\/}). There are different 
rules, for the different positions the meltnegrel can be in. The rules have as 
many subrules as necessary, to deal with all words possible in that specific 
position. The pairs that are dealt with are:\\
somebody - nobody; someone - no one; something - nothing; somewhere - nowhere; 
ever - never.

\vspace{1 cm}
\begin{description}
\item[Name] TSNegAdapt
\item[Task] To incorporate a negation in a word directly following it, when 
both are under S.
\item[File] english:TC\_NegAdaptation.mrule (mrules8.mrule)
\item[Semantics] --
\item[Example] not somebody did see me $\rightarrow$ nobody did see me
\item[Remarks]
\end{description}

\vspace{1 cm}
\begin{description}
\item[Name] TVPNegAdapt
\item[Task] To incorporate a negation in a word directly following it, when 
both are in the VP.
\item[File] english:TC\_NegAdaptation.mrule (mrules8.mrule)
\item[Semantics] --
\item[Example] I did [give them not something] $\rightarrow$ I did [give them 
nothing]
\item[Remarks] 
\end{description}

\vspace{1 cm}
\begin{description}
\item[Name] TPPSNegAdapt
\item[Task] To incorporate a negation in a word directly following the prep 
following it, when both are under S.
\item[File] english:TC\_NegAdaptation.mrule (mrules8.mrule)
\item[Semantics] --
\item[Example] Not to somebody could I tell my story $\rightarrow$ To nobody 
could I tell my story
\item[Remarks]
\end{description}

\vspace{1 cm}
\begin{description}
\item[Name] TPPVPNegAdapt
\item[Task] To incorporate a negation in a word directly following the prep 
following it, when both are in the VERBP.
\item[File] english:TC\_NegAdaptation.mrule (mrules8.mrule)
\item[Semantics] --
\item[Example] I did [talk not to someone] $\rightarrow$ I did [talk to no one]
\item[Remarks]
\end{description}

\vspace{1 cm}
\begin{description}
\item[Name] FPostNegAdapt
\item[Task] To assure that the iterative (and hence optional) transformation 
class has worked as often as it should in generation: there may not be a 
meltnegrel left.
\item[File] english:TC\_NegAdaptation.mrule (mrules8.mrule)
\item[Semantics] --
\item[Example] 
\item[Remarks]
\end{description}

\end{description}

\newpage
\subsection{TC\_PPorSoutofNPADJP}
\begin{description}
\item[Kind] Non-existent Transformation Class
\item[Task] This transformation class is intended to cover things like {
\em A book appeared by Chomsky\/}. The restrictions that hold for this kind of 
shift are hard to formulate, and hence this transformation class does 
not exist yet.
\end{description}

\newpage
\subsection{TC\_PronounOK}
\begin{description}
\item[Kind] at the moment, just one Obligatory Filter
\item[Task] To check whether there is an unallowed sequence of a personal 
pronoun directly following another NP. It is assumed that one day, extra rules 
will be written to produce a PrepObj variant ({\em to x1\/}) for some verbs 
that only allow 
it in this particular context (if there are any; all examples seem to be 
questionable, for instance the verb {\em teach\/}: although {\em I taught 
history to them\/} seems doubtful, Longman gives the pattern with {\em to\/} 
irrespective of the kind of direct object). No solution has been found for 
structures that are filtered out by this 
filter, but have no alternative formulation, like {\em permit\/} ($^*$I 
permitted it them, $^*$I permitted it to them, Ik stond het hun toe).

\vspace{1 cm}
\begin{description}
\item[Name] FNoPerspro
\item[Task] To check whether there is an unallowed sequence of a personal 
pronoun directly following another NP. 
\item[File] english:TC\_PronounOK.mrule (mrules6.mrule)
\item[Semantics] --
\item[Example] I did give them it
\item[Remarks] Perhaps the restrictions should be less restrictive, since now 
also {\em I offered him the new me\/} etc.\ are filtered out. However, it is 
not clear what the regularities are.
\end{description}

\end{description}

\newpage
\subsection{TC\_TopicCleft}
\begin{description}
\item[Kind] Obligatory Transformation Class
\item[Task] To deal further with topicalized elements. The transformation class 
has not been written yet, but will at least have to move a topicalized element 
in shiftrel away to a leftdislocrel position, so that it will not trigger 
inversion anymore\footnote{A provisional version of this transformation, {\em 
TMoveTopic\/}, had been written when this document was approved. It moves ALL 
topicalized elements in a finite `main' sentence (determined by its (super)
deixis, since the mood rules have not worked yet) to leftdislocrel, so that 
constructions like {\em No 
sooner than\/}, which do trigger inversion, cannot be handled yet. The 
transformation {\em TNoTopicCleft\/} lets all structures that do not have a 
topicalized element pass this transformation class. Both transformations can be 
found in file {\bf english: tc\_topiccleft.mrule}, which is {\em 
mrules96.mrule\/}.}. An 
alternative rule would be clefting of the topicalized element: {\em My father 
the house pleased\/} $\rightarrow$ {\em It was my father that the house 
pleased\/}. However, that rule will be quite complicated, since it will have to 
determine the correct mood for the subsentence, deal with (super)deixis etc.

\end{description}

\newpage
\subsection{RC\_MoodDetermination}
\begin{description}
\item[Kind] Obligatory Rule Class
\item[Task] to transform a CLAUSE to a main or subordinate SENTENCE, sentential 
NP or CLOSEDVERBPPROP, assigning
a value to the attributes {\bf mood} and {\bf modus} of the sentence. The 
embedded main verb also receives a value for {\bf modus}. Several moodrules for 
subordinate sentences have been given conditions on {\bf aspect}, to prevent 
unlikely or wrong combinations of aspect and mood.

Because the output of the moodrules can be a SENTENCE, an NP or a 
CLOSEDVERBPPROP, all transformations and rules following this rule class must
be able to deal with these categories as input. In case the transformation
class consists of one transformation for special cases plus a filter, there is
no difficulty, but when complementary rules or transformations must be written,
they must contain subrules or extra rules/transformations for all of these
input categories (see for instance the aspect neutralization rules below, in
TC\_FinalTempTransf). This is not practial at all, and should be prevented as 
much as possible.

\vspace{1 cm}
\begin{description}
\item[Name] RIndicMoodMain
\item[Task] To form a declarative, indicative main sentence
\item[File] english:RC\_MoodDetermination1.mrule (mrules14.mrule)
\item[Semantics]
\item[Example] $_{CL}$[x1 do buy a book] $\rightarrow$ $_S$[x1 does buy a book]
\item[Remarks]
\end{description}

\vspace{1 cm}
\begin{description}
\item[Name] RIndicMoodSub 
\item[Task] to form a declarative, indicative subordinate sentence, introducing 
the conjunction {\em THAT\/}.
\item[File] english:RC\_MoodDetermination1.mrule (mrules14.mrule)
\item[Semantics]
\item[Example] $_{CL}$[he have seen me] $\rightarrow$ $_S$[that he has seen me] 
\item[Remarks] If needed (e.g.\ for sentences with a wh-shifted subject),
the conjunction {\em That\/} is deleted in a separate 
transformation class, which is called only after the indicative subordinate 
sentence mood rules. 
See below, TC\_ConjThatDel.
\end{description}

\vspace{1 cm}
\begin{description}
\item[Name] RIndicFutMoodSub1
\item[Task] Generative rule to form a declarative, indicative subordinate 
sentence which has 
the auxiliary {\em will\/} instead of {\em do\/} and which begins with the 
conjunction {\em that\/}. 
\item[File] english:RC\_MoodDetermination1.mrule (mrules14.mrule)
\item[Semantics]
\item[Example] $_{CL}$[he did see me] $\rightarrow$ $_S$[that he would see me]
(He promised that he would see me)
\item[Remarks] This rule deals with the finite translation of LInjunSub (the 
Dutch OpenOmteinfMood: {\em Hij beloofde (om) me op te zoeken\/}. Only examples 
found so far: {\em beloven - promise, erop staan - insist on\/}). In analysis, 
a finite subsentence with {\em will\/} should be dealt with by the ordinary 
finite mood rules, translating into a Dutch finite ({\em .. dat hij me op zou 
zoeken\/}).

If needed (e.g.\ for sentences with a wh-shifted subject), the conjunction 
{\em that\/} is deleted in a separate 
transformation class, which is called only after the indicative subordinate 
sentence mood rules. See below, TC\_ConjThatDel.

No rules have been written yet for subsentences taking an obligatory {\em 
should\/} instead of {\em would\/}, like the complements of {\em order\/}: 
{\em Ik beval hem (om) weg te gaan - I ordered that he should leave\/}.

\end{description}

\vspace{1 cm}
\begin{description}
\item[Name] RIndicFutMoodSub2
\item[Task] Generative rule to form a declarative, indicative subordinate 
sentence which has the auxiliary {\em will\/} added (there is no {\em do\/}) 
and which begins with the conjunction {\em that\/}.
\item[File] english:RC\_MoodDetermination1.mrule (mrules14.mrule)
\item[Semantics]
\item[Example] $_{CL}$[he had seen the doctor by then] $\rightarrow$ $_S$[that 
he would have seen the doctor by then]
\item[Remarks] See above under RIndicFutMoodSub1
\end{description}

\vspace{1 cm}
\begin{description}
\item[Name] RIndicWhMoodMain
\item[Task] To form an indicative main wh-interrogative sentence
\item[File] english:RC\_MoodDetermination1.mrule (mrules14.mrule)
\item[Semantics]
\item[Example] $_{CL}$[who did buy books yesterday] $\rightarrow$ $_S$[who did 
buy books yesterday] (Who bought books yesterday?)
\item[Remarks]
\end{description}

\vspace{1 cm}
\begin{description}
\item[Name] RIndicWhMoodSub
\item[Task] To form an indicative subordinate wh-interrogative  sentence
\item[File] english:RC\_MoodDetermination1.mrule (mrules14.mrule)
\item[Semantics]
\item[Example] $_{CL}$[who did buy books yesterday] $\rightarrow$ $_S$[who did 
buy books yesterday] (I don't know who bought books yesterday)
\item[Remarks]
\end{description}

\vspace{1 cm}
\begin{description}
\item[Name] RIndicYesNoMoodMain
\item[Task] To form an indicative main yes/no-interrogative  sentence
\item[File] english:RC\_MoodDetermination1.mrule (mrules14.mrule)
\item[Semantics]
\item[Example] $_{CL}$[he do ever buy books] $\rightarrow$ $_S$[he does ever 
buy books]
\item[Remarks] The inversion will be done below, in TC\_AuxToComp.
\end{description}

\vspace{1 cm}
\begin{description}
\item[Name] RIndicYesNoMoodSub
\item[Task] To form an indicative subordinate yes/no-interrogative  sentence, 
introducing the conjunction {\em whether\/}. In analysis, the conjunction {\em 
if\/} is also accepted.
\item[File] english:RC\_MoodDetermination1.mrule (mrules14.mrule)
\item[Semantics]
\item[Example] $_{CL}$[he should go there] $\rightarrow$ $_S$[whether (if) he 
should go there]
\item[Remarks] It is assumed that when {\em if\/} is not allowed, it is 
stopped where that can be decided (e.g.\ in the relevant PropSubst rule).
\end{description}

\vspace{1 cm}
\begin{description}
\item[Name] RImpMood
\item[Task] To form an imperative (main) sentence, deleting the subject {\em 
You\/}. The rule uses a parameter {\bf numberpar} to determine the correct 
number of imperative (which in English is simply the infinitive form) and 
subject. Note that an English imperative will always be ambiguous between 
singular and plural {\em you\/}, unless a reflexive solves the ambiguity.
\item[File] english:RC\_MoodDetermination1.mrule (mrules14.mrule)
\item[Semantics]
\item[Example] $_{CL}$[you do buy a book for John] $\rightarrow$ $_S$[do buy a 
book for John]
\item[Remarks] No rules have been written in English yet for the `polite' 
imperative with {\em Do\/}: {\em Do come in, Do wash yourselves\/}. This might 
be a `PosImpMood', introducing POS (and hence blocking do-deletion). Perhaps 
there is an interrelation with the {\em U-imperatief\/} in Dutch: {\em Gaat 
U verder - Do continue\/}.
\end{description}

\vspace{1 cm}
\begin{description}
\item[Name] RFinRelMood
\item[Task] To form an indicative relative subordinate sentence, introducing 
the conjunction {\em that\/}.
\item[File] english:RC\_MoodDetermination1.mrule (mrules14.mrule)
\item[Semantics]
\item[Example] $_{CL}$[x1 he did see] $\rightarrow$ $_S$[x1 that he did see] 
(Is this the man (whom/that) he saw?)
\item[Remarks] In RCNmodRELSENT, the conjunction is deleted again and replaced 
by nothing, or by {\em who/which\/} or the RELPRO {\em that\/}. Thus, the 
current rule might as well not have introduced a conjunction.
\end{description}

\vspace{1 cm}
\begin{description}
\item[Name] RWhModMood
\item[Task] To form an indicative (wh-) modifying subordinate sentence
\item[File] english:RC\_MoodDetermination1.mrule (mrules14.mrule)
\item[Semantics]
\item[Example] $_{CL}$[Peter could read] $\rightarrow$ $_S$[Peter could read] 
(He bought more books than Peter could read)
\item[Remarks] It might well be that English does not need a separate WhMod 
mood and that relative mood suffices (cf.\ Dutch, where a wh-word surfaces: 
{\em Hij heeft meer problemen opgelost dan {\em waar} ik op gerekend had\/}). If 
it is decided to throw the WhMod rules away, and use the ordinary relative 
rules, the conjunction {\em that\/} must be deleted (either in the relative rule
(see the remark made there), or in the NPrule introducing a modifier sentence).
\end{description}

\vspace{1 cm}
\begin{description}
\item[Name] ROpenInfMood
\item[Task] To form an (open) subordinate sentence with an infinitive without 
{\em to\/}. This mood is allowed for perfective clauses only.
\item[File] english:RC\_MoodDetermination2.mrule (mrules13.mrule)
\item[Semantics]
\item[Example] $_{CL}$[x1 come] $\rightarrow$ $_S$[x1 come] (I can come)
\item[Remarks] No rule has been written yet to deal with the introduction of 
{\em to\/} in case the sentence is complement in a passive clause: {\em The 
sight of all that money made him sing - He was made {\em to} sing\/}. Such a 
rule might perhaps best be added after the proposition substitution rules, or 
after the passive clause formation rule.
\end{description}

\vspace{1 cm}
\begin{description}
\item[Name] RClosedInfMood
\item[Task] To form a (closed) subordinate sentence with an infinitive without {
\em to\/}.
\item[File] english:RC\_MoodDetermination2.mrule (mrules13.mrule)
\item[Semantics]
\item[Example] $_{CL}$[he come] $\rightarrow$ $_S$[he come] (I saw him come)
\item[Remarks]
\end{description}

\vspace{1 cm}
\begin{description}
\item[Name] ROpenToinfMood
\item[Task] To form an (open) subordinate sentence with an infinitive with {
\em to\/}.
\item[File] english:RC\_MoodDetermination2.mrule (mrules13.mrule)
\item[Semantics]
\item[Example] $_{CL}$[x1 buy books] $\rightarrow$ $_S$[x1 to buy books] (I 
want to buy books)
\item[Remarks] It is assumed that nothing can come between {\em to\/} and the 
infinitive, i.e.\ no construction like {\em I wanted to quickly buy books\/} 
should occur.
\end{description}

\vspace{1 cm}
\begin{description}
\item[Name] RClosedToinfMood
\item[Task] To form a (closed) subordinate sentence with an infinitive with {
\em to\/}.
\item[File] english:RC\_MoodDetermination2.mrule (mrules13.mrule)
\item[Semantics]
\item[Example] $_{CL}$[John have bought books] $\rightarrow$ $_S$[John to have 
bought books] (I believe John to have bought books)
\item[Remarks]
\end{description}

\vspace{1 cm}
\begin{description}
\item[Name] RWhToinfMood
\item[Task] To form an (open) subordinate wh-interrogative sentence with an 
infinitive with {\em to\/}. This mood is allowed for perfective clauses only.
\item[File] english:RC\_MoodDetermination2.mrule (mrules13.mrule)
\item[Semantics]
\item[Example] $_{CL}$[what step x1 take now] $\rightarrow$ $_S$[what step x1 
to take now] (I don't know what step to take now)
\item[Remarks]
\end{description}

\vspace{1 cm}
\begin{description}
\item[Name] RToinfRelMood
\item[Task] To form an (open) subordinate relative sentence with an 
infinitive with {\em to\/}. This mood is allowed for perfective clauses only.
\item[File] english:RC\_MoodDetermination2.mrule (mrules13.mrule)
\item[Semantics]
\item[Example] $_{CL}$[x1 read] $\rightarrow$ $_S$[x1 to read] (The books to 
read are on the table)
\item[Remarks] 
\end{description}

\vspace{1 cm}
\begin{description}
\item[Name] RToinfWhModMood
\item[Task] To form an (open) subordinate (wh-)modifying sentence with an 
infinitive with {\em to\/}. This mood is allowed for perfective clauses only.
\item[File] english:RC\_MoodDetermination2.mrule (mrules13.mrule)
\item[Semantics]
\item[Example] $_{CL}$[x1 talk to] $\rightarrow$ $_S$[x1 to talk to] (He was 
too stubborn to talk to)
\item[Remarks] See what was said under RWhModMood about the existence of WhMods 
in English. Of course, for infinitives there is no conjunction {\em that\/}.
\end{description}

\vspace{1 cm}
\begin{description}
\item[Name] RFortoInfMood
\item[Task] To form a (closed) subordinate declarative sentence starting with 
the 
conjunction {\em for\/}, and having an infinitive with {\em to\/}. The subject 
is assigned oblique case. This mood is allowed for perfective clauses only.
\item[File] english:RC\_MoodDetermination2.mrule (mrules13.mrule)
\item[Semantics]
\item[Example] $_{CL}$[John go] $\rightarrow$ $_S$[for John to go] (I want for 
John to go)
\item[Remarks] It is assumed that in case a fortoinf has an {\em ing-form\/}, 
it must be translated into a Dutch sentence with {\em aan het\/}: {\em I prefer 
for the children to be playing\/} $\rightarrow$ {\em Ik prefereer dat de 
kinderen 
aan het spelen zijn\/}. The same holds for other cases where only perfective 
aspect is allowed.
\end{description}

\vspace{1 cm}
\begin{description}
\item[Name] RFortoInfRelMood
\item[Task] To form a (closed) subordinate relative sentence starting with the
conjunction {\em for\/}, and having an infinitive with {\em to\/}.
This mood is allowed for perfective clauses only.
\item[File] english:RC\_MoodDetermination2.mrule (mrules13.mrule)
\item[Semantics]
\item[Example] $_{CL}$[x1 Peter read] $\rightarrow$ $_S$[x1 for Peter to read]
(The books for Peter to read are on the table)
\item[Remarks]
\end{description}

\vspace{1 cm}
\begin{description}
\item[Name] RFortoinfWhModMood
\item[Task] To form a (closed) subordinate (wh-)modifying sentence starting 
with the
conjunction {\em for\/}, and having an infinitive with {\em to\/}.
This mood is allowed for perfective clauses only.
\item[File] english:RC\_MoodDetermination2.mrule (mrules13.mrule)
\item[Semantics]
\item[Example] $_{CL}$[x1 we talk to] $\rightarrow$ $_S$[x1 for us to talk to]
(He is too stubborn for us to talk to)
\item[Remarks] See what was said under RWhModMood about the existence of WhMods 
in English. Of course, for infinitives there is no conjunction {\em that\/}.
\end{description}

\vspace{1 cm}
\begin{description}
\item[Name] ROpenIngMood
\item[Task] To form an open subordinate declarative sentence with a gerund.
This mood is allowed for perfective clauses only, so there will not be two {\em 
-ing\/} forms following each other.
\item[File] english:RC\_MoodDetermination2.mrule (mrules13.mrule)
\item[Semantics]
\item[Example] $_{CL}$[x1 be called a lady] $\rightarrow$ $_S$[x1 being called 
a lady] (I like being called a lady)
\item[Remarks] This rule is an alternative to RNPOpenIngMood, where the same 
structure is provided but headed by a new NP node, which takes the SENTENCE as 
its head. The current rule just produces a SENTENCE; it will usually function 
as the complement of verbs.
\end{description}

\vspace{1 cm}
\begin{description}
\item[Name] RAnterelIngMood
\item[Task] To form an open subordinate relative sentence with a gerund.
This mood is allowed for imperfective clauses only. 
In case the auxiliary of 
the progressive {\em be\/} is present, it is deleted.
\item[File] english:RC\_MoodDetermination2.mrule (mrules13.mrule)
\item[Semantics]
\item[Example] $_{CL}$[x1 be falling from that tree] $\rightarrow$ 
$_S$[x1 falling from that tree] (The leaves falling from that tree 
indicate acid rain)
\item[Remarks] The name `anterelative' is retained for the SENTENCE {\bf mood} 
attribute, although English uses 
post-relatives (cf.\ Dutch {\em De appels etende ezel\/}).
\end{description}

\vspace{1 cm}
\begin{description}
\item[Name] RAccIngMood
\item[Task] To form a closed subordinate declarative sentence with a gerund.
This mood rule accepts both perfective and imperfective clauses.
\item[File] english:RC\_MoodDetermination2.mrule (mrules13.mrule)
\item[Semantics]
\item[Example] $_{CL}$[what he do] $\rightarrow$ $_S$[what he doing] (What did 
you see him doing?)
\item[Remarks] The subject will receive case in the next cycle (see 
TExceptCaseAssign). In case there is an auxiliary of the progressive, the 
sentence will get two subsequent {\em -ing\/} forms: {\em I heard her being 
hitting the boy\/}. This will have to be changed.
\end{description}

\vspace{1 cm}
\begin{description}
\item[Name] RPastPartMood
\item[Task] To form an open subordinate relative sentence with a past 
participle. If there is an auxiliary of the passive, it is deleted; if there 
is an auxiliary of the perfect (retrospective reading), it is deleted too. 
The rule works for passives and ergatives.
This mood is allowed for perfective clauses only. 
\item[File] english:RC\_MoodDetermination2.mrule (mrules13.mrule)
\item[Semantics]
\item[Example] $_{CL}$[x1 be killed by his enemies] $\rightarrow$ $_S$[x1 
killed by his enemies] (The man killed by his enemies]
\item[Remarks] The name `anterelative' is retained for the SENTENCE {\bf mood} 
attribute, although English uses 
post-relatives (cf.\ Dutch {\em De door zijn vijanden gedode man\/})
\end{description}

\vspace{1 cm}
\begin{description}
\item[Name] RNPPossIngMood
\item[Task] To form a closed subordinate declarative sentence with a gerund, 
and with a possessive as a subject. The sentence is given a new top node, 
NP. 
This mood is allowed for perfective clauses only. 
\item[File] english:RC\_MoodDetermination3.mrule (mrules12.mrule)
\item[Semantics]
\item[Example] $_{CL}$[My father not know the precise time of arrival] 
$\rightarrow$ $_{NP}$[$_S$[My father's not knowing the precise time of arrival]
] (caused many inconveniences)
\item[Remarks]
An alternative analysis of PossIng sentences might be one where the possessive 
is a modifier of the rest of the sentence as an openIng NP (see the rule below),
together forming 
one large NP: $_{NP}$[ $_{mod}$[my father's] $_{NP}$[$_S$[not knowing ...]] ].
However, usually modification takes place at the CN-level.
\end{description}


\vspace{1 cm}
\begin{description}
\item[Name] RNPOpenIngMood
\item[Task] To form an open subordinate declarative sentence with a gerund. The 
sentence is given a new top node, NP.
This mood is allowed for perfective clauses only. 
\item[File] english:RC\_MoodDetermination3.mrule (mrules12.mrule)
\item[Semantics]
\item[Example] $_{CL}$[x1 join the expedition] $\rightarrow$ $_{NP}$[$_S$[x1 
joining the expedition]] (I still hesitate about joining the expedition)
\item[Remarks] This rule is an alternative to ROpenIngMood, where the same 
structure is provided but not headed by a new NP node
and just producing a SENTENCE; an NP will usually function 
as the object of a preposition, or as a subject.
\end{description}

\vspace{1 cm}
\begin{description}
\item[Name] RLetsMood
\item[Task] To form a main sentence starting with {\em Let us ...\/}. 
This mood is allowed for perfective clauses only. 
\item[File] english:RC\_MoodDetermination3.mrule (mrules12.mrule)
\item[Semantics]
\item[Example] $_{CL}$[we do go home] $\rightarrow$ $_S$[Let us go home]
\item[Remarks] No GLUE for contraction is provided yet ({\em Let's ...\/}),
although morphology already has a glue-rule for this case\footnote{This had 
been remedied by the time this document was approved.}.
\end{description}

\vspace{1 cm}
\begin{description}
\item[Name] RVPPROPmood
\item[Task] To reduce a clause to a CLOSEDVERBPPROP. This rule is needed 
because no separate VERBPPROP-grammar exists (it would simply duplicate many 
rules of the current sentence grammars).
This mood is allowed for perfective clauses only. 
\item[File] english:RC\_MoodDetermination3.mrule (mrules12.mrule)
\item[Semantics]
\item[Example] $_{CL}$[the house be built by a famous architect] $\rightarrow$
\\ 
$_{CLOSEDVERBPPROP}$[the house built by a famous architect] \\ 
(He had the house built by him)
\item[Remarks] The subject must receive case in a next cycle (see 
TExceptCaseAssign). A ClosedVerbpprop only seems needed as complement to the 
verbs {\em have\/} and {\em get\/} (if the latter verb is vp010).
\end{description}

\end{description}

\newpage
\subsection{TC\_ConjThatDel}
\begin{description}
\item[Kind] Obligatory Transformation Class, called only after RIndicMoodSub,
RIndicFutMoodSub1, and RIndicFutMoodSub2
\item[Task] To delete the conjunction {\em that\/} where necessary (i.e.\ when
the subject is shifted); in analysis, sentences without {\em that\/} are 
accepted even when there is a subject, and the conjunction is added. Note that 
sentences which have 
a finite extraposed subsentence starting without the conjunction {\em that\/}
will be stopped, analytically speaking, in TExtraposition1, so all sentences 
coming to this transformation will have the conjunction if they came from an 
extraposrel. If they came from a complrel, {\em that\/} may be absent if the 
verb allows it (see RComplSentSubst in the previous subgrammar; note that there 
is a mistake in the documentation to that rule: reference is made to 
RIndicWhMoodSub instead of to the current transformation, TConjThatDel. In 
analysis, RComplSentSubst does not check whether there is a shifted subject).
The problem mentioned in TCompl\-SentSubst for verbs not allowing 
{\em that\/}-deletion and trying to make a finite complement sentence with a 
shifted subject ({\em $^{*}$Who did they acknowledge (that) was defeated?\/})
has not been solved yet. A sentence like {\em Who did you say that 
has left\/} will be stopped here, because the embedded sentence contains a 
shifted subject but still has a conjunction.

\vspace{1 cm}
\begin{description}
\item[Name] TConjThatDeletion 
\item[Task] To delete the conjunction {\em that\/} when
the subject is shifted; in analysis, sentences without {\em that\/} are 
accepted even when there is a subject.
\item[File] english:RC\_MoodDetermination1.mrule (mrules14.mrule)
\item[Semantics] --
\item[Example]  who that gave me this $\rightarrow$ who gave me this (Who do 
you think gave me this)
\item[Remarks] Although doc.\ 150 mentions a transformation class 
TC\_Conjunction-Deletion, this probably is a copying error, and should be 
replaced by TC\_DoDeletion. Thus, the current class was not foreseen in doc.\ 
150.
\end{description}

\end{description}

\newpage
\subsection{RC\_ConjSent}
\begin{description}
\item[Kind] Optional Rule Class
\item[Task] To form adverbial subordinate sentences (these were not formed in 
the mood\-rules: there, 
only the subordinate sentence was formed, but no conjunction 
was added yet). In case the sentence consists of a sentential NP and the 
conjunction is a preposition, the output of this rule class is a PREPP. Hence, 
all further rule and transformation classes must be able to deal not only with 
SENTENCEs, NPs and CLOSEDVERBPPROPs, but also with PREPPs.

\vspace{1 cm}
\begin{description}
\item[Name] RConjFinSubsent
\item[Task] To form an adverbial sentence from a finite subordinate sentence, 
deleting the conjunction {\em that\/} and introducing a new adverbial 
conjunction.
\item[File] english:RC\_ConjSent.mrule (mrules92.mrule)
\item[Semantics]
\item[Example] \mbox{}\\
that he did leave + when $\rightarrow$ when he did leave\\
 that he had gone mad + as if $\rightarrow$ as if he had gone mad
\item[Remarks]
\end{description}

\vspace{1 cm}
\begin{description}
\item[Name] RConjIngSubsent
\item[Task] To form an adverbial sentence from an open infinite subordinate 
sentence with {\em -ing\/}, introducing an adverbial conjunction.
\item[File] english:RC\_ConjSent.mrule (mrules92.mrule)
\item[Semantics]
\item[Example] x1 coming home + when $\rightarrow$ when x1 coming home (When 
coming home, I always am anxious about burglars)
\item[Remarks]
\end{description}

\vspace{1 cm}
\begin{description}
\item[Name] RPrepFinSubsent
\item[Task] To form an adverbial sentence from a finite subordinate 
sentence, deleting the conjunction {\em that\/} and introducing a 
preposition which functions as an adverbial conjunction.
\item[File] english:RC\_ConjSent.mrule (mrules92.mrule)
\item[Semantics]
\item[Example] that he left + before $\rightarrow$ before he left
\item[Remarks] The rule has two condition-action pairs: if the PREP has the 
value {\em temp\/} in its {\bf subcs}, a temporal sentence is made; if there is 
more in the {\bf subcs} of the PREP, then also a non-temporal sentence is made.
\end{description}

\vspace{1 cm}
\begin{description}
\item[Name] RPrepIngSubsent
\item[Task] To form an adverbial PREPP from an NP heading an (open or 
closed) infinite 
subordinate sentence with {\em -ing\/}, introducing a preposition functioning 
as an adverbial conjunction.
\item[File] english:RC\_ConjSent.mrule (mrules92.mrule)
\item[Semantics]
\item[Example] \mbox{}\\
x1 leaving + before $\rightarrow$ before x1 leaving\\
my noticing it + without $\rightarrow$ without my noticing it
\item[Remarks] The rule has two condition-action pairs: if the PREP has the 
value {\em temp\/} in its {\bf subcs}, a temporal PREPP is made (indicated by 
setting the attribute {\bf actsubcefs} of the PREPP at the value {\em 
[temp]\/}; a PREPP does not have an attribute {\bf temporal}); if there is 
more in the {\bf subcs} of the PREP, then also a non-temporal PREPP is made.
\end{description}

\vspace{1 cm}
\begin{description}
\item[Name] RConjToinfSubsent
\item[Task] To form an adverbial sentence from an open infinite subordinate 
sentence with {\em to\/}, introducing an adverbial conjunction.
\item[File] english:RC\_ConjSent.mrule (mrules92.mrule)
\item[Semantics]
\item[Example] x1 to come + in order $\rightarrow$ in order x1 to come (In 
order to come, he took some days off)
\item[Remarks]
\end{description}

\end{description}

\newpage
\subsection{TC\_TempEMPTYDeletion}
\begin{description}
\item[Kind] Two subsequent Optional Transformations, each followed by an 
Obligatory Filter
\item[Task] To delete abstract temporal adverbs (or to insert them, in 
analysis). This transformation class was 
not specified in doc.\ 150. For more information, see doc.\ 263, {\em 
Documentation of the rules for the translation of temporal expressions in 
Rosetta3, part I\/}.

\vspace{1 cm}
\begin{description}
\item[Name] TRefDeletion
\item[Task] To delete an abstract temporal (present or past) reference adverb 
\item[File] english:TC\_TempEMPTYDeletion.mrule (mrules77.mrule)
\item[Semantics] --
\item[Example] He does walk REFEMPTY $\rightarrow$ He does walk
\item[Remarks] This optional transformation is followed by an obligatory 
filter, to assure its proper application.
\end{description}

\vspace{1 cm}
\begin{description}
\item[Name] FRefDeletion 
\item[Task] To assure the proper application in generation of the optional 
transformation TRefDeletion: all structures that still have an abstract 
temporal reference adverb left are filtered out.
\item[File] english:TC\_TempEMPTYDeletion.mrule (mrules77.mrule)
\item[Example] 
\item[Remarks]
\end{description}

\vspace{1 cm}
\begin{description}
\item[Name] TRetroDeletion
\item[Task] To delete an abstract temporal retrospective adverb 
\item[File] english:TC\_TempEMPTYDeletion.mrule (mrules77.mrule)
\item[Semantics] --
\item[Example] He has corrected the first version RETROEMPTY 
$\rightarrow$ He has corrected the first version
\item[Remarks] This optional transformation is followed by an obligatory 
filter, to assure its proper application.
\end{description}

\vspace{1 cm}
\begin{description}
\item[Name] FRetroDeletion
\item[Task] To assure the proper application in generation of the optional 
transformation TRetroDeletion: all structures that still have an abstract 
temporal retrospective adverb left are filtered out.
\item[File] english:TC\_TempEMPTYDeletion.mrule (mrules77.mrule)
\item[Example] 
\item[Remarks]
\end{description}

\end{description}

\newpage
\subsection{TC\_DeixisRetroAdaptation}
\begin{description}
\item[Kind] Optional Transformation Class, followed by Obligatory Filter
\item[Task] To reset the value of the attribute {\bf deixis} of an infinite 
perfective SENTENCE with {\em have\/},
which was assigned in special deixis rules (RinfinPastdeixisSpec and 
RInfinPastSuperdeixisSpec),
to a default value, and to give the SENTENCE a value {\em true\/} for the 
attribute {\bf retro}. This is needed for efficiency of the Surface Parser, 
which cannot recognize special deixis configurations.\\
(NB.\ The Surface Parser cannot recognize retrospective configurations either. 
Therefore, Tretroneutralization will reset the value of {\bf retro} to {\em 
false}.)

This transformation class was 
not specified in doc.\ 150. For more information, see docs 263, {\em 
Documentation of the rules for the translation of temporal expressions in 
Rosetta3, part I\/}, and 320, {\em Superdeixis in Rosetta3\/}.

\vspace{1 cm}
\begin{description}
\item[Name] TDeixisRetroAdaptation1
\item[Task] To reset the value of the attribute {\bf deixis} of an infinite 
perfective non-retrospective SENTENCE with {\em have\/}
to a default value (if there only is a value for {\bf superdeixis}, nothing
is done), and to give the SENTENCE a value {\em true\/} for the 
attribute {\bf retro}. 
\item[File] english:TC\_DeixisRetroAdaptation.mrule (mrules75.mrule)
\item[Semantics] --
\item[Example] [Bill to have killed John yesterday]$_{\frac{pastdeixis}
{retro=false}}$ $\rightarrow$ [Bill to have killed John yesterday]$_{\frac
{omegadeixis}{retro=true}}$
\item[Remarks] This optional transformation is followed by an obligatory 
filter, to assure its proper application.
\end{description}

\vspace{1 cm}
\begin{description}
\item[Name] FDeixisRetroAdaptation1
\item[Task] To assure the proper application in generation of the optional 
transformation TDeixisRetroAdaptation1: all infinite perfective non-
retrospective sentences with {\em have\/} that still have a non-default value 
for (super)deixis are filtered out.
\item[File] english:TC\_DeixisRetroAdaptation.mrule (mrules75.mrule)
\item[Example] 
\item[Remarks]
\end{description}

\end{description}

\newpage
\subsection{TC\_FinalTempTransf}
\begin{description}
\item[Kind] Optional Transformation, followed by an Obligatory Filter, followed 
by an Obligatory Transformation Class
\item[Task] To reset a number of attributes to their default values, in order 
to allow for a more simple Surface Parser. There are rules for the sentence 
attributes {\bf retro} and {\bf aspect}.

This transformation class was 
not specified in doc.\ 150. For more information, see doc.\ 263, {\em 
Documentation of the rules for the translation of temporal expressions in 
Rosetta3, part I\/}.

\vspace{1 cm}
\begin{description}
\item[Name] TRetroNeutralization
\item[Task] To reset the value of the attribute {\bf retro} of (infinite) 
sentences that take {\em have\/} as first auxiliary to {\em false\/}.
\item[File] english:TC\_FinalTempTransf.mrule (mrules34.mrule)
\item[Semantics] --
\item[Example] [Bill to have killed John]$_{retro=true}$ 
$\rightarrow$ [Bill to have killed John]$_{retro=false}$
\item[Remarks] This optional transformation is followed by an obligatory 
filter, to assure its proper application. An extra rule will be added for 
finite sentences, together with the accompanying filter\footnote{This had been 
done by the time this document was approved. The original rule is now called 
TRetroNeutralization1, and TRetroNeutralization2 is added. The original filter 
was renamed to FRetroNeutralization1, and FRetroNeutralization2 was added.}.
\end{description}

\vspace{1 cm}
\begin{description}
\item[Name] FRetroNeutralization
\item[Task] To assure the proper application in generation of the optional 
transformation TRetroNeutralization: all (infinite) 
retrospective sentences with {\em have\/} as first auxiliary are filtered out.
\item[File] english:TC\_FinalTempTransf.mrule (mrules34.mrule)
\item[Example] 
\item[Remarks] See the footnote to TRetroNeutralization.
\end{description}

\vspace{1 cm}
\begin{description}
\item[Name] TAspectNeutralization
\item[Task] To reset the value of the sentence attribute {\bf aspect} 
(which must be 
{\em perfective, imperfective\/} or {\em habitual\/}) to {\em omegaaspect\/} 
in generation, and to introduce these non-default values in analysis. For 
sentences with imperfective aspect, there may not be a duration adverbial.
\item[File] english:TC\_FinalTempTransf.mrule (mrules34.mrule)
\item[Semantics] --
\item[Example] [John works]$_{perfective}$ $\rightarrow$ [John works]$_
{omegaaspect}$
\item[Remarks]
\end{description}

\vspace{1 cm}
\begin{description}
\item[Name] TaspectNeutralization2
\item[Task] To reset the value of the attributes {\bf aspect} of a SENTENCE and 
of its head NP (which must be 
{\em perfective, imperfective\/} or {\em habitual\/}) to {\em omegaaspect\/} 
in generation, and to introduce these non-default values in analysis.
\item[File] english:TC\_FinalTempTransf.mrule (mrules34.mrule)
\item[Semantics] --
\item[Example] [His working]$_{imperfective}$ $\rightarrow$ [His working]$_
{omegaaspect}$
\item[Remarks]
\end{description}

\vspace{1 cm}
\begin{description}
\item[Name] TaspectNeutralization3
\item[Task] To reset the value of the attributes {\bf aspect} of a SENTENCE and 
of its head NP and of the governing sentential PREPP (the value must be 
{\em perfective, imperfective\/} or {\em habitual\/}) to {\em omegaaspect\/} 
in generation, and to introduce these non-default values in analysis.
\item[File] english:TC\_FinalTempTransf.mrule (mrules34.mrule)
\item[Semantics] --
\item[Example] [Without always realising it]$_{habitual}$ $\rightarrow$ 
[Without always realising it]$_{omegaaspect}$
\item[Remarks]
\end{description}

\vspace{1 cm}
\begin{description}
\item[Name] TaspectNeutralization4
\item[Task] To let CLOSEDVERBPPROPs pass this obligatory transformation class
\item[File] english:TC\_FinalTempTransf.mrule (mrules34.mrule)
\item[Semantics] --
\item[Example] [the house built] $\rightarrow$ [the house built]
\item[Remarks] This is not elegant; there should be a separate subgrammar for 
VERBPFORMULAtoPROP, in which this transformation class would not 
exist. 
\end{description}

\end{description}

\newpage
\subsection{TC\_DoBeDeletion}
\begin{description}
\item[Kind] Obligatory Transformation Class
\item[Task] To delete the auxiliary {\em do\/} for verbs that never need it, 
viz.\ the copula {\em be\/} and (in analysis only) the main verb {\em have\/} 
(in generation, {\em have\/} is treated as an ordinary main verb). Remember 
that {\em do\/} was introduced even for {\em be\/}, to provide a uniform 
structure for all clauses (the main verb always is in the VERBP, and there 
always is one auxiliary to the left of the VERBP). Imperatives,
which may require a combination of {\em do\/} and 
the copula {\em be\/} in case there is a negation, are not 
provided for yet: {\em Don't (you) be impudent to me, young man!\/}\footnote{
This had been remedied by the time this document was approved: in case the 
sentence is imperative and contains a negation in S, {\em do\/} is not 
deleted.}.

This specific case of {\em do\/}-deletion is crucially ordered before 
TC\_AuxToComp, or the auxiliary will end up in the Comp-position, instead of 
the main verb. `Ordinary' {\em do\/}-deletion occurs after TC\_AuxToComp. The 
transformation class was not foreseen in doc.\ 150. 

\vspace{1 cm}
\begin{description}
\item[Name] TDoBeDeletion
\item[Task] To delete the auxiliary {\em do\/} for verbs that never need it, 
viz.\ the copula {\em be\/} and (in analysis only) the main verb {\em have\/}. 
The copula is moved out of the VERBP to the position for auxiliaries 
(leaving the VERBP without a head).
\item[File] english:TC\_DoBeDeletion.mrule (mrules82.mrule, begin)
\item[Semantics] --
\item[Example] He does not be a doctor $\rightarrow$ He is not a doctor\\
He does have a book $\leftarrow$ He has a book
\item[Remarks]
\end{description}

\vspace{1 cm}
\begin{description}
\item[Name] TNoDoBeDeletion
\item[Task] Vacuous rule, to let structures in which there is no {\em do\/} to 
delete pass this transformation class.
\item[File] english:TC\_DoBeDeletion.mrule (mrules82.mrule, begin)
\item[Semantics] --
\item[Example] He does sing, etc.
\item[Remarks]
\end{description}

\end{description}

\newpage
\subsection{TC\_ShallWillSwap}
\begin{description}
\item[Kind] Optional Transformation Class, preceded by Obligatory Filter
\item[Task] To replace the auxiliary {\em will\/} by the form {\em shall\/} in 
main sentence present tense questions, in case there is a 1st person subject. 
In analysis, {\em shall\/} is accepted even when the sentence is not a 
question. This transformation class has not been written yet\footnote{By the 
time this document was approved, it had been written. The rule is called {\em 
TShallWill\/}, and the prefilter is {\em FNoShall\/}. They can be found in file 
{\bf english:TC\_DoBeDeletion}, which is {\em mrules82.mrule\/}. The 
transformation is optional, 
so in generation both {\em shall\/} and {\em will\/} are produced for first 
person subjects.}.

\end{description}

\newpage
\subsection{TC\_NegAuxAdapt}
\begin{description}
\item[Kind] Optional Transformation Class, preceded by Obligatory Filter
\item[Task] To contract the negation and a preceding modal or auxiliary.
This contraction must occur before TC\_AuxToComp, to allow the contracted form 
to move as a whole. It is crucially ordered after TC\_DoBeDeletion, since 
otherwise {\em 
be\/} and {\em not\/} would not be adjacent yet. The transformation class was 
not thought of in doc.\ 150. No conditions have been added yet to 
guide generation: both contracted and uncontracted forms are generated. It 
still has to be decided whether both forms are indeed equivalent, and which one 
is preferred in generation.

\vspace{1 cm}
\begin{description}
\item[Name] FPreNegAuxAdapt
\item[Task] Speed filter, to assure that in analysis the optional TNegAuxAdapt 
has worked where needed: no contracted negations may be left.
\item[File] english:TC\_AuxToComp.mrule (mrules11.mrule, begin)
\item[Example] 
\item[Remarks]
\end{description}

\vspace{1 cm}
\begin{description}
\item[Name] TNegAuxAdapt
\item[Task] To contract a negation and the modal or auxiliary just preceding 
it, forming a new (complex) VERB-node.
\item[File] english:TC\_AuxToComp.mrule (mrules11.mrule, begin)
\item[Semantics] --
\item[Example] They are not at home $\rightarrow$ They aren't at home\\
He must not see me $\rightarrow$ He mustn't see me
\item[Remarks] \mbox{}
\begin{itemize}
\item A number of glue rules only exist in analysis, e.g.\ {\em can not\/} 
$\rightarrow$ {\em can't\/}. For 
the modal {\em may\/}, no glue rule exists at all (correctly, of course).
Since the
current rule only excludes {\em am\/} from contraction (only in generation, so 
that in analysis {\em aren't I?\/} can be dealt with; the rule has not yet been 
restricted to accept {\em aren't I\/} only in tag questions), 
other forms for which 
no rule exists in (generative) segmentation will crash. This still has to be 
amended!\footnote{This had been done by the time this document was approved: 
the modal {\em may\/} was excluded from the rule. {\em Might\/} need not be 
excluded: the form {\em mightn't\/} exists. Furthermore, the glue rule for {\em 
can't\/} was made reversible. In analysis, restrictions were put on {\em aren't 
} and {\em ain't\/}: they are accepted only in questions.}
\item For {\em cannot\/}, no glue rule exists at all; there is an extra problem 
there, since just specifying {\em can\/} + {\em GLUE\/} + {\em not\/} in the 
segmentation rules would interfere with the other GLUErule for {\em can\/}, 
returning {\em can't\/}. Perhaps two different {\em not\/}s must be 
used\footnote{This had been implemented by the time this document was 
approved: a reversible
glue rule for {\em cannot\/} was added, using a `special' {\em n\`{o}t\/}. 
Since this {\em n\`{o}t\/} has its own s-key, it is not involved in the current 
transformation. It is used in a new transformation class, {\em 
TC\_CanNegIncorp\/} (see below).}.

Another point is that a construction like {\em Cannot you be quiet\/} seems 
awkward, 
while {\em He cannot come\/} is perfectly OK. Perhaps {\em Cannot\/} must be 
excluded explicitly from TC\_AuxToComp\footnote{Because {\em Cannot\/} is not 
formed by the current rule (see the previous footnote), the construction with {
\em Cannot he come\/} will never be produced. For {\em He cannot come\/}, 
see the new TC\_CanNegIncorp.}.
\end{itemize}
\end{description}

\end{description}

\newpage
\subsection{TC\_AuxToComp}
\begin{description}
\item[Kind] Obligatory Transformation Class
\item[Task] To invert subject and finite auxiliary (operator) for yes/no 
questions and sentences with a shiftrel.

\vspace{1 cm}
\begin{description}
\item[Name] TAuxToComp
\item[Task] To invert subject and operator (which may be complex) for 
yes/no questions and sentences with a shiftrel.
\item[File] english:TC\_AuxToComp.mrule (mrules11.mrule, end)
\item[Semantics] --
\item[Example] \mbox{}\\
Whom you did see leave $\rightarrow$ Whom did you see leave \\
You didn't leave $\rightarrow$ Didn't you leave \\
You did not leave $\rightarrow$ Did you not leave
\item[Remarks] Not all elements are in shiftrel 
yet that should be there, e.g.\ optative subjunctives {\em Far be it ...\/} and 
certain topicalized elements (they have all been moved to leftdislocrel): {\em 
So dark was it that ...\/}. Also, no rules have been written to deal with 
subject - verb inversion (i.e.\ without any auxiliary), like in the ergative 
{\em Here comes the bus\/} and in topicalized constructions like {\em So say 
all of us\/}.
\end{description}

\vspace{1 cm}
\begin{description}
\item[Name] TNoAuxToComp
\item[Task] Vacuous rule, to let structures which do not need inversion pass 
this transformation class.
\item[File] english:TC\_AuxToComp.mrule (mrules11.mrule, end)
\item[Semantics] --
\item[Example] Who did go away, how to solve this, I do walk in the garden
\item[Remarks]
\end{description}

\end{description}

\newpage
\subsection{TC\_CanNegIncorp}
\begin{description}
\item[Kind] Optional Transformation Class, preceded by Obligatory Filter
\item[Task] To glue the negation {\em n\`{o}t\/} to the modal {\em can\/}, 
giving {\em cannot\/},
in case they are adjacent (only when the sentence is not a question). The 
transformation class is ordered crucially after TC\_AuxToComp, because 
structures like {\em Cannot he come?\/} must be prevented. The transformation 
class has not been written yet, since the glue rules do not have a separate 
{\em n\`{o}t\/} yet; {\em can\/} + GLUE + {\em not\/} will give {\em 
can't\/}\footnote{This had been remedied by the time this document was 
approved. The glue rule was added, and the current transformation class was 
written. The transformation is called {\em TCanNegIncorp\/}, and the prefilter 
is 
{\em FPreCanNegIncorp\/}. They can be found in file {\bf 
english:TC\_AuxToComp}, which is {\em mrules11.mrule\/}. In generation, both 
the glued form and the normal form are produced, since the transformation is 
fully reversible and optional. This must still be adapted.}.

\end{description}

\newpage
\subsection{TC\_DoDeletion}
\begin{description}
\item[Kind] Obligatory Transformation Class
\item[Task] To delete either the auxiliary {\em do\/} or the abstract POS, 
where needed. This transformation class is not mentioned in doc.\ 150, but that 
probably is caused by a copying error (there is mention of a 
TC\_ConjunctionDeletion, which is a Dutch class). Also, TC\_DoDeletion {\em is 
\/} mentioned in doc.\ 153.

\vspace{1 cm}
\begin{description}
\item[Name] TDoDeletion
\item[Task] To delete the auxiliary {\em do\/} in case it is adjacent to the 
main verb (although certain adverbs may intervene). Person, number, modus and 
tense of {\em do\/} are copied to the main verb. In analysis, {\em do\/} is 
inserted if no operator is present while the sentence is finite.
\item[File] english:TC\_DoDeletion.mrule (mrules82.mrule, end)
\item[Semantics] --
\item[Example] she does (almost) [make the day begin] $\rightarrow$ she 
(almost) [makes the day begin]
\item[Remarks] Because this rule and TPosDeletion are alternatives, it is 
impossible to delete both POS and {\em do\/}. Since the current rule does not 
work if there is a POS intervening, {\em do\/} will remain in the structure.
\end{description}

\vspace{1 cm}
\begin{description}
\item[Name] TPosDeletion
\item[Task] To delete the abstract category POS. In analysis, POS is introduced 
only for those cases where there is a {\em do\/} which cannot be explained 
otherwise.
\item[File] english:TC\_DoDeletion.mrule (mrules82.mrule, end)
\item[Semantics] --
\item[Example] He did POS go to church $\rightarrow$ He did go to church\\
(Generation only:) He can POS come $\rightarrow$ He can come
\item[Remarks] See the remark above under TDoDeletion
\end{description}

\vspace{1 cm}
\begin{description}
\item[Name] TNoDeletion1
\item[Task] To let SENTENCEs in which there is no POS or {\em do\/} to delete 
pass this transformation class
\item[File] english:TC\_DoDeletion.mrule (mrules82.mrule, end)
\item[Semantics] --
\item[Example] He has never seen me, Didn't you know
\item[Remarks]
\end{description}

\vspace{1 cm}
\begin{description}
\item[Name] TNoDeletion2
\item[Task] To let other structures than SENTENCEs (i.e.\ NPs, PREPPs and 
CLOSEDVERBPPROPs) pass this transformation class.
\item[File] english:TC\_DoDeletion.mrule (mrules82.mrule, end)
\item[Semantics] --
\item[Example] (without) (my) knowing this
\item[Remarks] It is assumed that these structures (derived from infinite 
sentences) have neither {\em do\/} nor POS.
\end{description}

\end{description}

\newpage
\subsection{TC\_VPDeletion}
\begin{description}
\item[Kind] Obligatory Transformation Class
\item[Task] To prune the VERBP node in case it does not dominate a head/VERB 
(this only occurs in case of the copula {\em be\/}, and, in analysis, for 
questions with the main verb {\em have\/}; see TDoBeDeletion)\footnote{By the 
time this document was approved, the rule had been changed to delete the VERBP 
only in case it does not dominate anything anymore, i.e.\ when the predicate 
going with the copula {\em be \/} is wh-shifted, as in {\em How ill is he?, 
Where are you?\/}.
This is easier for the Surface Parser.}.
This transformation is needed for efficiency of the Surface Parser only. The 
class was not mentioned in doc.\ 150.

\vspace{1 cm}
\begin{description}
\item[Name] TVPDeletion
\item[Task] To prune the VERBP node in case it does not dominate a head/VERB.
\item[File] english:TC\_VPDeletion.mrule (mrules10.mrule)
\item[Semantics] --
\item[Example] He is [ill] $\rightarrow$ he is ill; (in analysis only:) Has he 
[a book] $\rightarrow$ Has he a book
\item[Remarks]
\end{description}

\vspace{1 cm}
\begin{description}
\item[Name] TNoVPDeletion
\item[Task] To let structures in which the VERBP cannot be deleted pass this 
transformation class
\item[File] english:TC\_VPDeletion.mrule (mrules10.mrule)
\item[Semantics]
\item[Example] He bought a book, x1 to be ill, etc.
\item[Remarks]
\end{description}

\end{description}

\newpage
\subsection{RC\_Punc}
\begin{description}
\item[Kind] Obligatory Rule Class
\item[Task] To provide a main sentence with the required (final) punctuation 
marks. There are no conditions on co-occurrence of punctuation marks yet, i.e.\ 
a punctuation mark will be added irrespective of whether the sentence already 
had a final punctuation mark. The correct conditions on co-occurrence still 
have to be formulated.

This rule class was not mentioned explicitly in doc.\ 150.

\vspace{1 cm}
\begin{description}
\item[Name] RNoPunc
\item[Task] Default rule, to let structures which do not need a final 
punctuation mark pass this rule class (i.e.\ for subordinate SENTENCEs, and for 
NPs, PREPPs and CLOSEDVERBPPROPs)
\item[File] english:RC\_Punc.mrule (begin of mrules9.mrule)
\item[Semantics]
\item[Example] that he left, (my) leaving the house, for you to go there, etc.
\item[Remarks]
\end{description}

\vspace{1 cm}
\begin{description}
\item[Name] RQuMark
\item[Task] To introduce a question mark for main sentence questions. In 
analysis, a question without question mark is also accepted.
\item[File] english:RC\_Punc.mrule (begin of mrules9.mrule)
\item[Semantics]
\item[Example] Did you see her $\rightarrow$ Did you see her ?
\item[Remarks]
\end{description}

\vspace{1 cm}
\begin{description}
\item[Name] RExclam
\item[Task] To introduce an exclamation mark for a (main sentence) imperative. 
The rule is also used to introduce a period for {\em Lets-mood\/} sentences. 
This is needed because in Dutch, {\em Laten-mood\/} sentences always takes an 
exclamation mark, and the mapping of punctuation rules in transfer is only 
one-to-one.
In analysis, an imperative or {\em lets-mood\/} without punctuation mark is 
also accepted, and a {\em letsmood\/} with an exclamation mark is accepted too.
\item[File] english:RC\_Punc.mrule (begin of mrules9.mrule)
\item[Semantics]
\item[Example] Go home $\rightarrow$ Go home !\\
Let's see where she is $\leftarrow$ Let's see where she is.
\item[Remarks]
\end{description}

\vspace{1 cm}
\begin{description}
\item[Name] RPeriod
\item[Task] To introduce a period at the end of a main sentence declarative.
In analysis, a declarative sentence without punctuation mark is also accepted.
\item[File] english:RC\_Punc.mrule (begin of mrules9.mrule)
\item[Semantics]
\item[Example] That was enough for today $\rightarrow$ That was enough for 
today.
\item[Remarks] Periods going with abbreviations might interfere with this rule. 
No solution has been found for that yet. Also, periods occurring within 
quotation marks cannot be handled yet.
\end{description}

\end{description}

\newpage
\subsection{TC\_CommaIncorp}

\begin{description}
\item[Kind] Optional Transformation Class, surrounded by two Obligatory Filters
\item[Task] To separate the comma coming at the end of a (leftdislocrel) 
sentence or PREPP from it and place it outside the sentence, directly under the 
higher sentence. In analysis, the 
comma is incorporated in the embedded sentence. This transformation class is 
needed for 
analysis, because the Surface Parser cannot put the comma in the embedded 
sentence, and the M-grammar may not be able to deal with a `stray' comma 
(which only is 
deleted in the ConjSentSubst rules). Note that at present, a
comma is generated for leftdislocrel adverbial sentences in the proposition 
substitution rules. 

The transformation class was not foreseen in doc.\ 150.

\vspace{1 cm}
\begin{description}
\item[Name] FPreCommaIncorp
\item[Task] To assure the proper application of the optional TCommaIncorp: in 
analysis, there may be no `stray' comma left in the higher sentence.
\item[File] english:RC\_Punc.mrule (end of mrules9.mrule)
\item[Example] 
\item[Remarks]
\end{description}

\vspace{1 cm}
\begin{description}
\item[Name] TCommaIncorp
\item[Task] To separate the comma coming at the end of a (leftdislocrel) 
sentence or PREPP from it and place it outside the sentence, directly under the 
higher sentence. In analysis, the 
comma is incorporated in the embedded sentence. 
\item[File] english:RC\_Punc.mrule (end of mrules9.mrule)
\item[Semantics] --
\item[Example] [Before he left,] he closed the door. $\rightarrow$ [Before he 
left], he closed the door.
\item[Remarks]
\end{description}

\vspace{1 cm}
\begin{description}
\item[Name] FPostCommaIncorp
\item[Task] To assure the proper application of the optional TCommaIncorp in 
generation: no structures with an incorporated comma are allowed. (This is to 
prevent two paths from going to Leaves, one with and one without an 
incorporated comma).
\item[File] english:RC\_Punc.mrule (end of mrules9.mrule)
\item[Example] 
\item[Remarks]
\end{description}

\end{description}

\end{document}
