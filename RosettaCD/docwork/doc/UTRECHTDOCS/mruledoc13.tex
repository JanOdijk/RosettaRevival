\documentstyle{Rosetta}
\begin{document}
   \RosTopic{Rosetta3.doc.Mrules.English}
   \RosTitle{Rosetta3 English M-rules: ADVPformation}
   \RosAuthor{Margreet Sanders}
   \RosDocNr{391}
   \RosDate{\today}
   \RosStatus{concept}
   \RosSupersedes{-}
   \RosDistribution{Project}
   \RosClearance{Project}
   \RosKeywords{English, documentation, Mrules, ADVPformation}
   \MakeRosTitle
%
%

\section{ADVPFormation}
PREPPs and ADVPs are used in Rosetta3 as non-argument locatives and temporals.
Contrary to what was assumed in the definition phase of Rosetta3 (see e.g.\ 
doc.\ 150, {\em Subgrammars of English\/}), it is no longer considered 
necessary to introduce these categories as full PREPPPROPs or ADVPPROPs that 
have to be pruned. For a full discussion of this subject, see the document on 
the Treatment of Adverbs by Jan Odijk (to appear).

Since the ADVPformation subgrammar need NOT be isomorphic with other XPPROP or 
sentence grammars
(because it is impossible anyway in Rosetta3 to translate a simple 
phrase (PREPP or ADVP) into something with a propositional structure),
the exact contents of the ADVPFormation subgrammar are mainly determined 
by the requirements posed by the sentence the ADVP is 
substituted in. Note that the 
subgrammar is isomorphic with the PREPPformation subgrammar, since PREPs and 
ADVs may have to translate into each other. The current documentation is in 
fact an adapted copy of the documentation on the PREPPformation subgrammar. 

Input to the grammar is a SUBADV, 
coming from the AdvDerivation subgrammar (see doc.\ 316, {\em Rosetta3 English 
M-rules: Derivation Subgrammars\/}). For testing purposes, the ADVP formed in 
the current subgrammar can also be given a top node UTT in the Utterance 
grammar, to allow translation of bare ADVPs. 
In doc.\ 150, {\em Subgrammars of English\/}, where the plans for the general 
lay-out of the translation system were presented, the ADVPformation grammar was 
not explained in any detail. Instead, reference was made to an earlier 
proposal, in doc.\ 103 by Franciska de Jong, {\em The organisation of the 
ADVP subgrammar\/}. Since in that document no clear distinction is made between 
the ADVP subgrammar and the ADVPPROP formation subgrammar, I will not compare 
the original plans and the current implementation. Instead, similarities and 
discrepancies between the current subgrammar and the ADVPPROP subgrammars or 
the PREPPformation grammar have been indicated. 

As all subgrammars, the ADVPformation subgrammar consists of 
a number of rule classes and transformation classes. A rule class in its turn
consists of a number of rules and a transformation class of a number of 
transformations. The relative ordering of the rules and transformations in the
(sub)grammar is indicated by a {\em control expression}. A summary of this
control expression (i.e.\ a listing of the ordering of the rule classes, 
without explicit mentioning of the rules themselves) is also included here, 
and the initial (= head), import and export categories are given. 

In the section on the rules and transformations, only the rule names are given, 
but not the exact rule formulation. Most rules are an adapted copy of the rules 
in the three ADVPPROP subgrammars.
For every rule, an example is given. If it is uncertain whether the example is 
correct (either 
because it may not be an example of the phenomenon in question, or because it 
may not be correct English), it is preceded by a question mark. Note that all 
explanation of rules and transformations is given from a generative viewpoint
only, unless explicitly stated otherwise. The semantics of the rules 
has been left unspecified in the current documentation, since it is not at all 
clear.

Finally note that the rules described in this document have NOT been tested 
properly. English analysis is not possible yet (there is no Surface Parser), and 
English generation has only been tested in as far as the construction was the 
translation of a Dutch sentence to be tested.

\newpage
\section{Subgrammar Specification}
The subgrammar definition can be found in the file which also contains all the 
rules of this subgrammar, {\bf AdvpSubgrammars.mrule}, which is 
{\em mrules84.mrule\/}.

\begin{verbatim}
%SUBGRAMMAR ADVPFormation


   ( RC_StartAdvpRules )
.  ( TC_AdvpPatterns )
.  ( RC_AdvpSuperdeixis )
.  { RC_AdvpSubst }


\end{verbatim}

\begin{description}
  \item[Head]  SUBADV  \ \ \ \ FROM (ADVderivation)
  \item[Export] ADVP
  \item[Import] NPVAR, NP
\end{description}

\newpage
\section{Rules and Transformations}

\subsection{RC\_StartAdvpRules}
\begin{description}
\item[Kind] Obligatory Rule Class
\item[Task] To provide a SUBADV with its correct number of ADVP argument 
variables and build an ADVP above it. 

\vspace{1 cm}
\begin{description}
\item[Name] RStartAdvp0
\item[Task] To build an ADVP for an intransitive SUBADV, so with no arguments 
in the ADVP.
\item[File] english:AdvpSubgrammars.mrule (mrules84.mrule)
\item[Semantics]
\item[Example] yesterday $\rightarrow$ yesterday (We sailed yesterday)
\item[Remarks]
\end{description}

\vspace{1 cm}
\begin{description}
\item[Name] RStartAdvp1
\item[Task] To build an ADVP for a transitive SUBADV, so with one argument in 
the ADVP. The argument may only be an NPVAR (usu.\ a measure phrase; it is 
assumed that there are NP measure phrases, although elsewhere they are assumed 
NPPROPs). For further comment, see TAdvpPattern1 below.
\item[File] english:AdvpSubgrammars (mrules84.mrule)
\item[Semantics]
\item[Example] ago + x1 $\rightarrow$ ago x1 (Three hours ago)
\item[Remarks]
\end{description}

\end{description}

\newpage
\subsection{TC\_AdvpPatterns}
\begin{description}
\item[Kind] Obligatory Transformation Class
\item[Task] To check the category of the argument variable and the relation 
it bears in the ADVP against the Advpatterns specified for the ADV. The {\bf 
advpatternefs} attribute of the ADVP is set at the value 
actually chosen from the advpatterns of the ADV.

\vspace{1 cm}
\begin{description}
\item[Name] TAdvpPattern0
\item[Task] To let ADVs that have no ADVP argument pass this transformation 
class
\item[File] english:AdvpSubgrammars (mrules84.mrule)
\item[Semantics]
\item[Example] yesterday (We sailed yesterday)
\item[Remarks]
\end{description}

\vspace{1 cm}
\begin{description}
\item[Name] TAdvpPattern1
\item[Task] To specify the relation name (which must always be {\em modrel\/})
and the category (NPVAR) of the argument in the ADVP. Also, the {\em case\/} of 
the argument is assigned (always {\em [accusative]\/}); there is no 
separate Case-assignment Transformation.
\item[File] english:AdvpSubgrammars (mrules84.mrule)
\item[Semantics]
\item[Examples] ago x1 $\rightarrow$ x1 ago (Three hours ago)
\item[Remarks] There is no separate rule class for modification of ADVs in the 
ADVPformation subgrammar (cf.\ the PREPPformation grammar, which does have 
modification). This makes both subgrammars only partially isomorphic.
\end{description}

\end{description}

\newpage
\subsection{RC\_AdvpSuperdeixis}
\begin{description}
\item[Kind] Obligatory Rule Class
\item[Task] To provide the ADVP with a value for the attribute {\bf 
superdeixis}. The rule uses a 
parameter, {\em super\/}, to determine whether a present or past superdeixis 
value should be assigned.

\vspace{1 cm}
\begin{description}
\item[Name] RAdvpSuperdeixis
\item[Task] see above
\item[File] english:AdvpSubgrammars (mrules84.mrule)
\item[Semantics]
\item[Example] [yesterday]$_{omegasuperdeixis}$ $\rightarrow$ 
[yesterday]$_{pastsuperdeixis}$ (We sailed yesterday)
\item[Remarks]
\end{description}

\end{description}

\newpage
\subsection{RC\_AdvpSubst}
\begin{description}
\item[Kind] Iterative Rule Class
\item[Task] To substitute non-sentential expressions for their variable. There 
is no condition on substitution order (there is only one VAR anyway).
Conditions on genericity and superdeixis are the same as in the comparable 
class in the CLAUSEtoSENTENCE subgrammar (see doc.\ 370, p.\ 14). Reflexives 
etc.\ have not yet been excluded from the current rule explicitly. 

The rule uses the system parameter LEVEL to check whether the index of the 
variable is consistent with (the level of) the variable that is to be 
substituted for according to the rule parameter.

\vspace{1 cm}
\begin{description}
\item[Name] RAdvpSubst1
\item[Task] To substitute a non-generic NP for its modrel VAR
\item[File] english:AdvpSubgrammars (mrules84.mrule)
\item[Semantics]
\item[Example] [x1 ago] + three hours $\rightarrow$ [three hours ago]
\item[Remarks]
\end{description}

\end{description}


\end{document}

