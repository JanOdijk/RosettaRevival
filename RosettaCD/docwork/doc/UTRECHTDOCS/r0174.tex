\documentstyle{Rosetta}
\begin{document}
   \RosTopic{General}
   \RosTitle{Description Van Dale dictionary N~--~N}
   \RosAuthor{Harm Smit, Jeroen Medema}
   \RosDocNr{R174}
   \RosDate{July 10, 1987}
   \RosStatus{concept}
   \RosSupersedes{-}
   \RosDistribution{Project}
   \RosClearance{Project}
   \RosKeywords{N~--~N, Dictionary, Van Dale}
   \MakeRosTitle
%
%

\hyphenation{pijl-ver-gift pijl-gif lij-ke-gif lij-ke-gift dier-ge-lijk 
             der-ge-lijk zijn ko-len-mijn kool-mijn be-zit-te-lijk af-rij-den
             vij-gen-mand vij-gen-mat prijs-ver-schil}
\newcommand{\male}{
   \setlength{\unitlength}{.7ex}
   \begin{picture}(5,4)(1,1)
   \thicklines
   \put(2,2){\circle{2}}
   \thinlines
   \put(3.4,3.4){\vector(1,1){1.5}}
   \put(3.1,3.1){\circle*{0.1}}
   \put(3.2,3.2){\circle*{0.1}}
   \put(3.3,3.3){\circle*{0.1}}
   \put(3.4,3.4){\circle*{0.1}}
   \put(3.5,3.5){\circle*{0.1}}
   \put(3.6,3.6){\circle*{0.1}}
   \put(3.7,3.7){\circle*{0.1}}
   \put(3.8,3.8){\circle*{0.1}}
   \put(3.9,3.9){\circle*{0.1}}
   \put(4,4){\circle*{0.1}}
   \end{picture}
                  }

\newcommand{\female}{
   \setlength{\unitlength}{.7ex}
   \begin{picture}(5,4)(1,1)
   \thicklines
   \put(2,3.1){\circle{2}}
   \put(2,0){\line(0,1){2}}
   \put(1,1){\line(1,0){2}}
   \thinlines
   \end{picture}
                    }

\section{Introduction}
The dictionaries for Dutch and English in the Rosetta machine translation 
project are developed by using files of `Van Dale Lexicografie'.
We have disposal of three different files, the N~--~N (Dutch - Dutch), the 
N~--~E
(Dutch - English) and the E~--~N (English - Dutch). These files correspond to 
three volumes of a series of printed dictionaries, published by Van Dale.
Because the Van Dale files are not intended to be used for machine translation,
they cannot be applied directly in Rosetta, but have to be adapted.
Thus, the information of the files can be used to make dictionaries which are 
fitted for Rosetta. 

In this document, an overview will be given of the type of 
information that is contained by one of these files, the N~--~N, and of the way
this information is structured. Also, some of the information has been 
extracted from the file, and, because this might be interesting for the
linguistic work in Rosetta, part of this information has been included in this 
document (see section 6).
\newpage
\section{The structure of the N~--~N file}
\subsection{Different versions of the N~--~N file}
Originally, the N~--~N file was sent on tape, split up in many (over 1000) small
files. Out of these files Joep Rous constructed 26 files, one per letter.
Thus, all words starting with the same letter were in the same file, which made
search processes easier; a disadvantage, however, was that some of these 26
files were very small (the `Q', `X' and `Y'), others, on the other hand, were
very large (the `S', for instance). From these files, Harm Smit has made a 
single file
that contains only the entry word and the main grammatical information. For
some search processes, this file is preferrable, because it can be processed
quite fast due to its limited size. Jeroen Medema made files for all (main) 
word categories (like nouns, verbs, adjectives, pronouns, etc.), which can be 
used when only words of a special category should be analysed. 

These various files can be found in the following directories:
\begin{itemize}
   \item the files made by Joep Rous:

         directory: rosdisk2:[rosetta.vandale.nn.conv]

         files: nna.txt, nnb.txt, nnc.txt, ... ,nnz.txt
   \item the file made by Harm Smit:

         directory: [smit.vandale.entries]

         file: entries.dat 
   \item the files made by Jeroen Medema:

         directory: rosdisk2:[rosetta.vandale.catnn]

         files: adjective.txt, adverb.txt, 
                conjunction.txt, interjection.txt, 
                noun.txt, numeral.txt, preposition.txt, 
                pronoun.txt, reference.txt, 
                sundries.txt, verb.txt.
\end{itemize}
Of course, also other files have been derived from the original ones;
some of these are intermediate versions in the process of developping the
dictionaries for Rosetta. Others contain specific information extracted 
for linguistic investigation.
\newpage
\section{Microstructure in the N~--~N}
The entry is organised in a systematic way: each part of the information is
preceded by a code, and for every type of information, a special code exists.
So, the entry word is always preceded by the entry word code, meanings are
preceded by the meaning code, synonyms by the synonym code, etc. 
Unfortunately,
the way this coding was done has never been put down in some `regular 
expression'. Without such a {\em syntax} it is hard to use the information 
thoroughly, and to be sure of its consistency.
We therefore developed a syntax ourselves.
Some entries had to be adjusted, in order to get them accepted by the syntax 
(without adjusting these entries, the syntax would become extremely complex).
In the meantime, Van Dale developed a syntax too, but this syntax is {\em not}
better than ours.
We presume that
the Van Dale syntax has been developed for a file that -slightly- differs 
from the N~--~N file we got, because it contains codes that are not present in
our file (like: `CV\#'), and, on the other hand, lacks codes like `SS\#', 
`UI\#' and
`UV\#'. Also, the codes are sometimes in the wrong order. Van Dale doesn't
speak of entries that are not accepted by their syntax, but we are sure that 
there will be {\em more} non-accepted entries than there were for our syntax.

In the next sections, 
a list of codes is given (together with an explanation of 
what they stand for and some examples) and the syntax is specified. Also an 
explanation of the {\em meaning} of the syntax is given,
and a list of the entries that had to be adjusted because they were not
accepted. 
\subsection{List of codes}

In this section, all codes mentioned in the syntax will be explained in detail.
In some cases, examples are given; in principle, this is done for the letter 
`J', and, for non-frequent codes, the number of occurences will be mentioned.
The `J' contains circa 0.5 \% of the total N~--~N file.
Of course, these figures should be interpreted carefully: it is possible that 
the words under `J' have been inserted by one and the same person, who might
have a special preference for some codes. Therefore, it is dangerous to expect
all codes in the {\em same} frequency for the other letters. Some codes
(CI, TV, UV, and TR) do {\em not} occur at all under `J'! The global frequency 
of all codes can be found in 5.1.

Van Dale sent us a list of codes in the begin of 1986. This list turned
out to be incomplete: the codes LI and SS were missing. Also, this list
contained a code that was {\em not} in the file: CV. Later, in december 1986,
we got the latest version of the so called `produktieboek' of Van Dale. This
`produktieboek' gives a list of codes that does have LI but still misses SS.
The CV code is also listed in the `produktieboek'.

Here, we give a list of all codes, including the CV code. References to entry
words of the N~--~N are given between double quotes; references to the manual of
the (printed) N~--~N are given by `[GBAW ...]'. On tape, the codes are allways 
placed at the begin of a line, and they consist of two letters and a `\#', 
followed by a {\em tab}. Behind the tab comes the information belonging to the 
particular code.

\begin{description}

   \item [AB] antonym

         This code occurs three times under `J' and therefore is quite
         rare. In the printed version this code is represented by
         `$\Leftrightarrow$' and followed by an antonym of the entry word.
         See ``jongleur''(with the -in our opinion- rather strange antonym
         ``troubadour'') and ``junior''(twice).

   \item [BB] meaning [GBAW 5, esp.\ 5.1]

         This code is followed by the description of the meaning of the entry
         word. Normally, the description consists of more than one word, but it
         also can be a synonym (compare SB). See also XX.  

   \item [CC] numbering examples [GBAW 5.2]

	This code is used to {\em number} the examples, like: `\&.\&', `\&.1',
        `\&.6', `1.\&', `1.1', `2.1', etc. The digit {\em before} the dot refers
        to the category of the word that the entry word is combined with in the 
        example. 
        The digit {\em behind} the dot corresponds with one of the 
        {\em meaning} numbers (compare NN). `\&' is used when there is no
        special combination category or no corresponding meaning number (i.e.
        for idioms). In the printed N~--~N `\&' is represented by `\P'.

   \item [CI] reference to `grammaticaal compendium'

        This code is used to refer to the `gram\-ma\-ti\-caal com\-pen\-di\-um'
        of the
        (prin\-ted) N~--~N. In print, this code looks like: `$\rightarrow$GC'.
        See ``aan\-ha\-lings\-teken''. There are no occurences of this code 
        under `J'. The CI code is always preceded by GI. 

   \item [CR] reference to `grammaticaal compendium', preceded by GR

        Like CI, but preceded by GR instead of GI. The code is rare; it doesn't
        occur under `J'. Examples: ``absoluut'', ``actief'' and ``punt''. 


   \item [CV] reference to `grammaticaal compendium', preceded by GV

        Like CI, but preceded by GV instead of GI. Note: this code does {\em 
        not} occur in our N~--~N file and therefore not in our syntax!
        
   \item [DI] updating / first code of entry

        Every entry of N~--~N {\em starts} with this code. It is followed by `6'
        or `0'. These numbers give information about an updating done by Van 
        Dale. There seems to be no (systematic) difference between words
        marked by `6' and those marked by `0'. 

   \item [FI] frequent word [GBAW 4.4]

        This code says that the entry word is frequent, and it is always
        followed by `f'. In the printed N~--~N, this is represented by:
        `$\langle~f~\rangle$'.

   \item [GI] grammatical information [GBAW 4.5] 

        This code gives the {\em category} of the entry word (which is coded
        as a two digit number), possibly 
        followed by special {\em forms}, such as: plural and diminutive 
        (for nouns),
        comparative and superlative (for adjectives), past tense
        and past participle
        (for verbs). Also, descriptions like: `mv', `g.mv.', `alleen attr.',
        `alleen pred.', etc. 

        For GI in sub-entries, see also GR.

        The following category numbers occur behind GI (between 
        brackets the representation in the printed N~--~N):

   \begin{itemize}
       \item 00 = {\bf abbreviations}, `strange' words, etc.
       \item 08 = neuter {\bf noun}, denoting a male person (`het $\sim$ \male')
         
09 = neuter {\bf noun}, denoting a female person (`het $\sim$ \female')

10 = {\bf noun}

11 = masculine {\bf noun} (`de $\sim$ (m.)')

12 = feminine {\bf noun} (`de $\sim$ (v.)')

13 = neuter {\bf noun} (`het $\sim$')

14 = feminine (masculine) {\bf noun} (`de $\sim$') [GBAW 4.5.1]

15 = masculine and neuter {\bf noun} (`het, de $\sim$ (m.)') 

16 = feminine and neuter {\bf noun} (`het, de $\sim$ (v.)') 

17 = masculine and feminine {\bf noun} (`de $\sim$') 

18 = masculine, feminine and neuter {\bf noun} (`het, de $\sim$') 

19 = feminine (masculine) and neuter {\bf noun} (`het, de $\sim$')

        \item 20 = {\bf adjective} (`bn.')

21 = {\bf adjective} (`bn.')

22 = {\bf adverb} (`bw.')

        \item 30 = infinitive {\bf verb} (`ww.')

31 = intransitive {\bf verb} (`onov.ww.')

32 = transitive {\bf verb} (`ov.ww.')

33 = reflexive {\bf verb} (`wk.ww.')

34 = absolute {\bf verb} 

35 = auxiliary {\bf verb} (`hww')

36 = copulative {\bf verb} (`kww')

37 = impersonal {\bf verb} (`onp.ww.') 

        \item 41 = personal {\bf pronoun} (`pers.vnw.')

42 = demonstrative {\bf pronoun} (`aanw.vnw.')

43 = possessive {\bf pronoun} (`bez.vnw.')

44 = relative {\bf pronoun} (`betr.vnw.')

45 = interrogative {\bf pronoun} (`vr.vnw.')

46 = reflexive {\bf pronoun} (`wdk.vnw.')

47 = reciprocal {\bf pronoun} (`wdg.vnw.')

48 = indefinite {\bf pronoun} (`onb.vnw.')

        \item 50 = {\bf adverbs} (`bw.')
 
        \item 60 = {\bf prepositions} (`vz.')

        \item 70 = {\bf numeral} 

71 = {\bf numeral}

72 = {\bf numeral} (`hoofdtelw.')

73 = {\bf ordinal} (`rangtelw.')

74 = special {\bf numeral}

75 = {\bf article} (`lidw.')

        \item 80 = {\bf cunjunction} (`vw.')
        \item 90 = {\bf interjection} (`tw.')
   \end{itemize}

        Only two verbs belong to number 34, ``onder$'$kruipen'' and 
        ``serreren''
        is unclear why 34 has been assigned to these verbs instead of the far
        more common 31.

        The numbers 71 and 74 are very rare too: 71 occurs only once in the 
        N~--~N file (``tienduizend'') and, in our opinion, is similar to 72; 74
        occurs twice (``hoeveelste'', ``zoveelste'').

        Some of the closed categories (like 40-49, 60, etc.), which are quite 
        small of course,  are listed in section 6.

        Also, one word (``hoezeer'') has a code that is {\em not} in this list: 
        81. In our opinion this must be an error, because the result of this code
        in the printed N~--~N was `8' instead of something like: `vw.'.

        There is a sub-entry of a word without grammatical code number; see:
        ``wat$^{2}$'', sub-entry IV.

        Often, a {\em combination} of category numbers is given; many adjectives
        for instance, have the combination `21,22', because they can act as
        adverb too. Also, some adverbs that are {\em not} `derived' from 
        adjectives have been put under the main category 2 (with `22'; example:
        ``vaak''); this is
        rather strange because `real' adverbs should be listed under main  
        category 5.

        Under `J', the following category numbers and combinations of category
        numbers occurred ({\em rare} category numbers are mentioned together 
        with the number of occurences under `J'):

\begin{verbatim}
        00
        08(3x), 10, 11, 12, 13, 14, 15, 17, 18(1x), 19(3x)
        20, 21(4x), 21,22, 22(1x)
        30(2x), 31, 31,32, 32
        41, 43(3x), 48(1x)
        50(2x)
        60
        90
\end{verbatim}

   \item [GR] general grammatical information for entries with sub-entries

        This code gives the same information as GI, but it occurs only in
        entries with sub-entries (see 3.2).
        The category behind GR is always coded as a one digit number that refers
        to a {\em main} category, like:
        \begin{itemize}

          \item 1  = {\bf noun} (`zn.')

          \item 2  = {\bf adjective} and {\bf adverb}

          \item 3  = {\bf verb} (`ww.')

          \item 4  = {\bf pronoun} (`vnw.')

          \item 7  = {\bf numerals}
        \end{itemize}
        In all following sub-entries the code GI occurs with grammatical
        information that holds for the particular subsection only; here
        we find a two-digit number of which the first corresponds to the
        (one) digit that follows GR (compare: GI).
        Example: `1' stands for the main category {\em noun}, and it can be
        combined with `11', `12', ..., `19' as category of a sub-entry.
        

   \item [GV] grammatical information for entries with morphological variants 

        Like GI, but preceded by VV (somewhere higher in the entry, see section
        3.2, syntax of the entry). Used when the morphological variant differs
        grammatically from the entry word (the morphological variant might have
        another gender, for instance).

   \item [I\,I] code for entry word [GBAW 4.1]

        This code precedes the entry word, which is marked for the stress (by
        `{\tt +;}'). A particular entry word can have more stresses. The entry word 
        is built out of the following characters:
     \begin{itemize}
        \item lower case and upper case letters; like: ``apache'', 
              ``Arabier'', ``A-omroep'', ``A.P.'', ``ALGOL'', ``alibi-Jet''.
         \item dots, apostrophs, hyphens, and spaces: ``c.q.'', ``aow'er'', 
              ``a-literair'', ``femme fatale'', etc.
        \item accents, such as:

              {\bf {\tt "}i} = \"{\i} (in: ``na\"{\i}ef'');

              {\bf \`\,a} = \`{a} (in: ``\`{a} propos'');

              {\bf \~\,e} = \'{e} (in: ``caf\'{e}'');

              {\bf +c8} = \c{c} (in: ``gar\c{c}on''); 

              {\bf +a4} = \aa (in: ``\aa ngstr\"{o}m-eenheid''); 

              {\bf +n1} = \~{n} (in: ``se\~{n}or'').
              
         \item the following special characters:

              {\bf +\%a} = a$^\circ$; 

              {\bf f[ps]o[pr]} = f$^\circ$;

              and, as already mentioned, for stress: {\bf {\tt +;}}.

         \item additional charectars are: `{\bf (}' and: `{\bf )}' for some
              kind of spelling variants: ``AB(N)'', ``bl(z).'', and: `{\bf ,}',
              that is used when more words (or even word-parts) are put
              together behind I\,I: ``kopi-, komple-'', or: ``gruizelementen, 
              gruizementen''.

     \end{itemize}
          The longest word in the file is: 
          ``ac{\tt +;}countants-admini{\tt +;}stratieconsulent''
          (38 characters, of which 4 
          for the accents). This is not the longest string preceded by `I\,I':
          there are cases of more words or word-parts. The longest is:
          ``konkl-, konko-, ..... '', which is 47 characters.

   \item [LB] explanation about the way of use, context, situation, etc.
              [GBAW 4.7]
 
         Like all codes starting with `L', LB explains the way the entry word
         should be used, whether or not a word is {\em slang}, etc. The codes
         starting with `L' are often followed by things like: `fig.', `inf.', 
         `euf.', `jur.', `kind.', `zelfst.', `iron.', etc. From the syntax 
         (see section 3.2) it is clear what the {\em scope} of a particular
         `L'-code
         can be. LB is quite rare, there are no occurences under `J'.

   \item [LC] like LB, but related to examples in the entry

   \item [LI] like LB, but related to the entry word itself

   \item [LN] like LB, but related to a meaning
 
         LN is often followed by more or less grammatical information, like:
         `als onbepaald bezittelijk vnw.', `2e persoon enkelvoud ....', 
         `alleen met hoofdletter', `uitdrukking van bevestiging, toestemming 
         of inwilliging', etc. As will be clear from the syntax, LN does not
         necessarily need to be followed by a BB-code (see ``ja''). 

   \item [LR] like LB, but related to a sub-entry. 

         This code is quite rare, only one occurence under `J'.

   \item [LX] like LB, but related to the meaning of examples (see: XX)

         This code is quite rare, only eight occurences under `J'.

   \item [NI] entry number for homonyms [GBAW 4.1.5]

         This code is used to number homonyms that have separate entries.
         Homonyms get separate entries if they differ in (main) category
         (like: ``ja$^{1}$'' and ``ja$^{2}$'', and: ``jammer$^{1}$'', 
         ``jammer$^{2}$'', and ``jammer$^{3}$''), or if they differ in 
         pronounciation (like: ``jus$^{1}$'', 
         and ``jus$^{2}$''). Note that a difference in {\em stress}
         leads to different
         entries which are {\em not} numbered (example: ``door$'$lopen'' and 
         ``$'$doorlopen''. Homonyms belonging to the same {\em main} category,
         which are pronounced the same, normally share one entry (example: 
         ``jacht''). There are some exceptions, however, like ``vorst$^{1}$'' 
         and
         ``vorst$^{2}$''; this is probably because the form ``vorstin'' belongs
         only to ``vorst$^{2}$'' (see also VV).

   \item [NN] meaning number

        This code is always followed by a meaning number, like: `0.1', `0.2', 
        `0.3', ... . The example numbers (see CC) refer to meaning numbers.

   \item [RR] sub-entry number [GBAW 4.1.5]

        This code is always followed by a roman number, that is used to number
        the sub-entries (`I', `II', `III', ...). A word has the same {\em main}
        category in all sub-entries (See GR).

   \item [SB] meaning related word (synonym) [GBAW 5.1].

         In the printed version this code is represented by
         `$\Rightarrow$' and it is followed by a {\em meaning related word}
         (more or less a synonym) of the entry word.

   \item [SI] spelling variant [GBAW 4.3.1]

        This code is used for alternative spelling of the entry word,
        and is followed by the alternative form. This may be a word, like
        ``Jesus''(see: ``Jezus$^{1}$'' and ``Jezus$^{2}$'') or part of it, like
        ``jacobs-''(see: ``jakobsladder'') or \mbox{``-diktie-''}(see: 
        ``jurisdictiegeschil''). In principle, the form preceded by I\,I is
        the preferred spelling (See VI). In the printed N~--~N, spelling 
        variants 
        are between ({\em round}) brackets; morphological variants (see VV) are 
        {\em not} between brackets.

   \item [SS] has been replaced by SB

        This code was found in very few entries (only one occurence under `J',
        ``jazegger''), and all
        entries had in common that their entry word lacked the stress-symbol
        {\bf {\tt +;}}. None of these words was listed in the printed N~--~N. The 
        function of this code was -in our opinion- equal to SB, so
        it has been replaced by that code in order to avoid unnecessarily
        complexity of the syntax. 
        
   \item [SV] spelling variant behind VV (see: SI)

        Very rare, no occurences under `J'. Example: ``athermaan''.

   \item [TI] reference to other language [GBAW 4.3.5]

        This code refers to other languages than Dutch, and will be followed by
        abbreviations like: `Am.', `Eng', `Eng.,Am', `Fr', `Lat', `Sp', `Hgd',
        `Jap', etc. The code is always preceded in the entry by GI.

   \item [TR] reference to other language, behind GR (see: TI)

        No occurences under `J'. Example: ``continu''.

   \item [TV] reference to other language, behind GV (see: TI)

        No occurences under `J'. Example: ``accoucheur''.

   \item [UI] change of stress pattern

        The UI-code indicates a (possible) change of the stress pattern.
        Under `J' it occurs once, and will probably be quite rare. See:
        ``jongstleden''.

   \item [UV] change of stress pattern, behind VV  

        Like UI, but for morphological variants. No occurences under `J'.
        Example: ``kruisgewijs''.

   \item [VC] Example [GBAW 5.2]
 
        This code is followed by a sample sentence. Every (group of) sample
        sentences is numbered by a CC-code, and thus corresponds to a 
        meaning (numbered by a NN-code). Some examples are not related to 
        a meaning; in general these sentences are idiomatic.

   \item [VI] reference to another entry [GBAW 4.3.4]

        This code is followed by the entry word of an {\em other} entry. In
        general, this entry word is a preferred spelling variant (see: SI).
        In the printed N~--~N this code is represented as $\rightarrow$
        Example: ``ju jitsu'' has a reference to ``jioe-jitsoe''.

   \item [VV] morphological variant [GBAW 4.3.2, 4.5.1]

        This code is followed by a morphological variant. Often, this is the
        form that refers to a female person: ``jubilaris'' has the
        variant ``jubilaresse''. The variant is {\em not} listed separate, 
        unless
        it differs basically in meaning. Therefore ``jubilaresse'' is not listed
        separately. ``Vorstin'' is a separate entry, next to ``vorst'' with the
        variant ``vorstin'', because it has the meaning {\em wife of a king}.
        A man who marries a queen never becomes a king, so it is clear that
        there is a difference: some queens reign, but {\em all} kings do.

        The second meaning of ``vorstin'', {\em female king}, is a bit 
        contradictory; here the N~--~N seems to be inconsistent. More
        inconsistencies of this kind exist; compare: ``jood'' and ``jodin''.

        Sometimes, the difference in meaning is more clear, compare ``zwemmer''
        and ``zwemster''.

   \item [XX] meaning of sample sentence [GBAW 5.2]

        This code is always followed by the meaning description of a sample
        sentence. It is used in particular when the meaning of the sample 
        sentence cannot be derived from (one of) the meanings of the entry word.
        This is especially the case when the example is an idiom.

   \item [ZI] reference to a proverb
        
        This code refers to a proverb by means of a number. The numbers 
        correspond to the numbers in the proverb-list in the printed N~--~N.
\end{description}
\subsection{The syntax of the entry of N~--~N}

The syntax of the entry of the N~--~N file is specified in table 1.

\begin{table}[htb]
\centering
\begin{tabular}{lcl} 
 Entry  &=& DI I\,I [ NI ]\\
        & & ( VI\\
        & & $|$ [ SI ]\\
        & & \hspace{2mm} ( \{ VV [ SV ] [ UV ] $|$ UI ) \} \\
        & & \hspace{2mm} [ FI ]\\
        & & \hspace{2mm} ( Head Rom Descr \{ Rom Descr \}+ \\
        & & \hspace{2mm} $|$ NoRom Descr )) ,\\ 
        & & \\
 Head   &=& GR [ TR ] [ CR ] [ LR ] \{ ZI \} , \\ 
        & & \\
 Rom    &=& RR GI [ TI ] [ CI ] \{ LI \} ,\\ 
        & & \\
 NoRom  &=& GI [ TI ]\\
        & & \{ \{ VV [ SV ] [ UV ] \}+ GV [ TV ] \}\\
        & & [ CI ] \{ LI \} \{ ZI \} ,\\ 
        & & \\
 Descr  &=& Nul [ Phrase ] $|$ Phrase ,\\ 
        & & \\
 Nul    &=& \{ NN\\
        & & \hspace{2mm} ( \{ LN \}+ [ BB ] $|$ BB )\\
        & & \hspace{2mm} \{ SB \} \{ AB \} \\
        & & \hspace{2mm} \{ [ LB ] BB \{ SB \} \{ AB \} \} \\
        & & \}+ ,\\ 
        & & \\
 Phrase &=& \{ CC\\
        & & \hspace{2mm} ( \{ LC \}+ [ VC ] $|$ VC ) \\
        & & \hspace{2mm} \{ LX $|$ XX \} \\
        & & \}+ . \\ 
\end{tabular} 
\caption{{\em syntax of the entry}}
\end{table}

\begin{verbatim}

Explanation:

A B     = A followed by B
A | B   = A or B
[ A ]   = A or nothing
{ A }   = zero or more times A
{ A }+  = A { A }

\end{verbatim}

Note: The position of `TR' is unknown: it can also {\em follow} `CR', 
because `TR' and `CR' never appear together in an entry.

The following words had an entry that was {\em not} accepted by the syntax,
and therefore have been adjusted:

\begin{table}[htb]
\begin{tabular}{lcl}
{\tt +;}basketbal        &-& contained an empty line \\
be{\tt +;}heersen        &-& had only one sub-entry, with roman `I' \\
{\tt +;}bieden           &-& space before FI-code \\

chan{\tt +;}teur         &-& contained a TI-code at a wrong place \\

decora{\tt +;}teur       &-& contained a TI-code at a wrong place \\
draadpotigen       &-& contained a SS-code \\
dubbel{\tt +;}blank      &-& had a NN behind a CC \\

enve{\tt +;}lop          &-& contained an empty line \\

geholpen           &-& contained a SS-code \\

{\tt +;}hanger           &-& had a NN behind a CC \\

ijslandisme        &-& contained a SS-code \\
{\tt +;}intekenlijst     &-& had a `{\bf -}' before the GI-code \\
{\tt +;}intekenprijs     &-& had a `{\bf -}' before the GI-code \\

jazegger           &-& contained a SS-code \\

{\tt +;}kleedgeld        &-& had a `{\bf -}' before the GI-code \\

{\tt +;}leugenachtigheid &-& had a `{\bf -}' before the GI-code \\
{\tt +;}leuningstoel     &-& had a `{\bf -}' before the GI-code \\

monotoom           &-& contained a SS-code \\

{\tt +;}notitie          &-& had a `@' before a VC-code \\

portkaraf          &-& contained a SS-code and LI instead of GI \\

roman{\tt +;}cier        &-& contained a TI-code at a wrong place \\
rook               &-& had only one sub-entry, with roman `II' \\

strikluier         &-& contained a SS-code \\

{\tt +;}tonica           &-& no GI-code, is given twice in the file! \\

uvarine            &-& contained a SS-code \\

ver{\tt +;}duisteren     &-& contained an empty line \\

waar               &-& space before RR, sub-entry II, waar$^{2}$ \\

zonvast            &-& contained a SS-code \\
\end{tabular}
\end{table}

\subsection{The `meaning' of the syntax}
The entries in the file are separated by empty lines; these lines are not coded.
the first code is `DI' which gives information about the source of the entry 
word (this code was {\em skipped} in the printing proces of the printed 
N - N). Next is `I\,I' which gives the entry word. `NI' says that there is another
entry word corresponding to the same string. Now we have the choice between 
reference (`VI') to another word (which implies that the entry ends) or give 
one or more meanings, that can be preceded by a spelling variant (`SI'), 
and either information about change of accent (`UI'), or one or more 
morphological variants (which, in their turn, can be accompanied by spelling 
variant,
information about change of accent, etc.) and the frequency code (`FI'), that
says whether the entry word is frequent (= belongs to the 4500 most-frequent 
words) or not.

In the next step, the choice between one -main- entry (`NoRom')
and (two or more) sub-entries (`Rom') is
made. Entries with sub-entries are marked by `GR' instead of `GI', and by 
`RR' (for each sub-entry) which is
followed by a roman number (the various sub-entries in an entry are numbered 
by roman digits). `GI' and `GR' give grammatical information that holds for the
whole entry, and, in case of sub-entries, every sub-entry has its own `GI' which
gives information that holds for the sub-entry only. Some other
information can be given in `TI', `CI', `LI' and `ZI', which have the same 
{\em scope} as the preceding `GI'. The same holds for `TR', `CR', `LR' and `ZI'
with respect to the preceding `GR'. {\em Inside} a `NoRom' there can be one or
more morphological variants, again with their own information (`GV', `FV', 
etc.).

Now, each main entry, and , in case of sub-entries, each sub-entry gets its
description (see: `Descr' in the syntax). This description consists of
either a meaning group (`Nul') and, optionally, an example group (`Phrase'),
or of just a single example group.

A meaning group consists of meanings (`BB') which are numbered (`NN'); per 
number more than one meaning is possible, and every meaning can be followed 
by synonyms (`SB') and antonyms (`AB'). `LN' and `LB' mark 
places were additional information about a particular meaning can be given 
(e.i.\ about the context or situation, in which this meaning is used, etc.).
It is possible that a certain meaning number is {\em not} followed by a 
`BB', but in that case, a `LN' with some infomation about the particular
meaning is present.

An example group consists of an example number (`CC'),
referring to a preceding meaning number, followed by a `VC' with 
example, or at least a `LC'. If an example has a meaning that cannot be
derived from the corresponding meaning(s) given in the meaning group
(think of idioms), the
`XX' is used as the place were this special meaning can be given.

Although our syntax guarantees a high degree of consistency, not everything 
is checked:
\begin{itemize}
  \item in case of an entry with `NI', we have no guarantee that there is
        indeed a homonym in the file. In fact, this cannot be checked
        as long as we limit our syntax to entries.
  \item there is no check on the numbers behind `NN' and `CC'; the highest
        number behind `CC' cannot be higher than the highest number behind
        `NN' in the same entry or sub-entry. If we want to be sure if this type
        of consistency, we need to look {\em deeper} than we do now.
 
        The same holds for the numbering {\em itself}; it must be correct
        for `NN' and -in roman- `RR'. For `CC', not all the numbers need to
        be present, but, of course, they should be given in the right order.

  \item it is possible that the scope of some of the codes may lead to a 
        contradiction; this is the case if, for example, in a certain entry
        the `TR' code says that a certain word originates the French language,
        and, a `TI' code that belongs to a sub-entry of the same language says
        that the word (in this sub-meaning) originates the German language.
        In fact, to check this degree of consistency is very hard: in many
        cases the contradiction can only be analysed by using `knowledge of
        the world'.
\end{itemize}
\newpage
\section{Inconsistencies and errors}
There are many inconsistencies and errors in the Van Dale N~--~N file. It seems 
that the `editors' of the file {\em typed} every single character; therefore 
there is no guarantee that all information of a certain type can be found by 
searching specific strings in the file. Typing errors can be found in the
codes, in the category numbers, in the entry words, in the meaning description,
etc. Some of these errors resulted in `syntactically' wrong entries; they have 
already been listed in section 3.2. Some inconsistencies are mentioned in 
the list of codes (see `VV'). Others are listed here. This list is 
{\em not} complete: it is meant as an inventory to give an impression of
the variaty of strange things, inconsistencies and errors that can be expected.
\begin{itemize}
  \item in many cases the `I\,I'-code is not followed by a {\em single} word, 
        but by more words or parts of words:
     \begin{itemize}
        \item ``kompi-, komple-''  (followed by `VI' with: ``compi-, comple-'')
        \item ``gruizele{\tt +;}menten, gruize{\tt +;}menten'' (followed by `VI' with: 
              ``gruzelementen'')
     \end{itemize}
  \item also, combinations of words and parts of words exist:
     \begin{itemize}
        \item  ``kommi-, kom{\tt +;}mode''
        \item  ``kol{\tt +;}lega, kollege(-)'' (``kollege'' doesn't have an
                accent)
     \end{itemize}
  \item in some cases pluralforms or other inflected forms miss, or are not 
        specified consistently:
     \begin{itemize}
        \item Normally, the pluralform is specified by the string that is added
              to the word (for ``kolenmijn'': \mbox{`-en'}, for ``komkommer'': 
              \mbox{`-s'}, etc.),
              and only when part of the word itself changes, (part of) the word
              is given (for ``kool'': `kolen', for ``koolmees'': `koolmezen').
              
              Sometimes this is not done consistently: ``koolmijn'' has the 
              pluralform \mbox{`-mijnen'} instead of \mbox{`-en'}. 

              Sometimes, errors have been made:
        \begin{itemize}
            \item
              ``oppas'' has \mbox{`-en'} instead of \mbox{`-sen'};
              ``rotteknip'' and ``stap'' have \mbox{`-en'} instead of \mbox{`-pen'};
              ``contactafdruk'', ``geleidingshek'' and ``steigerbok'' have
                 \mbox{`-en'} instead of \mbox{`-ken'}; 
              ``vijgenmat'', ``zoetstoftablet'' and ``zuigtablet'' have 
                 \mbox{`-en'} instead of \mbox{`-ten'};
              ``gel'' and ``prijsverschil'' have \mbox{`-en'} instead of 
                 \mbox{`-len'};
              ``tweepersoonsbed'' has \mbox{`-en'} instead of \mbox{`-den'};
            \item
              ``pijlvergif'' has only \mbox{`-en'}, which is correct for the
              variant form ``pijlvergift'', but not for ``pijlvergif'; the same 
              is the case for ``pijlgif''; ``lijkegif'' however, has only 
              \mbox{`-ten'}, although it also has the variant ``lijkegift''.
            \item
              ``vijgenmand'' has the plural form \mbox{`-ten'}, instead of 
              \mbox{`-en'}.
        \end{itemize}

       \item Quite a few nouns do not have information about their pluralform
             at all, although there is an explicit 
             value `g.mv.' for words that do not have a plural at all, like:
             ``oorlogstuig'' and ``kabeltelevisie''.
     \end{itemize}
  \item Verbs have a past form and a past participle behind the `GI'-code, and
        normally the past participle is preceded by `h.' or `i.' to make clear
        whether the verb gets `hebben' or `zijn' as auxiliary. When both
        `hebben' and `zijn' are possible, the past participle is preceded by
        `h.$/$i.'. Sometimes different strings are possible, like, for instance:
     \begin{itemize}
        \item `h' (without dot) instead of `h.': ``evoqueren'', ``greineren'',
              ``koffietafelen''.
        \item `s' (without dot) instead of `s.': ``afrijden'', ``minderen'',
              ``teruglopen'', ``toeschuiven'' and ``trappen''.
        \item no space between `h.' and the past participle: ``bebossen''.
        \item `heeft' instead of `h.': ``enerveren''.
        \item `h. en i.' instead of `h.$/$i.': ``expanderen''.
     \end{itemize}
  \item the past participle behind the `GI' of ``racen'' is followed by a `)'.
  \item ``wegstuffen'' has `{\bf -te}' ending in the past form, and `{\bf -d}' 
        in the past participle, which is inconsistent according to what is 
        normally considered the conjugation of weak verbs in Dutch.
  \item the past tense is also given in different notations, like:
     \begin{itemize}
      \item ``opstoten'': ` GI\#\ \ \ \ 32; stootte en stiet op, h. opgestoten '
      \item ``opvragen'': ` GI\#\ \ \ \ 32; vroeg, vraagde op, h. opgevraagd '
     \end{itemize}
  \item some words are not typed in correctly:
     \begin{itemize}
        \item ``renunci\"{e}ren'' contains a space between `c' and `i'.
        \item The `VV'-form of ``sponzen'' contains a space between the accent
              and the word itself.
        \item The words ``de{\tt +;}tailhandelsprijs'', ``{\tt +;}knippen'', 
              ``{\tt +;}kruisgewijs'', ``{\tt +;}Slavisch'', ``{\tt +;}top-veertig'', 
              ``{\tt +;}trammen'', ``{\tt +;}tramnet'' and ``{\tt +;}tramweg'' are {\em followed}
              by a space.
        \item The past participles of ``impregneren'' and ``overenten'' are not
              spelled correctly: `h. geimpregneerd' instead of `h. 
              ge{\tt "}impregneerd', `h. overg{\tt "}eent' instead of 
              `h. overge{\tt "}ent'.
        \item The entry word ``minstbe{\tt +;}bedeelden'' is not spelled 
              correctly.
        \item Many errors have been made in the examples; here a ``$\sim$'' has
              been used to denote the entry word; in principle, entry words are
              not inflected, and ``$\sim$'' therefore always refers to not 
              inflected forms. Examples of errors: ``kop'' (in 1.1) is 
              {\em both} spelled out {\em and} given as `$\sim$';  ``beetje''
              (in: 3.1) and ``beentje'' (in: 3.2) have `$\sim$jes' resp. 
              `$\sim$tjes' instead of `$\sim$s'; ``oogverblindend'' has
              (in: 2.1) `l$\sim$' instead of `$\sim$'.

     \end{itemize}

  \item in some cases there is an inconsistency between the `headcategory' 
        preceded by a `GR'-code and the category numbers preceded by the 
        various `GI'-codes of the same entry; the manual of the printed N~--~N 
        says that all sub-entries of an entry have the same `headcategory',
        which is specified in the `GR'-code part. So if a headword has 
        `GR\#\ \ \ 2', all sub-entries of this entry should have category 
        numbers 
        in the range 20 .. 29. This is {\em not} the case for a.o.: 
        ``aanmoedigen'' (`GR\#\ \ \ 2', combined with `GI\#\ \ \ 31,32' 
        and `GI\#\ \ \ 32');
       ``averechts'' (`GR\#\ \ \ 2', in combination with `GI\#\ \ \ 50' and 
        `GI\#\ \ \ 20');
        ``boze'' (`GR\#\ \ \ 2', in combination with `GI\#\ \ \ 11' and 
        `GI\#\ \ \ 13'.

        Some words have a wrong category number: the noun ``doling'' and the 
        verb ``verorberen'' both have 
        category number `21', which is a number for adjectives. The verb
        ``heupwiegen'' has got number `30', which stands for verbs that cannot
        be conjugated; directly behind this number, however, the conjugated forms
        `heupwiegde' and `h. geheupwiegd' are specified. The verb 
        ``scheuken'' has number `32', but is reflexive (which can be concluded
        from the conjugated forms that are accompanied by the word `zich').
        The verbs ``dubbelen'' and ``truten'' are listed as nouns, although
        they are inflected like verbs. The same seems to be the case for 
        ``onthalzen'', that doesn't have inflected forms. 

        One word didn't have a category number at all: ``wat'', sub-entry IV.
        
  \item Some entries have incorrectly numbered sub-entries:
     \begin{itemize}
        \item ``exerceren'' has `II' twice instead of `I' and `II';
        \item ``gans'' has `I' twice instead of `I' and `II';
        \item ``stijf'' has `II' twice instead of `I' and `II';
        \item ``verhongeren'' has `I' twice instead of `I' and `II';
        \item ``beheersen'' has only `I' and not `II'.
        \item ``rook'' has only `II' and not `I'.
    \end{itemize}
  \item Some words fail in the N~--~N, although they seem to be part of 
        the current Dutch vocabulary:
     \begin{itemize}
        \item the verb ``huishouden''.
        \item the noun ``bacterie''.
        \item the noun ``bel''.
    \end{itemize}
\end{itemize}
\newpage
\section{Statistics}
\subsection{Number of occurences of the codes}
The number of occurences of the codes has been determined:

\begin{table}[htb]
\centering
\begin{tabular}{lcl}
 Entry  &=& DI$_{(86550)}$ I\,I$_{(86550)}$ [ NI$_{(3135)}$ ]\\
        & & ( VI$_{(1886)}$\\
        & & $|$ [ SI$_{(5112)}$ ]\\
        & & \hspace{2mm} (\{ VV$_{(2190)}$ [ SV$_{(47)}$ ] [ UV$_{(8)}$ ] $|$ UI$_{(280)}$ ) \} \\
        & & \hspace{2mm} [ FI$_{(4810)}$ ]\\
        & & \hspace{2mm} ( Head$_{(3285)}$ Rom$_{(3285)}$ Descr$_{(3285)}$ \{ Rom$_{(3511)}$ Descr$_{(3511)}$ \}+ \\
        & & \hspace{2mm} $|$ NoRom$_{(81379)}$ Descr$_{(81379)}$ )) ,\\ \\
 Head   &=& GR$_{(3285)}$ [ TR$_{(27)}$ ] [ CR$_{(19)}$ ] [ LR$_{(85)}$ ] \{ ZI$_{(362)}$ \} , \\ \\
 Rom    &=& RR$_{(6796)}$ GI$_{(6796)}$ [ TI$_{(4)}$ ] [ CI$_{(1)}$ ] \{ LI$_{(327)}$ \} ,\\ \\
 NoRom  &=& GI$_{(81379)}$ [ TI$_{(3524)}$ ]\\
        & & \{ \{ VV$_{(903)}$ [ SV$_{(29)}$ ] [ UV$_{(1)}$ ] \}+ GV$_{(893)}$ [ TV$_{(9)}$ ] \}\\
        & & [ CI$_{(207)}$ ] \{ LI$_{(8441)}$ \} \{ ZI$_{(993)}$ \} ,\\ \\
 Descr  &=& Nul$_{(87556)}$ [ Phrase$_{(28327)}$ ] $|$ Phrase$_{(619)}$ ,\\ \\
 Nul    &=& \{ NN$_{(124012)}$\\
        & & \hspace{2mm} ( \{ LN$_{(11531)}$ \}+ [ BB$_{(10351)}$ ] $|$ BB$_{(112708)}$ )\\
        & & \hspace{2mm} \{ SB$_{(43492)}$ \} \{ AB$_{(1478)}$ \} \\
        & & \hspace{2mm} \{ [ LB$_{(71)}$ ] BB$_{(266)}$ \{ SB$_{(108)}$ \} \{ AB$_{(4)}$ \} \} \\
        & & \}+ ,\\ \\
 Phrase &=& \{ CC$_{(86498)}$\\
        & & \hspace{2mm} ( \{ LC$_{(6228)}$ \}+ [ VC$_{(6196)}$ ] $|$ VC$_{(80302)}$ ) \\
        & & \hspace{2mm} \{ LX$_{(2400)}$ $|$ XX$_{(31234)}$ \} \\
        & & \}+ . \\
\end{tabular}
\caption{{\em syntax of the entry, with frequency numbers}}
\end{table}

These figures show, that there are 86550 entries in the N~--~N, of which 1886
are references to other entries; 3285 entries have two or more sub-entries, and
86550 -- 3285 -- 1886 = 81379 entries have neither sub-entries nor reference,
and are `normal'. 

\subsection{Distribution of the wordlength}
The wordlength distribution of the words is shown in figure 1; here we didn't
count accents. 
%% This graph denotates the wordlength distribution in the N-N 

 \newcommand{\sety} [2]{\put(15,#1) {\makebox(0,0)[r]{\tiny #2}}
                        \put(20,#1) {\line(1,0){3}}}
 %% displays text #2 at 15,#1 justified right (in picture) and sets
 %% mark on y-axis at #1

 \newcommand{\setxy}[3]{\put(#1,#2) {\makebox(0,0)[b]{\tiny #3}}
                        \put(#1, 0) {\line(0,1){4}}}
 %% displays text #3 at #1,#2 (in picture) and sets mark on x-axis at #1

 \newcommand{\setxh}[2]{\setxy{#1}{-10}{#2}}
 %% displays text #2 at #1,-10 (in picture)

 \newcommand{\setxl}[2]{\setxy{#1}{-13}{#2}}
 %% displays text #2 at #1,-13 (in picture)

 \newcommand{\ptln} [2]{\put(#1,#2){\circle{3}}}
 %% displays circle at #1,#2

 \begin{figure}
 \begin{picture}(350,510)(0,0)

  %% y-axis
  \put(20,0){\line(0,1){505}}

  %% x-axis
  \put(20,0){\line(1,0){365}}

  %% y-axis numbers (number of counted words)
  \sety{  0}{    0}
  \sety{ 20}{  500}
  \sety{ 40}{ 1000}
  \sety{ 60}{ 1500}
  \sety{ 80}{ 2000}
  \sety{100}{ 2500}
  \sety{120}{ 3000}
  \sety{140}{ 3500}
  \sety{160}{ 4000}
  \sety{180}{ 4500}
  \sety{200}{ 5000}
  \sety{220}{ 5500}
  \sety{240}{ 6000}
  \sety{260}{ 6500}
  \sety{280}{ 7000}
  \sety{300}{ 7500}
  \sety{320}{ 8000}
  \sety{340}{ 8500}
  \sety{360}{ 9000}
  \sety{380}{ 9500}
  \sety{400}{10000}
  \sety{420}{10500}
  \sety{440}{11000}
  \sety{460}{11500}
  \sety{480}{12000}
  \sety{500}{12500}

  %% x-axis numbers (number of letters in word)
  \setxh{ 20}{ 0}
  \setxl{ 30}{ 1}
  \setxh{ 40}{ 2}
  \setxl{ 50}{ 3}
  \setxh{ 60}{ 4}
  \setxl{ 70}{ 5}
  \setxh{ 80}{ 6}
  \setxl{ 90}{ 7}
  \setxh{100}{ 8}
  \setxl{110}{ 9}
  \setxh{120}{10}
  \setxl{130}{11}
  \setxh{140}{12}
  \setxl{150}{13}
  \setxh{160}{14}
  \setxl{170}{15}
  \setxh{180}{16}
  \setxl{190}{17}
  \setxh{200}{18}
  \setxl{210}{19}
  \setxh{220}{20}
  \setxl{230}{21}
  \setxh{240}{22}
  \setxl{250}{23}
  \setxh{260}{24}
  \setxl{270}{25}
  \setxh{280}{26}
  \setxl{290}{27}
  \setxh{300}{28}
  \setxl{310}{29}
  \setxh{320}{30}
  \setxl{330}{31}
  \setxh{340}{32}
  \setxl{350}{33}
  \setxh{360}{34}
  \setxl{370}{35}
  \setxh{380}{36}

  %% Graph (number of counted words per length of word)
  \ptln{ 20}{  0}
  \ptln{ 30}{  2}
  \ptln{ 40}{ 12}
  \ptln{ 50}{ 42}
  \ptln{ 60}{ 98}
  \ptln{ 70}{120}
  \ptln{ 80}{212}
  \ptln{ 90}{285}
  \ptln{100}{418}
  \ptln{110}{499}
  \ptln{120}{475}
  \ptln{130}{386}
  \ptln{140}{282}
  \ptln{150}{203}
  \ptln{160}{144}
  \ptln{170}{100}
  \ptln{180}{ 64}
  \ptln{190}{ 44}
  \ptln{200}{ 28}
  \ptln{210}{ 17}
  \ptln{220}{ 13}
  \ptln{230}{  7}
  \ptln{240}{  4}
  \ptln{250}{  2}
  \ptln{260}{  2}
  \ptln{270}{  1}
  \ptln{280}{  0}
  \ptln{290}{  0}
  \ptln{300}{  0}
  \ptln{310}{  0}
  \ptln{320}{  0}
  \ptln{330}{  0}
  \ptln{340}{  0}
  \ptln{350}{  0}
  \ptln{360}{  0}
  \ptln{370}{  0}
  \ptln{380}{  0}

  %% x-axis text
  \put(250,50){\vector(1,0){30} 
               \parbox{1in}{\footnotesize\bf Number of letters of headword}}

  %% y-axis text
  \put(200,300){\vector(0,1){30}
                \parbox{18mm}{\footnotesize\bf Number of headwords}}

 \end{picture}
 \vspace{3mm}
 \caption{Distribution of the wordlength}
 \end{figure}

The longest word in the N~--~N is: ``ac{\tt +;}countants-admini{\tt
+;}stratieconsulent'', which is 38 characters (including the accents). The
average length of the entry words in the N~--~N is 12,73 characters (including
accents). 

\subsection{Number of occurences of the main categories}
 The following table contains the number of entry-words of the main categories
 per letter. The codes are 
 numbered from zero to nine and have the same meaning as explained in section 
 3.1 (see GR and GI), except for words with category number 08 and 09, which
 are treated as nouns (main category 1). 
 For instance: there are 795 verbs beginning with the 
 letter `a', 838 verbs beginning with `b', etc.
 `Ref' means that the word doesn't have a G-code at all; see VI in 
 section 3.1. 

 \begin{table}[hb]
  \begin{tabular}{|l|}\hline
   ~\\ \hline\hline
   A\\ \hline
   B\\ \hline
   C\\ \hline
   D\\ \hline
   E\\ \hline
   F\\ \hline
   G\\ \hline
   H\\ \hline
   I\\ \hline
   J\\ \hline
   K\\ \hline
   L\\ \hline
   M\\ \hline
   N\\ \hline
   O\\ \hline
   P\\ \hline
   Q\\ \hline
   R\\ \hline
   S\\ \hline
   T\\ \hline
   U\\ \hline
   V\\ \hline
   W\\ \hline
   X\\ \hline
   Y\\ \hline
   Z\\ \hline\hline
   Tot\\ \hline
  \end{tabular}
  \begin{tabular}{|r|r|r|r|r|r|r|r|r|r|r|%|r|
   }\hline
   0     &1     &2    &3   &4   &5   &6   &7  &8   &9  &Ref   %&Tot
   \\ \hline\hline
   78  &2540   &608  &795   &7  &32   &6   &9  &8  &25   &82   %&4190
   \\ \hline
   41  &4241   &684  &838   &0  &18  &13   &4  &2  &19   &66   %&5926
   \\ \hline
   47  &1890   &295  &186   &0   &0   &3   &0  &1   &2   &60   %&2484
   \\ \hline
   51  &2507   &488  &520  &16  &71   &2  &14  &8   &8   &62   %&3747
   \\ \hline
   49  &1109   &326  &108   &8  &57   &1   &9  &4   &4   &48   %&1723
   \\ \hline
   31  &1291   &238  &178   &0   &9   &0   &0  &0   &7   &52   %&1806
   \\ \hline
   43  &3069  &1064  &240   &5  &17   &3   &4  &1  &14   &84   %&4544
   \\ \hline
   54  &3109   &493  &236  &12  &54   &3   &7  &3  &38   &74   %&4083
   \\ \hline
   70  &1062   &366  &426  &10   &9   &6   &0  &2   &1   &55   %&2007
   \\ \hline
   17   &389    &39   &38   &6   &2   &1   &0  &0  &16   &13    %&521
   \\ \hline
   42  &4316   &489  &418   &0   &6   &1   &0  &0  &34  &258   %&5564
   \\ \hline
   47  &2824   &399  &249   &0  &11   &2   &0  &0   &2   &64   %&3598
   \\ \hline
   76  &3069   &502  &293   &8  &13   &4   &9  &3   &8   &95   %&4080
   \\ \hline
   54  &1330   &277  &231   &6   &0   &7  &12  &9   &5   &30   %&1961
   \\ \hline
   44  &2194  &1375 &1090   &1  &33  &12   &0  &6  &10   &65   %&4830
   \\ \hline
   75  &4811   &730  &414   &0  &25   &5   &0  &0  &35  &187   %&6282
   \\ \hline
   16    &70    &11    &4   &0   &1   &1   &1  &0   &0   &18    %&122
   \\ \hline
   30  &3296   &418  &421   &0   &7   &2   &0  &1  &12   &97   %&4284
   \\ \hline
   66  &7264  &1021  &739   &1  &30   &6   &1  &3  &25  &201   %&9357
   \\ \hline
   58  &2520   &403  &474   &0  &29   &9  &19  &7  &19  &100   %&3638
   \\ \hline
   10   &373   &113  &358   &2   &3   &3   &0  &2   &2    &6    %&872
   \\ \hline
   66  &3317   &788 &1024   &0  &16  &10  &14  &4  &21   &67   %&5327
   \\ \hline
   26  &2490   &379  &302   &6  &29   &1   &2  &6  &14   &58   %&3313
   \\ \hline
    6    &39     &2    &1   &0   &0   &0   &0  &0   &0    &4     %&52
   \\ \hline
    2    &23     &0    &1   &0   &0   &0   &0  &0   &1    &0     %&27
   \\ \hline
   36  &1648   &295  &123  &14  &26   &1  &12  &6  &11   &40   %&2212
   \\ \hline\hline
   1135 &60791 &11803 &9707 &102 &498 &102 &117 &76 &333 &1886  %&86550
   \\ \hline
  \end{tabular}
\caption{{\em number of occurences of the main categories}} 
\end{table}

For each entry, the first G-code (either GR or GI) is evaluated, or if no
G-code was present, the VI-code was counted (see `Ref').
Also the {\em totals} are given (all the letters added) 
so that the total number of words per category can be found in this table too 
(eg. 9707 verbs, 117 numerals, etc.).

\subsection{Number of meanings per word for the main categories}
The number of entries with a certain number of meanings has been determined
for the main categories.
Input for the program were the files made by Jeroen Medema (see section 1). 
For each entry the number of `NN\#'-codes has been counted. When there are two 
or more sub-meanings, each sub-entry is counted separately (of course, the total
number of meanings of the main entry has been ignored then).

\begin{table}[htb]
\begin{tabular}{|r||r|r|r|r|r|}\hline
   & \multicolumn{5}{c|}{\em Nouns}\\ \cline{2-6}
   & \multicolumn{2}{c|}{\bf $\neg$ Freq} &\multicolumn{2}{c|}{\bf Freq}
     & {\bf Totals}\\ \cline{2-5}
   \begin{picture}(10,10)
     \put(0,0){\shortstack{\#\\M}}
   \end{picture}
   &$\neg$ r&      r&$\neg$ r&     r&      \\ \hline\hline
 0 &    230&      7&     14&      2 &   253\\ \hline
 1 &  45929&    788&    535&     91 & 47343\\ \hline
 2 &   8809&    218&    501&     41 &  9569\\ \hline
 3 &   2128&     76&    359&     22 &  2585\\ \hline
 4 &    630&     44&    240&     23 &   937\\ \hline
 5 &    227&     20&    132&     12 &   391\\ \hline
 6 &     86&     12&     89&      4 &   191\\ \hline
 7 &     41&      3&     49&      7 &   100\\ \hline
 8 &      9&      2&     34&      4 &    49\\ \hline
 9 &      4&      2&     13&      4 &    23\\ \hline
10 &      6&      2&     13&      2 &    23\\ \hline
11 &      2&       &      8&      1 &    11\\ \hline
12 &       &       &     10&      1 &    11\\ \hline
13 &       &       &      7&      1 &     8\\ \hline
14 &       &       &      3&        &     3\\ \hline
15 &       &       &       &      1 &     1\\ \hline
16 &       &       &      1&        &     1\\ \hline
17 &       &       &      2&        &     2\\ \hline
18 &       &       &       &        &      \\ \hline
19 &       &       &       &        &      \\ \hline
20 &       &       &      1&        &     1\\ \hline
\end{tabular}
\begin{tabular}{r|r|r|r|r|}\hline
 \multicolumn{5}{c|}{\em Verbs}\\ \hline
 \multicolumn{2}{c|}{\bf $\neg$ Freq} &\multicolumn{2}{c|}{\bf Freq}
     & {\bf Totals}\\ \cline{1-4}
 $\neg$ r&      r&$\neg$ r&      r& \\ \hline\hline
   32&     22&      1&      2 &   57\\ \hline
 4581&   1897&     69&    264 & 6811\\ \hline
 1957&    713&    112&    163 & 2945\\ \hline
  641&    246&     90&    104 & 1081\\ \hline
  203&     94&     52&     61 &  410\\ \hline
   60&     40&     28&     37 &  165\\ \hline
   12&     19&     27&     21 &   79\\ \hline
    5&      8&     13&     13 &   39\\ \hline
    6&      5&      6&      8 &   25\\ \hline
     &      1&      3&      9 &   13\\ \hline
    1&      1&      5&      7 &   14\\ \hline
     &       &      2&      2 &    4\\ \hline
     &       &      1&      2 &    3\\ \hline
     &       &      2&      1 &    3\\ \hline
     &       &       &      1 &    1\\ \hline
     &      1&      1&      2 &    4\\ \hline
     &       &      1&        &    1\\ \hline
     &       &      2&      2 &    4\\ \hline
     &       &      1&        &    1\\ \hline
     &       &      1&      1 &    2\\ \hline
     &       &       &        &     \\ \hline
\end{tabular}
\caption{{\em Number of meanings per word for nouns and verbs}} 
\end{table}

Table 4 shows the distribution for the {\em nouns} and {\em verbs}.

The (total) number of entries is also subdivided into four groups (`r' means
`roman'): 
\begin{itemize}
  \item non-frequent entries (`{\em not} Freq., {\em not} r'), 
  \item non-frequent sub-entries (`{\em not} Freq., r'),
  \item frequent entries (`Freq., {\em not} r'),
  \item frequent sub-entries (`Freq., r').
\end{itemize}
For instance: there are 47343 nouns
with exactly {\em one} meaning; 45929 of them are {\em not} frequent and form
an entry themselves; 788 are {\em not} frequent and form a sub-entry; etc.

The highest number of meanings is 20; this is for a frequent non-roman noun.
A few words have {\em zero} meanings; this implies that they have only
a `Phrase' part (see section 3.2, syntax of the entry) and no `Nul'. To this 
group belong words that are only part of idioms and do not have a meaning 
as a single word, like ``brui''.

\begin{table}[htb]
\begin{tabular}{|r||r|r|r|r|r|}\hline
   & \multicolumn{5}{c|}{\em Adjectives, Adverbs}\\ \cline{2-6}
   & \multicolumn{2}{c|}{\bf $\neg$ Freq} &\multicolumn{2}{c|}{\bf Freq}
     & {\bf Totals}\\ \cline{2-5}
   \begin{picture}(10,10)
     \put(0,0){\shortstack{\#\\M}}
   \end{picture}
   &$\neg$ r&      r&$\neg$ r&      r& \\ \hline\hline
 0 &  160&      9&      5&      4 &  178\\ \hline
 1 & 7738&    683&    425&    217 & 9063\\ \hline
 2 & 1723&    196&    212&    142 & 2273\\ \hline
 3 &  397&     64&    113&     95 &  669\\ \hline
 4 &  102&     23&     55&     42 &  222\\ \hline
 5 &   27&      6&     25&     27 &   85\\ \hline
 6 &   12&     11&      5&     16 &   44\\ \hline
 7 &    4&      1&      6&      9 &   20\\ \hline
 8 &    1&      1&      6&      5 &   13\\ \hline
 9 &    1&      1&      3&      1 &    6\\ \hline
10 &     &       &      2&      2 &    4\\ \hline
11 &     &       &      3&        &    3\\ \hline
12 &     &       &      2&      2 &    4\\ \hline
13 &     &       &      4&        &    4\\ \hline
14 &     &       &       &      1 &    1\\ \hline
15 &     &       &       &      2 &    2\\ \hline
16 &     &       &       &      2 &    2\\ \hline
17 &     &       &       &        &     \\ \hline
18 &     &       &       &        &     \\ \hline
19 &     &       &       &        &     \\ \hline
20 &     &       &       &        &     \\ \hline
\end{tabular}
\begin{tabular}{r|r|r|r|r|}\hline
 \multicolumn{5}{c|}{\em Adverbs}\\ \hline
 \multicolumn{2}{c|}{\bf $\neg$ Freq} &\multicolumn{2}{c|}{\bf Freq}
     & {\bf Totals}\\ \cline{1-4}
 $\neg$ r&      r&$\neg$ r&      r&     \\ \hline\hline
        8&       &     14&        &   22\\ \hline
      176&       &    180&        &  356\\ \hline
       19&      1&     57&        &   77\\ \hline
        2&      1&     22&        &   25\\ \hline
        1&       &      4&        &    5\\ \hline
        1&       &      8&        &    9\\ \hline
         &       &       &        &     \\ \hline
         &       &      2&        &    2\\ \hline
         &       &      2&        &    2\\ \hline
         &       &      1&        &    1\\ \hline
         &       &       &        &     \\ \hline
         &       &       &        &     \\ \hline
         &       &       &        &     \\ \hline
         &       &       &        &     \\ \hline
         &       &       &        &     \\ \hline
         &       &       &        &     \\ \hline
         &       &       &        &     \\ \hline
         &       &       &        &     \\ \hline
         &       &       &        &     \\ \hline
         &       &       &        &     \\ \hline
         &       &       &        &     \\ \hline
\end{tabular}
\caption{{\em Number of meanings per word for adjectives and adverbs}} 
\end{table}

Table 5 contains the adjectives (and adverbs) of main category 2 and the
adverbs of main category 5.
The adverbs of category 5 form a small group: 498 entries, of which one is 
subdivided in two sub-entries 
(which makes 499 instances in the table). Many of these
adverbs are frequent. 

As was to be expected, frequent words have in general a higher number of 
meanings than non frequent.

The number of entries that are not split in sub-entries and have only one
meaning is very interesting (this is the case for $45929 + 535 = 46464$ nouns):
these words are {\em not}
ambiguous, except for those with NI-code. For the nouns, this number is rather 
high;
approximately 70\% of the entry-words of this category belong to this 
group. Of course, these words do not yield as much problems as
ambiguous nouns with respect to dictionary work like filling attributes, etc.

The distribution of the verbs differs from that of the nouns: there are 
relatively many sub-entries and also relatively many entries or sub-entries 
with two or more 
meanings. This means that verbs are -in general- often ambiguous.

Adjectives have a distribution that is more like the distribution of nouns,
except for the number of sub-entries, that is relatively high. The fact that 
many adjectives can also act as adverb might be an explanation for this fact.
\newpage
\section{Lists}
In this section, some lists derived from the N~--~N are given.

\subsection{Lists of some closed classes}

Lists of pronouns, prepositions and conjunctions:
\begin{itemize}
  \item {\bf Pronouns} (category 40-49):
  \begin{description}
      \item [A] al (48), alle (48), allehens (48), alleman (48), alles (48), 
                ander (48), anderman (48),
      \item [D] dat I (42; in het mv. die), II (44; in 't mv. die), 
                datgene (42),
                datzelfde (42), degene (42), dergelijk (42), 
                dewelke (44), deze (42),
                dezelfde (42), dezulke (42), die I (42), II 
                (00; bepalingaankondigend vnw.), III (44), diegene (42), 
                diergelijk (42), dies (42), diezelfde (42), dit (42),
                dusdanig (42),
      \item [E] een (48), eentje (48), elk (48), elkaar (47; ook mekaar), 
                elkeen (48), ene (48), enig (48), 
                er (41; oude tweede nv. van de derde pers. mv.),
      \item [G] ge (41), geen (48), gene (42), gij (41), gijlieden (41),
      \item [H] haar I (41), II (43), hem (41), hen (41), het I (41), II (48), 
                hetgeen I (42), II (44), hetwelk (44), hetzelfde (42), hij (41),
                hoedanig (45), hullie (41), hun I (41), II (43), hunner (41),
      \item [I] idem (42), ie (41), ieder (48), iedereen (48), iegelijk (48), 
                iemand (48), iets (48), ietwat (48), ik (41), ikke (41),
      \item [J] je I (41), II (48), III (43), jezelf (41), jij (41), jou (41),
                jouw (43), jullie I (41), II (43),
      \item [M] malkaar (47), me (41), mekaar (47), men (48), menigeen (48), 
                mezelf I (41), II (46), mij I (41), II (46), mijn (43),
      \item [N] niemand (48), niemendal (48), niet (48), niets (48), niks (48),
                noppes (48),
      \item [O] ons I (41), II (43),
      \item [S] sommige (48),
      \item [U] u (41), uw (43),
      \item [W] wat I (44), II (45; vgl. wat+1 bet. 4), III (48), 
                IV (uitroepend voornaamwoord), we (41), weinig 
                (48; minder, minst; vgl. aldaar), welk I (45), II (44), 
                III (48; met +'ook'), wie I (45), II (44), III (48), wij (41),
      \item [Z] ze I (41), II (48), zeker (48), zelf (42), zich (46), 
                zichzelf (46), zij (41), zijn (43), zodanig (42), zo'n (42), 
                zoveel (48), zowat (48), zulk (42), zulks (42), zullie 
                (41; mv.),
      \end{description}
   Note: the various sub-entries (`I', `II', etc.) and the category-number
   (`41', 42', etc.) have been specified. The same holds for grammatical 
   information like: `mv.', etc.

   \item {\bf Prepositions} (category 60):
      \begin{description}
      \item [A] `a, aan, aangaande, aangenomen, achter, ad,
      \item [B] behalve, behoudens, beneden, benevens, benoorden, betreffende,
                bezijden, bezuiden, bij, binnen, blijkens, boven, buiten,
      \item [C] con, contra, cum,
      \item [D] dankzij, door,
      \item [E] ex,
      \item [G] gedurende, getuige, gezien,
      \item [H] halfweg, halverwege, hangende,
      \item [I] in, ingaande, ingevolge, inter, intra, inzake,
      \item [J] jegens,
      \item [K] krachtens,
      \item [L] langs, langsheen,
      \item [M] met, middels, min, minus,
      \item [N] na, naar, naast, nabij, namens, niettegenstaande, nopens,
      \item [O] om, omstreeks, omtrent, ondanks, onder, onderlangs, ongeacht, 
                onverminderd, onverschillig, op, over, overeenkomstig,
      \item [P] per, plus, post, pro, propos (in vaste verbinding met `a),
      \item [Q] qua,
      \item [R] rond, rondom,
      \item [S] salva, sedert, sinds, sine, staande, sub,
      \item [T] te (met de lidw. 'den' en 'der' samengetr. tot 'ten' en 'ter'),
                tegen, tegenover (in verb. met 'er', 'hier', 'waar' enz. als 
                bijw. te besch.), ter, tijdens, tot, tot en met, trots, tussen,
      \item [U] uit, uitgenomen, uitgezonderd,
      \item [V] van, vanaf, vanuit, vanwege, versus, via, vis-`a-vis, volgens, 
                voor, voorbij,
      \item [W] wegens,
      \item [Z] zonder,
\end{description}
  Note: in some cases, grammatical information is specified.

  \item {\bf Conjunctions} (category 80):
    \begin{description}
      \item [A] aangezien, aleer, alhoewel, als, alsmede, alsof, alsook, annex, 
      \item [B] behalve, bijaldien, 
      \item [C] c.q., 
      \item [D] daar, dan, dat, dewijl, doch, doordat, doordien, dus, 
      \item [E] eer, en, enne, evenals, 
      \item [G] gelijk, 
      \item [H] hetzij, hoewel, hoezeer,
      \item [I] indien, ingeval, 
      \item [M] ma, maar, mits, 
      \item [N] naar, naardien, naargelang, naarmate, nadat, niettegenstaande,
                noch, nou, nu, 
      \item [O] of, ofschoon, ofte, oftewel, omdat, opdat, 
      \item [R] respectievelijk, 
      \item [S] schoon, sedert, sinds, 
      \item [T] teneinde, tenware, tenzij, terwijl, toen, tot, totdat, 
      \item [U] uitgenomen, uitgezonderd, 
      \item [V] vermits, voor, vooraleer, voordat, 
      \item [W] waar, wanneer, want, weshalve, weswege, wijl, 
      \item [Z] zo, zoals, zodat, zodra, zolang, zomede, 
   \end{description}
\end{itemize}
\subsection{Lists of some sub-categories of the verbs}

The sub-categories 35, 36 and 37 are listed in this section. 
For category 34
(which is not interesting) see 3.1, GI-code. Of course, there is no 
guarantee that the N~--~N is complete (the list of copulatives is shorter
than one might expect, for instance).

\begin{itemize}
  \item List of auxiliaries (category 35, between brackets grammatical 
       information 
       from the N~--~N): {\em
hebben, kunnen (van modaliteit), laten, moeten,
mogen, willen (van modaliteit), worden (van de lijdende vorm), zijn, zullen}.

  \item List of copulatives (category 36): {\em blijven, worden, zijn}.

  \item List of impersonal verbs (category 37): 
        {\em aankomen, bliksemen, buien, dagen,
        dauwen, donderen, donkeren, dooien, druppelen, 
        duisteren, ebben, falen, gaan, gelden, 
        gevallen, gieten, gutsen, hagelen, houden, 
        ijzelen, insneeuwen, inwaaien, inwateren, kwakkelen, 
        lekken, miezeren, misten, motregenen, nevelen, 
        onweren, piesen, plasregenen, plensregenen, plenzen, 
        regenen, rijmen, rijpen, sausen, slagregenen, 
        sneeuwen, spannen, spoken, sterven, stikken, 
        stormlopen, stortregenen, stuiven, tochten, vriezen, 
        waaien, winteren}. 
\end{itemize}

\end{document} 
