\documentstyle{Rosetta}
\begin{document}
   \RosTopic{Rosetta3.software}
   \RosTitle{Syntax of the Rosetta Lexicon Language}
   \RosAuthor{Joep Rous}
   \RosDocNr{305}
   \RosDate{April 12, 1989}
   \RosStatus{concept}
   \RosSupersedes{270}
   \RosDistribution{Project}
   \RosClearance{Project}
   \RosKeywords{Dictionary, testing}
   \MakeRosTitle
\section  {Introduction}
In this document we will describe the syntax of the language which is to be
used for specification of the Rosetta lexicons.
First a short overview of the Rosetta dictionaries
is given. In Chapter 3 some remarks are made on the lexicon compiler.
In Chapter 4 the syntax of the dictionary file will be 
given, elucidated with some examples. 

It is possible to specify constraints on 
the attribute values occurring in the lexicon. The syntax of the constraint
specification notation is given in Chapter 5.
The usage of the compiler is described
in Chapter 6. Finally, in Chapter 7 it is explained
how the keys that are used in the dictionary files can be defined\footnote{In 
the examples of chapter 4 I will use integer keys instead of the string keys 
which are introduced in chapter 6}.

\section {Overview}
\subsection {FIXIDDICT}
In Rosetta3 the FIXIDDICT lexicon is used for recognizing {\em fixed idioms}
in the input sentence. It contains a list of all the fixed idioms that
Rosetta3 must be able to deal with. Idioms can be treated as {\em fixed
idioms} if they obey the following criteria: \\
\begin {enumerate}
  \item The words of the idiom always occur in a fixed order (e.g. "by and 
large", "op en top").
  \item No other words can occur in between the idiom words.
  \item All idiom words, except the last one, are unchangeable, that is, not
sensible for inflection and/or derivation. The last word of the idiom may
take inflectional or derivational {\em suffixes}. (e.g. "kant en klaar" can be
treated as a fixed idiom, but "heer des huizes" cannot be a fixed idiom)
\end {enumerate}

If an idiom from the FIXIDDICT lexicon appears in the input sentence it is
treated by the Rosetta3 morphological module as if it were one single word.
Of course, the non-idiom reading is also tried by the system.

\subsection {MDICT}
The MDICT lexicon of Rosetta3 is used to make the transition from strings to
fkeys and vice versa. It contains 5-tuples of the form: 
\begin{verbatim}
   <stem, FON, CC, AG, fkey>

   Key: {(stem, fkey, FON)}

\end{verbatim}

Here, FON specifies the phonetical information associated with the word, CC
specifies the context condition and AG specifies whether the entry is only
relevant for analysis, generation or for both.

(Notice: a stem may also be a fixed idiom, e.g. "by and large")

\subsection {SDICT}
After the analytical segmentation fkeys are translated into skeys. The
fkey-skey mapping is specified in the SDICT lexicon. E.g. the lexicon contains
information for translating the fkey of "lopen" into the skeys belonging to
"lopen", "oplopen", "aflopen", etc.. This is accomplished by specifying a set
of pairs $<$skey, fkey$>$. Moreover, with each pair $<$skey, fkey$>$ a kind of 
condition on the context can be specified. This condition must be satified 
before the translation will be made. The context condition can be specified
by means of a number of fkeys, of which the appearance in the input sentence is
obligatory. In generation, the condition is not relevant because the M-grammar
guarantees that it is satisfied by the M-grammar.

For efficiency reasons we give the fkey and skey of most of the
$<$skey, fkey$>$ pairs the same value.
Only in cases where we have an 1 to n relation between fkey and skey their 
value will be different. In the implementation SDICT contains only these 
deviating $<$skey, fkey$>$ pairs. If the morphological analyzer cannot find an
fkey in SDICT it assumes that the corresponding  skey has the same value (in
generation the same holds for skeys and fkeys resp.).

\subsection {BLEX}
The BLEX dictionary contains, as it did in Rosetta2, pairs 
\begin{verbatim}
< skey, basic S-tree record >
\end{verbatim}

\subsection {ILDICT}
The ILDICT lexicon of Rosetta3 is used to translate skeys in the 
corresponding mkeys ( = meaning keys ). The structure of the lexicon
entries is as follows:

\begin{verbatim}
   <AG, skey, mkey, md, spref, mpref>

   Key: {(skey,mkey)}

\end{verbatim}
Here, {\em md} is a meaning description of the entry, {\em spref} specifies 
an order between entries 
with the same skey and {\em mpref} specifies an order between entries with the
same mkey. The attribute {\em AG} indicates whether the current entry is to be
used in analysis, in generation or in both.

\subsection{SEMI-IDDICT}

The SEMI-IDDICT lexicon defines a relation between basic expressions with a 
literal interpretation and their corresponding semi-idiomatic counterparts. The
dictionary contains tuples of the form (cf. document r0269):

\begin{verbatim}
   < skey-litteral, skey-argument, nr-argument, skey-semiidiom >

   Key: {(skey-semiidiom), (skey-litteral, skey-argument, nr-argument)}

\end{verbatim}

\subsection{IDDICT}

The IDDICT lexicon defines a relation between a sequence of basic 
expressions which are used in the idiom and a basic expression for the
idiom as a whole. The
dictionary contains tuples of the form:

\begin{verbatim}
   < skey-sequence, idiom-pattern, skey-idiom >

   Key: {(skey-sequence, idiom-pattern), (idiom-pattern, skey-idiom)}

\end{verbatim}

\section {The dictionary compiler}

The 7 lexicons that were mentioned in the previous sections should all be 
filled in a consistent way, e.g. if an idiom is present in FIXIDDICT it also
should appear in MDICT, each BLEX entry should have at least one corresponding
MDICT entry, etc.. It seemed to us that the linguist  filling the lexicon would 
get a better overview if (s)he could specify all information for 
all lexicons in one single dictionary in a clear way. 

The idea is that each dictionary entry is specified according to a fixed syntax,
which allows the user to specify in each dictionary-entry information for each 
of the 7 Rosetta3 lexicons.
The NEWDICTGEN compiler checks this and extracts from each entry specification 
the information needed to add one or more entries to the Rosetta3 lexicons.
One can look upon this process as a kind of normalization action that is 
performed by the dictionary compiler.

The dictionary will contain attribute values of which the type has been defined
in Domain T. The compiler checks whether these attribute values
are compatible with the current Domain. For that purpose the domain compiler
generates a number of PASCAL modules which are able to convert attribute
values of type STRING to the corresponding values which are defined in the
PASCAL version of Domain T. If this conversion fails the compiler knows that
it is an erroneous attribute value. 

This first version of the compiler is not very subtle in the sense that it does 
not perform any consistency checks as mentioned in the first alinea of this 
section. That is, however, no fundamental problem but merely a lack of time. 
The consistency checks can easily be incorporated in the compiler in the 
future.

\section {The dictionary syntax}

At the most global level the syntax of the dictionary is: 

\begin {tabbing}
   \= Dictionary \= :: \= \kill
   \> Dictionary \> :: \{ Entry-part \} "@" \\
   \> Entry-part \> :: MDICT-part ":" SDICT-part [ ":" SEMI-IDDICT-part ]\\
   \>            \>    \>[":" IDDICT-part ] ":" ILDICT-part ":" BLEX-part ";" 

\end {tabbing}

\subsection {The MDict part}

The syntax of the MDICT-part is as follows:

\begin {tabbing}
   \= MDICT-part \= \kill
   \> MDICT-part \> :: [ [ AG switch ] string-part [ fon-spec ] [ CC-spec ] ] \\
   \> fon-spec   \> :: "$<$" identifier \{ "," identifier \} "$>$" \\
   \> CC-spec    \> :: identifier \\
   \> AG-switch  \> :: "/" ( "A" $\mid$ "G" $\mid$ "AG" )

\end {tabbing}

The {\em MDICT-part} is optional because it must be possible to specify 
an abstract BLEX entry.
The {\em string-part} in an {\em MDICT-part} specification may not contain 
blanks and the string must at least contain one non-blank character. If the 
string is a fixed idiom, the words of the idiom should be separated by an 
underscore symbol "\_". At one place in the idiom the character "\#" may appear. 
This character is a directive for the compiler not to put the complete idiom in 
the FIXIDDICT lexicon but just the idiom string up to the "\#" character. 
A reserved character, e.g.
":" , "/" or "\#", in a string should be preceded by a backslash "\verb+\+" 
character.

In the
{\em fon-spec} part the attribute values according to the phonetical 
information should be specified. 

The {\em AGswitch} part specifies whether the string is relevant for
Analysis (={\bf /A}), Generation (={\bf /G}) or both (={\bf /AG}).

\begin{verbatim}
Example :

     Duitsland               : ........................
     actrice < true, false > : ........................
     op_en_top               : ........................
     kant_en_kla#ar          : ........................

     an      CCvowel         : ........................
     a       CCcons          : ........................
     
\end{verbatim}

\subsection {The SDict part}

The syntax of the SDICT-part is:

\begin {tabbing}
   \= SDICT-part \= \kill
   \> SDICT-part \> :: skey [ "," fkey [ context-set ] ] \\
   \> context-set \> :: "$<$" cckey \{ "," cckey \} "$>$" \\
   \> cckey       \> :: dictionary-key \\
   \> skey        \> :: dictionary-key \\
   \> fkey        \> :: dictionary-key
\end {tabbing}

In the Rosetta3 system the value of an fkey is by default the same as the
value of the corresponding skey. This fact is expressed in the above syntax
specification, if the fkey is omitted the compiler assumes that it has the
same value as the skey.
Only in the case that an fkey corresponds with a number of skeys it is 
allowed to specify the fkey. If the compiler discovers that an fkey is
specified it takes the MDICT-part specification of the entry as comment.
Therefore, the fkey in these entries should refer to an entry that does have a 
correct MDICT-part. E.g. the verbs "lopen", "aflopen", "oplopen" could have
the following entries:

\begin{verbatim}
   loop    : 525      : ..........
   loop_af : 526, 525 : ..........
   loop_op : 527, 525 : ..........
\end{verbatim}

In this example, the stem "loop" corresponds with fkey and skey 525. However,
the stem of the verb "aflopen" is also "loop" and corresponds with fkey 525,
whereas the skey is 526. The verb "oplopen" corresponds with fkey 525 and skey 
527.

In the above example we could also specify conditions on the context, because
the particle "af" should be present if fkey 525 is to be translated in skey 526
. Suppose the fkey value of "af" is 201 and that of "op" is 604 then the 
entries could also look like:

\begin{verbatim}
   loop    : 525              : ..........
   loop_af : 526, 525 < 201 > : ..........
   loop_op : 527, 525 < 604 > : ..........
\end{verbatim}

Notice that the strings "loop\_af" and "loop\_op" will not be found in MDICT 
because the MDICT-parts of these two entries will be viewed as comment.

Verbs with a particle, like "afkondigen" and "aankondigen", of which 
the form without the particle ("kondigen") does not exist, should be specified 
as follows:

\begin {verbatim}
   kondigen{_af}  : 932, 932 <201> : ...........
   kondigen_aan   : 933, 932 <620> : ...........
\end{verbatim}

The entry for the verb "afkondigen" is now used to define the fkey and the 
stem of the verb ( therefore "\_af" must placed between comment brackets ). 

\subsection {The Semi-IdDict part}

The syntax of the Semi-IdDict-part is:

\begin {tabbing}
   \= Semi-IdDict-part \= \kill
   \> Semi-IdDict-part \> :: sidkey-specs \\
   \> sidkey-specs\> :: sidkey-spec \{ "," sidkey-spec \} \\
   \> sidkey-spec \> :: "\#"arg-nr "=" arg-skey sid-skey sid-mkey \\
   \> arg-nr      \> :: dictionary-key \\
   \> arg-skey    \> :: dictionary-key \\
   \> sid-skey    \> :: dictionary-key \\
   \> sid-mkey    \> :: dictionary-key
\end {tabbing}

It is obvious that the semi-idiom part has to be an optional part of
a lexicon entry. This fact has already been expressed by the definition of the
{\em Entry-part} ( the first definition of section 4).
The {\em arg-nr} part specifies the argument position of
the semi-idiom argument, whereas the {\em arg-skey} part specifies the skey of 
the argument. For each semi-idiom a specific skey is introduced. This skey
is specified in the {\em sid-skey} part. The corresponding mkey is specified
in the {\em sid-mkey} part.
\begin{verbatim}
Examples:

les        : 234 : ..............;
voordracht : 745 : ..............;
geef       : 23  : #2=234 567 6789,
                   #2=745 890 2332 : ................;
\end{verbatim}


\subsection {The IdDict part}

The syntax of the IdDict-part is:

\begin {tabbing}
   \= IdDict-partttt \= \kill
   \> IdDict-part \> :: id-specs \\
   \> id-specs    \> :: id-spec \{ "," id-spec \} \\
   \> id-spec     \> :: skey-seq idpattern-set id-skey id-mkey \{ id-mkey \} \\
   \> skey-seq   \> :: "$<$" arg-skey \{ arg-skey \} "$>$"\\
   \> idpattern-set\> :: "[" idpattern \{ idpattern \} "]" \\
   \> arg-skey    \> :: dictionary-key \\
   \> id-skey    \> :: dictionary-key \\
   \> id-mkey    \> :: dictionary-key \\
   \> idpattern  \> :: identifier
\end {tabbing}

The idiom part is an optional part of
a lexicon entry as can be deduced from the definition of the
{\em Entry-part} ( the first definition of section 4).
An idiom specification consists of a sequence of skeys which occur in the
idiom (the {\em skey-seq} part), the set of idiom patterns which define the 
ways in which the idiom can be used in a sentence (the {\em idpattern-set} 
part), one specific skey for the idiom ({\em the id-skey}) and its 
corresponding mkey(s) ({\em id-mkey}).
\begin{verbatim}
Examples:

pijp       : 234 : ..............;
Maarten    : 745 : ..............;
aan        : 1256: ..............;
geef       : 23  : <23 234 1256 745> [ synidpat1 synidpat2 ] 23567 789
                 : ..............;
\end{verbatim}
\subsection {The ILDict part}

The syntax of the ILDICT-part is:

\begin {tabbing}
   \= ILDICT-part \= :: \= \kill
   \> ILDICT-part \> :: [ mkey-specs ] \\
   \> mkey-specs  \> :: mkey-spec \{ "," mkey-spec \} \\
   \> mkey-spec   \> :: mkey ["A"$|$"G"] ["s" pref] ["m" pref] [""" md """] [info]\\
   \> mkey        \> :: dictionary-key \\
   \> pref        \> :: number \\
   \> md          \> :: string \\
   \> info        \> :: "$<$" string "$>$"
\end {tabbing}

The ILDICT-part is optional because certain basic expressions are introduced
syncategorematically in the M-grammar. By means of the of the attribute "A" or
"G" it can be specified whether the translation is relevant for analysis or
for generation only. If the attribute is omitted the translation will be 
possible in both analysis and generation. The compiler uses the default value 
0 if the value of the preference bonusses ( {\em pref } ) is omitted. The "m"
preference bonus is used to distinguish between the alternatives during the
mkey $\rightarrow$ skey translation, whereas the "s" preference bonus is used 
to order alternatives during the skey $\rightarrow$ mkey translation.

The {\em info} part can be used to specify whether this mkey translation is meant for 
a specific target language. For instance, if the mkey is meant for a 
translation from Dutch to English, the info part will probably be:
$<$ DE $>$. The lexicon coordinator will define the strings that can be
used inside the {\em info } part. If no {\em info} part has been specified 
it is assumed that the translation is meant for all possible target languages.

\begin{verbatim}
Examples:

   loop : 525 : 10987 s1 "gaan"       ,
                10988 s2 "hard-lopen" 
                : BVERB(.........);
   loop : 788 : 12345 s1 "het lopen",
                54321 s2 "van geweer"
                : BNOUN(.........);
\end{verbatim}            
                                

\subsection {The BLex part}

The syntax of the BLEX-part is:

\begin {tabbing}
   \= BLEX-part \= \kill
   \> BLEX-part \> :: [ category "(" [ attrvalue \{ "," attrvalue \} ] ")" ] \\
\end {tabbing}

The order and values of the attribute in the BLEX-part specification should be 
according to DOMAIN-T. 
\begin{verbatim}
Example:

welk :176: 32400 :
         BDET (
         {req:}             omegapol,
         {possnietnp:}      false,
         {definite:}        indef,
         {posspred:}        false,
         {possnumbers:}     [singular, plural],
         {posscomas:}       [count, mass],
         {detnpmood:}       wh,
         {eFormation:}      RegEformation,
         {enFormation:}     false);
vulkaan :207: 457890 : BNOUN (
         {req:}              omegapol,
         {pluralforms:}      [enPlural],
         {genders:}          [mascgender],  {*morph* ++} 
         {class:}            omegaTimeAdvClass,
         {deixis:}           omegadeixis,
         {aspect:}           omegaAspect,
         {retro:}            false,
         {sexes:}            [],                    {9/3}
         {subcs:}            [othernoun],
         {temporal:}         false,
         {possgeni:}         false,
         {animate:}          inanimate,
         {human:}            nohuman,              {9/3}
         {posscomas:}        [count],
         {thetanp:}          omegathetanp,
         {nounpatterns:}     [],
         {prepkey:}          0,
         {personal:}         true);
\end{verbatim}

The BLEX-part is optional because it is possible that
several MDICT entries correspond with one and the same BLEX entry. In these 
cases the BLEX-part should be specified in one of the dictionary entries, e.g.
the determiners "a" and "an" in ENGLISH should be specified as:

\begin{verbatim}
   a   CCcons   : 512 : :DET();
   an  CCvowel  : 512 : :;

\end{verbatim}

\section{Lexicon constraints}

The lexicon constraints must be specified in a separate file with the
name {\bf constraints.constr}. In the file only constraints on the level of 
records may be specified, that is, dependencies between attributes or
sets of attributes within one record. The syntax of the constraint 
specification is as follows:

\begin {tabbing}
   \= constraint-spec \= \kill
   \> constraint-spec \> :: \{ cat-section \} "@" \\
   \> cat-section     \> :: "$<$" category \{ clause \} "$>$" \\
   \> clause          \> :: ":" bool-expression errormessage \\
   \> errormessage    \> :: [ peculiar ] "{\bf "}" string "{\bf "}" \\
   \> peculiar        \> :: "\$P" 
\end {tabbing}

The file {\bf constraints.constr} may contain for each lexical category
one {\em cat-section} in which the constraints for the records corresponding
that syntactic {\em category} can be specified. Such a specification may 
consist of several {\em clauses}. Each {\em clause} contains a boolean 
expression {\em bool-expression} on the set of attributes of the current 
category and an {\em errormessage}. The boolean expression is evaluated
for each record of syntactic category {\em category} in the lexicon. 
If the evaluation
yields the value "FALSE" then the constraint is violated 
and the specified {\em errormessage} is printed by the
lexicon compiler. The errormessage can be preceded by the {\em peculiar}
symbol. The expressions followed by the peculiar symbol are not interpreted
as real constraints but are used to inform the user of lexical
entries with a peculiar record value. Normal constraint violation
messages are preceded by the string "WARNING", peculiar messages, however,
are preceded by "PECULIAR".
Constraint conflicts are not interpreted as errors by the compiler,
they will only cause warning messages.

The syntax of the boolean expression is equivalent to the PASCAL syntax
of boolean expression, that is, operators like `AND', `OR' and `NOT' can be 
used. Furthermore, there are two predefined boolean functions, `IMPLIES'
and `IFF' each taking two parameters of type boolean. The semantics of
these functions is as follows:: \\
\begin{tabbing}
IMLPLIES$(a, b) \equiv \neg a \vee b$ \\
IFF$(a, b) \equiv (a \wedge b) \vee (\neg a \wedge \neg b) $
\end{tabbing}

\begin{verbatim}
Example

< BVERB
:  IMPLIES((particle <> 0), (verbraiser = noVR))    
      "Particle and verbraiser mutually exclusive"           
:  IMPLIES(((prepkey1 <> 0) AND 
             (verbraiser IN [optionalVR, obligatoryVR])),
             false)                                             
      $P "verbraisers take no prepositional objects"
>

< BNOUN
:  IMPLIES((human = yeshuman),(animate = yesanimate)) 
      "yeshuman implies yesanimate"
>
\end{verbatim}

\section { Compiler usage }

At the moment the lexicon is divided into a number of sub-lexicons. Each
sub-lexicon contains entries with a specific syntactical category, i.e.:

\begin{verbatim}
  bnoun.dict
  bverb.dict
  badv.dict
  badj.dict
  prep.dict
  particle.dict
  pronoun.dict
  conj.dict
  misc.dict
  auxverb.dict
  
\end{verbatim}

Each of these sub-lexicons must obey the syntax which is defined in the previous
sections of this document.

The correctness of the sub-lexicons, that is, the correctness according to
the lexicon syntax and the correctness according to DOMAIN-T, can be checked
by means of the following command:

\begin{verbatim}
   lbuild dutch:blexbnoun.isf
or
   sbuild english:blexprep.isf
etc.
\end{verbatim}

The final Rosetta3 lexicons ( that is, blex, fixid, sdict, siddict, iddict, 
mdict and 
ildict) for a specific language can be created by entering one single command, 
as follows:

\begin{verbatim}
   lbuild dutch:blex.isf
\end{verbatim}

\section{Key definitions}

The syntax definitions of the previous sections specified that the keys
of the lexicons should be {\em dictionary-keys}. In this section we will give
a definition of the the term dictionary-key.

In order to improve the readability of the dictionary we allow dictionary keys
to be of type string preceded by a dollar character `\$':
\begin {tabbing}
   \= dictionary-key \= \kill
   \> dictionary-key \> :: ("\$" string) $\mid$ integer \\
\end {tabbing}
(To be as flexible as possible, we also allow the user to specify the internal
integer dictionary key value. This, however, may cause errors because the
integer value may already be used by the dictionary compiler itself.)

\begin{verbatim}
Examples:

   loop : $s_loop_verb : $m_loop_gaan s1 "gaan"       ,
                         $m_loop_hard s2 "hard-lopen" 
                : BVERB(.........);
   loop : $s_loop_noun : $m_het_lopen s1 "het lopen",
                         $m_loop_v_geweer s2 "van geweer"
                : BNOUN(.........);
\end{verbatim}            

There will be 3 types of dictionary keys:
\begin{enumerate}
   \item dictionary-{\em fkeys}
   \item dictionary-{\em skeys}
   \item dictionary-{\em mkeys}
\end{enumerate}

All fkeys and skeys should be
specified in file {\bf language:skeydef.kdf}
and
mkeys in {\bf interlingua:mkeydef.kdf}. The syntax of these key definition 
files is as follows:\\

\begin {tabbing}
   \= KeyDefinition \= :: \= \kill
   \> KeyDefinition \> :: \{ string \} "@" \\
   \> string        \> :: (letter $\mid$ cypher $\mid$ "\_") \{ letter $\mid$ cypher $\mid$ "\_" \} \\
\end {tabbing}

\section {Conclusions}
The dictionary approach presented in this document has certainly some 
disadvantages. It has not the userfriendly userinterface of the 
Editor-Generator we previously had in mind,  instead, the standard text editor
must be used. 
The approach is not intended for a dictionary with a large number of 
entries. If the number of entries in the dictionary grows, the entry look-up 
speed for the user will decrease .

An advantage is that we have no serious B-LEX conversion problems ( cf. document 196) .
If DOMAIN-T is altered in such a way that it is incompatible with
the dictionary it will not be possible to integrate it. Because
the dictionaries depend on the DOMAIN-T file the RBS-system will try to 
generate 
new Rosetta3 dictionaries if DOMAIN-T changes. This will fail, however, if
they are incompatible. The user can only put his stuff into the archive if he 
also updates the lexicons.

\end{document}
ROSETTA.sty
\typeout{Document Style 'Rosetta'. Version 0.4 - released  24-DEC-1987}
% 24-DEC-1987:  Date of copyright notice changed
\def\@ptsize{1}
\@namedef{ds@10pt}{\def\@ptsize{0}}
\@namedef{ds@12pt}{\def\@ptsize{2}} 
\@twosidetrue
\@mparswitchtrue
\def\ds@draft{\overfullrule 5pt} 
\@options
\input art1\@ptsize.sty\relax


\def\labelenumi{\arabic{enumi}.} 
\def\theenumi{\arabic{enumi}} 
\def\labelenumii{(\alph{enumii})}
\def\theenumii{\alph{enumii}}
\def\p@enumii{\theenumi}
\def\labelenumiii{\roman{enumiii}.}
\def\theenumiii{\roman{enumiii}}
\def\p@enumiii{\theenumi(\theenumii)}
\def\labelenumiv{\Alph{enumiv}.}
\def\theenumiv{\Alph{enumiv}} 
\def\p@enumiv{\p@enumiii\theenumiii}
\def\labelitemi{$\bullet$}
\def\labelitemii{\bf --}
\def\labelitemiii{$\ast$}
\def\labelitemiv{$\cdot$}
\def\verse{
   \let\\=\@centercr 
   \list{}{\itemsep\z@ \itemindent -1.5em\listparindent \itemindent 
      \rightmargin\leftmargin\advance\leftmargin 1.5em}
   \item[]}
\let\endverse\endlist
\def\quotation{
   \list{}{\listparindent 1.5em
      \itemindent\listparindent
      \rightmargin\leftmargin \parsep 0pt plus 1pt}\item[]}
\let\endquotation=\endlist
\def\quote{
   \list{}{\rightmargin\leftmargin}\item[]}
\let\endquote=\endlist
\def\descriptionlabel#1{\hspace\labelsep \bf #1}
\def\description{
   \list{}{\labelwidth\z@ \itemindent-\leftmargin
      \let\makelabel\descriptionlabel}}
\let\enddescription\endlist


\def\@begintheorem#1#2{\it \trivlist \item[\hskip \labelsep{\bf #1\ #2}]}
\def\@endtheorem{\endtrivlist}
\def\theequation{\arabic{equation}}
\def\titlepage{
   \@restonecolfalse
   \if@twocolumn\@restonecoltrue\onecolumn
   \else \newpage
   \fi
   \thispagestyle{empty}\c@page\z@}
\def\endtitlepage{\if@restonecol\twocolumn \else \newpage \fi}
\arraycolsep 5pt \tabcolsep 6pt \arrayrulewidth .4pt \doublerulesep 2pt 
\tabbingsep \labelsep 
\skip\@mpfootins = \skip\footins
\fboxsep = 3pt \fboxrule = .4pt 


\newcounter{part}
\newcounter {section}
\newcounter {subsection}[section]
\newcounter {subsubsection}[subsection]
\newcounter {paragraph}[subsubsection]
\newcounter {subparagraph}[paragraph]
\def\thepart{\Roman{part}} \def\thesection {\arabic{section}}
\def\thesubsection {\thesection.\arabic{subsection}}
\def\thesubsubsection {\thesubsection .\arabic{subsubsection}}
\def\theparagraph {\thesubsubsection.\arabic{paragraph}}
\def\thesubparagraph {\theparagraph.\arabic{subparagraph}}


\def\@pnumwidth{1.55em}
\def\@tocrmarg {2.55em}
\def\@dotsep{4.5}
\setcounter{tocdepth}{3}
\def\tableofcontents{\section*{Contents\markboth{}{}}
\@starttoc{toc}}
\def\l@part#1#2{
   \addpenalty{-\@highpenalty}
   \addvspace{2.25em plus 1pt}
   \begingroup
      \@tempdima 3em \parindent \z@ \rightskip \@pnumwidth \parfillskip
      -\@pnumwidth {\large \bf \leavevmode #1\hfil \hbox to\@pnumwidth{\hss #2}}
      \par \nobreak
   \endgroup}
\def\l@section#1#2{
   \addpenalty{-\@highpenalty}
   \addvspace{1.0em plus 1pt}
   \@tempdima 1.5em
   \begingroup
      \parindent \z@ \rightskip \@pnumwidth 
      \parfillskip -\@pnumwidth 
      \bf \leavevmode #1\hfil \hbox to\@pnumwidth{\hss #2}
      \par
   \endgroup}
\def\l@subsection{\@dottedtocline{2}{1.5em}{2.3em}}
\def\l@subsubsection{\@dottedtocline{3}{3.8em}{3.2em}}
\def\l@paragraph{\@dottedtocline{4}{7.0em}{4.1em}}
\def\l@subparagraph{\@dottedtocline{5}{10em}{5em}}
\def\listoffigures{
   \section*{List of Figures\markboth{}{}}
   \@starttoc{lof}}
   \def\l@figure{\@dottedtocline{1}{1.5em}{2.3em}}
   \def\listoftables{\section*{List of Tables\markboth{}{}}
   \@starttoc{lot}}
\let\l@table\l@figure


\def\thebibliography#1{
   \addcontentsline{toc}
   {section}{References}\section*{References\markboth{}{}}
   \list{[\arabic{enumi}]}
        {\settowidth\labelwidth{[#1]}\leftmargin\labelwidth
         \advance\leftmargin\labelsep\usecounter{enumi}}}
\let\endthebibliography=\endlist


\newif\if@restonecol
\def\theindex{
   \@restonecoltrue\if@twocolumn\@restonecolfalse\fi
   \columnseprule \z@
   \columnsep 35pt\twocolumn[\section*{Index}]
   \markboth{}{}
   \thispagestyle{plain}\parindent\z@
   \parskip\z@ plus .3pt\relax
   \let\item\@idxitem}
\def\@idxitem{\par\hangindent 40pt}
\def\subitem{\par\hangindent 40pt \hspace*{20pt}}
\def\subsubitem{\par\hangindent 40pt \hspace*{30pt}}
\def\endtheindex{\if@restonecol\onecolumn\else\clearpage\fi}
\def\indexspace{\par \vskip 10pt plus 5pt minus 3pt\relax}


\def\footnoterule{
   \kern-1\p@ 
   \hrule width .4\columnwidth 
   \kern .6\p@} 
\long\def\@makefntext#1{
   \@setpar{\@@par\@tempdima \hsize 
   \advance\@tempdima-10pt\parshape \@ne 10pt \@tempdima}\par
   \parindent 1em\noindent \hbox to \z@{\hss$^{\@thefnmark}$}#1}


\setcounter{topnumber}{2}
\def\topfraction{.7}
\setcounter{bottomnumber}{1}
\def\bottomfraction{.3}
\setcounter{totalnumber}{3}
\def\textfraction{.2}
\def\floatpagefraction{.5}
\setcounter{dbltopnumber}{2}
\def\dbltopfraction{.7}
\def\dblfloatpagefraction{.5}
\long\def\@makecaption#1#2{
   \vskip 10pt 
   \setbox\@tempboxa\hbox{#1: #2}
   \ifdim \wd\@tempboxa >\hsize \unhbox\@tempboxa\par
   \else \hbox to\hsize{\hfil\box\@tempboxa\hfil} 
   \fi}
\newcounter{figure}
\def\thefigure{\@arabic\c@figure}
\def\fps@figure{tbp}
\def\ftype@figure{1}
\def\ext@figure{lof}
\def\fnum@figure{Figure \thefigure}
\def\figure{\@float{figure}}
\let\endfigure\end@float
\@namedef{figure*}{\@dblfloat{figure}}
\@namedef{endfigure*}{\end@dblfloat}
\newcounter{table}
\def\thetable{\@arabic\c@table}
\def\fps@table{tbp}
\def\ftype@table{2}
\def\ext@table{lot}
\def\fnum@table{Table \thetable}
\def\table{\@float{table}}
\let\endtable\end@float
\@namedef{table*}{\@dblfloat{table}}
\@namedef{endtable*}{\end@dblfloat}


\def\maketitle{
   \par
   \begingroup
      \def\thefootnote{\fnsymbol{footnote}}
      \def\@makefnmark{\hbox to 0pt{$^{\@thefnmark}$\hss}} 
      \if@twocolumn \twocolumn[\@maketitle] 
      \else \newpage \global\@topnum\z@ \@maketitle
      \fi
      \thispagestyle{plain}
      \@thanks
   \endgroup
   \setcounter{footnote}{0}
   \let\maketitle\relax
   \let\@maketitle\relax
   \gdef\@thanks{}
   \gdef\@author{}
   \gdef\@title{}
   \let\thanks\relax}
\def\@maketitle{
   \newpage
   \null
   \vskip 2em
   \begin{center}{\LARGE \@title \par}
      \vskip 1.5em
      {\large \lineskip .5em \begin{tabular}[t]{c}\@author \end{tabular}\par} 
      \vskip 1em {\large \@date}
   \end{center}
   \par
   \vskip 1.5em} 
\def\abstract{
   \if@twocolumn \section*{Abstract}
   \else
      \small 
      \begin{center} {\bf Abstract\vspace{-.5em}\vspace{0pt}} \end{center}
      \quotation 
   \fi}
\def\endabstract{\if@twocolumn\else\endquotation\fi}


\mark{{}{}} 
\if@twoside
   \def\ps@headings{
      \def\@oddfoot{Rosetta Doc. \@RosDocNr\hfil \@RosDate}
      \def\@evenfoot{Rosetta Doc. \@RosDocNr\hfil \@RosDate}
      \def\@evenhead{\rm\thepage\hfil \sl \rightmark}
      \def\@oddhead{\hbox{}\sl \leftmark \hfil\rm\thepage}
      \def\sectionmark##1{\markboth {}{}}
      \def\subsectionmark##1{}}
\else
   \def\ps@headings{
      \def\@oddfoot{Rosetta Doc. \@RosDocNr\hfil \@RosDate}
      \def\@evenfoot{Rosetta Doc. \@RosDocNr\hfil \@RosDate}
      \def\@oddhead{\hbox{}\sl \rightmark \hfil \rm\thepage}
      \def\sectionmark##1{\markboth {}{}}
      \def\subsectionmark##1{}}
\fi
\def\ps@myheadings{
   \def\@oddhead{\hbox{}\sl\@rhead \hfil \rm\thepage}
   \def\@oddfoot{}
   \def\@evenhead{\rm \thepage\hfil\sl\@lhead\hbox{}}
   \def\@evenfoot{}
   \def\sectionmark##1{}
   \def\subsectionmark##1{}}


\def\today{
   \ifcase\month\or January\or February\or March\or April\or May\or June\or
      July\or August\or September\or October\or November\or December
   \fi
   \space\number\day, \number\year}


\ps@plain \pagenumbering{arabic} \onecolumn \if@twoside\else\raggedbottom\fi 




% the Rosetta title page
\newcommand{\MakeRosTitle}{
   \begin{titlepage}
      \begin{large}
	 \begin{figure}[t]
	    \begin{picture}(405,100)(0,0)
	       \put(0,100){\line(1,0){404}}
	       \put(0,75){Project {\bf Rosetta}}
	       \put(93.5,75){:}
	       \put(108,75){Machine Translation}
	       \put(0,50){Topic}
	       \put(93.5,50){:}
	       \put(108,50){\@RosTopic}
	       \put(0,30){\line(1,0){404}}
	    \end{picture}
	 \end{figure}
	 \bigskip
	 \bigskip
	 \begin{list}{-}{\setlength{\leftmargin}{3.0cm}
			 \setlength{\labelwidth}{2.7cm}
			 \setlength{\topsep}{2cm}}
	    \item [{\rm Title \hfill :}] {{\bf \@RosTitle}}
	    \item [{\rm Author \hfill :}] {\@RosAuthor}
	    \bigskip
	    \bigskip
	    \bigskip
	    \item [{\rm Doc.Nr. \hfill :}] {\@RosDocNr}
	    \item [{\rm Date \hfill :}] {\@RosDate}
	    \item [{\rm Status \hfill :}] {\@RosStatus}
	    \item [{\rm Supersedes \hfill :}] {\@RosSupersedes}
	    \item [{\rm Distribution \hfill :}] {\@RosDistribution}
	    \item [{\rm Clearance \hfill :}] {\@RosClearance}
	    \item [{\rm Keywords \hfill :}] {\@RosKeywords}
	 \end{list}
      \end{large}
      \title{\@RosTitle}
      \begin{figure}[b]
	 \begin{picture}(404,64)(0,0)
	    \put(0,64){\line(1,0){404}}
	    \put(0,-4){\line(1,0){404}}
	    \put(0,59){\line(1,0){42}}
	    \begin{small}
	    \put(3,48){\sf PHILIPS}
	    \end{small}
	    \put(0,23){\line(0,1){36}}
	    \put(42,23){\line(0,1){36}}
	    \put(21,23){\oval(42,42)[bl]}
	    \put(21,23){\oval(42,42)[br]}
	    \put(21,23){\circle{40}}
	    \put(4,33){\line(1,0){10}}
	    \put(9,28){\line(0,1){10}}
	    \put(9,36){\line(1,0){6}}
	    \put(12,33){\line(0,1){6}}
	    \put(29,13){\line(1,0){10}}
	    \put(34,8){\line(0,1){10}}
	    \put(28,10){\line(1,0){6}}
	    \put(31,7){\line(0,1){6}}

	    \put(1,21){\line(1,0){0.5}}
	    \put(1.5,21.3){\line(1,0){0.5}}
	    \put(2,21.6){\line(1,0){0.5}}
	    \put(2.5,21.9){\line(1,0){0.5}}
	    \put(3,22.1){\line(1,0){0.5}}
	    \put(3.5,22.3){\line(1,0){0.5}}
	    \put(4,22.5){\line(1,0){0.5}}
	    \put(4.5,22.7){\line(1,0){0.5}}
	    \put(5,22.8){\line(1,0){0.5}}
	    \put(5.5,22.9){\line(1,0){0.5}}
	    \put(6,23){\line(1,0){0.5}}
	    \put(6.5,22.9){\line(1,0){0.5}}
	    \put(7,22.8){\line(1,0){0.5}}
	    \put(7.5,22.7){\line(1,0){0.5}}
	    \put(8,22.5){\line(1,0){0.5}}
	    \put(8.5,22.3){\line(1,0){0.5}}
	    \put(9,22.1){\line(1,0){0.5}}
	    \put(9.5,21.9){\line(1,0){0.5}}
	    \put(10,21.6){\line(1,0){0.5}}
	    \put(10.5,21.3){\line(1,0){0.5}}

	    \put(1,23){\line(1,0){0.5}}
	    \put(1.5,23.3){\line(1,0){0.5}}
	    \put(2,23.6){\line(1,0){0.5}}
	    \put(2.5,23.9){\line(1,0){0.5}}
	    \put(3,24.1){\line(1,0){0.5}}
	    \put(3.5,24.3){\line(1,0){0.5}}
	    \put(4,24.5){\line(1,0){0.5}}
	    \put(4.5,24.7){\line(1,0){0.5}}
	    \put(5,24.8){\line(1,0){0.5}}
	    \put(5.5,24.9){\line(1,0){0.5}}
	    \put(6,25){\line(1,0){0.5}}
	    \put(6.5,24.9){\line(1,0){0.5}}
	    \put(7,24.8){\line(1,0){0.5}}
	    \put(7.5,24.7){\line(1,0){0.5}}
	    \put(8,24.5){\line(1,0){0.5}}
	    \put(8.5,24.3){\line(1,0){0.5}}
	    \put(9,24.1){\line(1,0){0.5}}
	    \put(9.5,23.9){\line(1,0){0.5}}
	    \put(10,23.6){\line(1,0){0.5}}
	    \put(10.5,23.3){\line(1,0){0.5}}

	    \put(1,25){\line(1,0){0.5}}
	    \put(1.5,25.3){\line(1,0){0.5}}
	    \put(2,25.6){\line(1,0){0.5}}
	    \put(2.5,25.9){\line(1,0){0.5}}
	    \put(3,26.1){\line(1,0){0.5}}
	    \put(3.5,26.3){\line(1,0){0.5}}
	    \put(4,26.5){\line(1,0){0.5}}
	    \put(4.5,26.7){\line(1,0){0.5}}
	    \put(5,26.8){\line(1,0){0.5}}
	    \put(5.5,26.9){\line(1,0){0.5}}
	    \put(6,27){\line(1,0){0.5}}
	    \put(6.5,26.9){\line(1,0){0.5}}
	    \put(7,26.8){\line(1,0){0.5}}
	    \put(7.5,26.7){\line(1,0){0.5}}
	    \put(8,26.5){\line(1,0){0.5}}
	    \put(8.5,26.3){\line(1,0){0.5}}
	    \put(9,26.1){\line(1,0){0.5}}
	    \put(9.5,25.9){\line(1,0){0.5}}
	    \put(10,25.6){\line(1,0){0.5}}
	    \put(10.5,25.3){\line(1,0){0.5}}

	    \put(11,21){\line(1,0){0.5}}
	    \put(11.5,20.7){\line(1,0){0.5}}
	    \put(12,20.4){\line(1,0){0.5}}
	    \put(12.5,20.1){\line(1,0){0.5}}
	    \put(13,19.9){\line(1,0){0.5}}
	    \put(13.5,19.7){\line(1,0){0.5}}
	    \put(14,19.5){\line(1,0){0.5}}
	    \put(14.5,19.3){\line(1,0){0.5}}
	    \put(15,19.2){\line(1,0){0.5}}
	    \put(15.5,19.1){\line(1,0){0.5}}
	    \put(16,19){\line(1,0){0.5}}
	    \put(16.5,19.1){\line(1,0){0.5}}
	    \put(17,19.2){\line(1,0){0.5}}
	    \put(17.5,19.3){\line(1,0){0.5}}
	    \put(18,19.5){\line(1,0){0.5}}
	    \put(18.5,19.7){\line(1,0){0.5}}
	    \put(19,19.9){\line(1,0){0.5}}
	    \put(19.5,20.1){\line(1,0){0.5}}
	    \put(20,20.4){\line(1,0){0.5}}
	    \put(20.5,20.7){\line(1,0){0.5}}

	    \put(11,23){\line(1,0){0.5}}
	    \put(11.5,22.7){\line(1,0){0.5}}
	    \put(12,22.4){\line(1,0){0.5}}
	    \put(12.5,22.1){\line(1,0){0.5}}
	    \put(13,21.9){\line(1,0){0.5}}
	    \put(13.5,21.7){\line(1,0){0.5}}
	    \put(14,21.5){\line(1,0){0.5}}
	    \put(14.5,21.3){\line(1,0){0.5}}
	    \put(15,21.2){\line(1,0){0.5}}
	    \put(15.5,21.1){\line(1,0){0.5}}
	    \put(16,21){\line(1,0){0.5}}
	    \put(16.5,21.1){\line(1,0){0.5}}
	    \put(17,21.2){\line(1,0){0.5}}
	    \put(17.5,21.3){\line(1,0){0.5}}
	    \put(18,21.5){\line(1,0){0.5}}
	    \put(18.5,21.7){\line(1,0){0.5}}
	    \put(19,21.9){\line(1,0){0.5}}
	    \put(19.5,22.1){\line(1,0){0.5}}
	    \put(20,22.4){\line(1,0){0.5}}
	    \put(20.5,22.7){\line(1,0){0.5}}

	    \put(11,25){\line(1,0){0.5}}
	    \put(11.5,24.7){\line(1,0){0.5}}
	    \put(12,24.4){\line(1,0){0.5}}
	    \put(12.5,24.1){\line(1,0){0.5}}
	    \put(13,23.9){\line(1,0){0.5}}
	    \put(13.5,23.7){\line(1,0){0.5}}
	    \put(14,23.5){\line(1,0){0.5}}
	    \put(14.5,23.3){\line(1,0){0.5}}
	    \put(15,23.2){\line(1,0){0.5}}
	    \put(15.5,23.1){\line(1,0){0.5}}
	    \put(16,23){\line(1,0){0.5}}
	    \put(16.5,23.1){\line(1,0){0.5}}
	    \put(17,23.2){\line(1,0){0.5}}
	    \put(17.5,23.3){\line(1,0){0.5}}
	    \put(18,23.5){\line(1,0){0.5}}
	    \put(18.5,23.7){\line(1,0){0.5}}
	    \put(19,23.9){\line(1,0){0.5}}
	    \put(19.5,24.1){\line(1,0){0.5}}
	    \put(20,24.4){\line(1,0){0.5}}
	    \put(20.5,24.7){\line(1,0){0.5}}

	    \put(21,21){\line(1,0){0.5}}
	    \put(21.5,21.3){\line(1,0){0.5}}
	    \put(22,21.6){\line(1,0){0.5}}
	    \put(22.5,21.9){\line(1,0){0.5}}
	    \put(23,22.1){\line(1,0){0.5}}
	    \put(23.5,22.3){\line(1,0){0.5}}
	    \put(24,22.5){\line(1,0){0.5}}
	    \put(24.5,22.7){\line(1,0){0.5}}
	    \put(25,22.8){\line(1,0){0.5}}
	    \put(25.5,23.9){\line(1,0){0.5}}
	    \put(26,23){\line(1,0){0.5}}
	    \put(26.5,22.9){\line(1,0){0.5}}
	    \put(27,22.8){\line(1,0){0.5}}
	    \put(27.5,22.7){\line(1,0){0.5}}
	    \put(28,22.5){\line(1,0){0.5}}
	    \put(28.5,22.3){\line(1,0){0.5}}
	    \put(29,22.1){\line(1,0){0.5}}
	    \put(29.5,21.9){\line(1,0){0.5}}
	    \put(30,21.6){\line(1,0){0.5}}
	    \put(30.5,21.3){\line(1,0){0.5}}

	    \put(21,23){\line(1,0){0.5}}
	    \put(21.5,23.3){\line(1,0){0.5}}
	    \put(22,23.6){\line(1,0){0.5}}
	    \put(22.5,23.9){\line(1,0){0.5}}
	    \put(23,24.1){\line(1,0){0.5}}
	    \put(23.5,24.3){\line(1,0){0.5}}
	    \put(24,24.5){\line(1,0){0.5}}
	    \put(24.5,24.7){\line(1,0){0.5}}
	    \put(25,24.8){\line(1,0){0.5}}
	    \put(25.5,24.9){\line(1,0){0.5}}
	    \put(26,25){\line(1,0){0.5}}
	    \put(26.5,24.9){\line(1,0){0.5}}
	    \put(27,24.8){\line(1,0){0.5}}
	    \put(27.5,24.7){\line(1,0){0.5}}
	    \put(28,24.5){\line(1,0){0.5}}
	    \put(28.5,24.3){\line(1,0){0.5}}
	    \put(29,24.1){\line(1,0){0.5}}
	    \put(29.5,23.9){\line(1,0){0.5}}
	    \put(30,23.6){\line(1,0){0.5}}
	    \put(30.5,23.3){\line(1,0){0.5}}

	    \put(21,25){\line(1,0){0.5}}
	    \put(21.5,25.3){\line(1,0){0.5}}
	    \put(22,25.6){\line(1,0){0.5}}
	    \put(22.5,25.9){\line(1,0){0.5}}
	    \put(23,26.1){\line(1,0){0.5}}
	    \put(23.5,26.3){\line(1,0){0.5}}
	    \put(24,26.5){\line(1,0){0.5}}
	    \put(24.5,26.7){\line(1,0){0.5}}
	    \put(25,26.8){\line(1,0){0.5}}
	    \put(25.5,26.9){\line(1,0){0.5}}
	    \put(26,27){\line(1,0){0.5}}
	    \put(26.5,26.9){\line(1,0){0.5}}
	    \put(27,26.8){\line(1,0){0.5}}
	    \put(27.5,26.7){\line(1,0){0.5}}
	    \put(28,26.5){\line(1,0){0.5}}
	    \put(28.5,26.3){\line(1,0){0.5}}
	    \put(29,26.1){\line(1,0){0.5}}
	    \put(29.5,25.9){\line(1,0){0.5}}
	    \put(30,25.6){\line(1,0){0.5}}
	    \put(30.5,25.3){\line(1,0){0.5}}

	    \put(31,21){\line(1,0){0.5}}
	    \put(31.5,20.7){\line(1,0){0.5}}
	    \put(32,20.4){\line(1,0){0.5}}
	    \put(32.5,20.1){\line(1,0){0.5}}
	    \put(33,19.9){\line(1,0){0.5}}
	    \put(33.5,19.7){\line(1,0){0.5}}
	    \put(34,19.5){\line(1,0){0.5}}
	    \put(34.5,19.3){\line(1,0){0.5}}
	    \put(35,19.2){\line(1,0){0.5}}
	    \put(35.5,19.1){\line(1,0){0.5}}
	    \put(36,19){\line(1,0){0.5}}
	    \put(36.5,19.1){\line(1,0){0.5}}
	    \put(37,19.2){\line(1,0){0.5}}
	    \put(37.5,19.3){\line(1,0){0.5}}
	    \put(38,19.5){\line(1,0){0.5}}
	    \put(38.5,19.7){\line(1,0){0.5}}
	    \put(39,19.9){\line(1,0){0.5}}
	    \put(39.5,20.1){\line(1,0){0.5}}
	    \put(40,20.4){\line(1,0){0.5}}
	    \put(40.5,20.7){\line(1,0){0.5}}

	    \put(31,23){\line(1,0){0.5}}
	    \put(31.5,22.7){\line(1,0){0.5}}
	    \put(32,22.4){\line(1,0){0.5}}
	    \put(32.5,22.1){\line(1,0){0.5}}
	    \put(33,21.9){\line(1,0){0.5}}
	    \put(33.5,21.7){\line(1,0){0.5}}
	    \put(34,21.5){\line(1,0){0.5}}
	    \put(34.5,21.3){\line(1,0){0.5}}
	    \put(35,21.2){\line(1,0){0.5}}
	    \put(35.5,21.1){\line(1,0){0.5}}
	    \put(36,21){\line(1,0){0.5}}
	    \put(36.5,21.1){\line(1,0){0.5}}
	    \put(37,21.2){\line(1,0){0.5}}
	    \put(37.5,21.3){\line(1,0){0.5}}
	    \put(38,21.5){\line(1,0){0.5}}
	    \put(38.5,21.7){\line(1,0){0.5}}
	    \put(39,21.9){\line(1,0){0.5}}
	    \put(39.5,22.1){\line(1,0){0.5}}
	    \put(40,22.4){\line(1,0){0.5}}
	    \put(40.5,22.7){\line(1,0){0.5}}

	    \put(31,25){\line(1,0){0.5}}
	    \put(31.5,24.7){\line(1,0){0.5}}
	    \put(32,24.4){\line(1,0){0.5}}
	    \put(32.5,24.1){\line(1,0){0.5}}
	    \put(33,23.9){\line(1,0){0.5}}
	    \put(33.5,23.7){\line(1,0){0.5}}
	    \put(34,23.5){\line(1,0){0.5}}
	    \put(34.5,23.3){\line(1,0){0.5}}
	    \put(35,23.2){\line(1,0){0.5}}
	    \put(35.5,23.1){\line(1,0){0.5}}
	    \put(36,23){\line(1,0){0.5}}
	    \put(36.5,23.1){\line(1,0){0.5}}
	    \put(37,23.2){\line(1,0){0.5}}
	    \put(37.5,23.3){\line(1,0){0.5}}
	    \put(38,23.5){\line(1,0){0.5}}
	    \put(38.5,23.7){\line(1,0){0.5}}
	    \put(39,23.9){\line(1,0){0.5}}
	    \put(39.5,24.1){\line(1,0){0.5}}
	    \put(40,24.4){\line(1,0){0.5}}
	    \put(40.5,24.7){\line(1,0){0.5}}
	    \begin{large}
	       \put(60,45){Philips Research Laboratories}
	       \put(60,30){\copyright\ 1988 Nederlandse Philips Bedrijven B.V.}
	    \end{large}
	 \end{picture}
      \end{figure}
      \newpage
      \pagenumbering{roman}
      \tableofcontents
      \newpage
      \pagenumbering{arabic}
   \end{titlepage}
}
\title{}
\topmargin 0pt
\oddsidemargin 36pt
\evensidemargin 36pt
\textheight 600pt
\textwidth 405pt
\pagestyle{headings}
\newcommand{\@RosTopic}{General}
\newcommand{\@RosTitle}{-}
\newcommand{\@RosAuthor}{-}
\newcommand{\@RosDocNr}{}
\newcommand{\@RosDate}{}
\newcommand{\@RosStatus}{informal}
\newcommand{\@RosSupersedes}{-}
\newcommand{\@RosDistribution}{Project}
\newcommand{\@RosClearance}{Project}
\newcommand{\@RosKeywords}{}
\newcommand{\RosTopic}[1]{\renewcommand{\@RosTopic}{#1}}
\newcommand{\RosTitle}[1]{\renewcommand{\@RosTitle}{#1}}
\newcommand{\RosAuthor}[1]{\renewcommand{\@RosAuthor}{#1}}
\newcommand{\RosDocNr}[1]{\renewcommand{\@RosDocNr}{#1 (RWR-102-RO-90#1-RO)}}
\newcommand{\RosDate}[1]{\renewcommand{\@RosDate}{#1}}
\newcommand{\RosStatus}[1]{\renewcommand{\@RosStatus}{#1}}
\newcommand{\RosSupersedes}[1]{\renewcommand{\@RosSupersedes}{#1}}
\newcommand{\RosDistribution}[1]{\renewcommand{\@RosDistribution}{#1}}
\newcommand{\RosClearance}[1]{\renewcommand{\@RosClearance}{#1}}
\newcommand{\RosKeywords}[1]{\renewcommand{\@RosKeywords}{#1}}

