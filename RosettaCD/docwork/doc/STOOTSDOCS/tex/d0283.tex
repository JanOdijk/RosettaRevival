\documentstyle{Rosetta}
\begin{document}
   \RosTopic{Rosetta3.software}
   \RosTitle{Process Communication in Rosetta3}
   \RosAuthor{Joep Rous}
   \RosDocNr{283}
   \RosDate{July 28, 1989}
   \RosStatus{concept}
   \RosSupersedes{12 October, 1988}
   \RosDistribution{Project}
   \RosClearance{Project}
   \RosKeywords{}
   \MakeRosTitle
\section{Introduction}
The Rosetta3 system in execution 
consists of three processes: the Control process, the 
Analysis process and the Generation process. The choice for this structure is
motivated in document 73; "Global design decisions".
For the communication between the processes the mailbox concept is used.
There is, 
however, one exception. The result of applying Analysis to an input sentence is
a set of IL-trees. These IL-trees have to be passed to the Generation process.
Since for the implementation of such a set of IL-trees a large amount of data 
might be needed, mailboxes are not the best choice for efficiency
reasons. For this kind of data transfer we make use of a
global buffer which is shared between the Analysis and Generation process.


\section{Mailbox Communication}
\subsection{Mailbox functions}
In this section we will give an overview of the primitive operations on 
mailboxes that have been implemented in Rosetta.
\begin{description}
\item[MB\_Open(mbxname, access, create, \underline{var} mbxid)] Opens a mailbox for 
        use by the current process. \\
        - The name of the mailbox that is to be opened should be specified by 
        means of the parameter {\bf mbxname}. \\
        - The {\bf access} parameter can be used to 
        specify whether the mailbox is to be used for {\em taking} or 
        {\em putting}. \\
        - The {\bf create} parameter indicates whether a new mailbox 
        must be created or an existing one must be used. If a new mailbox must 
        be created, the value of the parameter {mbxname} must be unique with
        respect to the current process tree. If an existing one must be used
        that mailbox should already be created by another process with the name
        that is specified in {\bf mbxname}.\\
        - If the mailbox can be 
        opened a unique identification of the mailbox is returned by means of 
        the parameter {\bf mbxid}. This identification should be used in subsequent
        mailbox calls.\\
\item[MB\_Close(mbxid, delete)] Closes a mailbox. After a call to this
        function the mailbox cannot be used for communication anymore.\\
        - The unique identification of the mailbox that is to be closed should
        be specified by means of parameter {\bf mbxid}. \\
        - The {\bf delete} parameter specifies whether the mailbox should be 
        deleted after it has been closed. This is only possible if the current
        process has also created the mailbox. A mailbox can only be deleted if
        there is no connection to any other process.
\item[MB\_Take(mbxid, \underline{var} msg )] Takes a message from the 
        specified mailbox. If there is no message in the mailbox, the current
        process will be suspended until a message arrives. \\
        - The unique identification of the mailbox that is to be used should
        be specified by means of parameter {\bf mbxid}. \\
        - The message that is to be read is assigned to parameter {\bf msg}.
\item[MB\_Put(mbxid, msg)] Puts a message into the 
        specified mailbox. The current
        process will be suspended until the message is read by some other 
        process.\\
        - The unique identification of the mailbox that is to be used should
        be specified by means of parameter {\bf mbxid}. \\
        - The message that is to be put into the mailbox is specified by means
        of parameter {\bf msg}.
\end{description}
\subsection{Infra Structure}
In Rosetta each process will have its own mailbox. Furthermore, there will be 
only direct communication between the Analysis and the Control process and 
between the Generation and the Control process. Therefore, there will be no
connection between the Generation mailbox and the Analysis process and between
the Analysis mailbox and the Generation process.
\begin{verbatim}

                           CONTROL

                          mbxcontrol


                   mbxanalysis  mbxgeneration

              ANALYSIS               GENERATION
\end{verbatim}
The direction of the arrows in the above figure indicates whether the mailbox
is used by a process for {\em putting} or for {\em taking}.
\subsection{Process Synchronization}
Except for information exchange, mailboxes are used to synchronize the 
Analysis, Generation and Control processes. This is possible because MB\_Put and
MB\_Take are synchronous functions. The Rosetta system is activated by starting
the Control process. After an initialization phase in which Control creates its
own mailbox, the Analysis and Generation processes are created. The Control 
process waits after each process creation until it receives a message from the
created process. This message indicates that the process has finished its 
initialization phase in which it has opened the Control mailbox and in which
it has created its own mailbox. Next, the Control process can open the mailbox
of the created process: \\
\begin{tabbing}
123 \= \kill
{\bf program} Control; \\
{\bf begin} \\
\> MB\_Open('control', takeaccess, yes, mbxcontrol); \\
\> {\em ...create analysis process and wait for ready message....} \\
\> MB\_Take(mbxcontrol, analysis\_ready\_msg\_1) \\
\> MB\_Open('analysis', putaccess, no, mbxanalysis); \\
\> {\em ...create generation process and wait for ready message....} \\
\> MB\_Take(mbxcontrol, generation\_ready\_msg\_1) \\
\> MB\_Open('generation', putaccess, no, mbxgeneration); \\
\> ..... \\
{\bf end.}
\end{tabbing}
\begin{tabbing}
123 \= \kill
{\bf program} Analysis; \\
{\bf begin} \\
\> MB\_Open('analysis', takeaccess, yes, mbxanalysis); \\
\> MB\_Open('control', putaccess, no, mbxcontrol); \\
\> {\em ...initialize process and inform Control when ready....} \\
\> MB\_Put(mbxcontrol, analysis\_ready\_msg\_1) \\
\> ..... \\
{\bf end.}
\end{tabbing}
\begin{tabbing}
123 \= \kill
{\bf program} Generation; \\
{\bf begin} \\
\> MB\_Open('generation', takeaccess, yes, mbxgeneration); \\
\> MB\_Open('control', putaccess, no, mbxcontrol); \\
\> {\em ...initialize process and inform Control when ready....} \\
\> MB\_Put(mbxcontrol, generation\_ready\_msg\_1) \\
\> ..... \\
{\bf end.}
\end{tabbing}
Now Analysis and Generation are ready to execute the program body. First, the
Analysis process is started. Together with the start message Control sends 
information about debugging, printing etc.. The Analysis process executes its body
and informs the Control process when it is ready. This ready message also 
contains information about the result of Analysis. If the result is correct 
then the Control process activates the Generation process. The result of the
application of Generation may be ambiguous. Each time the Generation process 
finds a result it sends a synchronization message to Control. In Control it
is decided ( by the user )
whether the Generation process should be stopped, reset or continued.

If the Generation process is ready, that is, if it has presented all its 
results or if the user has decided to continue with ( the analysis ) of a 
new sentence, both Analysis and Generation have to clear their windows on the
terminal screen. First Generation clears its screen, sends a message to
Control and waits for the next sentence. Next, Analysis is triggered by a 
message from Control to clear its screen and to wait for the next sentence (if 
any). 
\begin{tabbing}
123\= 123\= 123\= 123\= \kill
{\bf program} Control; \\
{\bf begin} \\
\> ..... \\
\> {\em ...ask user about debugging, printing, etc....} \\
\> {\bf repeat} \\
\> \> MB\_Put(mbxanalysis, analysis\_start\_msg) \\
\> \> {\em ...wait until analysis is ready....} \\
\> \> MB\_Take(mbxcontrol, analysis\_ready\_msg\_2) \\
\> \> {\bf if} {\em analysis result is correct} {\bf then} \\
\> \> \>{\bf repeat} \\
\> \> \> \> MB\_Put(mbxgeneration, generation\_start\_msg) \\
\> \> \> \> {\em ...wait until generation is ready....} \\
\> \> \> \> MB\_Take(mbxcontrol, generation\_ready\_msg\_2) \\
\> \> \> \> {\em ask user whether generation should be stopped, reset, continued..} \\
\> \> \> {\bf until} {\em stop or reset} \\
\> \> \> MB\_Put(mbxgeneration, generation\_stop-reset\_msg) \\
\> \> \> {\em ...wait until generation ready....} \\
\> \> \> MB\_Take(mbxcontrol, generation\_ready\_msg\_3) \\
\> \> {\bf else} \\
\> \> \> {\em ask user whether system should be stopped or reset..} \\
\> \> {\bf fi} \\
\> \> MB\_Put(mbxanalysis, analysis\_clear\_window\_msg) \\
\> \> {\em ...ask user about debugging, printing, etc....} \\
\> {\bf until } {\em stop } \\
\> {\em ...stop the analysis and generation process ...} \\
\> MB\_Put(mbxanalysis, analysis\_stop\_msg) \\
\> MB\_Put(mbxgeneration, generation\_stop\_msg) \\
\> ..... \\
{\bf end.} \\
\end{tabbing}
\begin{tabbing}
123\= 123\= 123\= 123\= \kill
{\bf program} Analysis; \\
{\bf begin} \\
\> ..... \\
\> {\em ...wait for permission to continue with process body...} \\
\> MB\_Take(mbxanalysis, analysis\_start\_msg); \\
\> {\bf while} {\em message contains no stop instruction} {\bf do} \\
\> \> {\em ..execute analysis body and evaluate result..} \\
\> \> MB\_Put(mbxcontrol, analysis\_ready\_msg\_2) \\
\> \> MB\_Take(mbxanalysis, analysis\_clear\_window\_msg); \\
\> \> {\em ...clear the analysis window...} \\
\> \> MB\_Take(mbxanalysis, analysis\_start/stop\_msg); \\
\> {\bf od} \\
\> ..... \\
{\bf end.} \\
\end{tabbing}
\begin{tabbing}
123\= 123\= 123\= 123\= \kill
{\bf program} Generation; \\
{\bf begin} \\
\> ..... \\
\> {\em ...wait for permission to continue with process body...} \\
\> MB\_Take(mbxgeneration, generation\_start\_msg); \\
\> {\bf while} {\em message contains no stop instruction} {\bf do} \\
\> \> {\em ..execute generation body (backtracking) and evaluate result..} \\
\> \> {\bf while} {\em still results} {\bf and not} {\em (stop or reset)} {\bf do} \\
\> \> \> MB\_Put(mbxcontrol, generation\_ready\_msg\_2) \\
\> \> \> MB\_Take(mbxgeneration, generation\_start/stop-reset\_msg); \\
\> \> {\bf od} \\
\> \> {\bf if} {\em no results} {\bf then} \\
\> \> \> MB\_Put(mbxcontrol, generation\_ready\_msg\_2) \\
\> \> \> MB\_Take(mbxgeneration, generation\_stop-reset\_msg); \\
\> \> {\bf fi} \\
\> \> {\em ...clear the generation window...} \\
\> \> MB\_Put(mbxcontrol, generation\_ready\_msg\_3) \\
\> \> MB\_Take(mbxgeneration, generation\_start/stop\_msg); \\
\> {\bf od} \\
\> ..... \\
{\bf end.} \\
\end{tabbing}
\subsection{Message Structure}
All messages that are passed between the Analysis, Generation and Control 
process have the same structure. The message is implemented as a record
in which different fields are defined for storing different kinds of information.
If a process receives a message, it reads only the fields which are relevant at
that specific point in the program. If a process receives a wrong message
the behaviour of the system is undefined. The system will probably crash 
because of a run-time error or it will 
be put into a wait state from which it cannot be activated. 

The structure of the message record is :
\begin{verbatim}
   communicationblock = record
                           msg                : msgtype;
                           nextsyn,
                           nextlexsense,
                           nextstructsense    : boolean;
                           clearwindow        : boolean;
                           stoplevel          : leveltype;
                           intmode,
                           batchmode          : boolean;
                           printerf,
                           debug              : array[leveltype] of boolean;
                           ifdescr            : ifdescrtype
                        end;
\end{verbatim}
The meaning of the fields in this recordtype is as follows:
\begin{description}
   \item[msg] Will be explained below.
   \item[nextsyn] Used by the generation process to inform control that there
                  are syntactic ambiguous translations
                  (used in: generation\_ready\_msg\_2).
   \item[nextlexsense] Used by the generation process to inform control that
                  there are lexical ambiguous translations
                   (used in: generation\_ready\_msg\_2)
   \item[nextstructsense] Used by generation to inform control that there 
                  are structural ambiguous translations
                   (used in: generation\_ready\_msg\_2).
   \item[clearwindow] If TRUE then the screen window of the current process 
           should 
           be cleared. (used in : generation\_stop-reset\_msg,
           analysis\_clear\_window\_msg)
   \item[stoplevel] Specifies the last component of the system which is to be 
           applied. 
           (used in : analysis\_start\_message, 
           generation\_start\_message)
   \item[intmode] Specifies whether the system  runs in 
           interactive mode.
           (used in : analysis\_start\_message, 
           generation\_start\_message)
    \item[batchmode] Specifies whether the system  runs in 
           batchmode.
           (used in : analysis\_start\_message, 
           generation\_start\_message)
   \item[debug] Specifies which component should be executed in debug mode.
           (used in : analysis\_start\_message, 
           generation\_start\_message)
   \item[printerf] Specifies which component interfaces should be presented on
           the terminal screen.
           (used in : analysis\_start\_message, 
           generation\_start\_message)
   \item[ifdescr] Is used to pass the final result of the Analysis process to
           Generation. (used in: analysis\_ready\_msg\_2, 
           generation\_start\_msg)
\end{description}

The {\bf msg} field can have the following values:
\begin{description}
  \item[startmessage, stopmessage] Indicates that a process should continue or
                 stop. (used in : analysis\_start\_msg,
                                  generation\_start\_msg,
                                  generation\_stop-reset\_msg,
                                  analysis\_stop\_msg,
                                  generation\_stop\_msg)
  \item[EmptyAnResult] The Analysis process has not produced a result.
                        (used in: analysis\_ready\_msg\_2)
  \item[CorrectAnResult] The Analysis process has produced a result.
                        (used in: analysis\_ready\_msg\_2)
  \item[ILNextSyn] The Generation process has to present the next syntactic
                   ambiguity.
                        (used in: generation\_start\_msg)
  \item[ILNextLexSense] The Generation process has to present the first 
                result
                of the evaluation of G-Transfer of the current IL-tree.
                        (used in: generation\_start\_msg)
  \item[ILNextStructSense] The Generation process has to present the first 
                result
                of the evaluation of the next IL-tree.
                        (used in: generation\_start\_msg)
  \item[ILFinished] There are no ambiguities belonging to the current
                 IL-tree and the current IL-tree is the last one.
                 (used in: generation\_ready\_msg\_2)
  \item[AmbigInfo] There are still structural, lexical or syntactic
                 ambiguities.
                 (used in: generation\_ready\_msg\_2)
\end{description}
\section{Global Buffer Communication}
Communication via global buffers is especially useful for the transfer
of bulk data. A global buffer is shared between the communicating processes.
Therefore no physical transport of the data is needed, process A can write
the data into the buffer while process B reads them from the same buffer. Of
course these actions must be synchronized in some way or the other. In the
Rosetta system reading and writing is synchronized on a high level.
The Control process guarantees that Analysis and Generation are mutual 
exclusive, that is, either Analysis is active or Generation is active.

There will be one single function for global buffer communication:
\begin{description}
   \item[GlobBuf\_CreateBuffer(name, bytesize, \underline{var} startaddress)] Creates
       a global buffer of name {\bf name} and of size {\bf bytesize}. The 
       startaddress of the buffer is returned in parameter {\bf startaddress}.
       If a global buffer with name {\bf name} does not exist it is created. If
       the buffer has already been created by another process, only the address
       of the existing buffer is returned. A process is not allowed to create 
       two buffers with the same name.
\end{description}

From this interface description it is clear that the function cannot be used 
using ISO-Pascal. The usage in Rosetta is as follows. First a buffer type is 
defined :
\begin{verbatim}
   TYPE buffertype  = array[1..MaxElts] of bufferelt;
        pbuffertype = ^buffertype;
\end{verbatim}
Next, a buffer is created using the function described above:
\begin{verbatim}
   VAR  buffer     : pbuffertype;
        bufferaddr : INTEGER;
   BEGIN
      GlobBuf_CreateBuffer('DataBuffer1', Size(buffertype), bufferaddr);
\end{verbatim}
Finally, we use a {\em typecast} operation to give variable {\bf buffer} the
correct value.
\begin{verbatim}
      buffer := bufferaddr::pbuffertype;
\end{verbatim}
If these actions have been performed the rest of the software acts as if the 
variable {\bf buffer} got its value by means of an ordinary {bf NEW(buffer)} 
statement. However, there is a snake in the grass. It is not useful to pass
pointer values from one process to another by means of a global buffer since 
they are only relevant 
within the process space of the current process. For example, if the 
bufferelements in the buffer would be connected by means of a linked list by 
process A, then the connections would have no meaning for process B. If
process B would access a bufferelement via a linked list pointer this 
access would probably result in a run-time error.
\section{Module Overview}

\begin{description}
\item[GENERAL:MB] Contains the implementation of the abstract datatype {\em 
                  mailbox}.

\item[GENERAL:ANALYSIS] Body of the Analysis process, including process 
                  communication and synchronization. 

\item[GENERAL:GENERATION] Body of the Generation process, including process 
                  communication and synchronization. 

\item[GENERAL:CONTROL] Body of the Control process, including process 
                  communication and synchronization. 

\item[VMS:GLOBBUF] Implementation of the abstract datatype {\em shared global 
                  buffer}.

\item[GENERAL:HILTREE] Implementation of the Analysis-Generation interface 
                  datastructure together with some primitive operations. This
                  module uses the {\em global buffer} concept.
\end{description}
\end{document}

ROSETTA.sty
\typeout{Document Style 'Rosetta'. Version 0.4 - released  24-DEC-1987}
% 24-DEC-1987:  Date of copyright notice changed
\def\@ptsize{1}
\@namedef{ds@10pt}{\def\@ptsize{0}}
\@namedef{ds@12pt}{\def\@ptsize{2}} 
\@twosidetrue
\@mparswitchtrue
\def\ds@draft{\overfullrule 5pt} 
\@options
\input art1\@ptsize.sty\relax


\def\labelenumi{\arabic{enumi}.} 
\def\theenumi{\arabic{enumi}} 
\def\labelenumii{(\alph{enumii})}
\def\theenumii{\alph{enumii}}
\def\p@enumii{\theenumi}
\def\labelenumiii{\roman{enumiii}.}
\def\theenumiii{\roman{enumiii}}
\def\p@enumiii{\theenumi(\theenumii)}
\def\labelenumiv{\Alph{enumiv}.}
\def\theenumiv{\Alph{enumiv}} 
\def\p@enumiv{\p@enumiii\theenumiii}
\def\labelitemi{$\bullet$}
\def\labelitemii{\bf --}
\def\labelitemiii{$\ast$}
\def\labelitemiv{$\cdot$}
\def\verse{
   \let\\=\@centercr 
   \list{}{\itemsep\z@ \itemindent -1.5em\listparindent \itemindent 
      \rightmargin\leftmargin\advance\leftmargin 1.5em}
   \item[]}
\let\endverse\endlist
\def\quotation{
   \list{}{\listparindent 1.5em
      \itemindent\listparindent
      \rightmargin\leftmargin \parsep 0pt plus 1pt}\item[]}
\let\endquotation=\endlist
\def\quote{
   \list{}{\rightmargin\leftmargin}\item[]}
\let\endquote=\endlist
\def\descriptionlabel#1{\hspace\labelsep \bf #1}
\def\description{
   \list{}{\labelwidth\z@ \itemindent-\leftmargin
      \let\makelabel\descriptionlabel}}
\let\enddescription\endlist


\def\@begintheorem#1#2{\it \trivlist \item[\hskip \labelsep{\bf #1\ #2}]}
\def\@endtheorem{\endtrivlist}
\def\theequation{\arabic{equation}}
\def\titlepage{
   \@restonecolfalse
   \if@twocolumn\@restonecoltrue\onecolumn
   \else \newpage
   \fi
   \thispagestyle{empty}\c@page\z@}
\def\endtitlepage{\if@restonecol\twocolumn \else \newpage \fi}
\arraycolsep 5pt \tabcolsep 6pt \arrayrulewidth .4pt \doublerulesep 2pt 
\tabbingsep \labelsep 
\skip\@mpfootins = \skip\footins
\fboxsep = 3pt \fboxrule = .4pt 


\newcounter{part}
\newcounter {section}
\newcounter {subsection}[section]
\newcounter {subsubsection}[subsection]
\newcounter {paragraph}[subsubsection]
\newcounter {subparagraph}[paragraph]
\def\thepart{\Roman{part}} \def\thesection {\arabic{section}}
\def\thesubsection {\thesection.\arabic{subsection}}
\def\thesubsubsection {\thesubsection .\arabic{subsubsection}}
\def\theparagraph {\thesubsubsection.\arabic{paragraph}}
\def\thesubparagraph {\theparagraph.\arabic{subparagraph}}


\def\@pnumwidth{1.55em}
\def\@tocrmarg {2.55em}
\def\@dotsep{4.5}
\setcounter{tocdepth}{3}
\def\tableofcontents{\section*{Contents\markboth{}{}}
\@starttoc{toc}}
\def\l@part#1#2{
   \addpenalty{-\@highpenalty}
   \addvspace{2.25em plus 1pt}
   \begingroup
      \@tempdima 3em \parindent \z@ \rightskip \@pnumwidth \parfillskip
      -\@pnumwidth {\large \bf \leavevmode #1\hfil \hbox to\@pnumwidth{\hss #2}}
      \par \nobreak
   \endgroup}
\def\l@section#1#2{
   \addpenalty{-\@highpenalty}
   \addvspace{1.0em plus 1pt}
   \@tempdima 1.5em
   \begingroup
      \parindent \z@ \rightskip \@pnumwidth 
      \parfillskip -\@pnumwidth 
      \bf \leavevmode #1\hfil \hbox to\@pnumwidth{\hss #2}
      \par
   \endgroup}
\def\l@subsection{\@dottedtocline{2}{1.5em}{2.3em}}
\def\l@subsubsection{\@dottedtocline{3}{3.8em}{3.2em}}
\def\l@paragraph{\@dottedtocline{4}{7.0em}{4.1em}}
\def\l@subparagraph{\@dottedtocline{5}{10em}{5em}}
\def\listoffigures{
   \section*{List of Figures\markboth{}{}}
   \@starttoc{lof}}
   \def\l@figure{\@dottedtocline{1}{1.5em}{2.3em}}
   \def\listoftables{\section*{List of Tables\markboth{}{}}
   \@starttoc{lot}}
\let\l@table\l@figure


\def\thebibliography#1{
   \addcontentsline{toc}
   {section}{References}\section*{References\markboth{}{}}
   \list{[\arabic{enumi}]}
        {\settowidth\labelwidth{[#1]}\leftmargin\labelwidth
         \advance\leftmargin\labelsep\usecounter{enumi}}}
\let\endthebibliography=\endlist


\newif\if@restonecol
\def\theindex{
   \@restonecoltrue\if@twocolumn\@restonecolfalse\fi
   \columnseprule \z@
   \columnsep 35pt\twocolumn[\section*{Index}]
   \markboth{}{}
   \thispagestyle{plain}\parindent\z@
   \parskip\z@ plus .3pt\relax
   \let\item\@idxitem}
\def\@idxitem{\par\hangindent 40pt}
\def\subitem{\par\hangindent 40pt \hspace*{20pt}}
\def\subsubitem{\par\hangindent 40pt \hspace*{30pt}}
\def\endtheindex{\if@restonecol\onecolumn\else\clearpage\fi}
\def\indexspace{\par \vskip 10pt plus 5pt minus 3pt\relax}


\def\footnoterule{
   \kern-1\p@ 
   \hrule width .4\columnwidth 
   \kern .6\p@} 
\long\def\@makefntext#1{
   \@setpar{\@@par\@tempdima \hsize 
   \advance\@tempdima-10pt\parshape \@ne 10pt \@tempdima}\par
   \parindent 1em\noindent \hbox to \z@{\hss$^{\@thefnmark}$}#1}


\setcounter{topnumber}{2}
\def\topfraction{.7}
\setcounter{bottomnumber}{1}
\def\bottomfraction{.3}
\setcounter{totalnumber}{3}
\def\textfraction{.2}
\def\floatpagefraction{.5}
\setcounter{dbltopnumber}{2}
\def\dbltopfraction{.7}
\def\dblfloatpagefraction{.5}
\long\def\@makecaption#1#2{
   \vskip 10pt 
   \setbox\@tempboxa\hbox{#1: #2}
   \ifdim \wd\@tempboxa >\hsize \unhbox\@tempboxa\par
   \else \hbox to\hsize{\hfil\box\@tempboxa\hfil} 
   \fi}
\newcounter{figure}
\def\thefigure{\@arabic\c@figure}
\def\fps@figure{tbp}
\def\ftype@figure{1}
\def\ext@figure{lof}
\def\fnum@figure{Figure \thefigure}
\def\figure{\@float{figure}}
\let\endfigure\end@float
\@namedef{figure*}{\@dblfloat{figure}}
\@namedef{endfigure*}{\end@dblfloat}
\newcounter{table}
\def\thetable{\@arabic\c@table}
\def\fps@table{tbp}
\def\ftype@table{2}
\def\ext@table{lot}
\def\fnum@table{Table \thetable}
\def\table{\@float{table}}
\let\endtable\end@float
\@namedef{table*}{\@dblfloat{table}}
\@namedef{endtable*}{\end@dblfloat}


\def\maketitle{
   \par
   \begingroup
      \def\thefootnote{\fnsymbol{footnote}}
      \def\@makefnmark{\hbox to 0pt{$^{\@thefnmark}$\hss}} 
      \if@twocolumn \twocolumn[\@maketitle] 
      \else \newpage \global\@topnum\z@ \@maketitle
      \fi
      \thispagestyle{plain}
      \@thanks
   \endgroup
   \setcounter{footnote}{0}
   \let\maketitle\relax
   \let\@maketitle\relax
   \gdef\@thanks{}
   \gdef\@author{}
   \gdef\@title{}
   \let\thanks\relax}
\def\@maketitle{
   \newpage
   \null
   \vskip 2em
   \begin{center}{\LARGE \@title \par}
      \vskip 1.5em
      {\large \lineskip .5em \begin{tabular}[t]{c}\@author \end{tabular}\par} 
      \vskip 1em {\large \@date}
   \end{center}
   \par
   \vskip 1.5em} 
\def\abstract{
   \if@twocolumn \section*{Abstract}
   \else
      \small 
      \begin{center} {\bf Abstract\vspace{-.5em}\vspace{0pt}} \end{center}
      \quotation 
   \fi}
\def\endabstract{\if@twocolumn\else\endquotation\fi}


\mark{{}{}} 
\if@twoside
   \def\ps@headings{
      \def\@oddfoot{Rosetta Doc. \@RosDocNr\hfil \@RosDate}
      \def\@evenfoot{Rosetta Doc. \@RosDocNr\hfil \@RosDate}
      \def\@evenhead{\rm\thepage\hfil \sl \rightmark}
      \def\@oddhead{\hbox{}\sl \leftmark \hfil\rm\thepage}
      \def\sectionmark##1{\markboth {}{}}
      \def\subsectionmark##1{}}
\else
   \def\ps@headings{
      \def\@oddfoot{Rosetta Doc. \@RosDocNr\hfil \@RosDate}
      \def\@evenfoot{Rosetta Doc. \@RosDocNr\hfil \@RosDate}
      \def\@oddhead{\hbox{}\sl \rightmark \hfil \rm\thepage}
      \def\sectionmark##1{\markboth {}{}}
      \def\subsectionmark##1{}}
\fi
\def\ps@myheadings{
   \def\@oddhead{\hbox{}\sl\@rhead \hfil \rm\thepage}
   \def\@oddfoot{}
   \def\@evenhead{\rm \thepage\hfil\sl\@lhead\hbox{}}
   \def\@evenfoot{}
   \def\sectionmark##1{}
   \def\subsectionmark##1{}}


\def\today{
   \ifcase\month\or January\or February\or March\or April\or May\or June\or
      July\or August\or September\or October\or November\or December
   \fi
   \space\number\day, \number\year}


\ps@plain \pagenumbering{arabic} \onecolumn \if@twoside\else\raggedbottom\fi 




% the Rosetta title page
\newcommand{\MakeRosTitle}{
   \begin{titlepage}
      \begin{large}
	 \begin{figure}[t]
	    \begin{picture}(405,100)(0,0)
	       \put(0,100){\line(1,0){404}}
	       \put(0,75){Project {\bf Rosetta}}
	       \put(93.5,75){:}
	       \put(108,75){Machine Translation}
	       \put(0,50){Topic}
	       \put(93.5,50){:}
	       \put(108,50){\@RosTopic}
	       \put(0,30){\line(1,0){404}}
	    \end{picture}
	 \end{figure}
	 \bigskip
	 \bigskip
	 \begin{list}{-}{\setlength{\leftmargin}{3.0cm}
			 \setlength{\labelwidth}{2.7cm}
			 \setlength{\topsep}{2cm}}
	    \item [{\rm Title \hfill :}] {{\bf \@RosTitle}}
	    \item [{\rm Author \hfill :}] {\@RosAuthor}
	    \bigskip
	    \bigskip
	    \bigskip
	    \item [{\rm Doc.Nr. \hfill :}] {\@RosDocNr}
	    \item [{\rm Date \hfill :}] {\@RosDate}
	    \item [{\rm Status \hfill :}] {\@RosStatus}
	    \item [{\rm Supersedes \hfill :}] {\@RosSupersedes}
	    \item [{\rm Distribution \hfill :}] {\@RosDistribution}
	    \item [{\rm Clearance \hfill :}] {\@RosClearance}
	    \item [{\rm Keywords \hfill :}] {\@RosKeywords}
	 \end{list}
      \end{large}
      \title{\@RosTitle}
      \begin{figure}[b]
	 \begin{picture}(404,64)(0,0)
	    \put(0,64){\line(1,0){404}}
	    \put(0,-4){\line(1,0){404}}
	    \put(0,59){\line(1,0){42}}
	    \begin{small}
	    \put(3,48){\sf PHILIPS}
	    \end{small}
	    \put(0,23){\line(0,1){36}}
	    \put(42,23){\line(0,1){36}}
	    \put(21,23){\oval(42,42)[bl]}
	    \put(21,23){\oval(42,42)[br]}
	    \put(21,23){\circle{40}}
	    \put(4,33){\line(1,0){10}}
	    \put(9,28){\line(0,1){10}}
	    \put(9,36){\line(1,0){6}}
	    \put(12,33){\line(0,1){6}}
	    \put(29,13){\line(1,0){10}}
	    \put(34,8){\line(0,1){10}}
	    \put(28,10){\line(1,0){6}}
	    \put(31,7){\line(0,1){6}}

	    \put(1,21){\line(1,0){0.5}}
	    \put(1.5,21.3){\line(1,0){0.5}}
	    \put(2,21.6){\line(1,0){0.5}}
	    \put(2.5,21.9){\line(1,0){0.5}}
	    \put(3,22.1){\line(1,0){0.5}}
	    \put(3.5,22.3){\line(1,0){0.5}}
	    \put(4,22.5){\line(1,0){0.5}}
	    \put(4.5,22.7){\line(1,0){0.5}}
	    \put(5,22.8){\line(1,0){0.5}}
	    \put(5.5,22.9){\line(1,0){0.5}}
	    \put(6,23){\line(1,0){0.5}}
	    \put(6.5,22.9){\line(1,0){0.5}}
	    \put(7,22.8){\line(1,0){0.5}}
	    \put(7.5,22.7){\line(1,0){0.5}}
	    \put(8,22.5){\line(1,0){0.5}}
	    \put(8.5,22.3){\line(1,0){0.5}}
	    \put(9,22.1){\line(1,0){0.5}}
	    \put(9.5,21.9){\line(1,0){0.5}}
	    \put(10,21.6){\line(1,0){0.5}}
	    \put(10.5,21.3){\line(1,0){0.5}}

	    \put(1,23){\line(1,0){0.5}}
	    \put(1.5,23.3){\line(1,0){0.5}}
	    \put(2,23.6){\line(1,0){0.5}}
	    \put(2.5,23.9){\line(1,0){0.5}}
	    \put(3,24.1){\line(1,0){0.5}}
	    \put(3.5,24.3){\line(1,0){0.5}}
	    \put(4,24.5){\line(1,0){0.5}}
	    \put(4.5,24.7){\line(1,0){0.5}}
	    \put(5,24.8){\line(1,0){0.5}}
	    \put(5.5,24.9){\line(1,0){0.5}}
	    \put(6,25){\line(1,0){0.5}}
	    \put(6.5,24.9){\line(1,0){0.5}}
	    \put(7,24.8){\line(1,0){0.5}}
	    \put(7.5,24.7){\line(1,0){0.5}}
	    \put(8,24.5){\line(1,0){0.5}}
	    \put(8.5,24.3){\line(1,0){0.5}}
	    \put(9,24.1){\line(1,0){0.5}}
	    \put(9.5,23.9){\line(1,0){0.5}}
	    \put(10,23.6){\line(1,0){0.5}}
	    \put(10.5,23.3){\line(1,0){0.5}}

	    \put(1,25){\line(1,0){0.5}}
	    \put(1.5,25.3){\line(1,0){0.5}}
	    \put(2,25.6){\line(1,0){0.5}}
	    \put(2.5,25.9){\line(1,0){0.5}}
	    \put(3,26.1){\line(1,0){0.5}}
	    \put(3.5,26.3){\line(1,0){0.5}}
	    \put(4,26.5){\line(1,0){0.5}}
	    \put(4.5,26.7){\line(1,0){0.5}}
	    \put(5,26.8){\line(1,0){0.5}}
	    \put(5.5,26.9){\line(1,0){0.5}}
	    \put(6,27){\line(1,0){0.5}}
	    \put(6.5,26.9){\line(1,0){0.5}}
	    \put(7,26.8){\line(1,0){0.5}}
	    \put(7.5,26.7){\line(1,0){0.5}}
	    \put(8,26.5){\line(1,0){0.5}}
	    \put(8.5,26.3){\line(1,0){0.5}}
	    \put(9,26.1){\line(1,0){0.5}}
	    \put(9.5,25.9){\line(1,0){0.5}}
	    \put(10,25.6){\line(1,0){0.5}}
	    \put(10.5,25.3){\line(1,0){0.5}}

	    \put(11,21){\line(1,0){0.5}}
	    \put(11.5,20.7){\line(1,0){0.5}}
	    \put(12,20.4){\line(1,0){0.5}}
	    \put(12.5,20.1){\line(1,0){0.5}}
	    \put(13,19.9){\line(1,0){0.5}}
	    \put(13.5,19.7){\line(1,0){0.5}}
	    \put(14,19.5){\line(1,0){0.5}}
	    \put(14.5,19.3){\line(1,0){0.5}}
	    \put(15,19.2){\line(1,0){0.5}}
	    \put(15.5,19.1){\line(1,0){0.5}}
	    \put(16,19){\line(1,0){0.5}}
	    \put(16.5,19.1){\line(1,0){0.5}}
	    \put(17,19.2){\line(1,0){0.5}}
	    \put(17.5,19.3){\line(1,0){0.5}}
	    \put(18,19.5){\line(1,0){0.5}}
	    \put(18.5,19.7){\line(1,0){0.5}}
	    \put(19,19.9){\line(1,0){0.5}}
	    \put(19.5,20.1){\line(1,0){0.5}}
	    \put(20,20.4){\line(1,0){0.5}}
	    \put(20.5,20.7){\line(1,0){0.5}}

	    \put(11,23){\line(1,0){0.5}}
	    \put(11.5,22.7){\line(1,0){0.5}}
	    \put(12,22.4){\line(1,0){0.5}}
	    \put(12.5,22.1){\line(1,0){0.5}}
	    \put(13,21.9){\line(1,0){0.5}}
	    \put(13.5,21.7){\line(1,0){0.5}}
	    \put(14,21.5){\line(1,0){0.5}}
	    \put(14.5,21.3){\line(1,0){0.5}}
	    \put(15,21.2){\line(1,0){0.5}}
	    \put(15.5,21.1){\line(1,0){0.5}}
	    \put(16,21){\line(1,0){0.5}}
	    \put(16.5,21.1){\line(1,0){0.5}}
	    \put(17,21.2){\line(1,0){0.5}}
	    \put(17.5,21.3){\line(1,0){0.5}}
	    \put(18,21.5){\line(1,0){0.5}}
	    \put(18.5,21.7){\line(1,0){0.5}}
	    \put(19,21.9){\line(1,0){0.5}}
	    \put(19.5,22.1){\line(1,0){0.5}}
	    \put(20,22.4){\line(1,0){0.5}}
	    \put(20.5,22.7){\line(1,0){0.5}}

	    \put(11,25){\line(1,0){0.5}}
	    \put(11.5,24.7){\line(1,0){0.5}}
	    \put(12,24.4){\line(1,0){0.5}}
	    \put(12.5,24.1){\line(1,0){0.5}}
	    \put(13,23.9){\line(1,0){0.5}}
	    \put(13.5,23.7){\line(1,0){0.5}}
	    \put(14,23.5){\line(1,0){0.5}}
	    \put(14.5,23.3){\line(1,0){0.5}}
	    \put(15,23.2){\line(1,0){0.5}}
	    \put(15.5,23.1){\line(1,0){0.5}}
	    \put(16,23){\line(1,0){0.5}}
	    \put(16.5,23.1){\line(1,0){0.5}}
	    \put(17,23.2){\line(1,0){0.5}}
	    \put(17.5,23.3){\line(1,0){0.5}}
	    \put(18,23.5){\line(1,0){0.5}}
	    \put(18.5,23.7){\line(1,0){0.5}}
	    \put(19,23.9){\line(1,0){0.5}}
	    \put(19.5,24.1){\line(1,0){0.5}}
	    \put(20,24.4){\line(1,0){0.5}}
	    \put(20.5,24.7){\line(1,0){0.5}}

	    \put(21,21){\line(1,0){0.5}}
	    \put(21.5,21.3){\line(1,0){0.5}}
	    \put(22,21.6){\line(1,0){0.5}}
	    \put(22.5,21.9){\line(1,0){0.5}}
	    \put(23,22.1){\line(1,0){0.5}}
	    \put(23.5,22.3){\line(1,0){0.5}}
	    \put(24,22.5){\line(1,0){0.5}}
	    \put(24.5,22.7){\line(1,0){0.5}}
	    \put(25,22.8){\line(1,0){0.5}}
	    \put(25.5,23.9){\line(1,0){0.5}}
	    \put(26,23){\line(1,0){0.5}}
	    \put(26.5,22.9){\line(1,0){0.5}}
	    \put(27,22.8){\line(1,0){0.5}}
	    \put(27.5,22.7){\line(1,0){0.5}}
	    \put(28,22.5){\line(1,0){0.5}}
	    \put(28.5,22.3){\line(1,0){0.5}}
	    \put(29,22.1){\line(1,0){0.5}}
	    \put(29.5,21.9){\line(1,0){0.5}}
	    \put(30,21.6){\line(1,0){0.5}}
	    \put(30.5,21.3){\line(1,0){0.5}}

	    \put(21,23){\line(1,0){0.5}}
	    \put(21.5,23.3){\line(1,0){0.5}}
	    \put(22,23.6){\line(1,0){0.5}}
	    \put(22.5,23.9){\line(1,0){0.5}}
	    \put(23,24.1){\line(1,0){0.5}}
	    \put(23.5,24.3){\line(1,0){0.5}}
	    \put(24,24.5){\line(1,0){0.5}}
	    \put(24.5,24.7){\line(1,0){0.5}}
	    \put(25,24.8){\line(1,0){0.5}}
	    \put(25.5,24.9){\line(1,0){0.5}}
	    \put(26,25){\line(1,0){0.5}}
	    \put(26.5,24.9){\line(1,0){0.5}}
	    \put(27,24.8){\line(1,0){0.5}}
	    \put(27.5,24.7){\line(1,0){0.5}}
	    \put(28,24.5){\line(1,0){0.5}}
	    \put(28.5,24.3){\line(1,0){0.5}}
	    \put(29,24.1){\line(1,0){0.5}}
	    \put(29.5,23.9){\line(1,0){0.5}}
	    \put(30,23.6){\line(1,0){0.5}}
	    \put(30.5,23.3){\line(1,0){0.5}}

	    \put(21,25){\line(1,0){0.5}}
	    \put(21.5,25.3){\line(1,0){0.5}}
	    \put(22,25.6){\line(1,0){0.5}}
	    \put(22.5,25.9){\line(1,0){0.5}}
	    \put(23,26.1){\line(1,0){0.5}}
	    \put(23.5,26.3){\line(1,0){0.5}}
	    \put(24,26.5){\line(1,0){0.5}}
	    \put(24.5,26.7){\line(1,0){0.5}}
	    \put(25,26.8){\line(1,0){0.5}}
	    \put(25.5,26.9){\line(1,0){0.5}}
	    \put(26,27){\line(1,0){0.5}}
	    \put(26.5,26.9){\line(1,0){0.5}}
	    \put(27,26.8){\line(1,0){0.5}}
	    \put(27.5,26.7){\line(1,0){0.5}}
	    \put(28,26.5){\line(1,0){0.5}}
	    \put(28.5,26.3){\line(1,0){0.5}}
	    \put(29,26.1){\line(1,0){0.5}}
	    \put(29.5,25.9){\line(1,0){0.5}}
	    \put(30,25.6){\line(1,0){0.5}}
	    \put(30.5,25.3){\line(1,0){0.5}}

	    \put(31,21){\line(1,0){0.5}}
	    \put(31.5,20.7){\line(1,0){0.5}}
	    \put(32,20.4){\line(1,0){0.5}}
	    \put(32.5,20.1){\line(1,0){0.5}}
	    \put(33,19.9){\line(1,0){0.5}}
	    \put(33.5,19.7){\line(1,0){0.5}}
	    \put(34,19.5){\line(1,0){0.5}}
	    \put(34.5,19.3){\line(1,0){0.5}}
	    \put(35,19.2){\line(1,0){0.5}}
	    \put(35.5,19.1){\line(1,0){0.5}}
	    \put(36,19){\line(1,0){0.5}}
	    \put(36.5,19.1){\line(1,0){0.5}}
	    \put(37,19.2){\line(1,0){0.5}}
	    \put(37.5,19.3){\line(1,0){0.5}}
	    \put(38,19.5){\line(1,0){0.5}}
	    \put(38.5,19.7){\line(1,0){0.5}}
	    \put(39,19.9){\line(1,0){0.5}}
	    \put(39.5,20.1){\line(1,0){0.5}}
	    \put(40,20.4){\line(1,0){0.5}}
	    \put(40.5,20.7){\line(1,0){0.5}}

	    \put(31,23){\line(1,0){0.5}}
	    \put(31.5,22.7){\line(1,0){0.5}}
	    \put(32,22.4){\line(1,0){0.5}}
	    \put(32.5,22.1){\line(1,0){0.5}}
	    \put(33,21.9){\line(1,0){0.5}}
	    \put(33.5,21.7){\line(1,0){0.5}}
	    \put(34,21.5){\line(1,0){0.5}}
	    \put(34.5,21.3){\line(1,0){0.5}}
	    \put(35,21.2){\line(1,0){0.5}}
	    \put(35.5,21.1){\line(1,0){0.5}}
	    \put(36,21){\line(1,0){0.5}}
	    \put(36.5,21.1){\line(1,0){0.5}}
	    \put(37,21.2){\line(1,0){0.5}}
	    \put(37.5,21.3){\line(1,0){0.5}}
	    \put(38,21.5){\line(1,0){0.5}}
	    \put(38.5,21.7){\line(1,0){0.5}}
	    \put(39,21.9){\line(1,0){0.5}}
	    \put(39.5,22.1){\line(1,0){0.5}}
	    \put(40,22.4){\line(1,0){0.5}}
	    \put(40.5,22.7){\line(1,0){0.5}}

	    \put(31,25){\line(1,0){0.5}}
	    \put(31.5,24.7){\line(1,0){0.5}}
	    \put(32,24.4){\line(1,0){0.5}}
	    \put(32.5,24.1){\line(1,0){0.5}}
	    \put(33,23.9){\line(1,0){0.5}}
	    \put(33.5,23.7){\line(1,0){0.5}}
	    \put(34,23.5){\line(1,0){0.5}}
	    \put(34.5,23.3){\line(1,0){0.5}}
	    \put(35,23.2){\line(1,0){0.5}}
	    \put(35.5,23.1){\line(1,0){0.5}}
	    \put(36,23){\line(1,0){0.5}}
	    \put(36.5,23.1){\line(1,0){0.5}}
	    \put(37,23.2){\line(1,0){0.5}}
	    \put(37.5,23.3){\line(1,0){0.5}}
	    \put(38,23.5){\line(1,0){0.5}}
	    \put(38.5,23.7){\line(1,0){0.5}}
	    \put(39,23.9){\line(1,0){0.5}}
	    \put(39.5,24.1){\line(1,0){0.5}}
	    \put(40,24.4){\line(1,0){0.5}}
	    \put(40.5,24.7){\line(1,0){0.5}}
	    \begin{large}
	       \put(60,45){Philips Research Laboratories}
	       \put(60,30){\copyright\ 1988 Nederlandse Philips Bedrijven B.V.}
	    \end{large}
	 \end{picture}
      \end{figure}
      \newpage
      \pagenumbering{roman}
      \tableofcontents
      \newpage
      \pagenumbering{arabic}
   \end{titlepage}
}
\title{}
\topmargin 0pt
\oddsidemargin 36pt
\evensidemargin 36pt
\textheight 600pt
\textwidth 405pt
\pagestyle{headings}
\newcommand{\@RosTopic}{General}
\newcommand{\@RosTitle}{-}
\newcommand{\@RosAuthor}{-}
\newcommand{\@RosDocNr}{}
\newcommand{\@RosDate}{}
\newcommand{\@RosStatus}{informal}
\newcommand{\@RosSupersedes}{-}
\newcommand{\@RosDistribution}{Project}
\newcommand{\@RosClearance}{Project}
\newcommand{\@RosKeywords}{}
\newcommand{\RosTopic}[1]{\renewcommand{\@RosTopic}{#1}}
\newcommand{\RosTitle}[1]{\renewcommand{\@RosTitle}{#1}}
\newcommand{\RosAuthor}[1]{\renewcommand{\@RosAuthor}{#1}}
\newcommand{\RosDocNr}[1]{\renewcommand{\@RosDocNr}{#1 (RWR-102-RO-90#1-RO)}}
\newcommand{\RosDate}[1]{\renewcommand{\@RosDate}{#1}}
\newcommand{\RosStatus}[1]{\renewcommand{\@RosStatus}{#1}}
\newcommand{\RosSupersedes}[1]{\renewcommand{\@RosSupersedes}{#1}}
\newcommand{\RosDistribution}[1]{\renewcommand{\@RosDistribution}{#1}}
\newcommand{\RosClearance}[1]{\renewcommand{\@RosClearance}{#1}}
\newcommand{\RosKeywords}[1]{\renewcommand{\@RosKeywords}{#1}}

