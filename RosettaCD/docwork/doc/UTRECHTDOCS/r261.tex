\documentstyle{Rosetta}
\begin{document}
   \RosTopic{General}
   \RosTitle{Notulen linguistenvergadering dd. 15-03-88}
   \RosAuthor{Franciska de Jong}
   \RosDocNr{R261}
   \RosDate{\today}
   \RosStatus{informal}
   \RosDistribution{lLinguists}
   \RosClearance{Project}
   \RosKeywords{EMPTY's, relations, CNVAR}
   \MakeRosTitle
%
%

\noindent
{\bf Verslag linguistenvergadering 15 maart 1988}
\begin{description}
\item[Aanwezig:] Lisette Appelo, Elly van Munster, Jan Odijk, Margreet Sanders,
Harm Smit, Andr\'{e} Schenk, Franciska de Jong (not).
\item[Afwezig:] -
\item[Agenda:] \mbox{}
\begin{enumerate}
\item EMPTY's in IL
\item Links-rechts hierarchie van relaties
\item Rol van CN en CNVAR
\end{enumerate}
\end{description}


\section{EMPTY's in IL} 
 Met het oog op het Spaans worden in IL vier 
verschillende EMPTY's onderscheiden 
voor de volgende gevallen:
 \begin{itemize}
  \item men - se - one
  \item ze(gen.)  - \emptyset/ello - they
  \item je - uno - one
  \item syntactisch niet gerealiseerde argumenten
\end{itemize}
Jan O. schrijft over dit onderwerp en document.\\

Zijdelings  is de rol van bonustoekenning ter sprake geweest. 
Naast incorrectheidsbonus (correctheid niet gegarandeerd) is er wellicht ook
een speciale bonus nodig 
om aan te geven dat een bepaalde (correcte) vertaling alleen 
dan gegenereerd wordt als er geen enkele andere correcte vertaling doorkomt.

\section{Links-rechts hierarchie van relaties}
Er is een proleem met regels waarbij geinserteerd wordt 
in een sequentie van meer dan 
twee mu's. Tot nu toe werd de juiste  positie voor het te inserteren element 
beregeld met behulp van relatie-sets in het AUX-domein. Bijvoorbeeld: 
preobjrels, postobjrels. Bezwaar is dat er veel sets zijn, en dat de 
achterliggende systematiek niet wordt uitgedrukt.

Bovendien zijn er gevallen 
waarvoor het niet werkt: er zijn relaties, zoals advrel 
die kunnen optreden temidden van 
een set van gelijksoortige relaties. In principe zijn daarbinnen alle volgordes
mogelijk. Verder zijn er elementen waarvoor geen unieke positie 
gespecificeerd kan worden, zoals PUNCs en GLUE. 

Oplossing: een hulpfunctie die de links-rechts hierarchie (so to speak) van 
relaties beregelt. Jan O. coordineert de implementatie hiervan.

\section{Rol van CNVAR en CN}
De rol van CNVAR wordt nog eens uiteengezet. 
Terzijde wordt 
opgemerkt dat INDEFPRO's volgens 
de nu bestaande regels niet door een CN-knoop gedomineerd worden, maar 
alleen door een NP-knoop. Dat impliceert dat de modificatie in 
NP's als {\em iemand die rood haar heeft} 
ten onrechte  als uitbreidend wordt gezien. Verbetering hiervan
wordt uitgesteld tot na 1 november. 
\end{document}
