\documentstyle{Rosetta}
\begin{document}
   \RosTopic{Rosetta3.doc.morphology.English}
   \RosTitle{Rosetta3 English Morphology: Glue Rules and \\
               Segmentation Rules}
   \RosAuthor{Margreet Sanders}
   \RosDocNr{423}
   \RosDate{December 14, 1989}
   \RosStatus{concept}
   \RosSupersedes{-}
   \RosDistribution{Project}
   \RosClearance{Project}
   \RosKeywords{English, morphology, Glue rules, Segmentation Rules, 
                documentation}
   \MakeRosTitle

\section{Introduction}
When in analysis of an input sentence a word has to be analysed to find its 
stem, several rules are needed in Rosetta. First, there are glue rules that 
divide a word into its composing words. Then, the segmentation rules 
divide the word into a possible stem and affixes, if any can be detected. 
The result will usually be a whole set of possible combinations. Then, a 
search is done for each of these stems whether the dictionary indeed contains 
such a stem. If it does 
not, the analysis found by the segmentation rules is discarded. If the stem is 
found, the 
information the dictionary gives on the attributes of the stem are used by 
the lextree rules (also called W-parser) to check whether the combination of 
stem and affixes is indeed allowed. The formal aspects of this system are 
described in doc.\ 309 by Joep Rous, {\em Formal Specification of the Rosetta3 
Morphological Components\/}. The lextree rules are described in doc.\ 306 by 
Margreet Sanders, {\em Rosetta3 English Morphology: Lextree Rules\/}. The 
current document describes the glue rules and the segmentation rules. 
It can be seen as an updated version of doc.\ 115 by Agnes Mijnhout, {\em 
Rosetta3 English Morphology, inflection\/}, sections 3.1 and 8 on glue rules 
and sections 3.2 and 9 on segmentation rules.
Document 115 contains a full description of English 
morphology as it was implemented in Rosetta3 at the end of 1986. Since then, 
there have been very few major changes. All major changes are explicitly 
indicated in the present document. Minor changes and improvements have not been 
indicated explicitly.

\newpage
\section{Glue Rules}
Glue rules are rules that separate a word into two words by putting a GLUE 
between them, analytically 
speaking. They are distinct from segmentation rules, which divide a word into
its stem and affixes and store the appropriate suffixkey or prefix key. In 
analysis, the glue rules are applied first, because they decide what the words 
are on which the segmentation rules will have to work. They are not iterative, 
but may be n-ary, i.e.\ they may take more than two arguments at a time. For 
English, there are only binary rules. The glue rules can be found in files 
{\bf rglue.seg} and {\bf 
mglue.seg}, containing right glue rules and middle glue rules respectively.
The difference between these two kinds of rules is that middle glue rules 
expect fully specified strings as arguments, while right glue rules separate a 
word from the right-hand side of any partly specified string. This is indicated 
by means of the `care-not' symbol {\bf *}, which stands for any string 
(possibly empty). Left glue rules, that separate a 
word from the left-hand side of a string, do not exist for English.
An example of a right glue rule is the rule that separates the genitive {\em 
's\/} from a string:
\begin{verbatim}
/A  *s    +    's    :: *s's;
\end{verbatim}
In generation, the rule is read from left to right, and in analysis from right 
to left. If a rule is prefixed by either /A or /G it works only in analysis or 
generation, respectively. The `+' symbol stands for a GLUE. The M-grammar must 
contain rules that remove the GLUE in analysis; the segmentation rules 
interpret a GLUE as a sign that the strings on either side are separate words, 
but the do not throw the GLUE away. In generation, the rules of the 
M-grammar must explicitly introduce a GLUE if a contracted form is to be 
produced. Glue rules may also use variables, i.e.\ the name of a predefined set 
of strings, e.g.
\begin{verbatim}
VAR
B  =  [he, she, it];

TABLE
(B)   +  has   :: (B)'s;
\end{verbatim}
The variable B is defined as a set consisting of the strings {\em he, she\/}
and {\em it\/}. If any of these strings is followed by an {\em 's\/}, this {\em 
's\/} will be interpreted as the word {\em has\/} in analysis. Glue rules and 
segmentation rules are sensitive to the distinction between upper and lower 
case. Fortunately, the lay-out component that precedes the glue rules in 
analysis replaces all sentence-initial upper cases by lower cases, so that only 
upper case in the middle of a sentence has to be accounted for.

\newpage
\subsection{Middle Glue Rules}
This section contains the contents of the file {\bf mglue.seg}.
There are four kinds of rules:\\
1) The first set of rules make the contracted form of an operator and a 
negation. These rules were extended with a special rule for {\em cannot\/}: the 
basic expression {\em n\`{o}t\/} is introduced there to be able to guide the 
generation of either {\em cannot\/} or {\em can't\/} (when they are both formed 
from the same words, M-grammar will never be able to select only one of 
them in generation, and morphology should never have to filter out the 
structures supplied by M-grammar, so it does not help to have one of the rules 
in analysis only). For more comment, see also the relevant rule in M-grammar, 
TCanNegIncorp in subgrammar CLAUSEtoSENTENCE. If M-grammar can deal with tag 
questions, a similar strategy will be needed for the distinction between {\em 
aren't\/} and {\em ain't\/}, both made from {\em am\/} + {\em not\/}.\\
2) The second set of rules deals with the contraction of an operator and a 
personal pronoun subject. These rules work only in generation, exept for the 
rule contracting {\em I\/} + {\em am\/}. M-grammar must decide whether these 
contracted forms are 
ever needed. For analysis, any string is accepted as input, not just personal 
pronouns. Thus, the analytical rules are left glue rules, and are described in 
the next subsection. The generative rules have been adapted, so that the form 
{\em It'd\/} is not generated any longer. Perhaps {\em It'll\/} must be 
excluded too, but this has not been implemented yet. The slang form {\em 
ain't\/} as contraction of any form of the verb {\em be\/} (not just {\em am\/})
 and the negation is not dealt with.\\
3) The third kind of rule deals with {\em wanna\/}-contraction.\\
4) The last kind of rule is for {\em let's\/}.

\begin{verbatim}

TYPE  MGLUE2;

VAR

A = [I, you, he, she, it, we, you, they];
B = [he, she, it];
C = [you, we, they];
D = [I, you, we, they];
E = [I, you, he, she, we, you, they];

TABLE

   shall  + not :: shan't;       
   should + not :: shouldn't;  
   will   + not :: won't;        
   would  + not :: wouldn't;  
   can    + not :: can't;         
   can    + n`ot :: cannot;
   could  + not :: couldn't;
   must   + not :: mustn't;  
   might  + not :: mightn't;
   ought  + not :: oughtn't;
   used   + not :: usedn't;
   need   + not :: needn't;
   dare   + not :: daren't; 
   do     + not :: don't;
   does   + not :: doesn't;
   did    + not :: didn't;
   have   + not :: haven't;
   has    + not :: hasn't;
   had    + not :: hadn't;
   are    + not :: aren't;
   am     + not :: aren't;       (only in tag questions: aren't I?)
   am     + not :: ain't;   
   is     + not :: isn't;
   was    + not :: wasn't;
   were   + not :: weren't;

/G   (A)   + will  :: (A)'ll;         
/G   (E)   + had   :: (E)'d; 
/G   (E)   + would :: (E)'d;          
/G   (B)   + has   :: (B)'s;          
/G   (B)   + is    :: (B)'s;
      I    + am    ::  I'm;
/G   (C)   + are   :: (C)'re;
/G   (D)   + have  :: (D)'ve;         

   want + to :: wanna;
 
   let  + us :: let's;

END

\end{verbatim}

\newpage
\subsection{Right Glue Rules}
This subsection contains the contents of the file {\bf rglue.seg}. There are 
two sets of rules:\\
1) rules to make genitive forms by adding a {\em 's\/}.\\
2) analytical rules for contraction of an operator and any preceding string.

There is a major problem with the rules of the first subset.
Actually, these rules should be sensitive to phonetic information. The first
rule attaches only an apostrophe if the string ends in an {\em s\/}. However, 
irregular plurals ending in an {\em s\/} in pronunciation but not in writing 
should
also receive only an apostrophe: {\em beaux'\/} (at least according to the OED, 
Quirk
is not very clear on this point). This form is rejected by the rules:
they attach {\em 's\/} if the string does not end in 
an {\em s\/}. The constraint on the final {\em s\/} is not always correct 
either: singular words (Quirk gives only examples
with names)  ending in an {\em s\/}  in pronunciation
should also receive {\em 's\/}: {\em Ross's, mistress's\/}. These forms 
cannot be generated by the current rules.
Furthermore, with singular words (again only examples with names in 
Quirk) ending in a {\em z\/} in pronunciation, there is a
choice between {\em ' } and {\em 's\/}: {\em Dickens'\/}  or {\em Dickens's\/}.
 Only the first form
of these two is generated by the rules. A rule is added that accepts 
{\em Dickens's\/}. This rule will now incorrectly allow other strings in analysis
too, however. The set NOTs was devised to be able to deal with abbreviations
and numerals as well: it contains capitals and a period. Thus, the genitives
{\em M.P.'s\/} and {\em NP's\/} and the form {\em 1960's\/} 
can be dealt with too. This last
item is given as an example of a genitive form in the OED, but Quirk claims
that it can be plural as well, next to {\em 1960s\/}.

In Rosetta3, phonetic constraints can be used to check segmentation processes.
In that case, the dictionary must contain phonetic information, and this 
information is checked against the phonetic constraints that have been 
specified in the segmentation rule. However, such constraints are as yet 
impossible for glue rules, which operate before the segmentation rules 
(analytically speaking) and have no immediate possibility of checking any 
specification in the dictionary. Hence, 
there is no direct solution to the problem over over-acceptance but 
under-production for strings 
ending in {\em s's\/} yet. In doc.\ 115, it was suggested to treat genitive 
{\em 's\/} as a segmentation phenomenon instead of a word contraction 
phenomenon.
At the time, it was thought that this would be problematic, because the 
attribute specifying whether a genitive was needed should go with the NP-node, 
but the actual {\em 's\/} attaches to the 
noun. However, there already is a function CheckGen to check whether the last 
element in the NP can take a {\em 's\/}. A function AddGen that sets the value 
of a new attribute {\em gen-s\/} could easily be 
added. Therefore, it seems to me that Agnes's suggestion should be taken 
seriously.

Although the rules of the second subset also clearly accept many forms that are 
wrong, the conditions on what to accept and what not cannot be translated to a 
simple table of rules. Other forms are inherently ambiguous: a string like 
{\em He's eaten\/} could both be {\em He is eaten\/} and {\em He has eaten\/}. 
The only operator that must always go with a specific 
string is {\em am\/}; hence, the analytical glue rule for *{\em 'm\/} can be 
more fully specified as a rule for {\em I'm\/}, and is incorporated in the 
middle glue rules (see the previous subsection). 
Note that M-grammar does not have any rules yet that can deal with subject - 
auxiliary contraction, so these forms can only be dealt with by morphology now.
\begin{verbatim}

TYPE RGLUE2;

VAR

NOTs = [ a, b, c, d, e, f, g, h, i, j, k, l, m, n, o, p, q, r,
         t, u, v, w, x, y, z, 
         A, B, C, D, E, F, G, H, I, J, K, L, M, N, O, P, Q, R,
         T, U, V, W, X, Y, Z, 
         ., 0, 1, 2, 3, 4, 5, 6, 7, 8, 9, 0];

TABLE

   *s      + 's :: *s';         
   *S      + 's :: *S';
   *(NOTs) + 's :: *(NOTs)'s;
/A *s      + 's :: *s's;


/A * + will   :: *'ll;
/A * + had    :: *'d;
/A * + would  :: *'d;
/A * + has    :: *'s;
/A * + is     :: *'s;
/A * + are    :: *'re;
/A * + have   :: *'ve;

END
\end{verbatim}

\newpage
\section{Segmentation Rules}
As was explained in the introduction to this document, the segmentation rules 
work after the glue rules in analysis, taking the words separated by the glue 
rules as their domain. They divide a word into a possible stem and an affix.
The rules are binary and iterative, and know some internal ordering of 
application: first, 
prefixes are removed; then, there is a check whether the resulting string 
could be a compound,
and finally suffixes are removed. Since Rosetta3 does not deal with 
compounds yet, and 
since English inflection only uses suffixes (derivation is hardly dealt with 
yet), the suffix rules are the only rules to be described in this section.

The suffix rules do not produce two separate words in analysis, but a word and 
a suffix key. The suffix keys that exist for English are defined in {\bf 
english:lsdomaint.dom}, together with some prefix keys that may be needed when 
derivation is dealt with. The syntax of the rules is much the same as for left 
glue rules:
\begin{verbatim}
*come   +   SFKirrpt   :: *came;
\end{verbatim}
The example rule says that the combination of the irregular past tense key and 
all strings ending in {\em come\/} (become, come, overcome etc.) will gave 
a new string ending in {\em came\/}, generatively speaking. Segmentation rules 
may also contain variables, just as glue rules, and are also sensitive to the 
distinction between upper and lower case. Note that the `+' 
symbol in the segmentation rules does not stand for a GLUE. It is simply a 
symbol to separate the two arguments of the rule. In analysis, the form {\em 
*came\/} is thus recognised as a combination of a stem *{\em come\/} and the 
suffix key SFKirrpt. After the segmentation rules a dictionary look-up follows, 
in which the stem *{\em come\/} is connected to a full verb record. This record 
 contains, amongst others, an attribute that indicates the conjugation 
class(es) of 
the verb. In the lextree rules it is checked whether this attribute is 
compatible with the suffix key found by the segmentation rule (i.e.\ whether 
*{\em come\/} has an irregular past tense). If the combination is indeed 
allowed, the suffix key is `thrown away' and instead a special attribute value 
is set. Hence, suffix keys play no role whatsoever in M-grammar.

Because the order in which affixes can be attached to a stem is usually fixed, 
a control expression has been written to state this constraint. 
For English, it is very simple, because
there can be only one suffix, except for the very irregular steps 
of comparison of the adverbs {\em well\/} and {\em badly\/}, which have 
themselves been derived from the adjectives {\em good\/} and {\em bad\/} by 
means of a derivational suffix key.
The distinction that used to exist between rsuffix (regular, recursive), 
freefix (for fully specified combinations, without * ) and other suffix rules 
(for irregular forms that change the stem) has been removed, now that a
control expression has been defined.

In the sections to follow, the segmentation rules as present in file 
{\bf english:suffix.seg} are given in full. They are ordered more or less by 
lexical category, giving first the rules for Verbs, then for Nouns, then for 
Adjectives and Adverbs, and finally for Pronouns. There also is a section on 
Derivation Rules. The variables that are 
defined at the top of the file are reproduced here separately for each 
category, for easier reference.


\subsection{Control Expression}
The control expression which indicates the order of application of the affix 
rules for English can be found in file {\bf english:morphexpr.afxpr}. 
\begin{verbatim}

AFFIXEXPR =
[
( 
SFKam | SFKis | SFKwas | SFKare | SFKwere | 
SFKirrpp | SFKirring | SFKirrptpp | SFKirrS | SFKmodpt | 
SFKirrnc | SFKirrpt | SFKirrpt2 | SFKirrpp2 | SFKirrpp3 | 
SFKregptpp | SFKcdptpp | SFKcding | SFKreging | 
SFKregS | SFKregEs | 
( [ SFKirrcomp | SFKirrsuper ] SFKregly ) |
( [ SFKirrcomp | SFKirrsuper ] SFKirregly ) |
SFKnoly | 
SFKregable | SFKcdable | SFKtruncable |
SFKvoicingS | SFKirrplur | SFKlatplur | 
SFKirrcomp | SFKirrsuper | SFKirrcomp2 | SFKirrsuper2 | 
SFKcdcomp | SFKcdsuper | SFKregcomp | SFKregsuper | 
SFKacc | SFKposs | SFKpredposs 
)
]
EXPREND

\end{verbatim}

\newpage
\subsection{Rules for Verbs}
\subsubsection{Very Irregular Verbs}
These rules are for the very irregular verbs {\em be, have\/}, and the modals.
The numbers  in front of the verbs indicate the conjugation class. In the 
lextree rules, this conjugation class will be used to determine whether the 
combination of verb and suffix key as found in the segmentation rules 
was correct. For ing-forms and present tense forms, there is no such check, and 
the class has not been indicated. 

The suffix keys used in the rules are:\\
SFKam = for 1st person singular present tense of `be'\\
SFKis = for 3rd person singular present tense of `be'\\
SFKwas = for singular past tense of `be'\\
SFKare = for plural present tense of `be'\\
SFKwere = for plural past tense of `be'\\
SFKirring = for -ing form of `be' and other verbs (see section 3.2.4)\\
SFKirrptpp = for past and participle of `have' and other verbs (see section 
3.2.3)\\
SFKirrS = for s-form of `have' and other verbs (see section 3.3)\\
SFKmodpt = for the past tense of modals.

\begin{verbatim}
(0) be   + SFKam       :: am;     
(0) be   + SFKis       :: is;     
(0) be   + SFKwas      :: was;    
(0) be   + SFKare      :: are;    
(0) be   + SFKwere     :: were;   
(0) be   + SFKirrpp    :: been;  
    be   + SFKirring   :: being;

(2) have + SFKirrptpp  :: had;    (pt + pp)

    have + SFKirrS     :: has;

(12) can   + SFKmodpt    :: could;  (pt)
(12) may   + SFKmodpt    :: might;  (pt)
(12) shall + SFKmodpt    :: should; (pt)
(12) will  + SFKmodpt    :: would;  (pt)
(12) dare  + SFKmodpt    :: dared;  (pt)
\end{verbatim}
The form {\em dared\/} might also have been handled by the
regular rule, but then a separate conjugation class would be necessary for
{\em dare\/}, as the only regular modal. 
For {\em durst\/}, see the irregular past tense rules below.


\subsubsection{Regular Past and Perfect}
These rules make past tense and past participle forms for regular verbs. Names
 of the suffix keys used in these rules: \\
SFKregptpp = regular past tense and past participle form\\
SFKcdptpp  = consonant doubling past tense and past participle form.\\
Several variables have been defined:
\begin{description}
\item[C] the set of consonants. If these letters precede `y', `y' changes 
into `i' 
if `ed' , `s', `er' or `est' is added. For the current rules, the set is 
needed because of its importance for {\em -ed\/} attachment: \\
{\em try - tried; stay - stayed\/}.
\item[V] the set of vowels, excluding `y'. If these letters precede `y', 
`y' doesn't change into `i'.
\item[Else]  the set of all letters except `e' and `y'. This set allows normal 
addition of `ed'; `e' and `y' are special if `ed', `er' or `est' is added.
\item[D] the set of letters that can be doubled before `-ed'  and `-ing'.
\end{description}
Both C and Else may be too large; it may be the case that some forms 
allowed by this formulation do not actually exist. 
Especially `i' in Else may give unwarranted ambiguities in analysis of 
regular past tenses and past participles (see the rule), but since there 
is the `normal' verb {\em ski\/} ending in `i', it cannot be removed from 
this set.

\begin{verbatim}

VAR

C    = [b, c, d, f, g, h, j, k, l, m, n, p, q, r, s, t, v, w, x, z] ;
V    = [a, e, i, o, u];
Else = [a, b, c, d, f, g, h, i, j, k, l, m, n, o, p, q, r, s, t, 
        u, v, w, x, z];
D    = [b, d, f, g, k, l, m, n, p, r, s, t, z]; 


TABLE

(10)  *e             + SFKregptpp   :: *ed;        (pt + pp)
(10)  *(C)y          + SFKregptpp   :: *(C)ied;    (pt + pp)
(10)  *(V)y          + SFKregptpp   :: *(V)yed;    (pt + pp)
(10)  *(Else)        + SFKregptpp   :: *(Else)ed;  (pt + pp)
(11)  *(D)           + SFKcdptpp    :: *(D)(D)ed;  (pt + pp)
(11)  *c             + SFKcdptpp    :: *cked;      (pt + pp)

\end{verbatim}

\subsubsection{Irregular Past and Perfect}
This subsection descibes the rules that make past tense and past participle 
forms  for irregular verbs. Between brackets is indicated whether the rule 
is for a past tense (pt) or for a past participle (pp), or both. 
The numbers  in front of the verbs indicate the conjugation class. In the 
lextree rules, this conjugation class will be used to determine whether the 
combination of verb and suffix key as found in the segmentation rules 
was correct.  The rules are ordered on alphabetical ending of the stem, 
so alphabetically when reading the stem from right to left.
The names of the suffix keys are:\\
SFKirrpt = irregular past tense form\\
SFKirrpp = irregular past participle form\\
SFKirrptpp = irregular past tense and past participle form\\
SFKirrpt2 = second irregular past tense form\\
SFKirrpp2 = second irregular past participle form\\
SFKirrpp3 = third irregular past participle form \\
SFKirrnc = irregular, no change of form
\\[1 ex]
\begin{verbatim}

(2)   *lead          + SFKirrptpp  :: *led;       (pt + pp)
(3)   *spread        + SFKirrnc    :: *spread;    (pt + pp)
(1)   *tread         + SFKirrpt    :: *trod;      (pt)
(1)   *tread         + SFKirrpp    :: *trodden;   (pp)
(2)   *feed          + SFKirrptpp  :: *fed;       (pt + pp)
(2)   *bleed         + SFKirrptpp  :: *bled;      (pt + pp)
(2)   *speed         + SFKirrptpp  :: *sped;      (pt + pp)
(2)   *breed         + SFKirrptpp  :: *bred;      (pt + pp)
(3)   *shed          + SFKirrnc    :: *shed;      (pt + pp)
(3)   *wed           + SFKirrnc    :: *wed;       (pt + pp, also regular)
(3)   *bid           + SFKirrnc    :: *bid;       (pt + pp)
(1)   *bid           + SFKirrpt    :: *bade;      (pt)
(1)   *bid           + SFKirrpp    :: *bidden;    (pp)
(5)   *forbid        + SFKirrpt2   :: *forbad;    (pt)
(6)   *forbid        + SFKirrpp2   :: *forbid;    (pp)
(3)   *rid           + SFKirrnc    :: *rid;       (pt + pp. also regular)
(5)   *gild          + SFKirrpt2   :: *gilt;      (pt)
(2)   *build         + SFKirrptpp  :: *built;     (pt + pp)
(2)   *hold          + SFKirrptpp  :: *held;      (pt + pp)
(2)   *stand         + SFKirrptpp  :: *stood;     (pt + pp)
(2)   *bend          + SFKirrptpp  :: *bent;      (pt + pp)
(2)   *lend          + SFKirrptpp  :: *lent;      (pt + pp)
(2)   *spend         + SFKirrptpp  :: *spent;     (pt + pp)
(2)   *rend          + SFKirrptpp  :: *rent;      (pt + pp)
(2)   *send          + SFKirrptpp  :: *sent;      (pt + pp)
(2)   *bind          + SFKirrptpp  :: *bound;     (pt + pp)
(2)   *find          + SFKirrptpp  :: *found;     (pt + pp)
(2)   *grind         + SFKirrptpp  :: *ground;    (pt + pp)
(2)   *wind          + SFKirrptpp  :: *wound;     (pt + pp)
(2)   *gird          + SFKirrptpp  :: *girt;      (pt + pp)

(4)   *lade          + SFKirrpp    :: *laden;     (pp)
(2)   *bide          + SFKirrptpp  :: *bode;      (pt + pp)
(2)   *hide          + SFKirrptpp  :: *hid;       (pt + pp)
(6)   *hide          + SFKirrpp2   :: *hidden;    (pp)
(2)   *chide         + SFKirrptpp  :: *chid;      (pt + pp, also regular)
(6)   *chide         + SFKirrpp2   :: *chidden;   (pp, also regular)
(2)   *slide         + SFKirrptpp  :: *slid;      (pt + pp)
(6)   *slide         + SFKirrpp2   :: *slidden;   (pp, as an adjective)
(1)   *ride          + SFKirrpt    :: *rode;      (pt)
(1)   *ride          + SFKirrpp    :: *ridden;    (pp)
(2)   *stride        + SFKirrptpp  :: *strode;    (pt + pp)
(7)   *stride        + SFKirrpp3   :: *stridden;  (pp, rare)
(6)   *stride        + SFKirrpp2   :: *strid;     (pp)
(2)   *flee          + SFKirrptpp  :: *fled;      (pt + pp)
(1)   *see           + SFKirrpt    :: *saw;       (pt)
(1)   *see           + SFKirrpp    :: *seen;      (pp)
(1)   *lie           + SFKirrpt    :: *lay;       (pt)
(1)   *lie           + SFKirrpp    :: *lain;      (pp)
(1)   *shake         + SFKirrpt    :: *shook;     (pt)
(1)   *shake         + SFKirrpp    :: *shaken;    (pp)
(2)   *make          + SFKirrptpp  :: *made;      (pt + pp)
(1)   *forsake       + SFKirrpt    :: *forsook;   (pt)
(1)   *forsake       + SFKirrpp    :: *forsaken;  (pp)
(1)   *take          + SFKirrpt    :: *took;      (pt)
(1)   *take          + SFKirrpp    :: *taken;     (pp)
(1)   *wake          + SFKirrpt    :: *woke;      (pt, also regular)
(1)   *wake          + SFKirrpp    :: *woken;     (pp, also regular)
(2)   *strike        + SFKirrptpp  :: *struck;    (pt + pp)
(8)   *come          + SFKirrpt    :: *came;      (pt)
(8)   *come          + SFKirrnc    :: *come;      (pp)
(2)   *shine         + SFKirrptpp  :: *shone;     (pt + pp)
(2)   *shoe          + SFKirrptpp  :: *shod;      (pt + pp)
(5)   *dare          + SFKirrpt2   :: *durst;     (pt, archaic; see very 
                                                   irregular verbs for 
                                                   `dared')
(1)   *rise          + SFKirrpt    :: *rose;      (pt)
(1)   *rise          + SFKirrpp    :: *risen;     (pp)
(2)   *lose          + SFKirrptpp  :: *lost;      (pt + pp)
(1)   *choose        + SFKirrpt    :: *chose;     (pt)
(1)   *choose        + SFKirrpp    :: *chosen;    (pp)
(1)   *bite          + SFKirrpt    :: *bit;       (pt)
(1)   *bite          + SFKirrpp    :: *bitten;    (pp)
(1)   *smite         + SFKirrpt    :: *smote;     (pt)
(1)   *smite         + SFKirrpp    :: *smitten;   (pp)
(1)   *write         + SFKirrpt    :: *wrote;     (pt)
(1)   *write         + SFKirrpp    :: *written;   (pp)
(2)   *heave         + SFKirrptpp  :: *hove;      (pt + pp)
(2)   *leave         + SFKirrptpp  :: *left;      (pt + pp)
(1)   *cleave        + SFKirrpt    :: *clove;     (pt)
(1)   *cleave        + SFKirrpp    :: *cloven;    (pp)
(2)   *bereave       + SFKirrptpp  :: *bereft;    (pt + pp, also regular, 
                                                   less common)
(1)   *weave         + SFKirrpt     :: *wove;      (pt)
(1)   *weave         + SFKirrpp     :: *woven;     (pp)
(6)   *grave         + SFKirrpp2    :: *graven;    (pp, also regular)
(2)   *stave         + SFKirrptpp   :: *stove;     (pt + pp, also regular)
(2)   *reeve         + SFKirrptpp   :: *rove;      (pt + pp)
(5)   *dive          + SFKirrpt2    :: *dove;      (pt AmE, 
                                                    regular in BrE)
(1)   *give          + SFKirrpt     :: *gave;      (pt)
(1)   *give          + SFKirrpp     :: *given;     (pp)
(6)   *rive          + SFKirrpp2    :: *riven;     (pp)
(1)   *drive         + SFKirrpt     :: *drove;     (pt)
(1)   *drive         + SFKirrpp     :: *driven;    (pp)
(1)   *shrive        + SFKirrpt     :: *shrove;    (pt)
(1)   *shrive        + SFKirrpp     :: *shriven;   (pp)
(5)   *thrive        + SFKirrpt2    :: *throve;    (pt, also regular)
(6)   *thrive        + SFKirrpp2    :: *thriven;   (pp, also regular)
(1)   *strive        + SFKirrpt     :: *strove;    (pt)
(1)   *strive        + SFKirrpp     :: *striven;   (pp)
(6)   *prove         + SFKirrpp2    :: *proven;    (pp, also regular)
(1)   *freeze        + SFKirrpt     :: *froze;     (pt)
(1)   *freeze        + SFKirrpp     :: *frozen;    (pp)


(2)   *dig           + SFKirrptpp   :: *dug;       (pt + pp)
(2)   *hang          + SFKirrptpp   :: *hung;      (pt + pp, regular in 
                                                    meaning
                                                    `put to death') 
(2)   *cling         + SFKirrptpp   :: *clung;     (pt + pp)
(2)   *fling         + SFKirrptpp   :: *flung;     (pt + pp)
(1)   *ring          + SFKirrpt     :: *rang;      (pt)
(1)   *ring          + SFKirrpp     :: *rung;      (pp)
(2)   *bring         + SFKirrptpp   :: *brought;   (pt + pp)
(2)   *sling         + SFKirrptpp   :: *slung;     (pt + pp)
(2)   *string        + SFKirrptpp   :: *strung;    (pt + pp)
(2)   *wring         + SFKirrptpp   :: *wrung;     (pt + pp)
(1)   *sing          + SFKirrpt     :: *sang;      (pt)
(1)   *sing          + SFKirrpp     :: *sung;      (pp)
(2)   *sting         + SFKirrptpp   :: *stung;     (pt + pp)
(2)   *swing         + SFKirrptpp   :: *swung;     (pt + pp)

(2)   *teach         + SFKirrptpp   :: *taught;    (pt + pp)  
(2)   *beseech       + SFKirrptpp   :: *besought;  (pt + pp)
(2)   *catch         + SFKirrptpp   :: *caught;    (pt + pp)

(1)   *speak         + SFKirrpt     :: *spoke;     (pt)
(1)   *speak         + SFKirrpp     :: *spoken;    (pp)
(1)   *break         + SFKirrpt     :: *broke;     (pt)
(1)   *break         + SFKirrpp     :: *broken;    (pp)
(2)   *stick         + SFKirrptpp   :: *stuck;     (pt + pp)
(2)   *seek          + SFKirrptpp   :: *sought;    (pt + pp)
(2)   *think         + SFKirrptpp   :: *thought;   (pt + pp)
(2)   *slink         + SFKirrptpp   :: *slunk;     (pt + pp)
(1)   *drink         + SFKirrpt     :: *drank;     (pt)
(1)   *drink         + SFKirrpp     :: *drunk;     (pp)
(2)   *shrink        + SFKirrptpp   :: *shrunk;    (pt + pp)
(1)   *shrink        + SFKirrpt     :: *shrank;    (pt)
(1)   *shrink        + SFKirrpp     :: *shrunken;  (pp, as an adjective)
(1)   *sink          + SFKirrpt     :: *sank;      (pt)
(1)   *sink          + SFKirrpp     :: *sunk;      (pp)
(5)   *stink         + SFKirrpt2    :: *stank;     (pt)
(2)   *stink         + SFKirrptpp   :: *stunk;     (pt + pp)

(2)   *deal          + SFKirrptpp   :: *dealt;     (pt + pp)
(1)   *steal         + SFKirrpt     :: *stole;     (pt)
(1)   *steal         + SFKirrpp     :: *stolen;    (pp)
(2)   *feel          + SFKirrptpp   :: *felt;      (pt + pp) 
(2)   *kneel         + SFKirrptpp   :: *knelt;     (pt + pp, 
                                                    also regular, AmE)
(2)   *spoil         + SFKirrptpp   :: *spoilt;    (pt + pp, 
                                                    also regular, AmE)
(1)   *fall          + SFKirrpt     :: *fell;      (pt)
(1)   *fall          + SFKirrpp     :: *fallen;    (pp)
(2)   *smell         + SFKirrptpp   :: *smelt;     (pt + pp, 
                                                    also regular, AmE)
(2)   *spell         + SFKirrptpp   :: *spellt;    (pt + pp, 
                                                    also regular, AmE)
(2)   *sell          + SFKirrptpp   :: *sold;      (pt + pp)
(2)   *tell          + SFKirrptpp   :: *told;      (pt + pp)
(2)   *dwell         + SFKirrptpp   :: *dwelt;     (pt + pp, 
                                                    also regular, AmE)
(6)   *swell         + SFKirrpp2    :: *swollen;   (pp, also regular)
(2)   *spill         + SFKirrptpp   :: *spilt;     (pt + pp, 
                                                    also regular, AmE)

(2)   *dream         + SFKirrptpp   :: *dreamt;    (pt + pp, 
                                                    also regular, AmE)
(1)   *swim          + SFKirrpt     :: *swam;      (pt)
(1)   *swim          + SFKirrpp     :: *swum;      (pp)

(2)   *lean          + SFKirrptpp   :: *leant;     (pt + pp, 
                                                    also regular, AmE)
(2)   *mean          + SFKirrptpp   :: *meant;     (pt + pp)
(1)   *begin         + SFKirrpt     :: *began;     (pt)
(1)   *begin         + SFKirrpp     :: *begun;     (pp)
(2)   *spin          + SFKirrptpp   :: *spun;      (pt + pp)
(5)   *spin          + SFKirrpt2    :: *span;      (pt, archaic)
(2)   *win           + SFKirrptpp   :: *won;       (pt + pp)
(2)   *learn         + SFKirrptpp   :: *learnt;    (pt + pp, 
                                                    also regular, AmE)
(2)   *burn          + SFKirrptpp   :: *burnt;     (pt + pp, 
                                                    also regular, AmE)
(8)   *run           + SFKirrpt     :: *ran;       (pt)
(8)   *run           + SFKirrnc     :: *run;       (pp)

(1)   *do            + SFKirrpt     :: *did;       (pt)
(1)   *do            + SFKirrpp     :: *done;      (pp)
(1)   *go            + SFKirrpt     :: *went;      (pt)
(1)   *go            + SFKirrpp     :: *gone;      (pp)

(2)   *leap          + SFKirrptpp   :: *leapt;     (pt + pp, 
                                                    also regular, AmE)
(2)   *sleep         + SFKirrptpp   :: *slept;     (pt + pp)
(2)   *keep          + SFKirrptpp   :: *kept;      (pt + pp)
(2)   *creep         + SFKirrptpp   :: *crept;     (pt + pp)
(2)   *weep          + SFKirrptpp   :: *wept;      (pt + pp)

(1)   *bear          + SFKirrpt     :: *bore;      (pt)
(1)   *bear          + SFKirrpp     :: *borne;     (pp)
(2)   *hear          + SFKirrptpp   :: *heard;     (pt + pp)
(6)   *shear         + SFKirrpp2    :: *shorn;     (pp, also regular)
(1)   *tear          + SFKirrpt     :: *tore;      (pt)
(1)   *tear          + SFKirrpp     :: *torn;      (pp)
(1)   *wear          + SFKirrpt     :: *wore;      (pt)

(2)   *bless         + SFKirrptpp   :: *blest;     (pt + pp)


(1)   *eat           + SFKirrpt     :: *ate;       (pt)
(1,9) *eat           + SFKirrpp     :: *eaten;     (pp)    
(9)   *beat          + SFKirrnc     :: *beat;      (pt)
(6)   *beat          + SFKirrpp2    :: *beat;      (pp)
(3)   *sweat         + SFKirrnc     :: *sweat;     (pt + pp, also regular)
(2)   *meet          + SFKirrptpp   :: *met;       (pt + pp)
(3)   *bet           + SFKirrnc     :: *bet;       (pt + pp)
(2)   *get           + SFKirrptpp   :: *got;       (pt + pp)
(6)   *get           + SFKirrpp2    :: *gotten;    (pt + pp)
(1)   *beget         + SFKirrpt     :: *begot;     (pt)
(1)   *beget         + SFKirrpp     :: *begotten;  (pp)
(1)   *forget        + SFKirrpt     :: *forgot;    (pt)
(1)   *forget        + SFKirrpp     :: *forgotten; (pp)
(3)   *let           + SFKirrnc     :: *let;       (pt + pp)
(3)   *set           + SFKirrnc     :: *set;       (pt + pp)
(3)   *wet           + SFKirrnc     :: *wet;       (pt + pp, also regular)
(2)   *fight         + SFKirrptpp   :: *fought;    (pt + pp)
(2)   *light         + SFKirrptpp   :: *lit;       (pt + pp, also regular)
(3)   *hit           + SFKirrnc     :: *hit;       (pt + pp)
(3)   *slit          + SFKirrnc     :: *slit;      (pt + pp)
(3)   *split         + SFKirrnc     :: *split;     (pt + pp)
(3)   *knit          + SFKirrnc     :: *knit;      (pt + pp, also regular) 
(2)   *spit          + SFKirrptpp   :: *spat;      (pt + pp)
(2)   *sit           + SFKirrptpp   :: *sat;       (pt + pp)
(3)   *quit          + SFKirrnc     :: *quit;      (pt + pp, also regular)
(2)   *shoot         + SFKirrptpp   :: *shot;      (pt + pp)
(3)   *hurt          + SFKirrnc     :: *hurt;      (pt + pp)
(3)   *cast          + SFKirrnc     :: *cast;      (pt + pp)
(3)   *cost          + SFKirrnc     :: *cost;      (pt + pp)   
(3)   *burst         + SFKirrnc     :: *burst;     (pt + pp)
(3)   *thrust        + SFKirrnc     :: *thrust;    (pt + pp)
(3)   *cut           + SFKirrnc     :: *cut;       (pt + pp)
(3)   *shut          + SFKirrnc     :: *shut;      (pt + pp)
(3)   *put           + SFKirrnc     :: *put;       (pt + pp)

(1)   *draw          + SFKirrpt     :: *drew;      (pt)
(1)   *draw          + SFKirrpp     :: *drawn;     (pp)   
(6)   *saw           + SFKirrpp2    :: *sawn;      (pp, also regular)
(6)   *hew           + SFKirrpp2    :: *hewn;      (pp, also regular)
(6)   *strew         + SFKirrpp2    :: *strewn;    (pp, also regular)
(6)   *sew           + SFKirrpp2    :: *sewn;      (pp, also regular)
(6)   *show          + SFKirrpp2    :: *shown;     (pp, also regular)
(1)   *blow          + SFKirrpt     :: *blew;      (pt)
(1)   *blow          + SFKirrpp     :: *blown;     (pp)
(6)   *mow           + SFKirrpp2    :: *mown;      (pp, also regular)
(1)   *know          + SFKirrpt     :: *knew;      (pt)
(1)   *know          + SFKirrpp     :: *known;     (pp)
(5)   *crow          + SFKirrpt2    :: *crew;      (pt, archaic)
(1)   *grow          + SFKirrpt     :: *grew;      (pt)
(1)   *grow          + SFKirrpp     :: *grown;     (pp)
(1)   *throw         + SFKirrpt     :: *threw;     (pt)
(1)   *throw         + SFKirrpp     :: *thrown;    (pp)
(6)   *sow           + SFKirrpp2    :: *sown;      (pp, also regular)

(2)   *lay           + SFKirrptpp   :: *laid;      (pt + pp)
(1)   *slay          + SFKirrpt     :: *slew;      (pt)
(1)   *slay          + SFKirrpp     :: *slain;     (pp)
(2)   *pay           + SFKirrptpp   :: *paid;      (pt + pp)
(2)   *say           + SFKirrptpp   :: *said;      (pt + pp)
(1)   *fly           + SFKirrpt     :: *flew;      (pt)
(1)   *fly           + SFKirrpp     :: *flown;     (pp)   
(2)   *buy           + SFKirrptpp   :: *bought;    (pt + pp)

\end{verbatim}
A few forms have been removed from the set, because they are not thought to be 
likely or because they only apply to adjectives. Some other forms which also 
occur mostly as adjectives have not been removed yet. Removed are: 
\begin{verbatim}
     *strike        + SFKirrpp2    :: *stricken; adj, but rare as verb
     *bite          + SFKirrpp2    :: *bit; rare as verb
     *shave         + SFKirrpp2    :: *shaven; only adjective
     *sink          + SFKirrpp2    :: *sunken; only adjective
     *melt          + SFKirrpp2    :: *molten; only adjective
\end{verbatim}
The conjugation classes for these verbs have been adapted accordingly, in as 
far as they were in the test dictionary. The large dictionaries have NOT been 
adapted yet, since they are not ordered alphabetically for English yet. Thus, 
there may be mistakes there! Since a number of minor changes were made in the 
rules, the list of doc.\ 115 specifying for irregular verbs in which 
conjugation class(es) they fall is now out of date. This list should be adapted
too.

The verbs that are irregular in British English but regular in American English 
are now filled with both conjugation classes. No restrictions have been laid 
down yet on the working of these rules, so all forms that are possible will be 
produced in generation.

\subsubsection{Ing-form}
This subsection describes the rules that make `-ing' forms of verbs, 
for progressive and nominalizations.
 Verbs have an attribute, {\em ingform\/}, to indicate how they form their -ing 
form, if any. For more information, see doc.\ 115 or the relevant rules in 
doc.\ 306 on English Lextree rules. The suffix keys that are used in the 
segmentation rules reflect the attribute values:\\
SFKreging = for regular ing-forms\\
SFKcding  = for consonant doubling ing-forms\\
SFKirring = for irregular ing-forms\\
Several variables had to be defined to make the segmentation rules work:
\begin{description}
\item[NotE] The set of letters that allow `-ing' to be attached to them without 
any change of form. Perhaps the set is too large.
\item[A] the set of letters that make a following `e' drop before `-ing'.
\item[CompA]  the complement of A, so the set of letters that do not make 
a following `e' drop when `-ing' is added. Notice that CompA does 
not include  `i' to prevent wrong rule-application (see rules). Perhaps the set 
is too large.
\item[D] the set of letters that can be doubled before `-ed'  and `-ing', so 
this set was also used in the past tense rules.
\end{description}

\begin{verbatim}

VAR

NotE  = [a, b, c, d, f, g, h, i, j, k, l, m, n, o, p, 
         q, r, s, t, u, v, w, x, y, z]; 
A     = [e, y, o];  
CompA = [a, b, c, d, f, g, h, j, k, l, m, n, p, q, r, s, t, 
         u, v, w, x, z];
D     = [b, d, f, g, k, l, m, n, p, r, s, t, z]; 
        
TABLE

*(D)                 + SFKcding     :: *(D)(D)ing;
*c                   + SFKcding     :: *cking;
*ie                  + SFKreging    :: *ying;       (die)
*(A)e                + SFKreging    :: *(A)eing;
*(CompA)e            + SFKreging    :: *(CompA)ing;
*(NotE)              + SFKreging    :: *(NotE)ing;
*ge                  + SFKirring    :: *geing;      (age, singe)    
\end{verbatim}

\newpage
\subsection{S-form}
The rules described in the current section make an `s'-form of a word. 
This can be either the third person
 singular form of a verb, or the plural of a noun. 

Verbs have an attribute  {\em sform\/}, to indicate what kind of s-form they 
take, if any. For more information, see doc.\ 115 or the relevant rule in 
doc.\ 306 on the lextree rules.

 Nouns have an attribute {\em plurforms\/}, the value of which is a set; nouns
 can have several plural forms. For the values of this attribute and their
 meaning see again doc.\ 115 or the relevant rule in doc.\ 306 on the lextree 
rules. 

The suffix keys that are used in the rules are:\\
SFKregS = the value corresponding to `regplur' (regular plural) of nouns and 
`regS' of verbs\\
SFKregEs = the value corresponding to `regEplur' (regular e-plural) of nouns 
and `regEs' of verbs\\
SFKvoicingS = the value corresponding to `voicingplur' of nouns (not 
important for verbs)\\
 SFKirrS = the value for nouns ending in (C)y that have `irrSplural'.\\
Two variables had to be defined for the s-formation rules:
\begin{description}
\item[CompCy] This set defines the complement of strings ending in 
C(onsonant)y and `quy'.  This set is necessary for the regular affixation 
of `s': \\
{\em buy - buys; try - tries\/}.
\item[F]  This set allows the plural to be formed by both `s' and ` 's'. 
This is     necessary for plurals of abbreviations and dates. See also the 
remarks made under the glue rules for genitives. 
\end{description}


\begin{verbatim}

VAR

CompCy = [a, b, c, d, e, f, g, h, i, j, k, l, m, n, o, p, 
          q, r, s, t, u, v, w, x, z, 
          ay, ey, oy, buy, guy, puy ];
F      = [A, B, C, D, E, F, G, H, I, J, K, L, M, N, O, P, 
          Q, R, S, T, U, V, W, X, Y, Z, 
          ., 0, 1, 2, 3, 4, 5, 6, 7, 8, 9];

TABLE

   *(CompCy)           + SFKregS        :: *(CompCy)s;
   *(C)y               + SFKregS        :: *(C)ies;
   *(C)y               + SFKirrS        :: *(C)ys;    (Germanys; drys)
   *(F)                + SFKregS        :: *(F)s;
/A *(F)                + SFKregS        :: *(F)'s;
   *quy                + SFKregS        :: *quies;
   *                   + SFKregEs       :: *es;       (hero, peach)
   *f                  + SFKvoicingS    :: *ves;
   *ife                + SFKvoicingS    :: *ives;
   *do                 + SFKirrS        :: *does;
   *go                 + SFKirrS        :: *goes;

\end{verbatim}
The different attribute-values for regS and regEs are necessary, because it is
 not possible to predict from the string whether the added affix must be `s' or
 `es'. This is because it is really a phonological process, and the spelling
 is not always the same as the pronunciation. This applies especially to words
 ending in `ch'. Most of the times this is a sibilant, and the added affix is 
 `es', but sometimes the pronunciation is /k/, e.g, in {\em monarch\/}, 
{\em patriarch\/}
 etc. The solution that has been chosen here is to mark every verb in the 
 dictionary with an attribute-value that spells out this phonetic information.
 Another possibility would be to use phonetic markers in these rules. The
 rules would be a little more complicated. 

\subsection{Rules for Nouns}
\subsubsection{Regular Plural}
The regular plural of nouns has been dealt with in the previous subsection, on 
sForms.
\subsubsection{Irregular Plural}
This subsection contains the rules for irregular plurals,. There are two kinds 
of irregular plurals:\\
SFKirrplur = irregular plurals\\
SFKlatplur = latinate (or other foreign) plurals\\
First the plurals with mutation of the vowel are given, then -`en' plurals,
then foreign plurals, and finally other irregular plurals. When the large 
dictionaries were filled automatically with information from the Van Dale 
N-E dictionary, several new irregular forms were discovered. They have mostly 
been given the value {\em irrplur\/}, because that gave the least problems with 
other words having a {\em latplur\/}.

To speed up analysis, a variable {\em All\/} was defined that basically spells 
out the non-empty set.
     This set is used to exclude the empty string in rules that originally
     ended in *{\em a\/}; now they end in *{\em (All)a\/}, so that the number 
of wrong segmentations of the article `a' is greatly reduced.

\begin{verbatim}

VAR

All = [a,b,c,d,e,f,g,h,i,j,k,l,m,n,o,p,q,r,s,t,u,v,w,x,y,z,�,�];

TABLE

   *foot          + SFKirrplur      :: *feet;
   *tooth         + SFKirrplur      :: *teeth;
   *goose         + SFKirrplur      :: *geese;
   *louse         + SFKirrplur      :: *lice;
   *mouse         + SFKirrplur      :: *mice;
   *man           + SFKirrplur      :: *men;

   *brother       + SFKirrplur      :: *brethren;  (= `fellow 
                                        members of a religious society')
   *child         + SFKirrplur      :: *children;
   *ox            + SFKirrplur      :: *oxen;

   *us            + SFKlatplur      :: *i;
   *a             + SFKlatplur      :: *ae;
   *(All)um       + SFKlatplur      :: *(All)a;
   *ex            + SFKlatplur      :: *ices;
   *ix            + SFKlatplur      :: *ices;
   *is            + SFKlatplur      :: *es;
   *(All)on       + SFKlatplur      :: *(All)a;
   *eau           + SFKlatplur      :: *eaux;
   *ieu           + SFKlatplur      :: *ieux;
   *o             + SFKlatplur      :: *i;

   *iau           + SFKlatplur      :: *iaux;     
   *ux            + SFKlatplur      :: *uces;      

   *fez           + SFKirrplur      :: *fezzes;
   *quiz          + SFKirrplur      :: *quizzes;
   *bus           + SFKirrplur      :: *busses;   (AmE, BrE is regular)
   *die           + SFKirrplur      :: *dice;
   *penny         + SFKirrplur      :: *pence;

   *corpus        + SFKirrplur      :: *corpora;    (also regular)
   *genus         + SFKirrplur      :: *genera;
   *ma            + SFKirrplur      :: *mata;   
   *bandit        + SFKirrplur      :: *banditti;
   *dilettante    + SFKirrplur      :: *dilettanti;

   *cherub        + SFKirrplur      :: *cherubim;     (also regular)
   *seraph        + SFKirrplur      :: *seraphim;     (also regular)
   *kibbutz       + SFKirrplur      :: *kibbutzim;

   *judoka        + SFKirrplur      :: *judokas;   (pluralform not found
                                                    in dictinoaries...)
   *sol           + SFKirrplur      :: *soles;
   *jus           + SFKirrplur      :: *jura;       
   *opus          + SFKirrplur      :: *opera;      
   *lira          + SFKirrplur      :: *lire;    
   *polis         + SFKirrplur      :: *poleis;  
   *ancon         + SFKirrplur      :: *ancones; 
   *bijou         + SFKirrplur      :: *bijoux;    
   *markka        + SFKirrplur      :: *markkaa;    
   *stotinka      + SFKirrplur      :: *stotinki;   
   *walker-on     + SFKirrplur      :: *walkers-on;  
   *looker-on     + SFKirrplur      :: *lookers-on; 
\end{verbatim}
Two words were removed from the list: *{\em by\/} (for {\em lay-by\/}) and 
{\em dry\/} (meaning `prohibitionist'). These forms can be made by the 
regular plural rules (see above under {\em S-form\/}), assuming that the 
entries were specified as {\em irrS\/} in the lexicon. A compound form 
like {\em passer-by\/}, which probably has a plural {\em passers-by\/}, is not 
dealt with yet.


\newpage
\subsection{Rules for Adjectives and Adverbs}
\subsubsection{Regular Comparative and Superlative}
This section describes the rules that make comparative and superlative forms 
of adjectives and adverbs.
 Adjectives and adverbs have an inherent attribute `compformations', the value
 of which is a set. Its values (see doc.\ 115 or the relevant rules in doc.\ 
306 on lextree rules) are directly related to the suffix keys used in the 
segmentation rules for the steps of comparison. The attribute has one value 
for both comparative and superlative (they are both either regular or 
irregular). In the suffix keys, a distinction must be made between comparatives 
and superlatives. The values for the regular rules are:\\
SFKregcomp : for regular comparative forms,\\
SFKregsuper : for regular superlative forms,\\
SFKcdcomp : for comparative forms that show consonant doubling,\\
SFKcdsuper : for superlative forms that show consonant doubling
\\[1 ex]
Several variables have been defined:
\begin{description}
\item[C] the set of consonants. If these letters precede `y', `y' changes 
into `i' 
if `ed' , `s', `er' or `est' is added. For the current rules, the set is 
needed because of its importance for {\em er/est\/} attachment: \\
{\em dry - drier; coy - coyer(?)\/}.
\item[V] the set of vowels, excluding `y'. If these letters precede `y', 
`y' doesn't change into `i'.
\item[Else]  the set of all letters except `e' and `y'. This set allows normal 
addition of `er/est'; `e' and `y' are special if `ed', `er' or `est' is added.
\item[E] is the same as D (used for consonant doubling of -ing and -ed forms), 
except that it also contains `c'. These letters can be
     doubled in comparative and superlative adjectives.
\end{description}

\begin{verbatim}

VAR
C    = [b, c, d, f, g, h, j, k, l, m, n, p, q, r, s, t, v, w, x, z] ;
V    = [a, e, i, o, u];
Else = [a, b, c, d, f, g, h, i, j, k, l, m, n, o, p, q, r, s, t, 
        u, v, w, x, z];
E    = [b, c, d, f, g, k, l, m, n, p, r, s, t, z];

TABLE 

*(E)                + SFKcdcomp           :: *(E)(E)er;
*(E)                + SFKcdsuper          :: *(E)(E)est;
*(C)y               + SFKregcomp          :: *(C)ier;
*(V)y               + SFKregcomp          :: *(V)yer;
*(C)y               + SFKregsuper         :: *(C)iest;
*(V)y               + SFKregsuper         :: *(V)yest;
*e                  + SFKregcomp          :: *er;
*e                  + SFKregsuper         :: *est;
*(Else)             + SFKregcomp          :: *(Else)er;
*(Else)             + SFKregsuper         :: *(Else)est;

\end{verbatim}
Forms that end in {\em -(C)y\/} but do not change the `y' to an `i' are treated 
in the next subsection.

\subsubsection{Irregular Comparative and Superlative}
This section describes the rules for irregular comparative and superlative 
forms of adjectives
 and adverbs. Some adjectives and adverbs have the same form; in that case
 both can occur in comparative and superlative form. There are, however, some
 adverbs not corresponding to adjectives, that undergo comparative formation.
The suffix keys used in the rules are:\\
SFKirrcomp = for irregular comparative forms\\
SFKirrsuper = for irregular superlative forms\\
SFKirrcomp2   = for a second irregular comparative form\\
SFKirrsuper2  = for a second irregular comparative form
\\[1 ex]
\begin{verbatim}

   *good         + SFKirrcomp       :: *better;
   *good         + SFKirrsuper      :: *best;
   *bad          + SFKirrcomp       :: *worse;
   *bad          + SFKirrsuper      :: *worst;
   *far          + SFKirrcomp       :: *farther;
   *far          + SFKirrcomp2      :: *further;
   *far          + SFKirrsuper      :: *farthest;
   *far          + SFKirrsuper2     :: *furthest;
   *old          + SFKirrcomp       :: *elder;     (also regular)
   *old          + SFKirrsuper      :: *eldest;    (also regular)

   *well         + SFKirrcomp       :: *better;
   *well         + SFKirrsuper      :: *best;
   *badly        + SFKirrcomp       :: *worse;
   *badly        + SFKirrsuper      :: *worst;
   *little       + SFKirrcomp       :: *less;
   *little       + SFKirrsuper      :: *least;
   *much         + SFKirrcomp       :: *more;
   *much         + SFKirrsuper      :: *most;
   *many         + SFKirrcomp       :: *more;     
   *many         + SFKirrsuper      :: *most;    

   *shy          + SFKirrcomp       :: *shyer;    
   *shy          + SFKirrsuper      :: *shyest;   
   *sly          + SFKirrcomp       :: *slyer;  
   *sly          + SFKirrsuper      :: *slyest;

/A *wry          + SFKirrcomp       :: *wryer;     
/A *wry          + SFKirrsuper      :: *wryest;     
/A *dry          + SFKirrcomp       :: *dryer;     
/A *dry          + SFKirrsuper      :: *dryest;

\end{verbatim}

\newpage
\subsection{Rules for Pronouns}
There are rules for the accusative of personal pronouns (PERSPROs), for 
adjectival possessive pronouns (POSSADJs) and predicative possessive pronouns 
(POSSPROs), and for interrogative pronouns (WHPROs and WhPOSSADJs).

\subsubsection{Perspro}
This subsection gives the rules for PERSPROs. There is only one suffix key:\\
SFKacc = for accusative form.
\begin{verbatim}

I          + SFKacc       :: me;
you        + SFKacc       :: you;
he         + SFKacc       :: him;
she        + SFKacc       :: her;
it         + SFKacc       :: it;
we         + SFKacc       :: us;
they       + SFKacc       :: them;

\end{verbatim}

\subsubsection{Possadj}
This subsection gives the rules for POSSADJs. There is only one suffix key:\\
SFKposs = for possessive form of the personal pronoun, used attributively.
\begin{verbatim}

I          + SFKposs      :: my;
you        + SFKposs      :: your;
he         + SFKposs      :: his;
she        + SFKposs      :: her;
it         + SFKposs      :: its;
we         + SFKposs      :: our;
they       + SFKposs      :: their;

\end{verbatim}

\subsubsection{Posspro}
This subsection gives the rules for POSSPROs. There is only one suffix key:\\
SFKpredposs = for possessive form of the personal pronoun, used predicatively.
\begin{verbatim}

I          + SFKpredposs  :: mine;
you        + SFKpredposs  :: yours;
he         + SFKpredposs  :: his;
she        + SFKpredposs  :: hers;
we         + SFKpredposs  :: ours;
they       + SFKpredposs  :: theirs;

\end{verbatim}

\subsubsection{Whpro}
This subsection gives the rules for WHPROs. There are two rules, one to make
the accusative of an interrogative pronoun (WHPRO), one to make an adjectival 
interrogative possessive pronoun (WhPOSSADJ). There are two suffix keys:\\
SFKacc = for the accusative form of the interrogative pronoun\\
SFKposs = for the possessive form of the interrogative pronoun
\begin{verbatim}

who        + SFKacc       :: whom;

who        + SFKposs      :: whose;

\end{verbatim}

\newpage
\subsection{Derivation Rules}
As was mentioned in the introduction, Rosetta3 has not incorporated many rules 
for derivation yet. There is one kind of derivation that is dealt with 
extensively: adjective to adverb derivation. This is a fairly regular process, 
and must be dealt with because many adverbs that are related to an adjective 
are not in the dictionary. A first step to extending the set of derivation 
rules was to include rules for affixation of {\em -able\/} to verbs. However, 
the M-grammar has not incorporated these rules yet, so presently only 
morphology can deal with these forms.

\subsubsection{Adjective - Adverb}
This subsection gives the rules for the derivation of Advs from Adjs by means 
of the  derivational affix `-ly'.
 It is assumed that the adjectives marked with an asterisk in the current 
rules are given the value {\em irregAdv\/} in the 
 dictionary for the attribute `advformationtype' (even though some of them take
 a seemingly regular -ly suffix), and that the lextree-rules link this 
 attribute-value with the SFKirregly. All other adjectives are regAdv or 
 zeroAdv, which means that the form for adjective and adverb is the same. 
 They may of course also be noAdv.
 Note that, contrary to tradition, {\em true\/} and {\em due\/} are marked as 
exceptions 
 in advformation, assuming that words like {\em vague\/} and {\em unique\/}, 
also ending  in -ue, are regular. The suffix keys that are used are:\\
SFKregly = for regular ly-affixation\\
SFKirregly = for irregular -ly affixation\\
SFKnoly = for derivation without change of form\\
Several variables were defined:  
\begin{description}
\item[ADJ] The letters in this set allow for a regular -ly affixation; 
for `c', `e', `l' and `y' (not in the set) several options exist
\item[notL] The set of letters that allow regular -ly affixation if they 
precede `e'
\item[notI] The set of letters that allow regular -ly affixation if they 
precede `c'.
\end{description}

\begin{verbatim}

VAR

ADJ  = [a,b,d,f,g,h,i,j,k,m,n,o,p,q,r,s,t,u,v,w,x,z];
notL = [a,b,c,d,e,f,g,h,i,j,k,m,n,o,p,q,r,s,t,u,v,w,x,y,z];
notI = [a,b,c,d,e,f,g,h,j,k,l,m,n,o,p,q,r,s,t,u,v,w,x,y,z];

TABLE

*(ADJ)    + SFKregly     :: *(ADJ)ly;       (regular adverb formation)

*(notI)c  + SFKregly     :: *(notI)cly;     (regular adverb formation)
*ic       + SFKregly     :: *ically;        (mechanic etc.)
*ic       + SFKirregly   :: *icly;          (*public)

*(notL)e  + SFKregly     :: *(notL)ely;     (regular adverb formation)
*le       + SFKregly     :: *ly;            (double, idle etc.)
*sole     + SFKirregly   :: *solely;        (*sole)
*whole    + SFKirregly   :: *wholly;        (*whole)

*true     + SFKirregly   :: *truly;         (*true)
*due      + SFKirregly   :: *duly;          (*due)

*(notL)l  + SFKregly     :: *(notL)lly;     (regular adverb formation)
*ll       + SFKregly     :: *lly;           (full, shrill etc.)

good      + SFKirregly   :: well;           

*y        + SFKregly     :: *ily;           (gay, dry, weary etc.)
*y        + SFKirregly   :: *yly;           (*shy, *sly, *spry, *wry)

*         + SFKnoly      :: *;              (adj = adv)

\end{verbatim}

\subsubsection{-Able}
This subsection gives the tentative rules for the derivation of verb to 
adjective by means of the suffix {\em -able\/}. There 
are no lextree rules for this derivation yet, because there is no attribute in 
the dictionary yet that indicates in what way the suffix should be attached.
When 
the grammar can also deal with -able derivation, perhaps the rules will
have to be modified. There are still only three suffix keys:\\
SFKregable = for regular -able affixation\\
SFKtruncable = for the cases where the stem ending {\em -ate\/} is truncated\\
SFKcdable = for affixation of -able that occurs whith consonant doubling\\
No irregular affixation has been discovered yet, but see the remarks below.
The variables that are used for the present rules are:
\begin{description}
\item[C] The set of consonants; these allow regular affixation of `-able'
\item[B] The set of letters that do not cause a following final `e' to drop 
before `-able' 
\item[CompB] Complement of B, so the set of letters that cause a following 
final `e' to drop before `-able' 
\item[D] the set of letters that can be doubled before `-ed'  and `-ing', so 
this set was also used in the past tense and ing-form rules.
\end{description}

\begin{verbatim}

VAR

C     = [b, c, d, f, g, h, j, k, l, m, n, p, q, r, s, t, v, w, x, z] ;
B     = [c, e, g, o, y];
CompB = [a, b, d, f, h, i, j, k, l, m, n, p, q, r, s, t, u, v, w, x, z];
D     = [b, d, f, g, k, l, m, n, p, r, s, t, z]; 

TABLE

   *(C)                + SFKregable          :: *(C)able;
   *(B)e               + SFKregable          :: *(B)eable;
   *(CompB)e           + SFKregable          :: *(CompB)able;
/A *(CompB)e           + SFKregable          :: *(CompB)eable;
   *(C)y               + SFKregable          :: *(C)iable;
   *(D)                + SFKcdable           :: *(D)(D)able;
   *ate                + SFKtruncable        :: *able;

\end{verbatim}

Note that only the final stem vowels `e' and `y' can be dealt with presently.
A verb like {\em ski\/} cannot form {\em skiable\/} yet, but is unclear whether 
it should. The third rule accepts forms in analysis where the `e' has not been 
dropped although the rule expected it, to be able to deal with `extra' forms 
like {\em sizeable\/} and {\em tameable\/}, that exist next to the regular 
forms. If {\em inflame\/} is the verb that underlies {\em inflammable\/}, there 
is an example of an irregular form.
The last rule is for words that end in the string
 `-ate'. Probably the rule should work only for those cases in which the string 
does not bear the accent:
 {\em demonstrate - demonstrable\/} vs.\ {\em state - *stable\/}. This can 
easily be indicated in the dictionary by means of the attribute value for -able 
affixation.

\end{document}


