%R0188.tex


   \documentstyle{Rosetta}
   \begin{document}
      \RosTopic{formalism}
      \RosTitle{Auxiliary-Domain Compiler}
      \RosAuthor{Chris Hazenberg}
      \RosDocNr{188}
      \RosDate{\today}
      \RosStatus{concept}
      \RosSupersedes{-}
      \RosDistribution{Linguists,Ren\'{e} Leermakers}
      \RosClearance{Project}
      \RosKeywords{auxiliary-domain,compiler}
      \MakeRosTitle
\section{Introduction}
This document describes an auxiliary-domain compiler i.e.\ a program that
converts a standard grammar into Pascal.The standard grammar predefines
some basic datatypes and may be a support for the linguists to write M-rules.
To be more exact, it contains a description of predefined keys(i.e.integers),
abbreviations and records for a certain language.

The output consists of a definition file and an implementation file.These modules
may be used as "inherited" ones at the ultimate Pascal version of the
M-rules.

In creating the auxiliary-domain compiler, use is made of the compiler
generator decribed in documents R0167 and R0172.

Section 2 deals with the syntax of the input. Section 3 describes the output.
And a small example will be given in section 4.
\section{Input}
\subsection{Syntax}
The syntax for the input grammar is as follows:
\begin{verbatim}
UTT          =  LANGVERSION . [KEYSECTION] . [RECSECTION] . [ABBRSECTION]

LANGVERSION  =  language .  colon . IDENTIFIER

KEYSECTION   =  keys . leftarrow . {ARGUMENTS} . rightarrow
ARGUMENTS    =  IDENTIFIER .  equivalent . ARGLIST
ARGLIST      =  leftarrow . key . equivalent . NUMBER .
                            term . equivalent . TERMARGUMENT .
                            category . equivalent . IDENTIFIER .
                rightarrow
NUMBER       =  IDENTIFIER
TERMARGUMENT =  IDENTIFIER | abstract | PUNCTUATION

RECSECTION   =  standard . records . leftarrow . {NUMRECORDS} . rightarrow
NUMRECORDS   =  IDENTIFIER . colon . TYPESECTION . FIELDLIST
TYPESECTION  =  IDENTIFIER
FIELDLIST    =  curlyopen . {IDENTIFIER . equivalent . FIELDVALUE} .
                curlyclose
FIELDVALUE   =  IDENTIFIER |(squareopen .(squareclose |
                                         (IDENTIFIER .{comma . IDENTIFIER}.
                                          squareclose)))
\end{verbatim}
\pagebreak
\begin{verbatim}

ABBRSECTION  =  abbrsets . leftarrow . {NUMABBRS} . rightarrow
NUMABBRS     =  IDENTIFIER . equivalent . ABBRLIST
ABBRLIST     =  squareopen . {IDENTIFIER . comma} . IDENTIFIER .
                squareclose

\end{verbatim}

The CAPITAL-typed strings are the non-terminals and the others the terminal strings.
An IDENTIFIER consists of characters and digits.A NUMBER may only consist of
digits.A PUNCTUATION may only be one of the most often used punctuations,
defined on the keyboard.

The next lists enumerate the terminal symbols of the grammar:
\begin{verbatim}

   leftarrow   : '<'
   rightarrow  : '>'
   squareopen  : '['
   squareclose : ']'
   curlyopen   : '{'
   curlyclose  : '}'
   equivalent  : '='
   comma       : ','
   colon       : ':'
\end{verbatim}
and the teminal strings of the grammar:
\begin{verbatim}
   abbreviations : 'ABBREVIATIONS'
   abstract      : 'ABSTRACT'
   category      : 'CATEGORY'
   keys          : 'KEYS'
   key           : 'KEY'
   language      : 'LANGUAGE'
   records       : 'RECORDS'
   abbrsets      : 'SETS'
   standard      : 'STANDARD'
   term          : 'WORD'
\end{verbatim}
The allowed punctuations are:
{\bf `} , {\bf '} , {\bf "} , {\bf @} , {\bf \#} , {\bf \$} , {\bf \%} ,
 {\bf \verb+^+} , {\bf \&} , {\bf *} , {\bf (} , {\bf )} , {\bf \_} ,
 {\bf -} , {\bf +} , {\bf =} , {\bf \{} ,{\bf \}} , {\bf [} , {\bf ]} ,
 {\bf $>$} , {\bf $<$} , {\bf :} , {\bf ;} , {\bf /} , {\bf $|$} ,
 {\bf $\backslash$} , {\bf ?} , {\bf ?`} , {\bf !} , {\bf !`} , {\bf ,}
 and {\bf .} .

\subsection{Explanation}
The input file enables the linguist to increase his domain, used in the
M-rules for a certain language.The input file consists of four separate
sections.

In the section {\bf LANGUAGE} you have to define the language.
In the section {\bf KEYS} it is possible to define keys that belong to words,
abstract data and punctuations.When a range for the different kinds of
categories is fixed, a small extension will create the possibility to check
the {\bf KEY} and its corresponding {\bf TERM} and {\bf CATEGORY}.
The {\bf STANDARD RECORDS} are implemented in such a way, that it is
possible to compare a certain record with this Standard Record.
In the section {\bf SETS} abbreviations of sets are defined.

When defining this auxiliary domain, attention must be paid to the following
facts:
\begin{itemize}
\item The input file has to be named: $<$ {\em Language} $>$:
{\em lsauxdomain.auxdom},where '{\em Language}' is the same as defined in
the input file.
\item The chosen language has to be one for which the {\em lsdomaint}-file is
accessable.
\item The record-types have to be declared in this {\em lsdomaint}-file.
\item A defined {\bf KEY} has to be an integer.
\item The abbreviations are implemented as sets ; identifiers, separated by
commas,between square brackets.
\end{itemize}
\section {Output}
The output created by the auxiliary-domain compiler,consists of a definition
file ({\em LSAUXDOM.ENV}) and an implementation-file ({\em LSAUXDOM.PAS}),
both Pascal versions.To create these files one has to type the following
command :
\begin{verbatim}
               lbuild  < language >:LSAUXDOM.OPT

\end{verbatim}

In these files, variable names, used in other modules,are concatenated
with the module-name.
\subsection{Definition-file}
The definition file inherits some other files:
\begin {itemize}
\item The file STRING.ENV enables string-manipulations.
\item The file FILES.ENV enables file-manipulations.
\item The file language:LSDOMAINT.ENV declares the types,used in the standard
records;'language' must be stated in the auxiliary domain.
\end{itemize}
The predefined keys are converted to Pascal constants.The names of the
RecordCompare functions are a concatenation of de string 'LSAUXDOM\_Compare'
and the Standard Record identifier.Thus, comparing a Standard Record with
an other record is possible by calling the functionname derived from the
Standard Record.
An external procedure WriteAbbr enables the program to handle the abbreviations.
\subsection{Implementation-file}
The implementation file inherits the same files for the same reasons
as the definition file.The implementation file of course inherits also its
environment-file.

The implementation-file consists of a few procedures.

The RecordCompare functions compares an input record with a Standard Record.

WriteAbbr writes the complete string to a predefined outputfile (of1),
when dealing with its abbreviation.Note that a Case-statement is not allowed
here; the selector has to be an integer.

\section{A small example}
The following example may illustrate how to use the auxiliary-domain and its
compiler.It shows the input(auxiliary-domain) and output(implementation and
definition file).
\subsection{Input}
The input is written in the file named: DUTCH:lsauxdomain.auxdom.
\begin{verbatim}

LANGUAGE : DUTCH

KEYS
      <  DatKey = < KEY      = 1755126
                    WORD     = dat
                    CATEGORY = RelPro
                  >
         AbstrKey = < KEY      = 80675
                      WORD     = ABSTRACT
                      CATEGORY = Noun
                   >
         DotKey = < KEY      = 120000
                    WORD     = .
                    CATEGORY = punct
                   >
      >

STANDARD RECORDS
      <
        ADJPPROP1record: ADJPPROPrecord
                   { req             = omegapol
                     class           = omegaTimeAdvClass
                     deixis          = omegadeixis
                     aspect          = omegaAspect
                     retro           = false
                     aktionsart      = stative
                     superdeixis     = omegadeixis
                     actsubcefs      = [otheradj]
                     thetaadj        = omegathetaadj
                     adjpatternefs   = []
                     PROsubject      = false
                   }
        ASP1record : ASPrecord
                   { req         = omegapol
                     thanascompl = NPcompl
                   }
      >

SETS
      <
        NAW = [NAAM, ADRES ,WOONPLAATS]
        SENT= [VERB, NOUN]
        ONE = [onlyone]
      >

\end{verbatim}
\subsection{Output}
The output is created by the following command:
\begin{verbatim}
               lbuild : DUTCH:lsauxdom.opt

\end{verbatim}
\subsubsection{definition}
\begin{verbatim}

[ENVIRONMENT('DUTCH:lsauxdom'),
     INHERIT('GENERAL:string',
             'GENERAL:files',
             'DUTCH:lsdomaint')]

MODULE LSAUXDOM;

CONST
  LSAUXDOM_DATKEY   = 1755126;
  LSAUXDOM_ABSTRKEY = 80675;
  LSAUXDOM_DOTKEY   = 120000;
\end{verbatim}
\pagebreak
\begin{verbatim}

[EXTERNAL] FUNCTION LSAUXDOM_CompareADJPPROP1RECORD
                         (rec:LSDOMAINT_ADJPPROPRECORD):BOOLEAN;
           EXTERN;

[EXTERNAL] FUNCTION LSAUXDOM_CompareASP1RECORD
                         (rec:LSDOMAINT_ASPRECORD):BOOLEAN;
           EXTERN;

[EXTERNAL] PROCEDURE LSAUXDOM_WriteAbbr(VAR of1:FILES_text;
                               abbrev:STRING_string);
           EXTERN;


END. {LSAUXDOM}

\end{verbatim}
\subsubsection{implementation}
\begin{verbatim}

[INHERIT ('DUTCH:lsauxdom',
          'GENERAL:string',
          'GENERAL:files',
          'DUTCH:lsdomaint')]

MODULE LSAUXDOM(output);


[GLOBAL] FUNCTION LSAUXDOM_CompareADJPPROP1RECORD
                         (rec:LSDOMAINT_ADJPPROPRECORD):BOOLEAN;
VAR Bool : BOOLEAN;
BEGIN
  Bool := TRUE;
  WITH rec DO
  BEGIN
    IF Bool THEN
      Bool:= (REQ = OMEGAPOL )
    ELSE IF Bool THEN
      Bool:= (CLASS = OMEGATIMEADVCLASS )
    ELSE IF Bool THEN
      Bool:= (DEIXIS = OMEGADEIXIS )
    ELSE IF Bool THEN
      Bool:= (ASPECT = OMEGAASPECT )
    ELSE IF Bool THEN
      Bool:= (RETRO = FALSE )
    ELSE IF Bool THEN
      Bool:= (AKTIONSART = STATIVE )
    ELSE IF Bool THEN
      Bool:= (SUPERDEIXIS = OMEGADEIXIS )
    ELSE IF Bool THEN
      Bool:= (ACTSUBCEFS = [OTHERADJ] )
    ELSE IF Bool THEN
      Bool:= (THETAADJ = OMEGATHETAADJ )
    ELSE IF Bool THEN
      Bool:= (ADJPATTERNEFS = [] )
    ELSE IF Bool THEN
      Bool:= (PROSUBJECT = FALSE );
  END;{with}
  LSAUXDOM_CompareADJPPROP1RECORD:=Bool;
END;{function}



[GLOBAL] FUNCTION LSAUXDOM_CompareASP1RECORD
                         (rec:LSDOMAINT_ASPRECORD):BOOLEAN;
VAR Bool : BOOLEAN;
BEGIN
  Bool := TRUE;
  WITH rec DO
  BEGIN
    IF Bool THEN
      Bool:= (REQ = OMEGAPOL )
    ELSE IF Bool THEN
      Bool:= (THANASCOMPL = NPCOMPL );
  END;{with}
  LSAUXDOM_CompareASP1RECORD:=Bool;
END;{function}
\end{verbatim}
\pagebreak
\begin{verbatim}
[GLOBAL] PROCEDURE LSAUXDOM_WriteAbbr(VAR of1:FILES_text;
                             abbrev:STRING_string);

BEGIN
  Files_Open(of1,'un_specified',Files_MaxName,5);
  IF abbrev = 'NAW' THEN
     Files_WriteStr(of1,'[NAAM,ADRES,WOONPLAATS]',Files_MaxIO,Files_MaxIO,True)
  ELSE IF abbrev = 'SENT' THEN
     Files_WriteStr(of1,'[VERB,NOUN]',Files_MaxIO,Files_MaxIO,True)
  ELSE IF abbrev = 'ONE' THEN
     Files_WriteStr(of1,'[ONLYONE]',Files_MaxIO,Files_MaxIO,True);
  Files_Close(of1);
END;

END. {LSAUXDOM}

\end{verbatim}
\end{document}


ROSETTA.sty
\typeout{Document Style 'Rosetta'. Version 0.3 - released  feb-1987}
% feb-1987:  Date of copyright notice changed
\def\@ptsize{1}
\@namedef{ds@10pt}{\def\@ptsize{0}}
\@namedef{ds@12pt}{\def\@ptsize{2}}
\@twosidetrue
\@mparswitchtrue
\def\ds@draft{\overfullrule 5pt}
\@options
\input art1\@ptsize.sty\relax


\def\labelenumi{\arabic{enumi}.}
\def\theenumi{\arabic{enumi}}
\def\labelenumii{(\alph{enumii})}
\def\theenumii{\alph{enumii}}
\def\p@enumii{\theenumi}
\def\labelenumiii{\roman{enumiii}.}
\def\theenumiii{\roman{enumiii}}
\def\p@enumiii{\theenumi(\theenumii)}
\def\labelenumiv{\Alph{enumiv}.}
\def\theenumiv{\Alph{enumiv}}
\def\p@enumiv{\p@enumiii\theenumiii}
\def\labelitemi{$\bullet$}
\def\labelitemii{\bf --}
\def\labelitemiii{$\ast$}
\def\labelitemiv{$\cdot$}
\def\verse{
   \let\\=\@centercr
   \list{}{\itemsep\z@ \itemindent -1.5em\listparindent \itemindent
      \rightmargin\leftmargin\advance\leftmargin 1.5em}
   \item[]}
\let\endverse\endlist
\def\quotation{
   \list{}{\listparindent 1.5em
      \itemindent\listparindent
      \rightmargin\leftmargin \parsep 0pt plus 1pt}\item[]}
\let\endquotation=\endlist
\def\quote{
   \list{}{\rightmargin\leftmargin}\item[]}
\let\endquote=\endlist
\def\descriptionlabel#1{\hspace\labelsep \bf #1}
\def\description{
   \list{}{\labelwidth\z@ \itemindent-\leftmargin
      \let\makelabel\descriptionlabel}}
\let\enddescription\endlist


\def\@begintheorem#1#2{\it \trivlist \item[\hskip \labelsep{\bf #1\ #2}]}
\def\@endtheorem{\endtrivlist}
\def\theequation{\arabic{equation}}
\def\titlepage{
   \@restonecolfalse
   \if@twocolumn\@restonecoltrue\onecolumn
   \else \newpage
   \fi
   \thispagestyle{empty}\c@page\z@}
\def\endtitlepage{\if@restonecol\twocolumn \else \newpage \fi}
\arraycolsep 5pt \tabcolsep 6pt \arrayrulewidth .4pt \doublerulesep 2pt
\tabbingsep \labelsep
\skip\@mpfootins = \skip\footins
\fboxsep = 3pt \fboxrule = .4pt


\newcounter{part}
\newcounter {section}
\newcounter {subsection}[section]
\newcounter {subsubsection}[subsection]
\newcounter {paragraph}[subsubsection]
\newcounter {subparagraph}[paragraph]
\def\thepart{\Roman{part}} \def\thesection {\arabic{section}}
\def\thesubsection {\thesection.\arabic{subsection}}
\def\thesubsubsection {\thesubsection .\arabic{subsubsection}}
\def\theparagraph {\thesubsubsection.\arabic{paragraph}}
\def\thesubparagraph {\theparagraph.\arabic{subparagraph}}


\def\@pnumwidth{1.55em}
\def\@tocrmarg {2.55em}
\def\@dotsep{4.5}
\setcounter{tocdepth}{3}
\def\tableofcontents{\section*{Contents\markboth{}{}}
\@starttoc{toc}}
\def\l@part#1#2{
   \addpenalty{-\@highpenalty}
   \addvspace{2.25em plus 1pt}
   \begingroup
      \@tempdima 3em \parindent \z@ \rightskip \@pnumwidth \parfillskip
      -\@pnumwidth {\large \bf \leavevmode #1\hfil \hbox to\@pnumwidth{\hss #2}}
      \par \nobreak
   \endgroup}
\def\l@section#1#2{
   \addpenalty{-\@highpenalty}
   \addvspace{1.0em plus 1pt}
   \@tempdima 1.5em
   \begingroup
      \parindent \z@ \rightskip \@pnumwidth
      \parfillskip -\@pnumwidth
      \bf \leavevmode #1\hfil \hbox to\@pnumwidth{\hss #2}
      \par
   \endgroup}
\def\l@subsection{\@dottedtocline{2}{1.5em}{2.3em}}
\def\l@subsubsection{\@dottedtocline{3}{3.8em}{3.2em}}
\def\l@paragraph{\@dottedtocline{4}{7.0em}{4.1em}}
\def\l@subparagraph{\@dottedtocline{5}{10em}{5em}}
\def\listoffigures{
   \section*{List of Figures\markboth{}{}}
   \@starttoc{lof}}
   \def\l@figure{\@dottedtocline{1}{1.5em}{2.3em}}
   \def\listoftables{\section*{List of Tables\markboth{}{}}
   \@starttoc{lot}}
\let\l@table\l@figure


\def\thebibliography#1{
   \addcontentsline{toc}
   {section}{References}\section*{References\markboth{}{}}
   \list{[\arabic{enumi}]}
        {\settowidth\labelwidth{[#1]}\leftmargin\labelwidth
         \advance\leftmargin\labelsep\usecounter{enumi}}}
\let\endthebibliography=\endlist


\newif\if@restonecol
\def\theindex{
   \@restonecoltrue\if@twocolumn\@restonecolfalse\fi
   \columnseprule \z@
   \columnsep 35pt\twocolumn[\section*{Index}]
   \markboth{}{}
   \thispagestyle{plain}\parindent\z@
   \parskip\z@ plus .3pt\relax
   \let\item\@idxitem}
\def\@idxitem{\par\hangindent 40pt}
\def\subitem{\par\hangindent 40pt \hspace*{20pt}}
\def\subsubitem{\par\hangindent 40pt \hspace*{30pt}}
\def\endtheindex{\if@restonecol\onecolumn\else\clearpage\fi}
\def\indexspace{\par \vskip 10pt plus 5pt minus 3pt\relax}


\def\footnoterule{
   \kern-1\p@
   \hrule width .4\columnwidth
   \kern .6\p@}
\long\def\@makefntext#1{
   \@setpar{\@@par\@tempdima \hsize
   \advance\@tempdima-10pt\parshape \@ne 10pt \@tempdima}\par
   \parindent 1em\noindent \hbox to \z@{\hss$^{\@thefnmark}$}#1}


\setcounter{topnumber}{2}
\def\topfraction{.7}
\setcounter{bottomnumber}{1}
\def\bottomfraction{.3}
\setcounter{totalnumber}{3}
\def\textfraction{.2}
\def\floatpagefraction{.5}
\setcounter{dbltopnumber}{2}
\def\dbltopfraction{.7}
\def\dblfloatpagefraction{.5}
\long\def\@makecaption#1#2{
   \vskip 10pt
   \setbox\@tempboxa\hbox{#1: #2}
   \ifdim \wd\@tempboxa >\hsize \unhbox\@tempboxa\par
   \else \hbox to\hsize{\hfil\box\@tempboxa\hfil}
   \fi}
\newcounter{figure}
\def\thefigure{\@arabic\c@figure}
\def\fps@figure{tbp}
\def\ftype@figure{1}
\def\ext@figure{lof}
\def\fnum@figure{Figure \thefigure}
\def\figure{\@float{figure}}
\let\endfigure\end@float
\@namedef{figure*}{\@dblfloat{figure}}
\@namedef{endfigure*}{\end@dblfloat}
\newcounter{table}
\def\thetable{\@arabic\c@table}
\def\fps@table{tbp}
\def\ftype@table{2}
\def\ext@table{lot}
\def\fnum@table{Table \thetable}
\def\table{\@float{table}}
\let\endtable\end@float
\@namedef{table*}{\@dblfloat{table}}
\@namedef{endtable*}{\end@dblfloat}


\def\maketitle{
   \par
   \begingroup
      \def\thefootnote{\fnsymbol{footnote}}
      \def\@makefnmark{\hbox to 0pt{$^{\@thefnmark}$\hss}}
      \if@twocolumn \twocolumn[\@maketitle]
      \else \newpage \global\@topnum\z@ \@maketitle
      \fi
      \thispagestyle{plain}
      \@thanks
   \endgroup
   \setcounter{footnote}{0}
   \let\maketitle\relax
   \let\@maketitle\relax
   \gdef\@thanks{}
   \gdef\@author{}
   \gdef\@title{}
   \let\thanks\relax}
\def\@maketitle{
   \newpage
   \null
   \vskip 2em
   \begin{center}{\LARGE \@title \par}
      \vskip 1.5em
      {\large \lineskip .5em \begin{tabular}[t]{c}\@author \end{tabular}\par}
      \vskip 1em {\large \@date}
   \end{center}
   \par
   \vskip 1.5em}
\def\abstract{
   \if@twocolumn \section*{Abstract}
   \else
      \small
      \begin{center} {\bf Abstract\vspace{-.5em}\vspace{0pt}} \end{center}
      \quotation
   \fi}
\def\endabstract{\if@twocolumn\else\endquotation\fi}


\mark{{}{}}
\if@twoside
   \def\ps@headings{
      \def\@oddfoot{Rosetta Doc. \@RosDocNr\hfil \@RosDate}
      \def\@evenfoot{Rosetta Doc. \@RosDocNr\hfil \@RosDate}
      \def\@evenhead{\rm\thepage\hfil \sl \rightmark}
      \def\@oddhead{\hbox{}\sl \leftmark \hfil\rm\thepage}
      \def\sectionmark##1{\markboth {}{}}
      \def\subsectionmark##1{}}
\else
   \def\ps@headings{
      \def\@oddfoot{Rosetta Doc. \@RosDocNr\hfil \@RosDate}
      \def\@evenfoot{Rosetta Doc. \@RosDocNr\hfil \@RosDate}
      \def\@oddhead{\hbox{}\sl \rightmark \hfil \rm\thepage}
      \def\sectionmark##1{\markboth {}{}}
      \def\subsectionmark##1{}}
\fi
\def\ps@myheadings{
   \def\@oddhead{\hbox{}\sl\@rhead \hfil \rm\thepage}
   \def\@oddfoot{}
   \def\@evenhead{\rm \thepage\hfil\sl\@lhead\hbox{}}
   \def\@evenfoot{}
   \def\sectionmark##1{}
   \def\subsectionmark##1{}}


\def\today{
   \ifcase\month\or January\or February\or March\or April\or May\or June\or
      July\or August\or September\or October\or November\or December
   \fi
   \space\number\day, \number\year}


\ps@plain \pagenumbering{arabic} \onecolumn \if@twoside\else\raggedbottom\fi




% the Rosetta title page
\newcommand{\MakeRosTitle}{
   \begin{titlepage}
      \begin{large}
     \begin{figure}[t]
        \begin{picture}(405,100)(0,0)
           \put(0,100){\line(1,0){404}}
           \put(0,75){Project {\bf Rosetta}}
           \put(93.5,75){:}
           \put(108,75){Machine Translation}
           \put(0,50){Topic}
           \put(93.5,50){:}
           \put(108,50){\@RosTopic}
           \put(0,30){\line(1,0){404}}
        \end{picture}
     \end{figure}
     \bigskip
     \bigskip
     \begin{list}{-}{\setlength{\leftmargin}{3.0cm}
             \setlength{\labelwidth}{2.7cm}
             \setlength{\topsep}{2cm}}
        \item [{\rm Title \hfill :}] {{\bf \@RosTitle}}
        \item [{\rm Author \hfill :}] {\@RosAuthor}
        \bigskip
        \bigskip
        \bigskip
        \item [{\rm Doc.Nr. \hfill :}] {\@RosDocNr}
        \item [{\rm Date \hfill :}] {\@RosDate}
        \item [{\rm Status \hfill :}] {\@RosStatus}
        \item [{\rm Supersedes \hfill :}] {\@RosSupersedes}
        \item [{\rm Distribution \hfill :}] {\@RosDistribution}
        \item [{\rm Clearance \hfill :}] {\@RosClearance}
        \item [{\rm Keywords \hfill :}] {\@RosKeywords}
     \end{list}
      \end{large}
      \title{\@RosTitle}
      \begin{figure}[b]
     \begin{picture}(404,64)(0,0)
        \put(0,64){\line(1,0){404}}
        \put(0,-4){\line(1,0){404}}
        \put(0,59){\line(1,0){42}}
        \begin{small}
        \put(3,48){\sf PHILIPS}
        \end{small}
        \put(0,23){\line(0,1){36}}
        \put(42,23){\line(0,1){36}}
        \put(21,23){\oval(42,42)[bl]}
        \put(21,23){\oval(42,42)[br]}
        \put(21,23){\circle{40}}
        \put(4,33){\line(1,0){10}}
        \put(9,28){\line(0,1){10}}
        \put(9,36){\line(1,0){6}}
        \put(12,33){\line(0,1){6}}
        \put(29,13){\line(1,0){10}}
        \put(34,8){\line(0,1){10}}
        \put(28,10){\line(1,0){6}}
        \put(31,7){\line(0,1){6}}

        \put(1,21){\line(1,0){0.5}}
        \put(1.5,21.3){\line(1,0){0.5}}
        \put(2,21.6){\line(1,0){0.5}}
        \put(2.5,21.9){\line(1,0){0.5}}
        \put(3,22.1){\line(1,0){0.5}}
        \put(3.5,22.3){\line(1,0){0.5}}
        \put(4,22.5){\line(1,0){0.5}}
        \put(4.5,22.7){\line(1,0){0.5}}
        \put(5,22.8){\line(1,0){0.5}}
        \put(5.5,22.9){\line(1,0){0.5}}
        \put(6,23){\line(1,0){0.5}}
        \put(6.5,22.9){\line(1,0){0.5}}
        \put(7,22.8){\line(1,0){0.5}}
        \put(7.5,22.7){\line(1,0){0.5}}
        \put(8,22.5){\line(1,0){0.5}}
        \put(8.5,22.3){\line(1,0){0.5}}
        \put(9,22.1){\line(1,0){0.5}}
        \put(9.5,21.9){\line(1,0){0.5}}
        \put(10,21.6){\line(1,0){0.5}}
        \put(10.5,21.3){\line(1,0){0.5}}

        \put(1,23){\line(1,0){0.5}}
        \put(1.5,23.3){\line(1,0){0.5}}
        \put(2,23.6){\line(1,0){0.5}}
        \put(2.5,23.9){\line(1,0){0.5}}
        \put(3,24.1){\line(1,0){0.5}}
        \put(3.5,24.3){\line(1,0){0.5}}
        \put(4,24.5){\line(1,0){0.5}}
        \put(4.5,24.7){\line(1,0){0.5}}
        \put(5,24.8){\line(1,0){0.5}}
        \put(5.5,24.9){\line(1,0){0.5}}
        \put(6,25){\line(1,0){0.5}}
        \put(6.5,24.9){\line(1,0){0.5}}
        \put(7,24.8){\line(1,0){0.5}}
        \put(7.5,24.7){\line(1,0){0.5}}
        \put(8,24.5){\line(1,0){0.5}}
        \put(8.5,24.3){\line(1,0){0.5}}
        \put(9,24.1){\line(1,0){0.5}}
        \put(9.5,23.9){\line(1,0){0.5}}
        \put(10,23.6){\line(1,0){0.5}}
        \put(10.5,23.3){\line(1,0){0.5}}

        \put(1,25){\line(1,0){0.5}}
        \put(1.5,25.3){\line(1,0){0.5}}
        \put(2,25.6){\line(1,0){0.5}}
        \put(2.5,25.9){\line(1,0){0.5}}
        \put(3,26.1){\line(1,0){0.5}}
        \put(3.5,26.3){\line(1,0){0.5}}
        \put(4,26.5){\line(1,0){0.5}}
        \put(4.5,26.7){\line(1,0){0.5}}
        \put(5,26.8){\line(1,0){0.5}}
        \put(5.5,26.9){\line(1,0){0.5}}
        \put(6,27){\line(1,0){0.5}}
        \put(6.5,26.9){\line(1,0){0.5}}
        \put(7,26.8){\line(1,0){0.5}}
        \put(7.5,26.7){\line(1,0){0.5}}
        \put(8,26.5){\line(1,0){0.5}}
        \put(8.5,26.3){\line(1,0){0.5}}
        \put(9,26.1){\line(1,0){0.5}}
        \put(9.5,25.9){\line(1,0){0.5}}
        \put(10,25.6){\line(1,0){0.5}}
        \put(10.5,25.3){\line(1,0){0.5}}

        \put(11,21){\line(1,0){0.5}}
        \put(11.5,20.7){\line(1,0){0.5}}
        \put(12,20.4){\line(1,0){0.5}}
        \put(12.5,20.1){\line(1,0){0.5}}
        \put(13,19.9){\line(1,0){0.5}}
        \put(13.5,19.7){\line(1,0){0.5}}
        \put(14,19.5){\line(1,0){0.5}}
        \put(14.5,19.3){\line(1,0){0.5}}
        \put(15,19.2){\line(1,0){0.5}}
        \put(15.5,19.1){\line(1,0){0.5}}
        \put(16,19){\line(1,0){0.5}}
        \put(16.5,19.1){\line(1,0){0.5}}
        \put(17,19.2){\line(1,0){0.5}}
        \put(17.5,19.3){\line(1,0){0.5}}
        \put(18,19.5){\line(1,0){0.5}}
        \put(18.5,19.7){\line(1,0){0.5}}
        \put(19,19.9){\line(1,0){0.5}}
        \put(19.5,20.1){\line(1,0){0.5}}
        \put(20,20.4){\line(1,0){0.5}}
        \put(20.5,20.7){\line(1,0){0.5}}

        \put(11,23){\line(1,0){0.5}}
        \put(11.5,22.7){\line(1,0){0.5}}
        \put(12,22.4){\line(1,0){0.5}}
        \put(12.5,22.1){\line(1,0){0.5}}
        \put(13,21.9){\line(1,0){0.5}}
        \put(13.5,21.7){\line(1,0){0.5}}
        \put(14,21.5){\line(1,0){0.5}}
        \put(14.5,21.3){\line(1,0){0.5}}
        \put(15,21.2){\line(1,0){0.5}}
        \put(15.5,21.1){\line(1,0){0.5}}
        \put(16,21){\line(1,0){0.5}}
        \put(16.5,21.1){\line(1,0){0.5}}
        \put(17,21.2){\line(1,0){0.5}}
        \put(17.5,21.3){\line(1,0){0.5}}
        \put(18,21.5){\line(1,0){0.5}}
        \put(18.5,21.7){\line(1,0){0.5}}
        \put(19,21.9){\line(1,0){0.5}}
        \put(19.5,22.1){\line(1,0){0.5}}
        \put(20,22.4){\line(1,0){0.5}}
        \put(20.5,22.7){\line(1,0){0.5}}

        \put(11,25){\line(1,0){0.5}}
        \put(11.5,24.7){\line(1,0){0.5}}
        \put(12,24.4){\line(1,0){0.5}}
        \put(12.5,24.1){\line(1,0){0.5}}
        \put(13,23.9){\line(1,0){0.5}}
        \put(13.5,23.7){\line(1,0){0.5}}
        \put(14,23.5){\line(1,0){0.5}}
        \put(14.5,23.3){\line(1,0){0.5}}
        \put(15,23.2){\line(1,0){0.5}}
        \put(15.5,23.1){\line(1,0){0.5}}
        \put(16,23){\line(1,0){0.5}}
        \put(16.5,23.1){\line(1,0){0.5}}
        \put(17,23.2){\line(1,0){0.5}}
        \put(17.5,23.3){\line(1,0){0.5}}
        \put(18,23.5){\line(1,0){0.5}}
        \put(18.5,23.7){\line(1,0){0.5}}
        \put(19,23.9){\line(1,0){0.5}}
        \put(19.5,24.1){\line(1,0){0.5}}
        \put(20,24.4){\line(1,0){0.5}}
        \put(20.5,24.7){\line(1,0){0.5}}

        \put(21,21){\line(1,0){0.5}}
        \put(21.5,21.3){\line(1,0){0.5}}
        \put(22,21.6){\line(1,0){0.5}}
        \put(22.5,21.9){\line(1,0){0.5}}
        \put(23,22.1){\line(1,0){0.5}}
        \put(23.5,22.3){\line(1,0){0.5}}
        \put(24,22.5){\line(1,0){0.5}}
        \put(24.5,22.7){\line(1,0){0.5}}
        \put(25,22.8){\line(1,0){0.5}}
        \put(25.5,23.9){\line(1,0){0.5}}
        \put(26,23){\line(1,0){0.5}}
        \put(26.5,22.9){\line(1,0){0.5}}
        \put(27,22.8){\line(1,0){0.5}}
        \put(27.5,22.7){\line(1,0){0.5}}
        \put(28,22.5){\line(1,0){0.5}}
        \put(28.5,22.3){\line(1,0){0.5}}
        \put(29,22.1){\line(1,0){0.5}}
        \put(29.5,21.9){\line(1,0){0.5}}
        \put(30,21.6){\line(1,0){0.5}}
        \put(30.5,21.3){\line(1,0){0.5}}

        \put(21,23){\line(1,0){0.5}}
        \put(21.5,23.3){\line(1,0){0.5}}
        \put(22,23.6){\line(1,0){0.5}}
        \put(22.5,23.9){\line(1,0){0.5}}
        \put(23,24.1){\line(1,0){0.5}}
        \put(23.5,24.3){\line(1,0){0.5}}
        \put(24,24.5){\line(1,0){0.5}}
        \put(24.5,24.7){\line(1,0){0.5}}
        \put(25,24.8){\line(1,0){0.5}}
        \put(25.5,24.9){\line(1,0){0.5}}
        \put(26,25){\line(1,0){0.5}}
        \put(26.5,24.9){\line(1,0){0.5}}
        \put(27,24.8){\line(1,0){0.5}}
        \put(27.5,24.7){\line(1,0){0.5}}
        \put(28,24.5){\line(1,0){0.5}}
        \put(28.5,24.3){\line(1,0){0.5}}
        \put(29,24.1){\line(1,0){0.5}}
        \put(29.5,23.9){\line(1,0){0.5}}
        \put(30,23.6){\line(1,0){0.5}}
        \put(30.5,23.3){\line(1,0){0.5}}

        \put(21,25){\line(1,0){0.5}}
        \put(21.5,25.3){\line(1,0){0.5}}
        \put(22,25.6){\line(1,0){0.5}}
        \put(22.5,25.9){\line(1,0){0.5}}
        \put(23,26.1){\line(1,0){0.5}}
        \put(23.5,26.3){\line(1,0){0.5}}
        \put(24,26.5){\line(1,0){0.5}}
        \put(24.5,26.7){\line(1,0){0.5}}
        \put(25,26.8){\line(1,0){0.5}}
        \put(25.5,26.9){\line(1,0){0.5}}
        \put(26,27){\line(1,0){0.5}}
        \put(26.5,26.9){\line(1,0){0.5}}
        \put(27,26.8){\line(1,0){0.5}}
        \put(27.5,26.7){\line(1,0){0.5}}
        \put(28,26.5){\line(1,0){0.5}}
        \put(28.5,26.3){\line(1,0){0.5}}
        \put(29,26.1){\line(1,0){0.5}}
        \put(29.5,25.9){\line(1,0){0.5}}
        \put(30,25.6){\line(1,0){0.5}}
        \put(30.5,25.3){\line(1,0){0.5}}

        \put(31,21){\line(1,0){0.5}}
        \put(31.5,20.7){\line(1,0){0.5}}
        \put(32,20.4){\line(1,0){0.5}}
        \put(32.5,20.1){\line(1,0){0.5}}
        \put(33,19.9){\line(1,0){0.5}}
        \put(33.5,19.7){\line(1,0){0.5}}
        \put(34,19.5){\line(1,0){0.5}}
        \put(34.5,19.3){\line(1,0){0.5}}
        \put(35,19.2){\line(1,0){0.5}}
        \put(35.5,19.1){\line(1,0){0.5}}
        \put(36,19){\line(1,0){0.5}}
        \put(36.5,19.1){\line(1,0){0.5}}
        \put(37,19.2){\line(1,0){0.5}}
        \put(37.5,19.3){\line(1,0){0.5}}
        \put(38,19.5){\line(1,0){0.5}}
        \put(38.5,19.7){\line(1,0){0.5}}
        \put(39,19.9){\line(1,0){0.5}}
        \put(39.5,20.1){\line(1,0){0.5}}
        \put(40,20.4){\line(1,0){0.5}}
        \put(40.5,20.7){\line(1,0){0.5}}

        \put(31,23){\line(1,0){0.5}}
        \put(31.5,22.7){\line(1,0){0.5}}
        \put(32,22.4){\line(1,0){0.5}}
        \put(32.5,22.1){\line(1,0){0.5}}
        \put(33,21.9){\line(1,0){0.5}}
        \put(33.5,21.7){\line(1,0){0.5}}
        \put(34,21.5){\line(1,0){0.5}}
        \put(34.5,21.3){\line(1,0){0.5}}
        \put(35,21.2){\line(1,0){0.5}}
        \put(35.5,21.1){\line(1,0){0.5}}
        \put(36,21){\line(1,0){0.5}}
        \put(36.5,21.1){\line(1,0){0.5}}
        \put(37,21.2){\line(1,0){0.5}}
        \put(37.5,21.3){\line(1,0){0.5}}
        \put(38,21.5){\line(1,0){0.5}}
        \put(38.5,21.7){\line(1,0){0.5}}
        \put(39,21.9){\line(1,0){0.5}}
        \put(39.5,22.1){\line(1,0){0.5}}
        \put(40,22.4){\line(1,0){0.5}}
        \put(40.5,22.7){\line(1,0){0.5}}

        \put(31,25){\line(1,0){0.5}}
        \put(31.5,24.7){\line(1,0){0.5}}
        \put(32,24.4){\line(1,0){0.5}}
        \put(32.5,24.1){\line(1,0){0.5}}
        \put(33,23.9){\line(1,0){0.5}}
        \put(33.5,23.7){\line(1,0){0.5}}
        \put(34,23.5){\line(1,0){0.5}}
        \put(34.5,23.3){\line(1,0){0.5}}
        \put(35,23.2){\line(1,0){0.5}}
        \put(35.5,23.1){\line(1,0){0.5}}
        \put(36,23){\line(1,0){0.5}}
        \put(36.5,23.1){\line(1,0){0.5}}
        \put(37,23.2){\line(1,0){0.5}}
        \put(37.5,23.3){\line(1,0){0.5}}
        \put(38,23.5){\line(1,0){0.5}}
        \put(38.5,23.7){\line(1,0){0.5}}
        \put(39,23.9){\line(1,0){0.5}}
        \put(39.5,24.1){\line(1,0){0.5}}
        \put(40,24.4){\line(1,0){0.5}}
        \put(40.5,24.7){\line(1,0){0.5}}
        \begin{large}
           \put(60,45){Philips Research Laboratories}
           \put(60,30){\copyright\ 1987 Nederlandse Philips Bedrijven B.V.}
        \end{large}
     \end{picture}
      \end{figure}
      \newpage
      \pagenumbering{roman}
      \tableofcontents
      \newpage
      \pagenumbering{arabic}
   \end{titlepage}
}
\title{}
\topmargin 0pt
\oddsidemargin 36pt
\evensidemargin 36pt
\textheight 600pt
\textwidth 405pt
\pagestyle{headings}
\newcommand{\@RosTopic}{General}
\newcommand{\@RosTitle}{-}
\newcommand{\@RosAuthor}{-}
\newcommand{\@RosDocNr}{}
\newcommand{\@RosDate}{\today}
\newcommand{\@RosStatus}{informal}
\newcommand{\@RosSupersedes}{-}
\newcommand{\@RosDistribution}{Project}
\newcommand{\@RosClearance}{Project}
\newcommand{\@RosKeywords}{}
\newcommand{\RosTopic}[1]{\renewcommand{\@RosTopic}{#1}}
\newcommand{\RosTitle}[1]{\renewcommand{\@RosTitle}{#1}}
\newcommand{\RosAuthor}[1]{\renewcommand{\@RosAuthor}{#1}}
\newcommand{\RosDocNr}[1]{\renewcommand{\@RosDocNr}{#1}}
\newcommand{\RosDate}[1]{\renewcommand{\@RosDate}{#1}}
\newcommand{\RosStatus}[1]{\renewcommand{\@RosStatus}{#1}}
\newcommand{\RosSupersedes}[1]{\renewcommand{\@RosSupersedes}{#1}}
\newcommand{\RosDistribution}[1]{\renewcommand{\@RosDistribution}{#1}}
\newcommand{\RosClearance}[1]{\renewcommand{\@RosClearance}{#1}}
\newcommand{\RosKeywords}[1]{\renewcommand{\@RosKeywords}{#1}}

