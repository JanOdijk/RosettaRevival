\documentstyle{Rosetta}
\begin{document}
   \RosTopic{General}
   \RosTitle{Notulen Groepsvergadering 8-1-1990}
   \RosAuthor{Harm Smit}
   \RosDocNr{426}
   \RosDate{\today}
   \RosStatus{approved}
   \RosSupersedes{-}
   \RosDistribution{Project}
   \RosClearance{Project}
   \RosKeywords{minutes}
   \MakeRosTitle
\begin{itemize}
  \item {\bf aanwezig}: Andr\'{e} Schenk, Jan Landsbergen, Lisette Appelo,
                     Franciska de Jong, Ren\'{e} Leermakers, Joep Rous,
                     Jan Odijk, Harm Smit.
  \item {\bf afwezig}: Petra de Wit, Elly van Munster, Elena Pinillos.
  \item {\bf Agenda}:
    \begin{enumerate}
       \item Opening en notulen
       \item Diverse mededelingen en rondvraag
       \item Esprit projecten
       \item Rosetta 3D
    \end{enumerate}
  \item Aansluitend aan de vergadering is document 416 besproken en daarna
        vond de afscheidbijeenkomst voor Margreet Sanders plaats.

\section {Opening en notulen}
De notulen van de vorige vergadering werden met enkele wijzigingen aangenomen.
\section {Diverse mededelingen en rondvraag}
\begin{enumerate}
   \item Jan L. heet Petra de Wit (helaas ziek vandaag) welkom.
   \item Er komen twee nieuwe stagiairs; \'{e}\'{e}n ervan (Josien Willems) komt
         per 1 februari, de andere, een tweede klasser van de HTS zal volgende
         week al arriveren en blijft twee maanden.
         Beiden zullen voorlopig ergens ondergebracht worden (wellicht bij Joep
         en Lisette) totdat we meer ruimte krijgen; dit zal waarschijnlijk in
         de loop van februari zo zijn wanneer de oude kamer van Wijnand weer
         beschikbaar komt.
   \item Er is een nieuwe personeelschef: Andrien van Abkoude, zij zal vrijdag
         a.s. bij ons komen kennismaken.
   \item CSO-cursus voor gasten: Waumans neigt ertoe het goed te vinden dat de 
         drie betreffende gasten de CSO-cursus mogen volgen. Dit moet nog wel
         met personeelszaken overlegd worden.
   \item Jan L. en Harm zijn bij IBM in Amsterdam op bezoek geweest en hebben 
         daar het Critique-systeem (een syntax-checker) bekeken. Critique
         maakt onderdeel uit van Procesmaster dat op mainframes draait.
   \item Franciska doet verslag van het Montague symposium.
   \item Joep meldt dat Fred een lijst van documenten heeft die nooit verschenen
         zijn of nog niet electronisch naar Frank Stoots verstuurd zijn. De 
         lijst ligt bij Ren\'{e} ter inzage en iedereen moet even controleren of
         er documenten van hem of haar bij zijn.
\end{enumerate}

\section {Esprit projecten}

Momenteel doet Rosetta mee aan twee `proposals' voor de Esprit-2 ronde. 

Het eerste betreft een project om standaards voor woordenboeken te maken; we 
doen hierin mee met 6 manmaanden over 2 jaar verspreid.

Het tweede betreft het `Pragmatics-based Language Understanding System' en hierin
doen we mee met 5 manjaren (1 volledige kracht).

Voor beide projecten geldt dat er alleen `echte' Philips medewerkers aan mee
mogen doen; gasten komen hiervoor dus niet in aanmerking.

\section {Rosetta 3D}

Joep heeft de verschillende taken voor de verschillende projectleden in overleg 
met de betrokkenen op elkaar afgestemd en in een schema bijeengebracht. Dit 
schema is uitgedeeld en besproken.

Lisette merkt nog op dat files niet onnodig lokaal gehouden mogen worden. Zorg
ervoor dat ze zo snel mogelijk terug zijn in het archief zodat anderen ze kunnen
gebruiken.

\end{document}
