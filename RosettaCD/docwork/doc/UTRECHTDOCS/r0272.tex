
\documentstyle{Rosetta}
\begin{document}
   \RosTopic{General}
   \RosTitle{Notulen Linguistenvergadering 31-05-88}
   \RosAuthor{Margreet Sanders}
   \RosDocNr{0272}
   \RosDate{June 1, 1988}
   \RosStatus{informal}
   \RosSupersedes{-}
   \RosDistribution{Linguist, Joep Rous}
   \RosClearance{Project}
   \RosKeywords{minutes, NEG, POS, (Var)Preps, SUBST, Recipro}
   \MakeRosTitle
%
%
\begin{description}
\item[Aanwezig:] Lisette Appelo, Franciska de Jong, 
                 Jan Odijk, Joep Rous, Margreet Sanders (not),
                 Andr\'{e} Schenk, Harm Smit
\item[Afwezig:] Elly van Munster
\item[Agenda:]\mbox{}
  \begin{enumerate}
  \item NEG en POS
  \item (VAR)PREPPs
  \item SUBST
  \item RECIPRO
  \end{enumerate}
\end{description}

\section{NEG en POS}
Naar aanleiding van een eerdere discussie op de linguistenvergadering van 3
mei (waarvan geen offici\"{e}le notulen zijn gemaakt) wordt alsnog officieel 
besloten om 
syncategorematisch een NEGVAR (`nepvar') te introduceren als plaatsmarkeerder 
alvorens NEG te introduceren in de substitutieregels. 
Het probleem is nl.\ dat als er geen NEGVAR bestaat, in analyse NEG op 
verschillende momenten gedesubstitueerd kan worden, omdat de substorder 
condition (die zegt dat er in analyse links van het gedesubstitueerde element 
nog geen VARs mogen staan) geen rekening hoeft te houden met NEG, bv.\\
\begin{verbatim}
Iemand koopt niet een boek  (Van: Iemand koopt geen boek)

Substorder: 1) Iemand   2) niet     3) een boek  (correct), maar ook:
            1) niet     2) iemand   3) een boek

\end{verbatim}
De laatste volgorde is fout, want `niet' heeft nu scope over `iemand', en de 
vertaling zal dus worden {\em Niemand koopt een boek\/}.

De NEGVAR-introductie kan het best gebeuren vlak voor de substitutieregels. 
Omdat er geen aanwijzingen zijn waar de VAR moet komen (een substorder condition 
is alleen zinvol als sommige VARs al gedesubstitueerd zijn), moet 
deze TC een NEGVAR op alle 
mogelijke plaatsen cre\"{e}ren. Er wordt afgezien van een alternatieve oplossing
waarbij de substitutieregel eenplaatsig is, en de NEGVAR vervangen wordt door 
een syncategorematische NEG. Ook een alternatief waarbij NEGVAR een 
basisexpressie wordt lijkt minder aantrekkelijk.

Voor POS geldt een vergelijkbaar probleem. Op het eerste gezicht lijkt daar de
variant met de eenplaatsige regel aantrekkelijker (er is dan dus geen 
basisexpressie meer in de vertaling!). De vertaling naar het Spaans 
wordt nog een probleem.

Na de substitutieregels moet een filter komen dat alle zinnen waarin nog 
NEGVARs staan uitfiltert.

\section{(VAR)PREPPs}
In het Nederlands bestaat de conventie dat PREPPs waarin nog een VAR zit 
aangeduid worden met VARPREPP. In het Engels leverde dit problemen op, omdat 
niet altijd op het juiste moment direct zichtbaar is of zo'n VAR bestaat. 
Daarom heeft het Engels ook PREPPs waarin een VAR kan voorkomen. Dat heeft
voornamelijk gevolgen voor de functie `Substorder-condition'. 
Voorzetsel-voorwerpen worden overigens ook in het Engels altijd gevormd door 
een VARPREPP.

\section{SUBST}
In de M-regelnotatie is een wijziging aangebracht: in substitutie-regels moet 
v\'{o}\'{o}r het model van de substituent de string {\bf SUBST:} worden 
toegevoegd, bv.\\
\ \ \ SUBST:\\
\ \ \ m2: NP\{NPrec1\}[mu1].\\
 In de regels die in het archief stonden is dit al door de informatici 
zelf veranderd. Locale versies moeten hier nog op gechecked worden.

Verder heeft Ren\'{e} iets gemaakt waardoor bij het builden {\em warnings\/}
verschijnen als in het hoofdmodel dezelfde variabele wordt gebruikt als in een 
subregel. Deze warnings doen verder niets (de regel kan gewoon ge\"{i}ntegreerd 
worden), maar vestigen alleen de aandacht op een mogelijke vergissing.

\section{Recipro}
De uitspelling van argument reflexieven gebeurt in een transformatie op basis 
van identiteit van twee variabelen:\\
 x$_{1}$ ziet x$_{1}$ $\rightarrow$ x$_{1}$ ziet ..zelf (waarbij de juiste vorm 
van het reflexief bepaald wordt door het antecedent).

Reciproken (altijd argumenten) kunnen niet op dezelfde manier door een 
transformatie worden vertaald, want dan zou je {\em Zij zagen elkaar\/} ook in 
{\em Zij zagen zichzelf\/} vertalen; beide hebben immers de representatie 
x$_{1}$ ziet x$_{1}$. Daarom is reciproken-uitspelling een regelklasse, geen
transformatieklasse.

Dit lost nog niet alle problemen op: als er een zin is waarin drie identieke 
variabelen voorkomen, waarvan er een als reflexief en een als reciprook 
uitgespeld moet worden, kan momenteel nergens worden vastgelegd welke variabele
nu wat was. Bv.\ {\em Zij stelden zichzelf aan elkaar voor\/} wordt via 
{\em x$_{1}$ stelt x$_{1}$ aan x$_{1}$ voor\/}
ook vertaald in {\em Zij stelden elkaar aan zichzelf voor\/}, hetgeen toch echt
iets anders betekent. 

De oplossing bestaat uit twee delen: ten eerste moet de variabele voor het 
reciproke voornaamwoord een unieke index krijgen, in plaats van dezelfde index
als zijn antecedent. Deze index moet als parameter aan de Reciprocal regel
worden meegegeven (in een substitutieregel gaat dat vanzelf, via LEVEL). 
Natuurlijk moet nu op een andere manier worden vastgelegd 
wat het antecedent is van het reciprook; dit kan door de index van het 
antecedent als tweede parameter mee te geven aan de Reciprocal regel. Het 
antecedent kan zowel het object zijn als het subject ({\em Zij stelden hen aan 
elkaar voor\/} kan betekenen: a stelde b aan c voor en c aan b, en: a stelde b 
aan c voor en c stelde b aan a voor).
Verder moet het reciprook dezelfde eigenschappen als zijn antecedent hebben; 
dit kan het makkelijkst als de Reciprocal-substitutieregel als substituent een 
NP neemt, waar in het Nederlands de 
basisexpressie RECIPRO(AUX\_elkaar) onder hangt. 

Een nadeel van de gegeven oplossing is dat er voor talen waarin de vorm van het 
reciprook afhangt van eigenschappen van het antecedent (bv.\ een speciale 
uitgang voor mannelijk/vrouwelijk) tijdelijk een heleboel 
foute paden worden gemaakt: alle mogelijke vormen van recipro-NPVAR worden 
gecre\"{e}erd, en pas
bij substitutie, als de link met het antecedent wordt gelegd, blijkt welke vorm 
de goede was. Hiervoor is nog geen oplossing gevonden. Misschien is het 
mogelijk om deze recipro-NP(VAR) in talen waar het probleem niet speelt 
ongespecifieerd te laten voor sexes.

De reciprocal regels komen v\'{o}\'{o}r de echte substitutieregels.


\end{document}
