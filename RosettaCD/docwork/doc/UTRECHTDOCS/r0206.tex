
\documentstyle{Rosetta}
\begin{document}
   \RosTopic{General}
   \RosTitle{Notulen Rosetta vergadering 11-6-1987}
   \RosAuthor{Margreet Sanders}
   \RosDocNr{0206}
   \RosDate{July 1, 1987}
   \RosStatus{approved}
   \RosSupersedes{-}
   \RosDistribution{Project}
   \RosClearance{Project}
   \RosKeywords{minutes}
   \MakeRosTitle
%
%
\begin{description}
\item[Aanwezig:] Lisette Appelo, Carel Fellinger, Natalia Grygierczyk,
                 Chris Hazenberg, Franciska de Jong, 
                 Jan Landsbergen, Ren\'{e} Leermakers, 
                 Jeroen Medema, Elly van Munster, 
                 Jan Odijk, Joep Rous, Margreet Sanders (not),
                 Andr\'{e} Schenk, Harm Smit
\item[Afwezig:]  ---
\item[Agenda:]\mbox{}
  \begin{enumerate}
  \item Opening en notulen
  \item Mededelingen
  \item Stand van zaken m.b.t. SPIN en workstation project
  \item Besproken en/of nieuw verschenen documenten
  \item Rondvraag en sluiting
  \end{enumerate}
\end{description}

\section{Opening en notulen}
De notulen van de vorige keer worden na enkele wijzigingen goedgekeurd.

\section{Mededelingen}
\begin{enumerate}
\item Op vrijdag 19 juni is vanaf 18.00 uur (na werktijd, dus) 
iedereen (ook aanhang) welkom bij Jan Landsbergen, Treurenburgstr.\ 
15, tel.\ 444659 voor een {\em borrel} en sociale contacten.
\item {\em Jeroen Groenendijk} zal op dond. 25 juni om 14.00 uur een praatje 
houden voor ge\"{i}nteresseerden over {\em Dynamic Predicate Logic}. Plaats van 
handeling is nog niet bekend.
\item Per 1 sept.\ wordt de groep versterkt met een nieuwe {\em stagiaire}, 
Joleen Schippers. Haar begeleiders zullen Jan Odijk en Wim Zonneveld (Utrecht) 
zijn. Joleen zal onderzoek doen naar een fonetische versie van de morfologie 
van Rosetta (grafeem-foneem omzetting).
\item Een vriendelijk verzoek om Jan O.\ zo min mogelijk te storen (bij 
voorkeur alleen tussen 13 en 14 uur) gezien zijn werk aan zijn proefschrift.
\item Andr\'{e} meldt dat het boek {\em Computational Complexity and Natural 
Language} van Barton, Berwick en Ristad ter inzage ligt op k.\ 355.
\item Joep wijst op de {\em zomercursus Computerlinguistiek} in Amsterdam, van 
26 t/m 28 augustus. De onderwerpen zullen zijn theorie en gereedschap (incl.\ 
een praatje van Joep), automatisch ontleden en toepassingen.
\end{enumerate}

\section{Stand van zaken m.b.t. SPIN en workstation project}
{\bf Spin:} De enige projectmatige uitbreiding naar natuurlijke taalverwerking
die voor SPIN-subsidie in aanmerking lijkt te kunnen komen is een Q/A-systeem.
Contacten van Jan L. met uitgevers daarover zijn niet bemoedigend. Kluwer's
databanken (Datalex: jurisprudentie en Van Loghum Slaterus: conferentie-oorden, 
pius-almanak) zijn niet zo geschikt voor deductieve vragen. Elsevier (Excerpta 
Medica) zit nog in de molen. De speelfilmencyclopedie van Rostrum is 
geschikter, maar zou voornamelijk voor individuele consumenten van belang zijn, 
en niet voor (kapitaalkrachtiger) be\-drijven. Als de directie accoord gaat, 
kunnen SPIN-contacten misschien beter op een laag pitje.\\
{\bf Workstation project:} Jaap Berkhoff van het Nat.\ Lab.\ Geldrop wil in 
principe wel een vertaal-applicatie op zijn workstations (zie notulen 27-4),
maar verdere ontwikkelingen zijn mede 
afhankelijk van wat we precies kunnen bieden. Jan L.\ werkt aan een
project-voorstel. \\
Over CARIN is er nog geen nieuws.

\section{Besproken en/of nieuw verschenen documenten}
\begin{description}
\item [Besproken:]\mbox{}
  \begin{itemize}
  \item Andr\'{e} Schenk: Idioms and the Dictionary (194). Het voorstel is globaal 
geaccepteerd, maar door latere documenten al weer enigszins achterhaald.
  \item Jan Odijk en Andr\'{e} Schenk: Dictionaries in Rosetta3 (201). Het 
voorstel is geaccepteerd, met de restrictie dat een scheiding in een 
morfologisch en een syntactisch woordenboek wel in de editor mogelijk wordt, 
maar niet in de feitelijke implementatie, omdat de vertaling tussen beide 
woordenboeken erg moeilijk zou worden. Over de definitie van het lexicon met 
basisexpressies B moet nog een nader besluit worden genomen, dat mede afhangt 
van de bespreking van doc.\ 205.
  \end{itemize}
\item [Verschenen:] \mbox{}
  \begin{itemize}
  \item Jan Odijk en Andr\'{e} Schenk: Dictionaries in Rosetta3 Revisited (205)
. Dit document wordt aansluitend op de vergadering besproken.
  \end{itemize}
\end{description}

\section{Rondvraag en sluiting}
{\bf Lisette} vraagt hoe het staat met de bespreking van andere vertaalsystemen
. Voorlopig wordt Ren\'{e} (met Margreet?) voor augustus ingeroosterd (Franse 
systemen), en zullen Joep en Jan O. (en evt.\ Jan L.) daarna een keer iets over 
Japanse systemen vertellen.

Hiermee wordt de vergadering om 14.15 uur gesloten. Bijna iedereen is weer present 
om 14.30 uur om {\em doc.\ 205} te bespreken. Globale conclusie: het voorstel wordt 
aangenomen, met de wijziging dat DERIVTREES een dynamisch woordenboek wordt, 
dat de boom van literals pas gaat uitrekenen (op basis van attribuutwaardes) 
als het aangeroepen wordt. Hiermee komt dus in feite een deel van de grammatica 
in het woor\-den\-boek terecht. Jan O. merkte op dat de regels voor 
derivationele morfologie dan ook in het woordenboek terecht komen. Dit probleem 
zette de zaak weer op losse schroeven.

\end{document}

