\documentstyle{Rosetta}
\begin{document}
   \RosTopic{Rosetta3.morphology}
   \RosTitle{Rosetta3 Dutch Morphology, compounds}
   \RosAuthor{Harm Smit}
   \RosDocNr{0485}
   \RosDate{\today}
   \RosStatus{concept}
   \RosSupersedes{-}
   \RosDistribution{Project}
   \RosClearance{Project}
   \RosKeywords{Dutch, morphology, compounds, documentation}
   \MakeRosTitle


\section{Preface}

This document gives an impression of the treatment of compounds 
in the Dutch morphological component of Rosetta at the moment. 

\section{Compounding}

Compounding is a linguistic phenomenon that should be considered in any 
language system. It is also a rather problematic subject: it is not easy
to analyse and generate compounds; also, the translation of a compound may be
difficult.

At the moment, large dictionaries are not available for Rosetta. This implies 
that there are many words which cannot be accepted by Rosetta, although they are
often very regular compounds. In order to make Rosetta more robust we developed
some rules for compounding. One should note that these rules are blocked when a
wordform can be analysed in any way by the `normal' segmentation of the 
morphology. 

There are major differences between languages with respect 
to compounding: 
in Dutch we write {\em bloedgroep} as a single 
word, in English {\em blood group} separated by a space. 
In Dutch compounds can 
be `glued' by {\em -s-}, {\em -e-} or hyphen: {\em loonsvermindering}, 
{\em hondehok} or {\em amateur-wielrenner}. In Dutch, it is difficult to
predict how the parts of a specific compound will be glued. Often more
ways are possible. The next sections 
show how the Dutch morphological component has been adapted for compounds.

\newpage
\section{Current implementation}

In the current implementation the rules for compounding are used only when the
normal morphological analysis does not yield any results. The compounding rules
try to split a word in two parts from left to right (the right part as big as 
possible). When the right part yields a result after trying the normal 
morphological rules, the left part is analysed and as soon as the left part
turns out to be a word too, the analysis stops to limit the number of
trees. Also, the left part may
be a compound itself. Examples: {\em lastdier} is split in {\em last} and
{\em dier} because {\em dier} is the longest right-part that is a noun and
{\em last} is also a word. For nouns, the compound-rules accept plurals
and genitives as right-part and plurals as left part. Also diminutives are
accepted. Other examples: {\em gerstebier} is split in {\em gerst} and 
{\em bier}, {\em mannenwerk} in {\em mannen} and {\em werk}, {\em volksstam}
in {\em volk} and {\em stam}.

At the moment two types of compounds are possible: noun combined with noun
(the left noun can be a compound itself) and verb (as left part) 
combined with noun (at the right side).

When a noun acts as left part, in generation the compound may be `glued' by 
{\em -s-}, {\em -e-} or hyphen, or both parts may be written together
without anything between them. Nouns that have a plural ending in 
{\em -eren} (like: {\em kinderen}) may be glued by {\em -er-} (example:
{\em kinderspeelgoed}). There are no well-defined conditions on the type of 
`glue', so there might be some over-accepting in analysis. In 
generation words are only generated with a hyphen (see section 4); 
in many cases this is not
wrong, and it helps to see that the word is generated by compound rules 
(although some words in the dictionary may contain a hyphen too).

When a verb acts as left part of a compound it should be the first person
singular present, which is equal to the stem except for
verbs like {\em snijden, rijden, houden} which have an extra form ({\em snij}
vs. {\em snijd}).

\section{Rules}

In this section we discuss the (two) rules for compounding 
that have been implemented at the 
moment; both result in SUBNOUN.

The rules for compounding are added for robustness; they are
not sophisticated and limited in generation capacity (in analyses,
the rules accept quite a few forms). Example: words like:
{\em kasteletocht, kasteelstocht, kasteel-tocht and kasteeltocht}
all will be accepted, whereas only {\em kasteel-tocht} will be generated.
This is done because the forms with "-" are never wrong (although they
will often be a bit unnatural) and it gives the user an impression of
how the system interpreted the compound.
      
\newpage
\subsection{Rule for compounds of the type noun-noun.}

In analysis, all possible `glues' are accepted, 
with the letters {\em -s-} or {\em -e-}, or hyphen, 
or both parts can be written together
( {\em kasteletocht, kasteelstocht, kasteel-tocht and kasteeltocht}); 
in generation only {\em kasteel-tocht} will be generated.

\begin{verbatim}
%NOUNcomp1

m1:   NOUN{NOUNrec1}[mu1]
m2:   SFCAT{SFCATrec1}
m3:   SUBNOUN{SUBNOUNrec1}[mu2]
m:    SUBNOUN{SUBNOUNrec2}
             [modrel/NOUN{NOUNrec1}[mu1],
              head/SUBNOUN{SUBNOUNrec1}[mu2]
             ]

comp:
         C1:  (NOUNrec1.geni = false) AND
              (SUBNOUNrec1.lastaffix = noaffix)
           C2:  SFCATrec1.key = SFKleegCompound
           A2:  SUBNOUNrec2 := SUBNOUNrec1
           C2:  SFCATrec1.key = SFKstreepCompound
           A2:  SUBNOUNrec2 := SUBNOUNrec1
           C2:  SFCATrec1.key = SFKsCompound
           A2:  SUBNOUNrec2 := SUBNOUNrec1
           C2:  SFCATrec1.key = SFKeCompound
           A2:  SUBNOUNrec2 := SUBNOUNrec1
        A1:  @

decomp:
         C1:  (NOUNrec1.geni = false) AND
              (SUBNOUNrec1.lastaffix = noaffix)
         A1:  SFCATrec1.key := SFKstreepCompound
&
\end{verbatim}
\newpage
\subsection{Rule for compounds of the type VERB - NOUN.}

In the combination verb-noun the verb has to be of the form
first person singular present tense: {\em gloeilamp, glijbaan,
rijweg} (not the stem: {\em glijdbaan} is not correct!); an
exception like {\em staanplaatsen} has to be in the dictionary.
It is not possible to bind the verb and noun with a "s" or "e"
(exceptions like: {\em zegsman, scheidsrechter, verjaarspartij}
have to be in the dictionary!).

\begin{verbatim}
%VerbNOUNcomp

m1:   VERB{VERBrec1}[mu1]
m2:   SFCAT{SFCATrec1}
m3:   SUBNOUN{SUBNOUNrec1}[mu2]
m:    SUBNOUN{SUBNOUNrec2}
             [modrel/VERB{VERBrec1}[mu1],
              head/SUBNOUN{SUBNOUNrec1}[mu2]
             ]

comp:
         C1:  (SUBNOUNrec1.lastaffix = noaffix)
              and (VERBrec1.tense = presenttense)
              and (VERBrec1.modus = indicative)
              and (VERBrec1.number = singular)
              and ([0,1] * VERBrec1.persons <> [])
              and (VERBrec1.eORenForm = NoForm)
           C2: (SFCATrec1.key = SFKleegCompound) 
           A2:  SUBNOUNrec2 := SUBNOUNrec1
           C2: (SFCATrec1.key = SFKstreepCompound) 
           A2:  SUBNOUNrec2 := SUBNOUNrec1
         A1:  @

decomp:
         C1:  (SUBNOUNrec1.lastaffix = noaffix)
              and (VERBrec1.tense = presenttense)
              and (VERBrec1.modus = indicative)
              and (VERBrec1.number = singular)
              and ([0,1] * VERBrec1.persons <> [])
              and (VERBrec1.eORenForm = NoForm)

         A1:  SFCATrec1.key := SFKstreepCompound
&
\end{verbatim}

\newpage
\subsection{Rule for segmentation of compounds}

The segmentation rules can be found in the file SUFFIX.SEG; all rules with the 
following suffixkeys are for compounding:
{\tt SFKstreepCompound, SFKleegCompound, SFKsCompound, SFKeCompound};
the suffixkey {\tt SFKstreepCompound} is used in only one rule, and this is the
only rule that works in generation. This rule makes strings like 
{\em kasteel-tocht}. Rules with {\tt SFKleegCompound} would generate 
{\em kasteeltocht}, rules with {\tt SFKsCompound} {\em kasteelstocht}, and rules
with {\tt SFKeCompound} {\em kasteletocht}. The suffixkey {\tt SFKeCompound} 
is also used for words with {\em -er}, like {\em kinderhand, hoenderhok}, etc.

The segmentationrules for compounds are in general identical to the normal 
segmentation
rules; the same processes on strings take place, like for instance consonant
doubling: {\em kip\underline{p}ehok}.

\section{Final remarks}

The rules for compounds that are implemented at the moment are only added to 
the system for robustness. A better solution for compounding will need a
more detailed study of this phenomenon. Another way to improve the system would
be the use of very large dictionaries containing all frequent compounds.

\newpage
\section{References}

\begin{description}
  \item [ANS (GEERTS, G. e.a.)]
     1984, {\bf Algemene Nederlandse Spraakkunst}, Groningen, Wolters-Noordhoff.

  \item [VAN DALE (VAN STERKENBURG, P.G.J. e.a.)]
     1984, {\bf Groot woordenboek van hedendaags Nederlands}, Utrecht, Van Dale 
           Lexicografie b.v.

\end{description}
\end{document}
