
\documentstyle{Rosetta}
\begin{document}
   \RosTopic{Rosetta3.doc.Mrules.English}
   \RosTitle{Rosetta3 English M-rules: VerbppropFormation}
   \RosAuthor{Margreet Sanders, Lisette Appelo, Andr\'{e} Schenk}
   \RosDocNr{310}
   \RosDate{November 15, 1989}
   \RosStatus{approved}
   \RosSupersedes{-}
   \RosDistribution{Project}
   \RosClearance{Project}
   \RosKeywords{English, M-rules, documentation, VerbppropFormation}
   \MakeRosTitle
%
%
\section{Introduction}
The English sentence grammar is divided in three parts, in the same way as all
other main category grammars. First, there is a subgrammar providing the 
PROP-structure, 
called {\bf VerbppropFormation}. Then, {\bf XPPROPtoCLAUSE} turns this prop
into a clause. Finally, {\bf ClauseToSentence} makes a full sentence of this
clause. Around the Sentence grammar there are two other, very small grammars.
Prior to VerbppropFormation, there is the {\bf VerbDerivation} grammar, and
following ClauseToSentence, there is the {\bf Utterance} grammar.

The current document describes the contents of 
the first Sentence subgrammar, Verbppropformation. The grammar consists of 
a number of rule classes and transformation classes. A rule class in its turn
consists of a number of rules and a transformation class of a number of 
transformations. The relative ordering of the rules and transformations in the
(sub)grammar is indicated by a {\em control expression}. A summary of this 
control expression (i.e.\ a listing of the ordering of the rule classes, 
without explicit mentioning of the rules themselves) is also included here, 
and the initial (= head), import and export categories are given. Basically, 
the document describes the system as it was when the document was first 
written, i.e.\ in May 1989. Modifications that took place later have been added 
in footnotes.

In the section on the rules and transformations, only the rule names 
are given, but not the exact rule formulation. What is attempted 
is to provide a detailed overview of the workings of the (sub)grammars, and 
how the different rule classes achieve this,
together with some comments on the problems still to be solved, the reasons 
behind certain choices, and perhaps possible alternatives. For every rule, an 
example is given. If it is uncertain whether the example is correct (either 
because it may not be an example of the phenomenon in question, or because it 
may not be correct English), it is preceded by a question mark. Note that all 
explanation of rules and transformations is given from a generative viewpoint 
only, unless explicitly stated otherwise. Often, the information given in this 
document is based strongly on the comment already present in the documentation
of the rules themselves. Discrepancies between what is stated here and what is 
said in the rule itself are usually caused by the fact that the rule file has 
not been updated, although insights have changed.

Whenever the current implementation differs widely from the strategy that was 
devised in the definition phase of Rosetta3 (as laid down for English in docs.\ 
150, {\em Subgrammars of English\/}, 153, {\em Rule and Transformation Classes 
of English\/}, and 155, {\em Rule and Transformation Classes common to all 
languages\/}, all written by Jan Odijk), this will be indicated explicitly in 
the current document. Conditions on crucial orderings of rule classes will be 
repeated here, even if they do not differ from the original strategy, to make 
the document as self-contained as possible.

Finally note that the rules described in this document have NOT been tested 
properly. English analysis is not possible yet (there is no Surface Parser), 
and English generation has only been tested in as far the construction was the 
translation of a Dutch sentence to be tested.

\newpage
In principle, the authors of the rule classes have each documented their own 
classes: Andr\'{e} Schenk wrote the sections on TC\_IdiomCheck, 
RC\_IdStartVerbpRules and TC\_PossAdjSpelling, and Lisette Appelo wrote 
the sections on TC\_AktionsartCalc and RC\_TempAdvVar. The other sections have 
been written by Margreet Sanders.

\newpage
\section{Verbppropformation}
The task of this subgrammar is to provide a subverb with its correct number of 
argument variables, thus creating a prop-structure. Furthermore, variables for 
a number of non-argument adverbials and subordinate clauses are introduced 
here, and the structure is given a voice. Transformations take care of the 
realization of verbpatterns, particles, inherent reflexives and Aktionsart.

The general strategy for this subgrammar is the same as was mentioned in doc.\ 
150. A few changes have been made, however. They mainly concern the 
introduction of variables for non-argument adverbials. Instead of being all in 
one large class, the rules have been separated into different classes. This was 
mainly caused by a wish to keep the different cases separated until the 
ordering of the rule classes and their interrelation had 
been understood enough to incorporate the rules in one large class.
This stage has not been reached yet, nor is it certain that for English 
separation is not adequate\footnote{By the time this document was approved, the 
classes were merged again. The strategy that is used to remove the variables 
in analysis in different than in Dutch. There, they are removed strictly
according to surface order, starting with the last one. In English, the strategy 
chosen was to remove them class by class: first, all tempadvs are removed, then 
the locadvs, and finally the conjsents and sentadvs. Thus, the presentation in 
the current document is somewhat misleading: it might seem as if locadvs are removed 
last. In generation, there only are conditions on the surface order of the 
variables, but not on their order of insertion: whatever is demanded by the 
source language is allowed. The strategy is to keep the surface order of 
locatives, conjsents and sentadvs the same in Dutch and English. Temporal 
elements have a reversed surface order in English compared to Dutch.}. 
Note that English does not 
have any rules yet for instrumental or comitative adverbials, which also 
belong somewhere in these rule classes.

Another change in strategy concerning variables for adverbials is the way in 
which they are introduced. Doc.\ 155 announced an abstract preposition and an 
OPENPREPPPROP for temporal, locative and causative adverbials. This has now 
been changed to introduction of a simple PREPPVAR or ADVPVAR for all 
non-sentential variables introduced in this class. 
VP-modifying adverbials (including `agvpadvs'), which do require a PROP 
variable, have been removed from this rule class and are now placed in the next 
subgrammar, XPPROPtoCLAUSE.
More information on this 
subject can be found in the document on The Treatment of Adverbs in Rosetta3 
(by Jan Odijk, to appear).

The calculation of Aktionsart has been given its own transformation class, at 
the end of the subgrammar. In doc.\ 155 it was assumed that this could be done 
in the start rules.

Finally, it should be noted that no rules have been written yet for pre-VP 
modification. This rule class should come at the end of the subgrammar.


\newpage
\section{Subgrammar Specification}
The subgrammar definition can be found in file 
{\bf english:Verbppropformation.mrule}, which is {\em mrules3.mrule\/}.

\begin{description}
\item[Control Expression] \mbox{}
\begin{verbatim}
%SUBGRAMMAR Verbppropformation

(  ( TIsNotIdiom . RC_StartVerbpRules:mrules50 )  
 | ( TIsIdiom . RC_IdStartVerbpRules:mrules107 )   )
.  ( TC_VerbPatternRules: mrules 49-45 )
.  ( TC_ParticleSpelling: mrules44 )
.  { RC_TempAdvVar: mrules36 }
.  { RC_SentAdvvar:mrules91, begin }
.  { RC_LocAdvvar:mrules91, end }
.  ( RC_VoiceRules:mrules43 )
.  ( TC_ReflexiveSpelling: mrules42, begin )
.  ( TC_PossAdjSpelling: mrules42, end )
.  ( TC_AktionsartCalc: mrules81 )

\end{verbatim}

  \item[Head]  SUBVERB FROM (VERBDERIVATION)
  \item[Export]  VERBPPROP 
  \item[Import] ADJPPROPVAR, ADVPPROPVAR, NPPROPVAR, PREPPPROPVAR, 
                NPVAR, SENTENCEVAR, VERBPPROPVAR, CNVAR, PROSENTVAR,
                EMPTYVAR, PREPPVAR, ADVPVAR 
\end{description}

\newpage
\section{Rules and Transformations}
\subsection{TC\_IdiomCheck}
\begin{description}
\item[Kind] Obligatory Transformation class
\item[Task] To determine whether the skey of the head (SUBVERB) is an idiom key 
or not. In case it is not, the ordinary startrules apply, in case it is, the
idiom start rules are called. These transformations are used because otherwise 
the `ordinary' startrules might be applied to an idiom verb, resulting in a 
crash in morhology because an idiom does not have a string directly connected 
with it. For more information, see doc.\ 273 on idioms.

\vspace{1 cm}
\begin{description}
\item[Name] TIsIdiom
\item[Task] To check whether the skey of the SUBVERB is an idiom key. If it is,
the special idiom startrules are called, and the ordinary startrules are 
skipped (see control expression).
\item[File] english:RC\_IdStartVerbpRules.mrule (mrules107.mrule)
\item[Semantics] --
\item[Example] lose (x1 lose\_x1's\_patience)
\item[Remarks]
\end{description}

\vspace{1 cm}
\begin{description}
\item[Name] TIsNotIdiom
\item[Task] To check whether the skey of the SUBVERB is not an idiom key. If it 
is not, the special idiom startrules are skipped, and the ordinary startrules 
can apply (see control expression).
\item[File] english:RC\_IdStartVerbpRules.mrule (mrules107.mrule)
\item[Semantics] --
\item[Example] lose (x1 lose x2)
\item[Remarks]
\end{description}

\vspace{1 cm}
\item[Remarks] This transformation class is new compared to doc.\ 150. 
\end{description}

\newpage
\subsection{RC\_StartVerbpRules}
\begin{description}
\item[Kind] Obligatory Rule class
\item[Task] To provide a (non-idiom) subverb with its correct number of 
argument variables and build a VERBP and VERBPPROP around it. There are rules
for ordinary verbs and rules for semi-idioms.
The import (i.e.\ the 
argument(s) the verb takes) to the former rules may be variables of any kind.
The import to the latter rules is subjected to an extra check.

The subverb is given a new top node VERB, with all new attributes still at 
default value. It was decided to write rules 
for ergatives in English too, although it is not clear at all that they exist.
For testing purposes, a few verbs were entered into the dictionary as 
ergatives. Of course, the startrules needed for ergatives are the same 
as for raising verbs, so they will be needed anyway.

In English, no rules have been written yet for verbs with a deviant order of 
arguments (esp.\ vp132). In case the surface order of VP-arguments is different 
from the ordering in the startrules, a verbpattern is assigned which swaps 
them.
No theoretical considerations underly this decision; if it 
is thought better, extra startrules may be written and current verbpattern rules 
may be replaced by others. This is especially relevant if two verbpatterns 
exist that differ only in the ordering of the VP arguments. In that case, one 
of the two patterns can be deleted\footnote{By the 
time this document was approved, a startrule for vp132 was written.}. 
If a thetavp like 210 will be used, the realization
cannot be taken care of by pattern rules anyway (the subject is not in the VP), 
and new startrules must be added.

\vspace{1 cm}
\begin{description}
\item[Name] RStartVPPROP000
\item[Task] To provide a prop structure for a verb that does not take any 
arguments at all (weather verbs). A dummy subject {\em It\/} is introduced 
here, of which it is not clear whether it is really non-referential. Perhaps it 
may even function as an antecedent for `reflexive weather verbs' like 
{\em It has rained itself out\/}. However, the reflexive spelling rules (see 
below) are not suited yet to deal with NP antecedents; they only work on 
variables.

\item[File] english:RC\_StartVerbpRules.mrule (mrules50.mrule)
\item[Semantics]
\item[Example] rain $\rightarrow$ it rain
\item[Remarks] For an extensive discussion of the argument status of 
{\em It\/}, see section 4.1 of doc.\ 308 on the Dutch VERBPPROPformation.
\end{description}

\vspace{1 cm}
\begin{description}
\item[Name] RStartVPPROP100
\item[Task] To provide a prop structure for a verb that takes one subject 
argument, and no arguments in the VERBP.
\item[File] english:RC\_StartVerbpRules.mrule (mrules50.mrule)
\item[Semantics]
\item[Example] swim $\rightarrow$ x1 swim
\item[Remarks]
\end{description}

\vspace{1 cm}
\begin{description}
\item[Name] RStartVPPROP010a
\item[Task] To provide a prop structure for a verb that takes no subject, and 
one argument in the VERBP. This may be an ergative, or a raising verb. The verb 
is not a case assigner. No dummy subject is inserted, because the subject slot 
must remain empty to allow raising.
\item[File] english:RC\_StartVerbpRules.mrule (mrules50.mrule)
\item[Semantics]
\item[Examples] ergative: melt $\rightarrow$ melt x1\\
\ \ \ \ \ \ \ \ raising verb: seem $\rightarrow$ seem x1
\item[Remarks]
\end{description}

\vspace{1 cm}
\begin{description}
\item[Name] RStartVPPROP010b
\item[Task] To provide a prop structure for a verb that takes no subject, and
one argument in the VERBP, while the verb is a case assigner. In this case, a 
dummy subject is inserted to prevent the object from being raised. 
\item[File] english:RC\_StartVerbpRules.mrule (mrules50.mrule)
\item[Semantics]
\item[Example] rain $\rightarrow$ it rain x1 (it rains invitations)
\item[Remarks]
\end{description}

\vspace{1 cm}
\begin{description}
\item[Name] RStartVPPROP120
\item[Task] To provide a prop structure for a verb that 
takes two arguments, one a subject and the other in the VERBP.
\item[File] english:RC\_StartVerbpRules.mrule (mrules50.mrule)
\item[Semantics]
\item[Example] hit $\rightarrow$ x1 hit x2; eat $\rightarrow$ x1 eat x2
\item[Remarks]
\end{description}

\vspace{1 cm}
\begin{description}
\item[Name] RStartVPPROP012
\item[Task] To provide a prop structure for a verb that takes two arguments, 
both of them in the VERBP. They may be ergatives or raising verbs.
\item[File] english:RC\_StartVerbpRules.mrule (mrules50.mrule)
\item[Semantics]
\item[Example] ergative: come $\rightarrow$ come x1 x2 (for testing purposes)\\
\ \ \ \ \ \ \ \ raising verb: seem $\rightarrow$ seem x1 x2
\item[Remarks]
\end{description}

\vspace{1 cm}
\begin{description}
\item[Name] RStartVPPROP123
\item[Task] To provide a prop structure for a verb that takes three arguments,
one a subject and the other two in the VERBP.
\item[File] english:RC\_StartVerbpRules.mrule (mrules50.mrule)
\item[Semantics]
\item[Example] give $\rightarrow$ x1 give x2 x3
\item[Remarks]
\end{description}

\vspace{1 cm}
\begin{description}
\item[Name] RSIDStartVPPROP120
\item[Task] To provide a prop structure for a verb which is part of a semi 
idiom and which takes two arguments,
one a subject and the other in the VERBP. The head of the argument NP in VERBP 
is specified as being equal to the key the bverb expects for the semi-idiom.
\item[File] english:RC\_StartVerbpRules.mrule (mrules50.mrule)
\item[Semantics]
\item[Example] give $\rightarrow$ x1 give x2=shout
\item[Remarks] The semi-idiom BVERB is replaced by an `ordinary' BVERB (which is 
the same except for its key) without changing the SUBVERB heading it. This is 
possible because the {\bf key}-attribute does not percolate upwards.
\end{description}

\vspace{1 cm}
\begin{description}
\item[Name] RSIDStartVPPROP123
\item[Task] To provide a prop structure for a verb which is part of a semi 
idiom and which takes three arguments,
one a subject and the other two in the VERBP. The head of the first (`object') 
argument NP in the VERBP is specified as being equal to the key the bverb 
expects for the semi-idiom.
\item[File] english:RC\_StartVerbpRules.mrule (mrules50.mrule)
\item[Semantics]
\item[Example] give $\rightarrow$ x1 give x2=demonstration x3
\item[Remarks] The semi-idiom BVERB is replaced by an `ordinary' BVERB (which is 
the same except for its key) without changing the SUBVERB heading it. This is 
possible because the {\bf key}-attribute does not percolate upwards.
\end{description}

\vspace{1 cm}
\item[Remarks] The calculation of the Aktionsart of the VERBPPROP is not done 
here, as was assumed in doc.\ 155, but in a separate transformation class 
coming at the end of the subgrammar. This is necessary since the calculation 
rules are quite complex and crucially need information on the kind of arguments
the verb takes, and not the just the number of arguments.
\end{description}

\newpage
\subsection{RC\_IdStartVerbpRules}

\begin{description}
\item[Kind] Obligatory Rule Class
\item[Task] On the basis of a syntactic derivation tree that is specified in 
the M-rule  model the syntactic S-tree for an idiom is built. This includes 
much more than just a VERBP and a VERBPPROP. For further explanation, see doc.\ 
273 on Idioms.

\vspace{1 cm}
\begin{description}
\item[Name] RIdDeriv1
\item[Task] Build the S-tree for the case of a verb with a free subject and 
one (direct) object singular noun argument in the VERBP as part of the idiom
\item[File] english:rc\_idstartverbprules.mrule (mrules107.mrule)
\item[Semantics]
\item[Example] bury $\rightarrow$ x1 bury the hatchet
\item[Remarks]
\end{description}

\vspace{1 cm}
\begin{description}
\item[Name] RIdDeriv3
\item[Task] Build the S-tree for the case of a verb with a free subject and 
one (direct) object 
mass or plural noun argument in the VERBP as part of the idiom. This argument 
contains a 
free element which is identical with the subject and will be realized as a 
possessive.
\item[File] english:rc\_idstartverbprules.mrule (mrules107.mrule)
\item[Semantics]
\item[Example] lose $\rightarrow$ x1 lose x1 patience
\item[Remarks]
\end{description}

\vspace{1 cm}
\begin{description}
\item[Name] RIdDeriv7
\item[Task] Build the S-tree for the case of a verb with a free subject,
one (free) object 
argument in the VERBP and one LocOpenPREPPPROP argument as part of the idiom.
The subject of the PROP is identical with the object.
\item[File] english:rc\_idstartverbprules.mrule (mrules107.mrule)
\item[Semantics]
\item[Example] leave $\rightarrow$ x1 leave x2 [x2 in the lurch]
\item[Remarks]
\end{description}

\vspace{1 cm}
\begin{description}
\item[Name] RIdDeriv8
\item[Task] Build the S-tree for the case of a verb with a free subject and 
one singular noun (direct) object 
argument in the VERBP as part of the idiom. This argument contains a free 
element which will be spelled out as a possessive.
\item[File] english:rc\_idstartverbprules.mrule (mrules107.mrule)
\item[Semantics]
\item[Example] break $\rightarrow$ x1 break x2 heart
\item[Remarks]
\end{description}

\vspace{1 cm}
\begin{description}
\item[Name] RIdDeriv9
\item[Task] Build the S-tree for the case of a verb with a free subject and 
two arguments in the 
VERBP, one singular noun (direct) object as part of the idiom, and one free 
(indirect) object.
\item[File] english:rc\_idstartverbprules.mrule (mrules107.mrule)
\item[Semantics]
\item[Example] give $\rightarrow$ x1 give x2 the boot
\item[Remarks]
\end{description}

\vspace{1 cm}
\begin{description}
\item[Name] RIdDeriv10
\item[Task] Build the S-tree for the case of a verb with a free subject and 
two arguments in the 
VERBP, one Loc/DirOpenPREPPPROP argument as part of the idiom, and one (free) 
object. The PROP contains a free 
element which will be spelled out as a possessive, and the subject of the PROP 
is identical to the verb object.
\item[File] english:rc\_idstartverbprules.mrule (mrules107.mrule)
\item[Semantics]
\item[Example] lay $\rightarrow$ x1 lay x2 [x2 at x3 door]
\item[Remarks]
\end{description}

\vspace{1 cm}
\begin{description}
\item[Name] RIdDeriv12
\item[Task] Build the S-tree for the case of a verb with a free subject and 
one (direct) object 
singular noun argument in the VERBP as part of the idiom. This argument contains
a free element which is identical with the subject and will be realized as a 
possessive.
\item[File] english:rc\_idstartverbprules.mrule (mrules107.mrule)
\item[Semantics]
\item[Example] sling $\rightarrow$ x1 sling x1 hook
\item[Remarks]
\end{description}

\vspace{1 cm}
\begin{description}
\item[Name] RIdStartVPPROP1VAR2ID0
\item[Task] This rule is an alternative to RIdDeriv1, and is NOT a part of the 
translation system. Instead of specifying in the rule the Derivation tree 
for the idiom parts (i.e.\ the rules 
to be applied in building the S-tree), this rule spells out the S-tree itself
(in this case an S-tree for a verb with a subject and 
one (direct) object argument in the VERBP as part of the idiom. The article 
going with the object is determined by the idiom key). For further comment, 
see doc.\ 273 on Idioms.
\item[File] english:rc\_idstartverbprules.mrule (mrules107.mrule)
\item[Semantics]
\item[Example] bury $\rightarrow$ x1 bury the hatchet
\item[Remarks]
\end{description}

\end{description}

\newpage
\subsection{TC\_VerbpatternRules}

\begin{description}
\item[Kind] Obligatory Transformation Class
\item[Task] Check the categories of the argument variables in the VERBP 
against the verbpatterns specified for the 
verb, and assign a specific relation name to these arguments.
The synvpefs attribute of the VERBPPROP is set at the value actually 
chosen from the synvps of the VERB. Some verbpatterns are still missing, 
because they had not been thought of yet. They will be added. 
For comment on the ordering of the two VP-arguments, see what was said in 
section RC\_StartVerbpRules.

The names of the verbpatterns are not given here, only the ordering of the two 
variables, their categories and their relations. For more information on 
verbpatterns, see doc.\ 248, {\em Verbpatterns of English\/}.

\vspace{1 cm}
\begin{description}
\item[Name] TVerbPattern0
\item[Task] To let verbs that have no VP arguments (vp000 and vp100) pass this 
transformation class.
\item[File] english:TC\_VP1.mrule (mrules49.mrule)
\item[Semantics] --
\item[Example] x1 sleep $\rightarrow$ x1 sleep
\item[Remarks]
\end{description}

\vspace{1 cm}
\begin{description}
\item[Name] TVerbPattern1
\item[Task] Specify the relation name and check the category of the VP-argument of 
verbs with vp010 and vp120
\item[File] english:TC\_VP1.mrule (mrules49.mrule)
\item[Semantics] --
\item[Examples] (ordered by subrule)
\begin{description}
  \item[1a] x1 beat argrel/VAR $\rightarrow$ x1 beat objrel/NPVAR \\
(Mary beat John)
  \item[1b] x1 beat argrel/VAR $\rightarrow$ x1 beat objrel/CNVAR \\
(The man whom Mary beat)
  \item[2] x1 eat argrel/VAR $\rightarrow$ x1 eat objrel/EMPTYVAR \\
(They ate in silence)
  \item[3a] become argrel/VAR $\rightarrow$ become complrel/(closed)NPPROPVAR
 \\
(She became President)
  \item[3b] ? x1 act argrel/VAR $\rightarrow$ x1 act complrel/(open)NPPROPVAR
 \\
(He is always acting the experienced man)
  \item[4a] seem argrel/VAR $\rightarrow$ seem complrel/(closed)ADJPPROPVAR 
\\
(She seemed ill)
  \item[4b] ? smell argrel/VAR $\rightarrow$ smell complrel/(open)ADJPPROPVAR
 \\
(The book smelt old)
  \item[5] x1 weigh argrel/VAR $\rightarrow$ x1 weigh 
complrel/(open-measurephrase)NPPROPVAR
(The lorry weighed a ton)
  \item[6] x1 act argrel/VAR $\rightarrow$ x1 act complrel/(asif)SENTENCEVAR
 \\
(He acted as if he had gone mad)
  \item[7a1] x1 put argrel/VAR $\rightarrow$ x1 put 
locargrel/(locclosed)PREPPPROPVAR 
(They put him in bed)
  \item[7a2] x1 put argrel/VAR $\rightarrow$ x1 put 
locargrel/(locclosed)ADVPPROPVAR 
(They put him there)
  \item[7b1] x1 stand argrel/VAR $\rightarrow$ x1 stand
locargrel/(locopen)PREPPPROPVAR 
(He stood behind the door)
  \item[7b2] x1 live argrel/VAR $\rightarrow$ x1 live
locargrel/(locopen)ADVPPROPVAR 
(He lives southwards)
  \item[8a1] ? x1 see argrel/VAR $\rightarrow$ x1 see
dirargrel/(dirclosed)PREPPPROPVAR 
(They saw him to the door)
  \item[8a2] x1 verb argrel/VAR $\rightarrow$ x1 verb
dirargrel/(dirclosed)ADVPPROPVAR 
(no example found)
  \item[8b1] x1 jump argrel/VAR $\rightarrow$ x1 jump
dirargrel/(diropen)PREPPPROPVAR 
(She jumped off the fence)
  \item[8b2] x1 swim argrel/VAR $\rightarrow$ x1 swim
dirargrel/(diropen)ADVPPROPVAR 
(They swam westwards)
  \item[9a1] ? seem argrel/VAR $\rightarrow$ seem
complrel/(otherclosed)PREPPPROPVAR 
(She seemed with child)
  \item[9a2] x1 verb argrel/VAR $\rightarrow$ x1 verb
complrel/(otherclosed)ADVPPROPVAR 
(no examples found)
  \item[9b1] x1 verb argrel/VAR $\rightarrow$ x1 verb
complrel/(otheropen)PREPPPROPVAR 
(no examples found)
  \item[9b2] x1 verb argrel/VAR $\rightarrow$ x1 verb
complrel/(otheropen)ADVPPROPVAR 
(no examples found)
  \item[10a] x1 have argrel/VAR $\rightarrow$ x1 have 
complrel/(closed)VERBPPROPVAR 
(They had a house built)
  \item[10b] x1 verb x2 argrel/VAR $\rightarrow$ x1 verb 
complrel/(open)VERBPPROPVAR 
(no examples found)
  \item[11a] can argrel/VAR $\rightarrow$ can 
complrel/(closedinf)SENTENCEVAR \\
(He can come - possibility)
  \item[11b] x1 can argrel/VAR $\rightarrow$ x1 can 
complrel/(openinf)SENTENCEVAR 
(He can come - ability)
  \item[12a] x1 expect argrel/VAR $\rightarrow$ x1 expect
complrel/(closedtoinf)SENTENCEVAR 
(I expect you to be there)
  \item[12b] x1 decide argrel/VAR $\rightarrow$ x1 decide
complrel/(opentoinf)SENTENCEVAR 
(We decided to leave)
  \item[13] x1 prefer argrel/VAR $\rightarrow$ x1 prefer
complrel/(fortoinf)SENTENCEVAR 
(I prefer for John to stay)
  \item[14] x1 know argrel/VAR $\rightarrow$ x1 know 
complrel/(that)SENTENCEVAR \\
(He knows that you cannot refuse)
  \item[15] x1 wonder argrel/VAR $\rightarrow$ x1 wonder
complrel/(q)SENTENCEVAR \\
(I wonder what to do know)
  \item[16a] x1 watch argrel/VAR $\rightarrow$ x1 watch
complrel/(accing)SENTENCEVAR 
(I watched him singing)
  \item[16b] x1 burst (out) argrel/VAR $\rightarrow$ x1 burst (out)
complrel/(opening)SENTENCEVAR 
(I burst out singing)
  \item[17] x1 wonder argrel/VAR $\rightarrow$ x1 wonder
complrel/PROSENTVAR \\
(I wonder)
  \item[18] x1 think argrel/VAR $\rightarrow$ x1 think 
complrel/(so)PROSENTVAR \\
(They thought so)
  \end{description}
\item[Remarks] Some subrules were written although no examples could be found
(just in case). Other verbpatterns have been discovered since, that have not 
been incorporated yet, e.g.\ a synFRONTSOPROSENT, to 
cover structures like {\em So he indicated\/}. 
\end{description}

\vspace{1 cm}
\begin{description}
\item[Name] TVerbPattern2
\item[Task] Specify the relation name and check the category of the VP-argument of 
verbs with vp010 or vp120 having the synpattern synITTHATSENT
\item[File] english:TC\_VP1.mrule (mrules49.mrule)
\item[Semantics] --
\item[Example] x1 believe argrel/VAR $\rightarrow$ x1 believe it 
complrel/(that)SENTENCEVAR 
(I can't believe it that I've won the lottery)
\item[Remarks] Perhaps this verbpattern may be used to disambiguate factive 
from non-factive readings (a factive reading would require `it', a non-factive
reading would not). This idea has not been worked out, however.
\end{description}

\vspace{1 cm}
\begin{description}
\item[Name] TVerbPattern3
\item[Task] Specify the relation name and check the category of the VP-argument of 
verbs with vp010 or vp120, in case this argument is a prepositional phrase. The 
preposition is introduced syncategorematically. The relation name always is 
prepobjrel.

English does not distinguish between {\em for\/}, {\em to\/} and other preps, as 
does Dutch, since there does not seem to be any difference in the behaviour of 
these preps, nor in their position in the sentence. If differences are found, 
extra rules can be added simply.

\item[File] english:TC\_VP2.mrule (mrules48.mrule)
\item[Semantics] --
\item[Examples] (ordered by subrule)
  \begin{description}
  \item[1a] x1 count argrel/VAR $\rightarrow$ x1 count [upon NPVAR] \\
(I count upon you)
  \item[1b] x1 count argrel/VAR $\rightarrow$ x1 count [on CNVAR] \\
(The relief which I was counting on)
  \item[2a] x1 verb argrel/VAR $\rightarrow$ x1 verb [prep (closed)NPPROPVAR] 
\\ (no examples found)
  \item[2b] x1 come (across) argrel/VAR $\rightarrow$ x1 come (across) [as
(open)NPPROPVAR] 
(He came across as a nice person) 
  \item[3a] x1 count argrel/VAR $\rightarrow$ x1 count [on (accing)SENTENCEVAR] 
\\
(You cannot count on the weather being fine)
  \item[3b] x1 hesitate argrel/VAR $\rightarrow$ x1 hesitate [about
(opening)NPVAR] \\
(I am hesitating about joining the expedition)
  \item[4] x1 wait argrel/VAR $\rightarrow$ x1 wait [for 
(closedtoinf)SENTENCEVAR] \\
(He waited for the president to speak)
  \item[5] x1 verb argrel/VAR $\rightarrow$ x1 verb [prep
(closed)ADJPPROPVAR] \\
(no examples found)
  \item[6] x1 talk argrel/VAR $\rightarrow$ x1 talk [about (q)SENTENCEVAR] \\
(We all talked about why he would have murdered here)
  \item[7] x1 verb argrel/VAR $\rightarrow$ x1 verb [prep
(otherclosed)PREPPPROPVAR] 
(no examples found)
  \item[8] x1 count argrel/VAR $\rightarrow$ x1 count [on (that)SENTENCEVAR] \\
(He counted on that the train would be late; it-insertion follows later)
  \end{description}

\item[Remarks] No definition has been found yet for a sensible check on the 
possible patterns going with the prep introduced in this rule. For the moment,
it is assumed the prep takes more or less the same pattern as the verb, minus 
the prep itself.

One new verbpattern with a prep has been discovered until now:\\
x1 amount argrel/VAR $\rightarrow$ x1 amount [to (open-measurephrase)NPPROP)] \\
(His debts amount to over \$1,000)
\end{description}

\vspace{1 cm}
\begin{description}
\item[Name] TVerbPattern4
\item[Task] Specify the relation names and check the categories of the VP-arguments
of verbs with vp012 or vp123, with the second VP argument becoming an indirect
object preceding the other argument.

There still are mistakes in the mapping of Dutch and English verbpatterns in 
the dictionary with respect to the ordering of the VP arguments. The verb 
{\em persuade\/} e.g.\ has synIONP\_OPENTOSENT in English (the IONP is supposed 
to be the second VAR in the VP), but DONP\_OPENOMTESENT in Dutch (with the DONP 
as the first VAR in the VP). This will have to be checked thoroughly. Perhaps 
some of the example verbs below will be given a new pattern as a result of this 
check.

\item[File] english:TC\_VP3.mrule (mrules47.mrule)
\item[Semantics] --
\item[Examples] (ordered by subrule)
  \begin{description}
  \item[1a] x1 give argrel/VAR1 argrel/VAR2 $\rightarrow$ \\
x1 give indobjrel/NPVAR objrel/NPVAR \\
(I gave them a kiss)
  \item[1b] x1 give argrel/VAR1 argrel/VAR2 $\rightarrow$ \\
x1 give indobjrel/NPVAR objrel/CNVAR \\
(The kiss which I gave them)
  \item[1c] x1 give argrel/VAR1 argrel/VAR2 $\rightarrow$ \\
x1 give indobjrel/CNVAR objrel/NPVAR \\
(The people whom I gave a kiss)
  \item[1d] x1 give argrel/VAR1 argrel/VAR2 $\rightarrow$ \\
x1 give indobjrel/CNVAR objrel/CNVAR \\
(?)
  \item[2a] x1 allow argrel/VAR1 argrel/VAR2 $\rightarrow$ \\
x1 allow
indobjrel/EMPTYVAR objrel/NPVAR (only for verbs that have no paraphrase with a 
preposition; in case they may take a prep, they are covered by 
TVerbpattern5) \\
(We'll allow it this time)
  \item[2b] x1 allow argrel/VAR1 argrel/VAR2 $\rightarrow$ \\
x1 allow indobjrel/EMPTYVAR objrel/CNVAR \\
(The things which we will allow this time)
  \item[3a] x1 tell argrel/VAR1 argrel/VAR2 $\rightarrow$ \\
x1 tell indobjrel/NPVAR complrel/(that)SENTENCEVAR \\
(We told them that she was ill)
  \item[3b] x1 tell argrel/VAR1 argrel/VAR2 $\rightarrow$ \\
x1 tell indobjrel/CNVAR complrel/(that)SENTENCEVAR \\
(The man whom we told that she was ill)
  \item[4] x1 promise argrel/VAR1 argrel/VAR2 $\rightarrow$ \\
x1 promise indobjrel/EMPTYVAR complrel/(that)SENTENCEVAR \\
(We promised that we would do it)
  \item[5a] x1 tell argrel/VAR1 argrel/VAR2 $\rightarrow$ \\
x1 tell indobjrel/NPVAR complrel/(so)PROSENTVAR \\
(We told you so)
  \item[5b] x1 tell argrel/VAR1 argrel/VAR2 $\rightarrow$ \\
x1 tell indobjrel/CNVAR complrel/(so)PROSENTVAR \\
(The man whom we told so)
  \item[6a] x1 ask argrel/VAR1 argrel/VAR2 $\rightarrow$ \\
x1 ask indobjrel/NPVAR complrel/(q)SENTENCEVAR \\
(They asked a servant where the entry was)
  \item[6b] x1 ask argrel/VAR1 argrel/VAR2 $\rightarrow$ \\
x1 ask indobjrel/CNVAR complrel/(q)SENTENCEVAR \\
(The man whom they asked where the entry was)
  \item[7a] x1 persuade argrel/VAR1 argrel/VAR2 $\rightarrow$ \\
x1 persuade indobjrel/NPVAR complrel/(opentoinf)SENTENCEVAR \\
(They persuaded him to stay)
  \item[7b] x1 persuade argrel/VAR1 argrel/VAR2 $\rightarrow$ \\
x1 persuade indobjrel/CNVAR complrel/(opentoinf)SENTENCEVAR \\
(They people whom they had persuaded to stay)
  \item[8] x1 promise argrel/VAR1 argrel/VAR2 $\rightarrow$ \\
x1 promise indobjrel/EMPTYVAR complrel/(opentoinf)SENTENCEVAR 
(We promised to come back)
  \item[9a] x1 cost argrel/VAR1 argrel/VAR2 $\rightarrow$ \\
x1 cost indobjrel/NPVAR complrel/(open-measurephrase)NPPROPVAR
(It cost me a fortune)
  \item[9b] x1 cost argrel/VAR1 argrel/VAR2 $\rightarrow$ \\
x1 cost indobjrel/CNVAR complrel/(open-measurephrase)NPPROPVAR
(The fortune which it cost me)
  \item[10] x1 cost argrel/VAR1 argrel/VAR2 $\rightarrow$ \\
x1 cost indobjrel/EMPTYVAR complrel/(open-measurephrase)NPPROPVAR
(It cost a fortune)
  \item[11] x1 ask argrel/VAR1 argrel/VAR2 $\rightarrow$ \\
x1 ask indobjrel/EMPTYVAR complrel/(q)SENTENCEVAR \\
(We asked where the entrance was)
  \item[12a] x1 tell argrel/VAR1 argrel/VAR2 $\rightarrow$ \\
x1 tell indobjrel/NPVAR complrel/PROSENTVAR \\
(I told you)
  \item[12b] x1 tell argrel/VAR1 argrel/VAR2 $\rightarrow$ \\
x1 tell indobjrel/CNVAR complrel/PROSENTVAR \\
(Whom did you tell?)
  \item[13] x1 ask argrel/VAR1 argrel/VAR2 $\rightarrow$ \\
x1 ask indobjrel/EMPTYVAR complrel/PROSENTVAR \\
(I'll ask)
  \item[14] x1 allow argrel/VAR1 argrel/VAR2 $\rightarrow$ \\
x1 allow indobjrel/EMPTYVAR complrel/(opening)SENTENCEVAR \\
(I don't allow smoking here)
  \end{description}

\item[Remarks] As a new verbpattern (not added to the present rules yet), the 
following construction was discovered:\\
x1 say argrel/VAR1 argrel/VAR2 $\rightarrow$ \\
x1 say indobjrel/EMPTYVAR complrel/(so)PROSENTVAR \\
(I said so)
\end{description}

\vspace{1 cm}
\begin{description}
\item[Name] TVerbPattern5
\item[Task] Specify the relation names and check the categories of the 
VP-arguments
of verbs with vp012 and vp123, with the first VP argument receiving objrel or 
complrel and preceding the second argument.
\item[File] english:TC\_VP4.mrule (mrules46.mrule)
\item[Semantics] --
\item[Examples] (ordered by subrule)
  \begin{description}
  \item[1a] x1 force argrel/VAR1 argrel/VAR2 $\rightarrow$ \\
x1 force objrel/NPVAR complrel/(opentoinf)SENTENCEVAR \\
(We forced them to cooperate)
  \item[1b] x1 force argrel/VAR1 argrel/VAR2 $\rightarrow$ \\
x1 force objrel/CNVAR complrel/(opentoinf)SENTENCEVAR \\
(The man whom we forced to cooperate)
  \item[2a] x1 elect argrel/VAR1 argrel/VAR2 $\rightarrow$ \\
x1 elect objrel/NPVAR complrel/(open)NPPROPVAR \\
(They elected him President)
  \item[2b] x1 elect argrel/VAR1 argrel/VAR2 $\rightarrow$ \\
x1 elect objrel/CNVAR complrel/(open)NPPROPVAR \\
(The man whom they elected President)
  \item[3a] x1 paint argrel/VAR1 argrel/VAR2 $\rightarrow$ \\
x1 paint objrel/NPVAR complrel/(open)ADJPPROPVAR \\
(We painted the door green)
  \item[3b] x1 paint argrel/VAR1 argrel/VAR2 $\rightarrow$ \\
x1 paint objrel/CNVAR complrel/(open)ADJPPROPVAR \\
(The door which we painted green)
  \item[4a1] x1 put argrel/VAR1 argrel/VAR2 $\rightarrow$ \\
x1 put objrel/NPVAR locargrel/(locopen)PREPPPROPVAR \\
(I put the book on the table)
  \item[4b1] x1 put argrel/VAR1 argrel/VAR2 $\rightarrow$ \\
x1 put objrel/CNVAR locargrel/(locopen)PREPPPROPVAR \\
(The book which I put on the table)
  \item[4a2] ? meet argrel/VAR1 argrel/VAR2 $\rightarrow$ \\
meet objrel/NPVAR locargrel/(locopen)ADVPPROPVAR \\
(The cars met head-on)
  \item[4b2] ? meet argrel/VAR1 argrel/VAR2 $\rightarrow$ \\
meet objrel/CNVAR locargrel/(locopen)ADVPPROPVAR \\
(The cars that met head-on)
  \item[5a1] x1 drive argrel/VAR1 argrel/VAR2 $\rightarrow$ \\
x1 drive objrel/NPVAR dirargrel/(diropen)PREPPPROPVAR \\
(He drove the car into the yard)
  \item[5b1] x1 drive argrel/VAR1 argrel/VAR2 $\rightarrow$ \\
x1 drive objrel/CNVAR dirargrel/(diropen)PREPPPROPVAR \\
(The car which he drove into the yard)
  \item[5a2] ? x1 send argrel/VAR1 argrel/VAR2 $\rightarrow$ \\
x1 send objrel/NPVAR dirargrel/(diropen)ADVPPROPVAR \\
(The sent the boy home)
  \item[5b2] ? x1 send argrel/VAR1 argrel/VAR2 $\rightarrow$ \\
x1 send    objrel/CNVAR dirargrel/(diropen)ADVPPROPVAR \\
(The boy whom they sent home)
  \item[6a] x1 tell argrel/VAR1 argrel/VAR2 $\rightarrow$ \\
x1 tell    objrel/NPVAR prepobjrel/EMPTYVAR \\
(He told the whole story)
  \item[6b] x1 tell argrel/VAR1 argrel/VAR2 $\rightarrow$ \\
x1 tell    objrel/CNVAR prepobjrel/EMPTYVAR \\
(The story which he told)
  \item[7] seem argrel/VAR1 argrel/VAR2 $\rightarrow$ \\
seem    complrel/(closed)ADJPPROPVAR prepobjrel/EMPTYVAR \\
(He seemed ill)
  \item[8] seem argrel/VAR1 argrel/VAR2 $\rightarrow$ \\
seem    complrel/(closed)NPPROPVAR prepobjrel/EMPTYVAR \\
(He seemed a fraud)
  \item[9] seem argrel/VAR1 argrel/VAR2 $\rightarrow$ \\
seem    complrel/(otherclosed)PREPPPROPVAR prepobjrel/EMPTYVAR
(He seemed against the proposal)
  \item[10a] x1 tear argrel/VAR1 argrel/VAR2 $\rightarrow$ \\
x1 tear    objrel/NPVAR complrel/(otheropen)PREPPPROPVAR \\
(He tore the letter to pieces)
  \item[10b] x1 tear argrel/VAR1 argrel/VAR2 $\rightarrow$ \\
x1 tear     objrel/CNVAR complrel/(otheropen)PREPPPROPVAR \\
(The letter which he tore to pieces)
  \item[11] say argrel/VAR1 argrel/VAR2 $\rightarrow$ \\
say    complrel/SENTENCEVAR locargrel/(locopen)PREPPPROPVAR 
(It said in the paper that the war is over)
  \end{description}
\item[Remarks]
\end{description}

\vspace{1 cm}
\begin{description}
\item[Name] TVerbPattern6
\item[Task] Specify the relation names and check the categories of the VP-arguments 
of verbs with vp012 and vp123, when the second VP argument (VAR2) must be realised 
as a full prepositional 
phrase and must precede the other argument in the VP (cf.\ patterns 8, 9 and 10)
\item[File] english:TC\_VP4.mrule (mrules46.mrule)
\item[Semantics] --
\item[Examples] (ordered by subrule)
  \begin{description}
  \item[1a] x1 mention argrel/VAR1 argrel/VAR2 $\rightarrow$ \\
x1 mention     prepobjrel/[to NPVAR] complrel/(that)SENTENCEVAR
(He mentioned to me that he would be back early)
  \item[1b] x1 mention argrel/VAR1 argrel/VAR2 $\rightarrow$ \\
x1 mention     prepobjrel/[to CNVAR] complrel/(that)SENTENCEVAR 
(The girl to whom he mentioned that he would be back early)
  \item[2a] x1 require argrel/VAR1 argrel/VAR2 $\rightarrow$ \\
x1 require    prepobjrel/[of NPVAR] complrel/(q)SENTENCEVAR \\
(I required of him what he wanted)
  \item[2b] x1 require argrel/VAR1 argrel/VAR2 $\rightarrow$ \\
x1 require    prepobjrel/[of CNVAR] complrel/(q)SENTENCEVAR \\
(The man of whom he inquired what he wanted)
  \item[3a] x1 sign argrel/VAR1 argrel/VAR2 $\rightarrow$ \\
x1 sign    prepobjrel/[to NPVAR] complrel/(opentoinf)SENTENCEVAR
(The policeman signed to me to stop)
  \item[3b] x1 sign argrel/VAR1 argrel/VAR2 $\rightarrow$ \\
x1 sign    prepobjrel/[to CNVAR] complrel/(opentoinf)SENTENCEVAR
(The man to whom the policeman signed to stop)
  \item[4a] x1 talk argrel/VAR1 argrel/VAR2 $\rightarrow$ \\
x1 talk    prep(1)objrel/[to NPVAR] prep(2)objrel/EMPTYVAR \\
(He talked to his father)
  \item[4b] x1 talk argrel/VAR1 argrel/VAR2 $\rightarrow$ \\
x1 talk    prep(1)objrel/[to CNVAR] prep(2)objrel/EMPTYVAR \\
(The guy whom he talked to)
  \item[5a] seem argrel/VAR1 argrel/VAR2 $\rightarrow$ \\
seem     prepobjrel/[to NPVAR] complrel/(closedtoinf)SENTENCEVAR
(He seems to me to be ill)
  \item[5b] seem argrel/VAR1 argrel/VAR2 $\rightarrow$ \\
seem     prepobjrel/[to CNVAR] complrel/(closedtoinf)SENTENCEVAR
(The doctor to whom he seemed to be ill)
  \end{description}
\item[Remarks] 
\end{description}

\vspace{1 cm}
\begin{description}
\item[Name] TVerbPattern7
\item[Task] Specify the relation names and check the categories of the VP-arguments 
of verbs with vp012 and vp123, in case both arguments are full prepositional 
phrases.
\item[File] english:TC\_VP5.mrule (mrules45.mrule)
\item[Semantics] --
\item[Examples] (ordered by subrule)
  \begin{description}
  \item[1a] x1 talk argrel/VAR1 argrel/VAR2 $\rightarrow$ \\
x1 talk    prep(1)objrel/NPVAR prep(2)objrel/NPVAR \\
(I talked to my father about his work)
  \item[1b] x1 talk argrel/VAR1 argrel/VAR2 $\rightarrow$ \\
x1 talk    prep(1)objrel/NPVAR prep(2)objrel/CNVAR \\
(The work about which I talked to my father)
  \item[1c] x1 talk argrel/VAR1 argrel/VAR2 $\rightarrow$ \\
x1 talk    prep(1)objrel/CNVAR prep(2)objrel/NPVAR \\
(The man to whom I talked about his work)
  \item[1d] x1 talk argrel/VAR1 argrel/VAR2 $\rightarrow$ \\
x1 talk    prep(1)objrel/CNVAR prep(2)objrel/CNVAR \\
(?)
  \end{description}
\item[Remarks]
\end{description}

\vspace{1 cm}
\begin{description}
\item[Name] TVerbPattern8
\item[Task] Specify the relation names and check the categories of the VP-arguments 
of verbs with vp012 and vp123, when the second VP argument (VAR2) must be realised 
as a full prepositional 
phrase and must follow the other argument in the VP (cf.\ patterns 6, 9 and 10) 
\item[File] english:TC\_VP5.mrule (mrules45.mrule)
\item[Semantics] --
\item[Examples] (ordered by subrule)
  \begin{description}
  \item[1a] x1 regard argrel/VAR1 argrel/VAR2 $\rightarrow$ \\
x1 regard    objrel/NPVAR prepobjrel/[as (open)NPPROPVAR] \\
(They regard her as a friend)
  \item[1b] x1 regard argrel/VAR1 argrel/VAR2 $\rightarrow$ \\
x1 regard    objrel/CNVAR prepobjrel/[as (open)NPPROPVAR] \\
(The girl whom they regard as a friend)
  \item[2a] x1 change argrel/VAR1 argrel/VAR2 $\rightarrow$ \\
x1 change    objrel/NPVAR prepobjrel/[into (open)ADJPPROPVAR] \\
(We changed the main colour from brown to red)
  \item[2b] x1 change argrel/VAR1 argrel/VAR2 $\rightarrow$ \\
x1 change    objrel/CNVAR prepobjrel/[into (open)ADJPPROPVAR] \\
(The colour which we changed from brown to red)
  \item[3a] x1 give argrel/VAR1 argrel/VAR2 $\rightarrow$ \\
x1 give    objrel/NPVAR prepobjrel/[to NPVAR] \\
(I gave the book to John)
  \item[3b] x1 give argrel/VAR1 argrel/VAR2 $\rightarrow$ \\
x1 give    objrel/CNVAR prepobjrel/[to NPVAR] \\
(The book which I gave to John)
  \item[3c] x1 give argrel/VAR1 argrel/VAR2 $\rightarrow$ \\
x1 give    objrel/NPVAR prepobjrel/[to CNVAR] \\
(The man to whom I gave the book)
  \item[3d] x1 give argrel/VAR1 argrel/VAR2 $\rightarrow$ \\
x1 give    objrel/CNVAR prepobjrel/[to CNVAR] \\
(?)
  \item[4a] x1 talk (out) argrel/VAR1 argrel/VAR2 $\rightarrow$ \\
x1 talk (out)    objrel/NPVAR prepobjrel/[of (openIng)NPVAR] \\
(They talked him out of doing it)
  \item[4b] x1 talk (out) argrel/VAR1 argrel/VAR2 $\rightarrow$ \\
x1 talk (out)    objrel/CNVAR prepobjrel/[of (openIng)NPVAR] \\
(? The man whom they talked out of drowning himself)
  \item[5a] seem argrel/VAR1 argrel/VAR2 $\rightarrow$ \\
seem    complrel/(closed)ADJPPROPVAR prepobjrel/[to NPVAR] \\
(He seemed ill to me)
  \item[5b] seem argrel/VAR1 argrel/VAR2 $\rightarrow$ \\
seem    complrel/(closed)ADJPPROPVAR prepobjrel/[to CNVAR] \\
(The man to whom he seemed ill)
  \item[6a] seem argrel/VAR1 argrel/VAR2 $\rightarrow$ \\
seem    complrel/(closed)NPPROPVAR prepobjrel/[to NPVAR] \\
(He seemed a fool to me)
  \item[6b] seem argrel/VAR1 argrel/VAR2 $\rightarrow$ \\
seem    complrel/(closed)NPPROPVAR prepobjrel/[to CNVAR] \\
(The man to whom he seemed a fool)
  \item[7a] seem argrel/VAR1 argrel/VAR2 $\rightarrow$ \\
seem    complrel/(otherclosed)PREPPPROPVAR prepobjrel/[to NPVAR] 
(He seemed against the analysis to me)
  \item[7b] seem argrel/VAR1 argrel/VAR2 $\rightarrow$ \\
seem    complrel/(otherclosed)PREPPPROPVAR prepobjrel/[to CNVAR] 
(The man to whom he seemed against the analysis)
  \item[8a] x1 regard argrel/VAR1 argrel/VAR2 $\rightarrow$ \\
x1 regard    objrel/NPVAR prepobjrel/[as (otheropen)PREPPPROPVAR]
(I regard him as without principles)
  \item[8b] x1 regard argrel/VAR1 argrel/VAR2 $\rightarrow$ \\
x1 regard    objrel/CNVAR prepobjrel/[as (otheropen)PREPPPROPVAR]
(The man whom I regard as without principles)
  \end{description}
\item[Remarks]
\end{description}

\vspace{1 cm}
\begin{description}
\item[Name] TVerbPattern9
\item[Task] Specify the relation names and check the categories of the VP-arguments 
of verbs with vp012 and vp123, when the first VP argument (VAR1) must be realised 
as a full prepositional 
phrase and must precede the other argument in the VP (cf.\ patterns 6, 8 and 10)
\item[File] english:TC\_VP5.mrule (mrules45.mrule)
\item[Semantics] --
\item[Examples] (ordered by subrule)
   \begin{description}
  \item[1a] strike argrel/VAR1 argrel/VAR2 $\rightarrow$ \\
strike    indobjrel/NPVAR prepobjrel/[as (closed)ADJPPROPVAR] \\
(He struck me as pompous)
  \item[1b] strike argrel/VAR1 argrel/VAR2 $\rightarrow$ \\
strike    indobjrel/CNVAR prepobjrel/[as (closed)ADJPPROPVAR] \\
(The man whom he struck as pompous)
  \item[2a] x1 ask argrel/VAR1 argrel/VAR2 $\rightarrow$ \\
x1 ask    indobjrel/NPVAR prepobjrel/[about NPVAR] \\
(I'll ask him about it)
  \item[2b] x1 ask argrel/VAR1 argrel/VAR2 $\rightarrow$ \\
x1 ask    indobjrel/NPVAR prepobjrel/[about CNVAR] \\
(The book which I asked him about)
  \item[2c] x1 ask argrel/VAR1 argrel/VAR2 $\rightarrow$ \\
x1 ask    indobjrel/CNVAR prepobjrel/[about NPVAR] \\
(The man whom I asked about it)
  \item[2d] x1 ask argrel/VAR1 argrel/VAR2 $\rightarrow$ \\
x1 ask    indobjrel/CNVAR prepobjrel/[about CNVAR] \\
(?)
   \end{description}
\item[Remarks]
\end{description}

\vspace{1 cm}
\begin{description}
\item[Name] TVerbPattern10
\item[Task] Specify the relation names and check the categories of the VP-arguments 
of verbs with vp012 and vp123, when the first VP argument (VAR1) must be realised 
as a full prepositional 
phrase and must follow the other argument in the VP (cf.\ patterns 6, 8 and 9)
\item[File] english:TC\_VP5.mrule (mrules45.mrule)
\item[Semantics] --
\item[Examples] (ordered by subrule)
   \begin{description}
  \item[1a] x1 talk argrel/VAR1 argrel/VAR2 $\rightarrow$ \\
x1 talk    prep(1)objrel/EMPTYVAR prep(2)objrel/[about NPVAR] \\
(He talked about a new car)
  \item[1b] x1 talk argrel/VAR1 argrel/VAR2 $\rightarrow$ \\
x1 talk    prep(1)objrel/EMPTYVAR prep(2)objrel/[about CNVAR] \\
(The new car which he talked about)
   \end{description}
\item[Remarks] In the current implementation, the verbpattern going with this 
rule is simply {\em synEMPTY\_PREPNP\/}. This is not enough, since there is an 
extra demand, viz.\ that there are two prepkeys for the verb (prepkey2 is the 
one that is realised). A sentence like {\em We asked EMPTY for a break\/} 
would also have to be made by the same pattern, and there is only one prepkey 
for the verb {\em ask\/}. Therefore, the example with {\em talk\/} given here 
will have to be renamed into something like {\em synEMPTY\_PREP2NP\/}, and the 
appropriate rule for the `one prep case' has to be added to the set of 
verbpatterns.

It must also be checked whether verbs 
having the pattern synEMPTY\_PREPNP really always expect the correct VAR in the 
VP to be EMPTY, for the same reasons as were stated in TVerbpattern4. In case 
they expect the `wrong' VAR to be EMPTY, the thetavp of the verb may be 
changed.

\end{description}

\vspace{1 cm}
\begin{description}
\item[Name] TIdVerbPattern
\item[Task] Vacuous transformation to let idioms get through this 
transformation class
\item[File] english:TC\_VP3.mrule (mrules48.mrule)
\item[Semantics] --
\item[Example] all idioms
\item[Remarks]
\end{description}

\end{description}

\newpage
\subsection{TC\_ParticleSpelling}

\begin{description}
\item[Kind] Obligatory Transformation Class
\item[Task] Find out if a verb needs a particle, and if so, to spell it out.

\vspace{1 cm}
\begin{description}
\item[Name] TNoParticleInsertion
\item[Task] To let verbs that do not take a particle pass this transformation 
class
\item[File] english:TC\_ParticleSpelling.mrule (mrules44.mrule)
\item[Semantics] --
\item[Example] buy $\rightarrow$ buy
\item[Remarks]
\end{description}

\vspace{1 cm}
\begin{description}
\item[Name] TParticleInsertion
\item[Task] To spell out the particle for verbs that take one (as specified in 
the attribute {\bf .particle}).
\item[File] english:TC\_ParticleSpelling.mrule (mrules44.mrule)
\item[Semantics] --
\item[Example] turn $\rightarrow$ turn on
\item[Remarks]
\end{description}

\vspace{1 cm}
\item[Remark] Contrary to what was said in doc.\ 153, all particles are put 
directly behind the verb, even if there is a simple NP argument. The full 
burden of accounting for the surface position of the particle is placed on the 
Particle Hop Transformations (at the end of the next subgrammar) rather than 
distributed over two transformation classes. 
\end{description}

\newpage
\subsection{RC\_TempAdvVar}

\begin{description}
\item[Kind] Iterative Rule Class
\item[Task] to introduce variables for time adverbials. The variable may be for 
a sentence, a prepp, or an advp (perhaps nps will be added too). The rule class is 
iterative, but the formulation of the rules is such that within the rule class 
there is only one 
ordering of var-introduction possible for the different adverbials in analysis.
 This is to 
prevent unnecessary ambiguities\footnote{At the time this document 
was approved, the three advvar classes (tempadv, locadv, and sentadv) had been 
merged again (cf. section 2). The rules are 
written in such a way that in generation, every order of insertion (loc - temp 
- sent) is possible, but in analysis tempadvvars are removed first.}. In 
English, the surface order of tempadvs is the reverse of Dutch.

\vspace{1 cm}
\begin{description}
\item[Name] RrefvarInsertion
\item[Task] To introduce a variable for a referential time adverbial that is 
not retrospective
\item[File] english:RC\_TempVar.mrule (mrules36.mrule)
\item[Semantics] 
\item[Example] x1 work $\rightarrow$ x1 work refVAR (He worked yesterday)
\item[Remarks] No decision has been taken yet how disjunct complex referential 
expressions must be introduced (Yesterday, he arrived at three o'clock)
\end{description}

\vspace{1 cm}
\begin{description}
\item[Name] RdurvarInsertion
\item[Task] To introduce a variable for a durational time adverbial 
\item[File] english:RC\_TempVar.mrule (mrules36.mrule)
\item[Semantics] 
\item[Example] x1 work $\rightarrow$ x1 work durVAR (He worked for three hours)
\item[Remarks]
\end{description}

\vspace{1 cm}
\begin{description}
\item[Name] RretrovarInsertion
\item[Task] To introduce a variable for a referential time adverbial that is 
retrospective
\item[File] english:RC\_TempVar.mrule (mrules36.mrule)
\item[Semantics] 
\item[Example] x1 work $\rightarrow$ x1 work retroVAR (He has worked for three 
hours)
\item[Remarks]
\end{description}

\vspace{1 cm}
\begin{description}
\item[Name] RfreqvarInsertion
\item[Task] To introduce a variable for a frequential time adverbial 
\item[File] english:RC\_TempVar.mrule (mrules36.mrule)
\item[Semantics] 
\item[Example] x1 work $\rightarrow$ x1 work freqVAR (He works every day)
\item[Remarks] This rule has NOT been added to the control expression and the 
transfer yet!
\end{description}

\end{description}

\newpage
\subsection{RC\_SentAdvVar}

\begin{description}
\item[Kind] Iterative Rule class
\item[Task] To introduce a variable for adverbial subordinate sentences and 
prepps in 
different positions and sentence or causal adverbs in 
initial position. The conjunction may also be a preposition. No rules have 
been written yet for abstract conjunctions. 

The rules are in an iterative rule class, but have been written in such a way 
that in analysis only one order of application is possible. This to prevent 
unnecessary ambiguities\footnote{At the time this document 
was approved, the three advvar classes (tempadv, locadv, and sentadv) had been 
merged again (cf. section 2). The rules are 
written in such a way that in generation, every order of insertion (loc - temp 
- sent) is possible, but in analysis sentadvvars and conjsentvars are removed 
last.}. In generation, the same surface order of sentadvs and conjsents should 
occur as in the source language, esp.\ since it might reflect on the reference 
of pronominal expressions ({\em The man$_{i}$ cried because he$_{i}$ had killed
her - Because he$_{i,j}$ had killed her, the man$_{i}$ cried\/}). 
However, since Dutch has a `middle' conjsent 
while English does not, this is not always possible. In the current rule class, 
all rules (covering all positions) are mapped onto all relevant rules of IL. 
In the proposition substitution rules, however, there is a preference for 
sentences of the same position to be mapped onto each other (see the 
XPPROPtoCLAUSE subgrammar, RConjSentSubst, RFinalConjSentSubst, etc.).

\vspace{1 cm}
\begin{description}
\item[Name] RConjSentVar
\item[Task] To introduce a variable for an adverbial subordinate sentence or 
sentential prepp in initial position.
\item[File] english:RC\_AdvvarRules.mrule (mrules91.mrule)
\item[Semantics]
\item[Example] x1 leave $\rightarrow$ sentPREPPVAR x1 leave (Without having 
warned us, he left)\\
x1 put on x2 $\rightarrow$ advSENTENCEVAR x1 put on x2 (In order to read the 
letter, she put on her glasses)
\item[Remarks] The comma that is probably obligatory after the subordinate 
sentence is not added here, but in the proposition substitution rules (see 
the XPPROPtoCLAUSE subgrammar, RConjSentSubst and RConjPrepNPSubst).

Although the rule class is iterative, there may be only one leftdislocrel. See 
RSentAdvVar below for a possible problem.
\end{description}

\vspace{1 cm}
\begin{description}
\item[Name] RFinalConjSentVar
\item[Task] To introduce a variable for an adverbial subordinate sentence or 
sentential PREPP in final position.
\item[File] english:RC\_AdvvarRules.mrule (mrules91.mrule)
\item[Semantics]
\item[Example] x1 leave $\rightarrow$ x1 leave sentPREPPVAR (He left without 
having warned us)\\
x1 put on x2 $\rightarrow$ x1 put on x2 advSENTENCEVAR (She put on her glasses
in order to read the letter)
\item[Remarks] The comma that may be present in analysis between the main 
sentence and the subordinate sentence is not accounted for here, but 
in the proposition substitution rules (see 
the XPPROPtoCLAUSE subgrammar, RFinalConjSentSubst and RFinalConjPrepNPSubst).

No precautions have been taken to check the position of the subordinate 
sentence relative to a temporal sentence also present in the VERBPPROP. Given 
the current formulation of the rules, temporal sentences (like tempadvs) always 
precede other final sentences.
\end{description}

\vspace{1 cm}
\begin{description}
\item[Name] RSentadvVar
\item[Task] To introduce a variable for a causal or sentence advp in 
initial position. 
\item[File] english:RC\_AdvvarRules.mrule (mrules91.mrule)
\item[Semantics]
\item[Example] x1 leave $\rightarrow$ sentADVPVAR x1 leave (Probably, he left);
\\
x1 leave $\rightarrow$ causADVPVAR x1 leave (Therefore, he left)
\item[Remarks] No rules have been written yet to account for other positions 
of the adverbial; this may cause problems when there already is a sentence in 
initial position, since there may be only one leftdislocrel. 
In that case, the adverb should be put elsewhere, in 
accordance with the values of its {\bf position} attribute. This has not been 
done yet.

The comma that is probably obligatory after the adverbial
is not added here, but in the substitution rules (see 
the CLAUSEtoSENTENCE subgrammar, RSentAdvSubst).
\end{description}

\vspace{1 cm}
\begin{description}
\item[Name] RSentpreppVar
\item[Task] To introduce a variable for a prepositional causal or 
sentence adverbial in initial position. 
\item[File] english:RC\_AdvvarRules.mrule (mrules91.mrule)
\item[Semantics]
\item[Example] x1 leave $\rightarrow$ sentPREPPVAR x1 leave (For that reason, 
he left)
\item[Remarks] No rules have been written yet to account for other positions 
of the prepp; this may cause problems when there already is a sentence in 
initial position, since there may be only one leftdislocrel. 
In that case, the prepp should be put elsewhere. Probably, the only possibility 
is to put it sentence-finally (there is no {\bf position} attribute for prepps).
 This has not been done yet.

The comma that is probably obligatory after the prepp
is not added here, but in the substitution rules (see 
the CLAUSEtoSENTENCE subgrammar, RSentPreppSubst).
\end{description}

\end{description}

\newpage
\subsection{RC\_LocadvVar}

\begin{description}
\item[Kind] Iterative Rule Class
\item[Task] To introduce a variable for a non-argument locative. The rule class 
is iterative, but the rules have been written in such a way that only one order 
of application is possible in analysis\footnote{At the time this document 
was approved, the three advvar classes (tempadv, locadv, and sentadv) had been 
merged again (cf. section 2). The rules are 
written in such a way that in generation, every order of insertion (loc - temp 
- sent) is possible, but in analysis tempadvvars are removed first, and then 
locadvs.}. In generation, the same surface order of locadvs will be produced as 
existed in the source language.

\vspace{1 cm}
\begin{description}
\item[Name] RlocadvVar
\item[Task] To introduce a variable for a non-argument adverbial locative
\item[File] english:RC\_AdvvarRules.mrule (mrules91.mrule)
\item[Semantics]
\item[Example] x1 paint x2 $\rightarrow$ x1 paint x2 locADVPVAR (He painted it 
there)
\item[Remarks]
\end{description}

\vspace{1 cm}
\begin{description}
\item[Name] RlocpreppVar
\item[Task] To introduce a variable for a non-argument prepositional locative
\item[File] english:RC\_AdvvarRules.mrule (mrules91.mrule)
\item[Semantics]
\item[Example] x1 paint x2 $\rightarrow$ x1 paint x2 locPREPPVAR (He painted it 
at the club)
\item[Remarks]
\end{description}

\item[Remarks] No rules have been written in English to account for other 
surface poositions of locatives, e.g.\ in initial (leftdislocrel?) position. 
They still have to be included.
\end{description}

\newpage
\subsection{RC\_VoiceRules}

\begin{description}
\item[Kind] Obligatory Rule Class
\item[Task] To give active or passive voice to the prop-structure, and to 
change the subject in a by-object and the verb into a participle 
in case of a passive. The rules make use of a 
parameter, {\em voicepar\/}, so that a certain voice in one language is mapped 
onto both voices 
in another language with the original voice in the parameter. Then the 
restriction holds that the rule with the `different' voice may apply only if 
the rule with the same voice is impossible in the target language. E.g.\ when 
a Dutch 
sentence has an active voice, in English both voice rules are called, with the 
parameter `active'. The passive rule (with parameter `active') may apply only 
if the verb in question does not have an active form\footnote{By the time this 
document had been approved, another case was allowed, viz.\ when the verb does 
have an active form, but there is an EMPTY subject in the sentence. This is 
especially useful for the translation of a Dutch AanActive: {\em Ik liet het 
boek lezen\/} will then translate in {\em I let the book be read\/}. }. 
This is explained in more 
detail in doc.\ 315, {\em Voice in Dutch\/}, by Jan Odijk.

\vspace{1 cm}
\begin{description}
\item[Name] RActive
\item[Task] To put a sentence in active voice, either because the original 
sentence was active (in analysis and in generation; voice parameter = active)
or because the English verb used in the translation cannot form a passive 
(in generation only; voice parameter = passive)
\item[File] english:RC\_VoiceRules.mrule (mrules43.mrule)
\item[Semantics]
\item[Example] x1 sing x2 $\rightarrow$ x1 sing x2 (John sang an aria)
\item[Remarks]
\end{description}

\vspace{1 cm}
\begin{description}
\item[Name] RPassive1
\item[Task] To put a sentence in passive voice, either because the original 
sentence was passive (in analysis and in generation; voice parameter = passive)
or because the English verb used in the translation cannot form an active
(in generation only; voice parameter = active). The subject of the sentence, 
which is an NP/CNVAR or an EMPTYVAR, is put in a by-object. The first verb is 
changed into a participle. The auxiliary of the passive will be introduced 
later (in the clause-formation rules).
\item[File] english:RC\_VoiceRules.mrule (mrules43.mrule)
\item[Semantics]
\item[Example] x1 sing x2 $\rightarrow$ sung x2 by x1 (The aria was sung by John)
\item[Remarks]
\end{description}

\vspace{1 cm}
\begin{description}
\item[Name] RPassive2
\item[Task] To put a sentence in passive voice, either because the original 
sentence was passive (in analysis and in generation; voice parameter = passive)
or because the English verb used in the translation cannot form an active
(in generation only; voice parameter = active). The subject of the sentence, 
which is a wh or yesno SENTENCEVAR, is put in a by-object. The first verb is 
changed into a participle. The auxiliary of the passive will be introduced 
later (in the clause-formation rules).
\item[File] english:RC\_VoiceRules.mrule (mrules43.mrule)
\item[Semantics]
\item[Example] ? x1 puzzle x2 $\rightarrow$ puzzled x2 by x1 (I was 
puzzled by what to do next)\\
NB: this may be incorrect English: Longman only gives {\em I 
was puzzled (about) what to do next\/}, suggesting only that there is an 
adjective {\em puzzled (about)\/}, but not a verb.
\item[Remarks] Perhaps this rule is not needed: all examples found are either 
doubtful English or may well be simple NPs, like in {\em I was surprised by 
$_{NP}$[what you told me] \/}.
\end{description}

\vspace{1 cm}
\item[Remarks] \mbox{}
  \begin{itemize}
  \item The voice rules are ordered crucially before TC\_ReflexiveSpelling and 
TC\_\-PossAdjSpelling, because those transformation classes need a subject as 
antecedent. In case of a passive structure, the subject has been moved to a by-
phrase, and hence reflexives or other subject-bound anaphora cannot occur.
  \item No rules have been written to account for middle voice ({\em This book 
sells well\/}). Analysis of the middle voice construction is not conclusive 
enough to implement any well-defined strategy.
  \end{itemize}
\end{description}

\newpage
\subsection{TC\_ReflexiveSpelling}

\begin{description}
\item[Kind] Obligatory Transformation Class
\item[Task] To spell out the reflexive pronoun for verbs that are inherently 
reflexive. It is assumed that all reflexives in English take the subject of the 
verb as antecedent (cf.\ Dutch, were {\em zich voordoen\/} is considered an 
ergative, and hence takes the object as antecedent). The number, person and 
gender of the subject determine the actual form of the reflexive. Note what was 
said in RStartVPPROP000: the rules only deal with VAR subjects.

Also, it is assumed that 
English does not have any inherently reciprocal reflexives. Perhaps the 
attribute 
value {\bf reciprocal} may still be used in English, to indicate that a verb
may delete the `reciprocal' argument: {\em They met each other $\rightarrow$ 
they met\/} (again cf.\ Dutch, 
where the verb {\em ontmoeten\/} takes an obligatory reciprocal if no other 
argument is specified: {\em Zij ontmoetten elkaar\/}). Note that argument 
reflexives and reciprocals are not covered by 
this rule class. They are dealt with in TC\_ArgReflSpelling and 
RC\_ReciprocalSpelling, which are in the next subgrammar.

In English, a difference is made between direct object reflexives (the usual 
case) and indirect object reflexives. Usually, the reflexive is put in direct 
object position, but indirect object reflexives receive an indobjrel
(only example so far: {\em have\/}: they had themselves a ball). 


\vspace{1 cm}
\begin{description}
\item[Name] TNoReflInsertion
\item[Task] To let verbs which are not reflexive (or perhaps reciprocal) pass 
this transformation class
\item[File] english:TC\_ReflexiveSpelling.mrule (mrules42.mrule)
\item[Semantics] --
\item[Example] x1 swim $\rightarrow$ x1 swim
\item[Remarks]
\end{description}

\vspace{1 cm}
\begin{description}
\item[Name] TObjReflInsertion1
\item[Task] To spell out the reflexive for ordinary inherently reflexive verbs, 
when the subject antecedent is an NPVAR
\item[File] english:TC\_ReflexiveSpelling.mrule (mrules42.mrule)
\item[Semantics] --
\item[Example] x1 perjure $\rightarrow$ x1 perjure himself/yourselves/oneself 
etc. (He perjured himself)
\item[Remarks]
\end{description}

\vspace{1 cm}
\begin{description}
\item[Name] TObjReflInsertion2
\item[Task] To spell out the reflexive for ordinary inherently reflexive verbs, 
when the subject antecedent is a CNVAR
\item[File] english:TC\_ReflexiveSpelling.mrule (mrules42.mrule)
\item[Semantics] --
\item[Example] x1 perjure $\rightarrow$ x1 perjure himself/herself 
etc. (The man who perjured himself)
\item[Remarks]
\end{description}

\vspace{1 cm}
\begin{description}
\item[Name] TIndObjReflInsertion1
\item[Task] To spell out the reflexive for an inherently indirect object 
reflexive verb ({\em have\/}), when the subject antecedent is an NPVAR
\item[File] english:TC\_ReflexiveSpelling.mrule (mrules42.mrule)
\item[Semantics] --
\item[Example] x1 have x2 $\rightarrow$ x1 have himself/yourselves/oneself x2
etc. (They had themselves a ball)
\item[Remarks]
\end{description}

\vspace{1 cm}
\begin{description}
\item[Name] TIndObjReflInsertion2
\item[Task] To spell out the reflexive for an inherently indirect object 
reflexive verb ({\em have\/}), when the subject antecedent is a CNVAR
\item[File] english:TC\_ReflexiveSpelling.mrule (mrules42.mrule)
\item[Semantics] --
\item[Example] x1 have x2 $\rightarrow$ x1 have himself/itself x2
etc. (The people who had themselves a ball)
\item[Remarks]
\end{description}

\vspace{1 cm}
\item[Remarks] \mbox{}
  \begin{itemize}
  \item In doc.\ 150, it was assumed that there was one transformation class, 
TC\_BoundAnaphorCheck. In the current implementation, this one class has been 
divided into two parts, one for reflexives, and one for idiom possessives (see 
below)
  \item This transformation class is ordered crucially after the voice rules. 
Reflexives are allowed only if there still is a subject, so passive structures 
cannot have a reflexive. Also, the transformations must be ordered before 
object-to-subject raising.
  \end{itemize}
\end{description}

\newpage
\subsection{TC\_PossAdjSpelling}

\begin{description}
\item[Kind] Optional Transformation, followed by Obligatory Filter
\item[Task] Check whether there is an object NP with a determiner NPVAR/CNVAR 
which is identical to the subject NPVAR/CNVAR, and spell the determiner out 
as a full 
adjectival possessive pronoun (POSSADJ), taking the subject as the antecedent.
This will only occur in case the object NP has been introduced by an idiom rule
(non-idioms still have VARs as object).

\vspace{1 cm}
\begin{description}
\item[Name] TPossAdjSpelling1
\item[Task] To spell out a determiner NPVAR/CNVAR in an object NP 
(introduced by an idiom rule) as a full 
adjectival possessive pronoun (POSSADJ), taking the subject as the antecedent.
\item[File] english:TC\_ReflexiveSpelling.mrule (mrules42.mrule)
\item[Semantics] --
\item[Example] x1 lose x1 patience $\rightarrow$ x1 lose his/my/their patience
\item[Remarks]
\end{description}

\vspace{1 cm}
\begin{description}
\item[Name] FPossAdjSpelling1
\item[Task] To check that all determiner NPVAR/CNVARs in an object NP 
(introduced by an idiom rule) are realised as full 
adjectival possessive pronouns (POSSADJ) in generation.
\item[File] english:TC\_ReflexiveSpelling.mrule (mrules42.mrule)
\item[Example] --
\item[Remarks] This filter is needed because the transformation is optional.
\end{description}

\vspace{1 cm}
\item[Remarks] \mbox{}
  \begin{itemize}
  \item In doc.\ 150, it was assumed that there was one transformation class, 
TC\_BoundAnaphorCheck. In the current implementation, this one class has been 
divided in two parts, one for reflexives (see above), and one for idiom 
possessives.
  \item This transformation class is ordered crucially after the voice rules. 
Idiomatic PossAdjs are allowed only if there still is a subject, so passive 
structures 
cannot have a PossAdj. Also, the transformation must be ordered before 
object-to-subject raising.
  \end{itemize}
\end{description}

\newpage
\subsection{TC\_AktionsartCalc}

\begin{description}
\item[Kind] Obligatory Transformation Class
\item[Task] To calculate the value for the attribute {\em aktionsarts\/} of the 
VERBPPROP on basis of properties of the main VERB (attribute {\em classes\/})
and properties of the arguments in the VP. There are 13 rules:

\vspace{1 cm}
\begin{description}
\item[Name] taktactivity1
\item[Task] To assign the value {\em activity\/} to the Aktionsart attribute of 
the VERBPPROP. This rule is for non-ergative verbs with type 
{\em durativeclass}. No object or prepobject argument is allowed, unless it 
is an EMPTYVAR or part of an idiom.
In this rule a subject argument is allowed (but not needed) without 
restrictions.
\item[File] english:TC\_AktionsartCalc.mrule (mrules81.mrule)
\item[Semantics] --
\item[Example] x1 sleep, x1 work, x1 write, x1 walk, rain, x1 eat EMPTY
\item[Remarks]
\end{description}

\vspace{1 cm}
\begin{description}
\item[Name] taktactivity2
\item[Task] To assign the value {\em activity\/} to the Aktionsart attribute of 
the VERBPPROP. This rule is for verbs with type {\em movementclass}.
In this rule a subject, object, indobject  or prepobject argument are
allowed (but not needed) and there are no restrictions 
on properties of these arguments. A directional argument is not allowed except 
for directional arguments that are part of an idiom or EMPTY.
\item[File] english:TC\_AktionsartCalc.mrule (mrules81.mrule)
\item[Semantics] --
\item[Example] x1 push x2, x1 push, x1 go
\item[Remarks]
\end{description}

\vspace{1 cm}
\begin{description}
\item[Name] taktactivity3
\item[Task] To assign the value {\em activity\/} to the Aktionsart attribute of 
the VERBPPROP. This rule is for non-ergative verbs with type 
{\em durativeclass\/} 
and non-specified quantity object or prepobject arguments that 
are not CNVAR or EMPTYVAR nor part of an idiom.
In this rule a subject argument is allowed (but not needed) without
restrictions.
\item[File] english:TC\_AktionsartCalc.mrule (mrules81.mrule)
\item[Semantics] --
\item[Example] x1 eat sandwiches, x1 write letters, x1 build houses, 
x1 play music
\item[Remarks]
\end{description}

\vspace{1 cm}
\begin{description}
\item[Name] taktactivity4
\item[Task] To assign the value {\em activity\/} to the Aktionsart attribute of 
the VERBPPROP. This rule is for verbs with type {\em movementclass}.
 In this rule a subject, object, indobject  or prepobject argument are
allowed (but not needed) and there are no restrictions 
on properties of these arguments. A non-specified quantity directional argument
must be present that is not EMPTY or part of an idiom.
\item[File] english:TC\_AktionsartCalc.mrule (mrules81.mrule)
\item[Semantics] --
\item[Example] x1 push x2 to houses, x1 goes to clients(?)
\item[Remarks]
\end{description}

\vspace{1 cm}
\begin{description}
\item[Name] taktactivity5
\item[Task] To assign the value {\em activity\/} to the Aktionsart attribute of 
the VERBPPROP. This rule is for non-ergative (3-place) verbs with type {\em 
movementclass\/} and a non-specified quantity object or prepobject argument 
that is not part of an idiom or EMPTY or CNVAR,
and a specified quantity directional argument.
In this rule a subject argument is allowed (but not needed) without
restrictions.
\item[File] english:TC\_AktionsartCalc.mrule (mrules81.mrule)
\item[Semantics] --
\item[Example] x1 push carts to the house
\item[Remarks]
\end{description}

\vspace{1 cm}
\begin{description}
\item[Name] taktactivity6
\item[Task] To assign the value {\em activity\/} to the Aktionsart attribute of 
the VERBPPROP. This rule is for ergative verbs with type {\em durativeclass} or 
verbs with type {\em iterativeclass}.
In this rule a subject argument is allowed (but not needed) without 
restrictions.
\item[File] english:TC\_AktionsartCalc.mrule (mrules81.mrule)
\item[Semantics] --
\item[Example] boil x1, x1 hit, x1 knock
\item[Remarks]
\end{description}

\vspace{1 cm}
\begin{description}
\item[Name] taktaccomplishment1
\item[Task] To assign the value {\em accomplishment\/} to the Aktionsart 
attribute of the 
VERBPPROP. This rule is for non-ergative verbs with type {\em 
durativeclass\/} and specified quantity object or prepobject arguments 
(including CN) that are not EMPTY or part of an idiom.
In this rule a subject argument is allowed (but not needed) without
restrictions.
\item[File] english:TC\_AktionsartCalc.mrule (mrules81.mrule)
\item[Semantics] --
\item[Example] x1 eat an apple, x1 write a letter, x1 build thirty houses
\item[Remarks]
\end{description}

\vspace{1 cm}
\begin{description}
\item[Name] taktaccomplishment2
\item[Task] To assign the value {\em accomplishment\/} to the Aktionsart 
attribute of the 
VERBPPROP. This rule is for (non-ergative) 3-place verbs with type {\em 
movementclass\/} and specified quantity object or prepobject arguments that 
are not EMPTY or part of an idiom, and a specified quantity directional 
argument.
There are no restrictions on the subject.
\item[File] english:TC\_AktionsartCalc.mrule (mrules81.mrule)
\item[Semantics] --
\item[Example] x1 push the cart to the house
\item[Remarks]
\end{description}

\vspace{1 cm}
\begin{description}
\item[Name] taktaccomplishment3
\item[Task] To assign the value {\em accomplishment\/} to the Aktionsart 
attribute of the 
VERBPPROP. This rule is for 2-place verbs with type {\em movementclass}.
In this rule subject, object, indobject  or prepobject arguments are
allowed (but not needed) and there are no restrictions 
on properties of these arguments. A specified quantity directional argument
must be present that is not EMPTY or part of an idiom.
\item[File] english:TC\_AktionsartCalc.mrule (mrules81.mrule)
\item[Semantics] --
\item[Example] x1 go to Groningen
\item[Remarks]
\end{description}

\vspace{1 cm}
\begin{description}
\item[Name] taktachievement1
\item[Task] To assign the value {\em achievement\/} to the Aktionsart attribute 
of the VERBPPROP. This rule is for verbs with type {\em momentaryclass}.
In this rule arguments are allowed (but not needed) without restrictions.
\item[File] english:TC\_AktionsartCalc.mrule (mrules81.mrule)
\item[Semantics] --
\item[Example] x1 die, x1 reach the top, x1 explode
\item[Remarks]
\end{description}

\vspace{1 cm}
\begin{description}
\item[Name] taktstative1
\item[Task] To assign the value {\em stative\/} to the Aktionsart attribute of 
the VERBPPROP. This rule is for verbs with type {\em stativeclass}.
In this rule arguments are allowed (but not needed) without restrictions.
\item[File] english:TC\_AktionsartCalc.mrule (mrules81.mrule)
\item[Semantics] --
\item[Example] x1 love x2, x1 possess x2, x1 be x2
\item[Remarks]
\end{description}

\vspace{1 cm}
\begin{description}
\item[Name] taktstative2
\item[Task] To assign the value {\em stative\/} to the Aktionsart attribute of 
the VERBPPROP. This rule is for verbs with type {\em dynstativeclass}.
In this rule arguments are allowed (but not needed) without restrictions.
\item[File] english:TC\_AktionsartCalc.mrule (mrules81.mrule)
\item[Semantics] --
\item[Example] x1 lie, x1 sit
\item[Remarks]
\end{description}

\vspace{1 cm}
\item[Remark] In doc.\ 155, it was assumed that Aktionsart calculation could be 
done in the Startrules. However, the rules are complex, and crucially need 
information about the kind of arguments the verb takes, not just about their 
number. Therefore, the calculation had to be postponed to at least after the 
pattern rules and the introduction of reflexives. More information on the 
subject can be found in docs.\ 53, {\em Translation of Temporal Expressions in 
Rosetta3\/}, and 263, {\em Documentation on the rules for the translation of 
temporal expressions in Rosetta3, part I\/}, both by Lisette Appelo.
\end{description}

\end{document}
