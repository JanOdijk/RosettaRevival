\documentstyle{Rosetta}
\begin{document}
   \RosTopic{General}
   \RosTitle{Notulen Groepsvergadering 19-2-1990}
   \RosAuthor{Harm Smit}
   \RosDocNr{428}
   \RosDate{\today}
   \RosStatus{approved}
   \RosSupersedes{-}
   \RosDistribution{Project}
   \RosClearance{Project}
   \RosKeywords{minutes}
   \MakeRosTitle
\hyphenation{schrij-ven}
\begin{itemize}
  \item {\bf aanwezig}: Andr\'{e} Schenk, Jan Landsbergen, Lisette Appelo,
                     Franciska de Jong, Petra de Wit, Elly van Munster, 
                     Elena Pinillos, Joep Rous, Josien Willems, 
                     Harold Leurs, Jan Odijk, Harm Smit.
  \item {\bf afwezig}: Ren\'{e} Leermakers.
  \item {\bf Agenda}:
    \begin{enumerate}
       \item Opening en notulen
       \item Personele zaken
       \item Boek
       \item Diversen extern
       \item Diversen intern
       \item Rosetta 3D
       \item Rondvraag
    \end{enumerate}
\end{itemize}

\section {Opening en notulen}
De notulen van de vorige vergadering werden zonder wijzigingen aangenomen.
Verder heet Jan L. Petra de Wit, alsmede de stagiairs Harold Leurs en
         Josien Willems welkom.

\section {Personele zaken}
\begin{enumerate}
   \item Elly en Elena zullen maandag 26 februari verhuizen naar kamer 337;
         verder kunnen we wellicht binnenkort de beschikking krijgen over een 
         grote kamer voor demonstraties, vergaderingen en (in geval van
         ruimtegebrek) stagiairs. Verder zal Joep waarschijnlijk zijn kamer
         voor kamer 331 kunnen ruilen.
   \item Op 1 maart komt Frank Uittenbogaard; hij zal bij Lisette op de
         kamer komen. Harold zal dan voor de rest van zijn stage elders een 
         plek moeten krijgen, waarschijnlijk bij Joep. Frank zal moeten worden
         ingewerkt; dit zullen Jan L., Ren\'{e}, Joep en Lisette regelen.
   \item Per 1 mei komt er een nieuwe stagiair, Ids Dijkstra van de TU Delft.
         Hij zal waarschijnlijk aan de semantische component (statistische
         methode) gaan werken en zal 9 maanden blijven.

         Er is waarschijnlijk in de toekomst nog wel ruimte voor een korte
         stage. Wie een onderwerp weet moet dit bij Jan L. melden; Jan
         heeft een lijst met potenti\"{e}le kandidaten.

   \item De CSO-cursussen voor Andr\'{e}, Franciska en Harm kunnen doorgang 
         vinden. Zij staan alle drie voor het komende najaar op de lijst.
\end{enumerate}

\section {Boek}

Theo Janssen heeft een brief gestuurd naar aanleiding van het idee een boek
te schrijven. Theo heeft een plan geschreven en zal op 9 februari om half twee
's middags langs komen om dit te bespreken. Jan L. deelt hiervan kopie\"{e}n uit
en op maandag 26 maart zullen we hierover gaan praten.

\section {Diversen extern}

\begin{enumerate}
    \item Ben Waumans dankt degenen die betrokken waren bij het programma ter 
          viering van zijn jubileum.
    \item Woensdagochtend (21 februari) zal dhr. Hurault op bezoek komen.
          Het gaat om een `vervolg'-bezoek en hij zal met Jan L. en Jan O.
          praten.
    \item Van het Esprit project `Pragmatics-based Language Understanding
          System' is een volledige beschrijving van de aanvraag binnengekomen,
          en hiervan zullen exemplaren verspreid worden over de verschillende 
          kamers. Wie (van de Philips-medewerkers) van ons aan dit project
          eventueel zullen meedoen zal later worden bekeken.
    \item Jan L. heeft een gesprek gehad met mensen van `PASS' die werkten
          aan EDI (`Electronic Data Interchange'). Het gaat hier om een soort
          formele taal, waarbij vertalen niet echt nuttig lijkt.
    \item Er zijn mensen van PTT telecom op bezoek geweest. Het lijkt erop dat 
          Rosetta voor hun `te ver weg' is. Ook is niet duidelijk op wat voor
          soort machine Rosetta zal gaan draaien.
    \item Vrijdag 23 februari gaan Jan L. en Joep bij dhr. Cuppen van Medical
          Systems op bezoek. Bij Medical Systems hebben ze het probleem dat
          veel tekstmateriaal in hun manuals e.d. regelmatig kleine 
          veranderingen ondergaat en daardoor opnieuw vertaald moet worden.
    \item Harm is op bezoek geweest bij de groep van Willem Meijs van het
          Acquilex-project. Zij werken ook met het Vandale-bestand van de N-N
          en zijn ge\"{\i}nteresseerd in uitwisseling van gegevens die op de N-N
          betrekking hebben.
\end{enumerate}

\section {Diversen intern}
\begin{enumerate}
    \item Jan L. deelt een lijst met Rosetta-publicaties uit. Iedereen moet
          controleren of de gegevens juist zijn.
    \item Jan L. vraagt zich af of de PROLOG-club nog bestaat. Momenteel zijn 
          er wegens tijdgebrek bij de deelnemers geen bijeenkomsten meer maar
          men wil er na mei weer mee verder gaan.
    \item Conversie naar de SUN: Harold en Josien houden zich hier nu mee
          bezig. Het gaat om een conversie naar PASCAL op de SUN onder UNIX.
          De voor dit doel aangeschafte compiler lijkt tamelijk goed al zijn er
          nog wat kleine problemen.
\end{enumerate}

\section {Rosetta 3D}

Joep meldt dat op de laatste 3D vergadering is afgesproken om op 1 april te 
bekijken wat er precies haalbaar is voor de CRE. Jan L. vraagt zich af of dit 
niet eerder moet. Maandag 26 februari gaan we dit verder bespreken.

Jan O. merkt op dat de Engelse verbs nog `crash-vrij' gemaakt worden.

Lisette merkt nogmaals op dat de het werk aan de Nederlandse adjectieven 
zo snel mogelijk moet worden afgemaakt, zodat zij de temporele adjectieven kan 
gaan vullen.

Harm merkt op dat het goed zou zijn de lexico-bestanden eens wat vaker 
uitgeprobeerd werden bij het draaien van Rosetta zodat eventuele tekortkomingen 
zo spoedig mogelijk aan het licht komen.

De archief-schijf raakt vol. Arnold heeft ons al een grotere beloofd.

Franciska heeft het idee dat het draaien van de testbank onnodig lang duurt, 
met name door de idiomen die er in staan. Het is niet duidelijk of dit wel echt 
vertragend werkt maar Jan O. merkt op dat het niet wenselijk is de testbank te
verkleinen omdat het weglaten van testen uiteindelijk leidt tot meer fouten in 
het systeem.


\section {Rondvraag}

Franciska is op bezoek geweest bij Kees van Deemter. Hij vertelde dat de 
overeenkomst met SRI is gesloten. Moeten wij hieraan mee doen? Jan L. merkt op 
dat we met 2 Esprit projekten, CSO-cursussen en het VanDale LEXIC-project al 
vol zitten, maar is wel ge\"{\i}nteresseerd in de plannen.

Joep merkt op dat er wel heel erg slecht wordt schoongemaakt op de kamers. Ook 
de anderen vinden dit. Jan L. gaat hier wat aan doen.

\end{document}
