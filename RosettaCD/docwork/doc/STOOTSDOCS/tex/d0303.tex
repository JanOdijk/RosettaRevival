%R0303.tex
\documentstyle{Rosetta}
\makeindex
\begin{document}
   \RosTopic{Rosetta3.doc.software}
   \RosTitle{The Rosetta Module Index}
   \RosAuthor{Joep Rous and Ren\'{e} Leermakers}
   \RosDocNr{303}
   \RosDate{\today}
   \RosStatus{concept}
   \RosSupersedes{-}
   \RosDistribution{Project}
   \RosClearance{Project}
   \RosKeywords{}
   \MakeRosTitle

\newcommand{\prog}[2]{\item[#1:#2\index{#2} (p)] : }
\newcommand{\modl}[2]{\item[#1:#2\index{#2} (m)] : }
\newcommand{\env}[2]{\item[#1:#2\index{#2} (oe)] : }
\newcommand{\envmodl}[2]{\item[#1:#2\index{#2} (em)] : }
\newcommand{\impmodl}[2]{\item[#1:#2\index{#2} (im)] : }
\newcommand{\ldmodl}[2]{\item[#1:#2\index{#2} (ld)] : }
\newcommand{\lsmodl}[2]{\item[#1:#2\index{#2} (ls)] : }
\newcommand{\data}[2]{\item[#1:#2\index{#2} (da)] : }
\newcommand{\exe}[2]{\item[#1:#2\index{#2} (exe)] : }
\newcommand{\com}[2]{\item[#1:#2\index{#2} (com)] : }
\newcommand{\prosefile}[2]{\item[#1:#2\index{#2} (pr)] : }
\newcommand{\prosedef}[2]{\item[#1:#2\index{#2} (pd)] : }
\newcommand{\ldimodl}[2]{\item[#1:#2\index{#2} (ldi)] : }
\newcommand{\bsc}{\begin{description}}
\newcommand{\esc}{\end{description}}
\newcommand{\mnm}[1]{{\bf #1}}
\newcommand{\indexentry}[2]{\item {\bf #1}, #2}

\section{Introduction}
This document gives an overview of the modules which together make up the
Rosetta software. The structure of the document reflects the
structure of the Rosetta Translator Generator.
The Translator Generator consists of three main parts
\begin{description}
\item [kernel] The implementation of the Rosetta components (e.g. A-MORPH, M-PARSER ).
\item [compiler generator] The compiler generator generates a compiler from
a given {\sf PRosE} dialect specification.
\item [{\sf PRosE} dialect specification] Specifications of all {\sf PRosE} dialects.
\end{description}

In order to generate a Translator for a specific natural language one has to
 define the various apsects of that
language, e.g. morphology, syntax and lexical information,
in the notation prescribed by the {\sf PRosE} dialects. The collection of
these  natural
language specifications, which is
the Translator Generator input, is called {\em lingware}.

The Rosetta kernel will be described in Chapter 3, the compiler generator in
chapter 4 and the {\sf PRosE} dialect definitions in chapter 5. Chapter 6
contains the file names that make up the lingware of the system.

In Chapter 7 the command files are described which are used to execute the
Rosetta system, the VMS compilers, the Rosetta compilers etc..

Chapters 3, 4, 5, 6 and 7 contain the description of all {\em genuine sources}
of the Rosetta software. The subsequent chapters contain module descriptions
of {\em targets}, that is, modules which are generated by compilers or linkers
from other modules. Chapter 8 specifies the targets of the compiler generator.
Chapter 9 contains a list of immediate targets of each
Rosetta compiler and in Chapter 10 all Rosetta compiler executables are listed.

Finally, there is an index containing an alphabetically ordered list of all
module names occurring in the document.

\section{Naming conventions and terminology}
In the document some naming conventions are used. All module names are
labeled with one of the following strings:
\begin{description}
   \item[(m) ] A normal PASCAL module consisting of an {\em environment} and
              {\em implementation}
              part which are in the same component.
   \item[(p) ] A PASCAL module which is the body of a program, that is, it
              only consists of an {\em implementation} part.
   \item[(oe) ] A PASCAL module which has no {\em implementation} part, only an
               {\em environment} part.
   \item[(em)] The environment part of a normal PASCAL module.
   \item[(im)] The implementation part of a normal PASCAL module.
   \item[(da)] The file doesn't contain PASCAL code but only data in a fixed
               format.
   \item[(exe)] The file is an executable
   \item[(com)] The file is a DCL command file.
   \item[(pr)] The file is a {\em lingware} file and contains a specification
               in a {\sf PRosE} dialect.
   \item[(pd)] The file is an inputfile for the compiler generator and
               contains a definition of a {\sf PRosE} dialect.
   \item[(ls)] Both the {\em environment} part and {\em implementation} part
               are {\em language specific} and have to be present in each
               language component.
   \item[(ld)] The module is language dependent, that is, the module has both
               an {\em environment} and {\em implementation} part. The
enviroment part is language independent and resides in the specified component.
The implementation part, however, is language specific. Therefore, each
language component (DUTCH, ENGLISH and SPANISH) will have its own version of
the implementation part of the module.
   \item[(ldi)] The module is language dependent, that is, the module has both
               an {\em environment} and {\em implementation} part. The
enviroment part is language independent and resides in the specified component.
The implementation part is language specific. The language specific parts for
each language differ {\bf only} with respect to the component names of the {\em
inherited} modules.
In particular, the DUTCH version of the implemention part inherits DUTCH
modules, the ENGLISH implementation part inherits ENGLISH components, etc..
In stead of having three almost identical implementation part there is just
one implementation part (residing in the same component as the environment part)
in which the modules are inherited from the virtual component LANGUAGE.
\end{description}

Most of the Rosetta compilers have a similar structure. A compiler module
with a specific task has been given a fixed name ending, e.g. the
name of a compiler module in which a lexical scanner is implemented will
end in `{\bf scanner}'.  Therefore the compiler modules which obey this
naming convention are described only once (at their first cccurence).

\newpage
\section{Rosetta kernel}
\subsection{Components}
The modules in this section are directly related to the definitions of the
Rosetta components in the formalism. There is a 1-1 corespondence between
these modules and the functions defined in the formalism. The only exception
is the module \mnm{control} which has no counterpart in the formalism.

\bsc
\prog{general}{control} Controls the execution of the analysis and generation process.
               It also passes information about debugging and interface printing
               to both processes.
\prog{general}{analysis}
   \bsc
   \modl{general}{alayout}
   \modl{general}{amorph}
      \bsc
      \modl{general}{asegm}
      \modl{general}{alex}
      \modl{general}{alextree}
      \esc
   \modl{general}{surfparser}
   \modl{general}{mparser}
   \modl{general}{atransfer}
   \esc
\prog{general}{generation}
   \bsc
   \modl{general}{gtransfer}
   \modl{general}{mgenerator}
   \modl{general}{linearizer}
   \modl{general}{gmorph}
      \bsc
      \modl{general}{glextree}
      \modl{general}{glex}
      \modl{general}{gsegm}
      \esc
   \modl{general}{glayout}
   \esc
\esc

\subsection{Component interface modules}

This section describes all modules in which the interface datastructures between
\mnm{alayout}, \mnm{amorph}, \mnm{surfparser}, \mnm{mparser}, \mnm{atransfer},
\mnm{gtransfer}, \mnm{mgenerator}, \mnm{linearizer}, \mnm{gmorph} and
\mnm{glayout} are defined,

\bsc
\env{general}{interfaces} Defines the general interface datastructure.
   \bsc
   \modl{general}{interface0} Interface between \mnm{alayout} and \mnm{amorph}.
   \modl{general}{interface1} Interface between \mnm{amorph} and \mnm{surfparser}.
   \env{general}{interface2} Interface between \mnm{surfparser} and \mnm{mparser}.
   \env{general}{interface3} Interface between \mnm{mparser} and \mnm{atransfer}.
   \env{general}{interface4} Interface between \mnm{analysis} and \mnm{generation}.
   \env{general}{interface5} Interface between \mnm{gtransfer} and \mnm{mgenerator}.
   \env{general}{interface6} Interface between \mnm{mgenerator} and \mnm{linearizer}.
   \modl{general}{interface7} Interface between \mnm{linearizer} and \mnm{gmorph}.
   \modl{general}{interface8} Interface between \mnm{gmorph} and \mnm{glayout}.
   \esc
\esc

\subsection{Interface print modules}

\bsc
\modl{general}{printerf} Can print (on screen) the datastructure defined in
               \mnm{interfaces}.
   \bsc
   \modl{general}{printerf1} Prints datastructures defined in \mnm{interface1}.
   \modl{general}{printerf2} Prints datastructures defined in \mnm{interface2}.
   \modl{general}{printerf3} Prints datastructures defined in \mnm{interface3}.
   \modl{general}{printerf4} Prints datastructures defined in \mnm{interface4}.
   \modl{general}{printerf5} Prints datastructures defined in \mnm{interface5}.
   \modl{general}{printerf6} Prints datastructures defined in \mnm{interface6}.
   \modl{general}{printerf7} Prints datastructures defined in \mnm{interface7}.
   \esc
\esc

\subsection{Short-circuit modules}

\bsc
\modl{general}{interface1to7} Converts output of \mnm{amorph} according to datastructure
                     definition \mnm{interface1} to data according to
                     \mnm{interface7} which can be used as input for
                     \mnm{gmorph}.
\modl{general}{interface3to5} Converts output of \mnm{mparser} according to datastructure
                     definition \mnm{interface3} to data according to
                     \mnm{interface5} which can be used as input for
                     \mnm{mgenerator}.
\esc

\subsection{O.S. interface}

\bsc
\modl{general}{string} Contains string manipulation routines (dynamic strings).
\modl{general}{str} Contains string manipulation routines (fixed length strings)
.
\modl{general}{files} Provides basic operations on text files.
\modl{general}{windows} Provides routines for window handling on the terminal
                       screen (window creation, deletion, text manipulation
                       in windows).
\modl{general}{pc} Contains process handling routines (process creation,
                  deletion).
\modl{general}{mb} Contains routines for mailbox handling (mailbox creation,
                  deletion, connection to already created mailboxes,
                  message handling). Mailboxes are not intended to
                  transport large amounts of data.
\modl{vms}{vmsrms} A file I/O package. Provides sequential and indexed file
                  access.
\modl{general}{memory} Implementation of the MARK and RELEASE function for
                      memory management.
\modl{vms}{globbuf} Enables the user to define buffers which can be shared
                   between several processes. The package is intended for the
                   transport of bulk-data.
\modl{general}{clock} Functions which give consumed CPU time, date, etc..
\esc
\bsc
\env{vms}{vmssmg} Low-level interface to the SMG package of VMS.
\env{vms}{vmsstr} Low-level interface to the STR package of VMS.
\env{vms}{libdef} Provides an interrupt generating function, which can be
                 used to generate a run-time error.
\esc

\subsection{Layout, phonetical and context-condition rules}

\bsc
\ldmodl{general}{ldmorfdef.env} Defines the interface to the layout rules, the phonetical
                   rules and the context-condition rules.
\esc

\subsection{Segmentation rule interface}

\bsc
\modl{general}{asegmrules} Takes care of the actual application of the
                          analytical segmentation rules.
\modl{general}{gsegmrules} Takes care of the actual application of the
                          generative segmentation rules.
   \bsc
   \modl{general}{segmrules} Provides I/O routines for handling the segmentation
                            rule files.
   \esc
\ldmodl{general}{ldsucc.env} The interface to the control expression which defines
                        the application order of the segmentation rules.
\esc


\subsection{W-Rule interface}

\bsc
\ldmodl{general}{anlexif.env} Interface for the analytical lextree rules.
\ldmodl{general}{genlexif.env} Interface for the generative lextree rules.
\esc

\subsection{Surface parser rule interface}

\bsc
\ldmodl{general}{surfrulesgraphs.env} Defines the interface to the control
                                 expression part of the surface rules.
\ldmodl{general}{surfrules.env} Defines the interface to the condition action part
                           of the surface rules.
\esc

\subsection{M-rule (interface) modules}

\bsc
\ldmodl{general}{ldmrules.env} Part of the M-rule interface that is the same both for
                     analysis and generation.
\ldmodl{general}{ldsubgrammars.env} Contains functions that give information about
                     the subgrammars which are defined in the M-rules.
\ldmodl{general}{ldanmrules.env} Analytical M-rule interface definition.
\ldmodl{general}{ldgenmrules.env} Generative M-rule interface definition.
\modl{general}{limrules} Converts the M-grammar control expressions into
                        M-rule successor (predecessor) relations.
\modl{general}{limatches} Provides functions for M-rule model matching.
\esc

\subsection{Transfer rule interface modules}

\bsc
\ldmodl{general}{ldanilrules.env} Analytical transfer-rule interface definition
\ldmodl{general}{ldgenilrules.env} Generative transfer-rule interface defintion
\esc

\subsection{Lexicon interface modules}

\bsc
\ldimodl{general}{ldmdict} Provides functions for accessing MDICT.
   \bsc
   \ldimodl{general}{mdictdef} Definition of the MDICT record structure.
   \esc
\modl{general}{lifixiddict} Functions for accessing the fixed idiom
                           dictionary FIXIDDICT.
\ldimodl{general}{ldblex} BLEX access functions.
\modl{general}{lisdict} SDICT ( defines the relation between fkeys and
                       skeys) access functions.
\modl{general}{lisiddict} SIDDICT (the semi-idiom dictionary) access
                         functions.
\modl{general}{liiddict} IDDICT (the idiom dictionary) access
                         functions.
\modl{general}{liildict} ILDICT access functions. Furthermore, it
                        includes a definition of the ILDICT record structure.

\ldimodl{general}{ldaffixlex} Defines functions for accessing the affix lexicon.
\ldimodl{general}{ldgluelex} Defines functions for accessing the glue lexicon.

\modl{general}{strtokey} A package access functions for the compiled skey, fkey
                        and mkey definition files.
   \bsc
   \env{general}{strkeyrecdef} Definition of the record structure of the
                              compiled key definition files.
   \esc
\esc

\subsection{Tree definition modules}

\bsc
\env{general}{lidomaint} Defines the language independent counterparts of
                        the types defined in module {\bf language:lsdomaint}.
\esc
\bsc
\modl{general}{listree} Defines the abstract datatype S-tree
\modl{general}{superdtree} Defines the abstract datatype Super D-tree
\modl{general}{hyperdtree} Defines the abstract datatype Hyper D-tree
\modl{general}{hiltree} Defines the abstract datatype Hyper IL-tree
\esc

\subsection{Tree print modules}

\bsc
\modl{general}{drawtree} Implementation module of the Tree-print algorithm. The module
                is able to print various types of trees. Drawtree implements
                the tree type-independent part of the algorithm.
\modl{general}{oldtree} Module {\bf oldtree} implements the tree-type dependent parts
               of the algorithm, e.g. GetSon, GetBrother, etc.
\esc
\bsc
\modl{general}{drawstree} Prints an S-tree on the screen.
\modl{general}{drawsuperdtree} Prints a Super D-tree on the screen.
\modl{general}{drawhyperdtree} Prints a Hyper D-tree on the screen.
\modl{general}{drawhiltree} Prints a Hyper IL-tree on the screen.
\esc
\bsc
\modl{general}{lirectoscreen} Prints the contents of a tree node on the screen.
\modl{general}{rectoscreen} Prints the contents of S-tree node record on the screen and
                   enables the user to edit the record.
\esc

\subsection{Debug support modules}

\bsc
\modl{general}{error} Prints a status line containing an error or status message
             on the screen.
\modl{general}{log} Writes a message to a log file. There is one log file for each
           process.
\modl{general}{debug} Writes debug information in a special window on the screen.
\modl{general}{debugmparser} Implementation of the M-parser debug mode.
\modl{general}{debugmgenerator} Implementation of the M-generator debug mode.
\esc

\subsection{Conversion modules}

\bsc
\ldmodl{general}{ldconvrec.env} Contains conversion routines for string datastructure to
                   record conversion and vice versa.
\ldmodl{general}{ldstrtotype.env} Conversion routines for string to category, relation,
                     context condition etc.
\ldmodl{general}{ldtypetostr.env} Inverse of module {\bf ldstrtotype}.
\ldmodl{general}{ldidpatterns.env} Conversion module from idiom pattern to string.
\esc

\subsection{Miscellaneous}

\subsubsection{Surface parser modules}

\bsc
\ldmodl{general}{ldprims.env} Surface parser help routines.
\env{general}{items}
\ldmodl{general}{ldequal.env}
\esc

\subsubsection{Mparser, Mgenerator modules}

\bsc
\modl{general}{mpstatistics} Puts a window on the
                           terminal screen at the end of an M-parse
                           with information about the number
                           of M-rule applications etc..
\modl{general}{globsubst} Makes the substituent S-tree of the substitution rules
                        globally available for other M-rules.
\esc

\subsubsection{Miscellaneous}

\bsc
\ldimodl{general}{lsstree} Definition of language specific S-tree.
\modl{general}{globdef} Module which has the function of a blackboard for the
                       system. It is used to store and provide general
                       information data.
\ldmodl{general}{ldgetkey.env} Returns the key value of basic S-trees. In case the S-tree is
                  a variable it returns the index value of the variable.
\ldmodl{general}{ldcatsets.env} Provides characteristic functions of several sets which
                   have been specified in the language domain.
\ldmodl{general}{ldsubsttovar.env} Gives for an S-tree a corresponding stree with a VAR
                    category that corresponds to the category of the input
                    S-tree.
\ldmodl{general}{ldstrtostr.env}
\esc


\newpage
\section{The compiler generator}
\subsection{Generical compiler modules}
\bsc
\prog{tools}{gencomp} Makes a copy of a generical module and changes the copy
                      for usage in the compiler to be generated.
\prog{tools}{gencom} Generical program body of a compiler.
\modl{tools}{gencomscanner} Generical lexical scanner.
\modl{tools}{gencomparser} Generical parser module.
\modl{tools}{gencomdecl} Generical datastructure module.
\modl{tools}{gencomgraph} Generical module  in which the datastructures are
                          defined that are used in the implementation of
                          EBNF expressions.
\envmodl{tools}{gencomgraphdef} Generical environment module. This module
                          defines the interface of the ENBNF expression
                          definitions.
\esc
\subsection{Domain specification evaluation}
\bsc
\prog{tools}{mrudomcom}
\modl{tools}{mrudomcomlangspec}
\envmodl{tools}{gencomscanner}
\impmodl{tools}{mrudomcomscanner}
\envmodl{tools}{gencomgraphdef}
\impmodl{tools}{mrudomcomgraphdef} Definition of the EBNF expressions defining
                          the global syntax of the language.
\modl{tools}{gencomgraph}
\modl{tools}{gencomdecl}
\modl{tools}{gencomparser}
\modl{tools}{mrudomcomrules} Definition of the actual attribute grammar.
\esc
\subsection{Syntax and Code generation evaluation}
\bsc
\prog{tools}{mrusurcom}
\modl{tools}{mrusurcomcode} Definition of the code generationpart of the
                            compiler.
\modl{tools}{mrusurcomdecl}
\modl{tools}{mrusurcomscanner}
\modl{tools}{mrusurcomgraph}
\modl{tools}{mrusurcomgraphdef}
\modl{tools}{mrusurcomparser}
\modl{tools}{mrusurcomrules}
\esc

\newpage
\section{{\sf PRosE} dialect compilers}
\subsection{Domain dialects}
\subsubsection{Domain T dialect}
\bsc
\prog{tools}{domcom} The program body of the domain compiler
\impmodl{tools}{domcomgraphdef} Specification of the EBNF expressions defining
                             the
                             syntax of the Domain T language.
\modl{tools}{domcomlangspec} Definition of the atribute grammar {\em domain}.
\modl{tools}{domcomrules} Definition of the attribute grammar, including the
                          code generation.
\impmodl{tools}{domcomscanner} The lexical scanner implementation part.
\envmodl{tools}{gencomscanner}
\modl{tools}{gencomparser}
\modl{tools}{gencomdecl}
\modl{tools}{gencomgraph}
\envmodl{tools}{gencomgraphdef}
\esc
\subsubsection{Auxiliary domain language}
\bsc
\prosedef{tools}{auxcom.gendom} Definition of the attribute grammar domain
\prosedef{tools}{auxcom.gensur} Definition of the attribute grammar syntax and
                                code generation.
\esc
\subsection{Lexicon dialects}
\subsubsection{Rosetta lexicon language}
\bsc
\prog{tools}{newdictgen} The lexicon compiler program which defines the syntax
                          of the Rosetta lexicon language.
\esc

\subsubsection{Lexicon linker}
\bsc
\prog{tools}{isfinit} Initialization of all Rosetta lexicons.
\prog{tools}{isfmerge} Merges two versions of the lexicon files.
\prog{tools}{fixidgen} The compiler of the fixed idiom specification files which
                       are generated by the lexicon compiler.
\esc


\subsubsection{Lexicon constraint language}
\bsc
\prog{tools}{constraintgen} The compiler which defines the language in which the
                     constraints of the Rosetta lexicon can be specified.
\esc
\subsubsection{Key definition language}
\bsc
\prog{tools}{strkey} Defines the (very simple) language in which lexicon
                    keys can be specified.
\esc
\subsection{Affix rules}
\subsubsection{Affix rule language}
\bsc
\prog{tools}{asegcom} Body of the analytical affix rule compiler.
\prog{tools}{gsegcom} Body of the generative affix rule compiler.
\modl{tools}{segcomdecl} c.f. previously described compilers.
\modl{tools}{segcomgraph}
\modl{tools}{segcomgraphdef}
\modl{tools}{segcomlangspec}
\modl{tools}{segcomparser}
\modl{tools}{segcomrules}
\modl{tools}{segcomscanner}
\esc
\subsubsection{Affix control expression language}
\bsc
\prosedef{tools}{afxpr.gendom} Domain of the affix control expression attribute
                         grammar.
\prosedef{tools}{afxpr.gensur} Syntax and code generation part of the affix
                           control expression grammar.
\esc
\subsection{W-Rule language}
\bsc
\prog{tools}{lexcom} Body of the W-Rule compiler.
\modl{tools}{lexcomgraphdef} c.f. previously described compilers.
\modl{tools}{lexcomgraph}
\modl{tools}{lexcomdecl}
\modl{tools}{lexcomparser}
\modl{tools}{lexcomscanner}
\modl{tools}{lexcomrules}
\modl{tools}{lexcomcode}
\esc
\subsubsection{W-Rule linker}
\bsc
\prog{tools}{lexlink} Generates a kind of switch over all W-Rules.
\esc


\subsection{Surface Rule language}
\bsc
\prog{tools}{surcom} Body of the Surface Rule compiler.
\modl{tools}{surcomcode} c.f. previously described compilers.
\modl{tools}{surcomdecl}
\modl{tools}{surcomscanner}
\modl{tools}{surcomparser}
\modl{tools}{surcomgraph}
\modl{tools}{surcomgraphdef}
\modl{tools}{surcomrules}
\esc
\subsection{M-Rule language}
\bsc
\prosedef{tools}{mrucom.gendom} Domain of the M-rule attribute grammar.
\prosedef{tools}{mrucom.gensur} Syntax and code generation part of the M-rule
                                attribute grammar.
\esc
\subsubsection{M-Rule linker}
\bsc
\prog{tools}{mrulelink} Collects global M-rule information, e.g all M-rule
                        names, and generates functions which enable other
                        modules to use this information.
\esc


\subsection{Transfer Rule language}
\bsc
\prosedef{tools}{tracom.gendom} Domain of the transfer rule attribute grammar.
\prosedef{tools}{tracom.gensur}  Syntax and code generation part of the
                               transfer rule attribute grammar.
\esc
\subsection{Interlingua Rule language}
\bsc
\prosedef{tools}{ilacom.gendom} Domain of the interlingua attribute grammar.
\prosedef{tools}{ilacom.gensur} Syntax and code generation part of the
                               interlingua attribute grammar.
\esc

\newpage
\section{Lingware modules}

\subsection{Domains}
\bsc
\prosefile{language}{lsdomaint.dom} Domain T specification.
\prosefile{language}{lsauxdomain.auxdom} Auxiliary domain specification.
\esc
\subsection{Lexicons}
Specification of the Rosetta dictionaries.
\bsc
\prosefile{language}{auxverb.dict} Specification of AUXVERB entries.
\prosefile{language}{badj.dict} Specification of BADJ entries
\prosefile{language}{badv.dict} Specification of BADV entries
\prosefile{language}{bnoun.dict} Specification of BNOUN entries
\prosefile{language}{bverb.dict} Specification of BVERB entries
\prosefile{language}{conj.dict} Specification of CONJ entries
\prosefile{language}{exclam.dict} Specification of EXCLAM entries
\prosefile{language}{misc.dict} Specification of entries for several categories.
\prosefile{language}{particle.dict} Specification of PARTICLE entries
\prosefile{language}{prep.dict} Specification of PREP entries
\prosefile{language}{pronoun.dict} Specification of PRONOUN entries
\esc
\bsc
\prosefile{language}{skeydef.kdf} Definition of the `skeys'.
\prosefile{language}{constraints.constr} Constraint specifications for BLEX.
\esc
\subsection{Morphology}
\subsubsection{String Phase}
\bsc
\prosefile{language}{lglue.seg} Specification of left GLUE rules.
\prosefile{language}{rglue.seg} Specification of right GLUE rules.
\prosefile{language}{mglue.seg} Specification of middle GLUE rules.
\prosefile{language}{suffix.seg} Specification of suffix rules.
\prosefile{language}{prefix.seg} Specification of prefix rules
\prosefile{language}{morphexpr.afxpr} Specification of the order in which prefix and suffix
                         rules must be applied.
\esc
\subsubsection{S-tree Phase}
There are three files available in which the, so-called, W-rules can be
specified:
\bsc
\prosefile{language}{lexrules1.lex}
\prosefile{language}{lexrules2.lex}
\prosefile{language}{lexrules3.lex}
\esc
\subsection{Surface Grammar}
There are three files in which the Surface rules can be specified:
\bsc
\prosefile{language}{surfrules1.sur}
\prosefile{language}{surfrules2.sur}
\prosefile{language}{surfrules3.sur}
\esc
\subsection{M-Rules}
The filenames of the files in which the M-rules have been specified have not
been prescribed. The extension, however, must be `.mrule'. In order for the
system to know which filenames have been used, they must be specified
in the file `mrules.mms'
\bsc
\prosefile{language}{*.mrule} M-Rule files.
\prosefile{language}{mrules.mms} Specification of M-Rule filenames.
\esc

\subsection{Transfer Rules}
\bsc
\prosefile{language}{transferrules.trans} Specification of transfer rules
 which establish the link between M-Rules and IL-Rules.
\esc
\subsection{PASCAL Lingware}
A small part of the Lingware has still to be specified in PASCAL.
\bsc
\lsmodl{language}{lsmruquo} PASCAL functions which are used in the M-rules
\lsmodl{language}{lsphondef} Specification of the phonetical information and the
      phonetical rules which are used in the morphology component.
\ldmodl{language}{ldidpatterns.pas} Conversion module from PASCAL idiom pattern value to
string value.
\ldmodl{language}{ldmorfdef.pas} Specification of layout rules and context conditions.
\esc
\subsection{Interlingua}
\bsc
\prosefile{interlingua}{mkeydef.kdf} Definition of the names of the basic
                                    meanings.
\prosefile{interlingua}{ildefinition.ilan} Definition of the IL rulenames
\esc

\newpage
\section{Command Files}
\subsection{The Rosetta system}
\bsc
\com{general}{runrosetta} Starts the Rosetta system in interactive mode.
\com{general}{batchrosetta} Asks the user about the system configuration
                            he wants to run in batch-mode and starts the
                            batch-mode job.
\com{general}{batchros} Starts the actual batch mode job.
\com{actions}{tstb} Starts a batch-mode job to run a test-bench.
\com{actions}{morftest} Starts a version of the Rosetta system in which the
                         morfological components have been shortcircuited.
\esc
\subsection{The compiler generator}
\bsc
\com{actions}{gen} Runs the compiler generator and creates a compiler.
\com{actions}{gencomp} Command file used during generation of a compiler
                       for preparing a generical module in order to be
                        used in the compiler.
\com{actions}{gendom} Command file used during compiler generation for
                      domain evaluation.
\com{actions}{gensur} Used for evaluation of the syntax specification and
                      code generation.
\com{actions}{genobj}
\esc
\subsection{The compilers}
\subsubsection{VMS compilers}
\bsc
\com{actions}{pas} Applies the PASCAL compiler to the implementation part of a
                   module.
\com{actions}{env} Applies the PASCAL compiler to the environment part of a
                   module.
\com{actions}{obj} Creates an .opt file.
\com{actions}{opt} Links a number of object into an executable.
\com{actions}{merge\_opt} Adds a name of an object file to a .opt file.
\com{actions}{vmssysenv} Copies the  compiled system library  routines from
                         the system library.
\esc
\subsubsection{Rosetta compilers}
\bsc
\com{actions}{dom} Invokes the domain T compiler.
\com{actions}{auxdom} Invokes the auxiliary domain compiler.
\com{actions}{dict} Invokes the lexicon compiler.
\com{actions}{fixid} Invokes the fixed idiom compiler.
\com{actions}{constraint} Invokes the lexicon constraint compiler.
\com{actions}{kdf} Invokes the key definition compiler.
\com{actions}{seg} Invokes the segmentation rule compiler.
\com{actions}{afxpr} Invokes the affix control expression compiler.
\com{actions}{lex} Invokes the W-Rule compiler.
\com{actions}{sur} Invokes the Surface Rule compiler.
\com{actions}{mru} Invokes the M-rule compiler.
\com{actions}{mruall} Invokes the M-rule compiler and compiles all immediate
                      targets.
\com{actions}{idioms} Invokes the M-rule compiler for the idiom mrule file(s).
\com{actions}{tra} Invokes the transfer rule compiler.
\com{actions}{ila} Invokes the interlingua compiler.
\esc
\subsection{The Linkers}
\bsc
\com{actions}{isfinit} Initializes the Rosetta lexicon.
\com{actions}{isfexit} Completes the creation of the Rosetta lexicon.
\com{actions}{isfmerge} Adds a sub-lexicon to the Rosetta lexicon.
\com{actions}{llk} Calls the W-rule linker.
\com{actions}{mlk} Calls the M-rule linker.
\esc
\newpage
\section{Compiler generator targets}
\bsc
\exe{tools/language}{$<${\em compilername}$>$.exe}
\esc

\newpage
\section{{\sf PRosE} compiler targets}
\subsection{Domain dialects}
\subsubsection{Domain T targets}
\bsc
\env{language}{lsdomaint}
\lsmodl{language}{lstypetostr}
\lsmodl{language}{lsstrtotype}
\lsmodl{language}{lsconvrec}
\lsmodl{language}{lsconvattr}
\lsmodl{language}{maket}
\lsmodl{language}{copyt}
\ldmodl{language}{ldstrtotype.pas}
\ldmodl{language}{ldtypetostr.pas}
\ldmodl{language}{ldstrtostr.pas}
\ldmodl{language}{ldgetkey.pas}
\ldmodl{language}{ldconvrec.pas}
\ldmodl{language}{ldcatsets.pas}
\ldmodl{language}{ldequal.pas}
\esc
\subsubsection{Auxiliary domain targets}
\bsc
\lsmodl{language}{lsauxdom}
\ldmodl{language}{ldsubsttovar.pas}
\esc
\subsection{Lexicon dialects}
\subsubsection{Rosetta lexicon targets}
For each of the lexicon sources (cf. ) the following targets are generated:
\bsc
\data{language}{blex$<${\em source}$>$.isf}
\data{language}{iddict$<${\em source}$>$.isf}
\data{language}{ildict$<${\em source}$>$.isf}
\data{language}{mdict$<${\em source}$>$.isf}
\data{language}{sdict$<${\em source}$>$.isf}
\data{language}{siddict$<${\em source}$>$.isf}
\data{language}{fixid$<${\em source}$>$.fixid}
\esc
\subsubsection{Lexicon linker targets}
\bsc
\data{language}{blex.isf}
\data{language}{iddict.isf}
\data{language}{ildict.isf}
\data{language}{mdict.isf}
\data{language}{sdict.isf}
\data{language}{siddict.isf}
\data{language}{fixid.fixid}
\data{language}{fixid.isf}
\esc

\subsubsection{Lexicon constraint targets}
\bsc
\lsmodl{language}{lsconstraints}
\esc
\subsubsection{Key definition targets}
\bsc
\data{language}{skeydef.rmskdf}
\data{language}{mkeydef.rmskdf}
\esc
\subsection{Affix rules}
\subsubsection{Affix rule targets}
For each of the affix rule sources (lglue, rglue, mglue, prefix, suffix) the
following targets are generated:
\bsc
\data{language}{a$<${\em source}$>$.sro}
\data{language}{a$<${\em source}$>$.sso}
\data{language}{a$<${\em source}$>$.svo}
\data{language}{a$<${\em source}$>$.sco}
\data{language}{a$<${\em source}$>$.svs}
\data{language}{g$<${\em source}$>$.sro}
\data{language}{g$<${\em source}$>$.sso}
\data{language}{g$<${\em source}$>$.svo}
\data{language}{g$<${\em source}$>$.sco}
\data{language}{g$<${\em source}$>$.svs}
\esc
\subsubsection {Affix control expression targets}
\bsc
\ldmodl{language}{ldsucc.pas}
\esc
\subsection{W-Rule targets}
For each of the three W-rule files the following targets are generated:
\bsc
\lsmodl{language}{comlexrules$<${\em i}$>$}
\lsmodl{language}{decomlexrules$<${\em i}$>$}
\esc
\subsubsection{W-Rule linker targets}
\bsc
\ldmodl{language}{anlexif.pas}
\ldmodl{language}{genlexif.pas}
\esc

\subsection{Surface Rule targets}
\bsc
\ldmodl{language}{surfrules.pas}
\ldmodl{language}{surfrulesgraphs.pas}
\ldmodl{language}{ldprims.pas}
\esc
\subsection{M-Rule targets}
\bsc
\lsmodl{language}{commrules$<${\em i}$>$}
\lsmodl{language}{decommrules$<${\em i}$>$}
\esc
\subsubsection{M-Rule linker targets}
\bsc
\ldmodl{language}{ldmrules.pas}
\ldmodl{language}{ldsubgrammars.pas}
\ldmodl{language}{ldanmrules.pas}
\ldmodl{language}{ldgenmrules.pas}
\lsmodl{language}{helpsubgrammars}
\env{language}{lsparams}
\esc

\subsection{Transfer Rule targets}
\bsc
\ldmodl{language}{ldanilrules.pas}
\ldmodl{language}{ldgenilrules.pas}
\esc
\subsection{Interlingua rule targets}
\bsc
\modl{interlingua}{liilrules}
\esc

\newpage
\section{Executables}
\subsection{The Rosetta system}
\bsc
\exe{general}{control} The control process.
\exe{language}{analysis} The analysis process.
\exe{language}{generation} The generation process.
\esc
\subsection{the compiler generator generation}
\bsc
\exe{tools}{gencomp} Generical module handling.
\esc
\subsection{The compiler generator}
\bsc
\exe{tools}{mrudomcom} Evaluates compiler domain definition.
\exe{tools}{mrusurcom} Evaluates compiler syntax and code generation definition.
\esc
\subsection{The compilers}
\bsc
\exe{tools}{domcom} The domain compiler.
\exe{language}{auxcom} The auxiliary domain compiler.
\exe{dutch}{newdictgen} The lexicon compiler.
\exe{tools}{fixidgen} The fixid idiom specification compiler.
\exe{tools}{constraintgen} The lexicon constraint compiler.
\exe{tools}{strkey} The key definition compiler.
\exe{language}{asegcom} The analytical affix-rule compiler.
\exe{language}{gsegcom} The generative affix-rule compiler.
\exe{tools}{afxpr} The affix control expression compiler.
\exe{language}{lexcom} The W-Rule compiler.
\exe{language}{renesurcom} The Surface Rule compiler.
\exe{language}{mrucom} The M-Rule compiler.
\exe{language}{tracom} The Transfe Rule compiler.
\exe{language}{ilacom} The Interlingua compiler.
\esc
\subsection{The linkers}
\bsc
\exe{language}{isfinit} The lexicon initializer.
\exe{language}{isfmerge} The lexicon merger.
\exe{language}{lexlink} The W-Rule linker.
\exe{language}{mrulelink} The M-Rule linker.
\esc
\begin{theindex}
\include{formimpidx}
\end{theindex}
\end{document}


ROSETTA.sty
\typeout{Document Style 'Rosetta'. Version 0.4 - released  24-DEC-1987}
% 24-DEC-1987:  Date of copyright notice changed
\def\@ptsize{1}
\@namedef{ds@10pt}{\def\@ptsize{0}}
\@namedef{ds@12pt}{\def\@ptsize{2}}
\@twosidetrue
\@mparswitchtrue
\def\ds@draft{\overfullrule 5pt}
\@options
\input art1\@ptsize.sty\relax


\def\labelenumi{\arabic{enumi}.}
\def\theenumi{\arabic{enumi}}
\def\labelenumii{(\alph{enumii})}
\def\theenumii{\alph{enumii}}
\def\p@enumii{\theenumi}
\def\labelenumiii{\roman{enumiii}.}
\def\theenumiii{\roman{enumiii}}
\def\p@enumiii{\theenumi(\theenumii)}
\def\labelenumiv{\Alph{enumiv}.}
\def\theenumiv{\Alph{enumiv}}
\def\p@enumiv{\p@enumiii\theenumiii}
\def\labelitemi{$\bullet$}
\def\labelitemii{\bf --}
\def\labelitemiii{$\ast$}
\def\labelitemiv{$\cdot$}
\def\verse{
   \let\\=\@centercr
   \list{}{\itemsep\z@ \itemindent -1.5em\listparindent \itemindent
      \rightmargin\leftmargin\advance\leftmargin 1.5em}
   \item[]}
\let\endverse\endlist
\def\quotation{
   \list{}{\listparindent 1.5em
      \itemindent\listparindent
      \rightmargin\leftmargin \parsep 0pt plus 1pt}\item[]}
\let\endquotation=\endlist
\def\quote{
   \list{}{\rightmargin\leftmargin}\item[]}
\let\endquote=\endlist
\def\descriptionlabel#1{\hspace\labelsep \bf #1}
\def\description{
   \list{}{\labelwidth\z@ \itemindent-\leftmargin
      \let\makelabel\descriptionlabel}}
\let\enddescription\endlist


\def\@begintheorem#1#2{\it \trivlist \item[\hskip \labelsep{\bf #1\ #2}]}
\def\@endtheorem{\endtrivlist}
\def\theequation{\arabic{equation}}
\def\titlepage{
   \@restonecolfalse
   \if@twocolumn\@restonecoltrue\onecolumn
   \else \newpage
   \fi
   \thispagestyle{empty}\c@page\z@}
\def\endtitlepage{\if@restonecol\twocolumn \else \newpage \fi}
\arraycolsep 5pt \tabcolsep 6pt \arrayrulewidth .4pt \doublerulesep 2pt
\tabbingsep \labelsep
\skip\@mpfootins = \skip\footins
\fboxsep = 3pt \fboxrule = .4pt


\newcounter{part}
\newcounter {section}
\newcounter {subsection}[section]
\newcounter {subsubsection}[subsection]
\newcounter {paragraph}[subsubsection]
\newcounter {subparagraph}[paragraph]
\def\thepart{\Roman{part}} \def\thesection {\arabic{section}}
\def\thesubsection {\thesection.\arabic{subsection}}
\def\thesubsubsection {\thesubsection .\arabic{subsubsection}}
\def\theparagraph {\thesubsubsection.\arabic{paragraph}}
\def\thesubparagraph {\theparagraph.\arabic{subparagraph}}


\def\@pnumwidth{1.55em}
\def\@tocrmarg {2.55em}
\def\@dotsep{4.5}
\setcounter{tocdepth}{3}
\def\tableofcontents{\section*{Contents\markboth{}{}}
\@starttoc{toc}}
\def\l@part#1#2{
   \addpenalty{-\@highpenalty}
   \addvspace{2.25em plus 1pt}
   \begingroup
      \@tempdima 3em \parindent \z@ \rightskip \@pnumwidth \parfillskip
      -\@pnumwidth {\large \bf \leavevmode #1\hfil \hbox to\@pnumwidth{\hss #2}}
      \par \nobreak
   \endgroup}
\def\l@section#1#2{
   \addpenalty{-\@highpenalty}
   \addvspace{1.0em plus 1pt}
   \@tempdima 1.5em
   \begingroup
      \parindent \z@ \rightskip \@pnumwidth
      \parfillskip -\@pnumwidth
      \bf \leavevmode #1\hfil \hbox to\@pnumwidth{\hss #2}
      \par
   \endgroup}
\def\l@subsection{\@dottedtocline{2}{1.5em}{2.3em}}
\def\l@subsubsection{\@dottedtocline{3}{3.8em}{3.2em}}
\def\l@paragraph{\@dottedtocline{4}{7.0em}{4.1em}}
\def\l@subparagraph{\@dottedtocline{5}{10em}{5em}}
\def\listoffigures{
   \section*{List of Figures\markboth{}{}}
   \@starttoc{lof}}
   \def\l@figure{\@dottedtocline{1}{1.5em}{2.3em}}
   \def\listoftables{\section*{List of Tables\markboth{}{}}
   \@starttoc{lot}}
\let\l@table\l@figure


\def\thebibliography#1{
   \addcontentsline{toc}
   {section}{References}\section*{References\markboth{}{}}
   \list{[\arabic{enumi}]}
        {\settowidth\labelwidth{[#1]}\leftmargin\labelwidth
         \advance\leftmargin\labelsep\usecounter{enumi}}}
\let\endthebibliography=\endlist


\newif\if@restonecol
\def\theindex{
   \@restonecoltrue\if@twocolumn\@restonecolfalse\fi
   \columnseprule \z@
   \columnsep 35pt\twocolumn[\section*{Index}]
   \markboth{}{}
   \thispagestyle{plain}\parindent\z@
   \parskip\z@ plus .3pt\relax
   \let\item\@idxitem}
\def\@idxitem{\par\hangindent 40pt}
\def\subitem{\par\hangindent 40pt \hspace*{20pt}}
\def\subsubitem{\par\hangindent 40pt \hspace*{30pt}}
\def\endtheindex{\if@restonecol\onecolumn\else\clearpage\fi}
\def\indexspace{\par \vskip 10pt plus 5pt minus 3pt\relax}


\def\footnoterule{
   \kern-1\p@
   \hrule width .4\columnwidth
   \kern .6\p@}
\long\def\@makefntext#1{
   \@setpar{\@@par\@tempdima \hsize
   \advance\@tempdima-10pt\parshape \@ne 10pt \@tempdima}\par
   \parindent 1em\noindent \hbox to \z@{\hss$^{\@thefnmark}$}#1}


\setcounter{topnumber}{2}
\def\topfraction{.7}
\setcounter{bottomnumber}{1}
\def\bottomfraction{.3}
\setcounter{totalnumber}{3}
\def\textfraction{.2}
\def\floatpagefraction{.5}
\setcounter{dbltopnumber}{2}
\def\dbltopfraction{.7}
\def\dblfloatpagefraction{.5}
\long\def\@makecaption#1#2{
   \vskip 10pt
   \setbox\@tempboxa\hbox{#1: #2}
   \ifdim \wd\@tempboxa >\hsize \unhbox\@tempboxa\par
   \else \hbox to\hsize{\hfil\box\@tempboxa\hfil}
   \fi}
\newcounter{figure}
\def\thefigure{\@arabic\c@figure}
\def\fps@figure{tbp}
\def\ftype@figure{1}
\def\ext@figure{lof}
\def\fnum@figure{Figure \thefigure}
\def\figure{\@float{figure}}
\let\endfigure\end@float
\@namedef{figure*}{\@dblfloat{figure}}
\@namedef{endfigure*}{\end@dblfloat}
\newcounter{table}
\def\thetable{\@arabic\c@table}
\def\fps@table{tbp}
\def\ftype@table{2}
\def\ext@table{lot}
\def\fnum@table{Table \thetable}
\def\table{\@float{table}}
\let\endtable\end@float
\@namedef{table*}{\@dblfloat{table}}
\@namedef{endtable*}{\end@dblfloat}


\def\maketitle{
   \par
   \begingroup
      \def\thefootnote{\fnsymbol{footnote}}
      \def\@makefnmark{\hbox to 0pt{$^{\@thefnmark}$\hss}}
      \if@twocolumn \twocolumn[\@maketitle]
      \else \newpage \global\@topnum\z@ \@maketitle
      \fi
      \thispagestyle{plain}
      \@thanks
   \endgroup
   \setcounter{footnote}{0}
   \let\maketitle\relax
   \let\@maketitle\relax
   \gdef\@thanks{}
   \gdef\@author{}
   \gdef\@title{}
   \let\thanks\relax}
\def\@maketitle{
   \newpage
   \null
   \vskip 2em
   \begin{center}{\LARGE \@title \par}
      \vskip 1.5em
      {\large \lineskip .5em \begin{tabular}[t]{c}\@author \end{tabular}\par}
      \vskip 1em {\large \@date}
   \end{center}
   \par
   \vskip 1.5em}
\def\abstract{
   \if@twocolumn \section*{Abstract}
   \else
      \small
      \begin{center} {\bf Abstract\vspace{-.5em}\vspace{0pt}} \end{center}
      \quotation
   \fi}
\def\endabstract{\if@twocolumn\else\endquotation\fi}


\mark{{}{}}
\if@twoside
   \def\ps@headings{
      \def\@oddfoot{Rosetta Doc. \@RosDocNr\hfil \@RosDate}
      \def\@evenfoot{Rosetta Doc. \@RosDocNr\hfil \@RosDate}
      \def\@evenhead{\rm\thepage\hfil \sl \rightmark}
      \def\@oddhead{\hbox{}\sl \leftmark \hfil\rm\thepage}
      \def\sectionmark##1{\markboth {}{}}
      \def\subsectionmark##1{}}
\else
   \def\ps@headings{
      \def\@oddfoot{Rosetta Doc. \@RosDocNr\hfil \@RosDate}
      \def\@evenfoot{Rosetta Doc. \@RosDocNr\hfil \@RosDate}
      \def\@oddhead{\hbox{}\sl \rightmark \hfil \rm\thepage}
      \def\sectionmark##1{\markboth {}{}}
      \def\subsectionmark##1{}}
\fi
\def\ps@myheadings{
   \def\@oddhead{\hbox{}\sl\@rhead \hfil \rm\thepage}
   \def\@oddfoot{}
   \def\@evenhead{\rm \thepage\hfil\sl\@lhead\hbox{}}
   \def\@evenfoot{}
   \def\sectionmark##1{}
   \def\subsectionmark##1{}}


\def\today{
   \ifcase\month\or January\or February\or March\or April\or May\or June\or
      July\or August\or September\or October\or November\or December
   \fi
   \space\number\day, \number\year}


\ps@plain \pagenumbering{arabic} \onecolumn \if@twoside\else\raggedbottom\fi




% the Rosetta title page
\newcommand{\MakeRosTitle}{
   \begin{titlepage}
      \begin{large}
     \begin{figure}[t]
        \begin{picture}(405,100)(0,0)
           \put(0,100){\line(1,0){404}}
           \put(0,75){Project {\bf Rosetta}}
           \put(93.5,75){:}
           \put(108,75){Machine Translation}
           \put(0,50){Topic}
           \put(93.5,50){:}
           \put(108,50){\@RosTopic}
           \put(0,30){\line(1,0){404}}
        \end{picture}
     \end{figure}
     \bigskip
     \bigskip
     \begin{list}{-}{\setlength{\leftmargin}{3.0cm}
             \setlength{\labelwidth}{2.7cm}
             \setlength{\topsep}{2cm}}
        \item [{\rm Title \hfill :}] {{\bf \@RosTitle}}
        \item [{\rm Author \hfill :}] {\@RosAuthor}
        \bigskip
        \bigskip
        \bigskip
        \item [{\rm Doc.Nr. \hfill :}] {\@RosDocNr}
        \item [{\rm Date \hfill :}] {\@RosDate}
        \item [{\rm Status \hfill :}] {\@RosStatus}
        \item [{\rm Supersedes \hfill :}] {\@RosSupersedes}
        \item [{\rm Distribution \hfill :}] {\@RosDistribution}
        \item [{\rm Clearance \hfill :}] {\@RosClearance}
        \item [{\rm Keywords \hfill :}] {\@RosKeywords}
     \end{list}
      \end{large}
      \title{\@RosTitle}
      \begin{figure}[b]
     \begin{picture}(404,64)(0,0)
        \put(0,64){\line(1,0){404}}
        \put(0,-4){\line(1,0){404}}
        \put(0,59){\line(1,0){42}}
        \begin{small}
        \put(3,48){\sf PHILIPS}
        \end{small}
        \put(0,23){\line(0,1){36}}
        \put(42,23){\line(0,1){36}}
        \put(21,23){\oval(42,42)[bl]}
        \put(21,23){\oval(42,42)[br]}
        \put(21,23){\circle{40}}
        \put(4,33){\line(1,0){10}}
        \put(9,28){\line(0,1){10}}
        \put(9,36){\line(1,0){6}}
        \put(12,33){\line(0,1){6}}
        \put(29,13){\line(1,0){10}}
        \put(34,8){\line(0,1){10}}
        \put(28,10){\line(1,0){6}}
        \put(31,7){\line(0,1){6}}

        \put(1,21){\line(1,0){0.5}}
        \put(1.5,21.3){\line(1,0){0.5}}
        \put(2,21.6){\line(1,0){0.5}}
        \put(2.5,21.9){\line(1,0){0.5}}
        \put(3,22.1){\line(1,0){0.5}}
        \put(3.5,22.3){\line(1,0){0.5}}
        \put(4,22.5){\line(1,0){0.5}}
        \put(4.5,22.7){\line(1,0){0.5}}
        \put(5,22.8){\line(1,0){0.5}}
        \put(5.5,22.9){\line(1,0){0.5}}
        \put(6,23){\line(1,0){0.5}}
        \put(6.5,22.9){\line(1,0){0.5}}
        \put(7,22.8){\line(1,0){0.5}}
        \put(7.5,22.7){\line(1,0){0.5}}
        \put(8,22.5){\line(1,0){0.5}}
        \put(8.5,22.3){\line(1,0){0.5}}
        \put(9,22.1){\line(1,0){0.5}}
        \put(9.5,21.9){\line(1,0){0.5}}
        \put(10,21.6){\line(1,0){0.5}}
        \put(10.5,21.3){\line(1,0){0.5}}

        \put(1,23){\line(1,0){0.5}}
        \put(1.5,23.3){\line(1,0){0.5}}
        \put(2,23.6){\line(1,0){0.5}}
        \put(2.5,23.9){\line(1,0){0.5}}
        \put(3,24.1){\line(1,0){0.5}}
        \put(3.5,24.3){\line(1,0){0.5}}
        \put(4,24.5){\line(1,0){0.5}}
        \put(4.5,24.7){\line(1,0){0.5}}
        \put(5,24.8){\line(1,0){0.5}}
        \put(5.5,24.9){\line(1,0){0.5}}
        \put(6,25){\line(1,0){0.5}}
        \put(6.5,24.9){\line(1,0){0.5}}
        \put(7,24.8){\line(1,0){0.5}}
        \put(7.5,24.7){\line(1,0){0.5}}
        \put(8,24.5){\line(1,0){0.5}}
        \put(8.5,24.3){\line(1,0){0.5}}
        \put(9,24.1){\line(1,0){0.5}}
        \put(9.5,23.9){\line(1,0){0.5}}
        \put(10,23.6){\line(1,0){0.5}}
        \put(10.5,23.3){\line(1,0){0.5}}

        \put(1,25){\line(1,0){0.5}}
        \put(1.5,25.3){\line(1,0){0.5}}
        \put(2,25.6){\line(1,0){0.5}}
        \put(2.5,25.9){\line(1,0){0.5}}
        \put(3,26.1){\line(1,0){0.5}}
        \put(3.5,26.3){\line(1,0){0.5}}
        \put(4,26.5){\line(1,0){0.5}}
        \put(4.5,26.7){\line(1,0){0.5}}
        \put(5,26.8){\line(1,0){0.5}}
        \put(5.5,26.9){\line(1,0){0.5}}
        \put(6,27){\line(1,0){0.5}}
        \put(6.5,26.9){\line(1,0){0.5}}
        \put(7,26.8){\line(1,0){0.5}}
        \put(7.5,26.7){\line(1,0){0.5}}
        \put(8,26.5){\line(1,0){0.5}}
        \put(8.5,26.3){\line(1,0){0.5}}
        \put(9,26.1){\line(1,0){0.5}}
        \put(9.5,25.9){\line(1,0){0.5}}
        \put(10,25.6){\line(1,0){0.5}}
        \put(10.5,25.3){\line(1,0){0.5}}

        \put(11,21){\line(1,0){0.5}}
        \put(11.5,20.7){\line(1,0){0.5}}
        \put(12,20.4){\line(1,0){0.5}}
        \put(12.5,20.1){\line(1,0){0.5}}
        \put(13,19.9){\line(1,0){0.5}}
        \put(13.5,19.7){\line(1,0){0.5}}
        \put(14,19.5){\line(1,0){0.5}}
        \put(14.5,19.3){\line(1,0){0.5}}
        \put(15,19.2){\line(1,0){0.5}}
        \put(15.5,19.1){\line(1,0){0.5}}
        \put(16,19){\line(1,0){0.5}}
        \put(16.5,19.1){\line(1,0){0.5}}
        \put(17,19.2){\line(1,0){0.5}}
        \put(17.5,19.3){\line(1,0){0.5}}
        \put(18,19.5){\line(1,0){0.5}}
        \put(18.5,19.7){\line(1,0){0.5}}
        \put(19,19.9){\line(1,0){0.5}}
        \put(19.5,20.1){\line(1,0){0.5}}
        \put(20,20.4){\line(1,0){0.5}}
        \put(20.5,20.7){\line(1,0){0.5}}

        \put(11,23){\line(1,0){0.5}}
        \put(11.5,22.7){\line(1,0){0.5}}
        \put(12,22.4){\line(1,0){0.5}}
        \put(12.5,22.1){\line(1,0){0.5}}
        \put(13,21.9){\line(1,0){0.5}}
        \put(13.5,21.7){\line(1,0){0.5}}
        \put(14,21.5){\line(1,0){0.5}}
        \put(14.5,21.3){\line(1,0){0.5}}
        \put(15,21.2){\line(1,0){0.5}}
        \put(15.5,21.1){\line(1,0){0.5}}
        \put(16,21){\line(1,0){0.5}}
        \put(16.5,21.1){\line(1,0){0.5}}
        \put(17,21.2){\line(1,0){0.5}}
        \put(17.5,21.3){\line(1,0){0.5}}
        \put(18,21.5){\line(1,0){0.5}}
        \put(18.5,21.7){\line(1,0){0.5}}
        \put(19,21.9){\line(1,0){0.5}}
        \put(19.5,22.1){\line(1,0){0.5}}
        \put(20,22.4){\line(1,0){0.5}}
        \put(20.5,22.7){\line(1,0){0.5}}

        \put(11,25){\line(1,0){0.5}}
        \put(11.5,24.7){\line(1,0){0.5}}
        \put(12,24.4){\line(1,0){0.5}}
        \put(12.5,24.1){\line(1,0){0.5}}
        \put(13,23.9){\line(1,0){0.5}}
        \put(13.5,23.7){\line(1,0){0.5}}
        \put(14,23.5){\line(1,0){0.5}}
        \put(14.5,23.3){\line(1,0){0.5}}
        \put(15,23.2){\line(1,0){0.5}}
        \put(15.5,23.1){\line(1,0){0.5}}
        \put(16,23){\line(1,0){0.5}}
        \put(16.5,23.1){\line(1,0){0.5}}
        \put(17,23.2){\line(1,0){0.5}}
        \put(17.5,23.3){\line(1,0){0.5}}
        \put(18,23.5){\line(1,0){0.5}}
        \put(18.5,23.7){\line(1,0){0.5}}
        \put(19,23.9){\line(1,0){0.5}}
        \put(19.5,24.1){\line(1,0){0.5}}
        \put(20,24.4){\line(1,0){0.5}}
        \put(20.5,24.7){\line(1,0){0.5}}

        \put(21,21){\line(1,0){0.5}}
        \put(21.5,21.3){\line(1,0){0.5}}
        \put(22,21.6){\line(1,0){0.5}}
        \put(22.5,21.9){\line(1,0){0.5}}
        \put(23,22.1){\line(1,0){0.5}}
        \put(23.5,22.3){\line(1,0){0.5}}
        \put(24,22.5){\line(1,0){0.5}}
        \put(24.5,22.7){\line(1,0){0.5}}
        \put(25,22.8){\line(1,0){0.5}}
        \put(25.5,23.9){\line(1,0){0.5}}
        \put(26,23){\line(1,0){0.5}}
        \put(26.5,22.9){\line(1,0){0.5}}
        \put(27,22.8){\line(1,0){0.5}}
        \put(27.5,22.7){\line(1,0){0.5}}
        \put(28,22.5){\line(1,0){0.5}}
        \put(28.5,22.3){\line(1,0){0.5}}
        \put(29,22.1){\line(1,0){0.5}}
        \put(29.5,21.9){\line(1,0){0.5}}
        \put(30,21.6){\line(1,0){0.5}}
        \put(30.5,21.3){\line(1,0){0.5}}

        \put(21,23){\line(1,0){0.5}}
        \put(21.5,23.3){\line(1,0){0.5}}
        \put(22,23.6){\line(1,0){0.5}}
        \put(22.5,23.9){\line(1,0){0.5}}
        \put(23,24.1){\line(1,0){0.5}}
        \put(23.5,24.3){\line(1,0){0.5}}
        \put(24,24.5){\line(1,0){0.5}}
        \put(24.5,24.7){\line(1,0){0.5}}
        \put(25,24.8){\line(1,0){0.5}}
        \put(25.5,24.9){\line(1,0){0.5}}
        \put(26,25){\line(1,0){0.5}}
        \put(26.5,24.9){\line(1,0){0.5}}
        \put(27,24.8){\line(1,0){0.5}}
        \put(27.5,24.7){\line(1,0){0.5}}
        \put(28,24.5){\line(1,0){0.5}}
        \put(28.5,24.3){\line(1,0){0.5}}
        \put(29,24.1){\line(1,0){0.5}}
        \put(29.5,23.9){\line(1,0){0.5}}
        \put(30,23.6){\line(1,0){0.5}}
        \put(30.5,23.3){\line(1,0){0.5}}

        \put(21,25){\line(1,0){0.5}}
        \put(21.5,25.3){\line(1,0){0.5}}
        \put(22,25.6){\line(1,0){0.5}}
        \put(22.5,25.9){\line(1,0){0.5}}
        \put(23,26.1){\line(1,0){0.5}}
        \put(23.5,26.3){\line(1,0){0.5}}
        \put(24,26.5){\line(1,0){0.5}}
        \put(24.5,26.7){\line(1,0){0.5}}
        \put(25,26.8){\line(1,0){0.5}}
        \put(25.5,26.9){\line(1,0){0.5}}
        \put(26,27){\line(1,0){0.5}}
        \put(26.5,26.9){\line(1,0){0.5}}
        \put(27,26.8){\line(1,0){0.5}}
        \put(27.5,26.7){\line(1,0){0.5}}
        \put(28,26.5){\line(1,0){0.5}}
        \put(28.5,26.3){\line(1,0){0.5}}
        \put(29,26.1){\line(1,0){0.5}}
        \put(29.5,25.9){\line(1,0){0.5}}
        \put(30,25.6){\line(1,0){0.5}}
        \put(30.5,25.3){\line(1,0){0.5}}

        \put(31,21){\line(1,0){0.5}}
        \put(31.5,20.7){\line(1,0){0.5}}
        \put(32,20.4){\line(1,0){0.5}}
        \put(32.5,20.1){\line(1,0){0.5}}
        \put(33,19.9){\line(1,0){0.5}}
        \put(33.5,19.7){\line(1,0){0.5}}
        \put(34,19.5){\line(1,0){0.5}}
        \put(34.5,19.3){\line(1,0){0.5}}
        \put(35,19.2){\line(1,0){0.5}}
        \put(35.5,19.1){\line(1,0){0.5}}
        \put(36,19){\line(1,0){0.5}}
        \put(36.5,19.1){\line(1,0){0.5}}
        \put(37,19.2){\line(1,0){0.5}}
        \put(37.5,19.3){\line(1,0){0.5}}
        \put(38,19.5){\line(1,0){0.5}}
        \put(38.5,19.7){\line(1,0){0.5}}
        \put(39,19.9){\line(1,0){0.5}}
        \put(39.5,20.1){\line(1,0){0.5}}
        \put(40,20.4){\line(1,0){0.5}}
        \put(40.5,20.7){\line(1,0){0.5}}

        \put(31,23){\line(1,0){0.5}}
        \put(31.5,22.7){\line(1,0){0.5}}
        \put(32,22.4){\line(1,0){0.5}}
        \put(32.5,22.1){\line(1,0){0.5}}
        \put(33,21.9){\line(1,0){0.5}}
        \put(33.5,21.7){\line(1,0){0.5}}
        \put(34,21.5){\line(1,0){0.5}}
        \put(34.5,21.3){\line(1,0){0.5}}
        \put(35,21.2){\line(1,0){0.5}}
        \put(35.5,21.1){\line(1,0){0.5}}
        \put(36,21){\line(1,0){0.5}}
        \put(36.5,21.1){\line(1,0){0.5}}
        \put(37,21.2){\line(1,0){0.5}}
        \put(37.5,21.3){\line(1,0){0.5}}
        \put(38,21.5){\line(1,0){0.5}}
        \put(38.5,21.7){\line(1,0){0.5}}
        \put(39,21.9){\line(1,0){0.5}}
        \put(39.5,22.1){\line(1,0){0.5}}
        \put(40,22.4){\line(1,0){0.5}}
        \put(40.5,22.7){\line(1,0){0.5}}

        \put(31,25){\line(1,0){0.5}}
        \put(31.5,24.7){\line(1,0){0.5}}
        \put(32,24.4){\line(1,0){0.5}}
        \put(32.5,24.1){\line(1,0){0.5}}
        \put(33,23.9){\line(1,0){0.5}}
        \put(33.5,23.7){\line(1,0){0.5}}
        \put(34,23.5){\line(1,0){0.5}}
        \put(34.5,23.3){\line(1,0){0.5}}
        \put(35,23.2){\line(1,0){0.5}}
        \put(35.5,23.1){\line(1,0){0.5}}
        \put(36,23){\line(1,0){0.5}}
        \put(36.5,23.1){\line(1,0){0.5}}
        \put(37,23.2){\line(1,0){0.5}}
        \put(37.5,23.3){\line(1,0){0.5}}
        \put(38,23.5){\line(1,0){0.5}}
        \put(38.5,23.7){\line(1,0){0.5}}
        \put(39,23.9){\line(1,0){0.5}}
        \put(39.5,24.1){\line(1,0){0.5}}
        \put(40,24.4){\line(1,0){0.5}}
        \put(40.5,24.7){\line(1,0){0.5}}
        \begin{large}
           \put(60,45){Philips Research Laboratories}
           \put(60,30){\copyright\ 1988 Nederlandse Philips Bedrijven B.V.}
        \end{large}
     \end{picture}
      \end{figure}
      \newpage
      \pagenumbering{roman}
      \tableofcontents
      \newpage
      \pagenumbering{arabic}
   \end{titlepage}
}
\title{}
\topmargin 0pt
\oddsidemargin 36pt
\evensidemargin 36pt
\textheight 600pt
\textwidth 405pt
\pagestyle{headings}
\newcommand{\@RosTopic}{General}
\newcommand{\@RosTitle}{-}
\newcommand{\@RosAuthor}{-}
\newcommand{\@RosDocNr}{}
\newcommand{\@RosDate}{\today}
\newcommand{\@RosStatus}{informal}
\newcommand{\@RosSupersedes}{-}
\newcommand{\@RosDistribution}{Project}
\newcommand{\@RosClearance}{Project}
\newcommand{\@RosKeywords}{}
\newcommand{\RosTopic}[1]{\renewcommand{\@RosTopic}{#1}}
\newcommand{\RosTitle}[1]{\renewcommand{\@RosTitle}{#1}}
\newcommand{\RosAuthor}[1]{\renewcommand{\@RosAuthor}{#1}}
\newcommand{\RosDocNr}[1]{\renewcommand{\@RosDocNr}{#1}}
\newcommand{\RosDate}[1]{\renewcommand{\@RosDate}{#1}}
\newcommand{\RosStatus}[1]{\renewcommand{\@RosStatus}{#1}}
\newcommand{\RosSupersedes}[1]{\renewcommand{\@RosSupersedes}{#1}}
\newcommand{\RosDistribution}[1]{\renewcommand{\@RosDistribution}{#1}}
\newcommand{\RosClearance}[1]{\renewcommand{\@RosClearance}{#1}}
\newcommand{\RosKeywords}[1]{\renewcommand{\@RosKeywords}{#1}}

