
\documentstyle{Rosetta}
\begin{document}
   \RosTopic{Linguistics}
   \RosTitle{Idioms and the Dictionary}
   \RosAuthor{Andr\'{e} Schenk}
   \RosDocNr{194}
   \RosDate{\today}
   \RosStatus{concept}
   \RosSupersedes{-}
   \RosDistribution{Project}
   \RosClearance{Project}
   \RosKeywords{Idioms, Dictionary, B-LEX}
   \MakeRosTitle
%
%

\hyphenation{RSTART-VERB proposition-substitution}

\section{Introduction}

\paragraph{Intro}
This paper discusses some alternative ways to list idioms in the dictionary. 
It will be assumed that idioms correspond to a basic meaning, i.e. the meaning 
of an idiom is represented in the semantic derivation tree as one node. This 
contrasts with proposals made for example in GPSG, in which parts of an idiom 
carry meaning.

\paragraph{Preview Theoretical Possibilities}
The assumption that an idiom corresponds to a basic meaning and that therefore
compositional semantics does not apply to its parts can be worked out in two
ways: 

\begin{itemize}
  \item Syntactic rules apply to form the syntactic surface tree of the idiom;
these rules are represented in the syntactic derivation tree, but do not get an
interpretation (in terms of the M-grammar formalism this means that the rules
that form the idiomatic constituent are represented in the syntactic, but not
in the semantic derivation tree); the idioms are listed as syntactic D-trees 
with its corresponding basic meaning in the transfer dictionary; this will be
called the D-tree method. 
  \item Idioms are listed as syntactic trees with its corresponding basic 
expression in the syntactic D-tree in the basic lexicon; this will be called
the S-tree method. 
\end{itemize}

\paragraph{Preview Solutions}
It will be shown that both theoretical options present difficulties, so a
third method will be given, which, from the point of view of transfer, is 
equivalent to the S-tree method. This option necessitates an extension of the 
basic lexicon. Note that for each method below several variants are
conceivable, which only differ in detail. 

\paragraph{Restrictions on M-grammar}
We assume that idioms are, at some point in the derivation, represented in
syntax as an S-tree in some canonical form. Restrictions have to be imposed on
the grammar in such a way that it can operate on such idiomatic S-trees (note
that this is not the case for Global Transfer (section~\ref{GT})). 

\paragraph{Criteria}
Solutions to the idioms and the dictionary problem must satisfy certain 
criteria against which they can be evaluated. These criteria are:
\begin{enumerate}
  \item Restrictions on S-trees. The form in which canonical idiomatic S-trees
can occur is not arbitrary, since (i) idioms are well-formed (i.e. in general, 
well-known exceptions are for example {\em by and large} and {\em trip the
light fantastic}) and (ii) idioms are not of an arbitrary size (eg. idioms in
English have only one verb). These restrictions should be imposed by and follow 
from the grammar. 
  \item Concurrence with formalism. Any method for listing idioms has 
implications for the formalism (except for the S-tree method in
section~\ref{STM}); an extension of the formalism should concur with the
existing formalism in an acceptable way. We are well aware of the fact that 
this is an intuitive notion, but it is hard to state in advance which 
extensions of the formalism are acceptable and which are not.
\end{enumerate}

\paragraph{Preview Sectioning}
The next section gives instances of the D-tree method.
In section~\ref{ST1} an instance of the D-tree method will be given in which 
the parts of the derivation tree that constitute the idiom are scattered. 
Section~\ref{ST2} presents a D-tree method in which the idiomatic 
part of the syntactic D-tree is local. Section~\ref{STM} gives an S-tree
method. From each of this methods the drawbacks will be shown. It will be shown
that the extended S-tree method in section~\ref{EST} does not have these
drawbacks. In the appendix a formalization of the extended S-tree method will 
be given.

\section{D-tree Method}

\subsection {Global Transfer}
\label{ST1}

\paragraph{Intro}
This section will discuss an instance of the D-tree method. We will assume an
M-grammar that is not specifically suited for idioms, i.e. we will try and
adapt an existing M-grammar in such a way that it accomodates idioms. To
illustrate we will give a (partial) derivation of a literal sentence (i.e. {\em
the beans were spilled by John}) and then we will use the same partial
derivation to show how the grammar could be adapted if the sentence were
read idiomatically. 

\paragraph{What is the figure about}
Figure~\ref{ED} gives part of the derivation of the sentence {\em The beans were
spilled by John} (in fact only an intermediate result is derived, i.e. {\em the 
beans be spill by VAR$_1$}). Other rules and transformations (for example the
rules that specify the tense) intervene between the ones represented here, but,
since they are irrelevant in the present discussion, they have been omitted.
The STARTCLAUSE rule and the VERBPATTERN transformation specify the syntactic
environment of the VERB {\em spill}, viz. VAR$_1$ becomes the subject and
VAR$_2$ the object of the CLAUSE. RVOICE upto TNP-RAISING take care of
passivization. RVOICE moves the subject into object position of the {\em by}
PREPP and puts the VERB into the passive form; RAUXSPELLING inserts the passive
auxilliary {\em be}. TNP-RAISING moves the object into subject position.
RNOWHSHIFTSUBSTITUTION1 substitutes the argument {\em John} for the VARiable
indexed 1; RNOWHSHIFTSUBSTITUTION2 substitutes the argument {\em John} for the
VARiable indexed 2. 

\begin{figure}[htb]
\par
\begingroup
\def\par{\leavevmode\endgraf}
\obeylines\obeyspaces
{\obeyspaces\global\let =\ }
             RNOWHSHIFTSUBST2,2

             RNOWHSHIFTSUBST1,1      RNP1

 RNP2           TNP-RAISING           BNOUN
                                                     bean
BNOUN	       RAUXSPELLING  
 John
                    RVOICE

                 TVERBPATTERN2

                 RSTARTVERB2

              BVERB  VAR$_1$  VAR$_2$
               spill
\endgroup
\caption{Example Derivation}
\label{ED}
\end{figure}

\paragraph{And now for idioms}
Suppose that we, partially, derive the sentence {\em The beans were spilled by
John} again, but now in its idiomatic reading. The derivation will be the same
as in the previous example; some rules are applied only in order to construct
the canonical form of the idiom. 

\paragraph{What is the figure about}
As argued above, an idiom should correspond to a basic meaning. This basic
meaning is represented as B$_1$ in figure~\ref{GT}. B$_1$ corresponds to the
rules and transformations that are demarcated by the frame, which are the rules
that are applicable to construct the idiomatic S-tree. So, on the basis of the
transfer dictionary basic meaning B$_1$ is translated into syntactic D-trees
which are placed in between other parts of the syntactic D-tree. It is assumed
that in figure~\ref{GT} the semantic counterparts of the passive rules are in
the {\em other part of semantic D-tree} at the top. 

\begin{figure}[htb]
\par
\begingroup
\def\par{\leavevmode\endgraf}
\obeylines\obeyspaces
{\obeyspaces\global\let =\ }
          other part of semantic D-tree

                           B$_1$
                           $\Updownarrow$
          other part of syntactic D-tree

             RNOWHSHIFTSUBST2,2

             RNOWHSHIFTSUBST1,1     RNP1

 RNP2           TNP-RAISING            BNOUN
                                                      bean
BNOUN          	RAUXSPELLING
 John
                       RVOICE

                   TVERBPATTERN2

                    RSTARTVERB2

                 BVERB  VAR$_1$  VAR$_2$
                   spill
\endgroup
\caption{Global Transfer}
\label{GT}
\end{figure}

\paragraph{Discussion}
As can be seen from the example, there is no guarantee that the syntactic
D-tree for the idiom is local. Other rules can intervene between the parts that
are idiomatic in the syntactic D-tree. The rules that build the idiom are
listed in the transfer dictionary. On the basis of that dictionary analytical
transfer has to recognise the idiomatic parts in the D-tree and generative
transfer  has to generate these parts in the correct places. 

\paragraph{Analysis}
It is easy to conceive of a mechanism in analysis to achieve global transfer;
the transfer rule that looks for the idiom on the basis of the transfer
dictionary skips nodes that are not part of the idiom and replaces the parts of
the derivation tree that form the idiom by a basic meaning in the semantic
derivation tree. The other, non-idiomatic rules are glued together on top of
the basic meaning. 

\paragraph{Generation}
In generation it is harder to conceive of such a mechanism; a set of syntactic
D-trees is generated; one element of this set is the correct one to make the
idiomatic sentence. The others will have to be filtered out by the grammar. 

\paragraph{A problem}
A consequence of this method is that restrictions have to be imposed on the
parts of the syntactic D-tree that intervene between the idiom parts, cf. the
sentence {\em the beans were too precious to spill}, which is analysed as
(stated in an informal manner) {\em x$_1$ were too precious for x$_2$ to spill
x$_1$}. Under the global transfer approach this sentence, if no special
measures were taken, also gets an idiomatic interpretation, while it is
non-idiomatic only. 

\paragraph{Evaluation}
Global transfer violates criterion 1 since no restrictions can be imposed on
the format of idioms by the grammar. Furthermore, it violates criterion 2 in
that the formalism needed does not concur with the existing formalism in an
acceptable way. Many rules in several combinations can be applicable in between
the idiom rules. 

\paragraph{Comment}
Since this method does not seem to be workable no attempt has been made to give
an example of a dictionary entry as will be done for each of the following
methods. 

\paragraph{Preview}
The next section will discuss a D-tree methods in which the derivation tree
that constitutes the idiom is local. By using a subgrammar for the formation 
of idioms restrictions can be imposed on the format of idioms.

\subsection {Local Transfer}
\label{ST2}

\paragraph{Intro}
This section discusses a D-tree method in which transfer can be local as a
result of the use of subgrammars. Transfer is local since the part in the
syntactic derivation that constitutes the idiom is not scattered as under the
global transfer approach. Under this method there is a special subgrammar that
creates the idiomatic S-tree. It contains the following rule and transformation
classes:\\ 

SUBGRAMMAR: IDIOMformation:\\

RC: Startverbrules . TC: Verbpatternrules . \{RC: Advvarrules\} . \{RC:
propositionsubstitution\} . TC: PROPOKrules  .  TC: Controlrules .  \{RC:
nowhshiftsubstitution\} 

\paragraph{Characteristics of subgrammar}
This subgrammar is for Dutch and is similar to the subgrammars that treat
'graag' cases. The idea is that the subgrammar creates a canonical clausal
constituent structure for idioms to which, in particular, time and mood have
not been assigned yet. Basically, the same rules and transformations are used
here as in other subgrammars. Note that the distinction between rules and
transformations does not have the same implications here as in the other
subgrammars, since none of the operations has meaning. The subgrammar for
English will presumably be the same; possibly the PROPOKrules are not necessary
in English. 

\paragraph{This subgrammar is not enough}
Next to this subgrammar for clauses other subgrammars will be needed for
PREPPs, NPs, etc. (some of which are used below in the example).

\paragraph{Free arguments}
The free arguments to an idiom are translated together with the idiom. So, 
instead of a function, a function application is translated.

\paragraph{What is the figure about}
An example is given in Figure~\ref{LT2}, where on the basis of the transfer
dictionary the basic meaning B$_1$ is translated together with the free
argument (VAR$_1$) into the part of the syntactic derivation tree that makes
the canonical structure of {\em VAR$_1$ de plaats poets} (note that obligatory
rules that apply vacuously are not included in the D-tree representation). In
this D-tree the rules and transformations starting from RSTARTVERB2 up to
RNOWHSHIFTSUBST construct the idiomatic S-tree. RSTARTVERB2 has a BVERB and two
variables (VAR$_1$ and VAR$_2$) as arguments. The VERBPATTERN transformation
creates the correct S-structure according to the verbpattern. RNOWHSHIFTSUBST,2
substitutes the NP {\em de plaat}, which is made by RNP1 on the basis of {\em
plaat}, for the formal variable VAR$_2$. The result of this is input to the
non-idiomatic subgrammars. The TIDVERBPATTERN transformation (not included
here) applies vacuously. 

\begin{figure}[htb]
\par
\begingroup
\def\par{\leavevmode\endgraf}
\obeylines\obeyspaces
{\obeyspaces\global\let =\ }
         other part of semantic D-tree

             LIDSTARTVERB1

                   B$_1$    LVAR$_1$
                        $\Updownarrow$
         other part of syntactic D-tree

             RIDSTARTVERB1 (5)
          
            RNOWHSHIFTSUBST,2 (3)

              TVERBPATTERN2 (2)    RNP1 (4)

              RSTARTVERB2 (1)       BNOUN
                                                  plaat
             BVERB  VAR$_1$ VAR$_2$
              poets
\endgroup
\caption{Local Transfer}
\label{LT2}
\end{figure}

\paragraph{Intermediate results of derivation}
The numbers between brackets behind the rules in the example correspond to the
numbers below. There the S-trees after application of the rules are given. 


\par
\begingroup
\def\par{\leavevmode\endgraf}
\obeylines\obeyspaces
{\obeyspaces\global\let =\ }
1) VERBPPROP[subj/VAR$_1$,
                       pred/VERBP[arg/VAR$_2$,
                                         head/VERB(poets)]]
2) VERBPPROP[subj/VAR$_1$,
                       pred/VERBP[obj/VAR$_2$,
                                         head/VERB(poets)]]
3) VERBPPROP[subj/VAR$_1$,
                       pred/VERBP[obj/NP[det/ART(de),
                                                   head/NOUN(plaat)],
                                       head/VERB(poets)]]
4) NP[det/ART(de),
          head/NOUN(plaat)]
5) VERBPPROP[subj/VAR$_1$,
                       pred/VERBP[obj/NP[det/ART(de),
                                                   head/NOUN(plaat)],
                                       head/VERB(poets)]]
\endgroup

\paragraph{Dictionary Entry}
The dictionary for {\em de plaat poets} is given in figure~\ref{DILT2}.

\paragraph{Formal variables}
The V's in figure~\ref{DILT2} are formal variables. These variables are 
placeholders for variables corresponding to free arguments or for variables 
that trigger substitution rules that are part of the idiom.

\begin{figure}[htb]
\par
\begingroup
\def\par{\leavevmode\endgraf}
\obeylines\obeyspaces
{\obeyspaces\global\let =\ }
                       B$_1$   V$_1$
                            $\Updownarrow$
              RNOWHSHIFTSUBST,2

              TVERBPATTERN2        RNP1

              RSTARTVERB2           BNOUN
                                                plaat
             BVERB    V$_1$   V$_2$
              poets
\endgroup
\caption{Dictionary Entry Local Transfer}
\label{DILT2}
\end{figure}

\paragraph{Evaluation}
The method discussed in this section does not have the drawbacks of the
previous method. The subgrammar IDIOMformation is set up in such a way that it
defines the set of possible idiomatic S-trees. So, criterion 1 is satisfied.
The formalism has to be extended slightly only, so that criterion 2 is met.
Yet, certain objections can be made made to this method: 
\begin{itemize}
  \item The first is that the arguments have to be translated together with the
idiom, so that a function application and not a function is translated as is
customary. 
  \item The second objection holds for the D-tree method in general and that is 
that structural transfer is necessary. One implication of this is that in 
analytical transfer also syntactically ill-formed derivation trees have to be 
filtered out, while normally there is no filtering at all. An example of an
ill-formed D-tree is given in figure~\ref{IFDT}, in which instead of the object
NP, which would have been correct, the subject NP has not been substituted.
\end{itemize}
\begin{figure}[htb]
\par
\begingroup
\def\par{\leavevmode\endgraf}
\obeylines\obeyspaces
{\obeyspaces\global\let =\ }
         other part of semantic D-tree

             LIDSTARTVERB1

                   B$_1$    LVAR$_1$
                        $\Updownarrow$
         other part of syntactic D-tree

             RIDSTARTVERB1
          
            RNOWHSHIFTSUBST,1

              TVERBPATTERN2        RNP1 

              RSTARTVERB2           BNOUN
                                                  jongen
             BVERB  VAR$_1$ VAR$_2$
              poets
\endgroup
\caption{Ill-formed D-tree}
\label{IFDT}
\end{figure}

\paragraph{Preview}
The S-tree method as discussed in the next section does not have these
objections since it does not burden transfer, i.e. an idiomatic S-tree
corresponds to a basic expression which corresponds to a basic meaning. 

\section{S-tree Method}
\label{STM}

\paragraph{Intro}
Under the S-tree approach a basic meaning in the semantic D-tree is mapped onto
a basic expression in the syntactic D-tree. This holds for idiomatic and non-
idiomatic basic expressions; the difference under the S-tree approach
is that the idiomatic basic expression has an internal structure, while literal
basic expressions are terminal nodes. This internal structure represents the
constituent structure of the idiom. 

\paragraph{Example of idiomatic S-tree}
An example of such an idiomatic basic S-tree is given in figure~\ref{DIKTB} for
the idiom {\em kick the bucket}. 

\begin{figure}[htb]
\par
\begingroup
\def\par{\leavevmode\endgraf}
\obeylines\obeyspaces
{\obeyspaces\global\let =\ }
                     BVERB

                  subj   pred

                V$_1$     VERBP

                      head      obj

                    VERB         NP
                     kick
                               det    head

                            ART     NOUN
                             the      bucket
\endgroup
\caption{kick the bucket}
\label{DIKTB}
\end{figure}

\paragraph{Where to list idiomatic S-trees}
Basic expressions are listed in the basic lexicon of a grammar. This is also
the case for idioms. Figure~\ref{DIKTB} is an example of such an entry. The
idioms are represented as a canonical surface tree structure. A canonical
surface tree structure is the default tree structure for a certain sentence,
phrase, etc., i.e. the structure to which no syntactic transformations or rules
have applied. For example: if there is a passive transformation, the canonical
structure is in the active form. 

\paragraph{Formal Variables}
The V nodes indicate that in these positions actual variables have to be
substituted. The Vs in the idiomatic S-trees make them different from literal
S-trees, since literal S-trees do not have Vs. The Vs are placeholders for
VARs. How the link between Vs and VARs is made will be shown below. Note that
in figure~\ref{DIKTB} an informal notation is used; for example the attribute
value pairs at the nodes have been omitted. The top node, the BVERB, contains a
set of attribute-value pairs that specifies idiosyncratic properties of the
idiom. 

Figure~\ref{SM} shows the part of the grammar that creates the canonical form 
of the idiomatic S-tree.

\begin{figure}[htb]
\par
\begingroup
\def\par{\leavevmode\endgraf}
\obeylines\obeyspaces
{\obeyspaces\global\let =\ }
                       other part of D-tree
 
                         TIDVERBPATTERN

                           RSTARTVERB

         BVERB - - - - BVERB    VAR$_1$

    subj    pred

   V$_1$       VERBP

          head    obj

       VERB      NP
       kick
               det   head

            ART    NOUN
             the     bucket
\endgroup
\caption{S-tree Method 1}
\label{SM}
\end{figure}

\paragraph{What is the figure about}
The STARTVERB rule applies to a complex BVERB and the number of arguments. In
TIDVERBPATTERN the formal variable V$_1$ is replaced by the actual variable
VAR$_1$. There is a set of IDPATTERNs each element of which indicates where and
how for a certain of idiom free arguments are realized (for IDPATTERN cf.
below). The top category is changed into VERBPPROP. The S-tree that is the
result of the part of the grammar in figure~\ref{SM} is shown in
figure~\ref{RES}. 

\begin{figure}[htb]
\par
\begingroup
\def\par{\leavevmode\endgraf}
\obeylines\obeyspaces
{\obeyspaces\global\let =\ }
                  VERBPPROP

                  subj   pred

                VAR$_1$     VERBP

                      head      obj

                    VERB         NP
                     kick
                               det    head

                            ART     NOUN
                             the      bucket
\endgroup
\caption{kick the bucket}
\label{RES}
\end{figure}

\paragraph{IDPATTERN}
An example of an IDPATTERN is the one in which the free argument is realized in
subject position in {\em kick the bucket} and {\em spill the beans}. Another
example is the IDPATTERN where the first argument is realized in subject
position and the second in determiner position of the object in for example
{\em break someone's heart} and {\em pull someone's leg}. 

\paragraph{Analysis}
Analysis is the reverse of the process given above. RSTARTVERB matches the
idiomatic basic expression with the dictionary. 

\paragraph{Evaluation}
Criterion 1: restrictions on the form of idiomatic S-trees do not follow from
the grammar. As for criterion 2 the formalism does not have to be extended.
From the point of view of implementation the objection can be made that storage
of idiomatic S-trees is expensive, since the attribute value pairs of every
node (including the non-terminals) have to be specified in the dictionary
entry. 

\paragraph{Preview}
The specification of attribute values in the dictionary for every node is not
necessary under the D-tree method, since these values are rendered by rules 
for non-terminals. The method in the next section combines the merits of the
D-tree and the S-tree methods, because transfer is strictly local (srictly 
local means that nodes are translated into nodes) and idiomatic S-trees are 
rendered by rules.

\section{The Extended S-tree Method}
\label{EST}

\paragraph{Intro}
To overcome the problems given above another method will be presented, which
does not have these drawbacks. As far as translation is concerned the method is
equivalent to the S-tree method, i.e. transfer is local. The definition of the
basic lexicon is extended in such a way that the problems with the S-tree
method are done away with. Note that the name of this method is somewhat 
misleading, since it is given solely from the point of view of transfer. The 
alternative, somewhat misleading, name would be the lexical D-tree method.

\paragraph{Extension of B-LEX}
The basic lexicon (B-LEX) now comprises two parts; the first part consists of 
the basic expressions that are terminal; the second part of the complex basic 
expressions, viz. the idioms:
\begin{enumerate}
  \item BARE-LEX: This part consists of the set of basic expressions, that do
not have an internal structure, and is equivalent to what was previously called
B-LEX. 
  \item ID-LEX: This part defines the set of complex basic expressions; it is
divided into three subparts:
\begin{enumerate}
  \item ID-LEAVES: The set of basic expressions that are leaves of idioms.
  \item ID-GRAMMAR: The set of rules that define well-formed idiomatic S-trees.
  \item ID-DERIV-TREES: The set of rules, every element of which defines an 
idiomatic S-tree.
\end{enumerate}
\end{enumerate}

\paragraph{ID-LEAVES}
ID-LEAVES is a set of pairs (terminal S-trees, names of basic expressions).
ID-LEAVES is not the same as BARE-LEX since 
\begin{enumerate}
  \item there are some words in ID-LEAVES which do not occur in BARE-LEX, for
example {\em lurch} in {\em leave someone in the lurch}, 
  \item there are several words in BARE-LEX that do not occur in ID-LEAVES,
  \item the words in ID-LEAVES have other properties than their literal 
equivalents in BARE-LEX. The properties that are shared by the intersection of 
BARE-LEX and ID-LEAVES are form properties (i.e. properties related to the
morphological component).
\end{enumerate}

\paragraph{Comment}
The fact that BARE-LEX and ID-LEAVES have properties in common suggests that it 
should be possible to make generalisations. Note that this also holds for 
literals. Homonyms, like {\em bank} in its different meanings, share properties 
which are not expressed in the lexicon. The words in idioms are similar to 
words like {\em it} and {\em there} which sometimes do and sometimes do not 
carry meaning. An alternative definition of the lexicon, in which these facts 
can be expressed, will be presented in a forthcoming document.

\paragraph{ID-GRAMMAR}
The set ID-GRAMMAR is a subset of M-grammar rules. This is minimally necessary
to guarantee that literal rules are applicable to the S-trees that are the
result of ID-GRAMMAR, so that the surface structure of a sentence containing an
idiom is defined by the grammar. Furthermore this rule pack is restricted in
such a way that it only generates the idiomatic S-trees that are in accordance
with the restrictions on idiom formats.\\ 

ID-GRAMMAR consists of the following rules:\\

C: Startverbrules . C: Verbpatternrules . \{C: Advvarrules\} . \{C:
propositionsubstitution\} . C: PROPOKrules  .  C: Controlrules .  \{C:
nowhshiftsubstitution\} 

\paragraph{ID-DERIV-TREES}
ID-DERIV-TREES is a set of pairs (derivation tree, name of basic expression).
It specifies which rules in ID-GRAMMAR have to be applied to which words in
ID-LEAVES to form an idiomatic S-tree. 

\paragraph{Example dictionary entry}
An example of an entry in ID-LEX for {\em poets de plaat} is the following:\\

The part of ID-LEAVES that is necessary for this example (the article {\em de} 
is introduced syncategorematically):\\

poets\{...\}

plaat\{...\}\\

The ID-DERIV-TREE is represented in figure~\ref{IDDE}.
\begin{figure}[htb]
\par
\begingroup
\def\par{\leavevmode\endgraf}
\obeylines\obeyspaces
{\obeyspaces\global\let =\ }
              RNOWHSHIFTSUBST,2

              TVERBPATTERN2        RNP1

              RSTARTVERB2           BNOUN
                                                 plaat
             BVERB    V$_1$   V$_2$
              poets
\endgroup
\caption{ID-DERIV-TREE ID-LEX}
\label{IDDE}
\end{figure}

\paragraph{Replacement of formal variables}
The replacement of the formal variables (in the example V$_1$) is taken care of
in the grammar in the VERBPATTERN transformationclass in the same way as in
section~\ref{STM}. 

\paragraph{Evaluation}
Criterion 1 is satisfied since possible idiomatic S-trees are defined by
ID-GRAMMAR which is a subset of M-grammar. Criterion 2: the definition of B-LEX 
has to be extended. In the appendix it is shown that the formalism can be
extended elegantly. 

\section{Appendix}

DEFINITIONS EXTENDED S-TREE METHOD\\

\begin{tabbing}
aaa \= aaaaaaaaaaaaaaaaaaaaaaaaaaaaaaaaaaaaaaaaaaaaaaaaa \= aaaaaaaaaa \= \kill
t   \>: S-tree\\
w   \>: terminal S-tree\\
b   \>: Basic Expression\\
\underline{b} \>: Name of a Basic Expression\\
d   \>: Syntactic D-tree\\
\end{tabbing}


B-LEX = (BARE-LEX, ID-LEX)

ID-LEX = (ID-LEAVES, ID-GRAMMAR, ID-DERIV-TREES)\\


BARE-LEX = \{(w, \underline{b}) \}\\

ID-LEAVES = \{(w, \underline{b}) \}\\

ID-DERIV-TREES = \{ (d, \underline{b})\}\\

ID-GRAMMAR : ID-GRAMMAR consists of ID-PARSER and ID-GENERATOR. See definition
of M-GRAMMAR in doc. 101. The set of basic expressions for ID-GRAMMAR is
defined in ID-LEAVES\\ 

AN-B-LEX(t) =\\

\{ \underline{b} $|$ (t, \underline{b}) $\in$ BARE-LEX \} $\cup$\\

\{ \underline{b} $|$ d $\in$ ID-PARSER(t) 
      $\wedge$ (d,  \underline{b}) $\in$ ID-DERIV-TREES \}\\\\\\


GEN-B-LEX(\underline{b}) =\\

\{ t  $|$ (t, \underline{b}) $\in$ BARE-LEX \} $\cup$\\

\{ t  $|$ t $\in$ ID-GENERATOR(d) $\wedge$ 
(d,  \underline{b}) $\in$ ID-DERIV-TREES \}\\

\paragraph{Comment}
According to the above definitions ID-PARSER(t) renders a set of derivations of
which d is an element. The element d has to be in ID-DERIV-TREES and has the
name \underline{b}. This algorithm implies that the set \{d$|$d $\in$
ID-PARSER(t)\} has to be generated and checked against every d in
ID-DERIV-TREES, though the d's that are candidate are already known due to the
idiom technique in analytical morphology. In fact it is known which set
\{\underline{b}\} belongs to a given t and this set typically has one element.
That is why we extend the definition AN-B-LEX slightly. \newline 


AN-B-LEX(t) =\\

\{ \underline{b} $|$ (t, \underline{b}) $\in$ BARE-LEX \} $\cup$\\

\{ \underline{b} $|$ d $\in$ ID-PARSER(t) 
      $\wedge$ (d,  \underline{b}) $\in$ ID-DERIV-TREES $\wedge$\\

THEAD(t)=DHEAD(d) \}\\\\

\paragraph{Comment}
THEAD(t)=DHEAD(d): THEAD(t) looks up information in the S-tree (t) that 
matches information in the D-tree (d) rendered by DHEAD(d), so it gives a 
unique indication of which (small) subset of ID-DERIV-TREES consists of
candidates. 

\end{document}

