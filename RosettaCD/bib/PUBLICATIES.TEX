\documentstyle[11pt]{article}

\setlength{\parindent}{0in}
\setlength{\textheight}{52pc}
\setlength{\textwidth}{40pc}
\setlength{\topmargin}{1pc}
\setlength{\oddsidemargin}{.5pc}
\setlength{\evensidemargin}{.5pc}
\renewcommand{\baselinestretch}{1.5}


\begin{document}



\section*{Publicaties over Rosetta}
\setlength{\parindent} {0 em}

\bigskip

(inclusief afstudeerverslagen)\\

\bigskip

[1] Landsbergen, J., {\em Adaptation of Montague grammar to the requirements
        of parsing}, Philips Research Reprint 7573. In: Groenendijk, J.A.G., 
        Janssen, T.M.V. and Stokhof, M.B.J., {\sf Formal methods in the Study of 
        Language Part 2}, MC Tract 136, Mathematical Centre, Amsterdam, 1981, 
        pp 399-420.
\bigskip

[2] Landsbergen, J., {\em Machine translation based on logically isomorphic
        Montague grammars}, Philips Research Reprint 8143. In: Horecky, J. 
        (ed.), {\sf COLING 82}, pp 175-182, North-Holland, 1982.
\bigskip

[3] Landsbergen, J., {\em Isomorphic grammars and their use in the Rosetta
translation system}, Philips Research M.S. 12.950.
Paper presented at the Tutorial on Machine Translation,
Lugano, 1984. 
In M. King (ed), {\sf Machine Translation the state of the
art}, Edinburg University Press, 1987.
\bigskip

[4] Leermakers, R. and J. Rous, {\em The Translation Method of Rosetta},
Philips Research M.S. 13.701,
{\sf Computers and Translation}, 1986, Vol. 1, Number 3, pp. 169-183.

\bigskip


[5] Appelo, L. and J. Landsbergen, {\em The Machine Translation Project
Rosetta}, Philips Research M.S. 13.801,
{\sf Proceedings First International Conference on State of the Art in 
Machine Translation}, Saarbr\"{u}cken, 1986, pp. 34-51.

\bigskip

[6] Appelo, L. {\em A Compositional approach to the Translation of Temporal
        Expressions in the Rosetta System}, Philips Research M.S. 13.677,
        {\sf Proceedings of the 11th Conference on Computational Linguistics},
        August 25 - 29, 1986, Bonn.

\bigskip

[7] Schenk, A., {\em Idioms in the Rosetta Machine Translation System},
        Philips Research M.S. 13.508, {\sf Proceedings of the 11th Conference on 
        Computational Linguistics}, August 25 - 29, 1986, Bonn.

\bigskip

[8] Landsbergen, J., {\em Montague Grammar and Machine Translation},
        Philips Research M.S. 14.026, in: Whitelock, P. et al. (eds), 
        {\sf Linguistic Theory and Computer Applications}, Academic
 Press, London, 1987.

\bigskip

[9] Appelo, L., C. Fellinger and J. Landsbergen, {\em Subgrammars, Rule Classes
        and Control in the Rosetta Translation System}, 
        Philips Research M.S. 14.131, in: {\sf Proceedings of European ACL
        Conference}, Copenhagen, 1987.

\bigskip


[10] De Jong, F. and L. Appelo, {\em Synonymy and Translation},
        Philips Research M.S. 14.269, in: {\sf Proceedings of the
        6th Amsterdam Colloquium}, 1987.

\bigskip

[11] De Jong, F. and L. Appelo, {\em Synonymie en Vertaling},
        Philips Research M.S. 14.545, in: {\sf Spektator} 18, 1, pp44-58, 1988.
        (Dutch version of [10])

\bigskip

[12] Van Munster, E., {\em The Treatment of Scope and Negation}, 
Philips Research M.S. 14.718, in {\sf 
Proceedings of the 12th Conference on Computational Linguistics}, August 22-27,
1988, Budapest.

\bigskip

[13] Sanders, M.J., {\em The Rosetta Translation System}, Philips Research M.S. 
15.103, to appear in {\sf Proceedings of the 2nd Eindhoven Symposium on 
LSP (Language for Special Purposes)}, Eindhoven, 1988.

\bigskip

[14] Landsbergen, J., {\em Dictionaries for Rosetta}, in {\sf Proceedings of 
International Symposium on Electronic Dictionaries}, Tokyo, 1988.

\bigskip

[15] Odijk, J., {\em The organisation of the Rosetta grammars}, 
to be presented at {\sf European ACL Conference}, Manchester, 1989.

\bigskip

[16] Leermakers, R., {\em How to cover a grammar},
in: {\sf Proceedings American ACL Conference}, Vancouver, 1989.

\bigskip

[17] Landsbergen, J., J. Odijk and A. Schenk, {\em The power of compositional 
translation}, Philips Research M.S. 15.427, 
to appear in: {\sf Linguistic and Literary Computing}, 1989. 

\bigskip

[18] Schenk, A., {\em The formation of idiomatic structures}, 
Philips Research M.S. 15.439, to appear in: 
{\sf Proceedings First Tilburg Workshop on Idioms}, 1989.

\bigskip

[19] Landsbergen, J., {\em The Rosetta Project}, Philips Research M.S. 15.505,
in: {\sf Proceedings Machine Translation Summit II}, Munich, 1989.

\bigskip

[20] Landsbergen, S.P.J., {\em Kunnen machines vertalen?} (in Dutch), oratie 
Rijksuniversiteit Utrecht, ISBN 90-9003208-8, 1989.

\bigskip

[21] De Jong, F., {\em Het machinaal vertaalproject ROSETTA} (in Dutch), 
Informatie, februari 1990, pp 170 - 182.

\bigskip

[22] Landsbergen, S.P.J., {\em Kunnen machines vertalen} (in Dutch),
{\em Traduction automatique: mythe ou r\'{e}alit\'{e}?} (in French),
Horizon, nr. 20, 1990.

\bigskip

[23] Smit, H.E., {\em Van Van Dale bestanden naar Rosetta woordenboeken}
(in Dutch), in: {\sf Tabu jrg. 20, nr. 2, 1990}.

\bigskip


[24] Leermakers, R., {\em Non-deterministic Recursive Ascent Parsing},
Philips Research M.S. 16.324, in: {\sf Proceedings European ACL Conference}, 
Berlin, 1991.
 
\bigskip

[25] Rous, J., {\em Computational Aspects of M-grammars},
Philips Research M.S. 16.462,
in: {\sf Proceedings European ACL Conference}, Berlin, 1991.

\bigskip

[26] Leermakers, R., L. Augusteijn, F.E.J. Kruseman Aretz,
{\em A functional LR-parser},
Philips Research M.S. 16.324 (revised), accepted for publication in: {\sf 
Theoretical Computer Science}. 

\bigskip


[27] Leermakers, R., {\em On the theory of LR-parsing}, accepted for 
publication in: {\sf Information Processing Letters}. 

\bigskip

[28] Leermakers, R., {\em Recursive ascent Marcus parsers},
Philips Research M.S. 16.227, accepted for publication in: {\sf 
Theoretical Computer Science}. 

\bigskip

[29] Leermakers, R., {\em Recursive ascent parsing},
Philips Research M.S. 16.528, in: {\sf 
Proceedings first Twente Workshop on Language Technology: Tomita's Algorithm, 
Extensions and Applications, 1991}. 


\newpage

{\bf Doctoraalscripties}

\bigskip

Munster, E. van, {\em The treatment of scope and negation in Rosetta: a Dutch - 
Spanish view.} Doctoraalscriptie Rijksuniversiteit Utrecht, 1985.

\bigskip

Smit, H. E., {\em Coordinatoren},  Doctoraalscriptie Universiteit van 
Amsterdam, 1985.
 
\bigskip

Van Hout, A., {\em Er-peculiarities in Rosetta: an analysis of Dutch `er'
and its translations in English and Spanish}, 
 Doctoraalscriptie Katholieke Universiteit Brabant, 1986.

\bigskip

Hazenberg, C., {\em M-rules in Pascal: Rosetta's M-rule compiler},
 Doctoraalscriptie Rijksuniversiteit Leiden, 1987.

\bigskip

Post, A., {\em Temporal predication, semantics of temporal elements in 
natural language},  Doctoraalscriptie Rijksuniversiteit Utrecht, 1987.

\bigskip

Kopinga, L., {\em Time adverbials in the machine translation system of
Rosetta3},  Doctoraalscriptie Vrije Universiteit Amsterdam, 1987.

\bigskip

Grygierczyk, N.J., {\em Semantische disambiguering} (in Dutch), 
 Doctoraalscriptie Rijksuniversiteit Utrecht, 1987.

\bigskip

Schipper, J., {\em Morphological analysis of Dutch in Rosetta}, doctoraalscriptie
Rijksuniversiteit Utrecht, 1988.

\bigskip

Zwarts, J., {\em An analysis of genericity},
 Doctoraalscriptie Rijksuniversiteit Utrecht, 1988. 

\bigskip

Dijkstra, I., {\em Ambiguity resolution in Rosetta},
doctoraalscriptie Technische Universiteit Delft, 1991.

\end{document}
