\documentstyle{Rosetta}
\begin{document}
   \RosTopic{Rosetta3.doc.Mrules.English}
   \RosTitle{Rosetta3 English M-rules: Utterance}
   \RosAuthor{Margreet Sanders}
   \RosDocNr{379}
   \RosDate{November 14, 1989}
   \RosStatus{approved}
   \RosSupersedes{-}
   \RosDistribution{Project}
   \RosClearance{Project}
   \RosKeywords{English, documentation, Mrules, Utterance}
   \MakeRosTitle
%
%

\section{Introduction}
Around the English Sentence and XPPROP grammars there are two other, very 
small grammar
classes. Generation begins with one of the {\bf Derivation} grammars (see doc.\ 
316, {\em Rosetta3 English M-rules: Derivation Subgrammars\/}), and always ends
with the {\bf Utterance} grammar.
The current document describes the contents of this Utterance grammar.

The grammar consists of only one rule class, with five rules.
Thus, the {\em control expression} is very simple.

In line with the other (sub)grammar documentation, the 
explanation of the rules is given from a generative viewpoint
only. The semantics of the rules 
has been left unspecified in the current documentation, since it is not at all 
clear.

As holds for all English rules, the rules described in this document have NOT 
been tested 
properly, since English analysis is not possible yet (there is no Surface 
Parser).

In the definition phase of Rosetta3 (as laid down for English in doc.\ 
150, {\em Subgrammars of English\/}, written by Jan Odijk), 
this grammar was not mentioned explicitly. However, for ease of the Surface 
Parser (one unique top node), a special Utterance Grammar is advisable.
Also, it seemed an easy place to harbour the check on Polarity (see the 
documentation below), and finally, it simply is the continuation of a Rosetta2 
practice to use the UTT node as a common top node.


\newpage
\section{Grammar Specification}
The grammar definition can be found in file {\bf english:Utterance.mrule}, 
which is {\em mrules100.mrule\/}. There is only one rule class.

\begin{verbatim}
%SUBGRAMMAR Utterance

RC_UttRules

\end{verbatim}

\begin{description}
  \item[Head] \mbox{}\\
    \begin{tabular}{ll}
SENTENCE & FROM (CLAUSEtoSENTENCE)\\
NP       & FROM (NPFORMATION)\\
ADVP     & FROM (ADVPFORMATION)\\
ADJP     & FROM (ADJPFORMULAtoADJPPROP)\\
PREPP    & FROM (PREPPFORMATION)\\
    \end{tabular}
  \item[Eport] UTT
  \item[Import] --
\end{description}

\newpage
\section{Rules}

\subsection{RC\_UttRules}
\begin{description}
\item[Kind] Obligatory Rule Class
\item[Task] To provide an extra top node, UTT, for any structure arriving 
in the grammar, and to check Polarity requirements. The latter is done by 
means of the special function {\em PolarityOK\/}. For more information on this 
subject, see doc.\ 369 on Polarity.

English does not have an utterance rule for a CLOSEDADVPPROP. If such a rule is 
needed (to test isolated adverbs that take a sentential complement, like 
agvpadvs), it can be added.

\vspace{1 cm}
\begin{description}
\item[Name] RUttSent
\item[Task] To provide an extra top node, UTT, for any finite main sentence 
arriving in the grammar.
\item[File] english:Utterance.mrule (mrules100.mrule)
\item[Semantics] --
\item[Example] $_{S}$[He went home] $\rightarrow$ $_{Utt}$[He went home]
\item[Remarks] In Dutch, infinite wh-sentences are accepted to ({\em Wat te 
doen als er brand uitbreekt?\/}). No provision has been taken for that in 
English yet.
\end{description}

\vspace{1 cm}
\begin{description}
\item[Name] RUttNP
\item[Task] To provide an extra top node, UTT, for any non-generic NP
arriving in the grammar. This rule is needed for testing purposes of isolated 
NPs.
\item[File] english:Utterance.mrule (mrules100.mrule)
\item[Semantics] --
\item[Example] $_{NP}$[the beautiful house] $\rightarrow$ $_{Utt}$[the beautiful 
house]
\item[Remarks] In the rule, the cases of the NP are set to a `random' real 
value, viz.\ {\em 
[nominative] \/}. In analysis, any case is accepted, but the case is not passed 
on (by means of a parameter or something) to the IL. This will be changed, so 
that the case found in analysis is reproduced in generation. When the input 
string is ambiguous with respect to case, a hierarchy of preference must be 
defined, so that the `most plausible' case is passed on.

The superdeixis of the NP is set to {\em omegadeixis\/} (in analysis: to {\em 
presentdeixis\/}). This is needed because in the NP grammar, a value for 
superdeixis is expected in analysis.

When generic NPs can be made, the rule will probably have to be adapted to 
allow for generic NPs as well. At the moment, only non-generic NPs are 
accepted, and the value of the attribute is reset to {\em omegageneric\/}, 
because the surface parser cannot recognise genericity.
\end{description}

\vspace{1 cm}
\begin{description}
\item[Name] RUttADVP
\item[Task] To provide an extra top node, UTT, for any (presentdeixis) ADVP
arriving in the grammar. This rule is needed for testing purposes of `ordinary'
isolated ADVPs (for agvpadvs, subjvpadvs etc.\ see the remarks made at the 
beginning of this subsection).
\item[File] english:Utterance.mrule (mrules100.mrule)
\item[Semantics] --
\item[Example] $_{ADVP}$[earlier than we expected] $\rightarrow$ 
$_{Utt}$[earlier than we expected]
\item[Remarks] In the rule, the superdeixis of the ADVP is set to {\em 
omegadeixis\/} (in analysis: to {\em presentdeixis\/}). This is needed because 
in the ADVP grammar, a value for superdeixis is expected in analysis.
\end{description}

\vspace{1 cm}
\begin{description}
\item[Name] RUttADJP
\item[Task] To provide an extra top node, UTT, for any (presentdeixis) ADJP
arriving in the grammar. This rule is needed for testing purposes of isolated 
ADJPs.
\item[File] english:Utterance.mrule (mrules100.mrule)
\item[Semantics] 
\item[Example] $_{ADJP}$[more beautiful than we expected] $\rightarrow$ 
$_{Utt}$[more beautiful than we expected]
\item[Remarks] In the rule, the superdeixis of the ADJP is set to {\em 
omegadeixis\/} (in analysis: to {\em presentdeixis\/}). This is needed because 
in the ADJPPROP grammar, a value for superdeixis is expected in analysis.
It is assumed that in 
the ADJPFORMULAtoADJPPROP subgrammar, there is a special rule to `reduce' an 
ADJPFORMULA to an ADJP again (there is no ADJP grammar!).
\end{description}

\vspace{1 cm}
\begin{description}
\item[Name] RUttPREPP
\item[Task] To provide an extra top node, UTT, for any (presentdeixis)
 PREPP
arriving in the grammar. This rule is needed for testing purposes of isolated 
PREPPs.
\item[File] english:Utterance.mrule (mrules100.mrule)
\item[Semantics] --
\item[Example] $_{PREPP}$[in a year or so] $\rightarrow$ 
$_{Utt}$[in a year or so]
\item[Remarks] In the rule, the superdeixis of the PREPP is set to {\em 
omegadeixis\/} (in analysis: to {\em presentdeixis\/}). This is needed because 
in the PREPP grammar, a value for superdeixis is expected in analysis.
\end{description}


\end{description}

\end{document}

