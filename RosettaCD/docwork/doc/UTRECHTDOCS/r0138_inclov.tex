{\bf
\hyphenation{ont-wik-ke-ling}
Overzicht\\ \\
\rm
In het overzicht is van een effectieve werktijd (t.b.v. 
Rosetta3 ontwikkeling) uitgegaan van 3 dagen per werkweek. 
De codes beginnend met een A staan voor niet afgeronde activiteiten terwijl 
codes die met een X beginnen afgeronde activiteiten zijn. De codes staan 
voor de volgende activiteiten:
\begin{description}
\item[{\bf A5}] Onderzoek subgrammatica's van Rene. 
\item[{\bf A7}] Implementatie van het in X4 gedefineerde ontwerp.
\item[{\bf A8}] Implementatie, documentatie en testen van morfologische 
componenten.
\item[{\bf A9}] Implementatie, documentatie en testen van surf. parser en 
linearizer.
\item[{\bf A12}] Co\"{o}rdinerende werkzaamheden.
\item[{\bf A13}] Onderzoek COLD-S te gebruiken voor documentatie surfparser.
\item[{\bf A17}] RMS interface onderzoek.
\item[{\bf A18}] Lextree regel compilatie.
\item[{\bf A19}] I/O interface ontwerp en implementatie.
\item[{\bf A21}] Documentatie formalisme van de morf. componenten.
\item[{\bf A22}] Implementatie rule compiler voor Surf. Parser.
\item[{\bf A23}] Implementatie definitie modules vanuit het Global Design.
\item[{\bf A25}] Donderdagochtendvoordrachtvoorbereiding.
\item[{\bf A27}] Control grammars.
\item[{\bf A28}] M-regel compilatie.
\item[{\bf A30}] \LaTeX\ faciliteiten implementatie.
\item[{\bf A31}] Finishing touch RBS.
\item[{\bf A32}] Van Dale onderzoek.
\item[{\bf A33}] Woordenboek editor.
\item[{\bf A34}] C-verslag T.H.-Informatica\\ \\
\item[{\bf X1}] Ontwerp en implementatie van een systeem ten behoeve van code 
control.
\item[{\bf X2}] Vooronderzoek aan de surface parser op basis van het Earley 
algoritme.
\item[{\bf X3}] Eerste gedeelte afstudeeropdracht Carel: formaliseren van de 
regels, implementeren van de transducer.
\item[{\bf X4}] Globaal ontwerp van Rosetta3, definitie van minimaal 
noodzakelijke abstracte datatypes ten behoeve van morfologische componenten. 
Tevens documentatie hiervan.
\item[{\bf X6}] Formalisme van de morfologische componenten.
\item[{\bf X10}] Ondersteuning software cursus door Rene.
\item[{\bf X11}] Tweede gedeelte afstudeerwerk Carel: het verslag.
\item[{\bf X14}] DEC-cursus.
\item[{\bf X15}] Birmingham conferentie.
\item[{\bf X16}] Hamburg conferentie.
\item[{\bf X20}] String abstract datatype implementatie.
\item[{\bf X24}] Afstudeerpraatje voorbereiden.
\item[{\bf X26}] Low level Screen Management routines.
\item[{\bf X29}] Domein T compilatie.
\end{description}}
