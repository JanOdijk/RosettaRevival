\documentstyle{Rosetta}
\begin{document}
   \RosTopic{Rosetta3.doc.morphology}
   \RosTitle{Rosetta3 Dutch Morphology, derivation}
   \RosAuthor{Harm Smit}
   \RosDocNr{330}
   \RosDate{\today}
   \RosStatus{concept}
   \RosSupersedes{-}
   \RosDistribution{Project}
   \RosClearance{Project}
   \RosKeywords{Dutch, morphology, derivation, documentation}
   \MakeRosTitle


\section{Preface}

This document gives an impression
of the state of the art of the Dutch {\em derivation} of Rosetta at the moment. 
Another document deals with Dutch {\em inflection}, and the general treatment 
of morphological phenomena.
The {\em rules} are listed in two separate documents.

\newpage
\section{Introduction}

This document consists of several parts, which are:
\begin{itemize}
  \item relevant aspects of domain T for Dutch (section 3),
  \item explanation of the way verbs are dealt with (section 4),
  \item idem, for nouns and propernouns (section 5),
  \item idem, for adjectives and adverbs (section 6),
  \item final remarks (section 7) with suggestions for further refinement and
        for changes in future versions of the Dutch (derivational) morphology.
\end{itemize} 

It should be noted that derivational phenomena are --in general-- not very
regular (this holds for both meaning and form), which makes it difficult to
capture these morphological processes in rules. In principle, many derived words
will be added to the dictionaries as {\em basic} expressions.

\newpage
\section{Domain T of Dutch morphology}

This section contains domain T of the Dutch morphology as far as it is
relevant for derivation: all categories and category-records, that are used
in derivation, are specified here. Also, the attributes (together with the 
possible attribute-values) and the suffix- and prefix-keys are listed here.

{\bf Categories:}

\begin{tabular}{lll}
BVERB,       &  SUBVERB,       &  VERB,  \\
BNOUN,       &  SUBNOUN,       &  NOUN,  \\ 
BPROPERNOUN, &  PROPERNOUN,    &         \\
BADJ,        &  SUBADJ,        &  ADJ,   \\  
BADV,        &  SUBADV,        &  ADV,   \\
BPERSPRO,    &  PERSPRO,       &         \\
BDET,        &  DET,           &         \\ 
BINDEFPRO,   &  INDEFPRO,      &         \\
POSSADJ,     &                 &         \\ 
POSSPRO,     &                 &         \\ 
BWHPRO,      &  WHPRO,         &         \\
DEMPRO,      &                 &         \\
SFCAT,       &                 &         \\ 
PFCAT,       &                 &         \\ 
\end{tabular}

{\bf Keys:}

\begin{tabbing}

SFKje \ \ \= SFKetje \ \ \= SFKonregdim \ \ \= SFKdimletterword  \ \ \=  
PFKaller \\
\end{tabbing}

{\bf Records (inherent attributes are bold printed):}
\\
\begin{tabular}{lll}
             &                    &              \\
\end{tabular}
\\
\begin{tabular}{lll}
BVERBrecord: &                    &              \\
             & {\bf particle:}    & key                               \\
\end{tabular}
\\
\begin{tabular}{lll}
SUBVERBrecord: &                    &              \\
               & {\bf particle:}    & key                               \\
               & lastaffix:     & partaffix, exaffix, hyperaffix, viceaffix,\\
               &                & antiaffix, proaffix, onaffix, achtigaffix,\\
               &                & dimaffix,  noaffix, advaffix  \\
\end{tabular}
\\
\begin{tabular}{lll}
VERBrecord:  &                 &              \\
             & {\bf particle:} & key                               \\
             & status:         & bareV, partV   \\
\end{tabular}
\\
\begin{tabular}{lll}
BNOUNrecord: &                 &              \\
             & {\bf pluralforms:} & subset of (enPlural, sPlural, aTOaaPlural,\\
             &                 & aTOeePlural, eTOeePlural, eiTOeePlural,    \\
             &                 & iTOeePlural, oTOooPlural, erenPlural,      \\
             &                 & ienPlural, denPlural, nenPlural,           \\
             &                 & ieAccentPlural, luiPlural, liedenPlural,   \\
             &                 & LatPlural, enIrregPlural, sIrregPlural,    \\
             &                 & LatIrregPlural, NoPlural, OnlyPlural)      \\  
             & {\bf dimforms:} & subset of (jeDim, etjeDim, irregDim, \\
             &                 & dimletterword, noDim)                \\
             & {\bf genders:}  & mascgender, femgender, neutgender,   \\
             &                 & omegagender                           \\
\end{tabular}
\\
\begin{tabular}{lll}
SUBNOUNrecord: &               &              \\
             & {\bf pluralforms:} & subset of (enPlural, sPlural, aTOaaPlural,\\
             &                 & aTOeePlural, eTOeePlural, eiTOeePlural,    \\
             &                 & iTOeePlural, oTOooPlural, erenPlural,      \\
             &                 & ienPlural, denPlural, nenPlural,           \\
             &                 & ieAccentPlural, luiPlural, liedenPlural,   \\
             &                 & LatPlural, enIrregPlural, sIrregPlural,    \\
             &                 & LatIrregPlural, NoPlural, OnlyPlural)      \\  
             & {\bf dimforms:} & subset of (jeDim, etjeDim, irregDim, \\
             &                 & dimletterword, noDim)                \\
             & {\bf genders:}  & mascgender, femgender, neutgender,   \\
             &                 & omegagender                           \\
             & lastaffix:      & partaffix, exaffix, hyperaffix, viceaffix,\\
             &                 & antiaffix, proaffix, onaffix, achtigaffix,\\
             &                 & dimaffix,  noaffix, advaffix  \\
\end{tabular}
\\
\begin{tabular}{lll}
NOUNrecord:  &                 &              \\
             & {\bf pluralforms:} & subset of (enPlural, sPlural, aTOaaPlural,\\
             &                 & aTOeePlural, eTOeePlural, eiTOeePlural,    \\
             &                 & iTOeePlural, oTOooPlural, erenPlural,      \\
             &                 & ienPlural, denPlural, nenPlural,           \\
             &                 & ieAccentPlural, luiPlural, liedenPlural,   \\
             &                 & LatPlural, enIrregPlural, sIrregPlural,    \\
             &                 & LatIrregPlural, NoPlural, OnlyPlural)      \\  
             & {\bf dimforms:} & subset of (jeDim, etjeDim, irregDim, \\
             &                 & dimletterword, noDim)                \\
             & {\bf genders:}  & mascgender, femgender, neutgender,   \\
             &                 & omegagender                           \\
\end{tabular}
\\
\begin{tabular}{lll}
BPROPERNOUNrecord: &           &              \\
             & {\bf pluralforms:} & subset of (enPlural, sPlural, aTOaaPlural,\\
             &                 & aTOeePlural, eTOeePlural, eiTOeePlural,    \\
             &                 & iTOeePlural, oTOooPlural, erenPlural,      \\
             &                 & ienPlural, denPlural, nenPlural,           \\
             &                 & ieAccentPlural, luiPlural, liedenPlural,   \\
             &                 & LatPlural, enIrregPlural, sIrregPlural,    \\
             &                 & LatIrregPlural, NoPlural, OnlyPlural)      \\
             & {\bf dimforms:} & subset of (jeDim, etjeDim, irregDim, \\
             &                 & dimletterword, noDim)                \\
             & {\bf genders:}  & mascgender, femgender, neutgender,   \\
             &                 & omegagender                           \\
\end{tabular}
\\
\begin{tabular}{lll}
BADJrecord: &                     &              \\
            & {\bf possadv:}      & true, false  \\
\end{tabular}
\\
\begin{tabular}{lll}
SUBADJrecord: &                   &              \\
              & {\bf possadv:}    & true, false  \\
\end{tabular}
\\
\begin{tabular}{lll}
ADJrecord: &                      &              \\
              & {\bf possadv:}    & true, false  \\
\end{tabular}
\\
\begin{tabular}{lll}
BADVrecord: &                     &              \\
\end{tabular}
\\
\begin{tabular}{lll}
SUBADVrecord: &                   &              \\
               & lastaffix:     & partaffix, exaffix, hyperaffix, viceaffix,\\
               &                & antiaffix, proaffix, onaffix, achtigaffix,\\
               &                & dimaffix,  noaffix, advaffix  \\
\end{tabular}
\\
\begin{tabular}{lll}
ADVrecord: &                      &              \\
\end{tabular}
\\
\begin{tabular}{lll}
SFCATrecord: &                 &              \\
             & {\bf key:}      & key     \\
\end{tabular}
\\

Note: In some cases, values are mentioned that do not (yet) 
      occur in the morphology.

\newpage

\section{Verbs}

For verbs, the following derivation rules exist:

\begin{itemize}
  \item rule for verbs without derivation.

        This rule makes a SUBVERB out of a BVERB for all verbs without 
        derivational affixes.

  \item rule for verbs with particle. 

        This rule makes all VERB-forms with a
        particle, like {\em oplost}, {\em opgelost}, {\em opgeloste}, etc.
        The input model has to be a VERB, because the particle precedes 
        {\em ge-}.

        Note that the 0th person is not excluded, although it can never occur 
        with particle  ({\em *wegga jij?}). The reason for this is 
        that these forms are excluded by the M-grammar, which excludes 
        {\em all} 
        forms with particle that precede the subject ({\em wegga ik?}). The 
        morphology handles only single words and therefore the relation between
        subject  and verb cannot be used for conditions in lextree rules.

        SUBVERB-forms (``stems'') with particle should not pass this 
        rule (therefore, no ``mu'' in the model of this rule). 
        These forms have to be made, however, because of forms like 
        {\em oplosbaar}, where the SUBVERB 
        serves as stem for the (SUB)ADJ. At the moment such derivation rules 
        are not present in the morphology.  

        Forms like {\em onopgelost}, where derivation ({\em on-}) and 
        inflection ({\em ge-}) are (apparently) combined, 
        cannot be derived by the
        particle rule. It seems, however, that forms like {\em onopgelost} are 
        real adjectives; and therefore a form like {\em opgelost} should be in
        the dictionary. This implies that the derivational suffix {\em on-} 
        cannot be combined with SUBVERB's (note that in {\em onoplosbaar} the
        prefix {\em on-} is combined with a SUBADJ).

        Note that future rules for derivational processes should work properly 
        for particle-verbs: a form like {\em losbaar} should not be accepted or
        generated for the verb {\em los\_op}; neither should a form like:
        {\em opgeoplost} be yielded.
\end{itemize}

\newpage

\section{Nouns}

For nouns, the following derivation rules exist:

\begin{itemize}
  \item rule for nouns without derivation.

        This rule makes a SUBNOUN out of a BNOUN for all nouns without 
        derivational affixes.

  \item rule for nouns with diminutive suffix.
    
        This rule changes a SUBNOUN without the value ``dimaffix'' (for the
        attribute ``lastaffix'') into a SUBNOUN {\em with} ``dimaffix''.
        Note that this implies that it is impossible to have two adjacent 
        diminutive suffixes ({\em *bootjetje}). This does not mean that two 
        diminutive suffixes can never occur together in one word:
        {\em jonget\underline{je}sachtigheid\underline{je}s}.

        Because there are several diminutive-endings (which can occur 
        as variant forms for one noun, as for instance: {\em rug}, with
        {\em rugje} and {\em ruggetje}), four suffix-keys have been introduced:
         \begin{itemize}
           \item SFKje   ({\em rugje, koninkje, halmpje}),
           \item SFKetje ({\em ruggetje, ringetje, gummetje}), 
           \item SFKonregdim ({\em blaadje, lootje, scheepje}), 
           \item SFKdimletterword ({\em PC'tje, IQ'tje}).
         \end{itemize}
        These four suffix-key are mapped on one abstract affix: ``BNOUNSUFF'',
        (with the value ``dimaffix'') that is in the dictionary.
        A few values for the SUBVERB with dimform are copied from the suffix:
        both ``genders'' and ``pluralforms'' are specified by the suffix 
        (``genders'' is always ``neutgender'', ``pluralforms'' is ``sPlural'').
\end{itemize}

Another rule derives a noun from a propernoun:

\begin{itemize}
   \item rule for SUBNOUN out of BPROPERNOUN.

         This rule can handle cases like {\em de beide Duitslanden}, {\em de 
         Kennedy's}, {\em het zonnige Itali\"{e}}.

         In this case, some attribute-values can be copied from the BPROPERNOUN
         (like ``pluralforms'', ``dimforms'', etc.);
         others have a special value (which is in many cases the default value).

         The attribute ``possgeni'' will be made false.

         Note that PROPERNOUN's can be singular only, either genitive or not 
         genitive. So, attributes like ``pluralforms'' and ``dimforms'' are 
         only relevant in combination with this rule.

\end{itemize}

And there are derivation-rules for the category  PROPERNOUN:

\begin{itemize}
   \item rule for propernoun with diminutive.
    
         This rule handles propernoun-forms like {\em Jantje}, {\em Harmpje},
         etc. Because propernouns don't have a SUB-level, derivation and 
         inflection are both done by putting making a PROPERNOUN out of a
         BPROPERNOUN. Therefore there are no special rules for making 
         propernouns without derivation.

   \item rule for propernoun with diminutive and genitive.

         This rule makes forms like {\em Jantjes (boek)}, {\em Harmpjes (jas)}.
         It deals with derivation (diminutive) and inflection (genitive) at 
         once. 
         
\end{itemize}

\newpage

\section{Adjectives and adverbs}

For adjectives and adverbs, the following rules exist:
\begin{itemize}
  \item rule for adjectives without derivation.

        This rule makes a SUBADJ out of a BADJ for all adjectives without 
        derivational affixes.

  \item rule for adverbs without derivation.

        This rule makes a SUBADV out of a BADV for all adverbs without 
        derivational affixes.

  \item Rule that makes a SUBADV out of a SUBADJ.

        This rule derives an adverb from an adjective. It is comparable to
        the English rule that sticks a {\em -ly}-suffix to an adjective.

        Some attribute-values can be copied from the ``BADVSUFF''-suffix 
        (with the value ``advaffix'') that
        is in the dictionary; others come from the SUBADJ-level.

\end{itemize}

\newpage

\section{Final remarks}

Of course, only few derivational processes have been captured in the given 
rules.  Many of these processes are in one way or another irregular. Some may
be added in the future.
If rules for derivational suffixes like {\em -baar} will be made it should be
noted that they also work for verbs with particles (see section 4 about verbs).


\newpage


\section{References}

\begin{description}
  \item [ANS (GEERTS, G. e.a.)]
     1984, {\bf Algemene Nederlandse Spraakkunst}, Groningen, Wolters-Noordhoff.

  \item [VAN DALE (VAN STERKENBURG, P.G.J. e.a.)]
     1984, {\bf Groot woordenboek van hedendaags Nederlands}, Utrecht, Van Dale 
           Lexicografie b.v.

  \item [NIEUWBORG, E.R.]
     1978, {\bf Retrograde Woordenboek van de Nederlandse Taal}, 
           Deventer, Kluwer Technische Boeken b.v.

  \item [MARTIN, W.]
     1971, {\bf Inverte Frequentielijst van het Nederlands}, 
           Leuven, Instituut voor toegepaste Linguistiek.

  \item [KLEIN, M. and VISSCHER, M.]
     1985, {\bf Praktische Cursus Spelling}, Groningen, Wolters-Noordhoff.

\end{description}
\end{document}
