\documentstyle{Rosetta}
\begin{document}
   \RosTopic{General}
   \RosTitle{Notulen Rosetta vergadering 23-2-1987}
   \RosAuthor{Harm Smit}
   \RosDocNr{0179}
   \RosDate{\today}
   \RosStatus{approved}
   \RosSupersedes{-}
   \RosDistribution{Project}
   \RosClearance{Project}
   \RosKeywords{minutes}
   \MakeRosTitle
\begin{itemize}
  \item {\bf aanwezig}: Lilian Kopinga, Ans Post, Ren\'{e} Leermakers, 
             Jeroen Medema, Joep Rous, Jan Landsbergen, Andr\'{e} Schenk, 
             Lisette Appelo, Natalia Grygierczyk, Carel Fellinger, Jan Odijk, 
             Elly van Munster, Harm Smit.
  \item {\bf afwezig}: Franciska de Jong, Jan Stevens.
  \item {\bf Agenda}:
    \begin{enumerate}
       \item Opening en notulen
       \item Diverse mededelingen
       \item Kontakten met HIS en TDS
       \item Besproken en/of nieuw verschenen documenten
       \item Rondvraag en sluiting
    \end{enumerate}
  \item Aansluitend vertelde Natalia Grygierczyk iets over de semantische 
        component van DLT.
\section {Opening en notulen}
De notulen van de vorige vergadering werden met enkele wijzigingen aangenomen.
\section {Diverse mededelingen}
\begin{enumerate}
  \item Op woensdag 4 maart komt Agnes voor de eerste keer voor het testen van 
        de Engelse morfologie. Voor die dag zullen de hulpmiddelen voor het 
        testen gereed zijn. Agnes zal in het totaal vier keer komen.
  \item Zowel het abstract van Carel en Lisette als dat van Jan L. voor de ACL 
        1987 in Kopenhagen zijn geaccepteerd.
  \item Carel meldt een fout in document 129, pagina 5, regel 17: {\bf F11}
        moet zijn {\bf F10}; het staat wel goed onder `GOLD HELP'.
\end{enumerate}
\section {Kontakten met HIS en TDS}
Jan L. heeft contact gehad met de directie van het Lab en de HIS over een 
mogelijke participatie van 
ROSETTA in het werk voor de lingu\"{\i}stische component van de VIDEO WRITER.
Er zijn echter geen concrete afspraken gemaakt en het lijkt er ook niet op dat 
die er nog komen.
Naar aanleiding van een brief van TDS Wenen aan de directie van het Lab m.b.t.  
de VOICE ACTIVATED TYPEWRITER zal er gekeken worden of er een stagiair kan 
worden gevonden die iets gaat doen aan foneem-grafeem omzetting. Te denken 
valt aan het ontwerpen van een hierop toegesneden variant van A-Morph.
\section {Besproken en/of nieuw verschenen documenten}
\begin{itemize}
  \item {\bf besproken}: document 163 van Jan O. over `superdeixis and 
        polarity'. Het
        ging hier vooral om een notationele kwestie m.b.t. Conditie-Actie 
        paren. Jan O. en Ren\'{e} zullen een nieuwe notatie voorstellen.
  \item {\bf verschenen}: document 177 van Harm over het tweede verslag aan
        Van Dale. Dit document is ter kennisgeving verspreid en hoeft niet 
        besproken te worden.
\end{itemize}
\section {Rondvraag en sluiting}
Er was niets voor de rondvraag.
\end{itemize}
\end{document}
