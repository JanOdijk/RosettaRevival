\documentstyle{Rosetta}
\begin{document}
   \RosTopic{General}
   \RosTitle{Types}
   \RosAuthor{Jan Odijk}
   \RosDocNr{0353}
   \RosDate{September 18, 1989}
   \RosStatus{informal}
   \RosSupersedes{-}
   \RosDistribution{Project}
   \RosClearance{Project}
   \RosKeywords{Type system, sorts, semantic component}
   \MakeRosTitle
%
%


\def\m#1{$\|$#1$\|$}
\def\to{$\rightarrow$}
\def\toset#1{$\uparrow$#1}
\def\td#1{$<$ #1 $>$}
\def\examples#1 {\em #1}
\section{Introduction}

This document deals with types in Rosetta. A proposal is made for a
type-system that is to be used in Rosetta. In section~\ref{typesec} some formal
aspects of the type-system are discussed. Everything proposed there is
fairly standard, I think, except for one part: the system proposed
allows for a simple way to deal with cross-classifications.
Section~\ref{subst}  deals with the substantive part of the types: Which types
exist, and how do we determine types? A strategy to solve this problem is
proposed, and some examples are given.
Section~\ref{impl} discusses some aspects of robustness and implementation.

The status of this paper is exploratory and the proposal tentative.
The system as proposed here fits in nicely with (a restricted version of)
the proposal by Joep Rous
concerning the semantic component (doc. R0401).

\section{Types}
\label{typesec}

A {\em type} is an object associated to set of meanings.
The set of meanings associated to a type is called the {\em  domain} of
the type.

Both basic meanings (e.g. meanings of single words and idioms)
and non basic meanings (e.g. the meaning of an NP or a sentence)
are members of a {\em type domain }.

I'll notate type domains as D$_{type}$. The meaning of some expression $\alpha$
is represented as \m{$\alpha$}.

\def\D#1{D$_{#1}$}

\begin{description}
  \item[Examples]\mbox{}\\
\begin{description}
  \item[] \D{human} = \{ \m{man}, \m{vrouw}, \m{minister}, ... \} 
  \item[] \D{abstract} = \{ \m{kunst}, \m{tijd},.. \}
  \item[] \D{mass} = \{ \m{vlees}, \m{goud}, ... \}
  \item[] \D{question} = \{ \m{of hij ziek is}, \m{Is dat correct?},
                       \m{wat heeft hij gedaan?},...\}
  \item[] \D{plural} = \{ \m{de mannen}, \m{enkele vrouwen}, \m{sommige auto's}, 
        ...\}
\end{description}
 
\end{description}

The types from Montague grammar can be represented in this way as well.
The type {\em t} has domain \D{t} = \{ 0,1 \}; the type {\em e}  
equals the set of all entities.
The type {\em c} suggested for the meaning of {\em  wheather-it} in doc. R0308
would be a type consisting of exactly one member.

Sometimes a distinction is made between {\em  types} and {\em  sorts}.
Types are restricted to the types {\em  e}, {\em  t} and complex types
built out of these elements, whereas {\em  sorts} further subdivide
types (in particular the type {\em  e}). This distinction is not made here.
Some of the types that will be proposed correspond clearly to {\em  sorts},
e.g. {\em  human, animate} etc. Others, however, will probably correspond
to complex types formulated in terms of {\em  e}'s and {\em  t}'s, e.g.
{\em  question}. In the system proposed here both have the same status, though
they probably will differ in their place on a hierarchy to be defined below.

The domains of types of basic meanings are large, but finite sets. 
The domains of types of nonbasic
meanings are infinite sets.
For these reasons sets of meanings cannot be used directly in a real system. 
For basic 
meanings it would be impractical because the sets are very large, for nonbasic
meanings it is in principle impossible, since the sets are infinite.
Instead {\em types} are used to achieve the desired effects.


\subsection{Cross classification}
Types are used in {\em (enclosed) type descriptions} to indicate  what
type domains  a meaning is a member of. 
Enclosed type descriptions are defined by
means of the following syntax:

\def\nt#1{$<$#1$>$}

\begin{description}
  \item[] \nt{enclosed type description} \to '$<$' \nt{type description} '$>$' 
  \item[] \nt{type description} \to \nt{signed type} \{\nt{type description}\}
  \item[] \nt{signed type} \to ['$\sim$'] \nt{type}
\end{description}

\begin{description}
  \item[Examples] \mbox{}
\begin{description}
  \item[] \td{animate} 
  \item[] \td{$\sim$animate}
  \item[] \m{de mannen} : $<$ human plural masculine$>$ 
  \item[] \m{handdoek}: $<$ concrete horizontal$>$
  \item[] \m{zwemt de man} : $<$ question activity $>$
\end{description}


\end{description}


\noindent
where {\em  horizontal} represents the class of meanings that can function
as the first argument of the meaning of the 
verb {\em  liggen}, but not of {\em  zitten} or {\em  
staan}, and where {\em  activity} is an {\em  aktionsart}. 


\def\t#1{\mbox{$\tau_{#1}$}}


Every meaning has an enclosed type description. If a certain meaning {\em M} has
an enclosed type-description of the form $<$ \t1..\t{n} $>$ 
(where \t{i} is a signed
type), then this is interpreted
in such a way that M is a member of \D{\t1} $\cap$ .. $\cap$ \D{\t{n}}.
The interpretation of {\em  $\sim$\t{}} (\t{} a type) is: a certain meaning
can have the specification  {\em  $\sim$\t{}} if it is a member 
of the complement of \D{\t{}}.
The set relative to which the complement is taken (the universe) 
is here the set of all 
meanings. It might perhaps be better to take some other set as the universe.
For discussion, see below.

In this way cross-classifications can be made in a systematic way.
These cross-classifications are absolutely necessary to be able to make a
realistic system. Once the system is fixed, it might be possible to do away
with the cross classifications by partitioning the set of meanings in 
sufficiently small subsets, though this probably will not serve any purpose.


\subsection{Hierarchy}

We define a relation $\leq$ on types (actually: signed types, see below), 
as follows:
\begin{description}
  \item[]
a $\leq$ b iff \D{a} $\subseteq$ \D{b}
 
\end{description}


Analogously, the relations $<$, $>$ and $\geq$ can be defined.

Since the actual system cannot deal with real sets, a partial ordering
of types w.r.t the operator $\leq$ must be specified in a real system.
The $\leq$ relation must be computed on the basis of such a specification.

\begin{description}
  \item[Example] The following might be parts of such a partial ordering:
\begin{description}
  \item[] human $<$ animate $<$ concrete
  \item[] drink $<$ fluid $<$ concrete 
  \item[] concrete $<$ e
  \item[] weapon $<$ instrument $<$   concrete
\end{description}
\end{description}


\subsection{Type requirements}

Every function specifies for each of its arguments to what type it must belong.
This is specified by means of a {\em type requirement} for each argument.

A type requirement  is defined by the following syntax:\footnote{The syntax
given is arbitrary to a certain extent, and it can be extended where necessary,
e.g. one might perhaps want to allow requirement descriptions of the form
{\em  horizontal (weapon | cloths)}, etc.}

\def\r#1{\mbox{$\rho_{#1}$}}
  

\begin{description} 
  \item[] \nt{req-descr} \to \nt{req-descr} '$\mid$' \nt{req-descr}
  \item[] \nt{req-descr} \to \nt{signed type} \{ \nt{signed type} \}
  \item[] \nt{signed type} \to ['$\sim$'] \nt{type}
\end{description}


\begin{description} 
  \item[Examples]\mbox{}\\
\begin{description} 
  \item[] animate
  \item[] animate $\mid$ t
  \item[] $\sim$animate $\mid$ t
  \item[] human feminine $\mid$ weapon
\end{description}

\end{description}


It is possible to make enclosed requirement descriptions from req-descr's
in the following manner:

\begin{description} 
  \item[] \nt{enclosed req-descr} \to '$<$ [\nt{req-descr}['-' \nt{req-descr} ['-' \nt{req-descr}]]] '$>$'
\end{description}

These can be used to specify the type requirements for each argument (0,1,2,3)
of an argument-taking meaning.

Compatibility of a type description \t{} with a requirement description \r{} 
is defined as follows:


\begin{description} 
\item[alternatives] if \r{}=\r1$\mid$\r2 (\r1 and \r2 requirement 
descriptions), and \t1 is a type description, 
then \t1 is
compatible with \r{} iff \t1 is compatible with \r1 or with \r2
  \item[concatenation] if \t{} =  \t1..\t{n}, \r{} =  \r1..\r{m} (\t{i} and 
\r{i}
signed types for all {\em i}), then \t{} is compatible with \r{} iff
$\forall_{\r{i}} \exists_{\t{j}} \t{j} \leq \r{i}$ 
\end{description}



\begin{description}
  \item[Example] Given the partial ordering sketched above, the following 
holds:

\begin{description}
  \item[concatenation] \td{plural human masculine} is compatible with {\em animate},
since {\em  animate} $\geq$ {\em  human} 
  \item[alternation] \td{human} is compatible with {\em abstract $\mid$ concrete }, 
since {\em  human} $\leq$ {\em  concrete}
\end{description}

 
\end{description}

It appears to be more convenient to define '$\sim$' as creating a complement
w.r.t. the {\em  root type} than w.r.t all types. The {\em  root type}
for some type \t{} is defined as follows:

\begin{description} 
  \item[] Roottype(\t{}) = \t{1} iff \t{1} $\geq$ \t{} and there is no type
           \t{2} such that \t{2} $>$ \t{1}
\end{description}
 

In words, the roottype of some type \t{} is the type occurring as the root of the 
hierarchy tree that \t{} is a member of.
This will give the result that e.g. the type description $\sim$animate is equivalent
to nonanimate entities (\td{$\sim$animate entity}), so that e.g. 
meanings of type
{\em  t} are not included, which appears to be the intention in most cases..

Other interpretations of '$\sim$' can be imagined as well. If no consistent,
intuitively appealing definition can be found, then it might perhaps be better
to do away with it, because it will then certainly create errors, or we might
replace it by a difference operator.


%In certain treatments maxtypes are called {\em  types}, our {\em  types} are
%called {\em sorts}.

It might perhaps be convenient to define a special type that is 
compatible with any type and one (perhaps the same) that any type is 
compatible with (e.g. {\em anytype}). 

%This might perhaps be useful for an argument taking meanings that does not 
%impose any requirement on one of its arguments, and it might perhaps 
%be useful as a 
%type for the meaning of e.g. {\em  het}, which can be of type {\em  entity}
%(it can refer to {\em  het boek}, of type {\em  t} (it can refer to
%finite declarative sentences), of type {\em  question} (it can refer to
%questions, e.g. {\em  Jan vroeg het}), of type {\em  c} 
%(as in {\em  het regent} ).
  

\subsection{How to use types in the semantics?}

The idea is that any argument-taking meaning has  an enclosed type 
requirement as part of its semantic specification and that any meaning
has a type description. Combining a argument-taking meaning with its
arguments entails:

\begin{description} 
  \item[1] checking whether the type description of each argument is 
compatible with the relevant type requirement as specified in the
argument taking meaning.
 \item[2] (if the conditions under {\bf 1} are satisfied) 
computing a new type for the result of the application of this rule.
\end{description}

Thus, the operation working on these types for an n-ary syntactic rule
is an n-ary function from 
an n-tuple of feature bundles to a set of  feature bundles.

If there is a type mismatch, no new type will be computed for the result of
the application  of the relevant rule. If no case passes the type compatibility
test, no resulting type will occur, and the associated semantic derivation tree
is (semantically) ill-formed.

In certain cases certain information from arguments or argument taking
meanings must be retained. A concrete example might be VP-modifying adverbs.
For such adverbs it might be necessary that we compute a 
type for the VP and store it in the 
featurebundle associated to the rule forming a VPPROP (or CLAUSE) in order
to be able to later compare the type requirement of the adverb with the type
of  the VP.
This would be artificial, but it would be 
a direct consequence of the treatment of such
adverbs of which we knew in advance that it was semantically probably not 
fully correct. 

A second case where this might be relevant concerns such words as 
{\em  krijgen} 'gedaan krijgen', {\em  laten} with complements such as 
{\em  hem in de kast},
{\em  hem in de tuin} etc., where it is perhaps necessary to record the
type of the PP (path, location, etc.) occurring in this complement. I am
not sure about this yet, but it should be investigated. At the moment, 
these verbs have been given (somewhat ad-hoc) type requirements 
combining a truth value with the type of the PP ({\em tlocation}, 
{\em tpath}, etc.).

Also the treatment of certain kinds of adjuncts as proposed in doc. R0308
might make it necessary to retain certain information of the head.

We should also investigate whether there are other cases of this
kind.
 

\section{How to determine types?}
\label{subst}

A more difficult problem than the formal part of the type system is the 
substantive part: How to determine which types should be assumed? This is
a very difficult problem, because it is possible to take subsets from
the sets of meanings in an infinite number of ways. The problem is:
how do we select a choice of subsets that has  linguistic significance, and
that is useful for disambiguation purposes?

My suggestion would be to tackle this problem in the following manner:

\begin{description}
  \item[1]  First, tentatively make type-requirements for all arguments of
meanings of many (ideally: all) argument-taking words. 
Specify the hierarchy required for
this system, and define the types postulated precisely (add tests etc.)
  \item[2] Try to fill all meanings of nouns (more generally: potential 
heads of argument phrases) etc. for their types, using the hierarchy of types
postulated. The hierarchy of types will consist of a set of trees: Pick for
each tree (and each subtree, etc.) the lowest leaf that is appropriate.
Form a type description by putting all these types in a list.
\end{description}


As an experiment, I started to do  the first part of this strategy for
some of the verbs in the current dictionaries that I met anyway doing other
dictionary activities. This led to a first list of types and an 
associated  hierarchy.
This list is supplied in appendix~\ref{alltypes}. The actual list in the 
dictionary has been extended slightly already.

While filling these type requirements I found out that it was very important
to have a clear idea of what types the rules of the current Rosetta3 system
would yield. Since I did not have very clear ideas about this, this will 
probably have yielded inconsistent fillings. To avoid this in the future,
I made  a first proposal  as to what kind of 
types are yielded by the final rules of subgrammars. They are listed
in appendix~\ref{typesofrules}.


\section{Implementation and Robustness}
\label{impl}
\subsection{Robustness and Metaphorical Uses}

Of course, the Rosetta system must be robust enough to deal with cases
that are not foreseen. Among these there might be simply semantically
illformed sentences, but also all kinds of metaphorical uses.

One way to guarantee this robustness, would be to give the type check the
status of an ordering imposing mechanism: the type check never filters, but
it orders trees according to plausibility. In this way the system 
will always be able to cope with metaphorical uses, etc.

It is my opinion that this should {\em  not} be done, and that the type check
should be a real filter. When considering the cases in Van Dale, one finds
that cases of metaphors etc. are limited to very special cases. If metaphors
are to be dealt with in a principled way, then a theory of metaphors should
be built in. Since there probably is no time (nor anyone who would like to do 
it) to develop such a theory, I think that cases of metaphor should be treated 
by general robustness measures. This is especially the case since Van Dale
already distinguishes several `metaphorical' uses as separate meanings.

Also such sentences as {\em  Dieren kunnen niet spreken} etc. should be dealt 
with by means of robustness measures. 

One way to implement robustness (suggested to me by Harm Smit), is to go up
higher in the hierarchy step by step for each requirement description that is
violated, if no semantic derivation tree would be well-formed otherwise, until
a semantic derivation tree is well-formed. Another would be to specify
such changes in a list explicitly (This could in fact be a first, 
very primitive, theory of metaphors). 
E.g. If {\em  human} is required and {\em  animate}
is found, then {\em  animate} may be lowered to {\em  human} to achieve
the effect of {\em  personalization}, one of a very limited number of
possible metaphors or related devices. Of course, also both methods can be 
used.

\subsection{Implementation}

The implementation of the system proposed can be done in many ways. 
The $\leq$-relation might be computed on the basis of a hierarchy specification
and represented in an incidence matrix, or in a tree.

The actual rules doing the type-check can be formulated in the rules that
will be used in the semantic component (see the document by Joep Rous).

As far as efficiency is concerned, the following should be noted.
Let us consider a typical case of a verb taking 3 
arguments, which is m-fold ambiguous. Let us assume that  the
arguments themselves are also ambigous, resp. {\em i,j,k}-fold ambigous.

In this situation (if no other relevant properties hold)
the current implementation of M-Generator 
would require in the worst case  {\em  i.j.k.m}
compatibility tests. If  $i=j=k=m=10$ this would imply $10^{4}$ comparisons.

However, if the implementation is done differently, in the way sketched
very informally below,
the number of comparisons can be reduced to $(i+j+k).m$ comparisons
(i.e. 300 if $i=j=k=m=10$).

The implementation should proceed as follows. First all substituents 
(arguments) are evaluated, for all their meanings. Next the pre-action
associates corresponding variables with the relevant semantic features.
Now
a set of such semantic features for each variable exists.

The derivation proceeds by evaluating each meaning of the verb separately.
For each of these meanings  a type check is done for each argument with 
all alternative semantic features. This will (in the worst case)
require {\em k} comparisons for the third argument, {\em j} comparisons
for the second and {\em  i} for the first argument. If some test yields
incompatibility (in the worst case this is never so), the relevant
variable and its associated substituent are thrown out. This yields in
the worst case $(i+j+k)$ comparisons.

This process is repeated for each meaning of the verb, i.e. {\em  m} times,
so that in total $(i+j+k).m$ comparisons need be made.

This implementation does presuppose  that there are no interdependencies
between arguments w.r.t. their type restrictions. I think that that is
indeed never the case, but if such cases are found, this should be reported,
for it probably makes this implementation impossible.

It should be investigated whether it is possible to make an implementation
as sketched here and whether it can be incorporated into the rest of
M-Generator, and also whether it is necessary.

\appendix
\newpage
\section{Types as occurring in the dictionaries now}
\label{alltypes}

The types mentioned occur in type requirements listed in the 
dutch dictionaries as comment in the notation proposed above.
(search for the string \{$<$, or use  the edit command file
[odijkje.mrules]typedt.edt to get them into a file).
The first subsection contains a list of all types encountered, the second 
subsection contains the hierarchy of types, the last subsection lists
the meaning keys with meaning description and their postulated typing.


\subsection{List of types}
The following additional conventions have been adopted:

\begin{description} 
  \item[*] an asterisk indicates that the typing should be reconsidered
  \item[''] Quotes are used to mention keys directly if no characterization
of the relevant class could be found.
  \item[non] the prefix $\sim$ has not been used yet. The prefix {\em non} or
{\em in} has
been used instead
  \item[loc,t] stands for the type {\em tlocation}, see the discussion above
w.r.t {\em  krijgen}.
\end{description}

The list contains type names, plus their number of occurrences (no number
means that they occurred once).
The list is up to date until July 14, 1989. 
After this date new type requirements have
been specified, which are not listed in this document.

\newpage
\begin{theindex}
\item 'koers'
\item  *abstract
\item  *infocontainer
\item  abstract 18
\item  animate 10
\item  bel,telefoon,zoemer
\item  builtobject
\item  c
\item  celestialbody
\item  cloths 2
\item  concrete 12
\item  deling
\item  drink 2
\item  entity 20
\item  exam
\item  feminine
\item  fluid 2
\item  food 3
\item  fuel
\item  game
\item  geounit
\item  human 88
\item  inanimate 2
\item  line 2
\item  loc,t
\item  location
\item  masculine
\item  medicine 2
\item  music
\item  musicalinstrument
\item  nonanimate 3
\item  nonfluid
\item  nonhuman 4
\item  path 3
\item  plural 2
\item  possessable
\item  punishment
\item  question
\item  ship
\item  sourcepath
\item  spacemeasure
\item  suitcase
\item  t 8
\item  taste
\item  timeref
\item  transportmeans
\end{theindex}

\subsection{Hierarchy of postulated types}


\begin{tabular}
{|p{.20\textwidth}|p{.20\textwidth}|p{.20\textwidth}|p{.20\textwidth}|p{.20\textwidth}|}
\hline
entity & concrete & animate  & human&\\
       &          &          & feminine&\\
       &          &          & masculine&\\
&&                possessable&&\\
&&&                 bel,telefoon,zoemer&\\
&&&                builtobject&\\
&&&                celestialbody&\\
&&&                cloths&\\
&&&                fluid  & drink\\
&&&                food&\\
&&&                fuel&\\
&&&                geounit&\\
&&&                medicine&\\
&&&                musicalinstrument&\\
&&&                ship&\\
&&&                suitcase&\\
&&&                transportmeans&\\
      & abstract & 'koers'&&\\
&&                infocontainer&&\\
&&                'deling'&&\\
&&                exam&&\\
&&                game&&\\
&&                line&&\\
&&                music&&\\
&&                punishment&&\\
&&                taste&&\\
c&&&&\\
t &&&&\\
loc,t &&&&\\
question&&&\\
location&&&\\
path & sourcepath& & &\\
plural &&&&\\
spacemeasure &&&&\\
timeref &&&&\\
\hline
\end{tabular}


\newpage
\subsection{Meaning keys and their type requirements}

\begin{tabular}{|c|c|p{.3\textwidth}|p{.3\textwidth}|}
\hline
mkey                 &\# & meaning descr & typing \\
m\_aV\_0002\_beveel\_aan & s1 & "aanraden" & {$<$human-entity-human$>$}\\
m\_aV\_0001\_pers\_af    & s1 & "dwingen te geven" & {$<$human-human$>$}\\
m\_aV\_0002\_pers\_af    & s2 & "geheel en al persen" & {$<$human-cloths$>$}\\
m\_aV\_0001\_betref     & s1 & "aangaan" & {$<$entity-human$>$}\\
m\_aV\_0002\_betref     & s2 & "handelen over" & {$<$entity-entity$>$}\\
m\_aV\_0001\_ga         & s1 & "zich verplaatsen,begeven" & {$<$concrete-path$>$}\\
m\_aV\_0002\_ga         & s1 & "vertrekken, weggaan" & {$<$concrete-timeref$>$}\\
m\_aV\_0013\_ga         & s2 & "begrepen zijn in,passen" & {$<$concrete-location$>$}\\
m\_aV\_0004\_ga         & s3 & "$<$+ onbep. wijs$>$beginnen te" & {$<$c$\mid$entity-t$>$}\\
m\_aV\_0005\_ga         & s5 & "rinkelen" & {$<$?bel,telefoon,zoemer$>$}\\
m\_aV\_0014\_ga         & s1 & "beheren" & {$<$human-entity$>$}\\
m\_aV\_0015\_ga         & s2 & "tot onderwerp hebben" & {$<$*infocontainer-entity$>$}\\
m\_aV\_0001\_hebben     & s1 & "bezitten" & {$<$human-possessable$>$}\\
m\_aV\_0005\_hebben     & s5 & "in de genoemde toestand verkeren" & {$<$animate-t$\mid$loc,t$>$}\\
m\_aV\_0014\_hebben     & s1 & "nut ondervinden van" & {$<$animate-entity$\mid$t$>$}\\
m\_aV\_0001\_help       & s1 & "bijstaan" & {$<$animate-entity$\mid$t-animate$>$}\\
m\_aV\_0002\_help       & s2 & "verzorgen" & {$<$human-human$>$}\\
m\_aV\_0006\_help       & s6 & "baten" & {$<$nonanimate$\mid$t$>$}\\
m\_aV\_0001\_neem\_in   &  s1&  "mbt. geneesmiddelen" & {$<$human-medicine$>$}\\
m\_aV\_0002\_neem\_in   &  s2 & "mbt. een plaatsruimte" & {$<$entity-spacemeasure$>$}\\
m\_aV\_0003\_neem\_in   &  s3 & "veroveren" & {$<$human-geounit$\mid$builtobject$>$}\\
m\_aV\_0006\_neem\_in   &  s6 & "aan boord nemen" & {$<$ship$\mid$human-fuel$\mid$food$\mid$fluid$>$}\\
m\_aV\_0007\_neem\_in   &  s7 & "inkorten" & {$<$human-cloths$>$}\\
m\_aV\_0101\_stort\_in  &  s1 & "neerstorten" & {$<$nonanimate$>$}\\
m\_aV\_0102\_stort\_in  &  s2 & "een inzinking krijgen" & {$<$human$>$}\\
m\_aV\_0001\_isoleer    & s1 & "afzonderen" & {$<$human-human$>$}\\
m\_aV\_0002\_isoleer    & s2 & "uit een geheel halen" & {$<$human-nonhuman$>$}\\
m\_aV\_0001\_kleef      & s1 & "vast blijven zitten" & {$<$concrete-concrete$>$}\\
m\_aV\_0002\_kleef      & s2 & "$<$fig.$>$ onverbrekelijk verbonden zijn met"                           & {$<$abstract-abstract$>$}\
\
m\_aV\_0002\_koester    & s2 & "vertroetelen" & {$<$human-animate$>$}\\
m\_aV\_0003\_koester    & s3 & "uit waardering beschermen" & {$<$human-entity$>$}\\
m\_aV\_0013\_laat       & s13 & "bij zijn dood nalaten" & {$<$human-concrete-human$>$}\\
m\_aV\_0014\_laat       & s14 & "afstaan" & {$<$human-entity-human$>$}\\
\hline
\end{tabular}
\newpage
\begin{tabular}{|c|c|p{.3\textwidth}|p{.3\textwidth}|}
\hline
mkey                 &\# & meaning descr & typing \\
\hline
m\_aV\_0001\_luister    & s1 & "horen om iets te vernemen" & {$<$human-entity$\mid$question$>$}\\
m\_aV\_0003\_luister    & s3 & "aandacht schenken aan" & {$<$human-human$>$}\\
m\_aV\_0004\_luister    & s4 & "gehoorzamen aan" & {$<$human-human$>$}\\
m\_aV\_2101\_moet       & s1 & "mogen, believen" & {$<$human-human$>$}\\
m\_aV\_0101\_naai       & s1 & "vasthechten (van boeken)" & {$<$human-inanimate$>$}\\
m\_aV\_0103\_naai       & s3 & "benadelen" & {$<$human-human$>$}\\
m\_aV\_0001\_noteer     & s1 & "aantekenen" & {$<$human-t$>$}\\
m\_aV\_0002\_noteer     & s2 & "bepalen, opgeven (van koersen)" & {$<$human-'koers'$>$}\\
m\_aV\_0001\_omhels     & s1 & "omarmen" & {$<$human-human$>$}\\
m\_aV\_0002\_omhels     & s2 & "$<$fig.$>$aannemen" & {$<$human-abstract$>$}\\
m\_aV\_0003\_omhels     & s3 & "omvatten" & {$<$nonanimate-entity$>$}\\
m\_aV\_0003\_onthaal    & s3 & "$<$fig.$>$vergasten op" & {$<$human-human-abstract$>$}\\
m\_aV\_0001\_ontmoet    & s1 & "onvoorzien tegenkomen" & {$<$human-human$>$}\\
m\_aV\_0002\_ontmoet    & s2 & "volgens afspraak treffen" & {$<$human-human$>$}\\
m\_aV\_0003\_ontmoet    & s3 & "ondervinden" & {$<$human-abstract$>$}\\
m\_aV\_0004\_ontmoet    & s4 & "$<$wisk.$>$ (van lijnen)" & {$<$line-line$>$}\\
m\_aV\_0101\_open       & s1 & "openmaken" & {$<$animate-concrete$>$}\\
m\_aV\_0103\_open       & s3 & "openstellen" & {$<$animate-abstract$>$}\\
m\_aV\_0002\_ga\_op     &  s2 & "mbt. de zon" & {$<$celestialbody$>$}\\
m\_aV\_0004\_ga\_op     &  s4 & "examen afleggen" & {$<$human-exam$>$}\\
m\_aV\_0005\_ga\_op     &  s5 & "opgegeten/opgedronken worden" & {$<$food$\mid$drink$>$}\\
m\_aV\_0006\_ga\_op     &  s6 & "juist zijn" & {$<$abstract$>$}\\
m\_aV\_0007\_ga\_op     &  s7 & "in beslag genomen worden" & {$<$human-abstract$>$}\\
m\_aV\_0008\_ga\_op     &  s8 & "mbt. delingen" & {$<$deling$>$}\\
m\_aV\_0009\_ga\_op     &  s9 & "in elkaar overgaan" & {$<$entity-entity$>$}\\
m\_aV\_0101\_hang\_op   &  s1 & "in de hoogte hangen" & {$<$human-inanimate$>$}\\
m\_aV\_0102\_hang\_op   &  s2 & "ter dood brengen" & {$<$human-human$>$}\\
m\_aV\_0101\_houd\_op   &  s1 & "omhooghouden" & {$<$human-concrete$>$}\\
m\_aV\_0102\_houd\_op   &  s2 & "verdedigen" & {$<$human-abstract$>$}\\
\hline
\end{tabular}
\newpage
\begin{tabular}{|c|c|p{.3\textwidth}|p{.3\textwidth}|}
\hline
mkey                 &\# & meaning descr & typing \\
\hline
m\_aV\_0001\_pak        & s1 & " tevoorschijn halen" & {$<$animate-concrete-sourcepath$>$}\\
m\_aV\_0002\_pak        & s2 & " vastnemen" & {$<$animate-entity}\\
m\_aV\_0003\_pak        & s3 & " betrappen" & {$<$human-human$>$}\\
m\_aV\_0004\_pak        & s4 & "inpakken" & {$<$human-suitcase$>$}\\
m\_aV\_0005\_pak        & s5 & "gebruik maken van" & {$<$human-transportmeans$>$}\\
m\_aV\_0006\_pak        & s6 & "mbt. drank" & {$<$human-drink$>$}\\
m\_aV\_0008\_pak        & s8 & "benadelen" & {$<$human-human-entity$>$}\\
m\_aV\_0010\_pak        & s10 & "seksueel gebruiken" & {$<$human-human$>$ ?$<$masculine-feminine$>$}\\
m\_aV\_0013\_pak        & s13 & "$<$sport, voetbal$>$ mishandelen" & {$<$human-human$>$}\\
m\_aV\_0101\_proef      & s1 & "een smaak gewaarworden" & {$<$human-food$\mid$taste$>$}\\
m\_aV\_0102\_proef      & s2 & "bemerken, bespeuren" & {$<$human-abstract$>$}\\
m\_aV\_0002\_redeneer   & s2 & "gevolgtrekkingen afleiden" & {$<$human$>$}\\
m\_aV\_0003\_redeneer   & s3 & "argumenteren, redetwisten" & {$<$human plural$>$}\\
m\_aV\_0001\_reinig     & s1 & "ontdoen van vuil" & {$<$human-nonhuman$>$}\\
m\_aV\_0002\_reinig     & s2 & "$<$relig.$>$" & {$<$human-human$>$}\\
m\_aV\_0102\_rem        & s2 & "$<$fig.$>$, vertragen, stoppen" & {$<$t-abstract$>$}\\
m\_aV\_0001\_roof       & s1 & "met geweld wegnemen" & {$<$human-nonhuman$>$}\\
m\_aV\_0003\_roof       & s3 & "gewelddadig wegvoeren" & {$<$human-human$>$}\\
m\_aV\_0001\_slik       & s1 & "innemen" & {$<$human-medicine$>$}\\
m\_aV\_0002\_slik       & s2 & "accepteren" & {$<$human-abstract$\mid$t$>$}\\
m\_aV\_0001\_speel      & s1 & "zich (met een spel) vermaken" & {$<$animate-game$>$}\\
m\_aV\_0003\_speel      & s3 & "bespelen" & {$<$human-musicalinstrument$>$}\\
m\_aV\_0004\_speel      & s4 & "uitvoeren" & {$<$human-music$>$}\\
m\_aV\_0006\_speel      & s6 & "van invloed, belang zijn"  & {$<$abstract$>$}\\
m\_aV\_0001\_spot       & s1 & "zich met scherts uiten" & {$<$human$>$}\\
m\_aV\_0003\_spot       & s3 & "zich niet storen aan" & {$<$human-abstract$>$}\\
m\_aV\_0009\_sta        & s9 & "ge�ist worden" & {$<$punishment-*abstract$>$}\\
\hline
\end{tabular}
\newpage
\begin{tabular}{|c|c|p{.3\textwidth}|p{.3\textwidth}|}
\hline
mkey                 &\# & meaning descr & typing \\
\hline
m\_aV\_0001\_stroom     & s1 & "met kracht vloeien" & {$<$fluid-path$>$}\\
m\_aV\_0002\_stroom     & s2 & "$<$fig.$>$in groten getale komen/gaan" & {$<$nonhuman nonfluid entity-path$>$}\\
m\_aV\_0003\_stroom     & s3 & "zich in groten getale voortbewegen" & {$<$human-path$>$}\\
m\_aV\_0001\_struikelen & s1 & "het evenwicht verliezen en evt. vallen" & {$<$human-concrete$>$}\\
m\_aV\_0003\_struikelen & s3 & "$<$fig.$>$ten val komen" & {$<$human-abstract$>$}\\
m\_aV\_0005\_struikelen & s5 & "$<$fig.$>$ (in grote hoeveelheden) aantreffen" & {$<$human-plural entity$>$}\\
m\_aV\_0003\_stuit      & s3 & "aantreffen" & {$<$human-concrete$>$}\\
m\_aV\_0004\_stuit      & s4 & "$<$fig.$>$geconfronteerd worden" & {$<$human$\mid$abstract-abstract$>$}\\
m\_aV\_0101\_stuit      & s1 & "tegenhouden" & {$<$human-concrete$>$}\\
m\_aV\_0102\_stuit      & s2 & "$<$fig.$>$ tegenhouden" & {$<$human-abstract$>$}\\
\hline
\end{tabular}

\newpage
\section{Types yielded by certain semantic rules}
\label{typesofrules}

The following list is a first, tentative (and incomplete) proposal
for a specification of the types that should be yielded by functions
associated to IL-rules. 

\begin{theindex}
\item     LAccIng t? 
\item     Ladhort adhortative
\item     Lclosedinf t
\item     Lclosednpp t
\item     Lclosedvpprop t
\item     Lclosedxpp t $\mid$ tpath $\mid$ tlocation $\mid$ ...
\item     LCNformation1 entity?
\item     LCNformation2 entity?
\item     LCNformation3 entity?
\item     LCNformation4 entity?
\item     Lconjsent adverbial?
\item     Ldeclmain  t
\item     Ldeclsub t
\item     Ldetpformation ?
\item     Lfinrel relative
\item     Lfinwhmod whmod?
\item     LFortoinf t?
\item     LFortoinfRel relative
\item     LFortoinfWhMod whmod?
\item     Limp            imperative
\item     Linjunsub omte
\item     LNPformation1 entity $\mid$ actionnoun?
\item     LNPformation2 entity $\mid$ actionnoun?
\item     LNPformation3 entity $\mid$ actionnoun?
\item     LNPformation4a entity $\mid$ actionnoun?
\item     LNPformationdef entity $\mid$ actionnoun?
\item     LNPformation6 entity $\mid$ actionnoun?
\item     LNPformation7 entity $\mid$ actionnoun?
\item     LNPformation8  entity $\mid$ actionnoun?
\item     LNPformation9 entity $\mid$ actionnoun?
\item     LNPformation10 entity $\mid$ actionnoun?
\item     LNPformation11 entity $\mid$ actionnoun?
\item     LNPformation12 entity $\mid$ actionnoun?
\item     LNPformation13 entity $\mid$ actionnoun?
\item     LNPformation14 entity $\mid$ actionnoun?
\item     LNPformation17 entity $\mid$ actionnoun?
\item     LNPOpenIng t?
\item     LNPpartitiveformation entity $\mid$ actionnoun?
\item     Lopendeclinf t
\item     LOpenIng t?
\item     Lopennpp pred
\item     Lopenxpp pred $\mid$ path $\mid$ location $\mid$ ...
\item     Lpartdetpform ?
\item     LPossIng t?
\item     Lprespart relative
\item     Lqtoqp ?
\item     LToinfRel relative
\item     LToinfWhMod whmod
\item     Lwhmain whquestion
\item     Lwhsub whquestion
\item     LwhToinf whquestion
\item     Lynmain yesnoquestion
\item     Lynsub yesnoquestion
\end{theindex}


%\section{Example Treatment of certain words}
%\subsection{{\em lopen}}
%
%I intr
%\begin{enumerate}
%  \item (zich te voet voortbewegen) walk          vp100 / vp012
%  \item (rennen) run                              vp100
%  \item (zich verplaatsen) go                     ??
%  \item (<mbt. zaken> voortbewogen worden) run    ??
%  \item (stromen) run                             vp100
%  \item (in werking zijn) run                     vp100
%  \item (voortduren) run                          vp100
%  \item (zich uitstrekken) run                    vp100
%  \item (zich ontwikkelen) run                    vp100
%  \item (blootgesteld worden aan) run             ?? gevaar lopen
%  \item (geschikt zijn om op/in  te lopen)        middle vp120
%  \item (+inf) bezig zijn met                     zij lopen te treuzelen vp120
%\end{enumerate}
%
%
%II trans
%\begin{enumerate}
%  \item (deelnemen aan) go to (attend) 
%\end{enumerate}
%
%III onp. werkw.
%\begin{enumerate}
%  \item (naderen) 
%\end{enumerate}
%
%
%Proposed treatment:
%
%lopen vp012 (syndonp\_diropenpreppprop perfauxs=[zijn])\\
%(zich te voet voortbewegen) walk\\
%  arg1 = $<$ animate$>$\\
%  arg2 = $<$ path $>$\\
%\examples{hij is naar huis gelopen.}\\
%(stromen) run                    \\
%  arg1 = $<$ fluid $>$\\
%  arg2 = $<$ path $>$\\
%\examples{Het water is uit het vat gelopen}\\
%
%
%lopen vp100 (synnovpargs) perfauxs=[hebben])\\
%(zich te voet voortbewegen) walk \\
%  arg1 = $<$ animate $>$\\
%topic= sport \\
%(rennen) run   \\                  
%  arg1 = $<$ animate$>$\\
%\examples{De atleten hebben (zich warm) gelopen}\\
%(in werking zijn) run            \\
%  arg1 = $<$ machine $>$\\
%\examples{de machine loopt, de motor loopt, de klok loopt, ?de kraan loopt}\\
%(voortduren) run                 \\
%? arg1 = $<$ abstract $>$\\
%\examples{de onderhandelingen lopen; ?de contracten lopen}\\
%(zich uitstrekken) run           \\
%  arg1 = $<$ geo-unit $\mid$ building $>$\\
%\examples{De weg loopt van A naar B; de rivier loopt naar de Noordzee;
%Het huis loopt van hier tot aan de volgende zijstraat}\\
%
%
%
%
%
%lopen vp120 synopenteinfsent perfauxs=[hebben]\\
%(bezig zijn met) be\\
%  arg1 = $<$ animate $>$\\
%  arg2 = $<$ activity $>$\\
%\examples{Hij loopt te treuzelen}\\
%lopen vp120 synnp\\
%?? semi-idiom(deelnemen aan) go to (attend) \\
%  arg1 = $<$ human $>$\\
%  arg2 = $<$ lesson $>$\\
%\examples{Hij liep een cursus/}\\
%lopen vp010 synprepnp prepkey=tegen perfauxs=[hebben]\\
%(naderen)\\
%  arg1 = $<$ temporal-reference $>$\\
%\examples{Het loopt tegen zevenen, tegen middernacht, tegen de middag}\\
%
%lopen vp100 synnovpargs adverb present!\\
%(geschikt zijn om op/in  te lopen)       \\ 
%  arg1 = $<$ cloths $\mid$ geo-unit $\mid$ distance$>$\\
%\examples{deze schoenen lopen lekker, de eerste kilometer liep het 
%gemakkelijkst, deze broek loopt niet zo fijn; deze berg loopt heerlijk}\\
%(zich ontwikkelen) run           \\
%? arg1 = $<$ abstract $>$\\
%\examples{De zaken lopen niet goed; de cursus loopt niet goed}\\
%
%
%
%NOt accounted for yet:\\
%
%(zich verplaatsen) go                     ??\\
%<mbt. zaken> voortbewogen worden) run    ??\\
%
%(blootgesteld worden aan) run             ?? gevaar lopen\\
%
%semi-idiom: zijn neus loopt\\
%
%\newpage
%\subsection{innemen}
%
%\begin{enumerate}
%  \item 1(mbt geneesmiddelen) take
%  \item 2(mbt. plaatsruimte) take up
%  \item 3(veroveren) take, capture, seize
%  \item 4(vertrouwen, genegenheid winnen) captivate
%??\item 5(binnenhalen) bring/take/fetch in
%  \item 6(aan boord nemen) take on
%  \item 7(inkorten (nauwer) take in
%  \item 7(inkorten (korter) take up
%??\item 8(verzamelen) collect
%\end{enumerate}
%
%
%innemen vp120 synnp perfauxs=[hebben]\\
%1(mbt geneesmiddelen) take\\
%  arg1 = $<$ human $>$\\
%  arg2 = $<$ medicine $>$\\
%2(mbt. plaatsruimte) take up\\
%  arg1 = $<$ concrete $>$\\
%  arg2 = $<$ ??$>$ plaats ruimte\\
%3(veroveren) take, capture, seize\\
%  arg1 = $<$ human $>$\\
%  arg2 = $<$ city $\mid$ county $\mid$ building $>$\\
%6(aan boord nemen) take on\\
%  arg1 = $<$ human $\mid$ ship $>$\\
%  arg2 = $<$ fuel $\mid$ fluid $\mid$ food $>$\\
%7(inkorten (nauwer) take in\\
%  arg1 = $<$ human $>$\\
%  arg2 = $<$ cloths $>$\\
%7(inkorten (korter) take up\\
%  arg1 = $<$ human $>$\\
%  arg2 = $<$ cloths $>$\\
%
%
%
\end{document}
ROSETTA.sty
\typeout{Document Style 'Rosetta'. Version 0.4 - released  24-DEC-1987}
% 24-DEC-1987:  Date of copyright notice changed
\def\@ptsize{1}
\@namedef{ds@10pt}{\def\@ptsize{0}}
\@namedef{ds@12pt}{\def\@ptsize{2}} 
\@twosidetrue
\@mparswitchtrue
\def\ds@draft{\overfullrule 5pt} 
\@options
\input art1\@ptsize.sty\relax


\def\labelenumi{\arabic{enumi}.} 
\def\theenumi{\arabic{enumi}} 
\def\labelenumii{(\alph{enumii})}
\def\theenumii{\alph{enumii}}
\def\p@enumii{\theenumi}
\def\labelenumiii{\roman{enumiii}.}
\def\theenumiii{\roman{enumiii}}
\def\p@enumiii{\theenumi(\theenumii)}
\def\labelenumiv{\Alph{enumiv}.}
\def\theenumiv{\Alph{enumiv}} 
\def\p@enumiv{\p@enumiii\theenumiii}
\def\labelitemi{$\bullet$}
\def\labelitemii{\bf --}
\def\labelitemiii{$\ast$}
\def\labelitemiv{$\cdot$}
\def\verse{
   \let\\=\@centercr 
   \list{}{\itemsep\z@ \itemindent -1.5em\listparindent \itemindent 
      \rightmargin\leftmargin\advance\leftmargin 1.5em}
   \item[]}
\let\endverse\endlist
\def\quotation{
   \list{}{\listparindent 1.5em
      \itemindent\listparindent
      \rightmargin\leftmargin \parsep 0pt plus 1pt}\item[]}
\let\endquotation=\endlist
\def\quote{
   \list{}{\rightmargin\leftmargin}\item[]}
\let\endquote=\endlist
\def\descriptionlabel#1{\hspace\labelsep \bf #1}
\def\description{
   \list{}{\labelwidth\z@ \itemindent-\leftmargin
      \let\makelabel\descriptionlabel}}
\let\enddescription\endlist


\def\@begintheorem#1#2{\it \trivlist \item[\hskip \labelsep{\bf #1\ #2}]}
\def\@endtheorem{\endtrivlist}
\def\theequation{\arabic{equation}}
\def\titlepage{
   \@restonecolfalse
   \if@twocolumn\@restonecoltrue\onecolumn
   \else \newpage
   \fi
   \thispagestyle{empty}\c@page\z@}
\def\endtitlepage{\if@restonecol\twocolumn \else \newpage \fi}
\arraycolsep 5pt \tabcolsep 6pt \arrayrulewidth .4pt \doublerulesep 2pt 
\tabbingsep \labelsep 
\skip\@mpfootins = \skip\footins
\fboxsep = 3pt \fboxrule = .4pt 


\newcounter{part}
\newcounter {section}
\newcounter {subsection}[section]
\newcounter {subsubsection}[subsection]
\newcounter {paragraph}[subsubsection]
\newcounter {subparagraph}[paragraph]
\def\thepart{\Roman{part}} \def\thesection {\arabic{section}}
\def\thesubsection {\thesection.\arabic{subsection}}
\def\thesubsubsection {\thesubsection .\arabic{subsubsection}}
\def\theparagraph {\thesubsubsection.\arabic{paragraph}}
\def\thesubparagraph {\theparagraph.\arabic{subparagraph}}


\def\@pnumwidth{1.55em}
\def\@tocrmarg {2.55em}
\def\@dotsep{4.5}
\setcounter{tocdepth}{3}
\def\tableofcontents{\section*{Contents\markboth{}{}}
\@starttoc{toc}}
\def\l@part#1#2{
   \addpenalty{-\@highpenalty}
   \addvspace{2.25em plus 1pt}
   \begingroup
      \@tempdima 3em \parindent \z@ \rightskip \@pnumwidth \parfillskip
      -\@pnumwidth {\large \bf \leavevmode #1\hfil \hbox to\@pnumwidth{\hss #2}}
      \par \nobreak
   \endgroup}
\def\l@section#1#2{
   \addpenalty{-\@highpenalty}
   \addvspace{1.0em plus 1pt}
   \@tempdima 1.5em
   \begingroup
      \parindent \z@ \rightskip \@pnumwidth 
      \parfillskip -\@pnumwidth 
      \bf \leavevmode #1\hfil \hbox to\@pnumwidth{\hss #2}
      \par
   \endgroup}
\def\l@subsection{\@dottedtocline{2}{1.5em}{2.3em}}
\def\l@subsubsection{\@dottedtocline{3}{3.8em}{3.2em}}
\def\l@paragraph{\@dottedtocline{4}{7.0em}{4.1em}}
\def\l@subparagraph{\@dottedtocline{5}{10em}{5em}}
\def\listoffigures{
   \section*{List of Figures\markboth{}{}}
   \@starttoc{lof}}
   \def\l@figure{\@dottedtocline{1}{1.5em}{2.3em}}
   \def\listoftables{\section*{List of Tables\markboth{}{}}
   \@starttoc{lot}}
\let\l@table\l@figure


\def\thebibliography#1{
   \addcontentsline{toc}
   {section}{References}\section*{References\markboth{}{}}
   \list{[\arabic{enumi}]}
        {\settowidth\labelwidth{[#1]}\leftmargin\labelwidth
         \advance\leftmargin\labelsep\usecounter{enumi}}}
\let\endthebibliography=\endlist


\newif\if@restonecol
\def\theindex{
   \@restonecoltrue\if@twocolumn\@restonecolfalse\fi
   \columnseprule \z@
   \columnsep 35pt\twocolumn[\section*{Index}]
   \markboth{}{}
   \thispagestyle{plain}\parindent\z@
   \parskip\z@ plus .3pt\relax
   \let\item\@idxitem}
\def\@idxitem{\par\hangindent 40pt}
\def\subitem{\par\hangindent 40pt \hspace*{20pt}}
\def\subsubitem{\par\hangindent 40pt \hspace*{30pt}}
\def\endtheindex{\if@restonecol\onecolumn\else\clearpage\fi}
\def\indexspace{\par \vskip 10pt plus 5pt minus 3pt\relax}


\def\footnoterule{
   \kern-1\p@ 
   \hrule width .4\columnwidth 
   \kern .6\p@} 
\long\def\@makefntext#1{
   \@setpar{\@@par\@tempdima \hsize 
   \advance\@tempdima-10pt\parshape \@ne 10pt \@tempdima}\par
   \parindent 1em\noindent \hbox to \z@{\hss$^{\@thefnmark}$}#1}


\setcounter{topnumber}{2}
\def\topfraction{.7}
\setcounter{bottomnumber}{1}
\def\bottomfraction{.3}
\setcounter{totalnumber}{3}
\def\textfraction{.2}
\def\floatpagefraction{.5}
\setcounter{dbltopnumber}{2}
\def\dbltopfraction{.7}
\def\dblfloatpagefraction{.5}
\long\def\@makecaption#1#2{
   \vskip 10pt 
   \setbox\@tempboxa\hbox{#1: #2}
   \ifdim \wd\@tempboxa >\hsize \unhbox\@tempboxa\par
   \else \hbox to\hsize{\hfil\box\@tempboxa\hfil} 
   \fi}
\newcounter{figure}
\def\thefigure{\@arabic\c@figure}
\def\fps@figure{tbp}
\def\ftype@figure{1}
\def\ext@figure{lof}
\def\fnum@figure{Figure \thefigure}
\def\figure{\@float{figure}}
\let\endfigure\end@float
\@namedef{figure*}{\@dblfloat{figure}}
\@namedef{endfigure*}{\end@dblfloat}
\newcounter{table}
\def\thetable{\@arabic\c@table}
\def\fps@table{tbp}
\def\ftype@table{2}
\def\ext@table{lot}
\def\fnum@table{Table \thetable}
\def\table{\@float{table}}
\let\endtable\end@float
\@namedef{table*}{\@dblfloat{table}}
\@namedef{endtable*}{\end@dblfloat}


\def\maketitle{
   \par
   \begingroup
      \def\thefootnote{\fnsymbol{footnote}}
      \def\@makefnmark{\hbox to 0pt{$^{\@thefnmark}$\hss}} 
      \if@twocolumn \twocolumn[\@maketitle] 
      \else \newpage \global\@topnum\z@ \@maketitle
      \fi
      \thispagestyle{plain}
      \@thanks
   \endgroup
   \setcounter{footnote}{0}
   \let\maketitle\relax
   \let\@maketitle\relax
   \gdef\@thanks{}
   \gdef\@author{}
   \gdef\@title{}
   \let\thanks\relax}
\def\@maketitle{
   \newpage
   \null
   \vskip 2em
   \begin{center}{\LARGE \@title \par}
      \vskip 1.5em
      {\large \lineskip .5em \begin{tabular}[t]{c}\@author \end{tabular}\par} 
      \vskip 1em {\large \@date}
   \end{center}
   \par
   \vskip 1.5em} 
\def\abstract{
   \if@twocolumn \section*{Abstract}
   \else
      \small 
      \begin{center} {\bf Abstract\vspace{-.5em}\vspace{0pt}} \end{center}
      \quotation 
   \fi}
\def\endabstract{\if@twocolumn\else\endquotation\fi}


\mark{{}{}} 
\if@twoside
   \def\ps@headings{
      \def\@oddfoot{Rosetta Doc. \@RosDocNr\hfil \@RosDate}
      \def\@evenfoot{Rosetta Doc. \@RosDocNr\hfil \@RosDate}
      \def\@evenhead{\rm\thepage\hfil \sl \rightmark}
      \def\@oddhead{\hbox{}\sl \leftmark \hfil\rm\thepage}
      \def\sectionmark##1{\markboth {}{}}
      \def\subsectionmark##1{}}
\else
   \def\ps@headings{
      \def\@oddfoot{Rosetta Doc. \@RosDocNr\hfil \@RosDate}
      \def\@evenfoot{Rosetta Doc. \@RosDocNr\hfil \@RosDate}
      \def\@oddhead{\hbox{}\sl \rightmark \hfil \rm\thepage}
      \def\sectionmark##1{\markboth {}{}}
      \def\subsectionmark##1{}}
\fi
\def\ps@myheadings{
   \def\@oddhead{\hbox{}\sl\@rhead \hfil \rm\thepage}
   \def\@oddfoot{}
   \def\@evenhead{\rm \thepage\hfil\sl\@lhead\hbox{}}
   \def\@evenfoot{}
   \def\sectionmark##1{}
   \def\subsectionmark##1{}}


\def\today{
   \ifcase\month\or January\or February\or March\or April\or May\or June\or
      July\or August\or September\or October\or November\or December
   \fi
   \space\number\day, \number\year}


\ps@plain \pagenumbering{arabic} \onecolumn \if@twoside\else\raggedbottom\fi 




% the Rosetta title page
\newcommand{\MakeRosTitle}{
   \begin{titlepage}
      \begin{large}
	 \begin{figure}[t]
	    \begin{picture}(405,100)(0,0)
	       \put(0,100){\line(1,0){404}}
	       \put(0,75){Project {\bf Rosetta}}
	       \put(93.5,75){:}
	       \put(108,75){Machine Translation}
	       \put(0,50){Topic}
	       \put(93.5,50){:}
	       \put(108,50){\@RosTopic}
	       \put(0,30){\line(1,0){404}}
	    \end{picture}
	 \end{figure}
	 \bigskip
	 \bigskip
	 \begin{list}{-}{\setlength{\leftmargin}{3.0cm}
			 \setlength{\labelwidth}{2.7cm}
			 \setlength{\topsep}{2cm}}
	    \item [{\rm Title \hfill :}] {{\bf \@RosTitle}}
	    \item [{\rm Author \hfill :}] {\@RosAuthor}
	    \bigskip
	    \bigskip
	    \bigskip
	    \item [{\rm Doc.Nr. \hfill :}] {\@RosDocNr}
	    \item [{\rm Date \hfill :}] {\@RosDate}
	    \item [{\rm Status \hfill :}] {\@RosStatus}
	    \item [{\rm Supersedes \hfill :}] {\@RosSupersedes}
	    \item [{\rm Distribution \hfill :}] {\@RosDistribution}
	    \item [{\rm Clearance \hfill :}] {\@RosClearance}
	    \item [{\rm Keywords \hfill :}] {\@RosKeywords}
	 \end{list}
      \end{large}
      \title{\@RosTitle}
      \begin{figure}[b]
	 \begin{picture}(404,64)(0,0)
	    \put(0,64){\line(1,0){404}}
	    \put(0,-4){\line(1,0){404}}
	    \put(0,59){\line(1,0){42}}
	    \begin{small}
	    \put(3,48){\sf PHILIPS}
	    \end{small}
	    \put(0,23){\line(0,1){36}}
	    \put(42,23){\line(0,1){36}}
	    \put(21,23){\oval(42,42)[bl]}
	    \put(21,23){\oval(42,42)[br]}
	    \put(21,23){\circle{40}}
	    \put(4,33){\line(1,0){10}}
	    \put(9,28){\line(0,1){10}}
	    \put(9,36){\line(1,0){6}}
	    \put(12,33){\line(0,1){6}}
	    \put(29,13){\line(1,0){10}}
	    \put(34,8){\line(0,1){10}}
	    \put(28,10){\line(1,0){6}}
	    \put(31,7){\line(0,1){6}}

	    \put(1,21){\line(1,0){0.5}}
	    \put(1.5,21.3){\line(1,0){0.5}}
	    \put(2,21.6){\line(1,0){0.5}}
	    \put(2.5,21.9){\line(1,0){0.5}}
	    \put(3,22.1){\line(1,0){0.5}}
	    \put(3.5,22.3){\line(1,0){0.5}}
	    \put(4,22.5){\line(1,0){0.5}}
	    \put(4.5,22.7){\line(1,0){0.5}}
	    \put(5,22.8){\line(1,0){0.5}}
	    \put(5.5,22.9){\line(1,0){0.5}}
	    \put(6,23){\line(1,0){0.5}}
	    \put(6.5,22.9){\line(1,0){0.5}}
	    \put(7,22.8){\line(1,0){0.5}}
	    \put(7.5,22.7){\line(1,0){0.5}}
	    \put(8,22.5){\line(1,0){0.5}}
	    \put(8.5,22.3){\line(1,0){0.5}}
	    \put(9,22.1){\line(1,0){0.5}}
	    \put(9.5,21.9){\line(1,0){0.5}}
	    \put(10,21.6){\line(1,0){0.5}}
	    \put(10.5,21.3){\line(1,0){0.5}}

	    \put(1,23){\line(1,0){0.5}}
	    \put(1.5,23.3){\line(1,0){0.5}}
	    \put(2,23.6){\line(1,0){0.5}}
	    \put(2.5,23.9){\line(1,0){0.5}}
	    \put(3,24.1){\line(1,0){0.5}}
	    \put(3.5,24.3){\line(1,0){0.5}}
	    \put(4,24.5){\line(1,0){0.5}}
	    \put(4.5,24.7){\line(1,0){0.5}}
	    \put(5,24.8){\line(1,0){0.5}}
	    \put(5.5,24.9){\line(1,0){0.5}}
	    \put(6,25){\line(1,0){0.5}}
	    \put(6.5,24.9){\line(1,0){0.5}}
	    \put(7,24.8){\line(1,0){0.5}}
	    \put(7.5,24.7){\line(1,0){0.5}}
	    \put(8,24.5){\line(1,0){0.5}}
	    \put(8.5,24.3){\line(1,0){0.5}}
	    \put(9,24.1){\line(1,0){0.5}}
	    \put(9.5,23.9){\line(1,0){0.5}}
	    \put(10,23.6){\line(1,0){0.5}}
	    \put(10.5,23.3){\line(1,0){0.5}}

	    \put(1,25){\line(1,0){0.5}}
	    \put(1.5,25.3){\line(1,0){0.5}}
	    \put(2,25.6){\line(1,0){0.5}}
	    \put(2.5,25.9){\line(1,0){0.5}}
	    \put(3,26.1){\line(1,0){0.5}}
	    \put(3.5,26.3){\line(1,0){0.5}}
	    \put(4,26.5){\line(1,0){0.5}}
	    \put(4.5,26.7){\line(1,0){0.5}}
	    \put(5,26.8){\line(1,0){0.5}}
	    \put(5.5,26.9){\line(1,0){0.5}}
	    \put(6,27){\line(1,0){0.5}}
	    \put(6.5,26.9){\line(1,0){0.5}}
	    \put(7,26.8){\line(1,0){0.5}}
	    \put(7.5,26.7){\line(1,0){0.5}}
	    \put(8,26.5){\line(1,0){0.5}}
	    \put(8.5,26.3){\line(1,0){0.5}}
	    \put(9,26.1){\line(1,0){0.5}}
	    \put(9.5,25.9){\line(1,0){0.5}}
	    \put(10,25.6){\line(1,0){0.5}}
	    \put(10.5,25.3){\line(1,0){0.5}}

	    \put(11,21){\line(1,0){0.5}}
	    \put(11.5,20.7){\line(1,0){0.5}}
	    \put(12,20.4){\line(1,0){0.5}}
	    \put(12.5,20.1){\line(1,0){0.5}}
	    \put(13,19.9){\line(1,0){0.5}}
	    \put(13.5,19.7){\line(1,0){0.5}}
	    \put(14,19.5){\line(1,0){0.5}}
	    \put(14.5,19.3){\line(1,0){0.5}}
	    \put(15,19.2){\line(1,0){0.5}}
	    \put(15.5,19.1){\line(1,0){0.5}}
	    \put(16,19){\line(1,0){0.5}}
	    \put(16.5,19.1){\line(1,0){0.5}}
	    \put(17,19.2){\line(1,0){0.5}}
	    \put(17.5,19.3){\line(1,0){0.5}}
	    \put(18,19.5){\line(1,0){0.5}}
	    \put(18.5,19.7){\line(1,0){0.5}}
	    \put(19,19.9){\line(1,0){0.5}}
	    \put(19.5,20.1){\line(1,0){0.5}}
	    \put(20,20.4){\line(1,0){0.5}}
	    \put(20.5,20.7){\line(1,0){0.5}}

	    \put(11,23){\line(1,0){0.5}}
	    \put(11.5,22.7){\line(1,0){0.5}}
	    \put(12,22.4){\line(1,0){0.5}}
	    \put(12.5,22.1){\line(1,0){0.5}}
	    \put(13,21.9){\line(1,0){0.5}}
	    \put(13.5,21.7){\line(1,0){0.5}}
	    \put(14,21.5){\line(1,0){0.5}}
	    \put(14.5,21.3){\line(1,0){0.5}}
	    \put(15,21.2){\line(1,0){0.5}}
	    \put(15.5,21.1){\line(1,0){0.5}}
	    \put(16,21){\line(1,0){0.5}}
	    \put(16.5,21.1){\line(1,0){0.5}}
	    \put(17,21.2){\line(1,0){0.5}}
	    \put(17.5,21.3){\line(1,0){0.5}}
	    \put(18,21.5){\line(1,0){0.5}}
	    \put(18.5,21.7){\line(1,0){0.5}}
	    \put(19,21.9){\line(1,0){0.5}}
	    \put(19.5,22.1){\line(1,0){0.5}}
	    \put(20,22.4){\line(1,0){0.5}}
	    \put(20.5,22.7){\line(1,0){0.5}}

	    \put(11,25){\line(1,0){0.5}}
	    \put(11.5,24.7){\line(1,0){0.5}}
	    \put(12,24.4){\line(1,0){0.5}}
	    \put(12.5,24.1){\line(1,0){0.5}}
	    \put(13,23.9){\line(1,0){0.5}}
	    \put(13.5,23.7){\line(1,0){0.5}}
	    \put(14,23.5){\line(1,0){0.5}}
	    \put(14.5,23.3){\line(1,0){0.5}}
	    \put(15,23.2){\line(1,0){0.5}}
	    \put(15.5,23.1){\line(1,0){0.5}}
	    \put(16,23){\line(1,0){0.5}}
	    \put(16.5,23.1){\line(1,0){0.5}}
	    \put(17,23.2){\line(1,0){0.5}}
	    \put(17.5,23.3){\line(1,0){0.5}}
	    \put(18,23.5){\line(1,0){0.5}}
	    \put(18.5,23.7){\line(1,0){0.5}}
	    \put(19,23.9){\line(1,0){0.5}}
	    \put(19.5,24.1){\line(1,0){0.5}}
	    \put(20,24.4){\line(1,0){0.5}}
	    \put(20.5,24.7){\line(1,0){0.5}}

	    \put(21,21){\line(1,0){0.5}}
	    \put(21.5,21.3){\line(1,0){0.5}}
	    \put(22,21.6){\line(1,0){0.5}}
	    \put(22.5,21.9){\line(1,0){0.5}}
	    \put(23,22.1){\line(1,0){0.5}}
	    \put(23.5,22.3){\line(1,0){0.5}}
	    \put(24,22.5){\line(1,0){0.5}}
	    \put(24.5,22.7){\line(1,0){0.5}}
	    \put(25,22.8){\line(1,0){0.5}}
	    \put(25.5,23.9){\line(1,0){0.5}}
	    \put(26,23){\line(1,0){0.5}}
	    \put(26.5,22.9){\line(1,0){0.5}}
	    \put(27,22.8){\line(1,0){0.5}}
	    \put(27.5,22.7){\line(1,0){0.5}}
	    \put(28,22.5){\line(1,0){0.5}}
	    \put(28.5,22.3){\line(1,0){0.5}}
	    \put(29,22.1){\line(1,0){0.5}}
	    \put(29.5,21.9){\line(1,0){0.5}}
	    \put(30,21.6){\line(1,0){0.5}}
	    \put(30.5,21.3){\line(1,0){0.5}}

	    \put(21,23){\line(1,0){0.5}}
	    \put(21.5,23.3){\line(1,0){0.5}}
	    \put(22,23.6){\line(1,0){0.5}}
	    \put(22.5,23.9){\line(1,0){0.5}}
	    \put(23,24.1){\line(1,0){0.5}}
	    \put(23.5,24.3){\line(1,0){0.5}}
	    \put(24,24.5){\line(1,0){0.5}}
	    \put(24.5,24.7){\line(1,0){0.5}}
	    \put(25,24.8){\line(1,0){0.5}}
	    \put(25.5,24.9){\line(1,0){0.5}}
	    \put(26,25){\line(1,0){0.5}}
	    \put(26.5,24.9){\line(1,0){0.5}}
	    \put(27,24.8){\line(1,0){0.5}}
	    \put(27.5,24.7){\line(1,0){0.5}}
	    \put(28,24.5){\line(1,0){0.5}}
	    \put(28.5,24.3){\line(1,0){0.5}}
	    \put(29,24.1){\line(1,0){0.5}}
	    \put(29.5,23.9){\line(1,0){0.5}}
	    \put(30,23.6){\line(1,0){0.5}}
	    \put(30.5,23.3){\line(1,0){0.5}}

	    \put(21,25){\line(1,0){0.5}}
	    \put(21.5,25.3){\line(1,0){0.5}}
	    \put(22,25.6){\line(1,0){0.5}}
	    \put(22.5,25.9){\line(1,0){0.5}}
	    \put(23,26.1){\line(1,0){0.5}}
	    \put(23.5,26.3){\line(1,0){0.5}}
	    \put(24,26.5){\line(1,0){0.5}}
	    \put(24.5,26.7){\line(1,0){0.5}}
	    \put(25,26.8){\line(1,0){0.5}}
	    \put(25.5,26.9){\line(1,0){0.5}}
	    \put(26,27){\line(1,0){0.5}}
	    \put(26.5,26.9){\line(1,0){0.5}}
	    \put(27,26.8){\line(1,0){0.5}}
	    \put(27.5,26.7){\line(1,0){0.5}}
	    \put(28,26.5){\line(1,0){0.5}}
	    \put(28.5,26.3){\line(1,0){0.5}}
	    \put(29,26.1){\line(1,0){0.5}}
	    \put(29.5,25.9){\line(1,0){0.5}}
	    \put(30,25.6){\line(1,0){0.5}}
	    \put(30.5,25.3){\line(1,0){0.5}}

	    \put(31,21){\line(1,0){0.5}}
	    \put(31.5,20.7){\line(1,0){0.5}}
	    \put(32,20.4){\line(1,0){0.5}}
	    \put(32.5,20.1){\line(1,0){0.5}}
	    \put(33,19.9){\line(1,0){0.5}}
	    \put(33.5,19.7){\line(1,0){0.5}}
	    \put(34,19.5){\line(1,0){0.5}}
	    \put(34.5,19.3){\line(1,0){0.5}}
	    \put(35,19.2){\line(1,0){0.5}}
	    \put(35.5,19.1){\line(1,0){0.5}}
	    \put(36,19){\line(1,0){0.5}}
	    \put(36.5,19.1){\line(1,0){0.5}}
	    \put(37,19.2){\line(1,0){0.5}}
	    \put(37.5,19.3){\line(1,0){0.5}}
	    \put(38,19.5){\line(1,0){0.5}}
	    \put(38.5,19.7){\line(1,0){0.5}}
	    \put(39,19.9){\line(1,0){0.5}}
	    \put(39.5,20.1){\line(1,0){0.5}}
	    \put(40,20.4){\line(1,0){0.5}}
	    \put(40.5,20.7){\line(1,0){0.5}}

	    \put(31,23){\line(1,0){0.5}}
	    \put(31.5,22.7){\line(1,0){0.5}}
	    \put(32,22.4){\line(1,0){0.5}}
	    \put(32.5,22.1){\line(1,0){0.5}}
	    \put(33,21.9){\line(1,0){0.5}}
	    \put(33.5,21.7){\line(1,0){0.5}}
	    \put(34,21.5){\line(1,0){0.5}}
	    \put(34.5,21.3){\line(1,0){0.5}}
	    \put(35,21.2){\line(1,0){0.5}}
	    \put(35.5,21.1){\line(1,0){0.5}}
	    \put(36,21){\line(1,0){0.5}}
	    \put(36.5,21.1){\line(1,0){0.5}}
	    \put(37,21.2){\line(1,0){0.5}}
	    \put(37.5,21.3){\line(1,0){0.5}}
	    \put(38,21.5){\line(1,0){0.5}}
	    \put(38.5,21.7){\line(1,0){0.5}}
	    \put(39,21.9){\line(1,0){0.5}}
	    \put(39.5,22.1){\line(1,0){0.5}}
	    \put(40,22.4){\line(1,0){0.5}}
	    \put(40.5,22.7){\line(1,0){0.5}}

	    \put(31,25){\line(1,0){0.5}}
	    \put(31.5,24.7){\line(1,0){0.5}}
	    \put(32,24.4){\line(1,0){0.5}}
	    \put(32.5,24.1){\line(1,0){0.5}}
	    \put(33,23.9){\line(1,0){0.5}}
	    \put(33.5,23.7){\line(1,0){0.5}}
	    \put(34,23.5){\line(1,0){0.5}}
	    \put(34.5,23.3){\line(1,0){0.5}}
	    \put(35,23.2){\line(1,0){0.5}}
	    \put(35.5,23.1){\line(1,0){0.5}}
	    \put(36,23){\line(1,0){0.5}}
	    \put(36.5,23.1){\line(1,0){0.5}}
	    \put(37,23.2){\line(1,0){0.5}}
	    \put(37.5,23.3){\line(1,0){0.5}}
	    \put(38,23.5){\line(1,0){0.5}}
	    \put(38.5,23.7){\line(1,0){0.5}}
	    \put(39,23.9){\line(1,0){0.5}}
	    \put(39.5,24.1){\line(1,0){0.5}}
	    \put(40,24.4){\line(1,0){0.5}}
	    \put(40.5,24.7){\line(1,0){0.5}}
	    \begin{large}
	       \put(60,45){Philips Research Laboratories}
	       \put(60,30){\copyright\ 1988 Nederlandse Philips Bedrijven B.V.}
	    \end{large}
	 \end{picture}
      \end{figure}
      \newpage
      \pagenumbering{roman}
      \tableofcontents
      \newpage
      \pagenumbering{arabic}
   \end{titlepage}
}
\title{}
\topmargin 0pt
\oddsidemargin 36pt
\evensidemargin 36pt
\textheight 600pt
\textwidth 405pt
\pagestyle{headings}
\newcommand{\@RosTopic}{General}
\newcommand{\@RosTitle}{-}
\newcommand{\@RosAuthor}{-}
\newcommand{\@RosDocNr}{}
\newcommand{\@RosDate}{}
\newcommand{\@RosStatus}{informal}
\newcommand{\@RosSupersedes}{-}
\newcommand{\@RosDistribution}{Project}
\newcommand{\@RosClearance}{Project}
\newcommand{\@RosKeywords}{}
\newcommand{\RosTopic}[1]{\renewcommand{\@RosTopic}{#1}}
\newcommand{\RosTitle}[1]{\renewcommand{\@RosTitle}{#1}}
\newcommand{\RosAuthor}[1]{\renewcommand{\@RosAuthor}{#1}}
\newcommand{\RosDocNr}[1]{\renewcommand{\@RosDocNr}{#1 (RWR-102-RO-90#1-RO)}}
\newcommand{\RosDate}[1]{\renewcommand{\@RosDate}{#1}}
\newcommand{\RosStatus}[1]{\renewcommand{\@RosStatus}{#1}}
\newcommand{\RosSupersedes}[1]{\renewcommand{\@RosSupersedes}{#1}}
\newcommand{\RosDistribution}[1]{\renewcommand{\@RosDistribution}{#1}}
\newcommand{\RosClearance}[1]{\renewcommand{\@RosClearance}{#1}}
\newcommand{\RosKeywords}[1]{\renewcommand{\@RosKeywords}{#1}}

