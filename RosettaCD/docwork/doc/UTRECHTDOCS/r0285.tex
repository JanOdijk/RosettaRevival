
\documentstyle{Rosetta}
\begin{document}
   \RosTopic{General}
   \RosTitle{Notulen Linguistenvergadering 6-10-88}
   \RosAuthor{Margreet Sanders}
   \RosDocNr{0285}
   \RosDate{October 6, 1988}
   \RosStatus{informal}
   \RosSupersedes{-}
   \RosDistribution{Linguists, Joep Rous}
   \RosClearance{Project}
   \RosKeywords{minutes, superdeixis, modalen, prioriteiten}
   \MakeRosTitle
%
%
\begin{description}
\item[Aanwezig:] Lisette Appelo, Franciska de Jong, Elly van Munster,
                 Jan Odijk, Margreet Sanders (not), Harm Smit
\item[Afwezig:] Andr\'{e} Schenk
\item[Agenda:]\mbox{}
  \begin{enumerate}
  \item Modalen
  \item Superdeixis in IdentPropFormation
  \item Testzinnen
  \item Prioriteiten
  \end{enumerate}
\end{description}

\section{Modalen}
Jan O.\ deelt een herziene versie uit van het document over modalen. 
Veranderingen t.o.v.\ de vorige versie zijn aangegeven met een ! in het schema. 
De wijzigingen staan toegelicht op de volgende bladzijden. Commentaar op dit 
voorstel kan tot en met de volgende linguistenvergadering (dinsdag 11 oktober) 
worden ingediend; anders wordt het voorstel door Jan O.\ (i.s.m.\ anderen) 
ge\"{i}mplementeerd.

\section{Superdeixis in IdentPropFormation}
In de subgrammatica voor de vorming van Identificationele structuren (Dit zijn 
de directeuren; Wat is dat, etc.) worden de woorden {\em het, dit\/} en {\em 
dat\/} in de startregels als kant en klare NP ge\"{i}ntroduceerd. In analyse moet 
deze NP dus verder in de NP-grammatica worden afgebroken. Probleem is dat de NP 
dan ook een waarde voor superdeixis moet hebben, terwijl de zin erboven 
allang geen deixis-waarde meer heeft. 

Het herschrijven van de superdeixis-regels zodat ze alleen checken of de 
gevonden waarde correct is, maar deze niet verzetten naar omegadeixis lijkt wel 
erg drastisch, ook in de consequenties voor andere (tijd)regels. Het werken met 
een variabele (NPVAR) in de startregels levert wat problemen op omdat in het 
Nederlands een aantal attributen van de NP verzet worden (bv. in de zin {\em 
Dit zijn de directeuren\/} wordt het attribuut {\em number\/} verzet naar 
`plural', om agreement te kunnen beregelen), en in het Engels zelfs een ander 
woord wordt ge\"{i}ntroduceerd ({\em This is ..\/} vs.\ {\em These are\/}).
De twee mogelijke oplossingen zijn: het syncategorematisch introduceren van de
betrokken woordjes samen met het parametriseren van de startregels, en 
(makkelijker, maar minder netjes) het simpelweg op {\em presentdeixis\/} zetten 
van de betrokken NPs. Aangezien ze toch nooit een ingebedde zin zullen nemen, 
kan dit niet tot ongewenste neveneffecten leiden. Jan O.\ zal samen met Elly en 
Margreet kijken welke van deze twee oplossingen het beste (en/of snelste) is.

Jan O.\ merkt op dat een dergelijke aanpak op dit moment in het Nederlands 
ook wordt gevolgd in de PREPP-subgrammatica, waar modificatie optreedt 
v\'{o}\'{o}r de superdeixis-regel ({\em drie dagen voor de oorlog\/}, etc.). In 
deze subgrammatica kan echter de volgorde van superdeixis en modificatie gewoon 
worden omgedraaid, zodat er helemaal geen problemen meer zijn.

\section{Testzinnen}
Iedereen moet het lijstje met interessante testzinnen of -constructies {\em 
vandaag, donderdag 6 oktober\/} inleveren bij Jan O., die dan voor verspreiding 
zal zorgen. Voor de volgende vergadering moet men bekeken hebben in hoeverre de 
lijstjes op elkaar af te stemmen zijn, zodat dan een definitieve testzinnenbank
kan worden vastgesteld.

\section{Prioriteiten}
Voor iedereen heeft het testen van wat er al geschreven is de hoogste 
prioriteit. Verder geldt: \\
{\bf Franciska} probeert zoveel mogelijk het Spaans (ADJP, NP) bij te werken, 
en zal met Jan de ADJP-subgrammatica afstemmen op de zinsgrammatica (controle-
expressies en transfer).\\
{\bf Lisette} is al bezig met regels voor de toekomende tijd, en wil de modalen 
in orde krijgen. Mocht er tijd over zijn, dan wil ze eventueel nog wel naar 
constructies met `aan het ...' kijken.\\
{\bf Elly} schrijft clitic-regels en als het nog lukt regels voor causatieven.
\\
{\bf Margreet} wil locatieve zinsadverbia aankunnen, en derivatie voor adverbia
(hier moet dan ook voor het Nederlands nog wat aan gebeuren). Als er nog tijd 
over is dan wil ze nog wel wat bijwoordelijke bijzinnen kunnen maken.\\
{\bf Jan O.} gaat Spaanse shiftregels schrijven, en een Spaanse ADVPPROP-
grammatica.\\
{\bf Harm} zal (samen met Jan O.) zorgen voor een apart woordenboek voor 
modalen, en gaat verder werken aan een programma dat de testwoordenboeken omzet 
naar de nieuwe notatie (met strings als m-key), zodat correspondenties tussen 
de woordenboeken beter bekeken kunnen worden.\\
{\bf Andr\'{e}} zal moeten werken aan het afstemmen van de Spaanse en Engelse 
idioomregels op de Nederlandse.\\

\noindent
Op dinsdag 1 november zullen we zien wat er allemaal van terecht is gekomen...
\end{document}

