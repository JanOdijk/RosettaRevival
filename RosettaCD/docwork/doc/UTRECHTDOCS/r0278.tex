
\documentstyle{Rosetta}
\begin{document}
   \RosTopic{General}
   \RosTitle{Vijfde halfjaarlijkse verslag aan Van Dale Lexicografie}
   \RosAuthor{H.E. Smit}
   \RosDocNr{0278}
   \RosDate{September 5, 1988}
   \RosStatus{informal}
   \RosSupersedes{-}
   \RosDistribution{Project}
   \RosClearance{Project}
   \RosKeywords{dictionary, Van Dale report}
   \MakeRosTitle
%
%
\hyphenation{woor-den-boe-ken werk-zaam-he-den je-ne-ver-smaak moei-lijk
             jaar-ope-ning}

\section{Inleiding}

In dit vijfde halfjaarlijkse verslag zal een beknopt 
overzicht gegeven worden van de 
werkzaamheden aan de N-N, N-E en E-N bestanden in de periode van februari 1988
tot en met augustus 1988.

Tevens is weer een lijst van gevonden fouten en inconsistenties toegevoegd
(zie sectie 3).

\section{Overzicht van de werkzaamheden}

In het afgelopen halfjaar is gewerkt aan de programma's die op
de intersectie van de N-E en N-N (waarbij de lemma's uit de N-E genomen
worden met daaraan toegevoegd de morfologische informatie uit de N-N) 
werken en daaruit Rosetta3 woordenboeken genereren. 
Zoals in voorgaande verslagen al is opgemerkt worden
daarbij alle niet-morfologische attributen in de Rosetta3-lemma's 
default gevuld. Er zijn verschillende programma's voor de open categorie\"{e}n 
gemaakt (voor de {\em gesloten} categorie\"{e}n zal, zoals in vorige verslagen 
is toegelicht, geen gebruik worden gemaakt van de Van Dale lemma's) die
de woordenboeken voor zowel het Nederlands als het Engels genereren, en
tevens de vertaalrelatie leggen. 

Voor een aantal Engelse woorden was de vertaling zoals in de N-E gegeven niet
geschikt om als vertaling in de Rosetta-woordenboeken te dienen; het gaat
om gevallen waar de N-E niet een woord of compound levert, maar een complexer
geheel, zoals o.a. bij {\em jaarclub} het geval is: hier is als vertaling
{\em society of students of the same year} gegeven. Andere soortgelijke 
gevallen 
zijn te vinden bij {\em jeneversmaak} en {\em jaarinkomsten}.
Het is ook mogelijk dat de in de N-E gegeven vertaling niet eenduidig 
te analyseren
is, zoals bij: {\em jeugdafdeling} met als vertaling (in de gedrukte versie van 
de N-E) {\em youth/young people's/young persons' section} waarbij het bereik
van de schuine strepen ongedefinieerd is, zodat foute vertalingen als `{\em
youth people's section}' of zelfs `{\em young young persons' section}' 
gegenereerd zouden kunnen worden.
Tenslotte zijn er nog gevallen waarbij in de N-E alleen een omschrijving 
gegeven is, zoals bij: {\em jaarmis} (0.2), {\em jaaropening}, en {\em JAC}, of 
alleen een verwijzing, zoals bij {\em bezig} (0.3) en bij {\em jaloersheid}.
In al deze gevallen zal met de hand een vertaling moeten worden toegevoegd.

Het programma dat de Rosetta3 woordenboeken voor nouns genereert is een keer
over het hele bestand gedraaid: dit resulteerde in 56.426 Rosetta3 lemma's 
(d.w.z. 56.426 ``{\em s-keys}'') voor het Nederlands en 77.214 betekenissen 
(zgn. ``{\em m-keys}'') in de interlingua. Van de oorspronkelijke 60.791 lemma's
van de categorie noun (met de grammaticale code nummers 08 tot en met 19)
zijn dus ruim vierduizend `verdwenen', de meeste bij het maken van de 
intersectie van
N-N en N-E, waarbij o.a. ook gekeken werd of de romeinse onderverdeling van 
lemma's in de beide bestanden niet verschilde.

Wegens gebrek aan mankracht zal (in eerste instantie) slechts een deel van de
woorden uit de intersectie van N-N en N-E gevuld kunnen worden.
Het programma dat de te vullen subset (nl. de woorden van ROSETTA2 
--circa vijfduizend woorden--, waarbij deze woorden met {\em al} 
hun vervoegingen en verbuigingen, en in al hun betekenissen opgenomen worden)
genereert is (in een nieuwe, verbeterde versie) geschreven.

\section{Fouten en inconsistenties}

De volgende fouten en inconsistenties zijn gevonden:
\begin{itemize}
   \item het woord {\em NAR} in de N-N heeft als variant {\em N.A.M.}; dit
         had echter {\em N.A.R.} moeten zijn.
   \item in vergelijking met de tape-versie blijkt in de gedrukte versie van 
         de N-E een vrij groot aantal woorden te ontbreken
         beginnend met `mes-' en `met-', zoals o.a. {\em messing}, 
         {\em mestvee}, {\em mestvork}, {\em met}, {\em metaal}, 
         {\em metaaldraad}, enz. 
   \item het (Nederlandse) woord {\em assistent} is in de N-E fout 
         gespeld; er staat: {\em assistant}.
         Hetzelfde is geval bij {\em barnstenen}, wat als {\em barstenen}
         gespeld is in de N-E.
   \item vaak is de spelling van een ingangswoord in N-N en N-E verschillend;
         dit is bijvoorbeeld het geval bij: {\em enveloppe} (in de N-E) vs.
         {\em envelop} (in de N-N), {\em carburator} (N-E) vs. {\em carburateur}
         (N-N),. Het zal duidelijk zijn dat een programma dat beide bestanden 
         vergelijkt hier problemen mee krijgt.
   \item veel woorden op `-ing' afgeleid van werkwoorden zijn (opzettelijk?) 
         niet opgenomen in de N-N, maar wel in de N-E. Voorbeelden: 
         {\em begroeting}, {\em bevestiging}, {\em overmaking},
         {\em reservering}. M.i.  
         hadden deze woorden ook in de N-N opgenomen moeten worden, gezien
         het feit dat ze vaak ook een betekenis hebben die niet of niet op
         regelmatige wijze van het bijbehorende werkwoord afleidbaar is.
         Overigens ontbreekt het woord {\em beoordeling} in beide bestanden.
   \item Bij het vergelijken
         van de nouns uit de intersectie van N-N en N-E met het 
         Rosetta2-woordenboek bleek dat twee Rosetta2 woorden in zowel N-N als
         N-E ontbraken: {\em transistorradio} en {\em brandblusapparaat}.

\end{itemize}


\end{document}
