\documentstyle{Rosetta}
\begin{document}
   \RosTopic{General}
   \RosTitle{Dutch M-rules:subgrammar NPformation}
   \RosAuthor{Franciska de Jong}
   \RosDocNr{0377}
   \RosDate{\today}
   \RosStatus{concept}
   \RosSupersedes{-}
   \RosDistribution{Project}
   \RosClearance{Project}
   \RosKeywords{Dutch, M-rules, NPformation}
   \MakeRosTitle
%
%

\section{Introduction}
\section{Subgrammar Specification}
\section{Control Expression}
The control expression is defined as follows:
\section{Rules and Transformations}
\begin{mruleclass}{RC_NPformation}
\begin{classdescr}
\kind \nokind
\classtask \notask
\end{classdescr}

\begin{members}
\begin{member}
\rulename RNPformation1
\ruletask
Making an NP out of a CN containing a NOUN, and a DETP.
The DETP may be headed by DET, DEMADJ, QP, DETP (in case of a partitive),
or a numeral. 
\file dutch:npformation1.mrule
\semantics \nosemantics
\example 
\begin{verbatim}
T1 = DET:   die (aardige) jongen(s) / die (zoete) suiker
            dat (mooie) boek / dat (doffe) zilver
            deze (goede) zanger(s) / deze (lekkere) kaas
            dit (mooie) boek / dit (donkere) goud
T1 = NUM:   drie (groene) truien
T1 = DET:   elke (oude) man -- elk (oud) boek    ('eForm' vs. 'noform';
            menig                                  count/singular       )
            ieder
            ---

            veel boeken  -- veel kaas            (always 'noform';
            alle                                  count/plural vs. mass/singular)
            sommige
              ---

           vele                                 (always 'noform';
                                                always count/plural )
\end{verbatim}
\remarks\mbox{}
\begin{enumerate}
\item  
Articles are introduced syncategorematically by separate rules.
(RNPformation4 ({\em de} and {\em het})  and RNPformation4a ({\em 'n} and 
{\em een}).
Among the numerals two cases of {\em een}, both to be translated into 
the English numeral {\em one},
are distinguished: one with accent, 
and one without accent. The latter is excluded from this rule. It is only 
accepted as part of partitive DETPs, e.g. {\em een van de}.
As a consequence NPs such as {\em een boek} have only one analysis (viz. via 
RNPformation4a), and only 
one translation. E.g. {\em a book}.
\item problems:\\
---> There has to be written also a rule for: DEMADJ + {\em EN}.\\ \\
Considering the following data: \\ \\
\ \ \ \ \ \ - *dit EN \\ \\
\ \ \ \ \ \ - *dat EN \\ \\
\ \ \ \ \ \ -  dit mooie EN (Welk boek ...?  Dit mooie.) \\ \\  
\ \ \ \ \ \ -  dat mooie EN
(NB. Hij leest een mooi boek. vs. Hij leest een mooi{\em e}.) \\ \\           
\item observations:\\

\end{enumerate}

\end{member}
\begin{member}
\rulename RNPformation2
\ruletask
The formation of an NP out of a CN with (a) an empty head (EN), and a DETP

blocking of 'alle', 'elke', 'iedere', etc. 
as bare NP (see above).
Agreement.
The formation of an NP out of a CN with (a) a non-empty head (NOUN), 
and a DETP.
TO BE ADDED: model for EN plus ADJP
\begin{verbatim}
            NB. there is a problem with 'beide':
                    *alle
                    *alle met een strikje
              
              but:   beide
                    *beide met een strikje
           
            If 'beide' is considered def, the problem is that 'beide'  
            will be blocked wrongly;
            if 'beide' is considered adef, the problem is that the 
            ungrammatical 'beide met een strikje' won't be blocked.

    --> The best solution will probably be to assign 'beide' two categories: 
                                       INDEFPRO / DET
        Assuming now 'beide' is def, NPformation1 will block 'beide', but 
        another rule for INDEFPRO's will accept it.

\end{verbatim}
\file dutch:npformation1.mrule
\semantics \nosemantics
\example
         EN                 --> drie EN, enkele EN, de meeste, 
         gele EN            --> drie gele EN, alle gele EN
         honderd EN         --> alle honderd EN
         EN met een rietje  --> drie EN met een rietje
       
                  ................. etc.

\remarks\mbox{}
\begin{enumerate}
\item problems:\\
    Not yet specified: the value of NPrec1.possnietnp
                                          .syntquant         ;
    It should be investigated whether bare demonstratives (always 
    definite!!) should be dealt with by this rule (which would possibly
    require an extra attribute), or by another one.
\item modifications: 26/10/88: Added superdeixis DETP\\

\end{enumerate}

\end{member}
\begin{member}
\rulename RNPformation3
\ruletask
Syncategorematic introduction of {\em de}.
\file dutch:npformation1.mrule
\semantics \nosemantics
\example {\em (drie) (dure) EN} --> 
{\em {\bf de} (drie) (dure) EN} (alleen plural, d.w.z. de gele via andere regel)
\remarks\mbox{}
\begin{enumerate}
\item problems:\\
\item modifications:\\

\end{enumerate}

\end{member}
\begin{member}
\rulename RNPformation4
\ruletask
Syncategorematic Introduction of 'de'/'het'/`een` in case of non empty NP-head ,
possibly  modified. The function of the matchcondition is to block the
application of this rule where RNPformation5 should apply, viz. for CN-strings
that contain a NP instead of a PREPP in posrel.
\file dutch:npformation1.mrule
\semantics \nosemantics
\example
\remarks\mbox{}
\begin{enumerate}
\item problems:\\
--->  C1 of SUBRULE 1 is equivalent to: \\ \\
                   NOUNrec1.genders <= [neutgender]
\item modifications:\\

\end{enumerate}

\end{member}
\begin{member}
\rulename RNPformation4a
\ruletask
\file dutch:npformation1.mrule
\semantics \nosemantics
\example
\remarks\mbox{}
\begin{enumerate}
\item problems:\\

\item modifications:\\

\end{enumerate}

\end{member}
\begin{member}
\rulename RNPformation5
\ruletask Changing a postnominal 'posrel' (modification of a CN 
containing a NOUN), into a prenominal 'detrel'. 
(boek ik --> mijn boek, etc.)
\file dutch:npformation1.mrule
\semantics \nosemantics
\example 
\begin{enumerate}
\remarks\mbox{}
\begin{enumerate}
\item problems:\\
Ad SUBRULE 3, A1:  NPrec2.definite := def
It is assumes that 'iemands boek' is def, considering:
               - *{\em Er} ligt iemands boek op tafel.
If this sentence turns out to be grammatical though, 'definite := def'  can
be changed into:   NPrec1.definite := INDEFPRO.definite
\item modifications:\\

\end{enumerate}

\end{member}
\begin{member}
\rulename RNPformation10
\ruletask making a bare NP out of a  CN.
\file dutch:npformation1.mrule
\semantics \nosemantics
\example (mooie) boeken, (lekkere) kaas.
\remarks\mbox{}
\begin{enumerate}
\item problems:\\

\item modifications:\\

\end{enumerate}

\end{member}
\begin{member}
\rulename RNPformation12
\ruletask Making an NP (with 'head/DETP') out of a DETP + EN 
(the EN referring to 'human beings' (humanpar = true) or 'non-human beings' 
(humanpar = false)).
\file dutch:npformation1.mrule
\semantics \nosemantics
\example velen, allen, sommigen, enkelen.
\remarks\mbox{}
\begin{enumerate}
\item problems:\\
This rule can probably be incorporated into 'RNPformation2'.
\item modifications: 26/10/88: Added superdeixis for DETP.\\

\end{enumerate}

\end{member}
\begin{member}
\rulename TNPhop
\ruletask To account for the occurrence of postnominal determiners (QPs).
\file dutch:npformation1.mrule
\semantics \nosemantics
\example genoeg tijd --> tijd genoeg
\remarks\mbox{}
\begin{enumerate}
\item problems:\\

\item modifications:\\

\end{enumerate}

\end{member}
\end{members}

\nofilters

\nospeedrules

\noplannedrules

\norulesnotince

\begin{comments}
\end{comments}

\end{mruleclass}

\end{document}
