
\documentstyle{Rosetta}
\begin{document}
   \RosTopic{General}
   \RosTitle{Notulen Linguistenvergadering 11-10-88}
   \RosAuthor{Margreet Sanders}
   \RosDocNr{0286}
   \RosDate{November 9, 1988}
   \RosStatus{approved}
   \RosSupersedes{-}
   \RosDistribution{Linguists, Joep Rous}
   \RosClearance{Project}
   \RosKeywords{minutes, superdeixis, modalen}
   \MakeRosTitle
%
%
\begin{description}
\item[Aanwezig:] Lisette Appelo, Elly van Munster, Jan Odijk,
                 Margreet Sanders (not), Andr\'{e} Schenk, Harm Smit
\item[Afwezig:] Franciska de Jong
\item[Agenda:]\mbox{}
  \begin{enumerate}
  \item Modalen
  \item Superdeixis
  \item Testzinnen
  \end{enumerate}
\end{description}

\section{Modalen}
Naar aanleiding van het nieuwe document over modalen vraagt Margreet of `may' 
in de betekenis van {\em permission\/} niet ook in het Nederlandse `kunnen' 
vertaald moet worden ({\em You may go: U kunt gaan\/}). Besloten wordt dit 
alleen in Engelse analyse toe te staan.

Bij nader inzien blijkt het toch niet wenselijk om `might' ook te beschouwen als 
een gewone verleden tijd van `may'. Margreet zal, na overleg met Lisette en Jan 
O., de Engelse morfologie zo wijzigen dat `may' en `might' aparte werkwoorden 
zijn, waarvan `may' geen verleden tijd heeft en `might' misschien ook niet.
Daarna zal Lisette bekijken wat er nog meer moet veranderen aan de 
(super)deixis-regels om de vertalingen goed te krijgen. Het is de vraag of dit 
alles lukt voor 1 november.

Het voorstel wordt/is verder ongewijzigd ge\"{i}mplementeerd. Overigens is al 
op de vorige vergadering opgemerkt dat er nog een kleine typefout staat op 
p.\ 4 in de 
derde alinea: {\em had\/} + {\em can\/} wordt natuurlijk herschreven in {\em 
could\/} + {\em have\/}.

\section{Superdeixis}
Het probleem dat Jan O.\ de vorige keer aanroerde over superdeixis in de 
identificationele subgrammatica (en dat hij daar heeft opgelost door 
superdeixis maar willekeurig op presentdeixis te zetten) blijkt nu ook voor te 
komen bij modificatie van adjectieven (bv. {\em 
Hij wordt\/} zeer oud; {\em Het is} te {\em mooi\/} om waar te zijn). 
Modificatie 
gebeurt namelijk al in de ADJPPROP-formatie, en superdeixis wordt in de 
XPPROPtoCLAUSE grammatica beregeld (als alternatief voor de superdeixis 
regels uit de AJPPROPtoFORMULA subgrammatica). In analyse betekent dit dat na 
het omega maken van de superdeixis, in de ADJPPROP-grammmatica uit het niets 
alsnog een superdeixiswaarde moet worden gevonden om aan modificerende 
ingebedde zinnen door te kunnen geven ({\em Hij\/} was {\em zo oud dat hij 
helemaal krom\/} liep). Omdat hier echte zinnen kunnen optreden als modificator
, i.t.t.\ de situatie bij identificationele NPs {\em dit, dat, het\/}, is het 
hier onmogelijk om zomaar een waarde te kiezen voor superdeixis. 
Jan O.\ formuleert nogmaals het voorstel om het 
vertaalaspect van superdeixis naar voren te verschuiven door al in de 
startregels een superdeixis-parameter mee te geven, en de superdeixisregels 
zelf (die dan ook transformaties mogen worden en eventueel ook op andere 
plaatsen in de subgrammaticas mogen worden gezet) te herschrijven zodat ze 
superdeixis niet meer verzetten, maar alleen op correctheid checken. Het 
tegenvoorstel van Lisette is om de modificatieregels naar achteren te 
verschuiven, na de superdeixis-bepaling. Dat betekent dat deze regels ook voor 
de CLAUSE-grammatica moeten worden geschreven. 

Aangezien niemand tijd lijkt te hebben om voor 1 november beide voorstellen uit 
te werken (inclusief alle consequenties voor andere regels), wordt de oplossing 
van dit probleem maar verschoven tot na de deadline, met in het achterhoofd ook 
nog dat het uiteindelijk toch allemaal anders zal moeten (men zou het 
liefst superdeixisregels helemaal weghalen uit het systeem, en alleen op 
syntactische derivatiebomen laten werken). Dat betekent dus ook dat modificatie 
van adjectieven {\em niet\/} op alle manieren mogelijk zal zijn op 1 november.

\section{Testzinnen}
De testzinnen zijn al wel ingeleverd, maar nog niet gekopieerd en rondgedeeld. 
Daarom wordt dit punt verschoven naar de volgende vergadering.

\end{document}

