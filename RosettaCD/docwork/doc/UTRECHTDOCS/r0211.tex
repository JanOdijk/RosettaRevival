
\documentstyle{Rosetta}
\begin{document}
   \RosTopic{General}
   \RosTitle{Local Dutch M-rules. Documentation.}
   \RosAuthor{Margreet Sanders}
   \RosDocNr{0211}
   \RosDate{\today}
   \RosStatus{concept}
   \RosSupersedes{-}
   \RosDistribution{Linguists, Joep Rous}
   \RosClearance{Project}
   \RosKeywords{Local M-rules}
   \MakeRosTitle
%
\newpage
\section{Introduction}
In this document the local Dutch M-rules are enumerated and explained. Also, 
some general decisions  concerning local rules are mentioned, together with 
problems related to specific rules. The rules themselves are not incorporated in 
this document: they can be found under 
[SANDERS.ROSETTA3.DUTCH.WORK]RLOCAL*.tex.
In Appendix A, a full list is given of these files and the rules they contain.
For syntactic categories that do not change in morphology (e.g.\ ART, CONJ, 
ORDINAL, PREP) no rules have been written, since they can be translated as 
basic expressions without any rule having applied to them.

Most rules are written on an ad hoc basis. Some general decisions are mentioned 
in the following section. All points on which further decisions should be made 
are marked with an asterisk in the margin.

\section{Goal of Local M-rules}
The local rules written for Dutch are intended for the word translation mode, 
i.e.\ when one word is entered into the computer, all suitable 
translations (consisting of as many words as necessary) should be returned. 
The rules are not meant to serve as
robustness measures in the ordinary translation. It is as yet unclear whether 
they could perhaps be used for that purpose.

The translations returned by the computer should comprise all semantic 
variants,
but not variants determined solely by syntactic constraints. For example, the 
attributive adjective {\em mooi} is given in this form only, and not also in 
the form {\em 
mooie}, since there are no semantic reasons for distinguishing the two. Of 
course, {\em mooie} will be generated when a nominalised adjective is concerned
(but see the remarks in the section on adjectives).
It was decided to generate all `lexico-morphological' equivalents of output 
words, i.e.\ 
when for example an adjective has both an inflectional and a periphrastic 
comparative, both are shown. This may be an implicit help for the user: when 
only one form is shown, the other obviously does not exist for the word 
concerned.
A final choice concerns the symmetry between analysis and generation components. 
In case the output exists of more than one word, symmetry is of course 
impossible. Since the rule compiler does not yet allow for rules without a 
DECOMP part, analysis of more than one input word still seems possible in 
certain cases. This possibility should be removed as soon as possible.
Another source of asymmetry is the decision to generate only 
unambiguous forms once a certain word is disambiguated in the IL. For example, 
the personal pronoun {\em mij\/} may be both an accusative and a dative. In IL, 
the form is mapped onto both possibilities, and the generative component will 
produce one {\em mij (acc)\/} and one {\em mij (dat)\/}. In implementation, certain 
measures may be taken to prevent a proliferation of identical output. (The fact 
that the cases of the two {\em mij\/}s differ cannot be seen by the user, since 
the attribute values are not meant to be accessible to him or her.)

In the rules described in this document, derivation is still disregarded.

The organisation of the local M-rules (one subgrammar, one ruleclass ?) has not 
been decided yet. Also, the way to get an expression through S-parser is still 
unspecified.

\section{The Rules}

\subsection{Verbs}

These rules form the present and past tense of indicatives, conjunctives and 
participles, and make infinitives and imperatives. Also, a rule is added for 
the eORenForm of participles.

The correct placement of particles should be accounted for by the ParticlePlace 
transformations, which apply optionally after Pres/Past Indic/Conj (both {\em opbel} and 
{\em bel op} are possible), obligatory after VerbImp, and never after Pres/Past 
Part and VerbInf. This strategy has not been formalized in any way yet.

No decision has been taken yet on the mapping of tenses in different languages. 
Also, rules will have to be added for the formation of periphrastic tenses of 
verbs from conjugation class [0] (e.g.\ {\em buikspreken\/}, with translation 
{\em ventriloquize\/}, a verb with full conjugation). It is unclear whether and how 
participles for conjugation class [13] ({\em plegen}) should be produced.

The subjunctive poses some problems: ANS claims that only the third person 
singular has one. If that is true, then what should the translation be of other 
subjunctive forms (esp.\ those coming from Spanish)?

Finally, the eORen rule cannot refer to an attribute eFormation or eNominalised
in verbs. Consequently, eORenFormation is allowed for all participles (present and 
past) at the moment. It seems that some restrictions can be formulated based on 
the conjugation class the verbs have. For remarks on the use of -e and -enForms, 
see the discussion in section 3.4 on adjectives.

\subsection{Nouns}

The rules produce singular and plural forms of nouns. Nouns that have 
[OnlyPlural] are dealt with in the singular rule. It is assumed that nouns with 
[NoPlural] are also basically singular. A translation problem exists for nouns 
that can take a plural in one language, but not in another 
({\em joys - vreugde, anti aircraft guns - afweergeschut}).

Propernouns that obligatorily take an article ({\em Rhine, Hebrides}) are listed 
as BNOUNs in the dictionary. Propernouns that only take an article when they 
are modified ({\em the beautiful Germany}) are converted into subnouns by the 
rule RlocalBproperToSubnoun (cf. following section).

The Noun rules also take care of genitive forms. It is assumed that an 
inflectional genitive can only occur for a limited number of singular nouns:
converted propernouns and the restricted set containing 
{\em moeder, (?)huis\/}. An even more restricted set of nouns (those of the 
former rule ending in -s when plural) also takes an 
inflectional genitive when plural ({\em ouders'\/}).
Other nouns get a periphrastic phrase with {\em van de/het\/}.

\subsection{Propernouns}

It is unclear what propernouns can do while remaining propernouns. Can they form 
plurals and genitives, or must they be converted to (sub)nouns for that? The 
current propernoun rules allow for the formation of all forms nouns can take, 
but perhaps all plural rules should simply be discarded. Also, it is doubtful 
whether [OnlyPlural] propernouns exist that do not take an article; no examples 
could be found (propernouns that always take an article, are listed as BNOUNs 
in the dictionary).

\subsection{Adjectives}

These rules take care of positives, comparatives and superlatives of 
adjectives. Positives and comparatives may have NoForm or sForm ({\em vlug(s), 
vlugger(s)}). It is not quite clear to what English rule the sForm corresponds. 
Probably, the formpar can simply be discarded there, yielding the same 
translation for both forms.

It is assumed that adjectives that are [NoComp] in one language 
will also be [NoComp] in other languages ({\em houten - wooden}). The rules 
also introduce {\em meer\/} and {\em meest\/} for adjectives that do not take 
an inflectional comparative or superlative. The AllerSuperlative will be 
analysed only; in IL, only one superlative is recognised.

The AdjeORen rule produces -e and -en endings for NoForm adjectives in all 
degrees of comparison; the -eForm 
both for attributive and nominalised adjectives, the -enForm only for the 
latter. However, as was mentioned in section 2, attributive -e should not be 
generated, since the distinction between attributive forms with and without a 
final -e is 
determined exclusively by syntactic considerations. On the other hand, it may 
be useful for the user to know whether an adjective can take an attributive -e 
ending, since not all adjectives do
({\em tevreden\/}, as contrasted to the 
nominalised form {\em tevredene\/}). This information would be more or less on 
the same line as spelling out whether an adjective has both
an inflectional and a
periphrastic comparative. Finally, the use of only one formparameter
for both attributive -e (no semantics) and nominalised -e (meaningful) does not 
seem correct. 
A well-formulated decision should be taken 
here. This decision should hold for eORenFormation of the participles of verbs 
and for determiners and adverbs as well.

\subsection{Adverbs}

As was mentioned in section 2, derivation has not been taken into account yet. 
It has been proposed to give Dutch adverbs an empty suffix, in line with the 
{\em -ly\/} 
suffix for English adverbs. How full isomorphism can be achieved between both 
languages with regard to adverbs is not quite clear yet.

Adverbs can also form comparatives and superlatives. It is unclear whether the 
generation of periphrastic comparatives, as provided for in the rules, is 
really needed. English adverbs do not normally have an inflectional 
comparative, and thus will not be input in the word translation mode. 
Exceptions are the combined adjective/adverbs like {\em early - earlier}, which
(perhaps accidentally) all translate into inflection comparatives ({\em vroeg - 
vroeger}). The superlative of adverbs takes the obligatory article {\em het\/}:
{\em het liefst, het meest priv\'{e}}. In analysis, superlatives without 
article are necessarily allowed. Again, allerSuperlatives are accepted in 
analysis only.

With adverbs, only superlatives may have -e endings ({\em het liefst(e) 
willen}). It is assumed that only inflectional superlatives can take a final -e;
periphrastic superlatives of adverbs cannot. It is doubtful whether this 
assumption is correct. However, it is also doubtful whether periphrastic 
superlatives of adverbs need to be generated at all (cf.\ the remarks made on 
this point in the previous section). Finally, the decision to generate -eForms 
is not consistent with what was said on this point in section 2; therefore, the 
decision asked for in the previous section should hold for adverbs as well.

\subsection{Personal Pronouns}
It is assumed that in IL, all 5 persons are distinguished both in singular and 
in plural. Dutch person 0 must not be included separately, since it does not 
have a meaning of its own. Instead, in generation all occurrences of person 2
(sing) must be mapped onto both person 2 and person 0.

Reduced forms are accepted in analysis, but will not be generated, except for 
the generalised {\em je: one, you} and {\em ze: they}. Care should be taken 
that morphology can handle all cases of the generalised perspros (
{\em je (acc,dat)\/} and {\em ze (acc,dat)\/}). As was 
mentioned in section 2, only unambiguous forms are generated with respect to 
case: even though the input is {\em hem (acc,dat)\/}, the output will be 
{\em hem (acc)\/} and {\em hem (dat)\/}.

It is not clear yet whether all forms going with the 5th person {\em gij\/} can 
be handled yet by the local m-rules and by morphology. Is {\em U\/} the dative/
accusative of {\em gij\/}? If so, will all Dutch occurrences of {\em U\/} also 
be mapped onto the English rule for {\em thou\/} or not? The same problem holds 
with respect to the possessive adjectives formed from {\em gij\/} and 
{\em U\/}. The alternative would be to treat {\em gij\/} and {\em thou\/} in 
analysis only; however, this will make a translation of {\em gij zult niet 
stelen\/} into {\em thou shalt not steal\/} impossible (unless perhaps the ten 
commandments are listed in the dictionary as idioms).

\subsection{Possessive Adjectives}

The rules accept reduced forms in analysis, but generate only full possessive 
adjectives. An exception is made for generalized {\em je\/} and {\em ze\/}, 
which have {\em je\/} and {\em hun\/} as possessives, respectively.
For {\em het\/} and {\em gij\/} no possessive forms exist; an extra 
rule is added to produce {\em ervan\/} as translation of English {\em its\/}.
It is assumed that in morphology a rule exists which spells out the possessive 
form of {\em gij\/} as {\em Uw\/}. However, in that case all occurrences of 
{\em Uw\/} will be ambiguous between the possessive of {\em U\/} and of 
{\em gij\/} (cf. the discussion in the previous section). A formal decision on 
this point should be taken. Finally, the rules produce both {\em ons\/} and 
{\em onze\/}. Perhaps this decision should be reconsidered when the general 
strategy concerning eORenFormation has been formulated.

The third possadj rule accounts for the genitive forms of possessive 
adjectives. Generalized personal pronouns, {\em jullie\/}, {\em gij\/} and 
{\em het\/} do not have this form. Since English and Spanish do not have this 
form either, no rules are added to produce an alternative solution.

\subsection{Possessive Pronouns}

Two separate rules are needed for analysis and for generation, since the 
translation of English PossPros needs an article {\em mine $\rightarrow$ de/het
mijne\/}, while in analysis only one word may be entered ({\em mijne\/}).
Generalized personal pronouns, {\em gij\/} and {\em het\/} do not form 
possessive pronouns. If {\em gij\/} is to be generated, morphology should have 
a rule spelling out the posspro of {\em gij\/} as {\em (de) Uwe\/}, as 
translation of English {\em thine\/}. 

Since {\em jullie\/} does not have a posspro either, a 
third rule is added to generate {\em die/dat van mij, jullie\/} etc. It could 
be argued that this rule should be reserved exclusively for the second person 
plural; however, there seems to be no harm in a few extra translations for the 
other persons.

No rule is formulated yet for -enForms of possessive pronouns, since the exact 
meaning of e.g.\ {\em de mijnen\/} is not clear. Personally, I think these 
forms are not simply plurals of {\em de mijne\/}, but have a distinct meaning 
roughly paraphrased as {\em those who are near to me\/}. Consequently, I also 
believe that {\em de mijne\/} is ambiguous between singular and plural 
reference. In case the plural form indeed has its own meaning, it should be 
entered in the dictionary as a basic expression; its correct translation into 
English will be a problem.

\subsection{Wh Possessive Adjectives}

It is assumed that three different forms of {\em wie\/} exist: ([masc], sing), 
([fem], sing) and ([], plur), to produce {\em wiens\/} and two times 
{\em wier\/} respectively. English has only one {\em whose\/}.

\subsection{Dem Possessive Adjectives}

It was decided to deal with {\em diens\/} in analysis only, since translation 
of {\em his\/} and {\em su\/} should not give rise to too many ambiguities.
As it is, {\em his\/} will produce {\em zijn, de/het zijne, die/dat van hem\/}, 
which seems quite enough to cover the meaning of {\em diens\/} as well. As is 
apparent from the previous, it is assumed that the Dutch DEMPRO {\em die\/} has 
as one of its possible translations the English PERSPRO {\em hij\/}. The Dutch 
DemPossAdj rule should then be mapped onto the English PossAdj rule. If Dutch 
demonstrative pronouns get an extra attribute  to distinguish {\em deze\/} from 
{\em die\/}, this attribute should also be mentioned in the rule.

\subsection{Determiners}

As with adjectives and adverbs, determiners are also generated with -e and -en 
forms, for attributive and nominalised use. Again, a decision as to the 
correctness of this approach should be made.

At the moment, no rule exists in morphology to form a genitive of determiners. 
However, forms like {\em beider\/} and {\em aller\/} seem to exist as genitives 
of {\em beiden\/} and {\em allen\/} respectively (does 
{\em veler\/} exist too?). Are these forms indeed determiners, derived from a 
nominalisation of {\em beide\/} and {\em alle\/}, or are they separate
indefinite pronouns (which can form genitives)? Should determiners get an 
attribute `possgeni'? Perhaps the same problem holds for {\em ieder(s)\/} and 
{\em elk(s)\/}. 

No rule exists at present to generate {\em \underline{de} meeste\/} as 
translation of English {\em most\/}. When all determiners are listed in the 
test dictionary, the rules will be scanned for other such deficiencies and 
will be adapted to current needs.

\subsection{Indefinite Pronouns}

It is assumed that all indefinite pronouns that can have a genitive in other 
languages (all pronouns referring to animate beings?), can also have one in 
Dutch, except for {\em men\/}. English 
{\em one's\/} is translated by a separate rule into {\em zijn\/}. Since 
{\em one\/} was not only mapped onto {\em men\/}, but also onto {\em je\/}, 
this means that the English IndefProGeni-rule should also be mapped onto the 
Dutch PossAdj-rule to produce the desired {\em je (gen)\/}.

When a full list of indefinite pronouns is available in the test dictionary, 
the rules will be checked for completeness.

\subsection{Wh Pronouns}

This is a vacuous rule, needed only because of the difference made in English 
between {\em who (nom, acc)\/} and {\em whom (acc)\/}.



\newpage
{\Large\bf Appendix}
\appendix
\section{List of Files with Local Dutch M-rules}

{\bf RlocalVerb}
\begin{enumerate}
\item RlocalSimpleSubVerb
\item TlocalParticleSpelling0 and 1
\item RlocalVerbPresIndic
\item RlocalVerbPastIndic
\item TlocalParticlePlace0 and 1
\item RlocalVerbPastPart
\item RlocalVerbPresPart
\item RlocalVerbInf
\item RlocalVerbImp(numberpar)
\item RlocalVerbPresConj
\item RlocalVerbPastConj
\item RlocalVerbeORen(eORenFormpar)
\end{enumerate}

\noindent
{\bf RlocalNoun}
\begin{enumerate}
\item RlocalSimpleSubNoun
\item RlocalNounSing
\item RlocalNounPlur
\item RlocalNounSingGeni1 and 2
\item RlocalNounPlurGeni1 and 2
\end{enumerate}

\noindent
{\bf RlocalPropernoun}
\begin{enumerate}
\item RlocalPropernounSing
\item RlocalPropernounPlur
\item RlocalPropernounSingGeni1 and 2
\item RlocalPropernounPlurGeni1 and 2
\item RlocalBProperToSubnoun
\end{enumerate}

\noindent
{\bf RlocalAdj}
\begin{enumerate}
\item RlocalSimpleSubAdj
\item RlocalBasicAdj(formpar)
\item RlocalAdjComp(formpar)
\item RlocalAdjSup
\item RlocalAdjeORen(eORenFormpar)
\end{enumerate}

\noindent
{\bf RlocalAdv}
\begin{enumerate}
\item RlocalSimpleSubAdv
\item RlocalBasicAdv
\item RlocalAdvComp
\item RlocalAdvSup(eORenFormpar)
\end{enumerate}

\noindent
{\bf RlocalPersPro}
\begin{enumerate}
\item RlocalPerspro(casespar)
\end{enumerate}

\noindent
{\bf RlocalPossAdj}
\begin{enumerate}
\item RlocalPossAdj1, 2, and 3
\end{enumerate}

\noindent
{\bf RlocalPossPro}
\begin{enumerate}
\item RlocalPossPro1, 2, and 3
\end{enumerate}

\noindent
{\bf RlocalWhPossAdj}
\begin{enumerate}
\item RlocalWhPossAdj(sexespar, numberpar)
\end{enumerate}

\noindent
{\bf RlocalDemPossAdj}
\begin{enumerate}
\item RlocalDemPossAdj
\end{enumerate}

\noindent
{\bf RlocalDet}
\begin{enumerate}
\item RlocalDet(eORenFormpar)
\end{enumerate}

\noindent
{\bf RlocalIndefPro}
\begin{enumerate}
\item RlocalIndefPro1(genitivepar) and 2
\end{enumerate}

\noindent
{\bf RlocalWhPro}
\begin{enumerate}
\item RlocalWhPro
\end{enumerate}

\end{document}
