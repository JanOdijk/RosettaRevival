
\documentstyle{Rosetta}
\begin{document}
   \RosTopic{English}
   \RosTitle{Testing English Analysis}
   \RosAuthor{Petra de Wit}
   \RosDocNr{468}
   \RosDate{\today}
   \RosStatus{concept}
   \RosSupersedes{-}
   \RosDistribution{Project}
   \RosClearance{Project}
   \RosKeywords{}
   \MakeRosTitle
%
%

\section {Introduction}
This document contains the sentences I used to test English analysis 
(and generation where needed) as well as the problems I encountered while 
testing. Also, an indication is given of which type of rules and constuctions 
contained the majority of the mistakes.\\

Apart from these constructions, the English grammar has also been tested by 
means of the Flickinger corpus. Results of these tests can be found in the 
"Flickinger Folder", currently available at Lisette's.\\

\section {Error Evaluation}
Rule classes where most errors occured:
\begin{description}
\item [Verbpatterns] especially rules and subrules dealing with PPs
\item [Clauseformation] All special clauseformation rules inserting a verb 
contained various mistakes.
\item [Controlrules] few, but copied mistakes
\item [Mood] especially the deviant mood rules
\end{description}
Type of mistakes which were commonly made writing M-rules:
\begin{itemize}
\item To copy the wrong record or to copy the record onto itself such that 
the constituant receives either the wrong or the default values.
\item To replace a hard value in a matchcondition by a related value while 
forgetting to add a colon since the type of matchcondition has now changed 
\item To assume the wrong default value (which is the default in another 
language)
\item To assume the wrong value for no reason
\item To mix up several T's 
\item To refer to a record in DECOMP that only exist in the input model (it 
would be nice if the compiler could give a warning !) 
\item EXCEPT FOR
\item NPONlex
\item Not reversible since condition was in Match Condition !!
\end{itemize}
\section {Test Sentences}
This section gives a list of rule classes plus sentences that were used to test 
their functioning. The examples given below are largely based on the sentences 
given in document R0292, together with all the necessary extensions and 
updating.


\section {Startrules}
\begin{description}
\item [* RstartvpProp000:] It is raining
\item [* RstartvpProp100:] I am swimming
\item [* RstartvpProp010a:] These things happen
\item [* RstartvpProp010b:] He seems ill
\item [* RstartvpProp120:] He loaded the ship
\item [* RstartvpProp012:] The prince turned into a frog
\item [* RstartvpProp123:] I offered him a drink
\item [* RstartvpProp132:] I envied her her house
\end{description}

\section {Adjuncts}
\begin{description}
\item[* Radjunctvar1:]\mbox{}
\begin{description}
\item (ResAP): He painted his face green
\item (ResNP): ??
\item (ResPP): He cut the onion into small pieces
\item (Locadjunct): The child hid behind the door
\item (Diradjunct): She blew the dust off the table/ He hurried into the house
\end{description}
\item[* Radjunctvar2:] He cooked me a meal
\item[* Radjunctvar3:] He cooked a meal for me
\item[* Radjunctvar4:]\mbox{}
\begin{description}
\item (SubjComit): ??
\item (ObjComit): ??
\end{description}
\end{description}

\section {Verbpatterns}
\subsection {TVerbpattern0}
\begin{description}
\item[* relevant examples:]\mbox{}
\begin{description}
\item (vp000): it rains
\item (vp100): x1 sleeps: He sleeps
\end{description}
\end{description}

\subsection{TVerbpattern1}
\begin{description}
\item[* relevant examples:] \mbox{}
\begin{description}
\item [subrule 1a/b] NP/CN\\
\ \ (vp010): melt x1: The ice melts\\
\ \ (vp120): x1 beat x2: Mary beats John\\
\ \ \ \ \  : x1 regret x2: I regret his coming home late\\
\item [subrule 2] EMPTY\\
\ \ (vp010): ? \\
\ \ (vp120): x1 eats x2: John eats EMPTY
\item [subrule 3a] CLOSEDNPPROP\\
\ \ (vp010): become x1: She became a poet\\
\ \ (vp120): x1 consider x2: They considered him a poet
\item [subrule 3b] OPENNPPROP\\
\ \ (vp120): x1 turn x2:The party turned out a success 
\item [subrule 4a] CLOSEDADJPPROP\\
\ \ (vp010): remain x1: He remained calm\\
\ \ (vp120): x1 consider x2: I consider him foolish
\item [subrule 4b] OPENADJPPROP\\
\ \ (vp120): ?x1 smell x2: This book smells old
\item [subrule 5] MEASUREPHRASE\\
\ \ (vp120): x1 weigh x2: This weighs a stone
\item [subrule 6] ASIFSENT\\
\ \ (vp010): look x1: It looks as if she is swimming\\
\ \ (vp120): x1 act x2: He acted as if he had gone mad
\item [subrule 7a1] LOCCLOSEDPREPPPROP\\
\ \ (vp010): remain x1: He remained in the garden\\
\ \ (vp120): x1 wish x2: I wished her on the moon\\
\item [subrule 7a2] LOCCLOSED'ADvpPROP'\\
\ \ (vp010): remain x1: He remained there\\
\ \ (vp120): x1 get x2: I got them ?home
\item [subrule 7b1] LOCOPENPREPPPROP\\
\ \ (vp120): x1 stand x2: He stood behind the door\\
\item [subrule 7b2] LOCOPEN'ADvpPROP' \\
\ \ (vp120): x1 live x2: Where does he live ?
\item [subrule 8a1] DIRCLOSEDPREPPPROP\\
\ \ (vp010): ?\\
\ \ (vp120): x1 scratch x2: He scratched the paint off the wall
\item [subrule 8a2] DIRCLOSED'ADvpPROP' \\
\ \ (vp010): x1 go x2: He went there\\
\ \ (vp120): x1 go x2: He drove her there\\
\item [subrule 8b1] DIROPENPREPPPROP\\
\ \ (vp120): x1 float x2: He floated into a tree 
\item [subrule 8b2] DIROPEN'ADvpPROP' \\
\ \ (vp120): x1 swim x2: He swam southwards
\item [subrule 9a1] OTHERCLOSEDPREPPPROP\\
\ \ (vp010): seem x1: He remained against the proposal\\
\ \ (vp120): x1 squeeze x2: She squeezed toothpaste out of a tube\\
\item [subrule 9a2] OTHERCLOSED'ADvpPROP'\\
\ \ (vp010): ?\\
\ \ (vp120): ?
\item [subrule 9b1] OTHEROPENPREPPPROP\\
\ \ (vp120): He declared himself against the plan
\item [subrule 9b1] OTHEROPEN'ADvpPROP'\\
\ \ (vp120): ?
\item [subrule 10a] CLOSEDVERBPPROP\\
\ \ (vp010): get x1: He got hit\\
\ \ (vp120): x1 have x2: He had a house built
\item [subrule 10b] OPENVERBPPROP\\
\ \ (vp120): ?
\item [subrule 11a] CLOSEDINFSENT\\
\ \ (vp010): can x1: He can come; He can be reached\\
\ \ (vp120): x1 see x2: I saw the man leave; I saw him be hit
\item [subrule 11b] OPENINFSENT\\
\ \ (vp120): x1 can x2: He can come
\item [subrule 12a] CLOSEDTOSENT\\
\ \ (vp010): appear x1: He appears to be your friend\\
\ \ (vp120): x1 expect x2: I consider him to be a fool
\item [subrule 12b] OPENTOSENT\\
\ \ (vp120): x1 decide x2: We decided to leave
\item [subrule 13] FORTOSENT\\
\ \ (vp120): x1 prefer x2: I prefer for John to go
\item [subrule 14] THATSENT\\
\ \ (vp010): seems x1: It seems that Margreet is coming\\
\ \ (vp120): x1 know x2: I know that Rosetta can't translate everything
\item [subrule 15] QSENT\\
\ \ (vp010): ?\\
\ \ (vp120): x1 wonder x2: I wonder where to go; I wonder where he lives
\item [subrule 16a] CLOSEDGERUND\\
\ \ (vp010): ?\\
\ \ (vp120): x1 watch x2: He watched the sun setting
\item [subrule 16b] OPENGERUND\\
\ \ (vp120): x1 confess x2: I confess hating the king
\item [subrule 17] PROSENT\\
\ \ (vp010): ?\\
\ \ (vp120): x1 wonder x2: I wonder
\item [subrule 18] SOPROSENT\\
\ \ (vp010): ?\\
\ \ (vp120): x1 think x2: I think so\\
\end{description}
\end{description}

\subsection{Tverbpattern2}
\begin{description}
\item[* relevant examples:] I cannot believe it that I've won the lottery\
\end{description}

\subsection{TVerbpattern3}
\begin{description}
\item[* relevant examples:] \mbox{}
\begin{description}
\item [subrule 1a/b] PREPNP/CN\\
\ \ (vp120): x1 count (up)on x2: I count (up)on John/ his finishing the 
job
\item [subrule 2a] PREPCLOSEDNPPROP\\
\ \ (vp120): ? (originally for `regard as')
\item [subrule 2b1] PREPOPENNPPROP\\
\ \ (vp120): x1 serve as x2: This room serves as a bathroom
\item [subrule 2b2] PREPMEASUREPHRASE\\
\ \ (vp120): x1 amount to x2: His debts amount to \$ 5000
\item [subrule 3a] PREPCLOSEDGERUND\\
\ \ (vp120): x1 count on x2: You can't count on the weather being fine
\item [subrule 3b] PREPOPENGERUND\\
\ \ (vp120): x1 succeed x2: She succeeded in identifying the murderer
\item [subrule 4] PREPCLOSEDTOSENT\\
\ \ (vp120): x1 count on x2: I count on you to help
\item [subrule 5] PREPCLOSEDADJPPROP\\
\ \ (vp120): ? (originally for `regard as')
\item [subrule 6] PREPQSENT\\
\ \ (vp120): x1 talk about x2: We all talked about how to prevent an accident 
(from) happening; - why he would have murdered her; - whether we should all 
leave early\\
\ \ (vp120): x1 hesitate about x2: I'm hesitating about what to do next\\
\ \ (vp120): x1 hesitate over x2: I'm hesitating over whether to join the 
expedition
\item [subrule 7] PREPOTHERCLOSEDPREPPPROP\\
\ \ (vp120): ? (originally for `regard as'
\item [subrule 8] PREPTHATSENT\\
\ \ (vp120): x1 count on x2: I count on it that you'll be there\\
\end{description}
\end{description}


\subsection {TVerbpattern4}
\begin{description}
\item[* relevant examples:] \mbox{}
\begin{description}
\item[subrule 1a/d] IONP\_DONP\\
  (vp123): x1 give x2 x3: I gave him the book.\\
  (vp123): x1 tell x2 x3: I told him a lie .\\
  (vp123): x1 charge x2 x3: How much do you charge me (for that)?\\
\item[subrule 2a/b] EMPTY\_DONP \\
  (vp123): ?\\
\item[subrule 3a/b] IONP\_THATSENT\\
  (vp123): x1 tell x2 x3: He told me that he would be back early.
\item[subrule 4] EMPTY\_THATSENT\\
  (vp123): x1 promise EMPTY x3: We promised that we would be home early.
\item[subrule 5a/b] IONP\_SOPROSENT\\
  (vp123): x1 tell x2 x3: I told him so.
\item[subrule 6a/b] IONP\_QSENT\\
  (vp123): x1 tell x2 x3: Tell me where to go; I asked him if he wanted tea.
\item[subrule 7a/b] IONP\_OPENTOSENT\\
  (vp123): x1 promise x2 x3: John promised Bill to see a doctor\\
  (vp123): x1 persuade x2 x3: John persuaded Bill to see a doctor.\\
\item[subrule 8] EMPTY\_OPENTOSENT\\
  (vp123): x1 promise x2 x3: I promised EMPTY to come.
\item[subrule 9a/b] IONP\_MEASUREPHRASE\\
  (vp123): x1 cost x2 x3: This cost me \$6; It cost him his head (?!\\
  (vp123): x1 charge x2 x3: How much do you charge me for that?\\
\item[subrule 10] EMPTY\_MEASUREPHRASE\\
  (vp123): x1 cost x2 x3: This costs EMPTY \$6. \\
\item[subrule 11] EMPTY\_QSENT\\
  (vp123): x1 ask x2 x3: I asked EMPTY why he wanted to leave; - if he wanted 
tea.
\item[subrule 12a/b] IONP\_PROSENT\\
  (vp123): x1 tell x2 x3: I told him.
\item[subrule 13] EMPTY\_PROSENT\\
  (vp123): x1 ask x2 x3: I'll ask.
\item[subrule 14] EMPTY\_OPENGERUND\\
  (vp123): x1 allow x2 x3: I don't allow smoking here.\\
\end{description}
\end{description}

\subsection{TVerbpattern5}
\begin{description}
\item[* relevant examples:] \mbox{}
\begin{description}
\item[subrule 1a/b] DONP\_OPENTOSENT\\
  (vp123): x1 force x2 x3: We forced them to cooperate\\
  (vp123): x1 name x2 x3: The President named her to be a nurse(Longman)
\item[subrule 2a/b] DONP\_OPENNPPROP\\
  (vp123): x1 call x2 x3: They called him a liar 
\item[subrule 3a/b] DONP\_OPENADJPPROP\\
  (vp123): x1 paint x2 x3: He paints the door green\\
\item[subrule 4a/b1] DONP\_LOCOPENPREPPPROP\\
  (vp123): x1 put x2 on x3: I put the book on the table\\
\item[subrule 4a/b2] DONP\_LOCOPEN'ADvpPROP'\\
  (vp012): ??meet x1 x2: The cars met head-on
\item[subrule 5a/b1] DONP\_DIROPENPREPPPROP\\
  (vp123): x1 drive x2 x3: He drove the car into the garage
\item[subrule 5a/b2] DONP\_DIROPEN'ADvpPROP'\\
  (vp123): ?x1 send x2 x3: I sent the boy home
\item[subrule 6a/b] DONP\_EMPTY\\
  (vp123): x1 tell x2 to EMPTY: He told the whole story\\
\item[subrule 7] CLOSEDADJPPROP\_EMPTY\\
  (vp012): seem x1 to EMPTY: He seemed ill\\
\item[subrule 8] CLOSEDNPPROP\_EMPTY\\
  (vp012): seem x1 to EMPTY: He seemed a fool 
\item[subrule 9] OTHERCLOSEDPREPPPROP\_EMPTY\\
  (vp012): seem x1 to EMPTY: He seemed against the proposal 
\item[subrule 10a/b] DONP\_OTHEROPENPREPPPROP\\
  (vp123): x1 tear x2 x3: He tore the letter to pieces
\item[subrule 11] THATSENT\_LOCOPENPREPPPROP\\
  (vp012): say x1 x2: It said in the papers that Gorbachev was killed.
\item[subrule 12a/b:]DONP\_OPENGERUND\\
  (vp123): x1 catch x2 x3: I caught him lying\\
\item[subrule 13a/b:]DONP\_PROSENT\\
  (vp123): x1 force x2 PROSENT: They forced me
\item[subrule 14a/b:]DONP\_QSENT\\
  (vp123): x1 instruct x2 x3: They instructed Reagan how to act like a 
president
\item[subrule 15a/b:]DONP\_THATSENT\\
  (vp123): x1 convince x2 x3: They convinced her that she could not afford it\\
\item[subrule 16a/b:]DONP\_OPENINFSENT\\
  (vp123): x1 help x2 x3: I helped her clean the windows\\
\end{description}
\end{description}

\subsection{TVerbpattern6}
\begin{description}
\item[* relevant examples:] \mbox{}
\begin{description}
\item[subrule 1a/b:] PREPNP\_THATSENT\\
  (vp012): seem to x1 x2: It seems to me that he is ill\\
  (vp123): x1 mention to x2 x3: He mentioned to me that he would be back early
\item[subrule 2a/b:] PREPNP\_QSENT\\
  (vp123): x1 inquire of x2 x3: I inquired of him what he wanted
\item[subrule 3a/b:] PREPNP\_OPENTOSENT\\
  (vp123): x1 sign to/?for x2 x3: The policeman signed to me to stop\\
\end{description}
\end{description}

\subsection{TVerbpattern7}
\begin{description}
\item[* relevant examples:] \mbox{}
\begin{description}
\item[subrule 1a/d:] PREPNP\_PREPNP\\
  (vp123): x1 talk to/with x2 about x3: I talked to/with my father about a 
  new car / about his financing our next car\\
  (vp123): x1 talk with x2 about x3: I talked with my 
  friends about our new house  
\item[subrule 2:] PREPNP\_PREPOPENGERUND\\
  (vp123): x1 agree with x2: up)on x3 (They agreed with him (up)on selling the 
  house\\
\end{description}
\end{description}

\subsection{TVerbpattern8}
\begin{description}
\item[* relevant examples:] \mbox{}
\begin{description}
\item[subrule 1a/b:] DONP\_PREPOPENNPPROP\\
  (vp012): turn x1 into x2: He turned into a frog \\
  (vp123): x1 name x2 as x3: The President named him as Secretary of State\\
  (vp123): x1 regard x2 as x3: They regard her as a friend\\
  (vp123): x1 turn x2 into x3: We turned our back yard into a workshop\\
  (vp123): x1 mistake x2 for x3: I mistook you for somebody else\\
  (vp123): ?x1 name x2 after x3: They named him after his father\\
  (vp123): ?x1 charge x2 on x3: They charge a tax on imported bottles of wine\\
  (vp123): x1 allow x2 on x3: The bank allows 5 per cent (interest) on money 
kept with it\\
  (vp123): ?x1 ask x2 for x3: How much are you asking(/do you want) for that 
  painting?\\
\item[subrule 2a/b:] DONP\_PREPOPENADJPPROP\\
  (vp123): x1 change x2: from x3a) to x3b: We changed the main colour from 
brown to red\\
  (vp123): x1 regard x2 as x3: We regard her as clever\\
\item[subrule 3a/d:] DONP\_PREPNP\\
  (vp123): x1 give x2 to x3: I gave the book to John
\item[subrule 4a/b:] DONP\_PREPOPENGERUND (now considered NPs!\\
  (vp123): x1 talk x2 out of x3: They talked him out of jumping from the 
  top of the building   \\
  (vp123): x1 keep x2 from x3: Can't you keep him from forgetting?\\
  (vp123): x1 regard x2 as x3: I regard him as being without principles
\item[subrule 5a/b:] CLOSEDADJPPROP\_PREPNP\\
  (vp012): seem x1 to x2: He seemed ill to me
\item[subrule 6a/b:] CLOSEDNPPPROP\_PREPNP\\
  (vp012): seem x1 to x2: He seemed a fool to me
\item[subrule 7a/b:] OTHERCLOSEDPREPPPPROP\_PREPNP\\
  (vp012): seem x1 to x2: He seemed against the analysis to me
\item[subrule 8a/b:] DONP\_PREPOTHEROPENPREPPPROP\\
  (vp123): x1 regard x2 as x3: I regard him as without principles
\item[subrule 9a/b:] DONP\_PREPQSENT\\
  (vp123): x1 instruct x2 in x3: They instructed him in how to do it
\item[subrule 10a/b:] EMPTY\_PREPNP\\
  (vp123): x1 refer (EMPTY) to x3: We refer to the manager \\
\item[subrule 11:] EMPTY\_PREPOPENGERUND\\
  (vp123): x1 advise (EMPTY) against x3: I strongly advise against eating 
British beef.
\item[subrule 12a/b:]OPENGERUND\_PREPNP\\
  (vp123): x1 leave x2 to x3: I'll leave cleaning to you\\
\end{description}
\end{description}

\subsection{TVerbpattern9}
\begin{description}
\item[* relevant examples:] \mbox{}
\begin{description}
 \item[subrule 1a/b:] IONP\_PREPCLOSEDADJPPROP\\
  (vp012): ?strike x1 as x2: He struck me as pompous\\
  (vp123): x1 ask/tell x2 about x3: Why don't you ask/tell him about it?\\
  (vp123): x1 provide x2 with x3: I provided them with the relevant documents\\
\end{description}
\end{description}

\subsection{TVerbpattern10}
\begin{description}
\item[* relevant examples:] \mbox{}
\begin{description}
 \item[subrule 1a/b:] EMPTY\_PREP2NP\\
  (vp123): x1 talk (to EMPTY) about x2: We talked about the accident\\
\end{description}
\end{description}

\subsection{TVerbpattern11}
\begin{description}
\item[* relevant examples:] \mbox{}
\begin{description}
\item[subrule 1a/b:]PROSENT\_PREPNP\\
(vp123): x1 say SO to x3: I said so to my father
\item[subrule 2:]PROSENT\_EMPTY\\
(vp123): x1 say SO (to EMPTY): I said so
\end{description}
\end{description}

\subsection{TVerbpattern12}
\begin{description}
\item[* relevant examples:] x1 leave IT x2 to x3 (I will leave it [to clean the 
windows] to you\\
\end{description}

\subsection{Tverbpattern13}
\begin{description}
\item[* relevant examples:] \mbox{}
\begin{description}
\item[subrule 4a/b:] PREPNP\_EMPTY\\
  (vp123): x1 talk to x3 about EMPTY: He talked to his father\\
\end{description}
\end{description}

\subsection{TIdVerbpattern}
\begin{description}
\item[* relevant examples:] Every idiom\\
\end{description}


\section {Particles}
\begin{description}
\item [* TNoparthop/Toptparthop:] He gave up his ambitions 
(A:He gave his ambitions up)\\
\item [* TOblparthop:] He gave it up\\
\end{description}

\section {Voices}
\begin{description}
\item [* Ractive/RActClauseFormation/Rindicmoodmain:] He loves her
\item [* Rpassive1/RpasClauseFormation/Rindicmoodmain:] He was loved; He was 
being loved
\item [* Rpassive2/RpasClauseFormation/Rindicmoodmain:] I was surprised by what 
they wanted; I was struck by lightning
\end{description}
\section {Inherent Reflexives}
\begin{description}
\item [* Tobjreflinsertion1:] He declared himself against the plan
\item [* Tobjreflinsertion2:] The man who declared himself against the plan
\item [* Tindobjreflinsertion1:] We had ourselves a good time
\item [* Tindobjreflinsertion21:] The peole who had themselves a good time
\end{description}
\section {Argument Reflexives}
\begin{description}
\item [* Targreflspelling1:] I hate myself, I gave myself a book
\item [* Targreflspelling2:] The man who hates himself, who gave himself a book
\item [* Targreflspelling3:] I count on myself, I gave a book to myself
\item [* Targreflspelling4:] The man who counts on himself, who gave a book to 
himself
\item [* Targreflspelling5:] I showed Mary herself (cf Larson LI 22.4, p. 121)
\item [* Targreflspelling6:] The man who showed Mary herself
\item [* Targreflspelling7:] ?I gave a book to itself
\item [* Targreflspelling8:] ?The man who gave a book to itself
\item [* Targreflspelling9:] I put the vase next to me
\item [* Targreflspelling10:] The man who put the vase next to him
\end{description}
\section {Reciprocals}
\begin{description}
\item [* ?:] They like each other
\item [* ?:] They gave each other a long look
\end{description}
\section {Propsubst}
\begin{description}
\item[* Rmodalcomplsentsubst1/2:] He can see it
\item[* RsubjNPsubst:] His seeing it surprised me
\item[* RClosedAdjPPropsubst/Tnocontroladjp:] She turned out old
\item[* RLocopeNPrepPPropsubst/Toblsubjcontrolprepp:] The tooth lay on the 
photocopy
\item[* Rotherclosedprepppropsubst/Tnocontroladvp:] She got the ox versus the 
music
\item[* RExtrasubjsefntsubst/Tnocontrol/Titsubjinsertion:] It surprised me that 
she left
\item[* Rldislocsubjsentsubst/Tnocontrol/Tthatdeletion:]That he left surprised 
me
\end{description}
\section {ClauseFormation}
\begin{description}
\item[* RadjpClauseFormation:] He is ill\\
\item[* RNPClauseFormation:] He is a doctor\\
\item[* RpreppClauseFormation:] The church is against the proposal
\item[* Rstartexist/RexistNPClauseFormation:] Is there a doctor
\item[* Rdemproidentpl/RidentNPClauseFormation:] There are cacti
\end{description}
\section {Extraposition}
\begin{description}
\item[* Textraposition1:] He thougt yesterday that he had lost the match
\item[* Textraposition2:] He counted on it yesterday that he would swim
\end{description}
\section {Obligatory Control}
\begin{description}
\item[* RComplsentsubst/ToblsubjControlcomplsent:] He promised him to come
\item[* RComplsentsubst/Toblobjcontrolcomplsent:] He persuaded him to come
\item[* RExtrasubjsentsubst/Textraposition1/Toblbyobjcontrolcomplsent:] It was 
decided by the actress to leave
\item[* Robjnpsentsubst/Toblsubjcontrolopeningnp:] He burst out singing
\end{description}
\section {Non-obligatory Control}
\section {VPadvs}
\section {Emptys}
\begin{description}
\item[* Rbyemptysubst:] He was hit
\item[* RPrepemptysubst:] They say so
\item[* Robjemptysubst:] Jan is eating
\end{description}
\section {Subjok}
\begin{description}
\item[* Titsubjinsertion:] It surprised me that he saw it
\item[* TTheresubjinsertion:] There flew a bird through the sky
\item[* TPrepobjtosubjraising:] The man was being looked at
\item[* Tobjtosubjraising:] The volcano melts
\item[* Tsubjok:] It is raining
\end{description}
\section {Case}
\begin{description}
\item[* Texceptcaseassign:] She considers them thieves
\end{description}
\section {Shift}
What did you buy\\
Who bought a book\\
What did he insist on\\
On what did he insist\\
What did you connect him to\\
To what did you connect him\\
What did he conclude with\\
With what did he conclude\\
What did he talk about\\
About what did he talk\\
Who do you think that I knew\\
Who do you think I knew\\
Who do you think that left\\
Who do you think left\\
Who do you think we talked about\\
I wonder what to do\\
The man that left \\
The book I bought\\
The book which I bought\\
The girl to marry\\
The girl whom to marry\\
\section {Negation}
\begin{description}
\item [*Rsentnegvar/Rsentnegsubst:] Has he not come
\item [*Rsentposvar/RPossubst/Tnododeletion:] He did come
\item [*Rsentnegvar/Rnegsubst/Tdodeletion:] You need not come
\item [*Rsentmeltnegvar/Rsentmeltnegsubst/tnegadaptation:] He never came
\item [*Rsentmeltnegvar/rsentmeltnegsubst/tnegadaptation:] He saw nobody
\end{description}
\section {Prosent}
\begin{description}
\item [*Rprosentsubst:] She wonders
\item [*Rsoprosentsubst:] He thinks so
\item [*Rnotprosentsubst:] He thinks not
\end{description}
\section {Mood}
\begin{description}
\item [*Rindicmoodmain:] He is swimming
\item [*Rindicyesnomoodmain:] Is he swimming
\item [*Rimpmood:] Swim
\item [*Ropentoinfmood:] She wants to swim
\item [*Rfortoinfmood:] It is unusual for him to swim
\item [*Ranterelingmood:] All those still swimming must go home
\end{description}
\section {Tense and Aspect}
I swim\\
I am swimming\\
I have swum\\
I have been swimming\\
I swam\\
I was swimming\\
I had swum\\
I had been swimming\\
I will swim\\
I will be swimming\\
I will have swum\\
I will have been swimming\\
\section {Conjsent}
\begin{description}
\item [*Rprepingsubsent:] Without swimming he went home
\item [*Rconjfinsubsent:] She went home because he swam
\end{description}
\section {vp deletion}
\begin{description}
\item [*TVPdeletion:] He bought a book
\end{description}
\end{document}
