
\documentstyle{Rosetta}
\begin{document}
   \RosTopic{Rosetta3.doc.linguistics.dutch}
   \RosTitle{Dutch M-rules:subgrammar NPFormation}
   \RosAuthor{Franciska de Jong, Lisette Appelo ??}
   \RosDocNr{410}
   \RosDate{\today}
   \RosStatus{concept}
   \RosSupersedes{-}
   \RosDistribution{Project}
   \RosClearance{Project}
   \RosKeywords{Dutch, M-rules, NPFormation}
   \MakeRosTitle
%
%
\input{[dejong.mrules]mrudocdef}

\section{Introduction}
In this document the subgrammar NPformation for Dutch is described.
In general the task of this subgrammar 
is to handle the phenomena involved in the proces of NPformation
that presume the application of either subgrammar CNformation (for NPs with a 
an intermediate CN-level, that is: NPs with a full NOUN 
or EN head) or the subgrammar NOUNderivation 
(for pronominal NPs and PROPERNOUNs without 
inflection). (Note that the names of the subgrammars mentioned here are rather
infelicitous, though understandable in view of the history of the subgrammars.)

The subgrammar NPformation is relevant to {\bf all} 
NPs. 
that are headed by CN. (This holds for NPs containing either a full NOUN or a 
EN (empty head).)
The subgrammar is not supposed to be isomorphic to any other subgrammar 
of Dutch. 
The parts on.. and ... have been written by 
Lisette Appelo, the other parts have been written by Franciska de Jong.


\section{Subgrammar Specification}

\begin{description}
  \item[Head] CN, PERSPRO, INDEFPRO, DEMPRO, WHPRO, PROPERNOUN, RECIPRO,
  \item[Export] NP
  \item[Import] DETP, NP, ADVP, SENTENCE      
  \item[File] dutch:npsubgrammars.mrule (mrules67)
\end{description}

\section{Control Expression}
The control expression can be defined as follows:
\begin{verbatim}
\end{verbatim}
For testing purposes some of the rules are not activated in the present 
system. 

\section{Rules and Transformations}
\end{document}
