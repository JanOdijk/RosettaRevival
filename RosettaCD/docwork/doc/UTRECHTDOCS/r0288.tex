\documentstyle{Rosetta}
\begin{document}
   \RosTopic{General}
   \RosTitle{Notulen Linguistenvergaderingen 20-10-88 en 3-11-88}
   \RosAuthor{Margreet Sanders}
   \RosDocNr{R0288}
   \RosDate{November 24, 1988}
   \RosStatus{approved}
   \RosSupersedes{-}
   \RosDistribution{Linguists, Joep Rous}
   \RosClearance{Project}
   \RosKeywords{minutes, adverbia, scope, superdeixis, modalen, BIGPRO, 
                controle}
   \MakeRosTitle
%
%
\begin{description}
\item[Aanwezig:] Lisette Appelo, Franciska de Jong, Elly van Munster, Jan Odijk,
                 Margreet Sanders (not), Andr\'{e} Schenk, Harm Smit
\item[Agenda 20-10-88:]\mbox{}
  \begin{enumerate}
  \item Testzinnen
  \item Scope en Adverbia
  \item Notulen
  \item Superdeixis
  \end{enumerate}
\item[Agenda 3-11-88:]\mbox{}
  \begin{enumerate}
  \setcounter{enumi}{4}
  \item Modalen
  \item Controle van BIGPROs en open Infinitieven
  \end{enumerate}
\end{description}

\section{Testzinnen}
De ingeleverde testzinnen en -constructies blijken voor twee verschillende 
doelen te 
zijn geschreven: testen van bepaalde regels, en leuke zinnen voor 
demonstraties. Er wordt besloten om eerst een testzinnenbank te maken
met simpele zinnen die een bepaald verschijnsel testen. Iedereen wordt 
daarom gevraagd om het ingeleverde lijstje met zinnen zo aan te passen dat er 
een zin staat, + een indicatie van wat die zin test, + de vertaling. Ook 
voor de morfologie van de drie talen moet er een lijstje met interessante 
woorden komen (Harm),
en Andr\'{e} zal een lijstje met verschillende idiomen en semi-idiomen maken.
Jan O.\ en Margreet zullen deze werkzaamheden co\"{o}rdineren, zodat op 1 
november een eerste versie klaar is.

Als het hele systeem werkt, en er van elk van deze zinnen vaststaat hoeveel
en welke vertalingen eruit moeten komen, kunnen ze ook gebruikt worden
voor geautomatiseerd testen (d.w.z. checken of na een verandering nog steeds 
dezelfde vertalingen mogelijk zijn).

Daarna kan er op basis van de testzinnen een soort demonstratiezinnen-bank 
worden opgesteld, met leuke zinnen waarin verschillende verschijnselen worden 
gecombineerd.

\section{Scope en Adverbia}
\subsection{AgVPAdv's}
AgVpAdv's zijn adverbia die 2-plaatsig zijn, d.w.z.\ die restricties kunnen 
uitoefenen op de agens van de zin en de VP modificeren. Aangezien we in Rosetta
de VP geen semantische status willen geven (anders is vertaling van ergatieven
(met hun eerste argument in de VP) naar niet-ergatieven (met hun eerste 
argument buiten de VP in S) niet goed mogelijk) modificeren deze AgVPAdv's 
in feite 
hele zinnen. Voorbeelden van dergelijke AgVPAdv's zijn {\em x1 enthousiast 
x2\/}, waarbij {\em x1\/} de agens is die moet verdwijnen op basis van 
identiteit met de agens in {\em clause x2\/}, en {\em x1 opzettelijk x2\/}.
AgVPAdv's worden beschouwd als ADVPPROPs.

Merk op dat AgVPAdv's zich anders gedragen dan adverbia zoals {\em graag\/},
die niet agens- maar subject-geori\"{e}nteerd zijn en dus een echte {\em zin\/} 
als argument nemen, en niet een clause (een zin als {\em Er werd 
graag gedanst\/} is ongrammaticaal omdat er geen overt subject is).
Hier mag de introductie van de ADVPPROP dus niet gebeuren v\'{o}\'{o}r 
EMPTY-substitutie en ObjectOK-regels.

Als een AgVPAdv voorkomt in een passieve zin met een lege by-phrase ({\em Er 
werd enthousiast geapplaudisseerd - door EMPTY - \/})
ontstaat er een probleem:
de ADVPPROP moet geintroduceerd worden  v\'{o}\'{o}rdat de EMPTY is 
gesubstitueerd en verdwenen (anders kan er geen restrictie op worden gelegd), 
maar n\'{a}dat POSVAR/NEGVAR is geintroduceerd, omdat anders het verschil 
in scope tussen {\em Het werd niet opzettelijk gedaan \/} en {\em Het werd
opzettelijk niet gedaan\/} niet verantwoord kan worden. Aangezien EMPTY-substitutie 
al in de XPPROPtoCLAUSE subgrammatica zit, en POSVAR/NEGVAR-introductie pas in 
de volgende subgrammatica (CLAUSEtoSENTENCE), is er dus een 
ordeningsprobleem. 

Verder is er ook onafhankelijk hiervan een probleem met de scope van de 
AgVPAdvs, omdat ze niet door een variabele worden geintroduceerd. Dit is
hetzelfde probleem als bestond voor POS en NEG (zie notulen 31-5-88): een zin
waarin {\em .. niet opzettelijk\/} voorkomt, kan in analyse zowel eerst 
desubstitutie hebben van {\em niet\/} (correct) als van {\em opzettelijk\/}, 
aangezien er in het laatste geval toch geen VAR voor in de plaats komt en de 
substorder conditie dus niet wordt geschonden.

Een oplossing van dit probleem wordt niet gevonden. De mogelijke restricties op 
het subject verhinderen dat AgVPAdv's als gewone 
ADVs worden behandeld en met een ADVP(VAR) werken, en het vervroegen van 
de EMPTY substitutie lijkt ook wat drastisch. Als niemand iets bedenkt voor 
eind oktober, zal Jan O.\ regels implementeren die AgVPAdv's vroeg (d.w.z.\ 
v\'{o}\'{o}r EMPTY-substitutie) introduceren en dus goed werken voor 
{\em enthousiast\/}, dat niet na NEGVAR kan optreden (Q-status = false), 
maar die {\em opzettelijk\/} (met Q-status - true) niet aankunnen (tenzij dat 
ten onrechte als SentAdv wordt gekarakteriseerd).

\subsection{SentAdv's}
SentAdv's (bv.\ waarschijnlijk, kennelijk, hopelijk) vertonen  hetzelfde 
scope-probleem als de AgVPAdv's, omdat ze via een ADVPPROP worden 
geintroduceerd, zonder een VAR. Verder bestaan er duidelijke restricties op 
het voorkomen van SentAdv's met een negatie: de negatie moet altijd volgen op
het adverbium ($^{*}$Hij zal niet hopelijk te laat zijn), en in combinatie met 
het subject {\em Niemand\/} moeten ze verplicht aan het begin van de zin staan 
($^{*}$Niemand is waarschijnlijk gekomen).

Er wordt besloten in dit geval aan te nemen dat SentAdv's niet meer een 
ADVPPROP zijn maar een ADVP, en dat deze ADVP al vroeg via een ADVPVAR wordt
geintroduceerd. Jan O.\ zal dit voor het Nederlands implementeren. 

Aangezien het probleem van scope en ADVP(PROP/VAR)s steeds weer terugkomt, 
pleit Lisette ervoor om eens een goede inventarisatie te maken de argumenten 
om adverbia op een bepaalde manier te behandelen, in samenhang met EMPTYs, 
modificatie van VPs, scope etc. Hieruit kan hopelijk voor Rosetta4 lering 
worden getrokken!

\section{Notulen}
Naar aanleiding van opmerkingen over de notulen van 11 oktober stelt Margreet 
voor om ook de notulen van de linguistenvergadering voortaan eerst als concept te 
laten verschijnen, en na goedkeuring ervan een approved versie te maken. Dit
voorstel wordt aanvaard. De notulen van 11 oktober zullen meteen als approved 
versie worden herschreven.

\section{Superdeixis}
Lisette heeft een alternatieve oplossing voor de problemen die bestaan met 
superdeixis van de ADJP-modificatoren. In de ADJPPROPtoFORMULA subgrammatica 
moeten SuperdeixisAdaptation transformaties komen, zodat van modificatoren 
van een ADJ (np, qp, advp en prepp, al dan niet met een ingebedde zin: 
te {\em mooi\/} om waar 
te zijn, {\em blij\/} met wandelende ezels) gecheckt wordt of de superdeixis 
klopt, en deze daarna op omega gezet wort, zodat in de ADVP, NP, QP en PREPP 
grammatica's zelf nooit meer superdeixis nodig is.

Verder wil Lisette de superdeixis-regels van de ADVP en PREPP subgrammtica's 
optioneel maken, net als in de NP-superdeixis (dat laatste was nodig voor 
idiomen). Het is niet duidelijk of hiermee niet extra paden worden 
gecre\"{e}erd. Lisette zal in elk geval de SuperdeixisAdaptation-operatie 
uitvoeren, en zich beramen op verdere optionele regels.

\section{Modalen}
Er zijn twee problem gebleken bij de implementatie van het modalen-voorstel van 
6 oktober. Ten eerste de mapping van {\bf toegestaan} en {\bf allowed}. In de zin 
{\em He is allowed to smoke\/} (waar {\em allowed\/} in elk geval een ADJ moet 
zijn en geen participium, want dan zou het drie argumenten nemen en dus nooit 
een vertaling van {\em mogen\/} kunnen zijn) treedt een andere volgorde op van 
argumenten dan in de Nederlandse vertaling {\em Het is hem toegestaan te roken
\/}, nl.\ allowed x1 x2 vs.\ toegestaan x2 x1. Dit is op twee manieren op te 
lossen:\\
1) {\em toegestaan\/} krijgt een apart adjpattern, en in een speciale patroon- 
regel worden de argumenten alsnog in de goede volgorde gezet, of\\
2) {\em toegestaan\/} krijgt een aparte waarde voor thetaadjp, nl.\ adjp210, en 
er wordt een speciale startregel voor geschreven. \\
De laatste oplossing lijkt theoretisch aantrekkelijker (er bestaat echt een 
andere argumentstructuur bij dit ADJ dan normaal), en zal worden 
geimplementeerd. Overigens is het {\em werkwoord\/} {\bf toestaan} een 3-
plaatsig werkwoord, dat op het eveneens 3-plaatsige {\bf allow} moet worden 
gemapt: {\em Wij stonden hen toe te roken - We allowed them to smoke\/}. Het 
lijkt dat deze werkwoorden geen passief toestaan: $^{*}$They were allowed to 
smoke by us; $^{*}$This was allowed them by us.

Het tweede probleem betreft de vertaling van het Spaanse {\bf deber}. Dit 
werkwoord mapt zowel op {\em moeten\/} (in niet-polaire context) als op {\em
hoeven\/} (in negatief-polaire context). Als er maar \'{e}\'{e}n {\em deber\/} 
bestaat dat op beide IL-expressies wordt gemapt, zal in een niet-polaire 
context de vertaling naar {\em hoeven\/} doodlopen. Een mogelijke oplossing is 
om de 
IL-expressie {\em hoeven\/} in generatie ook te mappen op het Nederlandse {\em
moeten\/}, maar dat lijkt semantisch niet aanvaardbaar. Daarom zal in het Spaans 
toch onderscheid gemaakt moeten worden tussen twee werkwoorden {\em deber\/}, 
waarvan de ene een negatief polaire context wil, en de andere geen eisen stelt.
De vertaling van een zin als {\em No debe hacerlo\/} wordt dan keurig zowel 
{\em Hij moet het niet doen\/} als {\em Hij hoeft het niet te doen\/}.

In het Engels lijkt voor het hoofdwerkwoord {\em need\/} een dergelijk probleem 
op te treden. Margreet zal dit nog nader bekijken.

\section{Controle van BIGPROs en open Infinitieven}
In een zin als {\em Er werd (door EMPTY1) (aan EMPTY2) voorgesteld om BIGPRO 
te vertrekken\/} moet de BIGPRO verdwijnen in de controleregels op 
basis van (parti\"{e}le) identiteit met een variabele in de hoofdzin. Op dit 
moment wordt echter nergens vastgelegd op basis van welke variabele dat nou 
was. Als er een werkwoord zou bestaan dat in een andere taal een andere 
variabele als controller zou nemen (het is de vraag of dit toegestaan is), 
wordt in de vertaling een verkeerde link gelegd. Om dit te voorkomen moet het 
record van een BIGPRO worden uitgebreid met een attribuut `index', waarin de 
index van de controlerende variabele kan worden opgeslagen, en moet deze index 
als parameter worden meegegeven aan de NPformatie-regel die van BIGPROs een NP 
maakt. Overigens gaan we ervan uit dat elk werkwoord maar \'{e}\'{e}n 
controller heeft. Jan O.\ heeft de meeste regels voor dit soort controle al 
geschreven.

Het geval dat het antecedent geen EMPTY is maar een `echte' variabele ("split
antecedents") is moeilijker, en wordt nog niet behandeld.

Voor `controle' van finiete zinnen (d.w.z.\ het vastleggen van de identiteit 
tussen een variabele 
uit de hoodfzin en een uit de bijzin: {\em Ik meen dit te moeten doen\/} vs.\ 
{\em I think that {\em I} have to do this\/}) moeten nog regels worden 
geschreven.\\[4ex]


In verband met de backup die voortaan elke donderdagochtend gemaakt wordt 
tussen  8 en 10 uur (en die tot nu toe altijd uitloopt) vergaderen we voortaan 
op donderdagochtend om 10 uur (na de donderdagochtend voordracht).

\end{document}
