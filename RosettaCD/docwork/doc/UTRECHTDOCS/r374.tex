
\documentstyle{Rosetta}
\begin{document}
   \RosTopic{General}
   \RosTitle{Adjpatterns of Dutch}
   \RosAuthor{Franciska de Jong}
   \RosDocNr{0374}
   \RosDate{\today}
   \RosStatus{concept}
   \RosSupersedes{-}
   \RosDistribution{Project}
   \RosClearance{Project}
   \RosKeywords{adjpatterns, Dutch, syntax}
   \MakeRosTitle
%
%


\section{Introduction}

This document gives an alphabetical 
section-wise list of all adjpatterns of Dutch.
Each adjpattern is the subject of a separate section. For each value of 
thetaadjp there is a separate subsection. The subsections supply 
additional information with the following headings:
\begin{description}
  \item [Rule] The name of the M-rule spelling out the adjpattern.  Sometimes 
the relevant subrules are indicated by .n (where n refers to the informal 
numbering convention in the M-rule files). These M-rules can be found in three 
files:
\begin{enumerate}
  \item dutch:tc\_adjpattern1.mrule (mrules33)
  \item dutch:tc\_adjpattern2.mrule (mrules34)
  \item dutch:rc\_startadjpprop.mrule (mrules32; infelicity of name due to recent 
extension)
\end{enumerate}

  \item [Canonical Surface Representations] An indication of how the pattern
        is realized in structures that are the output of M-generator, 
disregarding the effects of shiftrules etc.
     
        
  \item [Remarks] Any remark that seems appropriate

  \item [Examples Adjectives] 
  \item [Examples Sentences] 
\end{description}

\noindent 
For conventions in the naming of synpatterns, cf. doc 0240, section 2 (Jan O.).

\newpage
\section{noadjpargs}
  \subsection{adjp000}
\begin{description}
  \item [Rule] Tadjpattern0
  \item [Canonical Surface Representations] \mbox{}\\ head/ADJ
  \item [Remarks] No remarks
  \item [Example adjectives]\mbox{}
\begin{enumerate}
  \item koud
  \item laat
\end{enumerate}
  \item [Example sentences]\mbox{}
\begin{enumerate}
  \item het is koud
  \item het is later dan bedoeld
\end{enumerate}
\end{description}
\subsection{adjp100}
\begin{description}
  \item [Rule] Tadjpattern0
  \item [Canonical Surface Representations]\mbox{}\\ head/ADJ
  \item [Remarks] No remarks
  \item [Example adjectives]\mbox{}

\begin{enumerate}
  \item koud
  \item mooi
\end{enumerate}
  \item [Example sentences]\mbox{}\\
 
\begin{enumerate}
  \item de koffie is koud
  \item zij was mooier dan ooit
  \item dat hij niet komt is vervelend 
\end{enumerate}
\end{description}
\newpage
\section{synAANNP\_DONP}
\subsection{adjp123}
\begin{description}
  \item [Rule] TAdjpattern21
  \item [Canonical Surface Representations] 
 \mbox{}\\ aanobjrel/NP objrel/NP head/ADJ
  \item [Remarks] \mbox{}\\
\begin{enumerate}
  \item 
Ergativity is assumed 
in view of the surface order:
{\em ... dat aan hem het idee bekend is}.
  \item
In the case of ergative adjectives,
the direct object is moved into main clause subject-position if
there is no other element in shiftrel. 
\end{enumerate}

  \item [Example adjectives]\mbox{}\\
\begin{enumerate}
  \item bekend 
  \item duidelijk
\end{enumerate}
  \item [Example sentences]\mbox{}\\
\begin{enumerate}
  \item de uitkomst is aan ons bekend
  \item het idee  is aan ons niet duidelijk geworden
\end{enumerate}
\end{description}
\subsection{adjp123}
\begin{description}
  \item [Rule] TAdjpattern21 .1
  \item [Canonical Surface Representations]  \mbox{}\\
aanobjrel/.. objrel/.. head/ADJ
  \item [Remarks] No remarks

  \item [Example adjectives]\mbox{}\\
\begin{enumerate}
  \item schuldig
\end{enumerate}
  \item [Example sentences]\mbox{}\\
\begin{enumerate}
  \item Pim is aan mij een gulden schuldig
\end{enumerate}
\end{description}
\newpage
\section{synAANNP\_HETOPENOMTESENT}
\subsection{adjp123}
\begin{description}
  \item [Rule] TAdjpattern23
  \item [Canonical Surface Representations]  \mbox{}\\ aanobjrel/.. objrel/"het" 
complrel/SENTENCE head/ADJ
  \item [Remarks] No remarks

  \item [Example adjectives]\mbox{}\\
\begin{enumerate}
  \item verplicht
\end{enumerate}
  \item [Example sentences]\mbox{}\\
\begin{enumerate}
  \item Pim is het aan mij verplicht (om) te winnen
\end{enumerate}
\end{description}
\newpage
\section{synAANNP\_HETQSENT}
\subsection{adjp012}
\begin{description}
  \item [Rule] TAdjpattern25
  \item [Canonical Surface Representations]  \mbox{}\\ 
aanobjrel/NP objrel/"het" complrel/SENTENCE head/ADJ.
  \item [Remarks] \mbox{}\\
The direct object "het" is moved into subject-position.

  \item [Example adjectives]\mbox{}\\
\begin{enumerate}
  \item bekend 
  \item duidelijk
\end{enumerate}
  \item [Example sentences]\mbox{}\\
\begin{enumerate}
  \item Het is aan ons niet bekend of ...
  \item Het is aan ons niet duidelijk geworden wie ...
\end{enumerate}
\end{description}
\newpage
\section{synAANNP\_HETTHATSENT}
\subsection{adjp012}
\begin{description}
  \item [Rule] TAdjpattern25
  \item [Canonical Surface Representations]  \mbox{}\\ 
aanobjrel/NP objrel/"het" complrel/SENTENCE head/ADJ.
  \item [Remarks] \mbox{}\\
The direct object "het" is moved into subject-position.

  \item [Example adjectives]\mbox{}\
\begin{enumerate}
  \item bekend 
  \item duidelijk
\end{enumerate}
  \item [Example sentences]\mbox{}\\
\begin{enumerate}
  \item Het is aan ons bekend dat ...
  \item Het is aan ons  duidelijk geworden dat ...
\end{enumerate}
\end{description}
\newpage
\section{synAANNP\_OPENOMTESENT}
\subsection{adjp123}
\begin{description}
  \item [Rule] TAdjpattern21 .2
  \item [Canonical Surface Representations]  \mbox{}\\ aanobjrel/.. 
complrel/SENTENCE head/ADJ
  \item [Remarks] No remarks

  \item [Example adjectives]\mbox{}\\
\begin{enumerate}
  \item verplicht
\end{enumerate}
  \item [Example sentences]\mbox{}\\
\begin{enumerate}
  \item Pim is aan mij verplicht (om) te winnen
\end{enumerate}
\end{description}
\newpage
\section{synHETOPENOMTESENT}
\subsection{adjp120}
\begin{description}
  \item [Rule] TAdjpattern13
  \item [Canonical Surface Representations]  \mbox{}\\ 
objrel/"het" head/ADJ complrel/SENTENCE
  \item [Remarks] No remarks
  \item [Example adjectives] \mbox{}\\
\begin{enumerate}
  \item zat
\end{enumerate}
  \item [Example sentences] \mbox{}\\
\begin{enumerate}
  \item Pim is het zat (om) altijd te moeten koken
\end{enumerate}
\end{description}
\newpage
\section{synHETTHATSENT}
\subsection{adjp120}
\begin{description}
  \item [Rule] TAdjpattern13
  \item [Canonical Surface Representations]  \mbox{}\\ 
objrel/"het" head/ADJ complrel/SENTENCE
  \item [Remarks] No remarks
  \item [Example adjectives] \mbox{}\\
\begin{enumerate}
  \item zat
\end{enumerate}
  \item [Example sentences] \mbox{}\\
\begin{enumerate}
  \item Pim is het zat dat hij altijd moet koeken
\end{enumerate}
\end{description}
\newpage
\section{synIOEMPTY\_DONP}
  \subsection{adjp012}
\begin{description}
  \item [Rule] TAdjpattern22
  \item [Canonical Surface Representations]  \mbox{}\\ objrel/NP head/ADJ
  \item [Remarks] \mbox{}\\
 In main clauses the direct object is moved into subject position if 
there is no other element in shiftrel. 
  \item [Example adjectives]\mbox{}\\
\begin{enumerate}
  \item bekend
  \item duidelijk
\end{enumerate}
  \item [Example sentences]\mbox{}\\
\begin{enumerate}
  \item de uitkomst bekend
  \item het idee  is niet duidelijk geworden
\end{enumerate}
\end{description}
  \subsection{adjp123}
\begin{description}
  \item [Rule] TAdjpattern22 .2
  \item [Canonical Surface Representations]  \mbox{}\\ objrel/.. head/ADJ
  \item [Remarks]
Cf. the remarks on obligatory movement in the section on synNP.
  \item [Example adjectives]\mbox{}\\
\begin{enumerate}
  \item waard
\end{enumerate}
  \item [Example sentences]\mbox{}\\
\begin{enumerate}
  \item Pim is die inspanning waard
\end{enumerate}
\end{description}

\newpage
\section{synIOEMPTY\_HETQSENT}
  \subsection{adjp012}
\begin{description}
  \item [Rule] TAdjpattern26
  \item [Canonical Surface Representations]  \mbox{}\\ 
objrel/"het" complrel/SENTENCE head/ADJ
  \item [Remarks] \mbox{}\\
The direct object "het" is moved into subject-position.
  \item [Example adjectives]\mbox{}\\
\begin{enumerate}
  \item bekend
  \item duidelijk
\end{enumerate}
  \item [Example sentences]\mbox{}\\
\begin{enumerate}
  \item Het is niet bekend of ...
  \item Het niet duidelijk geworden wie ...
\end{enumerate}
\end{description}
\newpage
\section{synIOEMPTY\_HETOPENOMTESENT}
\subsection{adjp123}
\begin{description}
  \item [Rule] TAdjpattern24 .3
  \item [Canonical Surface Representations]  \mbox{}\\ objrel/"het" 
complrel/SENTENCE head/ADJ
  \item [Remarks] The synpattern is still to be added to the domain.
  \item [Example adjectives]\mbox{}\\
\begin{enumerate}
  \item verplicht
\end{enumerate}
  \item [Example sentences]\mbox{}\\
\begin{enumerate}
  \item Pim is het verplicht (om) te winnen
\end{enumerate}
\end{description}
\newpage
\section{synIOEMPTY\_HETTHATSENT}
  \subsection{adjp012}
\begin{description}
  \item [Rule] TAdjpattern26 .3
  \item [Canonical Surface Representations]  \mbox{}\\ 
objrel/"het" complrel/SENTENCE head/ADJ
  \item [Remarks] \mbox{}\\
The direct object "het" is moved into subject-position.
  \item [Example adjectives]\mbox{}\\
\begin{enumerate}
  \item bekend
  \item duidelijk
\end{enumerate}
  \item [Example sentences]\mbox{}\\
\begin{enumerate}
  \item Het werd bekend dat ... 
  \item Het is niet duidelijk geworden dat ...
\end{enumerate}
\end{description}
\newpage
\section{synIOEMPTY\_OPENOMTESENT}
\subsection{adjp123}
\begin{description}
  \item [Rule] TAdjpattern22. 3
  \item [Canonical Surface Representations] \mbox{}\\ 
complrel/SENTENCE head/ADJ
  \item [Remarks] The synpattern is still to be added to the domain.

  \item [Example adjectives]\mbox{}\\
\begin{enumerate}
  \item verplicht
\end{enumerate}
  \item [Example sentences]\mbox{}\\
\begin{enumerate}
  \item Pim is verplicht (om) te winnen
\end{enumerate}
\end{description}
\newpage
\section{synIOEMPTY\_QSENT}
  \subsection{adjp012}
\begin{description}
  \item [Rule] TAdjpattern22 .6
  \item [Canonical Surface Representations]  \mbox{}\\ 
complrel/SENTENCE head/ADJ
  \item [Remarks] \mbox{}\\
Expletive {\em er} must be inserted, either in shiftrel, or in erposrel.
  \item [Example adjectives]\mbox{}\\
\begin{enumerate}
  \item bekend
  \item duidelijk
\end{enumerate}
  \item [Example sentences]\mbox{}\\
\begin{enumerate}
  \item Er is niet bekend of ...
  \item Er is niet duidelijk geworden wie ...
\end{enumerate}
\end{description}
\newpage
\section{synIOEMPTY\_THATSENT}
  \subsection{adjp012}
\begin{description}
  \item [Rule] TAdjpattern22 .6
  \item [Canonical Surface Representations]  \mbox{}\\ 
complrel/SENTENCE head/ADJ
  \item [Remarks] \mbox{}\\
Expletive {\em er} must be inserted, either in shiftrel, or in erposrel.
  \item [Example adjectives]\mbox{}\\
\begin{enumerate}
  \item bekend
  \item duidelijk
\end{enumerate}
  \item [Example sentences]\mbox{}\\
\begin{enumerate}
  \item Er werd bekend dat ... 
  \item Er is niet duidelijk geworden dat ...
\end{enumerate}
\end{description}

\newpage
\section{synIONP}
  \subsection{adjp120}
\begin{description}
  \item [Rule] TADJPATTERN11.1a 
  \item [Canonical Surface Representations]  \mbox{}\\ 
indobjrel/NP head/ADJ
  \item [Remarks]  \mbox{}\\ 
For this pattern  no verbal examples exist. For all verbs with thetavp=
vp120 with an NP as the second argument this NP can be considered an indirect 
object. All other two-place verb are ergative. 
The example given below is exceptional. It does not seem to be an ergative 
adjective, neither is the second argument a direct object.

  \item [Example adjectives]\mbox{}\\ 
\begin{enumerate}
  \item ontrouw
\end{enumerate}
  \item [Example sentences]\mbox{}\\
\begin{enumerate}
  \item ...omdat hij mij ontrouw was
\end{enumerate}


\end{description}

\newpage
\section{synIONP\_DONP}
  \subsection{adjp012}
\begin{description}
  \item [Rule] TAdjpattern22
  \item [Canonical Surface Representations]  \mbox{}\\ 
  \item [Canonical Surface Representations]  \mbox{}\\ 
indobjrel/NP objrel/NP head/ADJ
  \item [Remarks]  \mbox{}\\ 
Ergativity is assumed because of the surface order:
..  dat hem het idee bekend is.
 In main clauses the direct object is moved into subject position if 
there is no other element in shiftrel. 
  \item [Example adjectives]\mbox{}\\
\begin{enumerate}
  \item bekend
  \item duidelijk
\end{enumerate}
  \item [Example sentences]\mbox{}\\
\begin{enumerate}
  \item de uitkomst is ons bekend
  \item het idee  is ons niet duidelijk geworden
\end{enumerate}
\end{description}
  \subsection{adjp123}
\begin{description}
  \item [Rule] TAdjpattern22 .1
  \item [Canonical Surface Representations]  \mbox{}\\ 
indobjrel/.. objrel/.. head/ADJ
  \item [Remarks] No remarks

  \item [Example adjectives]\mbox{}\\
\begin{enumerate}
  \item schuldig
\end{enumerate}
  \item [Example sentences]\mbox{}\\
\begin{enumerate}
  \item Pim is mij een gulden schuldig
\end{enumerate}
\end{description}
\newpage
\section{synIONP\_HETOPENOMTESENT}
\subsection{adjp123}
\begin{description}
  \item [Rule] TAdjpattern24
  \item [Canonical Surface Representations] \mbox{}\\  indobjrel/.. objrel/"het" 
complrel/SENTENCE head/ADJ
  \item [Remarks] No remarks

  \item [Example adjectives]\mbox{}\\
\begin{enumerate}
  \item verplicht
\end{enumerate}
  \item [Example sentences]\mbox{}\\
\begin{enumerate}
  \item Pim is het ons verplicht (om) te winnen
\end{enumerate}
\end{description}
\newpage
\section{synIONP\_HETQSENT}
  \subsection{adjp012}
\begin{description}
  \item [Rule] TAdjpattern26
  \item [Canonical Surface Representations] \mbox{}\\
indobjrel/NP  objrel/"het" complrel/SENTENCE head/ADJ
  \item [Remarks] \mbox{}\\
The direct object "het" is moved into subject-position.
  \item [Example adjectives]\mbox{}\\
\begin{enumerate}
  \item bekend
  \item duidelijk
\end{enumerate}
  \item [Example sentences]\mbox{}\\
\begin{enumerate}
  \item Het is ons niet bekend of ...
  \item Het ons niet duidelijk geworden wie ...
\end{enumerate}
\end{description}
\newpage
\section{synIONP\_HETTHATSENT}
  \subsection{adjp012}
\begin{description}
  \item [Rule] TAdjpattern26
  \item [Canonical Surface Representations] \mbox{}\\
indobjrel/NP  objrel/"het" complrel/SENTENCE head/ADJ
  \item [Remarks] \mbox{}\\
The direct object "het" is moved into subject-position.
  \item [Example adjectives]\mbox{}\\
\begin{enumerate}
  \item bekend
  \item duidelijk
\end{enumerate}
  \item [Example sentences]\mbox{}\\
\begin{enumerate}
  \item Het is ons bekend dat ...
  \item Het ons niet duidelijk geworden dat ...
\end{enumerate}
\end{description}
\newpage
\section{synIONP\_OPENOMTESENT}
\subsection{adjp123}
\begin{description}
  \item [Rule] TAdjpattern22 .3
  \item [Canonical Surface Representations] \mbox{}\\  indobjrel/..  
complrel/SENTENCE head/ADJ
  \item [Remarks] No remarks

  \item [Example adjectives]\mbox{}\\
\begin{enumerate}
  \item verplicht
\end{enumerate}
  \item [Example sentences]\mbox{}\\
\begin{enumerate}
  \item Pim is mij verplicht (om) te winnen
\end{enumerate}
\end{description}
\newpage
\section{synIONP\_QSENT}
  \subsection{adjp012}
\begin{description}
  \item [Rule] TAdjpattern22 .4 .5
  \item [Canonical Surface Representations] \mbox{}\\
indobjrel/NP complrel/SENTENCE head/ADJ
  \item [Remarks] \mbox{}\\
Expletive {\em er} must be inserted, either in shiftrel, or in erposrel.
  \item [Example adjectives]\mbox{}\\
\begin{enumerate}
  \item bekend
  \item duidelijk
\end{enumerate}
  \item [Example sentences]\mbox{}\\
\begin{enumerate}
  \item Er is ons niet bekend of ...
  \item Er ons niet duidelijk geworden wie ...
\end{enumerate}
\end{description}
\newpage
\section{synIONP\_THATSENT}
  \subsection{adjp012}
\begin{description}
  \item [Rule] TAdjpattern22 .4 .5
  \item [Canonical Surface Representations] \mbox{}\\
indobjrel/NP complrel/SENTENCE head/ADJ
  \item [Remarks] \mbox{}\\
Expletive {\em er} must be inserted, either in shiftrel, or in erposrel.
  \item [Example adjectives]\mbox{}\\
\begin{enumerate}
  \item bekend
  \item duidelijk
\end{enumerate}
  \item [Example sentences]\mbox{}\\
\begin{enumerate}
  \item Er is ons bekend dat ...
  \item Er ons niet duidelijk geworden dat ...
\end{enumerate}

\end{description}
\newpage
\section{synLOCPREPP}
\subsection{adjp120}
\begin{description}
  \item [Rule] TAdjpattern11 .5, .6
  \item [Canonical Surface Representations] \mbox{}\\
\begin{enumerate} 
  \item locargrel/PREPP head/ADJ
  \item locargrel/ADVP head/ADJ
\end{enumerate}
  \item [Remarks]  No remarks
  \item [Example adjectives] \mbox{}\\
\begin{enumerate}
  \item bekend
\end{enumerate}
  \item [Example sentences] \mbox{}\\
\begin{enumerate}
  \item Marie is in A. goed bekend
  \item Marie is hier behoorlijk bekend (not meaning: well known)
\end{enumerate}
\end{description}
\newpage
\section{synLOCEMPTY}
\subsection{adjp120}
\begin{description}
  \item [Rule] TAdjpattern11 .7
  \item [Canonical Surface Representations] \mbox{}\\
   locargrel/EMPTY head/ADJ
  \item [Remarks]  No remarks
  \item [Example adjectives] \mbox{}\\
\begin{enumerate}
  \item gewend
\end{enumerate}
  \item [Example sentences] \mbox{}\\
\begin{enumerate}
  \item Marie is al behoorlijk gewend 
\end{enumerate}
\end{description}
\newpage
\section{synNP}
\subsection{adjp120}
\begin{description}
  \item [Rule] TAdjpattern11 .1 .2
  \item [Canonical Surface Representations] \mbox{}\\objrel/NP head/ADJ
  \item [Remarks] \mbox{}\\ synNP is also the pattern that is assumed for the
superficially sentential complements in the following examples:
\begin{enumerate}
  \item Altijd te moeten koken wordt je snel zat.
  \item Dat hij altijd moet koken is Pim nu wel zat
\end{enumerate}
These cases are exceptional in that 
obligatory movement to 
the initial position is involved. 
There are no rules yet 
that acount for this fact. Presently it is assumed that the sentential 
complement is to be analysed as an NP dominating a SENTENCE. The prediction is
that 
all sentential direct object complement (type restriction: 
with propositional 
content) selected by adjectives, would be affected by the leftward movement 
obligatory.
  \item [Example adjectives] \mbox{}\\
\begin{enumerate}
  \item gewend
  \item beu
\end{enumerate}
  \item [Example sentences] \mbox{}\\
\begin{enumerate}
  \item Pim is de regen zat
  \item Pim is veel gewend
  \item Marie is de grote stad niet gewend
\end{enumerate}
\end{description}
\newpage
\section{synMEASUREPHRASE}
\subsection{adjp120}
\begin{description}
  \item [Rule] TAdjpattern11 .3
  \item [Canonical Surface Representations]\mbox{}\\  objrel/NP head/ADJ
  \item [Remarks] \mbox{}\\ The need for 
isomorphy with verbs such as {\em kosten} should be investigated
  \item [Example adjectives] \mbox{}\\
\begin{enumerate}
  \item waard
\end{enumerate}
  \item [Example sentences] \mbox{}\\
\begin{enumerate}
  \item Dat boek is tien gulden waard
\end{enumerate}
\end{description}
\newpage
\section{synOPENOMTESENT}
\subsection{adjp120}
\begin{description}
  \item [Rule] TAdjpattern16
  \item [Canonical Surface Representations]\mbox{}\\head/ADJ complrel/SENTENCE

  \item [Remarks] \mbox{}\\ 
  \item [Example adjectives] \mbox{}\\
\begin{enumerate}
  \item gewend
\end{enumerate}
  \item [Example sentences] \mbox{}\\
\begin{enumerate}
  \item Ik ben gewend (om) te winnen
\end{enumerate}
\end{description}
\newpage
\section{synOPENOMTESENTPROOBJ}
\subsection{adjp120}
\begin{description}
  \item [Rule] TAdjpattern17
  \item [Canonical Surface Representations]\mbox{}\\  complrel/SENTENCE head/ADJ

  \item [Remarks] \mbox{}\\  
\begin{enumerate}
\item There is no complete path for adjectives with this pattern yet. 
The phenomenon - complementation by means of a sentence with a PRO-object- 
needs to be 
studied still in some more detail. 
\item An abstract adjective MEANT is presumed in order to derive the cases 
without overt adjective. This is still to be added to the dictionary. 
\item problems:\\
At least the following 
two alternative treatments for the examples mentioned below are 
conceivable:
\begin{itemize}
  \item There is an additional empty argument (EMPTY,  or ALL, or whatever).
The pro-subject (BIGPRO) of the te-sentence is deleted 
under identity with EMPTY/ALL. 
(This would require adjp123 as the value for .thetaadj (and given the implicit 
naming convention, a different name for the patternrule).) 
  \item
The pro-subject is deleted without identity conditions to be met.
\end{itemize}
Pending a principled choice the present elaboration is compatible 
with the second alternative.

\item translation aspects: The dutch sentence 
{\em dit boek is niet om in te schrijven} should be 
translated into a passive construction of English: {\em this book is not meant 
to be
written in}. Sometimes a translation with a {\em for}-phrase is preferable:
{\em dit bier is niet om te drinken} translates into {\em this beer is not for 
drinking}. 
\item modifications:\\

\end{enumerate}

  \item [Example adjectives] \mbox{}\\
\begin{enumerate}
  \item geschikt
  \item MEANT (abstract adjective)
\end{enumerate}
  \item [Example sentences] \mbox{}\\
\begin{enumerate}
  \item Dit mes is geschikt om vis mee te snijden
  \item Dit mes is MEANT om vis mee te snijden
\end{enumerate}
\end{description}
\newpage
\section{synOPENTESENT}
\subsection{adjp120}
\begin{description}
  \item [Rule] TAdjpattern16
  \item [Canonical Surface Representations]  \mbox{}\\ head/ADJ complrel/SENTENCE
  \item [Remarks] \mbox{}\\
\begin{enumerate}
  \item 
This pattern accounts for adjectives with a sentential 
object with omitted preposition. In most cases the adjective has also
syn(PA)PREPOPEN(OM)TESENT or syn(PA)PREPTHATSENT in its patternset.  
  \item

Adjectives with this pattern do not allow the occurrence of {\em om}.
In general this pattern applies for a certain adjective if
a paraphrase with a (?static) that-sentence is possible. Compare:
\begin{description}
  \item []
Ik ben tevreden te kunnen vertrekken/dat ik kan vertrekken.
  \item []
Ik ben bang te verliezen/dat ik aan het verliezen ben
  \item []
Ik ben bang ziek te zijn/dat ik ziek ben
\end{description}
\item 
Some adjectives e.g. {\em bang} have both synHETOPENTESENT and 
synHETOPENOMTESENT in their patternset. This is the case where
the adjective has two meanings. In the one a paraphrase of the infinitival 
complement  with a that-sentence is possible. In the other
this possibility is absent. Instead there is the often the possibility of
a paraphrase with an explicit
preposition in combination with a infinitival complement.  Compare:
\begin{description}
  \item []
Ik ben bang (om) in de lift te stappen /Ik ben er bang voor (om) in de lift te 
stappen.
  \item Ik ben blij te kunnen vertrekken/*Ik ben er blij mee (om) te kunnen 
vertrekken
\end{description}

\end{enumerate}
  \item [Example adjectives] \mbox{}\\
\begin{enumerate}
  \item blij
  \item tevreden
\end{enumerate}
  \item [Example sentences] \mbox{}\\
\begin{enumerate}
  \item Pim is blij te kunnen vertrekken
  \item Pim is tevreden  te kunnen vertrekken
\end{enumerate}
\end{description}
\newpage
\section{synPATHPREPP}
  \subsection{adjp120}
\begin{description}
  \item [Rule] TAdjpattern11 .8
  \item [Canonical Surface Representations]\mbox{}\\
 dirargel/PREPNP head/ADJ
  \item [Remarks]  \mbox{}\\ 
\begin{enumerate}
  \item 
The PREPP-subgrammar does not define path-PREPPs yet.
  \item {\em Pim is onderweg} must be  
dealt with by TADJpattern14 (synPREPEMPTY).
\end{enumerate}

  \item [Example adjectives] \mbox{}\\
\begin{enumerate}
  \item onderweg
\end{enumerate}
  \item [Example sentences]\mbox{}\\
\begin{enumerate}
  \item Hij is onderweg (van A.) naar B.
\end{enumerate}
\end{description}
\newpage
\section{synPAPREPOPENOMTESENT}
  \subsection{adjp120}
\begin{description}
  \item [Rule] TAdjpattern15b
  \item [Canonical Surface Representations] \mbox{}\\ erposrel/"er" head/ADJ ../PREP complrel/
SENTENCE
  \item [Remarks] \mbox{}\\ This pattern accounts for adjectives with a sentential 
prepositional object with preadjectival {\em er} and the preposition 
obligatory in postadjectival position. 

  \item [Example adjectives] \mbox{}\\
\begin{enumerate}
  \item dol
  \item tevreden
\end{enumerate}
  \item [Example sentences]\mbox{}\\
\begin{enumerate}
  \item Hij is er dol op (om) in zee te zwemmen 
  \item Hij is er tevreden mee (om) in zee te kunnen zwemmen 
\end{enumerate}
\end{description}
\newpage
\section{synPAPREPQSENT}
  \subsection{adjp120}
\begin{description}
  \item [Rule] TAdjpattern15b
  \item [Canonical Surface Representations] \mbox{}\\
erposrel/"er" head/ADJ ../PREP complrel/
SENTENCE
  \item [Remarks] \mbox{}\\ This pattern accounts for adjectives with an interrogative
 sentential 
prepositional object with preadjectival {\em er} and the preposition 
obligatory in postadjectival position. 

  \item [Example adjectives] \mbox{}\\
\begin{enumerate}
  \item onzeker 
  \item onduidelijk
\end{enumerate}
  \item [Example sentences]\mbox{}\\
\begin{enumerate}
  \item Hij is er onzeker over of Jan komt
  \item Hij is er onduidelijk over wie er komt
\end{enumerate}

\end{description}
\newpage
\section{synPAPREPTHATSENT}
  \subsection{adjp120}
\begin{description}
  \item [Rule] TAdjpattern15b
  \item [Canonical Surface Representations] \mbox{}\\ erposrel/"er" head/ADJ ../PREP complrel/
SENTENCE
  \item [Remarks] \mbox{}\\ This pattern accounts for adjectives with a sentential 
prepositional object with preadjectival {\em er} and the preposition 
obligatory in postadjectival position. 

  \item [Example adjectives] \mbox{}\\
\begin{enumerate}
  \item blij
  \item tevreden
\end{enumerate}
  \item [Example sentences]\mbox{}\\
\begin{enumerate}
  \item Hij is er blij mee dat het gaat regenen 
  \item Hij is er tevreden over dat het niet regent
\end{enumerate}
\end{description}
\newpage
\section{synPOSTADJPREPNP}
  \subsection{adjp120}
\begin{description}
  \item [Rule] TAdjpattern12b
  \item [Canonical Surface Representations] \mbox{}\\
head/ADJ postadjrel/PREPP
  \item [Remarks]  \mbox{}\\The distinction between adjectives with the value 
synPREPNP  for .adjpattern and those with value synPOSTADJPREPNP is necessary
in order to account for their different distribution in combination with 
the {\em er}-forms. E.g. {\em Hij is erop bedacht}
 versus 
{\em *Hij is erop dol}. 
  \item [Example adjectives] \mbox{}\\
\begin{enumerate}
  \item dol
  \item bang
\end{enumerate}
  \item [Example sentences]\mbox{}\\
\begin{enumerate}
  \item Pim is dol op vis
  \item Jan is er bang voor
\end{enumerate}
\end{description}
\newpage
\section{synPREPEMPTY}
  \subsection{adjp120}
\begin{description}
  \item [Rule] TAdjpattern14
  \item [Canonical Surface Representations] \mbox{}\\
prepobjrel/EMPTY head/ADJ
  \item [Remarks]  No remarks
  \item [Example adjectives] \mbox{}\\
\begin{enumerate}
  \item verliefd
  \item bezig
  \item bang 
  \item verslaafd
\end{enumerate}
  \item [Example sentences]\mbox{}\\
\begin{enumerate}
  \item Pim is verliefd 
  \item Pim is bang
  \item Hij is verslaafd
\end{enumerate}
\end{description}
\newpage
\section{synPREPNP}
  \subsection{adjp120}
\begin{description}
  \item [Rule] TAdjpattern12a
  \item [Canonical Surface Representations] \mbox{}\\
prepobjrel/PREPNP head/ADJ
  \item [Remarks]  \mbox{}\\ The distinction between adjectives with the value 
synPREPNP  for .adjpattern and those with value synPOSTADJPREPNP is necessary
in order to account for their different distribution in combination with 
the {\em er}-forms. E.g. {\em Hij is erop bedacht} versus 
{\em *Hij is erop dol}.
  \item [Example adjectives] \mbox{}\\
\begin{enumerate}
  \item verliefd
  \item bezig
  \item tevreden
  \item beducht
\end{enumerate}
  \item [Example sentences]\mbox{}\\
\begin{enumerate}
  \item Pim is verliefd op Marie
  \item Pim is bezig met een opdracht
  \item Zij is tevreden met het resultaat
  \item Hij is beducht voor de uitslag
\end{enumerate}
\end{description}
\newpage
\section{synPREPOPENOMTESENT}
  \subsection{adjp120}
\begin{description}
  \item [Rule] TAdjpattern15a
  \item [Canonical Surface Representations] \mbox{}\\ prepobjrel/["er"  ../PREP] head/ADJ complrel/
SENTENCE
  \item [Remarks]  \mbox{}\\This pattern accounts for adjectives with a sentential 
prepositional object with preadjectival {\em er}-PREPP.

  \item [Example adjectives] \mbox{}\\
\begin{enumerate}
  \item verzot
  \item gebrand
  \item bedacht
  \item bewust
  \item bekend
\end{enumerate}
  \item [Example sentences]\mbox{}\\
\begin{enumerate}
  \item Hij is erop verzot (om) te laat te komen
  \item Hij is erop gebrand (om) op tijd te komen
  \item Hij is erop bedacht dat het gaat regenen 
  \item Hij is zich ervan bewust dat hij slecht ziet
  \item Hij is ermee bekend dat het bedrijf op vrijdag gesloten is 
\end{enumerate}
\end{description}
\newpage
\section{synPREPQSENT}
  \subsection{adjp120}
\begin{description}
  \item [Rule] TAdjpattern15a
  \item [Canonical Surface Representations] \mbox{}\\ ../"er" head/ADJ ../PREP complrel
SENTENCE
  \item [Remarks] \mbox{}\\ 
This pattern accounts for adjectives with an interrogative
 sentential 
prepositional object with preadjectival {\em er}-PREPP.

  \item [Example adjectives] \mbox{}\\
\begin{enumerate}
  \item onzeker 
  \item benieuwd
\end{enumerate}
  \item [Example sentences]\mbox{}\\
\begin{enumerate}
  \item Hij is erover onzeker of Jan komt
  \item Zij is ernaar benieuwd  of het gaat regenen 
\end{enumerate}
\end{description}
\newpage
\section{synPREPTHATSENT}
  \subsection{adjp120}
\begin{description}
  \item [Rule] TAdjpattern15a
  \item [Canonical Surface Representations] \mbox{}\\ ../"er" head/ADJ ../PREP complrel
SENTENCE
  \item [Remarks] \mbox{}\\ 
This pattern accounts for adjectives with a declarative
 sentential 
prepositional object with preadjectival {\em er}-PREPP.

  \item [Example adjectives] \mbox{}\\
\begin{enumerate}
  \item gewend
\end{enumerate}
  \item [Example sentences]\mbox{}\\
\begin{enumerate}
  \item Hij is eraan gewend dat het vaak regent
\end{enumerate}
\end{description}
\newpage
\section{synQSENT}
  \subsection{adjp120}
\begin{description}
  \item [Rule] TAdjpattern16
  \item [Canonical Surface Representations] \mbox{}\\ complrel/SENTENCE head/ADJ
  \item [Remarks] \mbox{}\\ The first example given below, with a YesNo-
interrogative as a complement, is marginal. Better is a variant
with an {\em er}-PREPP, or just a preposition: Pim is {\bf er} niet zeker 
{\bf van}
of zij op tijd is vertrokken; 
Pim is onzeker {\bf over}  of het weer zo 
mooi blijft. 
  \item [Example adjectives]\mbox{}\\
\begin{enumerate}
  \item onzeker
  \item benieuwd
\end{enumerate}
  \item [Example sentences]\mbox{}\\
\begin{enumerate}
  \item 
? Pim is onzeker of zij op tijd is vertrokken
  \item 
Pim is benieuwd wie er komt
\end{enumerate}
\end{description}
\newpage
\section{synTHATSENT}
  \subsection{adjp012}
\begin{description}
  \item [Rule] TAdjpattern16
  \item [Canonical Surface Representations]\mbox{}\\  complrel/SENTENCE head/ADJ
  \item [Remarks]  No remarks
  \item [Example adjectives]\mbox{}\\
\begin{enumerate}
  \item blij
\end{enumerate}
  \item [Example sentences]\mbox{}\\
\begin{enumerate}
  \item Pim is blij dat het gaat regenen
\end{enumerate}
\end{description}
\newpage
\section{synVOORNP}
  \subsection{adjp012}
\begin{description}
  \item [Rule] TAdjpattern18a
  \item [Canonical Surface Representations]\mbox{}\\  voorobjrel/NP head/ADJ
  \item [Remarks] \mbox{}\\
\begin{enumerate}
  \item 
This pattern is meant to treat "belanghebbende voorwerpen".
This class of complements does not allow local translation of the preposition.
For example, {\em lief voor dieren} translates into {\em kind to animals}. 
  \item 
The distinction between voorobj's and prepobj's is motivated mainly by the 
different restriction on ordering: degree-modifiers may precede prepobj's, but 
not voorobj's. Cf. {\em de zeer aan mij gehechte hond} versus 
*{\em de zeer voor mij belangrijke datum}.

  \item 
Not all PREPPs with preposition {\em voor} are to be treated as voorobj.
There 
are of course  also prepositional objects with a preposition {\em voor} that
should not be translated locally (relation: prepobjrel). Secondly 
there are modifiers with {\em voor} that allows local translation. These cases 
are dealt with by means of the modifier-introducing rule RADJVOOROBJMOD. 
(NB.  The value for .thetaadj is then adjp100 instead of adjp120.)
This rule is probably also relevant for cases such as the
{\em handig voor op reis/onderweg}, with a complement that is not an NP.
  \item 
Quite often an adjective that allows a voorobj (with a syncategoremtically 
introduced {\em voor} ) also allows a {\em voor}-modifier. For example, 
{\em belangrijk voor} is sometimes translated as 
{\em important for}. So {\em belangrijk} must have two entries in the 
dictionary: one that allows modification with thetadj = adjp100, 
the other with thetadj = adjp120.
  \item As the distinction between prepobj and voorobj has no counterpart in 
English TADJpattern18a has no counterpart in English M-grammar either.
\end{enumerate}
  \item [Example adjectives]\mbox{}\\
\begin{enumerate}
  \item belangrijk
  \item leuk
\end{enumerate}
  \item [Example sentences]\mbox{}\\
\begin{enumerate}
  \item  Pim is belangrijk voor mij (Eng: Pim is important {\bf to} me)
  \item  Pim is leuk voor dieren (versus: dit is leuk voor dieren)
\end{enumerate}
\end{description}
\newpage
\section{synVOOREMPTY}
  \subsection{adjp012}
\begin{description}
  \item [Rule] TAdjpattern18b
  \item [Canonical Surface Representations]\mbox{}\\  head/ADJ
  \item [Remarks]  No remarks
\begin{enumerate}
  \item
The presumed 
difference between synVOOREMPTY and synPREPEMPTY is
motivated by a difference in semantics.  The substitution of the EMPTYVAR 
introduced by TADJpattern18b is supposed not to 
amount to Existential Quantification, as 
is the case with TADJpattern14.
  \item  The bare occurrences of for example 
{\em belangrijk} as in {\em Zij is belangrijk} 
are supposed to have a separate dictionary entry with thetaadj = adjp100.
\end{enumerate}
  \item [Example adjectives]\mbox{}\\
\begin{enumerate}
  \item No relevant examples found yet
\end{enumerate}
  \item [Example sentences]\mbox{}\\
\begin{enumerate}
  \item No relevant examples found yet
\end{enumerate}
\end{description}
\end{document}


