\documentstyle{Rosetta}
\begin{document}
   \RosTopic{General}
   \RosTitle{Notulen groepsvergadering 11-2-1991}
   \RosAuthor{Franciska de Jong}
   \RosDocNr{459}
   \RosDate{12-2-1991}
   \RosStatus{approved}
   \RosSupersedes{-}
   \RosDistribution{Project}
   \RosClearance{Project}
   \RosKeywords{Notulen}
   \MakeRosTitle
\hyphenation 
{deel-tijd-hoog-leraar
ont-wik-keld
Hoge-school
af-ron-ding
na-tuur-lij-ke
de-fi-ni-ering
werk-zaam-heden
li-te-ra-tuur-ver-wij-zing
bij-voor-beeld}

%
%
\begin{description}
\item[Aanwezig:] Lisette Appelo,
                 Franciska de Jong, Ren\'{e} Leermakers,  
                 Jan Landsbergen (voorzitter),   
                 Jan Odijk, Elena Pinillos, 
                 Andr\'{e} Schenk,
                 Harm Smit,
                 Frank Uittenbogaard, Petra de Wit
                  
                  

\item[Afwezig:]  Joep Rous
\item[Agenda:]\mbox{}
  \begin{enumerate}
  \item Notulen/Actiepunten
  \item Taakverdeling '91
  \item CE/PIP
  \item Budget '91
  \item Reizen
  \item Bezoeken
  \item Evaluatiedag
  \item Rondvraag
  \end{enumerate}
\end{description}

\section{Notulen}
De notulen van de vorige vergadering worden -met correctie van enkele 
tikfouten- goedgekeurd.

\section{Afgeronde actiepunten}

Afgeronde actiepunten:
\begin{enumerate}

  \item De voor de inventarisatie van ons boekenbestand benodigde lijstjes
        zijn aan Margot verstuurd. 
\end{enumerate}

\noindent
Uit de redactie: 
\begin{enumerate}
  \item Er is een stuk van de hand van Lisette verspreid waarin is aangegeven
wie welk stuk kan gaan schrijven.
  \item Lisette heeft een mail verstuurd met richtlijnen voor het opnemen van 
literatuurverwijzingen en het gebruik van {\LaTeX}-labels.
\item De redactie is zich ervan bewust dat de 
opgegeven omvangsbeperking (10 pp.) in combinatie met Rosettastyle 
in een aantal gevallen te krap is. In overleg met de auteurs 
zal bekeken worden 
of de norm van 10 pp. per hoofdtuk bijgesteld moet worden. 
Het adagium "maak het niet langer dan strikt noodzakelijk" blijft evenwel van 
kracht. 

\end{enumerate}

\section{Taakverdeling '91}
Met het eerder besproken voorstel van Jan L. heeft in grote lijnen de 
instemming van alle betrokkenen.
De taak van Petra en Franciska in de periode na 1 juli a.s. 
zal t.z.t. worden vastgesteld.

De IT-voorstellen zijn inmiddels door Timmer goedgekeurd. Nadere details, 
bijvoorbeeld over de relatie tussen het IT-onderzoek en ander onderzoek in de
groep Landsbergen, zijn nog niet bekend.

\section{\bf CE/PIP}
Naar aanleiding van de resultaten van het 'notebook'-marktonderzoek 
hebben de product managers van de groep 'text products' besloten tot nader 
order
niet te investeren in de ontwikkeling van hardware, maar in de ontwikkeling van 
linguistische software voor PC's. Dit besluit wordt gesteund door hun 
superieuren.
Deze ontwikkeling lijkt voor de groep Landsbergen niet ongunstig.

In de komende maanden zal een aantal 
voorstellen nader worden uitgewerkt. Niet onwaarschijnlijk is dat het onderzoek 
naar spelling checkers minder relevant wordt. 

\section{\bf Budget '91}

\begin{enumerate}
  \item 
Zowel het aangevraagde budget voor de conversie, als de voorgestelde 
besteding ervan zijn goedgekeurd.
  \item
Voor het gebruik van het VAX-cluster is een budget toegekend van 
plm. 1 miljoen gulden. Het gebruik in januari 1991 (f. 78.000) 
loopt hiermee in de pas.

Overigens komt er
voor het gebruik van het VAX-
cluster geen maximumtarief  per gebruikersgroep. 
\item Van het aangevraagde reisbudget voor buitenlandse reizen   van 
f. 35.000 is maar f. 20.000 toegekend.
\end{enumerate}

\section{\bf Reizen '91}
Voor zover nu bekend staan de volgende reizen op het programma (als niet op 
kosten van het Philips zal worden gereisd 
staat dat aangegeven):

\begin{enumerate}
  \item Jan L. (OTS): ACL, workshop over omkeerbaarheid, MT-summit3 - VS
  \item Elena/Petra (STT), Joep, Ren\'{e} : ACL-Berlijn 
  \item Language Fair - London 
  \item Summerschool LLI - Saarbruecken 
  \item NLP and its applications - Avignon 
\end{enumerate}


\section{\bf Bezoeken}
\begin{enumerate}
  \item  
Arthur Dirksen (IPO) komt op donderdagmiddag  21 februari 1991  op bezoek bij 
Rosetta.
  \item 
De groep Landsbergen brengt op woensdag 27 februari 1991 een bezoek 
aan het IPO. Het bezoek begint met een lunch.
\end{enumerate}

  \section{\bf Evaluatiedag}
Er zal op initiatief van Joep een evaluatiedag worden gehouden over het 
Rosetta-project. 
De bedoeling is om met  een zekere distantie terug te kijken 
op de gang van zaken in de
afgelopen jaren, om zodoende  
een helderder beeld te krijgen
over vragen als: wat is er goed gegaan, wat had er anders gemoeten, wat is 
er niet gelukt en waarom niet.

Er zal een aantal thema's worden geselecteerd. Als mogelijke thema's zijn o.m. 
genoemd: 
syntactische capaciteit, complexiteit, isomorfie, tussentaal, evaluatie van
performance, inwerken.
Andere suggesties zijn welkom.
Elk thema zal  door een van de groepsleden kort worden ingeleid.

Jan L. zal een datum in maart reserveren en het programma voorbereiden.

\section{Actiepunten}
\begin{itemize}
  \item Jan O.: verspreiding van het Rosetta LEXIC-document
  \item (algemeen): Voor degenen die het aangaat: alvast
beginnen aan de hoofdstukken voor het Rosetta-boek.
\end{itemize}


\section{Volgende vergadering}
De volgende vergadering is op maandag 25 februari, 13.30 uur in WY7.\\
\end{document}

