
\documentstyle{Rosetta}
\begin{document}
   \RosTopic{Rosetta3.doc.linguistics}
   \RosTitle{The treatment of possessive modification}
   \RosAuthor{Franciska de Jong}
   \RosDocNr{413}
   \RosDate{November 29, 1991}
   \RosStatus{approved}
   \RosSupersedes{}
%concept of September 4, 1989}
   \RosDistribution{Project}
   \RosClearance{Project}
   \RosKeywords{M-rules, NPsubgrammars, possessive modification}
   \MakeRosTitle
%
%
%\input{[odijng.mrules]mrudocdef}
\input{[dejong]definitions}
\def\sg{subgrammar}
\section{Introduction}

This document describes 
the analysis of possessive 
modification in NPs. The relevant rules are part of the subgrammars 
CNformation, NPformation. For Dutch these rules are also 
described in the rule documentation documents
r482 and r480 (FdeJ).\\
The relevant morphological attributes for Dutch are discussed in document r329.


\noindent
Possessive modification pertains to the kind of modification that occurs in the 
NPs in (\ref{1}). The examples are restricted to constructions 
conatining a nominal head. Possessive modification in NPs with an empty head
 (EN), or in NPs lacking the syntactic level CN (e.g. pronomiminal NPs, such as 
{\em wie van ons}) 
are discussed separately in 
section~\ref{enh}. The Dutch examples given
have counterparts in English, though it is not evident that 
syntactic similarity always implies semantic identity. 
For Spanish a more restricted set of 
constructions can be used to express possessive modification. 
Possessives occurring as modifier to 
a non nominal head are discussed in section~\ref{nnh}.\\

\begin{lxam}
&a.&& Het boek {\em van Jan}\\ 
&b.&& {\em Jans} boek \\ 
&c.&& Twee van {\em Jans} boeken\\ 
&d.&& {\em Ons} boek\\ 
&e.&& Twee van {\em onze} boeken\\ 
\label{1}
&f.&& {\em Wiens} boeken\\
&g.&&De boeken {\em van mijn vader}\\
&h.&&{\em Mijn vaders} boeken\\
\end{lxam}

\noindent
In all the examples above the "possessor"
of the object(s) denoted by {\em boek(en)} is indicated by means of a modifier,
 either in prenominal position by means of a POSSADJ or a genitival NP,
or in postnominal position. 
In the latter case
the possessive expression (PE) 
is a {\em van}-PP. (In English or Spanish a {\em of}- 
or {\em de}-PP respectively.)\\



\noindent
In this document the following issues will be discussed separately:

\begin{itemize}
  \item Relevant attributes: cases, possgeni 
  \item Possessive modification and non-nominal NP heads.
  \item Possessive modification and empty NP heads.
  \item The relation with other postnominal {\em van}-PPs.
\end{itemize}

\section{Possessive modification in NPs with a nominal head}
\subsection{The problem}
The following two stages in 
the derivation of NPs with a nominal head 
are relevant here: \\

\begin{description}
  \item 
(i) the introduction of the S-tree level CN and 
the (optional) introduction of arguments and modifiers to the noun. 
This part 
of the derivation of NPs is dealt with in the subgrammar CNformation.
  \item 
(ii) the introduction of the S-tree level NP and the (optional) 
introduction of modifiers 
and the insertion of a determiner phrase as a sister to the CN.
This part 
of the derivation of NPs is dealt with in the subgrammar NPformation.
\end{description}

Schematically:\\


\begin{verbatim}
               NP             -output of subgrammar NPformation
           /       \
    detrel-
   expression        CN       -outputlevel of subgrammar  CNformation
  (optional)          |        
                     NOUN
                      |
                    SUBNOUN   -output of derivation-subgrammar
                      |
                     BNOUN
\end{verbatim}


\noindent
The distribution of possessive modification over these two 
stages in the process of deriving NPs is not 
self-evident.
PE's require special attention for two reasons:
\begin{description}
  \item 
(a) Languages may vary with respect to the preference for 
postnominal possessive modification or prenominal 
possessive modification. 
Prenominal PEs
are in complementary distribution with the occurrence of elements in 
{\em detrel}-position, a position immediately dominated by NP.

The two positions that may be occupied by a PE 
carry information about the possessor 
can be illustrated by the following structure:

\begin{verbatim}
           NP

     detrel   head

     PE        CN

      ...   head    possrel

            NOUN     van-PE

\end{verbatim}


If there are translational equivalents for which 
a prenominal PE must be related to postnominal PE
the analysis must 
accommodate the mapping of  postnominal PREPPs in the S-tree immediately dominated by CN, 
on non-PREPPs in the S-tree immediately dominated by NP.
The incongruency is complicated by
the fact that PP-modification is ordered before the account of 
{\em detrel}-elements.
  \item
(b) 
Prenominal possessives correspond to two semantic 
phenomena: they express both modification and determination.  
This interaction of possessive modification 
with the semantic operation determination may be illustrated 
by the equivalence of the NPs in (\ref{2}).\\


\begin{lxam}
&a.&& Mijn zwagers fiets\\ \label{2}
&b.&& De fiets van mijn zwager
\end{lxam}

\noindent
In (\ref{2}b) the definiteness of the NP as a whole is 
due to the occurrence of the definite article {\em de}. 
In (\ref{2}a) no explicit definite article shows up, while the distribution of 
the NP is restricted in the same way as for (\ref{2}a).
It might therefore be concluded that {\em mijn} does not only express
possessive modification, but also definiteness. As definiteness is 
generally related to the occurrence of a determiner out of a certain subset 
of determiners, and as the occurrence of POSSADJs excludes the occurrence of 
other DETPs, it can be concluded that 
{\em mijn} expresses determination as well. 
\end{description}


\noindent
The idea that in the case of possessive modification the interaction
of two phenomena  may be involved plays a dominant role
in the Rosetta approach towards the problem.

\subsection{Outline of the approach}
Semantically the 
possessor is taken as a
modifying element. In the initial stage of the derivation,  
possessors are denoted by expressions of the 
category NP, irrespective of the syntactic category 
that ends up as indicator of the possessor. 

Just like attributive adjectival modifiers in the NP, possessive modifiers 
are considered to express modification of nominal expressions of the 
$<e,t>$-type, that is of 
expressions that in Rosetta are associated with the syntactic category NOUN.
Therefore, the modifying function of possessives is dealt with in the subgrammar 
that deals with restricted modification of NOUNs: CNformation.
The relevant rules are: RCNmodposs1, RCNmodposs2 and RCNmodposs3.

Generally spoken, at most one PE occurs as a modifier. However, it is not clear 
whether this is a pragmatic or a syntactic constraint. For example, the status
of {\em Jans boek van ons} is unclear. The rules introducing 
a PE are member of a rule class that may be applied iteratively. 
The occurrence of more than one PE is prohibited by a condition on the 
relation {\em posrel}.


In the case of a postnominal PP, the CN-tree derived is similar 
to the CN-tree of the eventual NP. The possessive preposition ({\em van}) is
introduced syncategorematically. 
In the case of a prenominal possessive (or in Spanish in the case of a 
postnominal POSSADJ: {\em el libro mio}) the possessor NP will have to be 
transformed. Transforming a possessor NP into a modifying category 
(POSSADJ, genitival NP) 
is not performed by the subgrammar CNformation, but by the 
subgrammar NPformation.

\section{Tasks for the relevant subgrammars}

The task of either subgrammar involved can be summarized as follows:

\begin{description}
  \item [Subgrammar CNformation]\mbox{}\\ 
\noindent
Disregarding the occurrence of other modifiers or arguments, possessive 
modification starts with a two-place rule with CN (head) and NP  (possessor)
as its arguments. The S-tree yielded by this operation may vary. 
There are the following possibilities: 

\begin{itemize}
  \item 
a possessive preposition is introduced 
syncategorematically and the resulting CN contains a 
postnominal PREPP with relation {\em posrel}.\footnote{For reasons of 
efficiency, S-Parser does not assign the relationname $posrel$. Instead the 
more general $postmodrel$ is assigned. The 'analytical'
replacement of {\em postmodrel}
 by {\em posrel} is dealt with in subgrammar NPformation 
by transformation class TC\_NPpostopostmod} CNs thus defined correspond to 
a substructure of the NP to be derived, or 
  \item
the resulting CN consists of a head noun plus a postnominal NP with relation
{\em posrel}. CNs thus defined require the application of rules that replace the 
postnominal NP within the CN by a prenominal element
\item 
(Spanish only)  
the resulting CN consists of a head noun plus a postnominal POSSADJ 
with relation {\em posrel}: {\em libro mio} (eventually for example: 
{\em el libro mio}).
\end{itemize}

  \item [Subgrammar NPformation]\mbox{}\\ 
\noindent
In the case of a CN containing a PREPP under {\em posrel} a rule out of rule 
class NPformation will apply. This may be the 
rule introducing a definite article, but also the introduction of an other 
determiner, or the rule deriving a bare NP. 

In the case of a CN containing  NP under {\em posrel}, only two rules 
may be 
applied 
successfully: RNPformation5 or RNPpartitiveformation2, each  
yielding an NP in which the possessor 
NP is transformed into a prenominal element according to the following table.\\


\begin{tabular}{|l|l|l|} \hline 
head of possessor NP  & surface category & example\\ \hline
PROPERNOUN & NP with genitival case& Jan $\rightarrow$ Jans\\ 
allowing a genitival -s  &spelled out on proper noun&\\ \hline
CN allowing a genitival -s& NP with genitival case & mijn vader $\rightarrow$  mijn vaders\\
&spelled out on the noun &\\ \hline
PERSPRO & POSSADJ & ik $\rightarrow$ mijn \\ \hline
WHPRO & POSSADJ & wie $\rightarrow$ wiens \\ \hline
INDEFPRO & NP with genitival case 
& ieder $\rightarrow$ ieders\\ 
allowing a genitival -s  &spelled out on INDEFPRO&\\ \hline
\end{tabular}

\end{description}

\noindent
RNPformation5 yields NPs with a non-partitive expression in $detrel$.
It is a one-place rule taking a CN with an NP in $posrel$ as input.
RNPparttiveformation2 yields a partitive NP, that is a NP with a DETP that 
accomodates two DET-elements: DET$_1$ van DET$_2$ CN. (Cf. documentation on 
partitives, incorporated in doc:r480 (on subgrammar NPformation) 
and in doc:r484 (on subgrammar DETPformation).
It is a two-place rule taking as input a CN with an NP in $posrel$ as well as a 
DETP. In the resulting partitive the DET$_1$ slot is occupied by the input DETP, 
while the DET$_2$ slot contains the possessor expression. 
The preposition {\em van} is introduced syncategorematically. 
\\

\begin{tabular}{|l|l|l|} \hline
&      RNPformation5& RNPpartitiveformation2\\ \hline
input&CN[head/$boeken$ + posrel/$ik]$ &CN[head/$boeken$ + posrel/$ik$]\\
& - &DETP: head/$twee$\\ \hline
output&{\em mijn boeken}& detrel/{\em twee van mijn} +  head/$boeken$\\ \hline
\end{tabular}


\noindent
In general RNPformation5 yields definite NPs. However, in the case of 
interrogative NPs such as {\em 
wiens boeken}, which may occur in existential contexts, 
indef(inite) seems a more correct value for the attribute $
definite$

\section{The relevant attributes}

In the treatment of possessive modification, several attributes play a crucial 
role. An overview:

\begin{description}
  \item {\bf .possgeni} (Dutch), {\bf .poss} (English) (not relevant for 
Spanish)\\
The value of .possgeni and .poss indicates whether a noun can get a genitival form. 
Nouns for which .possgeni=false will never show up as a prenominal PE.
Nouns for which .possgeni=true may show up as a prenominal PE, but only if 
the nodes with which the noun forms an NP do not block this.
For example the noun {\em collega} may be realized as {\em collega's} 
but if 
it is modified this possibility vanishes, as in the case of 
{\em collega uit Parijs}. Consequently 
in the derivation of any NP with a head noun with possgeni=true
the corressponding value at CN-level may change. (NB. the attribute 
.possgeni is an inherent attribute. 
There is no corresponding attribute that records the actual value on the 
nominal projections. For these non-lexical levels .possgeni is used too).
The value is checked in rule RCNmodposs3    
which introduces a possessor NP in postnominal position
($posrel$) without introducing a PREPP-level. In \sg \ NPformation such NPs are 
transformed into genitival NPs (or POSSADJs).



  \item {\bf .cases} (CN, NP)\\
The value of .cases is set to [genitive] in rule RNPformation5 and 
RNPpartitiveformation2.
\item {\bf .geni}  (NOUN, PROPERNOUN, INDEPRO) \\ 
This attribute indicates whether a category is to be spelled out in its 
genitival form. Example: {\em vaders, Jans,  Jane's, iemands}.


\end{description}


\section{The relation with non-possessive postnominal {\em van/of/de}-PREPPs}

In Rosetta the distinction made between possessive 
{\em van/of/de}-PREPPs and other {\em van}-PREPPs yields 
an ambiguity that is 
not solved by the grammars. 
For all {\em van/of/de}-PREPPs 
a possessive derivation is available in addition to the other 
interpretations (directional, etc.).
Whether these PREPPs have  a complement status too (prepobj)
is determined by the attributes {\em thetanp} , {\em nounpatterns} and {\em 
prepkey}.\\

\noindent
For an NP such as {\em de afkeer van Rosetta}
three kinds of derivation are available:

\begin{enumerate}
  \item syncategorematic {\em van}: {\em Rosetta} as possessor
  \item syncategorematic {\em van}: {\em Rosetta} as prepobj to {\em afkeer}
  \item {\em van} as basic expression, e.g. with a directional interpretation 
indication the 'source' of a 'movement'.
\end{enumerate}


\noindent
The {\em van/of/de} occurring in partitives is not analyzed as introducing a 
PREPP. Cf. doc:r484 (FdeJ; subgrammar DETformation). 
So NPs such as {\em elk van de kinderen} does not involve a PREPP, and 
confusion with possessive constructions to be expected only in case an NP is 
ambiguous between a partitive interpretation and an interpretation as
a empty headed NP with a possessive 
prepositional modifier. Example: Ik heb er {\em twee van de kinderen} gezien. 


\section{Mismatches}
The translation of possessive expressions does not take into account that 
at least in English, significant differences in interpretation may 
exist between
prenominal PEs and postnominal possessive PREPPs
and that it is not always claer what kind of possession is expressed.
More research is needed here in order to determine
the correct translations without making reference to extralinguistic facts.


Examples of this vagueness:
\begin{enumerate}
  \item 
{\em Jane's book} means both 'the book by Jane' and 
'the book of Jane'
  \item
{\em The children's toy} may mean: 'the toys possessed by the children' and 
'the toys the children are playing with'
  \item
{\em the power of love} does not express a  standard possessive relation
  \item
{\em all the men of science}  does not express a  standard possessive relation
 has no prenominal counterpart: *{\em all science's men}
\end{enumerate}


\section{Possesive modification and empty nominal heads}
\label{enh}
Possessive modifiers may occur also in NPs lacking an over head. 
Examples:\\


\begin{lxam}
&a.&& die EN van ons\\
&b.&& Jans twee gele EN\\
&c.&& twee EN van de buren\\
\end{lxam}


\noindent
After formation of the CN-level via RC\_CNformation, the rules introducing 
possessive modification apply to these expressions in a way similar to 
the NPs with a full nominal head.

\section{Possesive modification and non-nominal heads}
\label{nnh}
A certain subset 
of NPs is lacking the syntactic level CN. 
This is the case with NPs formed on the basis of 
INDEFPROs, WHPROs, PROPERNOUNS. 
These NPs may be modified by an PE-like modifier too. \\

\begin{lxam}
&a.&&onze Jan, Jan van de buren\\
&b.&&wie van de deelnemers\\
&c.&&iemand van de club, iets van ons \\
\end{lxam}

\noindent
It is however not clear
whether in all these cases a possession-relation must be assumed. 
Pending further research, some relevant rules have been written.
They are not part of a control expression. 

\end{document}

