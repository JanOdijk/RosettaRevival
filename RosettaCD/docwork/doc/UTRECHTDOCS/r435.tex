\documentstyle{Rosetta}
\begin{document}
   \RosTopic{General}
   \RosTitle{Notulen Groepsvergadering 12-4-1990}
   \RosAuthor{Petra de Wit}
   \RosDocNr{435}
   \RosDate{1-6-1990}
   \RosStatus{approved}
   \RosSupersedes{-}
   \RosDistribution{Project}
   \RosClearance{Project}
   \RosKeywords{minutes}
   \MakeRosTitle



\hyphenation{be-oordelen}
\hyphenation{samen-werking}
\hyphenation{woorden-boek}
\begin{itemize}
  \item {\bf aanwezig}: Andr\'{e} Schenk, Jan Landsbergen, 
                     Franciska de Jong, Petra de Wit,  
                     Elena Pinillos, Joep Rous, Jan Odijk, Harm Smit,
                     Ren\'{e} Leermakers, Frank Uittenbogaard.

  \item {\bf afwezig}: Lisette Appelo, Elly van Munster, Josien Willems
  \item {\bf Agenda}:
    \begin{enumerate}
       \item Opening en notulen
       \item Diversen
       \item Groepsleiders seminar
       \item PLUS en STEM
       \item Contacten met CE
       \item Voortgang CRE:
         \begin{enumerate} 
  	 \item presentatie
         \item PC-demo
         \item Woordvertaler
         \item Conjugator
         \item Rosetta 3D
         \end{enumerate}
       \item Rondvraag
    \end{enumerate}
\end{itemize}

\section {Opening en notulen}
De notulen van de vergaderingen op 12 februari en 26 februari worden 
met enkele wijzigingen aangenomen.

\section {Diversen}
\begin{enumerate}
   \item Wij hebben een floppy ontvangen met daarop een overzicht van ongeveer 
600 Language Industries in Europa. Jan L. zal deze voor ge\"{i}nteresseerden 
bij de PC neerleggen. 
   \item Kamer WB331 moet nog gedeeltelijk opgeknapt worden. Er zal nagedacht 
moeten worden over een multi-functionele inrichting, aangezien de kamer zowel 
als stagiaire kamer als als demo-kamer zal fungeren.
   \item Er zijn enige problemen ontstaan met betrekking tot de 
reiskostenvergoeding voor de Utrechters. Dhr. Camstra heeft op basis van het 
ambtenarenreglement geconcludeerd dat werknemers met een vaste standplaats 
buiten Utrecht slechts \'{e}\'{e}n jaar recht hebben op volledige 
reiskostenvergoeding. Jan L. zal proberen via de STT tot een vervangende 
regeling te komen.
   \item Elly wil dit jaar haar dienstverband beeindigen. Jan L. heeft 
Elena verzocht 4 dagen in de week te gaan werken.
   \item Ten aanzien van de samenwerking met CAP zijn er 2 problemen gerezen. 
Cap wil alleen participeren in gesubsidieerde projecten en ziet zichzelf niet 
als commerciele partner in het geheel.
\end{enumerate}

\section {Groepsleiders seminar}
Jan L. is begonnen aan een seminar voor groepsleiders (3 maal 2 dagen). 
Hierbij zijn onder andere de volgende punten aan bod gekomen :
\begin{itemize}
    \item Het streven van van der Klugt naar 4\% winst werd als volgt 
gemotiveerd :
    \newline
    \newline eigen vermogen 40\%
    \newline vreemd vermogen 60\%
    \newline
    \newline groei omzet = 8\% per jaar (4,5 miljard)
    \newline omloopsnelheid (omzet/vermogen) = 1,1
    \newline
    \newline Het vermogen moet dus met 4,1 miljard groeien.
    \newline 1640 miljoen (40\%) + 510 miljoen (dividend) = 2150 miljoen
    \newline Dit is precies 4\%.
    \item De raad van bestuur  heeft besloten dat 3/10 van de 
research onvoorwaardelijk gefinancierd wordt en dat de rest van de financiering 
uit overleg met de PD's moet komen. Dit hoeft dus niet 7/10 te zijn, maar kan 
ook meer of minder zijn. Gezien de winstcijfers van de verschillende PD's  
en de afname van het aantal PD's, lijkt inkrimping onvermijdelijk.
    \item Er werd ook een cursus "omgaan met creativiteit" gegeven. Wellicht 
gaan wij hier iets mee doen.
\end{itemize}

\section {PLUS en STEM}
STEM staat als vierde genoemd in een lijst van 5 projecten. De EEG beveelt 
samenwerking aan met MULTILEX, dat zich niet buitengewoon enthousiast toont.
PLUS staat bovenaan in de sector. Er moet echter gestreefd worden naar een 
reductie van 50\%. Lisette en Jan L. zijn in het engelse laboratorium op 
bezoek geweest. Het accent ligt hier vooral op pragmatiek , de taalkundige
component is minimaal. Als applicatie wordt gedacht aan de 'yellow pages' 
(gouden gids). 

\section {Contacten met CE}
Dhr. Strik , productmanager van de tekstgroep die o.a. de videowriter op de 
markt heeft gebracht, is geinteresseerd in toepassingen van Rosetta binnen een 
wordprocessor. De Smit Corona personal wordprocessor wordt te slecht bevonden 
en zij zijn van plan zelf iets beters maken. In principe zijn zij ook in vertalen 
geinteresseerd. Woensdag 18 april moet er een demonstreerbaar systeem zijn. Ook 
de PC-demo kan dan gedemonstreerd worden.

\section {CRE}

\subsection {Presentatie} 
Er zijn reeds schetsen voor het bord gemaakt. De teksten moeten nog geschreven 
worden.

\subsection {PC demo} 
Andr\'{e} heeft teksten geschreven en een gedeelte van de interactie bedacht. Frank 
vraagt of de zinsstructuur hierdoor erg veranderd is. Andr\'{e} belooft dat hij
hier naar zal kijken.

\subsection {Woordvertaler}
De interface is af, de woordenboeken moeten nog getest worden. De adjectieven 
zijn aangemaakt en getest, de werkwoorden zijn nog niet getest. Joep vindt het 
niet verstandig om met deze woordvertaler op de CRE te gaan staan. Hij vindt de 
applicatie minder leuk, daar het veel minder is dan een electronisch 
woordenboek (geen synomiemen) en niet veel meer dan de VD zelf. Bovendien is de 
kans op problemen erg groot. Jan O. merkt op dat de woordvertaler gepresenteerd 
wordt als een research project dat nog veranderingen kan ondergaan. Wel is de 
kwaliteit van het geleverde belangrijk. Jan L. stelt voor de teksten voor de CRE
zodanig te formuleren dat de aanwezigheid van LOCALROS optioneel is. Midden 
volgende week zal de beslissing moeten vallen.

\subsection {Conjugator}
De morfologie is nagenoeg af. Jan O. vraagt zich af of de snelheid misschien 
problemen oplevert. Er wordt besloten de nieuw surface parser opnieuw in te 
perken en de CLAUSE regels aan te passen. Ren\'{e} merkt op dat ook meerdere 
generaties vertraging opleveren.

\subsection {Rosetta 3D}
Jan O. meldt dat er vooral woordenboekwerk verricht is en er enkele regels 
geschreven zijn. Verder is er aan data voor het Nederlands, Spaans en Engels 
gewerkt en is er een checker geschreven die waardes voor attributen in 
verschillende talen vergelijkt. De engelse adjectieven zijn inmiddels 
aangemaakt en kunnen gevuld worden.

Jan O. vraagt of hij nog robuustheid in moet bouwen. Jan L. moedigt dit aan, 
maar denkt dat hiervoor niet genoeg tijd meer is.

Franciska is begonnen met het weggooien van zinloze betekenissen. Misschien 
heeft ze hier in de toekomst hulp bij nodig.

\section {Rondvraag}

Er waren geen vragen voor de rondvraag.

\end{document}
