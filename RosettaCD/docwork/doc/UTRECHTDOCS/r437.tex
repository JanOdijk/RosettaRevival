\documentstyle{Rosetta}
\begin{document}
   \RosTopic{General}
   \RosTitle{Notulen Groepsvergadering 28-5-1990}
   \RosAuthor{Petra de Wit}
   \RosDocNr{437}
   \RosDate{11-6-1990}
   \RosStatus{approved}
   \RosSupersedes{-}
   \RosDistribution{Project}
   \RosClearance{Project}
   \RosKeywords{minutes}
   \MakeRosTitle



\begin{itemize}
  \item {\bf aanwezig}: Lisette Appelo, Andr\'{e} Schenk, Jan Landsbergen, 
                     Franciska de Jong, Petra de Wit,  
                     Elena Pinillos, Joep Rous, Jan Odijk,
                     Ren\'{e} Leermakers, Frank Uittenbogaard.

  \item {\bf afwezig}: Elly van Munster, Harm Smit, Josien Willems
  \item {\bf Agenda}:
    \begin{enumerate}
       \item Opening en notulen
       \item CRE
       \item LEXIC, PLUS en STEM
       \item Diversen
       \item Home Office
       \item Venture
       \item Hoe Verder ?
       \item Rondvraag
    \end{enumerate}
\end{itemize}

\section {Opening en notulen}
De notulen van de vergadering op 12 april worden met een enkele wijziging 
aangenomen.

\section {CRE}
Jan L. dankt iedereen voor zijn/haar hulp tijdens en voor de CRE. De 
demonstratoren hebben een brief van Bulthuis ontvangen waarin zij
bedankt worden. De stand heeft een goede indruk gemaakt, zeker ook de spin-offs 
die lieten zien dat Rosetta op meerdere gebieden behulpzaam kan zijn.

\section {LEXIC, PLUS en STEM}

\subsection {LEXIC}
Dhr. Vercouteren, voormalig onderhandelaar voor Van Dale, is projectleider 
geworden. Monique Meijer zal als assistent-projectleider fungeren. Op de vraag 
van Franciska wat de achtergrond van dhr. Vercouteren is, antwoordt Jan L. dat 
hij een econoom is. Inmiddels zijn zij gestart met inventariserende 
werkzaamheden met gelden van de STT. Op 8 juni zullen zij 's middags een bezoek 
aan Rosetta brengen. 

\subsection {PLUS}
Jan L. is maandag en dinsdag op een bijeenkomst in Bologna geweest, met als 
doel het schrijven van een Draft Technical Annex. Op 15 juni is er een 
bijeenkomst met de EEG-vertegenwoordiger, dhr J. Folkmanis. 
De gewenste reductie is reeds gerealiseerd zodat er in principe geen obstakels 
meer zijn. De Philips bijdrage 
ligt in de hoek van de pragmatiek (Dialogue-Manager en Dialogue-Grammar).
In september verwacht men te beginnen.

\subsection {STEM}
Er komt een overleg tussen MULTILEX en STEM over een eventuele fusie.
\section {Diversen}
\begin{itemize}
\begin{enumerate}
\item Vrijdagmiddag 29 juni zal een groep Utrechtse studenten 
computerlinguistiek Rosetta bezoeken.
\item Volgend jaar zal er weer een europese ACL plaatsvinden. Papers moeten 
voor september ingediend worden.
\item Het programma GRAMMATIK, een syntax checker voor het engels, staat op de 
harde schijf van de PC.
\item CAP heeft wederom bevestigd dat het geen projecten zonder 'funding' wil 
beginnen.
\item Vrijdagochtend om 10.00 uur houdt Josien een proefpraatje over een I/O 
module onder UNIX. Na afloop zal 
er taart zijn.
\end{enumerate}
\end{itemize}

\section {Home Office}
De contacten met Home Office zijn weer een stapje verder. Een spellingschecker
lijkt als begin-product geschikt, aangezien hierbij op korte termijn iets terug 
te verdienen valt. 
Waumans wil 
ook hoger binnen CE bekendheid geven aan Rosetta en haar producten.
Op de vraag van Lisette of Home Office ook in "meer" geinteresseerd is, 
antwoordt Jan L. dat ze in principe in alles geinteresseerd zijn, als het maar 
niet teveel kost. De woordenboeken die voor de checker gemaakt worden zouden 
bijvoorbeeld bij andere produkten van nut kunnen zijn. Jan L. lijkt dit 
traject, waarbij men van een klein produkt naar een groot produkt gaat, een 
goed traject.

\section {Venture}
Dhr. Vriezendorp van het Philips Venture Fund heeft  verteld dat goede 
ondernemers over 3 zaken moeten beschikken: Een product, een markt en 
een goed team dat bereid is 24 uur per dag te werken. 
Bovendien moet een deel van 
het kapitaal uit het team zelf komen. Dit alles houdt in dat er binnen 2 jaar 
een 
product op de markt moet verschijnen. Jan O. ziet niet dat wij binnen twee jaar een interactief 
vertaalsysteem ontwikkelen als product dat op de markt gebracht moet worden. 

\section {Hoe Verder}
Jan L. wil in deze vergadering een open gesprek houden over welke ombuigingen 
er eventueel in de werkzaamheden plaats moeten vinden en wat wij beogen over 
twee jaar te kunnen maken. Nu hebben we nog geen product waarmee we indruk 
kunnen maken. In de loop van juni moet er een concreet plan gemaakt worden.
Lisette merkt op dat het originele plan was om na de CRE aan het spaans te gaan 
werken. Ook zou het engels bijgewerkt kunnen worden naar het niveau van het 
nederlands. Lisette stelt een andere werkwijze voor, waarbij mensen niet zozeer 
per taal als wel per onderwerp werken. Uit ervaring meent zij dat dit 
efficienter werkt. Jan L. merkt op dat ook het LEXIC project stimulerend zal 
werken op het spaans. Franciska stelt voor een 
tijdje met de methode te experimenteren.\\

\noindent Jan O. stelt dat er nu een beslissing moet vallen over wat we eigenlijk 
willen. Moet de nadruk gelegd worden op fundamentele problemen (ellipsis, 
coordinatie,interactie) of op aantallen constructies.
Het eerste is niet zo zichtbaar voor de buitenwereld. Lisette stelt voor om aan 
gebruiksaanwijzingen te gaan werken. Jan O. merkt op dat als wij met zulke 
teksten met minder moeite overtuigend over kunnen komen, dit het overwegen 
waard is.\\

\noindent Jan L. meldt dat op 14 juni mevr Den Hommel van het 
vertaalbureau op bezoek komt.\\

\noindent Jan O. stelt voor de omkeerbaarheid van het systeem te testen door 
aan engelse analyse te gaan werken. Hieraan is een woordenboek probleem 
verbonden. Ren\'{e} merkt op dat hij vertaling van nederlands naar spaans 
en engels al overtuigend genoeg vindt.\\

\noindent Frank merkt op dat een systeem E-N met interactie in de doeltaal veel 
interessante toepassingen oplevert. Een voordeel is dat de 
kwaliteit van de vertaling minder 
hoog hoeft te zijn. Men kan hierbij denken aan toepassingen in het 
vakantiecircuit.\\

\noindent Frank vraagt, denkend aan PLUS, of we de komende jaren alleen met vertalen 
bezig gaan. Jan L. wil niet te veel aandacht besteden aan andere applicaties. 
Misschien moet er wel nagedacht worden over niet geheel isomorfe grammatica's, 
maar grammatica's met taalpaarafhankelijke gedeeltes (zoals nu bij sommige 
regels al het geval is). Jan L. vraagt aan de software-specialisten of zij
 reeds nagedacht hebben over een systeem met taalpaarafhankelijke gedeeltes
. Zij antwoorden dat zij dit nog niet gedaan hebben.\\

\noindent Jan O. deelt een lijstje uit met constructies waar in de verschillende talen 
aan gewerkt moet worden. Jan L. merkt hierbij op dat het onderzoek op het 
gebied van de semantische component gewoon door gaat.\\

\noindent Afsluitend stelt Lisette voor een deadline te stellen voor de inhaalactie 
spaans.\\

\section {Rondvraag}

Er waren geen vragen voor de rondvraag.\\

\noindent De volgende vergaderingen zullen op \underline {11 en 25 juni} 
plaats vinden. 

\end{document}
