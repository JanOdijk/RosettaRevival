


\documentstyle{Rosetta}
\begin{document}
   \RosTopic{Rosetta3.doc.linguistics.Dutch}
   \RosTitle{Dutch M-rules:subgrammar QPformation}
   \RosAuthor{Franciska de Jong}
%Lisette Appelo}
   \RosDocNr{483}
   \RosDate{December 4, 1991}
   \RosStatus{concept}
   \RosSupersedes{}
%concept of September 4, 1989}
   \RosDistribution{Project}
   \RosClearance{Project}
   \RosKeywords{Dutch, M-rules, QPformation}
   \MakeRosTitle
%
%
\input{[dejong.mrules]mrudocdef}
\input{[dejong]definitions}

\section{Introduction}
By this subgrammar QPs  are derived 
that occur in 
contexts 
non-predicative contexts:\\

(i)   Hij heeft genoeg boeken\\
(ii)  Hij heeft boeken genoeg/zat\\
(iii) Hij heeft genoeg boeken om een bibliotheek te beginnen\\
(iv)  Piet is minder lang (dan Jan) \\

Cf. also remarks to SUBGRAMMAR QPtoQPPROP

\section{Subgrammar Specification}

\begin{verbatim}
HEAD: Q

IMPORT:
<
Q                        BASIC EXPRESSION
BINDEFPRO                BASIC EXPRESSION
THANP
ASP
VARTHANP
VARASP
SENTENCE
>

EXPORT: QP
\end{verbatim}

\section{Control Expression}

The control expression is defined as follows:
\begin{verbatim}

 (RQTOQP1/1 | RQTOQP2/2 
%(* | RQTOQP3/3 *) )

. RQPsuperdeixis/12

. [RQPCOMPARATIVEcomplmod/5 | RQPSUPERLATIVEcomplmod/6 | RQPomcomplmod/7]

. {TQPNOMCaseAssignment/10 | TQPVANCOMPLCaseAssignment/11}

. [ RQPvoorobjmod/4 ]

. [RQPamountmod/8]
(* planned: afhandeling van complementen *)

. {RQPSubstitution1/9 | RQPSubstitution2/12} 
\end{verbatim}


\section{Planned}

A subgramar QPtoQPPROP is planned, containging as a stratrule 
RSTARTQPPROP1,  taking as import a QP thta is yielded by subgrammar 
QPFORMATION. 
This subgrammar is needed in order to account for the 
predicative use of QPs as in :\\

(i)  Nu is het genoeg\\
(ii) Dit is meer dan Piet kan velen \\


It is presumed that QP is the import category.
If there turn out to be arguments 
favouring a derivation  QPPPROP isomorphic to the derivation of other XPPROP 
categoriesd both this subgrammar and QPformation should be adapted.\\


\begin{mruleclass}{QPformation}
\begin{classdescr}
\kind \nokind
\classtask formation of a QP level
\classremarks

\nofilters

\nospeedrules

\noplannedrules

\norulesnotince

\rulelist

\end{classdescr}

\begin{members}
\begin{member}
\rulename RQTOQP1 
\ruletask formation of a QP level
\file dutch:qprules.mrule (mrules45)
\semantics \nosemantics
\example meer, minder, genoeg, veel, iets, meeste, meest
\remarks\mbox{}
\begin{enumerate}
\item problems:\\
om uit te zorgen dat `meest' (niet meeste) ook van Q tot QP wordt:
met toevoeging van `het'
zie RQTOQP2; DETP de/het meeste loopt via
RQTOQP1 en DETformation3a.
\item modifications:\\

\end{enumerate}

\end{member}
\begin{member}
\rulename  RQTOQP2
\ruletask formation of a QP level for 'meest', with introduction of 'het'
\file dutch:qprules.mrule (mrules45)
\semantics \nosemantics
\example {\em meest} (meestQkey) $\rightarrow$ {\em het meest}
\remarks\mbox{}
\begin{enumerate}
\item problems:\\

\item modifications:\\

\end{enumerate}

\end{member}
\begin{member}
\rulename RQPCOMPARATIVEcomplmod
\ruletask Introduction of a comparative complement to {\em meer} or
{\em minder}.
\file dutch:qprules.mrule (mrules45)
\semantics \nosemantics
\example minder dan Piet, meer dan vroeger
\remarks\mbox{}
\begin{enumerate}
\item problems: Probably it might be necesary to have an attribute for QP 
indicating the syntactic category of T1 under complrel. In earlier versions of
the domain
the attribute thanascompl in ASPrecord and THANPrecord was supposed to
keep track of this information. According to a domain modification in august 87
this attribute has a new dunction in QP ( and accordingly a new value set): it
indicates the kind of complement a certain quantifier or adverb inherently has.
\\ 
\item modifications: 26/10/88: Added superdeixis to XP category subrules.\\
\item Case is dealt with by TQPNomCaseAssignment
\end{enumerate}

\end{member}
\begin{member}
\rulename RQPSUPERLATIVEcomplmod
\ruletask Introduction of a superlative complement to {\em meest}.
\file dutch:qprules.mrule (mrules45)
\semantics \nosemantics
\example het meest .. van allemaal
\remarks\mbox{}
\begin{enumerate}
\item problems: see RQPCOMPARATIVEcomplmod

\item modifications: 26/10/88: Added superdeixis to XP subrules\\
\item Case is dealt with by TQPVANCOMPLCaseAssignment.
\end{enumerate}

\end{member}
\begin{member}
\rulename RQPmod
\ruletask To introduce a modifier to a QP.
\file dutch:qprules.mrule (mrules45)
\semantics \nosemantics
\example ongeveer genoeg, bijna meer
\remarks\mbox{}
\begin{enumerate}
\item problems:\\

\item modifications:\\

\end{enumerate}

\end{member}
\begin{member}
\rulename RQPamountmod
\ruletask Introduction of amount modifiers to a QP.
\file dutch:qprules.mrule (mrules45)
\semantics \nosemantics
\example een meter minder lang, iets verder weg, 
veel meer op zijn gemak
\remarks\mbox{}
\begin{enumerate}
\item problems:\\

\item modifications:\\

\end{enumerate}

\end{member}
\begin{member}
\rulename RQPCOMPARATIVEcomplmod
\ruletask Introduction of a infinitival om-te-modifier to a QP.
\file dutch:qprules.mrule (mrules45)
\semantics \nosemantics
\example genoeg om te komen (geinteresseerd genoeg om te 
komen);
genoeg om te kopen (mooi genoeg om te kopen).
\remarks\mbox{}
\begin{enumerate}
\item problems: 
\\ 

\end{enumerate}

\end{member}
\begin{member}
\rulename TQPNOMCASEAssignment
\ruletask Analysis: Reducing the set for NPVAR/NP-attribute {\em cases} 
from a set including nominative to
[Nominative].
\file dutch:qprules.mrule (mrules45)
\semantics \nosemantics
\example meer dan ik
\remarks\mbox{}
\begin{enumerate}
\item problems:\\

\item modifications:\\
\item Transformation applies in analysis only. In generation 
it would be applicable to its own results. Besides it would yield extra paths 
in e.g. QPsubstitution1.
\end{enumerate}

\end{member}
\begin{member}
\rulename TQPVANCOMPLCASEAssignment
\ruletask
\file dutch:qprules.mrule (mrules45)
\semantics \nosemantics
\example
\remarks\mbox{}
\begin{enumerate}
\item problems:\\

\item modifications:\\

\end{enumerate}

\end{member}
\begin{member}
\rulename RQPVOORobjMOD
\ruletask Introduction of a voorobj-modifier into a QP.
\file dutch:qprules.mrule (mrules45)
\semantics \nosemantics
\example voor mij genoeg (lang genoeg voor mij)
\remarks\mbox{}
\begin{enumerate}
\item problems:\\

\item modifications:\\

\end{enumerate}

\end{member}
\begin{member}
\rulename RQPsuperdeixis
\ruletask Assign the superdeixis value of the parameter to the QP node 
in generation and give the the parameter the value of the superdeixis attribute 
of the QP node  in analysis.
\file dutch:qprules.mrule (mrules45)
\semantics \nosemantics
\example
\remarks\mbox{}
\begin{enumerate}
\item problems:\\

\item modifications:\\

\end{enumerate}

\end{member}
\begin{member}
\rulename RQPSubstitution1
\ruletask Substitution of NPs that function as the complement of a THANAS.
\file dutch:qprules.mrule (mrules45)
\semantics \nosemantics
\example
\remarks\mbox{}
\begin{enumerate}
\item problems:\\ As in generation this rule applies before all other 
substitution rules the substituent will have narrow scope.

\item modifications:
24/10/88: Superdeixis conditions and assignments added. \\

\end{enumerate}

\end{member}
\begin{member}
\rulename RQPSubstitution2
\ruletask Substitution of NPs in VOORobj-modifiers.
\file dutch:qprules.mrule (mrules45)
\semantics \nosemantics
\example voor mij (lang) genoeg
\remarks\mbox{}
\begin{enumerate}
\item problems:\\ As in generation this rule applies before all other 
substitution rules the substituent will have narrow scope.

\item modifications:

\end{enumerate}

\end{member}
\begin{member}
\rulename TQPsuperdeixisadaptation1
\ruletask Check/assign superdeixis values.
\file dutch:qprules.mrule (mrules45)
\semantics \nosemantics
\example
\remarks\mbox{}
\begin{enumerate}
\item problems:\\

\item modifications:\\

\end{enumerate}

\end{member}
\begin{member}
\rulename TQPsuperdeixisadaptation2
\ruletask Check/assign superdeixis values.
\file dutch:qprules.mrule (mrules45)
\semantics \nosemantics
\example
\remarks\mbox{}
\begin{enumerate}
\item problems:\\

\item modifications:\\

\end{enumerate}

\end{member}
\begin{member}
\rulename TnoQPsuperdeixisadaptation
\ruletask Default rule for check/assignment superdeixis values.
\file dutch:qprules.mrule (mrules45)
\semantics \nosemantics
\example
\remarks\mbox{}
\begin{enumerate}
\item problems:\\

\item modifications:\\

\end{enumerate}

\end{member}
\end{members}

\end{mruleclass}

\end{document}
