\documentstyle{Rosetta}
\begin{document}
   \RosTopic{General}
   \RosTitle{Notulen Vergadering over R3D en het boek; 26-2-1990}
   \RosAuthor{Harm Smit}
   \RosDocNr{429}
   \RosDate{\today}
   \RosStatus{approved}
   \RosSupersedes{-}
   \RosDistribution{Project}
   \RosClearance{Project}
   \RosKeywords{minutes}
   \MakeRosTitle
\begin{itemize}
  \item {\bf aanwezig}: Andr\'{e} Schenk, Jan Landsbergen, Lisette Appelo,
                     Franciska de Jong, Petra de Wit, Elly van Munster, 
                     Elena Pinillos, Joep Rous, Ren\'{e} Leermakers,
                     Jan Odijk, Harm Smit.
\end{itemize}

\section{R3D en CRE}

Jan L. noteert drie vragen m.b.t. R3d en CRE:
\begin{enumerate}
  \item is het brievencorpus vertaalbaar op de CRE?
  \item zijn brieven over hotelreserveringen in het algemeen vertaalbaar op 
        de CRE?
  \item is interactief vertalen van brieven in het algemeen haalbaar binnen 
        drie \`{a} vier jaar?
\end{enumerate}

N.a.v. punt 1 wordt opgemerkt dat we dit `bijna' kunnen; 
de grootste problemen zijn:
\begin{itemize}
  \item de lengte van de zinnen in de brieven is groter dan wat bij testen 
        gebruikelijk is,
  \item we moeten nu met grote en realistischer woordenboeken gaan werken
        (hetgeen nieuwe ambigu\"{\i}teiten oplevert).
        
\end{itemize}
M.b.t. punt 2 kunnen we stellen dat het antwoord `nee' zal luiden. Elke nieuwe
brief zal weer nieuwe problemen aan het licht brengen en dat zal, ook als deze 
slechts triviaal van aard zijn, aanpassingen in het systeem vergen.

Over punt 3 vari\"{e}ren de meningen. Men is het er wel over eens dat dit een 
enorme klus is, misschien teveel, zelfs bij forse uitbreiding van aantal mensen.
De "interactie" lijkt een groot probleem te zijn, het aantal vragen kan veel te 
groot worden. Enkelen zien nog mogelijkheden als het aantal vragen beperkt kan 
worden, bijvoorbeeld door een domein te kiezen of sommige vragen 
in bepaalde situaties `uit te zetten'. Jan L. denkt dat R3D een mooie 
(domein)inperking is om het in punt 3 gestelde te testen. 
Waarschijnlijk geeft de DEMO op de PC een te gunstig beeld. Ook geeft de DEMO 
geen inzicht in het voorkomen van constructies door users die wij niet 
aankunnen.

Joep ziet meer in de combinatie van de huidige grammatica met RosBuiz (naar
aanleiding van een stuk van Ren\'{e}). 
Jan L. merkt op dat er ook in dat geval veel
aan onze grammatica zal moeten worden verbeterd. Bovendien denkt hij dat mensen
toch ook andere zinnen of constructies willen kunnen gebruiken. Ren\'{e} 
zegt dat
we zowiezo zoiets als in RosBuiz is voorgesteld moeten hebben omdat mensen 
anders bij kleine wijzigingen in hun tekst steeds weer alle vragen opnieuw
moeten beantwoorden.

De vraagt rijst of we \"{u}berhaupt wel op de CRE moeten staan. Er zijn 
echter door enkelen alternatieven bedacht. Het gaat om:
\begin{itemize}
  \item Het electronisch woordenboek dat verbogen vormen kent. Een gebruiker
        kan tijdens het typen van een Engelse tekst een Nederlandse woordvorm 
        invoeren en hiervan de vertalingen opvragen. Het gaat hier in feite om
        een variant van "localros", met een groot woordenboek en het talenpaar
        N-E.
  \item De "vervoegmachine", wellicht het interessantst voor het talenpaar N-S,
        waar een werkwoordsvorm van de ene taal in de andere wordt omgezet.
        Het mooiste zou zijn als er ook nog wat extra mogelijkheden zouden zijn
        bij het intypen zoals negatie, vragende vormen, clitics e.d.
\end{itemize}

Deze twee dingen zouden als "spin-off" van het gewone systeem gepresenteerd 
kunnen worden. 

De vergadering blijkt enthousiast te zijn voor de nieuwe plannen; het lijkt
ook een goede zaak te zijn dat we in elk geval wel op de CRE zijn en het is 
wellicht ook goed er met wat meer demonstratie materiaal te zijn. Met name 
C.E. zou in deze spin-offs ge\"{\i}nteresseerd kunnen zijn.

De vraag is natuurlijk of we dit op korte termijn nog wel kunnen halen.
Daarover vergaderen we vrijdagmiddag. Deze week moet een en ander dus bekeken
worden.

\section{Boek}

Over het boek is opgemerkt:
 \begin{enumerate}
  \item H. 1, P. 1: het wordt voor een breed publiek; de doorsnee van de kennis
        van de genoemde personen lijkt een leuk uitgangspunt.
  \item H. 1, P. 2: Jan O. merkt op dat het handig is dingen uit te leggen aan
        de hand van (het in het boek `maken' van) 
        een `mini-vertaalsysteem' waar simpele regels in voor komen.
  \item H. 1: het genoemde aantal pagina's is wel wat veel; Jan L. denkt aan 
        ongeveer 250, dus circa 12 per hoofdstuk.
  \item H. 2: Compositioneel vertalen lijkt een goede rode draad. Maar de 
        nadruk moet niet teveel gaan liggen op compositionaliteit in een ruimer
        verband!
  \item H. 3, P. 4:De taalkundige inleiding is een probleem: deze moet er wel
        in (en uitgebreider dan Theo voor ogen staat) maar waar? Een dergelijke
        inleiding is belangrijk om grammaticale begrippen (zoals de 
        categorie\"{e}n e.d.) uit te leggen en vooral ook om duidelijk te maken
        wat boomstructuren zijn. 
   
        Een ander probleem is of we dingen generatief of analytisch uitleggen. 
        Theo oppert het laatste maar waarschijnlijk maakt dat het uitleggen
        erg ingewikkeld. 

        Ergens moet ook beargumenteerd worden waarom wij bepaalde keuzes hebben
        gemaakt zoals het hebben van een morfologie en een surface-grammatica.

        Ad hoofdstuk 6: het is goed dat de lezer beseft dat er steeds veel paden
        zijn (en vaak veel toepasbare regels enz.); misschien is dit al te 
        illustreren aan het `mini-systeempje'. De algebra\"{\i}sche consequenties
        horen elders (in een van de laatste hoofdstukken thuis.

        Ad hoofdstuk 7: variabelen behoeven wel een uitvoerige toelichting.
        Wellicht zijn pronomina een goede mogelijkheid om variabelen uit te
        leggen en te beargumenteren.

        Hoofdstuk 8 en 9 kunnen samen.

        Hoofdstuk 10 is problematisch omdat de kennis uit hoofdstuk 12 tot en 
        met 17 daarvoor nodig is (of kan zijn).
       
        Hoofdstuk 15: misschien is een aanpak mogelijk als in het artikel van 
        Jan O., Jan L. en Andr\'{e}.

        Hoofdstuk 17: wat hier staat is volgens Ren\'{e} niet helemaal waar.
        Verder zal de surface parser al eerder aan de orde zijn.

        Hoofdstukken 18 en 19 als mathematische achteraan lijkt okay.

        Hoofdstuk 20: misschien moet er ook ergens nog een vergelijking komen
        tussen wat wij kunnen en `echte' tekst, en over onze inzichten 
        hieromtrent. 

        Als alternatieve indeling kunnen we overwegen het boek in te delen in
        3 delen:

        {\bf I Algemeen}, met de algemene zaken zoals o.a.:
        \begin{itemize}
          \item machinaal vertalen
          \item taalkundige inleiding
          \item compositionaliteit
        \end{itemize}

        {\bf II Rosetta}

        {\bf III Algebra\"{\i}sche zaken}, met hoofdstuk 18 en 19 en 
        de formelere kant van hoofdstuk 17.

        Eventueel hoofdstuk 20 in een afsluitend deel {\bf IV}.
       
        Hoeveel tijd kost ons dit boek: Jan L. heeft er nog niet met 
        Waumans over kunnen spreken, maar als het 1 maand per persoon kost is 
        het waarschijnlijk geen probleem.
        Ren\'{e} denkt dat het veel meer tijd zal
        kosten. Theo zal echter waarschijnlijk de redactie op zich nemen, wat
        een deel van de tijd zal zijn.

        Onder welke auteur wordt het boek gepubliceerd? Er zijn twee opties:
        "Dr. Rosetta Stone" of onder de namen van Jan en Theo (waarbij nog
        de vraag rijst wie er als eerste genoemd wordt). Theo heeft voor
        zover bekend een voorkeur voor de laatste optie. Wij hebben
        (na discussie) voorkeur voor "Dr. Rosetta Stone".
        
\end {enumerate}
\end{document}
