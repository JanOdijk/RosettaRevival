\documentstyle{Rosetta}
\setlength{\parindent}{0in}
\begin{document}
   \RosTopic{linguistics.spanish}
   \RosTitle{Spanish NPs and ADJPs: an inventarisation}
   \RosAuthor{Franciska de Jong}
   \RosDocNr{450}
   \RosDate{23-11-1990}
   \RosStatus{informal}
   \RosSupersedes{-}
   \RosDistribution{Project}
   \RosClearance{Project}
   \RosKeywords{ADJP, NP, QP, DETP, Spanish}
   \MakeRosTitle
%
%


This note contains an overview of what is presently 
included in the Spanish subgrammars for NPs and adjectival constructions.
It is meant to describe the degree in which Spanish is trailing behind.
Topics that are not dealt with in Dutch and/or English, e.g. proobjs, 
genericity 
are not mentioned here.

\section{ADJPs}

For the following phenomena/aspects the relevant 
rule classes have been implemeted and tested:
\begin{enumerate}
  \item startrules
  \item patternrules
  \item degree-modifcation 
  \item extraposition of degree-complements
  \item deixis-superdeixis-aspect
  \item substitution and case-assignment
  \item mood
  \item number-gender agreement 
\end{enumerate}
\mbox{}\\

Rules for the following topics have not been written yet:
\begin{enumerate}
  \item {\bf control}. The treatment of control should be identical 
to the treatment of control in clausal structures. 
But in addition to the various kinds of control for clausal structures, there 
are the kinds of control that -at least in Dutch- occur in the context
of adjectives only, namely control in combination with a pro-object 
(dutch: dit mes is geschikt om [mee PRO] te snijden), and 
control in the context of degree-modifiers (dit boek is te zwaar voor mij om te 
dragen). Cf. the remarks in the documentation on (dutch) patterns and on 
subgrammars ADJPPROPformation/ADJPPROPtoADJPFORMULA.
  \item {\bf non-degree modifcation}. Planned rules: 
RADJPARAOBJMOD, RADJMOD1, RADJPARAMOD1

\item {\bf extraposition}, other than for complements to THANPs, if such rules 
are needed at all.

\item {\bf movement} if such rules are needed at all.

\item {\bf reflexivity/reciprocals}
\end{enumerate}
\mbox{}\\

Rules to be tested/improved:
\begin{enumerate}
  \item rules for ADJPs in clausal contexts
  \item rules for sentential arguments
\end{enumerate}

\section{NPs}

For the following phenomena/aspects the relevant 
rule classes have been implemeted and tested:
\begin{enumerate}
  \item derivation
  \item number
  \item modification of CN (including relativisation)
  \item argumentlike modification and argument substitution
  \item possessive relations, for example {\em el buey de Juan, mi libro}
  \item partitive constructions
  \item superdeixis
  \item pronouns/proper names
  \item empty nominal heads
  \item apocope and glue  
\end{enumerate}

Rules for the following topics have not been written yet:
\begin{enumerate}
  \item gluerules for syncategorematically introduced PREPPs
  \item extraposition of DETP-complements
  \item modification of the NP-level (non-restrictive modification and 
        modification of non-CN heads).
\end{enumerate}

\section{DETPs}
For the following phenomena/aspects the relevant 
rule classes have been implemeted and tested:
\begin{enumerate}
  \item DETPs heading DETs, NUMs, QPs and DEMADJ's
  \item partitive DETP's
  \item superdeixis
\end{enumerate}
\mbox{}\\

Rules for the following topics have not been written yet:
\begin{enumerate}
  \item modification of determiners
\end{enumerate}

\section{QPs}
For the following phenomena/aspects the relevant 
rule classes have been implemeted and tested:
\begin{enumerate}
  \item superdeixis
  \item modification with complements triggered by the Q
  \item case assignment and substitution in complements
\end{enumerate}
\mbox{}\\

Rules for the following topics have not been written yet:
\begin{enumerate}
\item {\em para}-complements 
\item degree modification within QPs
\item adverbs occurring in QP-complements
\end{enumerate}

\section{Dictionaries}
\begin{description}
  \item[closed categories]
The dictionaries for the relevant closed categories are up to date.
  \item[nouns]

As argument structure for nouns
is always 
optional, it was possible to add patterns etc., without splitting the 
entries. 
But in order to 
get rid of a lot of 'no generation'-messages the 
noun-dictionaries should be submitted to a splitting action, especially for 
the attributes pertaining to argument structure. 

The filling of the values for {\em posscomas} must be checked.

\item [adjs]
The dictionaries for adjectives and adverbs have not been adapted yet to
the modifications with respect to argumentstructure of last spring (CRE).

Note that 
these dictionaries may still contain the garbage that originally occurred in 
the dutch dict-files as well: irregular forms, expressions that in Rosetta
have a special category status, etc. As the dutch equivalents have been 
removed, nothing needs to be done as long as there is no spanish analysis.



\end{description}
\end{document}
