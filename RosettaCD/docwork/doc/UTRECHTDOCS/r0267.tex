
\documentstyle{Rosetta}
\begin{document}
   \RosTopic{General}
   \RosTitle{Notulen Linguistenvergadering 19-4-88 en 26-4-88}
   \RosAuthor{Margreet Sanders}
   \RosDocNr{0267}
   \RosDate{\today}
   \RosStatus{informal}
   \RosSupersedes{-}
   \RosDistribution{Linguists, Joep Rous}
   \RosClearance{Project}
   \RosKeywords{minutes, shift, topicalisatie, relativisatie, WITH, expl.ER, 
                case assignment, filters}
   \MakeRosTitle
%
%

\noindent
{\bf Verslag linguistenvergadering 19 april 1988}
\begin{description}
\item[Aanwezig:] Jan Odijk, Lisette Appelo, Elly van Munster, Harm Smit, 
Andr\'{e} Schenk, Franciska de Jong, Margreet Sanders (not)
\item[Afwezig:] --
\item[Agenda:] \mbox{}
\begin{enumerate}
\item Uitbreiding M-regelnotatie: WITH
\item Shiftregels voor relativisatie en topicalisatie
\item Shiftregels en substitutie voor PREPPs
\item Filters
\end{enumerate}
\end{description}

\section{WITH}
De M-regelnotatie is weer uitgebreid met een nieuwe mogelijkheid. In de 
matchcondities kan een restrictie gelegd worden op een attribuut van de
variable T1, d.m.v.\ de term WITH:\\
T1.CAT IN [...] WITH T1.REC.attribute = value\\
T1.CAT IN [...] WITH value IN T1.REC.attribute, etc.\\
De compiler kan alleen gevallen aan waarin T1.CAT element is van een set, 
dus niet: T1.CAT = .... De conditie die achter WITH volgt moet enkelvoudig 
zijn, dus niet WITH (a = b AND c IN d) of zoiets.

\section{Shiftregels voor relativisatie en topicalisatie}
Topicalisatie en relativisatie gebeurt in twee stappen: eerst wordt de 
relevante VAR verplaatst, daarna wordt er voor de betreffende VAR 
gesubstitueerd. Beide stappen worden uitgevoerd door betekenisvolle regels: 
substitutie is een regel omdat de index van de betreffende variabele meegegeven 
moet worden, en topicalisatie-shift is ook vertaalrelevant. Alleen de betekenis 
van relativisatie-shift ligt minder voor de hand. Deze shift wordt toch een 
regel genoemd, omdat ook hier de index van de geshifte variabele als een 
parameter wordt meegegeven, omdat je in gevallen van vertaling in weer een 
shift voor effici\"{e}ntie graag wilt weten welke variabele er geshift moet 
gaan worden. Nadeel van deze aanpak is wel dat in alle subgrammatica's die 
isomorf aan de clause-grammatica zijn (ADJP, PREPP, ADVP) ook een dergelijke 
regel moet voorkomen, hoewel die daar loos is. Een beslissing of zo'n 
uitsmering van de semantiek van relativisatie over twee ver van elkaar 
verwijderde regels wel handig is, wordt tot volgende week uitgesteld.

\section{Shiftregels en substitutie voor PREPPVARs}
Locatieve en temporele relatieve adverbiale PREPPs ({\em Het moment waarop hij 
binnenkwam, 
de flat waaronder hij werkt\/}) beginnen als PREPPVAR. Deze VAR wordt in de 
relativisatie-shift naar voren verplaatst. Hierbij wordt de index {\em i\/} van 
de 
PREPPVAR als parameter aan de regel meegegeven, vanwege effici\"{e}ntie bij het 
vertalen in een shiftrel/.. van de doeltaal (zie vorige punt). In de 
substitutieregels wordt 
dan de 
PREPPVAR vervangen door een VARPREPP met een PREP en een CNVAR. Deze CNVAR 
heeft een index {\em j\/}. In de NP-grammatica kan daarna substitutie van de 
CNVARs plaatsvinden, maar het bijzondere is dat de variabele die 
gerelativiseerd wordt (CNVAR$_{j}$: {\em moment\/}) niet de variabele is 
waarvan de index aan de Relshift was meegegeven. Dat was namelijk de PREPPVAR 
met index {\em i\/}. In derivatie:\\[5 mm]
\begin{tabular}{ll}
              &   $_{S}$[ hij PREPPVAR$_{i}$ binnenkwam ] \\
Relshift(i):  &   $_{S}$[ shift/PREPPVAR$_{i}$ hij binnenkwam ] \\
Subst(i):     &   $_{S}$[ shift/VARPREPP[...CNVAR$_{j}$...] hij binnenkwam ] \\
CN-Subst:     &   $_{CN}$[ head/NOUN(moment), \\
              & \ \ \ \ \ \ mod/S [ shift/PREPP [...RELPRO(waar)...] hij 
                                                                binnenkwam ]]
\end{tabular}
\\[5 mm]
Verder moet het Nederlands een speciale regel hebben die bij tempadvrel/PREPPs 
specifieke PREPs kan deleren, om zinnen aan te kunnen als {\em De dag dat hij 
vertrok\/} 
(vs.\ het alternatief {\em De dag waarop hij vertrok\/}).

\section{Filters}
Bij `verplichte' optionele transformatieklassen moet er een
filter na de klasse zijn (generatief gesproken) om ongewenste vormen tegen te
houden. Dit filter moet zowel analytisch als generatief werken. Voor
niet-itererende transformatieklassen is echter een formulering met een
verplichte klasse van de vorm ( T1 $|$ T2 ), waarbij de inputmodellen van de
twee transformaties elkaars complement zijn, het meest effici\"{e}nt. Als dit
moeilijk te realiseren is, of als de klasse iteratief moet zijn, is het
alternatief om, indien mogelijk, ook een filter te defini\"{e}ren v\'{o}\'{o}r
de transformaties, dus F1.[ T ].F2. Dit scheelt soms 
veel uiteindelijk toch doodlopende paden. Ook dit filter kan het best zowel
analytisch als generatief werken. 

\newpage
\noindent
{\bf Verslag linguistenvergadering 25 april 1988}
\begin{description}
\item[Aanwezig:] Jan Odijk, Lisette Appelo, Elly van Munster, Harm Smit, 
Andr\'{e} Schenk, Franciska de Jong, Margreet Sanders (not)
\item[Afwezig:] --
\item[Agenda:] \mbox{}
\begin{enumerate}
\item Nogmaals: Shift en Substitutie bij Relativisatie en Topicalisatie
\item Case Assignment
\item Expletief ER
\end{enumerate}
\end{description}

\setcounter{section}{0}
\section{Nogmaals: Shift en Substitutie bij Relativisatie en Topicalisatie}
Ten eerste een aanvulling op het verhaal van de vorige keer: de 
substitutieregels voor een VAR die in de relativisatie-regels is geshift zitten 
in de NP-grammatica, die voor een getopicaliseerde VAR in de clause-grammatica. 
Verder lijken de shiftregels voor beide gevallen wel veel op elkaar, maar zijn 
toch niet helemaal hetzelfde: een subject mag namelijk niet topicaliseren, maar 
wel relativiseren. Hierdoor is niet mogelijk deze regels samen te klappen.

Een probleem van het voorstel dat vorige keer werd genoemd is dat het mogelijk 
is om een relativisatie-shift uit te voeren, maar vervolgens toch al in de 
clause-grammatica ook te substitu\"{e}ren (eigenlijk behorend bij 
topicalisatie). Het voor\-stel wordt daarom als volgt uitgebreid: maak voor 
relativisatie een betekenisvolle regel, die alleen de betreffende CNVAR 
markeert voor `+relative' (NB: weer een geval van toekenning van een 
attribuut-waarde aan een variable!). Daarna volgen shift-transformaties voor 
`+relative' 
VARs die de eigenlijke verplaatsing uitvoeren. Voor topicalisatie blijven nog 
gewoon vergelijkbare shift-regels bestaan, die nu alleen werken op `--relative' 
VARs. Ook de substitutieregels in de clause werken dan alleen nog op 
`--relative' CNVARs. Hiermee is relativisatie dus uitgebreid tot een proces in 
drie stappen: markering(R), shift(T), substitutie(R). Eventueel kan de 
markering ook 
plaatsvinden in de relativisatie-shift zelf; natuurlijk alleen voor gevallen 
waarin een VAR voor de eerste keer verhuist naar een shiftrel, en niet meer bij 
verdere verplaatsing van een bestaande shiftrel. Bij deze oplossing wordt de 
relativisatie-shift toch weer een regelklasse i.p.v.\ een transformatieklasse.

Het voorstel om de semantiek van relativisatie over twee regels te verdelen wordt 
aangenomen; wel wordt opgemerkt dat inmiddels een andere invulling van het 
begrip {\em vertaalrelevant\/} is ontstaan: ook de correspondentie van een 
bepaalde 
{\em structuur\/} in verschillende talen, bij een M-regel die in beide talen 
voorkomt maar waarvan de semantiek 
niet meteen voor de hand ligt, kan reden zijn om iets een {\em regel\/} te 
noemen, en niet meer 
alleen de correspondentie van {\em betekenis\/}.

\section{Case Assignment}
In het Nederlands is naamvalstoekenning een iteratieve transformatie, met een 
filter ervoor en erna. De regel was echter zo geschreven, dat alleen de eerste 
NPVAR/CNVAR in een boom gevonden werd; was dat toevallig een VAR die geen 
default waarde voor cases had (nominative), dan werd een volgende VAR nooit 
meer gevonden. Dit is opgelost door de voorwaarde van een default waarde voor 
cases op te nemen in de matchcondities.

\section{Expletief ER}
Het expletieve `er' ({\em Er zijn boeken, Er kwam een man, Er wordt gedanst, Er 
kocht iemand een boek\/}) 
staat niet in subjrel, maar in erposrel. Het feit dat een zin met 
zo'n expletief `er' geen subjrel bevat, levert op twee plaatsen problemen op: 
in de subject-verb agreement regel, waar agreement nu moet worden beregeld door 
allerlei andere elementen uit de zin (obj, postsubj etc.), en in de 
objectOKregels, die 
o.a.\ voor subject to subject raising moeten zorgen in zinnen als {\em Er 
schijnt iemand ziek te zijn\/}. De oplossing is om in de objectOK regels een 
abstract subject te defini\"{e}ren dat bestaat uit een NP 
met een EC eronder. De NP kan dan het attribuut `number' overnemen van een 
andere NP uit de zin voor zover dat van belang is voor agreement ({\em Er is 
een boek, Er 
zijn boeken\/}, en kan restricties opleggen aan deze NPs (postsubj, 
indefinite obj etc.). In de RADV-substitutieregels wordt 
de EC uitgespeld als RADV
nadat is bepaald of er eventueel nog meer `er's mee moeten samensmelten, bv.\ 
het prepositionele `er': {\em Er keek iemand naar\/} (in het Spaans treedt net 
zoiets op bij `se'-uitspelling). Een voorbeeld van derivatie:\\[5 mm]
{\em Er schijnen mensen te komen}\\
\begin{tabular}{ll}
            &   obj/x1(plur) \ te \ komen (ergatief!)\\
ObjectOK:   &   subj/NP(plur)[EC] \ obj/x1(plur, indef) \ te \ komen\\
SubjVerbAgr: & n.v.t. (infiniete zin)\\
Er$_{X}$-subst:   & n.v.t. (werkt alleen op finiete zinnen)\\
Subst:       &  subj/NP(plur)[EC] \ obj/mensen  te komen\\[5 mm]

PropSubst:   &  $_{S}$[ subj/NP(plur)[EC] \ obj/mensen te komen ] schijn\\
ObjectOK(SSR):    &  subj/NP(plur)[EC] $_{S}$[ obj/mensen te komen ] schijn\\
SubjVerbAgr:      &  subj/NP(plur)[EC] $_{S}$[ obj/mensen te komen ] schijnen\\
Er$_{X}$-subst:   &  erposrel/Er$_{X}$ \ $_{S}$[ obj/mensen te komen ] 
                                                    schijnen
\end{tabular}
\\[5 mm]
Na VerbRaising, VerbSecond etc. komt dan de goede zin tevoorschijn.

\end{document}

