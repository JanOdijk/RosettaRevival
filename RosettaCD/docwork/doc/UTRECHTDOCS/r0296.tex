
\documentstyle{Rosetta}
\begin{document}
   \RosTopic{Rosetta3.Linguistics.Minutes}
   \RosTitle{Notulen Linguistenvergadering 15-12-88}
   \RosAuthor{Andr\'{e} Schenk}
   \RosDocNr{0296}
   \RosDate{December 15, 1988}
   \RosStatus{concept}
   \RosSupersedes{-}
   \RosDistribution{Project}
   \RosClearance{Project}
   \RosKeywords{minutes}
   \MakeRosTitle
%
%
\begin{description}
\item[Aanwezig:] Lisette Appelo, Franciska de Jong, Elly van Munster,
                 Jan Odijk, Margreet Sanders,
                 Andr\'{e} Schenk,  Harm Smit
\item[Afwezig:]


\item[Agenda:]\mbox{}
  \begin{enumerate}
  \item Notulen
  \item Prep en Part
  \item Probleem met regels die moeten als ze kunnen
  \item Volgende vergadering
  \end{enumerate}
\end{description}

\section{Notulen}
De notulen van de vorige vergadering werden met enkele kleine wijzigingen
goedgekeurd. 

\section{Prep en Part}
Het stuk Prep en Part wordt besproken op de volgende vergadering. 

\section{Probleem met regels die moeten als ze kunnen}
Na een regel die moet als hij kan met als inputmodel CL[mu1,V,mu2] en als
outputmodel CL[mu1,V,Y,mu2], volstaat een filter met als model CL[mu1,V,mu2]
niet, omdat het outputmodel past op het filter. In dit soort gevallen heeft een
expressie [t]f dus niet gegarandeerd de juiste semantiek. Afhankelijk van de
situatie dienen de structuren aangescherpt te worden. 

\section{Volgende vergadering}
De volgende vergadering is op donderdag 19 januari 1989.

\end{document}


