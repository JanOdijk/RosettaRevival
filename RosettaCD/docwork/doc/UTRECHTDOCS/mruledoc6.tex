\documentstyle{Rosetta}
\begin{document}
   \RosTopic{Rosetta3.doc.Mrules.English}
   \RosTitle{Rosetta3 English M-rules: PREPPPROPformation}
   \RosAuthor{Margreet Sanders}
   \RosDocNr{380}
   \RosDate{November 15, 1989}
   \RosStatus{approved}
   \RosSupersedes{-}
   \RosDistribution{Project}
   \RosClearance{Project}
   \RosKeywords{English, documentation, Mrules, PREPPPROPformation}
   \MakeRosTitle
%
%

\section{Introduction}
As all main category grammars, the English PREPPPROP grammar is divided in 
three parts. First, {\bf PREPPPROPformation} forms the PROP-structure. Then, {
\bf PREPPPROPtoFORMULA} turns this PROP into an intermediate structure, called 
PREPPFORMULA
(comparable to the CLAUSE in the sentence grammar). Finally, {\bf 
PREPPFORMULAtoPROP} makes an open or closed PROP-structure, comparable to the 
SENTENCE level in the sentence grammar. Prior to PREPPPROPformation, there is 
the {\bf PrepDerivation} grammar, which was discussed in doc.\ 316, {\em 
Rosetta3 English M-rules: Derivation Subgrammars\/}. The final open or closed 
PREPPPROP is usually input to the Proposition Substitution Rules in the 
XPPROPtoCLAUSE subgrammar. A `bare' PREPPPROP may also form an expression on 
its own with help of the copula {\em be\/}. In that case, it leaves the 
PREPPPROP grammars after PREPPPROPformation, and 
is input to the ClauseFormation Rules of the XPPROPtoCLAUSE subgrammar. 
There is no rule to make a complete Utterance of a closed PROP.

The current document describes the contents of the first PREPPPROP subgrammar, 
PREPPPROPformation (the other two subgrammars are discussed in docs 381 and 
382). The subgrammar consists of 
a number of rule classes and transformation classes. A rule class in its turn
consists of a number of rules and a transformation class of a number of 
transformations. The relative ordering of the rules and transformations in the
(sub)grammar is indicated by a {\em control expression}. A summary of this
control expression (i.e.\ a listing of the ordering of the rule classes, 
without explicit mentioning of the rules themselves) is also included here, 
and the initial (= head), import and export categories are given. 

In the section on the rules and transformations, only the rule names are given, 
but not the exact rule formulation. What is attempted 
is to provide a detailed overview of the workings of the subgrammar, and 
how the different rule classes achieve this,
together with some comments on the problems still to be solved, the reasons 
behind certain choices, and perhaps possible alternatives. For every rule, an 
example is given. If it is uncertain whether the example is correct (either 
because it may not be an example of the phenomenon in question, or because it 
may not be correct English), it is preceded by a question mark. Note that all 
explanation of rules and transformations is given from a generative viewpoint
only, unless explicitly stated otherwise. Often, the information given in this 
document is based strongly on the comment already present in the documentation 
of the rules themselves. Discrepancies between what is stated here and what is 
said in the rule itself are usually caused by the fact that the rule file has 
not  been updated, although insights have changed. The semantics of the rules 
has been left unspecified in the current documentation, since it is not at all 
clear.

Whenever the current implementation differs widely from the strategy that was 
devised in the definition phase of Rosetta3 (as laid down for English in docs.\ 
150, {\em Subgrammars of English\/}, 153, {\em Rule and Transformation Classes 
of English\/}, and 155, {\em Rule and Transformation Classes common to all 
languages\/}, all written by Jan Odijk), this will be indicated explicitly in 
the current document. Conditions on crucial orderings of rule classes will be 
repeated here, even if they do not differ from the original strategy, to make 
the document as self-contained as possible.

Finally note that the rules described in this document have NOT been tested 
properly. English analysis is not possible yet (there is no Surface Parser), and 
English generation has only been tested in as far as the construction was the 
translation of a Dutch sentence to be tested.

\newpage
\section{PrepppropFormation}
The exact contents of the PrepppropFormation subgrammar are mainly determined 
by the requirements of isomorphy with the VerbppropFormation subgrammar. Thus, 
there are many rule classes which have only one rule, providing some sort of 
`default' value. Transformations are hardly needed, since the surface structure 
and many attribute values will be determined by the sentence the PROP is 
substituted in. 

Since the PREPPPROP may be pruned in the sentence grammar, just leaving a 
PREPP, the attributes of the PREPP are filled (for the most part, they are a 
copy of the PREP attributes). Cf.\ the VERBP-level in the sentence grammars, 
which has no such status and is not filled in the VERB startrules.

The contents of the subgrammar are much the same as was stipulated in doc.\ 
150. Rule classes have been added to introduce non-argument variables, in the 
same way as they were added in the VerbppropFormation subgrammar. The last rule 
class mentioned in doc.\ 150, RC\_PrePPModRules (for prexpadvs like {\em even, 
just\/}), has not been added yet; it does not exist in the sentence grammars 
either.

The contents of the three rule classes that introduce variables for 
non-arguments will be merged to form one large advvar rule class. This will not 
change the power of the grammar, but might make Dutch analysis easier.

\section{Subgrammar Specification}
The subgrammar definition can be found in the file which also contains all the 
rules of this subgrammar, {\bf english:PrepppropFormation.mrule}, 
which is {\em mrules86.mrule\/}.

\begin{verbatim}
%SUBGRAMMAR Prepppropformation


   ( RC_StartPPPRules )
.  ( TC_pppPatternRules )
.  { RC_pppTempVar }
.  { RC_pppSentAdvVar }
.  { RC_pppLocAdvVar }
.  ( RC_pppVoiceRules )
.  [ RC_pppModRules ]

\end{verbatim}

\begin{description}
  \item[Head]  SUBPREP  \ \ \ \ FROM (Prepderivation)
  \item[Export] PREPPPROP
  \item[Import] NPVAR, CNVAR, ADJPPROPVAR, ADVPPROPVAR, NPPROPVAR, 
PREPPPROPVAR, VERBPPROPVAR, SENTENCEVAR, EMPTYVAR, PROSENTVAR, ADVP, NP,
ADVPVAR, PREPPVAR, CLAUSEVAR
\end{description}

\newpage
\section{Rules and Transformations}

\subsection{RC\_StartPPPRules}
\begin{description}
\item[Kind] Obligatory Rule Class
\item[Task] To provide a SUBPREP with its correct number of argument variables 
and build a PREPP and PREPPPROP around it. The SUBPREP-level itself is deleted, 
just leaving the PREP. The reason for the existence of the SUBPREP level is 
explained in doc.\ 316, {\em Rosetta3 English M-rules: Derivation Subgrammars}. 

In the rules, the Aktionsarts of the PREPPPROP are set to {\em [stative]\/};
no separate transformation class is needed to to determine this `default' 
value.

\vspace{1 cm}
\begin{description}
\item[Name] RStartPPProp000
\item[Task] To build the PROP structure for a SUBPREP which does not take any 
arguments, and to introduce the dummy subject {\em It\/}.
\item[File] english:PrepppropFormation.mrule (mrules86.mrule)
\item[Semantics]
\item[Example] on $\rightarrow$ it on (It is on between Roger and my sister)
\item[Remarks] It is not clear whether the introduced subject is indeed a 
non-referential expression.
\end{description}

\vspace{1 cm}
\begin{description}
\item[Name] RStartPPProp100
\item[Task] To provide a PROP structure for a SUBPREP that takes one subject 
argument, and no arguments in the PREPP.
\item[File] english:PrepppropFormation.mrule (mrules86.mrule)
\item[Semantics]
\item[Example] out + x1 $\rightarrow$ x1 out (That ball is out)
\item[Remarks]
\end{description}

\vspace{1 cm}
\begin{description}
\item[Name] RStartPPProp120
\item[Task] To provide a PROP structure for a SUBPREP that takes one subject 
argument, and one argument in the PREPP. The attributes {\bf mood} and {\bf 
specQ} of the PREPP are determined on basis of the variable in it.
\item[File] english:PrepppropFormation.mrule (mrules86.mrule)
\item[Semantics]
\item[Example] in + x1 + x2 $\rightarrow$ x1 in x2 (He is in the house)
\item[Remarks]
\end{description}

\item[Remark] No rules have been written for `path' preps, like {\em from ... 
to ... (via ...)\/}. If a way can be found to deal with the explosion of 
ambiguities caused by the introduction of such double preps, rules should be 
added for them.

\end{description}

\newpage
\subsection{TC\_pppPatternRules}
\begin{description}
\item[Kind] Obligatory Transformation Class
\item[Task] To check the category of the argument variable and the relation 
it bears in the PREPP against the preppatterns specified for the PREP. The {\bf 
synppefs} attributes of the PREPP and the PREPPPROP are set at the value 
actually chosen from the synpps of the PREP.

In doc.\ 150, this transformation class is mentioned too, although it is 
described only in the document on Dutch rule classes (doc.\ 152). There, it is 
assumed that preps taking a prepp or advp argument must introduce a whole 
preppprop or advpprop and then prune it. This is no longer considered 
necessary. Preps functioning as a conjunction are not dealt with here, but in 
the sentence grammar (RC\_ConjSent in ClauseToSentence). `Approximate 
expressions', like {\em over a hundred, about five thousand\/}, are still 
thought to be examples of modification of numerals, and should be dealt with in 
the DETP grammar. Note that Dutch has many more of these prepp modifiers, often 
translating into an English adverb: {\em tegen de/rond de/om en nabij de 
vijftig - approximately fifty\/}.

\vspace{1 cm}
\begin{description}
\item[Name] TpppPattern0
\item[Task] To let preps that have no PREPP arguments pass this transformation 
class
\item[File] english:PrepppropFormation.mrule (mrules86.mrule)
\item[Semantics]
\item[Example] it in (The ball was in); x1 out (The countess was out)
\item[Remarks]
\end{description}

\vspace{1 cm}
\begin{description}
\item[Name] TpppPattern1
\item[Task] To specify the relation name (which must always be {\em objrel\/})
and the category of the argument in the PREPP 
\item[File] english:PrepppropFormation.mrule (mrules86.mrule)
\item[Semantics]
\item[Examples] \mbox{}\\
x1 in argrel/VAR $\rightarrow$ x1 in objrel/NPVAR\\
 (He was in the garden)\\
x1 in argrel/VAR $\rightarrow$ x1 in objrel/CNVAR \\
((the garden) which he was in)\\
x1 from  argrel/VAR $\rightarrow$ x1 from objrel/PREPPVAR\\ 
(She is from behind the Tower)\\
x1 until argrel/VAR $\rightarrow$ x1 until objrel/ADVPVAR \\
(? (He will not be there) until tomorrow)
\item[Remarks] It is not clear how the Dutch prep {\em tot\/} must be 
translated when it combines with another prep, as in {\em Hij liep tot in mijn 
tuin\/}. 
\end{description}

\end{description}

\newpage
\subsection{RC\_pppTempVar}
\begin{description}
\item[Kind] Iterative Rule Class
\item[Task] To introduce variables for time adverbials. The variable may be for 
a sentence, a prepp or an advp. The rule class is iterative, but the rules are 
formulated in such a way that only one ordering of var-introduction is possible 
for the different adverbials. This is to prevent unnecessary ambiguities in 
analysis.

The introduction of temporal adverbials is necessary only for those PREPPPROPs 
that will take the copula {\em be\/} and form an expression on their own. 
PREPPPROPs that go on to the next prep subgrammar are not supposed to receive 
their own time adverbials. In doc.\ 150, this rule class was not thought 
necessary for the prep grammars.

\vspace{1 cm}
\begin{description}
\item[Name] Rppprefvarinsertion
\item[Task] To introduce a variable for a referential time adverbial that is 
not retrospective.
\item[File] english:PrepppropFormation.mrule (mrules86.mrule)
\item[Semantics]
\item[Example] x1 out + refVAR $\rightarrow$ x1 out refVAR (The ball was out 
before he could spot it)
\item[Remarks]
\end{description}

\vspace{1 cm}
\begin{description}
\item[Name] Rpppdurvarinsertion
\item[Task] To introduce a variable for a durational time adverbial.
\item[File] english:PrepppropFormation.mrule (mrules86.mrule)
\item[Semantics]
\item[Example] x1 inside + durVAR $\rightarrow$ x1 inside durVAR (They were 
inside for three hours)
\item[Remarks]
\end{description}

\vspace{1 cm}
\begin{description}
\item[Name] Rpppretrovarinsertion
\item[Task] To introduce a variable for a referential time adverbial that is 
retrospective.
\item[File] english:PrepppropFormation.mrule (mrules86.mrule)
\item[Semantics]
\item[Example] x1 inside + retroVAR $\rightarrow$ x1 inside retroVAR (They have 
been inside for three hours)
\item[Remarks]
\end{description}

\end{description}

\newpage
\subsection{RC\_pppSentAdvVar}
\begin{description}
\item[Kind] Iterative Rule Class
\item[Task] To introduce variables for adverbial subordinate sentences in 
different positions and sentence or causal adverbials in initial position. The 
conjunction may also be a preposition. No rules have been written yet for 
abstract conjunctions.

The rules are in an iterative class, but have been written in such a way that 
only one order of application is possible. This to prevent unnecessary 
ambiguities in analysis. Also, the IL strategy in mapping the different 
conjsent rules is to preserve the surface order used in the source language as 
much as possible, because it may be of importance for pronominal reference.

In doc.\ 150, this rule class was not thought necessary for the prep grammars.

\vspace{1 cm}
\begin{description}
\item[Name] RpppConjsentVar
\item[Task] To introduce a variable for an adverbial subordinate sentence (or 
sentential PREPP) in initial (leftdislocrel) position.
\item[File] english:PrepppropFormation.mrule (mrules86.mrule)
\item[Semantics]
\item[Example] \mbox{}\\
x1 against x2 + advSENTENCEVAR $\rightarrow$ advSENTENCEVAR x1 
against x2 \\
(Although we do not live here yet, we are against foreigners in this 
neighbourhood)\\
x1 against x2 + advPREPPVAR $\rightarrow$ advPREPPVAR x1 against x2 \\
(Without wanting to discriminate anyone, we are against foreigners in this 
neighbourhood)
\item[Remarks] The comma that is probably obligatory after the adverbial 
sentence is not introduced here, but will be added in the proposition 
substitution rules of the XPPROPtoCLAUSE subgrammar (see RConjSentSubst and 
RConjPrepNPSubst).
\end{description}

\vspace{1 cm}
\begin{description}
\item[Name] RpppFinalConjsentVar
\item[Task] To introduce a variable for an adverbial subordinate sentence (or 
sentential PREPP) in final (postsentadvrel) position.
\item[File] english:PrepppropFormation.mrule (mrules86.mrule)
\item[Semantics]
\item[Example] \mbox{}\\
x1 against x2 + advSENTENCEVAR $\rightarrow$ x1 against x2 advSENTENCEVAR \\
(We are against foreigners in this neighbourhood, although we do not live here 
yet)\\
x1 against x2 + advPREPPVAR $\rightarrow$ x1 against x2 advPREPPVAR \\
(We are against foreigners in this neighbourhood, without wanting to 
discriminate anyone)
\item[Remarks] The comma that is probably obligatory before the adverbial 
sentence is not introduced here, but will be added in the proposition 
substitution rules of the XPPROPtoCLAUSE subgrammar (see RFinalConjSentSubst and 
RFinalConjPrepNPSubst).
\end{description}

\vspace{1 cm}
\begin{description}
\item[Name] RpppSentadvVar
\item[Task] To introduce a variable for a causal or sentence adverbial or a 
causal prepp in 
initial position. No rules have been written to account for any other position 
of the adverbial. Conditions to relate its position to that of an adverbial or 
temporal sentence also present in the clause will be added soon.
\item[File] english:PrepppropFormation.mrule (mrules86.mrule)
\item[Semantics]
\item[Example] \mbox{}\\
x1 out + sentADVVAR $\rightarrow$ sentADVVAR x1 out (Probably, it was out)\\
x1 against x2 + causPREPPVAR $\rightarrow$ causPREPPVAR x1 against x2 (For that 
reason, we are against the proposal)
\item[Remarks] The comma that is obligatory after most adverbs is not introduced 
here, but in the CLAUSEtoSENTENCE grammar in the substitution rules 
(RSentPreppSubst and RSentAdvSubst).
\end{description}

\end{description}

\newpage
\subsection{RC\_pppLocAdvVar}
\begin{description}
\item[Kind] Iterative Rule Class
\item[Task] To introduce variables for non-argument locatives (ADVP or PREPP) 
at a fixed position (locadvrel) following the PREPP. No rules have been 
written to account for any other position 
of the locative. Conditions to relate its position to that of an adverbial or 
temporal sentence also present in the clause will be added soon.

In doc.\ 150, this rule class was not thought necessary for the prep grammars.

\vspace{1 cm}
\begin{description}
\item[Name] RpppLocAdvVar
\item[Task] To introduce a variable for a non-argument locative ADVP 
\item[File] english:PrepppropFormation.mrule (mrules86.mrule)
\item[Semantics]
\item[Example] x1 against x2 + locADVVAR $\rightarrow$ x1 against x2 locADVVAR
(He was against the proposal there)
\item[Remarks]
\end{description}

\vspace{1 cm}
\begin{description}
\item[Name] RpppLocPreppVar
\item[Task] To introduce a variable for a non-argument locative PREPP
\item[File] english:PrepppropFormation.mrule (mrules86.mrule)
\item[Semantics]
\item[Example] x1 against x2 + locADVVAR $\rightarrow$ x1 against x2 locADVVAR
(He was against the proposal at the club)
\item[Remarks]
\end{description}

\end{description}

\newpage
\subsection{RC\_pppVoiceRules}
\begin{description}
\item[Kind] Obligatory Rule Class
\item[Task] To provide a rule at this place in the derivation that can function 
as an counterpart in the isomorphic scheme for the voice rules in the sentence 
grammar. The rule itself is vacuous: it does nothing whatsoever.

\vspace{1 cm}
\begin{description}
\item[Name] RPPPVoice
\item[Task] see above
\item[File] english:PrepppropFormation.mrule (mrules86.mrule)
\item[Semantics]
\item[Example] x1 out 
\item[Remarks]
\end{description}

\end{description}

\newpage
\subsection{RC\_pppModRules}
\begin{description}
\item[Kind] Optional Rule Class
\item[Task] To introduce modifiers (of the PREP) in the PREPP, as some kind of 
degree modification. In doc.\ 150, this class was thought to occur before the 
voice rules, but that ordering was not crucial.

This rule class is no part of any other subgrammar (except 
AdjppropFormation, but there it is situated differently, viz.\ before the 
TempAdv rules). The existence of 
this rule class makes the PrepppropFormation subgrammar only {\em partially\/} 
isomorphic to the other XPPROPformation subgrammars, even to the 
AdvppropFormation subgrammar.
Thus, application of this (optional) rule class makes it impossible to translate 
to another basic category, following another subgrammar path. Note that the
AdjppropFormation and the PrepppropFormation subgrammars would be isomorphic if 
the rules were ordered the same.

The rule class is placed here to allow for modified PREPs becoming a sentence 
on their own ({\em He was dead against every innovation\/}). However, there is 
a major problem with superdeixis: the PREPPPROP has not received a value for 
superdeixis yet, and thus there can be no check on the appropriateness of the 
superdeixis value of the modifier. But in the XPPROPtoCLAUSE subgrammar, where 
superdeixis of the CLAUSE is set, there can be no such check anymore because 
in the modification rules the superdeixis is set to omegadeixis. This problem 
has not been solved yet. Note that in the PREPP subgrammar, which need not be 
isomorphic to the XPPROP grammars, the order of superdeixis determination and 
modification have been reversed.

\vspace{1 cm}
\begin{description}
\item[Name] RPPPAdvpMod
\item[Task] To introduce an ADVP degree modifier of the PREP in the PREPP.
\item[File] english:PrepppropFormation.mrule (mrules86.mrule)
\item[Semantics]
\item[Example] x1 against x2 + dead $\rightarrow$ x1 dead against x2
\item[Remarks]
\end{description}

\vspace{1 cm}
\begin{description}
\item[Name] RPPPNpMod
\item[Task] To introduce a (unitnoun) NP modifier of the PREP in the PREPP.
\item[File] english:PrepppropFormation.mrule (mrules86.mrule)
\item[Semantics]
\item[Example] x1 behind x2 + three foot $\rightarrow$ x1 three foot behind 
x2 (She always stays three foot behind her husband)
\item[Remarks]
\end{description}

\end{description}


\end{document}

