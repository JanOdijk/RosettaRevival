\documentstyle{Rosetta}
\begin{document}
   \RosTopic{Rosetta3.doc.Mrules.English}
   \RosTitle{Rosetta3 English M-rules: toNPPROP}
   \RosAuthor{Margreet Sanders, Franciska de Jong}
   \RosDocNr{393}
   \RosDate{December 4, 1989}
   \RosStatus{concept}
   \RosSupersedes{-}
   \RosDistribution{Project}
   \RosClearance{Project}
   \RosKeywords{English, documentation, Mrules, toNPPROP}
   \MakeRosTitle
%
%

\section{Introduction}
This document describes the contents of the second NPPROP subgrammar, 
{\bf toNPPROP}. The first NPPROP subgrammar is discussed in doc.\ 392, 
{\em Rosetta3 English M-rules: NPPROPformation\/}. The introduction given in 
that document to explain the different treatment of NPPROPs as opposed to other 
XPPROPs is repeated here in full, to make the current document as self-
contained as possible.

NPPROP structures are structures with an NP as predicative head (under predrel).
The derivation of NPPROP structures is different from the derivation of all
other XPPROP and SENTENCE structures, and the isomorphy with other 
main category grammars is only partial. 
The reason for this deviation is the following.

Two kinds  of NPs may be distinguished on the level of
surface structures: (1) non-predicative NPs, 
that may function a.o. as subject, VP-complement and PREPP-complement, and 
(2) predicative NPs that occur in predicative position, e.g. in the context 
of verbs as {\em become\/} and {\em consider\/}. 
For most major categories (i.e.\ for X = ADJ, ADV, VERB, PREP), 
a predicate (XP) and a propositional structure (XPPROP) are built around the 
X in the XPPROPformation subgrammar. For X = NOUN, the situation is more 
complex: only for the derivation of predicative NPs is the 
creation of the NPPROP level semantically correct. 
Non-predicative NPs must be derived by a separate grammar, with its own 
idiosyncratic division into subgrammars (a.o CNformation, NPformation), not 
resulting in an NPPROP.

However, even when dealing with predicative NPs only there is a problem with 
isomorphy.
Many semantic and syntactic features of NP structures do not match the 
processes known for other XPs.
For example, NP determination (the introduction of determiners) has no clear
counterpart in any other main category grammar, and most 
modifications within the NP pertain to the NP-head, whereas in the  
other XPPROP grammars, modification pertains to the XPPROP-level, at 
least semantically. Hence, a derivation of 
NPPROPs that parallels the derivation of other XPPROPs is 
problematic, at least under the present implementation of XPPROP grammars.

Pending the development of a more satisfying approach to the mapping of 
NPPROPs and XPPROPs,  the present approach to the derivation of 
NPPROPs is one that has relaxed the conditions on isomorphy with other 
XPPROPsubgrammars, and requires a minimum number of rules specific for 
NPPROPs. Instead of building the NPPROP around the SUBNOUN provided by a 
derivation subgrammar, NPPROPs now take an NP formed by the NPformation 
subgrammar as their head. Thus, the rules existing for determination, 
modification etc.\ of 
non-predicative NPs need not be duplicated in the NPPROPformation subgrammar.
(Note that almost all NPs that may occur 
in non-predicative position may occur in predicative position as well. 
NPs consisting of a bare count noun ({\em 
Juliana  became queen in 1948}), which are excluded from all other contexts, may
also occur as predicate.) 
As a consequence, it is excluded that an NPPROP is translated into 
any other XPPROP. 

The rules working in the NPPROP subgrammars could not be written completely 
independent of the other XPPROP subgrammars, however. When an NPPROP is input 
to another subgrammar, it must have a general structure and attribute values 
compatible with what is expected there. Thus, there still is a general 
resemblence between the rules for NPPROPs and those for other XPPROPs.
Presently, of the three subgrammars that constitute the other XPPROP grammars 
only two have a counterpart in the NPPROP grammar.
The subgrammar {\bf NPPROPformation} introduces the NPPROP-level with the 
`finished' NP as predicate (predrel). 
Its output is either input to the clause formation rules of the 
{\bf XPPROPtoCLAUSE} subgrammar 
(in which sentences with the copula {\em be\/}, 
such as {\em She is a linguist\/} are derived), or 
it is input to the remaining part of the NPPROP grammar, called {\bf toNPPROP}.
For other XPPROPgrammars this remaining part consists of two subgrammars, 
XPPROPtoXPFORMULA and XPFORMULAtoXPPROP. In the `middle' subgrammar, aspect and 
superdeixis are dealt with. Since these have already been dealt with in the 
NP-subgrammar, they can be skipped for NPPROPs, and there is no need for an 
NPFORMULA 
level. The final subgrammar {\bf toNPPROP} turns the NPPROP directly into 
an open or a closed NPPROP structure.
This NPPROP is then import to the proposition substitution rules of
the subgrammar {\bf XPPROPtoCLAUSE}, and is incorporated into the sentential 
structure. Recently, the introduction of a `minimal' NPPROPtoNPFORMULA 
subgrammar has been proposed, to make translations between a small clause 
NPPROP like {\em He (seems) an imposter\/} and a full clause like {\em (It 
seems) that he is an imposter\/} possible. These clauses share the 
NP-derivation 
path, and only need parallel paths in the NPPROPtoNPFORMULA and XPPROPtoCLAUSE 
subgrammars to be possible. This change will probably be added in the near 
future.


As said above, the current document describes the contents of the subgrammar {
\bf toNPPROP}. The subgrammar consists of 
a number of rule classes and one transformation class. A rule class in its turn
consists of a number of rules and a transformation class of a number of 
transformations. The relative ordering of the rules and transformations in the
subgrammar is indicated by a {\em control expression}. A summary of this
control expression (i.e.\ a listing of the ordering of the rule classes, 
without explicit mentioning of the rules themselves) is also included here, 
and the initial (= head), import and export categories are given. 

In the section on the rules and transformations, only the rule names are given, 
but not the exact rule formulation. What is attempted 
is to provide a detailed overview of the workings of the subgrammar, and 
how the different rule classes achieve this. For every rule, an 
example is given. If it is uncertain whether the example is correct (either 
because it may not be an example of the phenomenon in question, or because it 
may not be correct English), it is preceded by a question mark. Note that all 
explanation of rules and transformations is given from a generative viewpoint
only, unless explicitly stated otherwise. Often, the information given in this 
document is based strongly on the comment already present in the documentation 
of the rules themselves. Discrepancies between what is stated here and what is 
said in the rule itself are usually caused by the fact that the rule file has 
not  been updated, although insights have changed. The semantics of the rules 
has been left unspecified in the current documentation, since it is not at all 
clear.

In doc.\ 150, {\em Subgrammars of English\/} (in which the definition of 
Rosetta3 for English was presented), the contents of the NPPROP subgrammars 
were not specified. Instead, reference was made to an earlier document by 
Franciska de Jong, doc.\ 117: {\em The subgrammars specific to Nominal 
Constituents\/}. For NPPROPs, this document specifies only two rule classes, 
Startrules and Modification rules, together with a transformation class for 
Pattern rules. None of these rule classes have anything in common with the rule 
classes now existing for the toNPPROP subgrammar. Hence, any comparison between 
the contents of doc.\ 117 and the rules described in the current document 
is useless.

Finally note that the rules described in this document have NOT been tested 
properly. English analysis is not possible yet (there is no Surface Parser), and 
English generation has only been tested in as far as the construction was the 
translation of a Dutch sentence to be tested.

\newpage
\section{toNPPROP}
As explained in the introduction to this document, the exact contents of the 
toNPPROP subgrammar are mainly determined 
by the requirements posed by the sentence in which the NPPROP is substituted.
Thus, the main function of the subgrammar is to make the proposition open or 
closed. Very few other rules have been added, since isomorphy with other 
categories is impossible anyway at present.

\section{Subgrammar Specification}
The subgrammar definition can be found in the file which also contains all the 
rules of this subgrammar, {\bf english:NPsubgrammars.mrule}, 
which is {\em mrules56.mrule\/}.

\begin{verbatim}
%SUBGRAMMAR toNPPROP


   ( TC_NPPPROstatus )
.  { RC_NPPsubst }
.  ( RC_NPPmood )
.  ( RC_NPPpunc )

\end{verbatim}

\begin{description}
  \item[Head]  NPPROP \ \ \ \ FROM (NPPROPformation)
  \item[Export] OPENNPPROP, CLOSEDNPPROP
  \item[Import] NP
\end{description}

\newpage
\section{Rules and Transformations}

\subsection{RC\_NPPPROstatus}
\begin{description}
\item[Kind] Obligatory Transformation Class
\item[Task] To assign a value to the attribute {\bf PROsubject}. The default 
value is 
{\em false\/}. In generation, there is a free choice between the two rules, and 
both an OPENNPPROP (PROsubject = {\em true\/}) and a CLOSEDNPPROP (PROsubject = 
{\em false\/}) are made. If a subject substitution rule is applied (see 
below), only the version with PROsubject = {\em false\/} is allowed; hence, the 
transformation class is ordered crucially before the substitution rules.

\vspace{1 cm}
\begin{description}
\item[Name] TNPnoPROsubj
\item[Task] Vacuous rule, leaving the {\bf PROsubject} attribute of the NPPROP 
at its default value, which is {\em false\/}.
\item[File] english:NPsubgrammars.mrule (mrules56.mrule)
\item[Semantics] --
\item[Example] [x1 a doctor]$_{PROsubject=false}$ (She became a doctor)
\item[Remarks]
\end{description}

\vspace{1 cm}
\begin{description}
\item[Name] TNPPROsubj
\item[Task] To set the {\bf PROsubject} attribute of the NPPROP 
at the value {\em true\/}.
\item[File] english:NPsubgrammars.mrule (mrules56.mrule)
\item[Semantics] --
\item[Example] [x1 a doctor]$_{PROsubject=false}$ $\rightarrow$ 
[x1 a doctor]$_{PROsubject=true}$ (She seemed a doctor; with {\em seem\/} as 
two place verb)
\item[Remarks]
\end{description}

\end{description}

\newpage
\subsection{RC\_NPPsubst}
\begin{description}
\item[Kind] Iterative Rule Class
\item[Task] To substitute a (non-sentential, non-generic) NP for its subject 
variable. The same constraints and remarks hold as for other substitution rules 
(see e.g.\ the comparable rule class in the English CLAUSEtoSENTENCE 
subgrammar, doc.\ 
370), except that reflexives etc.\ have not yet been excluded from the 
current rule explicitly. 

This rule was added only because in Spanish, some constructions never get a 
subject in the main clause (there only is something like {\em It seemed that 
the man was a doctor\/}). However, since the sentence grammar and the NPPROP 
grammars are not isomorphic anyway, this rule is still superfluous.

\vspace{1 cm}
\begin{description}
\item[Name] RNPPsubjSubst
\item[Task] see above
\item[File] english:NPsubgrammars.mrule (mrules56.mrule)
\item[Semantics]
\item[Example] x1 a junkie + the man $\rightarrow$ the man a junkie (The man 
became a junkie)
\item[Remarks]
\end{description}

\end{description}

\newpage
\subsection{RC\_NPPmood}
\begin{description}
\item[Kind] Obligatory Rule Class
\item[Task] To turn a `bare' NPPROP into an OPEN or CLOSED NPPROP, depending on 
the value of the attribute {\bf PROsubject}. This rule is the `XPPROP-version' 
of the mood rules in the sentence grammar, but there are no new attributes 
going with the new top node.

\vspace{1 cm}
\begin{description}
\item[Name] ROpenNPPROPformation
\item[Task] To turn an NPPROP into an OPENNPPROP
\item[File] english:NPsubgrammars.mrule (mrules56.mrule)
\item[Semantics]
\item[Example] $_{NPPROP}$[x1 a sailor] $\rightarrow$ $_{OPENNPPROP}$[x1 
a sailor] (He became a sailor)
\item[Remarks] 
\end{description}

\vspace{1 cm}
\begin{description}
\item[Name] RClosedNPPROPformation
\item[Task] To turn an NPPROP into a CLOSEDNPPROP
\item[File] english:NPsubgrammars.mrule (mrules56.mrule)
\item[Semantics]
\item[Example] $_{NPPROP}$[he a sailor] $\rightarrow$ $_{CLOSEDNPPROP}$[he
a sailor] (He seemed a sailor)
\item[Remarks] 
\end{description}

\end{description}

\newpage
\subsection{RC\_NPPpunc}
\begin{description}
\item[Kind] Obligatory Rule Class
\item[Task] Vacuous rule class, added when other subgrammars needed this rule 
for isomorphy reasons. Since the NPPROP grammar is not isomorphic anyway, this 
rule is still superfluous here.

\vspace{1 cm}
\begin{description}
\item[Name] RNPPnoPunc
\item[Task] see above
\item[File] english:NPsubgrammars.mrule (mrules56.mrule)
\item[Semantics]
\item[Example] x1 the king 
\item[Remarks]
\end{description}

\end{description}

\end{document}


