\documentstyle{Rosetta}
\begin{document}
   \RosTopic{General}
   \RosTitle{Notulen Vergadering over R3D; 2-3-1990}
   \RosAuthor{Harm Smit}
   \RosDocNr{430}
   \RosDate{\today}
   \RosStatus{approved}
   \RosSupersedes{-}
   \RosDistribution{Project}
   \RosClearance{Project}
   \RosKeywords{minutes}
   \MakeRosTitle
\begin{itemize}
  \item {\bf aanwezig}: Andr\'{e} Schenk, Jan Landsbergen, Lisette Appelo,
                     Franciska de Jong, Petra de Wit, Elly van Munster, 
                     Joep Rous, Ren\'{e} Leermakers,
                     Jan Odijk, Harm Smit, Frank Uittenbogaard.
\end{itemize}

\section{R3D en CRE}

Jan L. informeert of er nog iemand bezwaren heeft tegen het gewijzigde 
CRE-plan. Dit blijkt niet het geval te zijn. Het gaat dus om drie `versies' 
van Rosetta:

\begin{enumerate}
  \item R3D, niet meer specifiek voor brieven. Hierbij staat nu alleen nog
        het verbeteren van fouten centraal. Deze versie wordt als "zinnen
        vertaler" gepresenteerd; wel wordt het LEXICO-bestand gebruikt.
  \item Vervoegmachine.
  \item `Localros'.
\end{enumerate}

M.b.t. 2 en 3 worden tijdens de vergadering stukjes van respectievelijk
Jan L. en Joep uitgedeeld.

Als `voortrekkers' voor de verschillende versies worden aangewezen:
voor 1. het bestaande groepje, voor 2. Lisette en voor 3. Joep.

Lisette vraagt hoe zeker het is dat we op de CRE komen. Dit is nog niet voor 
honderd procent zeker; dat weten we pas per 1 april. Het werk wat we nu doen is 
echter ook interessant voor andere demonstraties, zoals bijv. bij een bezoek 
van mensen van CE.

Het woordenboekwerk wordt besproken. De punten 8, 9, 10, 12, 13, 14 vervallen
doordat we geen `hotelreserveringen' meer doen.

Tenslotte merkt Jan L. nog op dat iedereen nog punten voor het 
voortgangsverslag moet 
inleveren m.b.t. haar/zijn werk in het tweede halfjaar van 1989. 
\end{document}
