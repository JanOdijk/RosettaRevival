\documentstyle{Rosetta}
\begin{document}
      \RosTopic{Rosetta3.Linguistics.test}
      \RosTitle{Sentences and Constructions Used in Testing Rosetta3}
      \RosAuthor{all Linguists}
      \RosDocNr{R0292}
      \RosDate{\today}
      \RosStatus{concept}
      \RosSupersedes{-}
      \RosDistribution{Project}
      \RosClearance{Project}
      \RosKeywords{testing}
      \MakeRosTitle

\addtolength{\itemsep}{-2 ex}
\def\zinnen{\begin{description}}
\def\endzinnen{\end{description}}
\def\zin#1{\item[]{\em #1}}
\def\vertaling#1{\item[] - #1}
\def\Test#1{\vspace{1 ex} \item[Test:]{\bf #1}}
\def\test#1{\item[] - tests: #1}
\def\VR{\item VerbRaising}
\def\remark#1{\item[remark]#1}
\def\works{\item[status] operative}
\def\worksnotyet{\item[status] not yet operative}

\section{Morphology}
Some words that are covered by the morphology of Dutch, English and 
Spanish deserve special attention. Of course, there are words having 
irregular inflection, like {\em goed - beter - best, badly - worse - 
worst\/} for adjectives/adverbs.

For {\bf Dutch}, special measures are taken for adjectives like {\em grof\/},
which have weakening of the consonant in e-forms: {\em grove\/}. Other words
needing special attention are OnlyPlurals like {\em hersenen\/}, and
NoPlurals. Particles can be dealt with ({\em op-geven\/}), and for a word 
like {\em idee\/} the
set-value of the attribute {\em genders\/} is needed: {\em de idee, het idee\/}.
Diminutives are also covered: {\em tangootje, karretje\/}.

For {\bf English}, morphology can handle different types of genitive {\em 's\/}:
{\em parents', M.P.'s\/} though (in generation) not {\em boss's\/}. Contraction
of subject and auxiliary is dealt with in analysis: {\em He'd do it\/}. 
Contraction of negatives and an auxiliary is allowed both ways in morphology:
{\em won't, hasn't\/}. Special rules were written to deal with {\em sly - 
slyer\/} vs.\ {\em dry - drier/dryer\/}. Other special words morphology can 
handle are: {\em to ski - skiing, to traffic - trafficking, trafficked, to 
singe - singeing\/} (vs.\ to age - aging).

For Spanish, ...
\\[2 ex]
These `special' words have been incorporated in the test sentences following 
below.


\newpage

\section{Dutch Sentences}
(introduction)

These sentences may also include embedded XPPROPs, implicitly also testing the 
XPPROP grammars.

In principle, the `translation' of the sentences into Dutch is the same as the 
input. If more possibilities come up, this is indicated explicitly.
\\[2 ex]
Notation inside `tests':

\begin{description}
\item[a+b] the combination of {\em a} and {\em b}
\item[a,b] a with specification b
\item[a;b] a and separately, b
\end{description}

\begin{zinnen}
\zin{Jan zwemt}
\test{tense ott; subject substitutie; startvp100; TV21; Aktionsart}
\zin{Hij heeft gezwommen}
\test{tense vtt; TV23}
\zin{Hij wordt gedronken}
\test{passief; EMPTY-subst byobjrel; TV23; tense ott}
\zin{Hij is gedronken}
\test{passief; EMPTY-subst byobjrel; vtt; TV23}
\zin{Hij wordt door hem gedronken}
\test{passief; no EMPTY-subst;tense ott}
\zin{Hij is door hem gedronken}
\test{passief; no EMPTY-subst; vtt}
\zin{Hij belt hem op}
\test{V+ particle; TV22}
\zin{Jan at een appel}
\test{direct object; startvp120; Aktionsart}
\zin{Hij eet}
\test{EMPTY subst object}
\zin{Jij scheert je}
\test{Inh. reflexieven, 2 sg}
\zin{Hij zwemt wel}
\test{Positive}
\zin{Hij geeft hem hem  }
\test{startvp123; indobj}
\zin{Hij geeft hem  }
\test{startvp123; EMPTY-subst, indobj}
\zin{Hij is mooi}
\test{Clauseformation3; adj-clause interaction}
\zin{Hij wordt mooi}
\test{closedadjpprop; oblcontrol1a; start010; tense}
\zin{Er wordt gezwommen}
\test{Impersonal Passive; Emptysubst byobjrel}
\zin{Er wordt gegeten}
\test{Impersonal Passive; EMPTYsubst byobjrel + objrel }
\zin{Er wordt door hem gezwommen}
\test{Impersonal Passive, no EMPTYsubst }
\zin{Er wordt door hem gegeten}
\test{Impersonal Passive, no EMPTYsubst byobjrel; EMPTYsubst objrel }
\zin{Hij zegt dat hij zwemt}
\test{Rmood2; tense ott}
\zin{Hij heeft gezegd dat hij zwemt}
\test{Rmood2; tense vtt}
\zin{Hij probeert om te zwemmen}
\test{Rmoodopenomteinf; Romcomplsubst +om; subject control}
\zin{Hij probeert te zwemmen}
\test{Rmoodopenomteinf; Romcomplsubst -om; subject control; TV24; tense}
\zin{Hij pleegt te zwemmen}
\test{Rmoodopenteinf; Rcomplsubst; subject control}
\zin{Hij verblijft in Duitsland}
\test{PPP subgrammar; toblcontrol5a1; locargrel}
\zin{Hij komt}
\test{Ergative verbs; tobjectok1; start010}
\zin{Hij wil zwemmen}
\test{Verb raising; subject control; TV24}
\zin{Hij heeft willen zwemmen}
\test{TV24}
\zin{Zwem!}
\test{Imperative sg}
\zin{Zwemt!}
\test{Imperative pl}
\zin{Zwemt hij?}
\test{yes-no question}
\zin{Hij ziet het}
\test{het-clitic}
\zin{Hij ziet 't}
\test{'t-clitic}
\zin{Hij ziet 'm}
\test{do-clitic}
\zin{Hij geeft het hem}
\test{io-clitic + het-clitic}
\zin{Hij geeft 't hem}
\test{ioclitic + 't-clitic}
\zin{Hij geeft 't 'm}
\test{io-clitic 'm + 't-clitic}
\zin{Hij heeft met hem gesproken}
\test{prepositional object}
\zin{Hij zwemt er}
\test{ErL + V/2}
\zin{Hij heeft er gezwommen}
\test{ErL }
\zin{Hij heeft hier gezwommen}
\test{hier locative}
\zin{Hij heeft ergens gezwommen}
\test{ergens locative}
\zin{Hij zegt dat hij wil zwemmen}
\test{verb-raising in embedded fite clause}
\zin{Hij zegt dat hij aan wil dringen}
\test{separation of particle and verb in verb-raising, -te}
\remark{This sentence does not work anymore, pattern of V has changed}
\zin{Hij zegt dat hij probeert te zwemmen}
\test{verbraising in embedded finite clause, +te}
\zin{Hij zegt dat hij aan probeert te dringen}
\test{verbraising in embedded finite clause, separated particle +te}
\zin{Hij zegt dat hij hem wil zien}
\test{Verbraising with object in embedded infinitive, -te }
\zin{Hij zegt dat hij met hem wil spreken}
\test{\VR with prepobj in embedded infinitive, -te}
\zin{Hij zegt dat hij hem probeert te zien}
\test{\VR with obj in embedded infinitive, +te}
\zin{Hij zegt dat hij met hem probeert te spreken}
\test{\VR with prepobj in embedded infinitive, +te}
\zin{Hij ziet hem zwemmen}
\test{rmoodclosedinf; subj-removal}
\zin{Hij geeft aan hem hem}
\test{synaanNP\_donNP} 
\zin{Er komt iemand}
\test{ErX, ergative verb}
\zin{Er zwemt iemand}
\test{ErX, intransitive verb}
\zin{Er wordt iets gekocht}
\test{ErX, personal passive}
\zin{Hij heeft hem opgebeld}
\test{vtt}
\zin{Hij heeft aangedrongen}
\test{vtt}
\zin{Er wordt door hem geprobeerd te zwemmen}
\test{NOC, byobjrel}
\zin{Er kocht iemand iets}
\test{ErX, transitive verb}
\zin{Er zwemmen twee ossen}
\test{ERX, intransitive verb}
\zin{Hij zegt dat de man de os gegeven wordt}
\test{io+do + passive; tobjectok5}
\zin{Hij staat het de man toe te zwemmen}
\test{hetopenomtesent; io-control}
\zin{Hij staat 't 'm toe te zwemmen}
\test{hetopenomtesent; io-control}
\zin{Hij biedt 'm aan te zwemmen}
\test{openomtesent; control}
\zin{Hij biedt de man aan te zwemmen}
\test{openomtesent; control}
\zin{Hij biedt aan te zwemmen}
\test{openomtesent; control}
\zin{Hij staat het hem toe te zwemmen}
\test{hetopenomtesent}
\zin{Wat ziet hij}
\test{wh-shift, object}
\zin{Wat zegt hij dat hij ziet}
\test{wh-shift, long distance}
\zin{Wie geeft hij de os}
\test{wh-shift, indobj}
\zin{Aan wie geeft hij de os}
\test{wh-shift, aanobj}
\zin{ik zie mezelf}
\test{Argument reflexives 1sg}
\zin{het regent}
\test{start000}
\zin{het regent ossen}
\test{start010b}
\zin{hij blijkt te zwemmen}
\test{\VR; subj-tosubj-Raising}
\zin{hij lijkt mij groot}
\test{tnocontrol1; ionp\_closedadjpprop}
\zin{hij lijkt groot}
\test{tnocontrol1; empty\_closedadjpprop}
\zin{ik weet of hij komt}
\test{Rmoodynsub; synQSENT}
\zin{ik vraag of hij komt}
\test{Rmoodynsub; synempty\_QSENT}
\zin{ik laat de os aan Jan zien}
\test{RAAnactive}
\worksnotyet
\zin{ik heb aan hem gevraagd te zwemmen}
\test{aanobj control}
\worksnotyet
\zin{Door wie wordt hij gedood}
\test{wh-shift, byobjrel}
\zin{Het bevalt de man dat hij zwemt}
\test{synionp\_hetthatsent; start012}
\zin{Het bevalt de man te zwemmen}
\test{synionp\_hetopenomtesent}
\zin{Het bevalt de man de os te zien}
\test{synionp\_hetopenomtesent}
\zin{Hij betreurt het dat hij zwemt}
\test{synhetthatsent}
\zin{Hij verblijft te Frankrijk}
\test{ppp-subgrammar, locargrel; synlocopenpreppprop; toblcontrol5b1}
\zin{Hij heet Jan}
\test{synopennpprop}
\zin{Hij beschouwt hem als de vulkaan}
\test{syndonp\_prepopennpprop}
\worksnotyet
\zin{Aan wie geeft hij de os}
\test{wh-shift, aanobjrel}
\zin{Hij ziet de man niet}
\test{unmeted NEG + preadvrel}
\zin{wat laat de man de os zien}
\test{shift-removal}
\zin{wat vertelt hij de man te doen}
\test{wh-shift, long distance, extraposed infinitive}
\zin{dit kost de man drie ossen}
\test{synionp\_measurephrase}
\zin{ik weet het}
\test{synPROSENT}
\zin{dit kost drie ossen}
\test{synEMPTY\_measurephrase}
\zin{hij weegt drie ossen}
\test{synMEASUREphrase}
\zin{hij openbaart zich aan hem}
\test{vp012 + reflexive}
\zin{hij vraagt aan hem te zwemmen}
\test{aanobj control}
\zin{hij vindt de os mooi}
\test{synclosedadjpprop, tnocontrol1}
\zin{er blijkt dat hij zwemt}
\test{synTHATSENT + vp010}
\zin{het kan regenen}
\test{subj-tosubj-raising, -te}
\zin{hij laat de man zwemmen}
\test{synclosedinfsent; subj-removal}
\zin{hij verft de os mooi}
\test{syndonp\_openadjpprop; toblcontrol1a}
\zin{ik maak de os vulkaan}
\test{syndonp\_opennpprop}
\zin{hij beschuldigt de man van de os}
\test{syndonp\_prepnp}
\zin{hij bidt de man om de os}
\test{syndonp\_prepnp}
\zin{Ik betaal aan de man de os}
\test{synaannp\_donp}
\zin{Er verblijft iemand}
\test{ErX+ErL(arg)}
\zin{Hij kijkt er naar}
\test{prepobj + er}
\zin{Hij kijkt ernaar}
\test{prepobj + er; erglue}
\zin{Hij is de vulkaan ingegaan}
\test{donp\_diropenpreppprop; postprepok+ noV2}
\zin{Hij gaat de vulkaan in}
\test{donp\_diropenpreppprop; postprepok + V2}
\zin{Hij rekent er op dat hij komt (4,4,?)}
\test{sentential prepobj}
\zin{Hij rekent erop dat hij komt}
\test{sentential prepobj + erglue}
\zin{Er werd geprobeerd te zwemmen}
\test{noc, empty byobjrel}
\zin{Hij loopt naar de metropolis (4,4,4)}
\test{syndonp\_diropenpreppprop}
\zin{Hij verblijft er (2,2,2)}
\test{ErL(arg)}
\zin{Naar wie kijk ik?}
\test{wh-shift;prepobj}
\zin{Waar zwemt hij (2,2,?)}
\test{wh-shift;locadv}
\zin{Het huis dat hij ziet (2,2,?)}
\test{relativization, obj, neuter}
\zin{De man die hij ziet (2,2,?)}
\test{relativization, obj, masc}
\zin{De man die hij het huis geeft (?,?,?)}
\test{relativization, indobj}
\zin{De man die hem ziet (4,4,?)}
\test{relativization ,subj}
\zin{Waar kijkt hij naar (2,2,?)}
\test{wh-shift, stranding, prepobj}
\zin{Er kijkt iemand naar (2,2,?)}
\test{ErX+ErP, active}
\zin{Ik liet iemand de kerk bouwen (2,2,?)}
\test{synclosedinfsent}
\zin{Er wordt naar gekeken (2,2,?)}
\test{ErX+ErP, passive}
\zin{Hij vraagt het zich af (4,4,-) }
\test{obj+inh.refl, synPROSENT}
\end{zinnen}

\subsection{Identificationals and Existentials}
\begin{zinnen}
\zin{dit is de os}
\test{ident, dit}
\zin{het is de os}
\test{ident, het}
\zin{dit zijn de ossen}
\test{ident dit, pl}
\zin{wat is dit?}
\test{ident, dit, wat, sg}
\zin{wat zijn dit?}
\test{ident, dit, wat, pl}
\zin{er zijn ossen}
\test{existential, pl}
\zin{er is een os}
\test{existential, sg}
\end{zinnen}

\newpage
\section{English Sentences}
This is a description of the sentences used to test English M-rules in the 
clause grammars. The Dutch
sentence needed as input is given, together with the different translations and
the rules these translations test. No indication has been given yet of how 
often identical translations are produced, and why.\\
Notation inside `tests':
\begin{description}
\item[a+b] the combination of {\em a} and {\em b}
\item[a,b] a with specification b
\item[a;b] a and separately, b
\end{description}

\begin{zinnen}

\Test{1. StartRules and Verbpatterns}

\Test{2. Particles}
\zin{Hij gaf zijn ambities op }
\vertaling{He gave up his ambitions (A: He gave his ambitions up)}
\test{TParticleInsertion + TNoPartHop (A: TOptPartHop)}
\zin{Hij gaf het op }
\vertaling{He gave it up}
\test{TParticleInsertion + TOblPartHop}

\Test{3. AdvVars}

\Test{4. Voices}
\zin{Hij ziet het }
\vertaling{He sees it / He is seeing it}
\test{RActive; RActClauseFormation; RIndicMoodMain }
\zin{Het wordt gezien }
\vertaling{It is seen / it is being seen}
\test{RPassive1; RPasClauseFormation; RIndicMoodMain}
\zin{?}
\vertaling{I was surprised by what they wanted}
\test{RPassive2 etc}

\Test{5. Reflexives}
\zin{Zij pleegden meineed }
\vertaling{They perjured themselves}
\test{TObjReflInsertion1/2}
\zin{?}
\vertaling{They had themselves a ball}
\test{TIndObjReflInsertion1/2}

\Test{6. Propsubst}
\zin{Hij kan het zien }
\vertaling{He can see it (modal, 010 and 120)}
\test{RModalComplSentSubst1/2}
\zin{Het verbaasde me dat hij het zag }
\vertaling{(a.o.) His seeing it surprised me}
\test{RSubjNPSentSubst}
\zin{Zij bleek oud}
\vertaling{ (a.o.) She turned out old}
\test{RClosedAdjPPropSubst + TNoControlAdjp}
\zin{De tand lag op de fotocopie}
\vertaling{(a.o.) The tooth lay / was lying on the photocopy}
\test{RLocOpenPrepPPropSubst + TOblSubjControlPrepp}
\zin{Zij heeft de os contra de muziek gekregen}
\vertaling{She got / has got(ten) the ox versus / against the music}
\test{ROtherClosedPrepPPropSubst + TNoControlPrepp}
\zin{De sopraan raakte daar}
\vertaling{The soprano got / was getting over there}
\test{RLocClosedAdvPPropSubst + TNoControlAdvp}
\zin{Het verbaasde me dat hij wegging}
\vertaling{1. It surprised me that he left}
\test{RExtrapSubjSentSubst + TNoControl + TItSubjInsertion}
\vertaling{2. That he left surprised me}
\test{RLdislocSubjSentSubst + TNoControl + TThatDeletion}

\Test{7. ClauseFormation}
\zin{Zij is oud}
\vertaling{She is old}
\test{RAdjpClauseFormation}
\zin{De kerk is tegen het voorstel}
\vertaling{The church is against the proposal}
\test{RPreppClauseFormation}
\zin{Is er een dokter?}
\vertaling{Is there a doctor?}
\test{RStartExist + RExistNPClauseFormation}
\zin{Dat zijn cactussen}
\vertaling{These are cacti}
\test{RDemproIdentPl + RIdentNPClauseFormation}

\Test{8. Extraposition}
\zin{Hij dacht gisteren dat de aardappels kookten}
\vertaling{He thought / was thinking yesterday that the potatoes boiled / were 
boiling}
\test{TExtraposition1}
\zin{Ik rekende er gisteren op dat hij zou zwemmen}
\vertaling{I counted on it yesterday that he would swim}
\test{TExtraposition2 - for PrepSents}

\Test{9. Obligatory Control}
\zin{Hij beloofde hem te komen }
\vertaling{He promised him to come}
\test{RComplSentSubst + TOblSubjControlComplSent}
\zin{Hij overreedde hem te komen }
\vertaling{He persuaded him to come}
\test{RComplSentSubst + TOblObjControlComplSent}
\zin{Er werd door de actrice besloten om weg te gaan}
\vertaling{It was decided by the actress to leave}
\test{RExtrapSubjSentSubst + TExtraposition1 + TOblByobjControlComplSent}
\zin{Hij barstte in zingen uit}
\vertaling{He burst out singing}
\test{RObjNPSentSubst + TOblSubjControlOpenIngNP}

\Test{10. Non-obligatory Control}

\Test{11. VPAdvs}

\Test{12. Empty's}
\zin{Hij werd geslagen }
\vertaling{He was hit}
\test{RByEMPTYSubst}
\zin{Ze zeggen het }
\vertaling{They say so}
\test{RPrepEMPTYSubst}
\zin{Jan eet }
\vertaling{Jan eats / Jan is eating}
\test{RObjEMPTYSubst}

\Test{13. SubjOK}
\zin{Het verbaasde me dat hij het zag }
\vertaling{(a.o.) It surprised me that he saw it}
\test{TItSubjInsertion}
\zin{Er vloog een vogel door de lucht }
\vertaling{There flew / was flying a bird through the sky}
\test{TThereSubjInsertion}
\zin{Er werd naar de man gekeken }
\vertaling{The man was / was being looked at}
\test{TPrepobjToSubjRaising}
\zin{De vulkaan smelt}
\vertaling{The volcano melts}
\test{TObjToSubjRaising}
\zin{Het regent}
\vertaling{It is raining}
\test{TSubjOk}

\Test{14. Case }
\zin{Zij beschouwt hen als dieven }
\vertaling{She considers them thieves}
\test{TExceptCaseAssign + filters}

\Test{15. Argument reflexives}
\zin{Men moet zichzelf niet verlagen }
\vertaling{One should not degrade oneself}
\test{TArgReflSpelling1 + filters}

\Test{16. Reciprocals}
\zin{Zij gaven elkaar iets }
\vertaling{They gave each other something}
\test{RObjReciproSubst + filter}

\Test{17. RelMarking}

\Test{18. Shift}
\zin{Waar praatte je over met Jan? }
\vertaling{1. What did you talk to Jan about?}
\test{TWhShift}
\vertaling{2. About what did you talk to Jan?}
\test{TStrandedWhShift + RIndicWhMoodMain}
\zin{Het huis dat ik heb gekocht is mooi }
\vertaling{The house (which) I bought is beautiful}
\test{RelMarking + TWhShift}

\Test{19. Pos/Neg + NegAdaptation}
\zin{Is hij niet gekomen? }
\vertaling{Has he not come? (?Hasn't he come?)}
\test{RSentNegVar + RSentNegSubst}
\zin{Hij is wel gekomen }
\vertaling{1. He did come}
\test{RSentPosVar + RPosSubst + TNoDoDeletion}
\vertaling{2. (only in generation) He has come}
\test{RSentPosVar + RPosSubst + TPosDeletion}
\zin{Je hoeft niet te komen}
\vertaling{You need not come}
\test{RSentNEGVar + RNEGSubst + ?TDoDeletion}
\zin{Hij is nooit gekomen}
\vertaling{1. He never came / He has never come}
\test{RSentMeltnegVar + RSentMeltNegSubst + TNegAdaptation}
\zin{Hij zag niemand }
\vertaling{He saw nobody (no one) / (?)He did not see anybody (anyone)}
\test{RSentMeltnegVar + RSentMeltNegSubst + TNegAdaptation}

\Test{20. Substitution: ProSent}
\zin{Zij vraagt het zich af }
\vertaling{She wonders}
\test{RProSentSubst}
\zin{Hij denkt van wel }
\vertaling{he thinks so}
\test{RSoProSEntSubst}
\zin{Hij denkt van niet }
\vertaling{he thinks not (? he does not think so)}
\test{RNotProSentSubst}

\Test{21. Moods}
\zin{Hij zwemt }
\vertaling{He swims / He is swimming}
\test{RIndicMoodMain}
\zin{Zwemt hij? }
\vertaling{Does he swim / Is he swimming?}
\test{RIndicYesNoMoodMain}
\zin{Zwem(t)! }
\vertaling{Swim!}
\test{RImpMood}
\zin{Zij wil zwemmen }
\vertaling{She wants to swim / to be swimming }
\test{ROpenToinfMood}
\zin{Het is ongewoon dat hij zwemt }
\vertaling{It is unusual for him to swim}
\test{RFortoInfMood}
\zin{Iedereen die (nog) zwemt moet naar huis }
\vertaling{All those (still) swimming must go home}
\test{RAnterelIngMood}

\Test{22. ConjSents}
\zin{Zonder te zwemmen ging hij naar huis }
\vertaling{Without swimming he went home}
\test{RPrepIngSubsent}
\zin{Omdat hij zwom, ging ze naar huis }
\vertaling{She went home because he swam / was swimming}
\test{RConjFinSubsent}

\Test{23. VP-deletion}
\zin{Hij kocht een boek}
\vertaling{He bought a book}
\test{TVPDeletion}

\end{zinnen}

\newpage

\section{Spanish Sentences}
This is a description of the sentences used to test Spanish M-rules in the 
clause grammars. The Dutch
sentence needed as input is given, together with the different translations and
the rules these translations test. No indication has been given yet of how 
often identical translations are produced, and why.\\

\begin{zinnen}

\zin{Het regent}
\vertaling{Llueve}
\test{TredelloSUBJdel}
\zin{Jan komt}
\vertaling{Juan viene. Viene Juan.  }
\test{TPRONPVARdel}
\zin{Ik geef hem het boek. }
\vertaling{Le doy el libro (a \'{e}l).  }
\test{PronIOdel}
\zin{Ik geef jullie het boek.}
\vertaling{Os doy el libro (a vosotras.) }
\test{PronIOdel}
\zin{Ik zie haar.}
\vertaling{La veo a ella. }
\test{DOclplusPERSPRO)}
\vertaling{La veo.}
\test{DOclminusPERSPRO}
\zin{Ik zie het. }
\vertaling{Lo veo.}
\test{DOclminusPERSPRO}
\zin{Ik geef het hem. }
\vertaling{Se lo doy.}
\test{IOclCambioSE}
\zin{Ik geef ze hen.  }
\vertaling{Se las doy.}
\test{IOclCambioSE}
\zin{Er wordt gedanst.}
\vertaling{Se baila. }
\test{SEspelling}
\zin{Men danst.  }
\vertaling{Se baila. }
\test{SEspelling}
\zin{Men ziet Jan.}
\vertaling{Se ve a Juan.  }
\test{SEspelling}
\zin{Er worden boeken gekocht.  }
\vertaling{Se venden libros.}
\test{SEspelling1}
\zin{Men verkoopt boeken.  }
\vertaling{Se venden libros.}
\test{SEspelling1}
\zin{Men moet komen.  }
\vertaling{Hay que venir. }
\test{haberqueSubjdel}
\zin{Ik zie de man.}
\vertaling{Veo al hombre. }
\test{GLUE}
\zin{Het boek van de jongen.}
\vertaling{El libro del chico. }
\test{GLUE}
\zin{Hij praat met mij.}
\vertaling{Habla conmigo. }
\test{GLUE}
\zin{Hij wil haar zien.}
\vertaling{Quiere verla.  }
\test{GLUE}
\zin{Ze wil het me geven.  }
\vertaling{Quiere d\'{a}rmelo.}
\test{GLUE}
\zin{Hij wil het gedaan hebben. }
\vertaling{Quiere haberlo hecho.}
\test{GLUE}
\zin{Geef het!}
\vertaling{!Dalo!}
\test{GLUE}
\zin{Geef het me!}
\vertaling{!D\'{a}melo!  }
\test{GLUE}

\end{zinnen}

\newpage

\section{Tenses}
\subsection{Dutch main clauses}

\begin{zinnen}
\zin{Hij zwom}
\test{tense}
\zin{Hij had gezwommen}
\test{tense}
\zin{Hij zal zwemmen}
\test{tense}
\zin{Hij zou zwemmen}
\test{tense}
\zin{Hij zal gezwommen hebben}
\test{tense}
\zin{Hij zou gezwommen hebben}
\test{tense}
\zin{Hij werd geslagen}
\test{tense}
\zin{Hij was geslagen door Jan}
\test{tense}
\zin{Hij werd ziek}
\test{tense}
\zin{Hij is ziek geworden}
\test{tense}
\zin{Hij was ziek geworden}
\test{tense}

\end{zinnen}

\subsection{Dutch tense adverbials in main clauses}

\begin{zinnen}
\zin{Hij heeft gisteren gezwommen}
\test{tense}
\zin{Hij zwom gisteren}
\test{tense}
\zin{Hij komt morgen}
\test{tense}
\zin{Hij zal morgen komen}
\test{tense}
\zin{Hij komt om 3 uur}
\test{tense}
\zin{Hij kwam om 3 uur}
\test{tense}
\zin{Hij zwemt al drie uur}
\test{tense}
\zin{Hij heeft sinds gisteren drie uur (lang) gezwommen}
\test{tense}
\zin{Hij zwom drie uur lang}
\test{tense}
\zin{Hij duwde de kar drie uur lang}
\test{tense}
\zin{Het water kookte drie uur lang}
\test{tense}
\zin{Jan zat een uur in de tuin}
\test{tense}
\zin{Jan reed in een uur naar Utrecht}
\test{tense}
\zin{Hij bereikte de top in drie uur}
\test{tense}

\end{zinnen}

\subsection{Dutch embedded Clauses}
\subsubsection{complement clauses}

\begin{zinnen}
\zin{Hij zegt dat hij zwemt}
\test{tense}
\zin{Hij zei dat hij zwom}
\test{tense}
\zin{Hij zegt dat hij zwom}
\test{tense}
\zin{Hij zegt dat hij heeft gezwommen}
\test{tense}
\zin{Hij zegt te hebben gezwommen}
\test{tense}
\zin{Hij zegt gezwommen te hebben}
\test{tense}
\zin{Hij zegt gisteren te hebben gezwommen}
\test{tense}
\zin{Hij zegt geslagen te zijn}
\test{tense}
\zin{Hij probeert te zwemmen}
\test{tense}
\zin{Hij probeerde te zwemmen}
\test{tense}
\zin{Hij zegt dat hij zal komen}
\test{tense}
\zin{Hij zegt dat hij morgen zal komen}
\test{tense}
\zin{Hij zegt dat hij morgen komt}
\test{tense}
\zin{Hij zei dat hij zou komen}
\test{tense}
\zin{Hij zei dat hij morgen zou komen}
\test{tense}
\zin{Hij zei dat hij morgen kwam}
\test{tense}
\zin{Hij zei morgen te zullen komen}
\test{tense}
\zin{Hij heeft gisteren geprobeerd te zwemmen}
\test{tense}
\zin{Hij had gisteren geprobeerd te zwemmen}
\test{tense}

\end{zinnen}
\subsubsection{relative clauses}

\begin{zinnen}
\zin{Ik zie de man die ziek is}
\test{tense}
\zin{Ik zie de man die ziek was}
\test{tense}
\zin{Ik zie de man die ziek geweest is}
\test{tense}
\zin{Ik zie de man die gisteren ziek was}
\test{tense}
\zin{Ik zag de man die ziek is}
\test{tense}
\zin{Ik zag de man die ziek was}
\test{tense}
\zin{Ik zag de man die vorig jaar ziek was}
\test{tense}
\zin{Ik heb de man die ziek was gezien}
\test{tense}
\zin{Ik zie de man die drie uur heeft gezwommen}
\test{tense}
\zin{Ik zie de man die in drie uur de top bereikte}
\test{tense}
\zin{Ik zag het huis dat gisteren gekocht is}
\test{tense}

\end{zinnen}

\subsection{Tense in ADJPs}
\subsubsection{Dutch}

\begin{zinnen}
\zin{Ik zie de zieke man}
\test{tense}
\zin{Ik zie de zwemmende man}
\test{tense}
\zin{Ik zie de appels etende man}
\test{tense}
\zin{Ik heb het gekochte huis gezien}
\test{tense}
\zin{Ik heb het gisteren gekochte huis gezien}
\test{tense}

\end{zinnen}
\subsubsection{English}
\subsubsection{Spanish}

\newpage
\section{Aktionsart}
\subsection{Dutch}

\begin{zinnen}
\zin{Hij denkt}
\test{stative verbs; stative aktionsart}
\zin{Jan duwt}
\test{Aktionsart}
\zin{Jan duwt de kar}
\test{Aktionsart}
\zin{Jan reed}
\test{Aktionsart}
\zin{Jan koopt boeken}
\test{Aktionsart}
\zin{Jan duwde karren mest naar de vulkaan}
\test{Aktionsart}
\zin{Het water kookt}
\test{Aktionsart}
\zin{Hij klopte op de deur}
\test{Aktionsart}
\zin{Jan at een appel}
\test{Aktionsart}
\zin{Jan duwde de kar naar de vulkaan}
\test{Aktionsart}
\zin{Jan reed naar Utrecht}
\test{Aktionsart}
\zin{Jan houdt van Marie}
\test{Aktionsart}
\zin{Jan zit in de vulkaan}
\test{Aktionsart}
\zin{De bom ontploft}
\test{Aktionsart}
\zin{Jan sterft}
\test{Aktionsart}
\zin{Jan bereikte de top}
\test{Aktionsart}

\end{zinnen}
\subsection{English}
\subsection{Spanish}

\newpage
\section{Modals}
\subsection{Dutch}

\begin{zinnen}
\zin{Hij kan zwemmen}
\test{modal}
\zin{Hij kon zwemmen}
\test{modal}
\zin{Hij moet zwemmen}
\test{modal}
\zin{Hij moest zwemmen}
\test{modal}
\zin{Hij hoeft niet te zwemmen}
\test{modal}
\zin{Hij hoefde niet te zwemmen}
\test{modal}
\zin{Hij mag zwemmen}
\test{modal}
\zin{Hij mocht zwemmen}
\test{modal}
\zin{Hij wil zwemmen}
\test{modal}
\zin{Hij wou zwemmen}
\test{modal}
\zin{Hij zou kunnen zwemmen}
\test{modal}
\zin{Hij zou moeten zwemmen}
\test{modal}
\zin{Hij zou misschien zwemmen}
\test{modal}
\zin{Hij komt misschien}
\test{modal}
\zin{Er had gezwommen kunnen worden}
\test{modal}
\zin{Er had gezwommen moeten worden}
\test{modal}
\zin{Hij had kunnen komen}
\test{modal}
\zin{Hij zou hebben kunnen komen}
\test{modal}
\zin{Hij had gekomen kunnen zijn}
\test{modal}
\zin{Hij had moeten zwemmen}
\test{modal}
\zin{Hij zou hebben moeten zwemmen}
\test{modal}
\zin{Hij had gezwommen moeten hebben}
\test{modal}
\zin{Men moet kunnen zwemmen}
\test{modal}
\end{zinnen}

\subsection{English}
\subsection{Spanish}

\newpage

\section{PREPP(PROP)}
\subsection{Dutch}
\begin{zinnen}
\zin{de os op de os}
\test{PPinside NP; PPPsubgrammar}
\end{zinnen}

\subsection{English}
\subsection{Spanish}

\newpage

\section{NPs}
\subsection{Dutch NPs}
\subsubsection{NPs without nominal head}

\begin{zinnen}
\zin{Jan}
\test{proper names}
\zin{Duitsland}
\test{proper names}
\zin{ik/mij/mijn, hij/hem/zijn, u/uw}
\test{personal/possessive pronouns }
\zin{wie, wat}
\test{interrogative pronouns}
\zin{iets, iedereen, iemand}
\test{other pronouns}
\zin{dit/dat}
\test{other pronouns}
\end{zinnen}

\subsubsection{NPs with nominal head + determiner (no modifiers)}

\begin{zinnen}
\Test{1. simple determiner + noun}
\zin{ de man/mannen}
\test{determiner = article}
\zin{ het kind}
\test{determiner = article}
\zin{ een huis}
\test{determiner = article}
\zin{twee kinderen}
\test{determiner = numeral}
\zin{veel huizen}
\test{determiner = numeral}
\zin{deze kinderen }
\test{determiner = demonstrative}
\zin{dat project}
\test{determiner = demonstrative}
\zin{dit speelgoed}
\test{determiner = demonstrative}
\zin{die huizen}
\test{determiner = demonstrative}
\zin{mijn kinderen }
\test{determiner = possessive}
\zin{Jans huizen}
\test{determiner = possessive}
\zin{ons speelgoed}
\test{determiner = possessive}
\zin{alle kinderen}
\test{other determiners}
\zin{elk kind}
\test{other determiners}
\zin{beide huizen}
\test{other determiners}
\zin{sommige huizen}
\test{other determiners}
\zin{enkele huizen}
\test{other determiners}
\zin{verscheidene huizen}
\test{other determiners}
\zin{meer huizen}
\test{other determiners}
\zin{kaas genoeg}
\test{other determiners}

\Test{2. partitive determiners + noun }
\zin{twee van de kinderen}
\test{determiner+van+definite article}
\zin{sommige van de kinderen}
\test{determiner+van+definite article}
\zin{twee van mijn huizen}
\test{determiner+van+possessive determiner}
\zin{enkele van die huizen}
\test{determiner+van+demonstrative determiner}

\Test{3. other determiners}
\zin{meer melk dan kaas}
\test{determiners with complement}
\zin{minder dan drie kinderen}
\test{}
\end{zinnen}

\subsubsection{NPs with determiner + modified head}

\begin{zinnen}
\zin{de  zieke kinderen}
\test{modificator = adjective}
\zin{twee oude huizen}
\test{modificator = adjective}
\zin{sommige van de oude huizen }
\test{modificator = adjective}
\zin{een zwemmend kind}
\test{modificator = participle}
\zin{een door hem gekocht huis}
\test{modificator = participle}
\zin{alle honderd kinderen}
\test{modificator = numeral}
\zin{de twee oude huizen}
\test{modificator = numeral}
\zin{twee van de drie oude huizen}
\test{modificator = numeral}
\zin{het boek van Jan}
\test{modificator = PP}
\zin{de kaas op de tafel }
\test{modificator = PP}
\zin{ de kinderen die ziek zijn}
\test{modificator = clause}
\zin{mijn zus Margreet}
\test{modificator = NP (proper name)}
\zin{het project Rosetta }
\test{modificator = NP (proper name)}
\zin{de vrouw, een actrice}
\test{modificator = NP ("bijstelling") }
\zin{Margreet, mijn zus}
\test{modificator = NP ("bijstelling") }
\zin{een stuk kaas}
\test{other modifiers}
\zin{de flessen melk }
\test{other modifiers}
\zin{twee zakken oude aardappelen}
\test{other modifiers}
\end{zinnen}

\subsubsection{Nouns with a  complement}

\begin{zinnen}
\zin{de vraag of het regent}
\test{complement = clause}
\zin{het feit dat Jan ziek is}
\test{complement = clause.
 N.B. The complement sentence is not a relative sentence. Cf.
{\em de} opmerking {\em die} gemaakt werd.}
\end{zinnen}

\subsubsection{Nouns without determiner}

\begin{zinnen}
\zin{ kaas}
\test{bare mass noun}
\zin{ ossen}
\test{plural}
\zin{ speelgoed}
\test{bare mass noun}
\zin{ oude kaas}
\test{modified mass noun}
\zin{ kaas die oud is}
\test{modified mass noun}
\zin{ ossen uit Duitland}
\test{modified plurals }
\zin{flessen melk }
\test{modified plurals }
\end{zinnen}

\subsubsection{NP-external modifier}

\begin{zinnen}
\zin{ook hij, ook Rosetta }
\test{NP-external modifier}
\end{zinnen}

\subsection{English NPs}
\subsubsection{NPs without nominal head}

\begin{zinnen}
\zin{Jan}
\test{proper names}
\vertaling{ Jan }
\zin{Duitsland}
\test{proper names}
\vertaling{Germany}
\zin{ik/mij/mijn, hij/hem/zijn, u/uw}
\test{personal/possessive pronouns }
\vertaling{ I, she, him, hers, we, our, ours, yours}
\zin{wie, wat}
\test{interrogative pronouns}
\vertaling{ who, what }
\zin{iets, iedereen, iemand}
\test{other pronouns}
\vertaling{something/anything, everyone/everybody/anyone/anybody, 
someone/somebody/anyone/anybody}
\zin{dit/dat}
\test{other pronouns}
\vertaling{this, that}
\end{zinnen}

\subsubsection{NPs with nominal head + determiner (no modifiers)}

\begin{zinnen}
\Test{1. simple determiner + noun}
\zin{ de man/mannen}
\test{determiner = article}
\vertaling{the man/men}
\zin{ het kind}
\test{determiner = article}
\vertaling{the child }
\zin{ een huis}
\test{determiner = article}
\vertaling{a house}
\zin{twee kinderen}
\test{determiner = numeral}
\vertaling{two children}
\zin{veel huizen}
\test{determiner = numeral}
\vertaling{many houses }
\zin{deze kinderen }
\test{determiner = demonstrative}
\vertaling{ these children}
\zin{dat project}
\test{determiner = demonstrative}
\vertaling{  that project}
\zin{dit speelgoed}
\test{determiner = demonstrative}
\vertaling{ these toys}
\zin{die huizen}
\test{determiner = demonstrative}
\vertaling{  those boeken}
\zin{mijn kinderen }
\test{determiner = possessive}
\vertaling{ my children, the children of mine}
\zin{Jans huizen}
\test{determiner = possessive}
\vertaling{ the houses of John }
\zin{ons speelgoed}
\test{determiner = possessive}
\vertaling{ our toys }
\zin{alle kinderen}
\test{other determiners}
\vertaling{all children}
\zin{elk kind}
\test{other determiners}
\vertaling{every child}
\zin{beide huizen}
\test{other determiners}
\vertaling{ both houses}
\zin{sommige huizen}
\test{other determiners}
\vertaling{some houses}
\zin{enkele huizen}
\test{other determiners}
\vertaling{ some houses}
\zin{verscheidene huizen}
\test{other determiners}
\vertaling{ several houses}
\zin{meer huizen}
\test{other determiners}
\vertaling{ more houses}
\zin{kaas genoeg}
\test{other determiners}
\vertaling{enough cheese (cheese enough)}

\Test{2. partitive determiners + noun }
\zin{twee van de kinderen}
\test{determiner+van+definite article}
\vertaling{two of the children}
\zin{sommige van de kinderen}
\test{determiner+van+definite article}
\vertaling{some of the children}
\zin{twee van mijn huizen}
\test{determiner+van+possessive determiner}
\vertaling{two of the houses of mine (two of my houses)}
\zin{enkele van die huizen}
\test{determiner+van+demonstrative determiner}
\vertaling{ some of those houses}

\Test{3. other determiners}
\zin{meer melk dan kaas}
\test{determiners with complement}
\vertaling{more milk than cheese }
\zin{minder dan drie kinderen}
\test{}
\vertaling{ less than three children }
\end{zinnen}

\subsubsection{NPs with determiner + modified head}

\begin{zinnen}
\zin{de  zieke kinderen}
\test{modificator = adjective}
\vertaling{the ill children / the children that are ill}
\zin{twee oude huizen}
\test{modificator = adjective}
\vertaling{two old houses / two houses that are old}
\zin{sommige van de oude huizen }
\test{modificator = adjective}
\vertaling{some of the  old houses/some of the houses that are old}
\zin{een zwemmend kind}
\test{modificator = participle}
\vertaling{a child who is swimming}
\zin{een door hem gekocht huis}
\test{modificator = participle}
\vertaling{a house bought by him}
\zin{alle honderd kinderen}
\test{modificator = numeral}
\vertaling{ all hundred children}
\zin{de twee oude huizen}
\test{modificator = numeral}
\vertaling{the two old houses/ the two houses that are old}
\zin{twee van de drie oude huizen}
\test{modificator = numeral}
\vertaling{two of the three old houses }
\zin{het boek van Jan}
\test{modificator = PP}
\vertaling{the book of Jan / (Jan's book)
N.B. In English the possessive PP {\em the children of his} 
contains a possessive pronoun, whereas in the Dutch counterpart 
a personal pronoun occurs: {\em de kinderen van hem}  }
\zin{de kaas op de tafel }
\test{modificator = PP}
\vertaling{the cheese on the table / the cheese at the table}
\zin{ de kinderen die ziek zijn}
\test{modificator = clause}
\vertaling{ the children who are ill / the ill children}
\zin{mijn zus Margreet}
\test{modificator = NP (proper name)}
\vertaling{my sister Margreet}
\zin{het project Rosetta }
\test{modificator = NP (proper name)}
\vertaling{the Rosetta project}
\zin{de vrouw, een actrice}
\test{modificator = NP ("bijstelling") }
\vertaling{the woman, an actress}
\zin{Margreet, mijn zus}
\test{modificator = NP ("bijstelling") }
\vertaling{ Margreet, my sister}
\zin{een stuk kaas}
\test{other modifiers}
\vertaling{a piece of cheese}
\zin{de flessen melk }
\test{other modifiers}
\vertaling{the bottles of milk}
\zin{twee zakken oude aardappelen}
\test{other modifiers}
\vertaling{two bags of old potatoes.
N.B. In English the modifier must contain a preposition 
where in Dutch this is absent.}
\end{zinnen}

\subsubsection{Nouns with a  complement}

\begin{zinnen}
\zin{de vraag of het regent}
\test{complement = clause}
\vertaling{the question whether it is raining }
\zin{het feit dat Jan ziek is}
\test{complement = clause}
\vertaling{the fact that John is ill.
N.B. The complement sentence is not a relative sentence. Cf.
{\em de} opmerking {\em die} gemaakt werd.}
\end{zinnen}

\subsubsection{Nouns without determiner}

\begin{zinnen}
\zin{ kaas}
\test{bare mass noun}
\vertaling{cheese}
\zin{ ossen}
\test{plural}
\vertaling{oxen}
\zin{ speelgoed}
\test{bare mass nouns and plurals}
\vertaling{toys}
\zin{ oude kaas}
\test{modified mass nouns and plurals }
\vertaling{old cheese / cheese that is old}
\zin{ kaas die oud is}
\test{modified mass nouns and plurals }
\vertaling{old cheese / cheese that is old}
\zin{ ossen uit Duitland}
\test{modified mass nouns and plurals }
\vertaling{oxen from Germany / oxen out of Germany}
\zin{flessen melk }
\test{modified mass nouns and plurals }
\vertaling{ bottles of milk}
\end{zinnen}

\subsubsection{NP-external modifier}

\begin{zinnen}
\zin{ook hij, ook Rosetta }
\test{NP-external modifier}
\vertaling{he too/ he also/even he, Rosetta too/ Rosetta also/even Rosetta}
\end{zinnen}

\subsection{Spanish NPs}
\subsubsection{NPs without nominal head}

\begin{zinnen}
\zin{Jan}
\test{proper names}
\vertaling{Jan}
\zin{Duitsland}
\test{proper names}
\vertaling{ Alemania}
\zin{ik/mij/mijn, hij/hem/zijn, u/uw}
\test{personal/possessive pronouns }
\vertaling{ yo, ella, nos, nosotros, mi, m\'{i}o}
\zin{wie, wat}
\test{interrogative pronouns}
\vertaling{ qui\'{e}n, qui\'{e}nes, qu\'{e}}
\zin{iets, iedereen, iemand}
\test{other pronouns}
\vertaling{algo,  todo el mundo, alguien (nada, nadie)}
\zin{dit/dat}
\test{other pronouns}
\vertaling{esto, eso}
\end{zinnen}

\subsubsection{NPs with nominal head + determiner (no modifiers)}

\begin{zinnen}
\Test{1. simple determiner + noun}
\zin{ de man/mannen}
\test{determiner = article}
\vertaling{ el(los) hombre(s)}
\zin{ het kind}
\test{determiner = article}
\vertaling{ el ni\~{n}o}
\zin{ een huis}
\test{determiner = article}
\vertaling{una casa }
\zin{twee kinderen}
\test{determiner = numeral}
\vertaling{dos ni\~{n}os }
\zin{deze kinderen }
\test{determiner = demonstrative}
\vertaling{ estos ni\~{n}os}
\zin{dat project}
\test{determiner = demonstrative}
\vertaling{  ese proyecto/ aquello proyecto}
\zin{dit speelgoed}
\test{determiner = demonstrative}
\vertaling{ estos juguetes}
\zin{die huizen}
\test{determiner = demonstrative}
\vertaling{  esas casas/ aquellas casas}
\zin{mijn kinderen }
\test{determiner = possessive}
\vertaling{ los hijos de mi / mis hijos  }
\zin{Jans huizen}
\test{determiner = possessive}
\vertaling{las casas de Jan }
\zin{ons speelgoed}
\test{determiner = possessive}
\vertaling{nuestros juguetes de nosotros/nosotras.
N.B.  In Spanish, proper names cannot be used prenominally}
\zin{elk kind}
\test{other determiners}
\vertaling{cada ni\~{n}o}
\zin{meer huizen}
\test{other determiners}
\vertaling{ mas casas}
\zin{kaas genoeg}
\test{other determiners}
\vertaling{bastante queso}

\Test{2. partitive determiners + noun }
\zin{twee van de kinderen}
\test{determiner+van+definite article}
\vertaling{dos de los ni\~{n}os}
\zin{twee van mijn huizen}
\test{determiner+van+possessive determiner}
\vertaling{dos de las casas de mi}
\end{zinnen}

\subsubsection{NPs with determiner + modified head}

\begin{zinnen}
\zin{de  zieke kinderen}
\test{modificator = adjective}
\vertaling{los ni\~{n}os enfermos }
\zin{twee oude huizen}
\test{modificator = adjective}
\vertaling{dos casas viejas }
\zin{een zwemmend kind}
\test{modificator = participle}
\vertaling{un ni\~{n}o que nada}
\zin{de twee oude huizen}
\test{modificator = numeral}
\vertaling{las dos casas viejas}
\zin{twee van de drie oude huizen}
\test{modificator = numeral}
\vertaling{dos de las tres casas viejas}
\zin{het boek van Jan}
\test{modificator = PP}
\vertaling{el libro de Jan }
\zin{de kaas op de tafel }
\test{modificator = PP}
\vertaling{el queso en la mesa}
\zin{ de kinderen die ziek zijn}
\test{modificator = clause}
\vertaling{ los ni\~{n}os que son/sean enfermos }
\zin{mijn zus Margreet}
\test{modificator = NP (proper name)}
\vertaling{mi hermana Margreet}
\zin{het project Rosetta }
\test{modificator = NP (proper name)}
\vertaling{el proyecto Rosetta}
\zin{een stuk kaas}
\test{other modifiers}
\vertaling{un pedazo de queso}
\zin{de flessen melk }
\test{other modifiers}
\vertaling{las botellas de leche}
\zin{twee zakken oude aardappelen}
\test{other modifiers}
\vertaling{dos bolsas de patatas viejas.
N.B. In Spanish the modifier must contain a preposition 
where in Dutch this is absent.}
\end{zinnen}

\subsubsection{nouns without determiner}

\begin{zinnen}
\zin{ kaas}
\test{bare mass noun}
\vertaling{queso}
\zin{ ossen}
\test{plural}
\vertaling{bueyes}
\zin{ speelgoed}
\test{bare mass nouns and plurals}
\vertaling{juguetes}
\zin{ oude kaas}
\test{modified mass nouns and plurals }
\vertaling{ queso viejo}
\zin{ kaas die oud is}
\test{modified mass nouns and plurals }
\vertaling{queso que es viejo}
\zin{ ossen uit Duitland}
\test{modified mass nouns and plurals }
\vertaling{bueyes de Alemania}
\zin{flessen melk }
\test{modified mass nouns and plurals }
\vertaling{ botellas de leche}
\end{zinnen}

\subsubsection{NP-external modifier}

\begin{zinnen}
\zin{ook hij, ook Rosetta }
\test{NP-external modifier}
\vertaling{ tambi\'{e}n '{e}l, tambi\'{e}n Rosetta }
\end{zinnen}

\newpage

\section{Idioms, Semi-idioms and Translation Idioms}
\subsection{Idioms}
\subsubsection{Dutch Idioms}

\begin{zinnen}
\zin{Hij poetste de plaat}
\test{V + NP}
\zin{ Dit spant de kroon}
\test{V + NP}
\zin{ Hij stierf}
\test{V + NP}
\zin{ Hij begroef de strijdbijl}
\test{V + NP}
\zin{ Hij stal de show}
\test{V + NP}
\zin{ Hij zit op de wip}
\test{V + PP}
\zin{ Hij bindt de kat de bel aan}
\test{V + NP + NP}
\zin{ Hij verloor z'n geduld}
\test{V + NP, bound anaphor}
\zin{ Hij hield z'n mond}
\test{V + NP, bound anaphor}
\zin{ Hij gaf hem de zak}
\test{V + ARG + NP}
\zin{ Hij gaf haar de bons}
\test{V + ARG + NP}
\zin{ Hij liet haar in de steek}
\test{V + ARG + PP}
\zin{ Zij brak Jans hart}
\test{V + NP[ARG + n]}
\zin{ Hij heeft haar laten zitten}
\test{V + S}
\end{zinnen}

\subsubsection{English Idioms}

\begin{zinnen}
\zin{Hij poetste de plaat}
\test{V + NP}
\vertaling{He bolted, he slung his hook}
\zin{ Dit spant de kroon}
\test{V + NP}
\vertaling{ This takes the cake}
\zin{ Hij stierf}
\test{V + NP}
\vertaling{ He kicked the bucket}
\zin{ Hij begroef de strijdbijl}
\test{V + NP}
\vertaling{ He buried the hatchet}
\zin{ Hij stal de show}
\test{V + NP}
\vertaling{ He stole the show, he stole the scene}
\zin{ Hij zit op de wip}
\test{V + PP}
\vertaling{He has his job on the line}
\zin{ Hij bindt de kat de bel aan}
\test{V + NP + NP}
\vertaling{ He bells the cat}
\zin{ Hij verloor z'n geduld}
\test{V + NP, bound anaphor}
\vertaling{ He lost his cool, he lost his patience}
\zin{ Hij hield z'n mond}
\test{V + NP, bound anaphor}
\vertaling{ He held his tongue}
\zin{ Hij gaf hem de zak}
\test{V + ARG + NP}
\vertaling{ He gave him the sack, he gave him the boot}
\zin{ Hij gaf haar de bons}
\test{V + ARG + NP}
\vertaling{ He gave her the push}
\zin{ Hij liet haar in de steek}
\test{V + ARG + PP}
\vertaling{ He left her in the lurch}
\zin{ Zij brak Jans hart}
\test{V + NP[ARG + n]}
\vertaling{ She broke Jan's heart}
\zin{ Hij heeft haar laten zitten}
\test{V + S}
\vertaling{ He dumped her, he left her in the lurch}
\end{zinnen}

\subsubsection{Spanish Idioms}

\begin{zinnen}
\zin{Hij poetste de plaat}
\test{V + NP}
\vertaling{Escurr\'{\i}a el bulto, escurri\'{o} el bulto}
\zin{ Dit spant de kroon}
\test{V + NP}
\vertaling{Esto se lleva la mapa}
\zin{ Hij stierf}
\test{V + NP}
\vertaling{ Muri\'{o}}
\zin{ Hij verloor z'n geduld}
\test{V + NP, bound anaphor}
\vertaling{Perdi\'{o} / perd\'{\i}a la paciencia}
\end{zinnen}

\newpage
\subsection{Translation Idioms}
\subsubsection{English}
\begin{zinnen}
\zin{ Hij liet haar de os zien}
\vertaling{He showed her the ox, he showed the ox to her}
\test{V + S}
\zin{ Hij liet de os zien}
\vertaling{ He showed the ox}
\test{V + S}
\end{zinnen}

\subsubsection{Spanish}
\begin{zinnen}
\zin{ Hij liet haar de os zien}
\test{V + S}
\vertaling{Le mostr\'{o} el buey a \'{e}l}
\zin{ Hij liet de os zien}
\test{V + S}
\vertaling{Mostr\'{o} el buey}
\end{zinnen}

\newpage
\subsection{Semi-idioms}
\subsubsection{Dutch semi-idioms}
\begin{zinnen}
\zin{ Hij nam een bad}
\test{V + NP}
\zin{ Hij gaf een schreeuw}
\test{V + NP}
\zin{ Hij gaf hem een trap}
\test{V + NP + NP}
\zin{ Hij gaf aan hen een demonstratie}
\test{V + aan + NP + NP}
\zin{ Hij schonk aan hen aandacht}
\test{V + aan + NP + NP}
\end{zinnen}

\subsubsection{English semi-idioms}
\begin{zinnen}
\zin{ Hij nam een bad}
\test{V + NP}
\vertaling{He had a bath}
\zin{ Hij gaf een schreeuw}
\test{V + NP}
\vertaling{He gave a shout}
\zin{ Hij gaf hem een trap}
\test{V + NP + NP}
\vertaling{He gave him a kick}
\zin{ Hij gaf aan hen een demonstratie}
\test{V + aan +NP + NP}
\vertaling{He gave them a demonstration}
\zin{ Hij schonk aan hen aandacht}
\test{V + aan +NP + NP}
\vertaling{He paid attention to them}
\end{zinnen}

\subsubsection{Spanish semi-idioms}
\begin{zinnen}
\zin{ Hij nam een bad}
\test{V + NP}
\vertaling{Se daba / di\'{o} un ba\~{n}o}
\zin{ Hij gaf een schreeuw}
\test{V + NP}
\vertaling{Di\'{o} / daba un grito}
\zin{ Hij gaf hem een trap}
\test{V + NP + NP}
\vertaling{Peg\'{o} / pegaba un puntapi\'{e}}
\zin{ Hij gaf aan hen een demonstratie}
\test{V + aan +NP + NP}
\vertaling{Les hizo / hac\'{i}a una demonstraci\'{o}n}
\zin{ Hij schonk aan hen aandacht}
\test{V + aan +NP + NP}
\vertaling{Les daba / di\'{o} atenci\'{o}n a ellos/ellas}
\end{zinnen}

\end{document}
