

\documentstyle{Rosetta}
\begin{document}
   \RosTopic{Rosetta3.doc.linguistics.Dutch}
   \RosTitle{Dutch M-rules:subgrammar CNformation}
   \RosAuthor{Franciska de Jong}
%Lisette Appelo}
   \RosDocNr{482}
   \RosDate{December 13, 1991}
   \RosStatus{approved}
   \RosSupersedes{}
%concept of September 4, 1989}
   \RosDistribution{Project}
   \RosClearance{Project}
   \RosKeywords{Dutch, M-rules, CNformation, number, CNmodification}
   \MakeRosTitle
\def\ra{$\rightarrow$ }
%
%
\input{[dejong.mrules]mrudocdef}
\input{[dejong]definitions}

\section{Introduction}

Other than the subgrammars deriving clausal structures, 
the subgrammars crucial for the derivation of NPs
each pertain to a relatively isolated part of phrase structure.
This document deals with the rules accounting for the 
configuration dominated by the node CN.


A head is not obligatory for S-trees of the category NP 
but if there is one it is either a CN, a category that may be preceded by 
a determiner sisternode (DETP or otherwise), or the NP is a determinerless 
phrase headed by a pronominal expression or a proper name. 
The subgrammar NPformation generating the level NP is discussed in 
doc:r480 (FdeJ).
Together NPformation and CNformation
are the two major subgrammars involved in the generation of NPs.

Due to the relatively close relation between the derivational 
history of NPs and their S-tree structure
the division of labour between the subgrammars CNformation and NPformation 
can be pictured strightfowardly. Separate subgrammars deal with the derivation of 
characteristic input, such as DETPformation, and the various sugrammars yielding
ouput that may serve as input for the modification and complemenation of 
NOUNs.\\


\begin{verbatim}
               NP             -output of NPformation
           /       \
    detrel-
   expression        CN       -output of subgrammar CNformation
  (optional)          |
                     NOUN
                      |
                    SUBNOUN   -output of derivation-subgrammar
                      |
                     BNOUN
\end{verbatim}

\noindent
Separate documentation is available on the treatment of 
possessive modification. Cf. doc:r413 (FdeJ).\\
Separate documentation on the treatment of number can be found in section 
\ref{number}.

The subgrammar NPformation has not been attuned to any other subgrammar.

%\section{Phenomena not covered}

\section{Subgrammar Specification}

The formation of CNs is dealt with by 
the subgrammar CNformation:\\


\begin{description}
  \item[Head] SUBNOUN, EN 
  \item[Export] CN
  \item[Import] ADVP, ADJPPROP, PREPPPROP, NP, SENTENCE, NPVAR, SENTENCEVAR, 
  \item[File] dutch:npsubgrammars.mrule (mrules67)
\end{description}

\section{Control Expression}
The control expression has been defined as follows:

\begin{verbatim}

    [( RSUBNOUNTONOUN1/1 | RSUBNOUNTONOUN2/2)]

   .( RCNFORMATION1/3 | RCNFORMATION2/4 | RCNFORMATION3/5 | RCNFORMATION4/6  )

   . (RCNPresentSuperdeixis/100   | RCNPastSuperdeixis/101)

   . {RNNcompounds/28 | RVNcompounds/29}

   . [RNOUNargmod1/110 | RNOUNargmod2/111]

   . [RCNmodbareNP/23]

   . [RCNspecProperName/24]

   .{ RCNMODADJP/7       | RCNMODNUM/8    
      | RCNMODPP/12 | RCNMODADVP1/27 | RCNMODADVP2/26 | RCNMODRELSENT1/13 
      | RCNMODPOSS1/9   | RCNMODPOSS2/10 | RCNMODPOSS3/11 
      | RCNMODANTEREL1/22 | RCNmodinfrel/30
    }            

   . FCNNOUNposs/35 .  [TCNNOUNposs/34] 

   .(FPreinfrelcontrol/32)
   .{Tinfrelcontrol/31}
   .(Finfrelcontrol/33)
   . Ftempadjcheck/25

   . [Rnoundim/30]

\end{verbatim}

\section{Rules and transformations}


\begin{mruleclass}{RC\_NOUNformation}
\begin{classdescr}
\kind optional rule class (does not apply in case of an empty nominal head)
\classtask The formation of a NOUN level dominating SUBNOUN and the
assignment of a value to the attribute NOUN.number.
\classremarks

Together the two rules of this ruleclass must deal with the following three 
cases:\\
\begin{itemize}
  \item count singular
  \item mass singular
  \item count plural
\end{itemize}

\noindent
The question which alternative should apply is determined in part
by a parameter that keeps record of the number of the 
number used in the source language. If the same number value is available for 
the target language the number value is preserved under translation. 
In case of non-matching properties with respect the attributes NOUN.$posscomas$
and NOUN.$plurforms$ an alternative number assignment is available.
Separate documentation on the translation of number can be found in 
section~\ref{number}.

\nofilters

\nospeedrules

\noplannedrules

\norulesnotince

\rulelist

\end{classdescr}

\begin{members}

\begin{member}
\rulename RSUBNOUNtoNOUN1
\ruletask Introducing the level NOUN and determining the value for the 
attribute NOUN.number for 
count nouns which can occur in singular both in source and
in target language.
\file dutch:cnformation.mrule (mrules49)
\semantics number assignment
\example {\em tuin, tafel, boek, vent}
\remarks\mbox{}
NB. for Spanish and English the corresponding 
rule also deals with onlyplurals; so {\em een
schaar} is correctly mapped onto {\em scissors}
                             and {\em tijeras}. 
For Dutch no  example has been 
found yet that would motivate a similar
extension.

\end{member}
\begin{member}
\rulename RSUBNOUNtoNOUN2
\ruletask 
\begin{enumerate} 
\item Assigning plural number to nouns 
that can occur as plural including singular unitnouns.
\item Assigning singular number to mass nouns 
\end{enumerate}
\file dutch:cnformation.mrule (mrules49)
\semantics number assignment 
\example {\em tuinen,  hersenen, speelgoed, brood}
\remarks No remarks
\end{member}
\end{members}

\end{mruleclass}
\begin{mruleclass}{CNformation}
\begin{classdescr}
\kind obligatory rule class
%\kind \nokind
\classtask The formation of the syntactic level CN
\classremarks

\nofilters
\nospeedrules
\noplannedrules
\norulesnotince
\begin{comments}
\end{comments}
\rulelist

\end{classdescr}

\begin{members}

\begin{member}
\rulename CNformation1
\ruletask Making a CN out of a NOUN 
\file dutch:cnformation.mrule (mrules49)
\semantics LCNformation1
\example all NOUNs dominating a lexical noun
\remarks\mbox{}
\begin{enumerate}
\item The two subrules of this rule each apply twice to singular 
nouns with unitnoun in .subcs, such as {\em uur}. 
The CNrec1 has either [singular] as value for .numbers, or [plural].
In combination with what is said below this 
latter value is needed to guarantee that only plural determiners are 
combined with a unitnoun that has passed the plural rule RSUBNOUNtoNOUN2.
\item This rule is deviant as it involves number-assignment to a record that
in general is supposed to reflect lexical values. The strategy followed
here is more efficient than to
consider singular unitnouns ambiguous between a singular and a plural form.
There is a more elegant alternative to keep track of the number of the NP as a
whole, namely by defining a special path through the control expression that 
guarantees application of RSUBNOUNtoNOUN2 in case of a singular unitnoun  with 
a plural determiner.
\item *{\em de (vele) uur} vs. {\em de drie uur}; *{\em (vele) uur} vs. 
{\em drie uur};
"plural" unitnouns require a plural count indefinite determiner. 
\item *$uur$ vs. $uren$; "plural" unitnouns do not occur as bare plural;
\end{enumerate}


\end{member}
\begin{member}
\rulename CNformation2
\ruletask
Makes a CN (mass) out of an EN (key: massENkey)
\file dutch:cnformation.mrule (mrules49)
\semantics LCNformationmassplur
\example EN $\rightarrow$ CN[head/EN] ({\em veel} EN)
\remarks No remarks
\end{member}
\begin{member}
\rulename CNformation3
\ruletask Making a CN (count, singular) out of an EN (key: countENkey)

\file dutch:cnformation.mrule (mrules49)
\semantics LCNformationcountsing
\example EN $\rightarrow$ CN[head/EN] ({\em een EN})

\remarks No remarks
\end{member}
\begin{member}
\rulename CNformation4
\ruletask Making a CN (count, plural) out of an EN (key: countENkey)
\file dutch:cnformation.mrule (mrules49)
\semantics LCNformationmassplur
\example EN $\rightarrow$ CN[head/EN] ({\em  twee EN})
\remarks No remarks

\end{member}
\end{members}
\end{mruleclass}

\begin{mruleclass}{RC\_CNsuperdeixis}
\begin{classdescr}
\kind obligatory rule class
%\kind \nokind
\classtask Accounting for the value of CN.superdeixis
\classremarks

\nofilters
\nospeedrules
\noplannedrules
\norulesnotince
\begin{comments}
\end{comments}
\rulelist

\end{classdescr}

\begin{members}


\begin{member}
\rulename RCNPresentsuperdeixis
\ruletask In generation: set value for superdeixis at presentdeixis.\\
               In analysis: set value for superdeixis at omegadeixis.    
\file dutch:cnformation.mrule (mrules49)
\semantics LCNpresentsuperdeixis
\example all CNs
\end{member}
\begin{member}
\rulename RCNPastsuperdeixis
\ruletask In generation: set value for superdeixis at pastdeixis.\\
               In analysis: set value for superdeixis at omegadeixis.    
\file dutch:cnformation.mrule (mrules49)
\semantics LCNpastsuperdeixis
\example all CNs

\end{member}
\end{members}
\end{mruleclass}


\begin{mruleclass}{RC\_compounds}
\begin{classdescr}
\kind optional rule class
%\kind \nokind
\classtask The formation of compounds 
\classremarks

\begin{enumerate}
  \item 

This ruleclass is for robustness. 
It does not apply if a compound has an entry
in the lexicon.  (Relevant condition in morphology.)
\item 
There are special reasons why this rule class takes CNs as input.
\end{enumerate}
\nofilters
\nospeedrules
\noplannedrules
\norulesnotince
\begin{comments}
\end{comments}
\rulelist

\end{classdescr}

\begin{members}

\begin{member}
\rulename RNNcompounds
\ruletask To form a complex CN out of two CNs.
\file dutch:npcnvaria.mrule (mrules60)
\semantics LNNcompounds
\example {\em tafel} + {\em poot} \ra {\em tafelpoot}
\remarks No remarks
\end{member}

\begin{member}
\rulename RVNcompounds
\ruletask To form a complex CN out of a CN and a SUBVERB.
\file dutch:npcnvaria.mrule (mrules60)
\semantics LVNcompounds
\example {\em werk} + {\em tafel} \ra {\em werktafel}
\remarks No remarks
\end{member}
\end{members}   
\end{mruleclass}


\begin{mruleclass}{RC\_NOUNargmod}
\begin{classdescr}
\kind optional rule class
%\kind \nokind
\classtask The introduction of VARs of which the occurrence is dependent on the 
value of a non-default value for  attribute NOUN.thetanp.
\classremarks 
The name of this RC indicates the hybrid nature of the phenomenon 
it deals with.
The fact that the expressions introduced are always optionally present seems to 
argue for a modifier status, while the dependency of the NOUN attributes 
$.thetanp$ and $nounpatterns$ suggests an argument-like nature. Perhaps the 
notion of adjunct would be more appropriate. 
Cf. also doc. nr. 364 (LEXIC-document).
\nofilters
\nospeedrules
\noplannedrules
\norulesnotince
\begin{comments}
\end{comments}
\rulelist

\end{classdescr}

\begin{members}

\begin{member}
\rulename RNOUNargmod1
\ruletask To introduce an NPVAR to occur in a prepositional object.
\file dutch:npcnvaria.mrule (mrules130)
\semantics restrictive modification/complementation
\example\mbox{}
\begin{enumerate}
  \item 
{\em antwoord} $\rightarrow$ CN[head/{\em antwoord} prepobjrel/VARPREPP[op 
NPVAR]]] (het antwoord op de brief)
  \item
{\em vraag} $\rightarrow$ CN[head/{\em vraag} prepobjrel/VARPREPP[naar salami]]
 (de vraag naar salami)
\end{enumerate}
\remarks No remarks
\end{member}
\begin{member}
\rulename RNOUNargmod2
\ruletask To introduce a SENTENCEVAR as complement.
\file dutch:npcnvaria.mrule (mrules130)
\semantics restrictive modification/complementation
\example\mbox{}
\begin{enumerate}
  \item 
{\em feit} $\rightarrow$ CN[head/{\em feit } complrel/SENTENCEVAR] (dat het 
regent)
  \item
{\em vraag} $\rightarrow$ CN[head/{\em vraag} complrel/SENTENCEVAR] (of het 
regent)
\end{enumerate}
\remarks The term 'appositive' rather than complement 
is sometimes used to indicate these kinds of sentential expressions.
\end{member}


\end{members}   
\end{mruleclass}


\begin{mruleclass}{RC\_CNbareNP}
\begin{classdescr}
\kind optional rule class
%\kind \nokind
\classtask The introduction of a bare NP specifying the content of the head 
noun denotation.
\classremarks

\nofilters
\nospeedrules
\noplannedrules
\norulesnotince
\begin{comments}
\end{comments}
\rulelist

\end{classdescr}

\begin{members}
\begin{member}
\rulename RCNmodbareNP
\ruletask To specify the content of the head NOUN of the CN, by
introducing an NP in postnominal position.
\file dutch:cnformation.mrule (mrules49)
\semantics modification
\example

\begin{enumerate} 
  \item $fles$, NP[{\em melk}] $\rightarrow$ CN[head/$fles$ postmodrel/NP[$melk$]] (de fles melk
)
  \item $emmer$, NP[{\em bramen}] $\rightarrow$ CN[head/$emmer$ postmodrel/NP[$bramen$]] (twee 
emmers bramen)
  \item $uur$, NP[{\em werk}] $\rightarrow$ CN[head/$uur$ postmodrel/NP[$werk$]] (drie uur werk)
\end{enumerate}
\remarks The corresponding construction in English and Spanish have a PP rather 
than a bare NP as a postnominal  modifier. The prepositions ($van/de$)
are therefore 
introduced syncategorematically.
\end{member}
\end{members}   
\end{mruleclass}


\begin{mruleclass}{RC\_CNspecPROPERName}
\begin{classdescr}
\kind optional rule class
%\kind \nokind
\classtask The introduction of a PROPERNOUN as specifying modifier.
\classremarks
This rule class should carefully be distinghuished from RC\_NPapposition.
\nofilters
\nospeedrules
\noplannedrules
\norulesnotince
\begin{comments}
\end{comments}
\rulelist

\end{classdescr}

\begin{members}
\begin{member}
\rulename RCNspecProperName
\ruletask To add a specifying propername to a nominal head.
\file dutch:cnformation.mrule (mrules49)
\semantics restrictive modification
\example 
\begin{enumerate} 
\item
CN[project] + PROPERNOUN[Rosetta] $\rightarrow$ CN[project postmodrel/Rosetta]
({\em het project Rosetta})
\item
CN[$zus$] + PROPERNOUN[{\em Margreet}] $\rightarrow$ CN[{\em zus} 
postmodrel/{\em Margreet}]
({\em mijn zus Margreet})
\end{enumerate}
\remarks No derivation available for {\em het Rosetta project}, a form which is 
often
considered incorrect.
\end{member}

\end{members}   
\end{mruleclass}


\begin{mruleclass}{RC\_CNmodification}
\begin{classdescr}
\kind optional rule class
%\kind \nokind
\classtask The introduction of prenominal 
and postnominal modifiers that occur as daugther to CN.
\classremarks
Some of the rules can only be applied once. This is guaranteed by additonal 
constraints on relations.
\begin{filters}

\begin{members}
\begin{member}
\rulename FTempadjcheck
\ruletask To filter in generation a path in CNmodadjp for temporal adjp 
and temporal CN in which the CN did not receive the temporal attributes of the 
adjp. Note that this assignment also could have been done in a separate 
transformation of which this would be the filter.
\file dutch:cnformation.mrule (mrules49)
\end{member}
\end{members}
\end{filters}

\nospeedrules
\noplannedrules
\norulesnotince
\begin{comments}
\end{comments}
\rulelist

\end{classdescr}

\begin{members}

\begin{member}
\rulename RCNmodADJP
\ruletask Modification of a CN (with head = NOUN or EN)  by an ADJP(PROP).
\file dutch:cnformation.mrule (mrules49)
\semantics substitution/modification
\example ADJPPROP[subjrel/VAR head/ADJP] + CN[.. head/NOUN ..]
$\rightarrow$ CN[modrel/ADJP .. head/NOUN ..] 
({\em mooie boek, lange man}, etc.)
\remarks This rule may apply more than once. The modifier that is 
added last is supposed to have widest scope.
\end{member}
\begin{member}
\rulename CNmodNUM
\ruletask Making a {\em definite} CN out of a CN (with CN.definite =
omegadef) and a NUM.
\file dutch:cnformation.mrule (mrules49)
\semantics modification
\example\mbox{}

\begin{enumerate}
  \item 
CN[{\em (Spaanse) dames}] + {\em drie} $\rightarrow$ CN[modrel/{\em drie} 
(modrel/$Spaanse $) head/$dames$]
  \item
CN[{\em blonde (domme) heren}] + talloze $\rightarrow$ CN[modrel/$talloze$ 
modrel/{\em blonde} (modrel/{\em domme}) head/{\em heren}]

\end{enumerate}

\remarks\mbox{}
\begin{enumerate}
\item 
A CN into which a numerical modifier is inserted is assigned the value 
for .$definite$ that is eventually expected to be the value for the entire NP: 
$def$. This assignment must block the introduction of an indefinite determiner, 
as in *{\em vele drie heren}, *{\em drie enkele heren}. 
\item 
The solution requires that all determiners that may be preceded by a definite 
article are considered expressions of the category NUM. 
In additon to the cardinals this holds for {\em verscheidene, verschillende} (
unless it is taken as an adjective), {\em enkele}, {\em vele}.
\item 
A distinction must be made between {\em vele} and {\em veel}:\\
\begin{verbatim}
       -  veel/vele boeken.
       -  de vele boeken.
         *de veel boeken.
\end{verbatim}
Solution:     (i) $veel$ = BDET; $vele$ = NUM, or 
              (ii) $veel$ = BDET; $vele$ = BDET and  ADJ.
Soluton (i) is presently implemented. Some problems relating to the status of {
\em veel} and {\em vele} (and {\em weinig/weinige} are discussed in doc.483 
(FdeJ; in 
preparation ).
\end{enumerate}

\end{member}
\begin{member}
\rulename CNmodPP
\ruletask modification of a CN (with head = NOUN or EN) by a PREPPPROP.
\file dutch:cnformation.mrule (mrules49)
\semantics modification/substitution
\example PREPPPROP[subjrel/VAR head/PREPP] + CN[.. head/NOUN ..]
$\rightarrow$ CN[.. head/NOUN .. postmodrel/PREPP] 
({\em man met de hoed}
\remarks No derivation for temporal PREPPs as CN-modifier.
\end{member}

\begin{member}
\rulename CNmodADVP1
\ruletask modification of a CN by a non-temporal ADVPPROP.
\file dutch:cnmodification.mrule (mrules109)
\semantics modification/substitution
\example (die lange) man {\em daar}
\remarks\mbox{}

\end{member}
\begin{member}
\rulename CNmodADVP2
\ruletask modification of a CN by introduction of a temporal ADVP.
\file dutch:cnmodification.mrule (mrules109)
\semantics modification/substitution
\example (de) vergadering 
gisteren, maandag aanstaande (under the assumption that postnominal 
{\em aanstaande} 
adverbium is without flection; Cf.  het aanstaand weekend vs. het weekend 
aanstaande. 
\remarks There is no succesful  path for this kind of adverbial 
modification yet. 
The subgrammar ADVPformation is in need of refinement.
\end{member}

\begin{member}
\rulename RCNmodRELSENT1
\ruletask To substitue CNVAR in sentences by CN in order to realize 
modification of a CN by a restrictive, finite, relative SENTENCE.
\file dutch:cnmodification.mrule (mrules109)
\semantics substitution/modification 
\example\mbox{}
 
man + [x2 hij zag ] $\rightarrow$ [man [die hij zag],]\\
huis + [x2 hij zag ] $\rightarrow$ [huis [dat hij zag],]\\
{\em de stad waarvan ik droom} (in relsent RADV takes the place of CNVAR)\\
{\em de stad waar ik van droom} (in relsent RADV takes the place of CNVAR)\\
{\em de stad waar ik woon} (in relsent RADV takes the place of CNVAR; 
preposition is deleted)\\
{\em de man van wie ik droom} (in relsent WHPRO takes the place of CNVAR)\\

\remarks
\begin{enumerate}
\item 
Subrules have taken over the role of the original modrelsent1-5. 
\item 
With respect to superdeixis 
the deixis value for the relative clause is given a value in 
generation and set to omega in analysis. This is done for the surface parser. 
It is the same task as is carried out for complement sentences in the 
superdeixisadaptation transformations.\\
\item 
In addition to {\em de man van wie ik droom}, 
{\em de man waarvan ik droom} is accepted in analysis, but only the former
will be  generated.

\end{enumerate}

\end{member}

\begin{member}
\rulename CNmodposs1
\ruletask
Introduction of postnominal possessive $van$-modifiers into CNs headed by NOUN.
\file dutch:cnformation.mrule (mrules49)
\semantics modification
\example CN[$boek$] + NP[$Jan$] $\rightarrow$ CN {\em boek van Jan}; 
\remarks For comments and more examples, see 
separate documentation on possessives in doc. r413 (FdeJ).
\end{member}
\begin{member}
\rulename CNmodposs2
\ruletask
Introduction of postnominal possessive $van$-modifiers into CNs headed by EN.
\file dutch:cnformation.mrule (mrules49)
\semantics modification
\example EN + NP[$Jan$] $\rightarrow$ EN {\em van Jan}; 
relevant for the 
derivation of {\em die van Jan}.
\remarks For comments and more examples, see 
separate documentation on possessives in doc. r413 (FdeJ).
\end{member}
\begin{member}
\rulename CNmodposs3
\ruletask
Introduction of possessive modifiers that end up in prenominal position
via NPformation.  \\ \\
\file dutch:cnformation.mrule (mrules49)
\semantics modification
\example
CN[.. head/.. ..]  + NP  $\rightarrow$ CN[.. head/.. posrel/NP] \\
({\em boek ik/mijn vader} (via NPformation) $\rightarrow$ {\em 
mijn/mijn vaders boek}
\remarks For comments and more examples, see 
separate documentation on possessives in doc. r413 (FdeJ).

\end{member}

\begin{member}
\rulename CNmodanterel1
\ruletask To modify a CN by an anterelative sentence containing
a present or past participle.
\file dutch:cnformation.mrule (mrules49)
\semantics modification
\example
kind + [x1 lopend zijnd] $\rightarrow$ [lopend zijnd kind]\\
man  + [x1 z'n kinderen hatend zijnd] $\rightarrow$ [z'n kinderen hatend zijnd man]\\
ossen + [x1 door Piet gekocht ] $\rightarrow$ [door Piet gekocht ossen]\\
\remarks No remarks
\end{member}

\begin{member}
\rulename RCNmodInfrel
\ruletask to substitue CNVAR in sentences by CN; 
\file dutch:cnmodification.mrule (mrules109)
\semantics modification/substitution
\example  
de boeken {\em om te lezen} (liggen op tafel);\\
de auto's {\em om mee te spelen} (liggen op tafel);\\
de plaats {\em om te wonen} is A'dam;\\
de dag {\em om te komen} is woensdag;\\
\remarks
With respect to superdeixis the deixis value for the relative clause is given a value in 
generation and set to omega in analysis. This is done for the surface parser. 
It is the same task as is carried out for complement sentences in the 
superdeixisadaptation transformations.
\end{member}

\begin{member}
\rulename CNmodPostParticiplemod
\ruletask modification of a CN by a non-temporal participle.
\file dutch:cnmodification.mrule (mrules109)
\semantics modification
\example (die lange) man {\em komend van links}
\remarks\mbox{}
\begin{enumerate} 
  \item This rule is written, but not in the subgrammar
\item Presently the sentence grammar  cannot deal with this participle.
  \item The distribution of comma's is still to be accounted for
\end{enumerate}


\end{member}
\end{members}
\end{mruleclass}


\begin{mruleclass}{TC\_CNNOUNpos}
\begin{classdescr}
\kind optional rule class
%\kind \nokind
\classtask Accounting for the value of $.possgeni$ in analysis.
\classremarks
\begin{filters}
\begin{member}
\rulename FCNNOUNposs
\ruletask To guarantee the correct application of TCNNOUNposs in analysis.
\file dutch:npcnvaria.mrule (mrules130)
\end{member}
\end{filters}
\nospeedrules
\noplannedrules
\norulesnotince
\begin{comments}
\end{comments}
\rulelist

\end{classdescr}
\begin{members}
\begin{member}
\rulename TCNNOUNposs
\ruletask Analysis only: replacing the value $false$ 
for CN.possgeni by 
the value of the NOUN, anticipating the set of rules that in analysis 
peels off postnominal modifiers. 
\file dutch:npcnvaria.mrule (mrules130)
\semantics \nosemantics
\example CN's with a postnominal modifier:
{\em de vrouw uit de film}, {\em de vrouw die loopt}.
\remarks This transformation is a correction on the fact that the modification 
rules originally did not account for the shift of .$posgeni$.
\end{member}

\end{members}

\end{mruleclass}

\begin{mruleclass}{TC\_infrelcontrol}
\begin{classdescr}
\kind optional rule class
%\kind \nokind
\classtask 
\classremarks
\begin{filters}
\begin{members}
\begin{member}
\rulename Finfrelcontrol
\ruletask To guarantee the correct application of Tinfrelcontrol
\file dutch:npcnvaria.mrule (mrules111)
\end{member}
\end{members}
\end{filters}
\begin{speedrules}
\begin{members}
\begin{member}
\rulename Fpreinfrelcontrol
\ruletask To speed up analysis
\file dutch:npcnvaria.mrule (mrules111)
\end{member}
\end{members}
\end{speedrules}
\noplannedrules
\norulesnotince
\begin{comments}
\end{comments}
\rulelist

\end{classdescr}
\begin{members}
\begin{member}
\rulename Tinfrelcontrol
\ruletask  Deleting BIGPRO in infinitival relative sentences
\file dutch:npcnvaria.mrule (mrules131)
\semantics \nosemantics
\example {\em de boeken om te lezen}, 
{\em het mes om de kaas mee te snijden}
\end{member}

\end{members}

\end{mruleclass}

\begin{mruleclass}{RC\_noundim}
\begin{classdescr}
\kind optional rule class
%\kind \nokind
\classtask 
\classremarks
\nofilters
\nospeedrules
\noplannedrules
\norulesnotince
\begin{comments}
\end{comments}
\rulelist

\end{classdescr}
\begin{members}
\begin{member}
\rulename Rnoundim
\ruletask Deriving a diminutive form for a noun. 
\file dutch:npcnvaria.mrule (mrules60)
\semantics modification
\example {\em boek} \ra {\em boekje}
\remarks In Spanish and English the corresponding rule introduces an adjective 
expressing diminutivity.
\end{member}

\end{members}

\end{mruleclass}
\section{The translation of number}
\label{number}

The main issue to solve is the fact that number is not necessarily preserved 
under
translation: some singular NPs may translate into plurals and vice versa.
Pending a more extensive documentation on this subject, a pointwise 
"summary" of the most important assumptions for the treatment of number is 
given here, applied to Dutch and English.

TRANSFER\\ 

\begin{tabular}{|lllll|} \hline
Dutch && IL & &  English \\ \hline \hline

singular & $\leftrightarrow$  & S  & $\leftrightarrow$ & singular \\ \hline
 
plural/mass & $\leftrightarrow$  &  PM  & $\leftrightarrow$  & plural/mass
 \\ \hline
\end{tabular}\\ \\

CRUCIAL ATTRIBUTES:
\begin{description}
\item[.number] for NOUN; values: singular, plural
\item[.pluralforms] for SUBNOUN;  especially the values Noplural and OnlyPlural 
(Dutch) and singandplur (English)
\item[.posscomas] for  SUBNOUN; values: [count], [mass]
([count] for nouns that combine with the indefinite article {\em een}, [mass] 
for the other nouns)
\end{description}


EXAMPLES\\ 

\begin{tabular}{|l|l|l|} \hline
boek          & S &  book\\        
boeken       & PM &    books\\ \hline

vent               & S &     bloke \\
kerels              & PM &      blokes\\ \hline

schoen              & S&    shoe\\
schoeisel, schoenen              & PM &     shoes\\ \hline
               
politie (noplur, mass) & PM & police (onlyplur, mass)\\ \hline

speelgoed,speeltjes              & PM &    toys\\ 
?speeltje              & S & toy  \\ \hline

melk              &PM &    milk\\ \hline

kaas1              & PM &     cheese1\\
kaas2              & S &     cheese2\\ \hline

brood1              & PM  &    bread1\\
brood2              &  S &    (loaf of bread2)\\
broden2             & PM &    (loaves of bread2)\\ \hline

hersenen1 (onlyplur, mass)               & PM &   brains1  \\
((een stel) hersenen2)                   & S &   (a) brain2 \\
((twee stel) hersenen2)              & PM &    (two) brains2\\ \hline
\end{tabular}


\end{document}
