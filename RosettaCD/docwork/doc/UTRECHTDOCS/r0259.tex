\documentstyle{Rosetta}
\begin{document}
   \RosTopic{Linguistics}
   \RosTitle{Notulen Linguistenvergadering 29-3-88 en 5-4-1988} 
   \RosAuthor{Margreet Sanders}
   \RosDocNr{0259}
   \RosDate{5 april 1988}
   \RosStatus{informal}
   \RosSupersedes{-}
   \RosDistribution{Linguists, Joep Rous}
   \RosClearance{Project}
   \RosKeywords{minutes, ALL, POSITIVE, N1, QUOTE, BIGPRO, RADV}
   \MakeRosTitle

\noindent
{\bf Verslag linguistenvergadering 29 maart 1988}
\begin{description}
\item[Aanwezig:] Jan Odijk, Lisette Appelo, Elly van Munster, Harm Smit, 
Andr\'{e} Schenk, Margreet Sanders (not)
\item[Afwezig:] Franciska de Jong
\item[Agenda:] \mbox{}
\begin{enumerate}
\item Woordenboek
\item Regard as
\item Het Probleem van de drie mu's
\item Nieuwe mogelijkheden in M-regelnotatie
\item QUOTE
\item Controle van BIGPRO
\end{enumerate}
\end{description}

\section{Woordenboek}
Harm deelt een stencil uit met de `Tien Geboden voor Woordenboekgebruik'. 
Conclusie: verander zo weinig mogelijk aan een woordenboek zonder dat Harm het 
weet.

\section{Regard as}
Margreet heeft een probleem met het tweeplaatsige Engelse werkwoord `regard as' 
(met een syncategorematisch geintroduceerd voorzetsel `as') in de controle 
regels: i.t.t.\ de situatie voor gesloten XPPROPs is er voor gesloten 
complementzinnen in een PREPP geen regel die het ingebedde
subject verandert in een object in de hogere zin, zodat er een verkeerde surface 
volgorde ontstaat: {\em I regard [as [he being my closest friend]]\/} (vgl.\ 
{\em
I count [on [the weather being fine]]\/}, waar wel de goede surface-volgorde 
uitkomt). 
Dit komt in verdere regels niet meer goed. De simpelste oplossing is om `regard 
as' drieplaatsig te maken, met een open complement ({\em x1 regard x2 as [x2 
being my closest friend]\/}. Aangezien dit niet de vertaalproblemen schijnt op 
te leveren die aanvankelijk de reden waren om `regard as' tweeplaatsig te 
noemen, wordt `regard as' nu voortaan drieplaatsig.

\section{Het Probleem van de drie mu's}
Jan O.\ waarschuwt 
voor regels die met drie achtereenvolgende mu's werken: een formulering als: \\
SENTENCE\{\}[ I1:mu1, I2:mu2, I3mu3 ]\\
I1: NOT EXIST (mu1, [I11: rel1/..]) \ \ \  I11: rel1 IN AUX\_insideVPrels\\
I2: ALL (mu2, I21:rel2/..) \ \ \ \ \ \ \ \ \ \ I21: rel2 IN AUX\_insideVPrels\\
I3: NOT EXIST (mu3, [I31: rel3/..]) \ \ \  I31: rel3 IN AUX\_insideVPrels\\
stopt in geval van een lege mu2 automatisch alles in mu3, aangezien de parser 
van links naar rechts probeert alles zoveel mogelijk voor zich uit te schuiven 
en stopt bij de eerste gevonden match. (Dit geval doet zich voor als er uit de 
surface parser een SENTENCE is gekomen waar geen VERBPknoop in 
herkend kon worden, en dat alsnog in de Mparser geregeld moet worden. Er 
bestaan dan zinnen waar mu2 leeg is, en het de bedoeling is een shiftrel in mu1 
te stoppen en een conjrel in mu3. Beide relaties zitten niet in InsideVPrels). 
De oplossing is natuurlijk om de restricties op mu1 en mu3 explicieter en niet 
overlappend te formuleren, liefst met een positieve conditie i.p.v. een NOT 
EXIST.

\section{Nieuwe mogelijkheden in M-regelnotatie}
\subsection{ALL}
In principe zijn momenteel regels van de vorm \\
m: SENTENCE{}[mu1, rel1/B, mu2] \\
m1: SENTENCE{}[mu1, rel1/D, mu2]\\
niet echt omkeerbaar, daar generatief altijd de meest linkse B omgezet wordt in 
D (onafhankelijk van of er misschien al Ds voorkwamen in de mu's), maar 
analytisch altijd de meest linkse D omgezet wordt in B. Als je wilt dat 
{\em alle\/} mogelijke Ds uit het 
rechter model omgezet worden in Bs kun je nu in de matchconditie gebruik maken 
van de notatie:\\
m : ALL (in het hoofdmodel)\\
T1: ALL (in subregels)\\
Dat betekent dus dat {\em alle\/} matches die mogelijk zijn worden 
uitgeprobeerd, dus niet alleen de meest linkse D wordt omgezet, maar er komt 
ook een boom uit waar deze D in mu1 is opgenomen en een volgende D aan de beurt 
komt. Als je niet wilt dat letterlijk alle mogelijke lijsten worden 
uitgeprobeerd, moet je het `invariante' gedeelte (met de mu's die niet van 
invloed zijn op de omzetting) verplaatsen naar een subregel 
als je de ALL op je hoofdmodel wilt toepassen, of het `variante' gedeelte naar 
een subregel verplaatsen en de ALL alleen in die subregel declareren.
Let op: ALL produceert een heleboel output als de mu's een beetje groot zijn, 
dus gebruik hem alleen als het echt nodig is.

\subsection{POSITIVE}
Filters werken als volgt: als een input boom voldoet aan het model dat in de 
filter beschreven is, wordt de betreffende boom {\em tegengehouden\/}. Alle 
andere bomen 
worden doorgelaten. Er is nu een mogelijkheid om {\em positieve\/} filters 
te defini\"{e}ren, die een boom alleen {\em doorlaten\/} als hij aan de 
beschrijving voldoet; andere bomen worden tegengehouden. Deze positieve filters 
worden gedeclareerd door het woord POSITIVE v\'{o}\'{o}r het woord FILTER in de 
naam van de regel te zetten.

\subsection{Variabele keys}
Het is nu mogelijk om in het hoofdmodel de key van een bepaalde basiscategorie 
als een variabele op te geven, en pas later uit te spellen om 
welke specifieke keys het gaat, dus bv.\ in de matchcondities:\\
NP\{\}[ detrel/I1::ART(key1){ARTrec1}, mu2 ]\\
I1: key1 = AUX\_deARTkey OR key1 = AUX\_hetARTkey\\
Nota bene: een formulering in de matchcondities met `key1 IN [..]' is {\em niet
\/} toegestaan, daar dat niet in Pascal kan worden ge\"{i}mplementeerd. Let bij 
het schrijven van zulke regels wel op de omkeerbaarheid. De variabele `key1' 
kan verder in de regel naar believen gebruikt worden, dus ook in de CA-paren.

\subsection{N1}
Ren\'{e} gaat nog implementeren dat in het hoofdmodel een N1 gebruikt kan 
worden, waarnaar later verwezen kan worden met N1.REC.attribuut. Idem voor 
T1.REC = T2.REC.

\section{QUOTE}
Jan O. heeft voor het Nederlands een aantal nieuwe functies toegevoegd aan de 
files LSMRUQUO: 
\subsection{CheckAktArts en AssignAktArts}
Deze functies controleren c.q.\ berekenen de aktionsarts van een verb of 
sentence (eerste argument van de functie) op basis van de aktionsarts van het 
verb dat als tweede argument 
meegegeven wordt aan de functie. In het Nederlands is 
dit nodig bij Verb Raising.

\subsection{AssignEform}
Deze functie kent een eForm toe aan adjectieven.

\subsection{SrelPrec, VPrelPrec, ADJPrelPrec}
Deze functies kijken of de relatie die als eerste argument wordt meegegeven aan 
de functie `voorafgaat' (prec = precede) aan de relatie die als tweede argument 
is meegegeven. Hierbij wordt aangenomen dat de volgorde van relaties in 
LSMRUQUO.pas is gedefini\"{e}erd, bv.\ shiftrel $<$ subjrel $<$ tempadvrel, 
locadvrel, preadvrel $<$ predrel. Onderling niet ge\"{o}rdende relaties staan 
dus, door komma's gescheiden, samen tussen de < van de rest. Vergelijking van 
zulke `gelijkwaardige' posities onderling levert altijd {\em true\/} op. \\
Puncrel en gluerel 
zijn niet opgenomen in deze hi\"{e}rarchie. Vergelijkingen van een relatie met 
een van deze twee levert ook altijd {\em true\/} op.\\
Voorbeelden: I1: ALL (mu1, I11:rel1/..) \ \ \ I11: QUOTE\_VPrelPrec(rel1, 
objrel)\\
of: I11: QUOTE\_VPrelPrec(objrel, rel1)\\
waarbij gecontroleerd wordt of rel1 voorafgaat aan resp.\ volgt op objrel. Let 
op: `voorafgaan' moet in ruime zin worden gezien, dus niet als `direct 
voorafgaan'.

In het Nederlands is de definitie van de volgorde niet goed stringent te maken, 
omdat {\em tempadvrel\/} zowat overal tussen kan staan.

Verder zijn er vier functies al wel beschreven in LSMRUQUO.env, maar nog niet 
ge\"{i}mplementeerd in de .pas-versie: {\em firstcat, firstrel, lastcat, 
lastrel}. Deze functies vergelijken of de eerste 
categorie of relatie in een mu (eerste argument van de functie) gelijk is aan 
een gespecifieerde waarde (tweede argument).

Ren\'{e} zal begin april iets implementeren waardoor bij simpele toevoegingen 
(niet: bij veranderingen!) in het auxdomein of in LSMRUQUO.env niet alle 
M-regels opnieuw gecompileerd hoeven te worden.

\section{Controle van BIGPRO}
In normale controle wordt een variabele in de ingebedde zin gedeleerd op basis 
van volstrekte identiteit met 
een andere variable in de hoofdzin. Bij BIGPROs vindt controle plaats op basis 
van andere condities dan volstrekte identiteit, bv.\ het antecedent en de 
BIGPRO moeten beide eerste persoon zijn. Aangezien de controle-regels 
transformaties zijn, is het theoretisch mogelijk dat in de doeltaal een ander
antecedent wordt gevonden dan in de brontaal aanwezig was. Hiermee is dan een 
verkeerde relatie van antecedenten gelegd. Mocht de ingebedde zin een reflexief 
bevatten, dan komt deze verkeerde relatie ook aan de oppervlakte.

Om dit te voorkomen stelt Jan O. voor om de controle-regels die met een BIGPRO 
werken niet als transformaties te schrijven maar als echte regels, die twee 
parameters nemen: de index van het antecedent en de 
index van de BIGPRO. Deze laatste index moet dus toegevoegd worden als 
attribuut aan BIGPRO (NIET als het attribuut INDEX dat bij variabelen voorkomt, 
maar bv.\ als `BIGPROindex'
met het type `indextype'). In de controle-regels ligt dan vast welke relatie 
gelegd moet worden.
In de substitutie-regels moet de index van de BIGPRO generatief weer op 0 gezet 
worden, en analytisch op de waarde van LEVEL.

\newpage
\noindent
{\bf Verslag linguistenvergadering 5 april 1988}
\begin{description}
\item[Aanwezig:] Jan Odijk, Lisette Appelo, Elly van Munster, Harm Smit, 
Andr\'{e} Schenk, Franciska de Jong, Margreet Sanders (not)
\item[Afwezig:] --
\item[Agenda:] \mbox{}
\begin{enumerate}
\item Shift en Substitutie
\end{enumerate}
\end{description}

\section{Shift en Substitutie}
Jan O.\ legt het probleem uit dat `er' veroorzaakt als er eerst shiftregels 
zijn en pas daarna substitutie:\\
als `er' complement is bij een voorzetsel mag het niet meteen uit de PREPP naar 
bv.\ een shiftrel, maar moet het eerst naar een speciale {\em erposrel\/} net buiten 
de PREPP, omdat als 
er meerdere {\em er\/}s voorkomen, deze soms moeten versmelten: {\em Er keek 
iemand naar\/}
heeft maar \'{e}\'{e}n {\em er\/}, dat zowel expletief is als 
prepositioneel. In andere gevallen geldt dat aanwezigheid van een bepaalde `er' 
introductie van nog een `er' blokkeert, bv.\ bij locatieven. Ditzelfde verhaal 
geldt ook voor andere RADVs als `waar'. Als nu 
de prepositionele `er' al geshift is uit de erposrel ({\em Waar keek hij 
naar?}), is de erposrel 
weer leeg, en kan er in de substitutie regels nog een locatief `er' neergezet 
worden, wat niet de bedoeling is:
{\em Waar keek hij er gisteren naar?\/} mag alleen voorkomen in de 
interpretatie dat {\em Waar\/} de locatief is, en niet de NPVAR die 
vanuit de PREPP 
is verplaatst. Er zijn twee oplossingen mogelijk: introduceer RADVPVARs (voor 
locatieven) al v\'{o}\'{o}r 
de shiftregels in de erposrel, zodat er geen `er' meer uit de PREPP kan worden 
verplaatst omdat deze plaats al bezet is, of laat een spoor achter van een 
geshifte `er'. Dit spoor bestaat dan uit een lege categorie RADVP, die niets 
domineert en een kopie is van de geshifte RADVPVAR. Deze laatste oplossing is 
voorlopig 
ge\"{i}mplementeerd. 

NPs in een PREPP die RADV moeten worden (bv.\  het 
syncategorematisch ge\"{i}ntro\-duceerde {\em het\/} van een ge\"{e}xtraponeerde 
zin: {\em Ik reken op het dat hij komt\/}) worden al binnen de PREPP omgezet 
in een RADV.

Het lijkt erg lastig om RADVs alle attributen van gewone ADVs te laten 
meeslepen. Daarom is de basiscategorie RADV ingevoerd. Echter, de derivaties in 
verschillende talen moeten wel isomorf blijven. Daarom is er voor het 
Nederlands in de derivatie-subgrammatica een regel toegevoegd die een RADV 
omzet in een SUBADV (dus naast de BADV - SUBADV regel). SUBADV en alle hogere 
ADV categorie\"{e}n hebben nu het 
attribuut `radvb' (`b' van BOOLEAN) erbijgekregen om aan te geven wat hun hoofd 
is.

Er is nu nog een probleem met het kwantitatieve `er', dat niet in de erposrel 
staat maar in de erqrel. In de substitutie-regels kan voor een NPVAR 
een NP met een +count EN head worden ge\"{i}ntroduceerd, mits er ook meteen 
het kwantitatieve `er' in de betreffende zin wordt gezet:
{\em Hij kocht er $_{NP}$[twee]\/}. Als de NPVAR echter geshift is naar een 
hogere zin,
is niet meer duidelijk in welke zin nu het kwantitatieve `er' 
thuishoort: {\em Hoeveel dacht je dat hij er kocht?\/}. Daarom moeten NPVARs 
een nieuwe attribuut-waarde van het attribuut NPhead
erbij krijgen: enNP. Bij het shiften van een +count en-NPVAR moet meteen de 
erqrel gevuld worden, en wordt de waarde van NPhead verzet naar enOKNP.
Let op: dit is dus toekenning van een 
attribuutwaarde aan een variabele! Volgens het nieuwe voorstel van Joep 
voor variabelen (doc.\ 258) is dat voortaan toegestaan.

\end{document}

