\documentstyle{Rosetta}
\begin{document}
   \RosTopic{General}
   \RosTitle{Notulen groepsvergadering 25-2-1991}
   \RosAuthor{Franciska de Jong}
   \RosDocNr{460}
   \RosDate{25-2-1991}
   \RosStatus{approved}
   \RosSupersedes{-}
   \RosDistribution{Project}
   \RosClearance{Project}
   \RosKeywords{Notulen}
   \MakeRosTitle
%
%
\begin{description}
\item[Aanwezig:] Lisette Appelo,
Franciska de Jong, Ren\'{e} Leermakers,  
                 Jan Landsbergen (voorzitter),   
                 Elena Pinillos, 
                 Jan Odijk, 
                 Joep Rous,
                 Andr\'{e} Schenk,
                 Harm Smit,
                 Frank Uittenbogaard 
                  
                  

\item[Afwezig:]  
Petra de Wit

\item[Agenda:]\mbox{}
  \begin{enumerate}
  \item Notulen
  \item Actiepunten
  \item Diversen
  \item Rondvraag
  \end{enumerate}
\end{description}

\section{Notulen}
De notulen van de vorige keer worden met enkele wijzigingen goedgekeurd.

\section{Afgeronde actiepunten}

Afgeronde actiepunten:
\begin{enumerate}
\item Jan O. heeft het Rosetta-document ten behoeve van Lexic verspreid.
\item Alle auteurs zijn begonnen aan hun boek-bijdrage.
\end{enumerate}

\section{Diversen}
\begin{description}
\item {\bf ACL}\\
Zowel Petra als Elena gaan op kosten van het STT naar de ACL in Berlijn.

\item {\bf Conferenties in 1992}\\
Jan wijst op de aankondiging van twee conferenties in 1992: de Coling-92 
(deadline voor abstracts: 1 november 1991) en de 
3rd Conference on Applied Natural Language Processing
(deadline voor abstracts: 10 september 1991).

\item {\bf Bezoek}

\begin{enumerate}
  \item Tijdens het bezoek van Arthur Dirksen van 21 februari jl. is een aantal 
onderwerpen wegens tijdgebrek niet aan bod gekomen. 
Tijdens het aanstaande bezoek van de groep Landsbergen aan het IPO zal
Andr\'{e} een afspraak maken om te praten over grammar checkers.
  \item Op donderdag 7 maart  komt Martin Everaert met een aantal mensen uit 
Utrecht op bezoek. 
  \item Op donderdagochtend  
4 april komt Wietske Sijtsma met een aantal mensen van het MMC-project  
(Tilburg) op bezoek. 
  \item 
Op vrijdagmiddag 12 april komt Angeliek van Hout met een aantal mensen uit 
Tilburg op bezoek
\end{enumerate}

\item {\bf IPO-werkbezoek}\\
Het werkbezoek van de groep Landsbergen aan het IPO op 27 februari moet volgens 
Jan L. onder meer 
als doel hebben om aanknopingspunten te vinden voor gezamenlijk onderzoek 
in de toekomst. De discussies die op het programma staan zouden  
met dat idee op de achtergrond gevoerd moeten worden.

\item {\bf Uit de redactie}\\
Op donderdag 28 februari is er een redactievergadering met Theo Jansen erbij. 
\item {\bf Evaluatiedag}\\
De evaluatiedag is definitief vastgesteld op donderdag 21 maart. Het programma 
begint 
met het oog op de treinreizigers
om 9.15 uur. Suggesties voor thema's zijn nog steeds welkom. 
\item {\bf CE}\\
In het kader van de plannen voor de ontwikkeling van 
lingu\"{\i}stische software wordt bij CE overwogen om onder meer 
een studie op te zetten
van bestaande grammar checkers. 
Een definitieve beslissing hierover wordt binnenkort
genomen. 

Onze  kennis van 
spelling checkers is 
onlangs door Joep overgedragen aan de heer Estl (Wenen). 

\item {\bf Cursus Bedrijfskunde}\\
Ren\'{e} wordt verzocht zich aan te melden voor de cursus die in het najaar van 
start gaat. Lisette, Jan O. en Frank kunnen een inschrijfverzoek 
verwachten voor de cursussen van 1992.

\item
{\bf Conversie}\\
De conversie is opgesplitst in drie periodes. Voor iedere periode is een apart 
contract afgesloten. Na de tweede periode, die afloopt in april, zou Rosetta 
onder UNIX in de batch moeten kunnen draaien.

\item
{\bf Nieuwe Activiteiten}\\
Jan L. memoreert 1 maart als de startdatum voor
de twee nieuwe activiteiten binnen de groep: brievengenerator en 
grammar checking. 
 In de toekomst zal tijdens de
groepsvergaderingen ruimte worden ingeruimd om elkaar op de hoogte te houden 
van de voortgang van de drie onderzoeksthema's.
Voor elk onderzoeksthema zal een apart documentatiesysteem 
moeten komen.  
\end{description}

\section{Rondvraag}
Er zijn geen vragen en/of opmerkingen.

\section{Actiepunten}

Alle actiepunten zijn afgerond.

\section{Volgende vergadering}
De volgende vergadering is op maandag 25 maart, 13.30 uur in WY7.\\
\end{document}


