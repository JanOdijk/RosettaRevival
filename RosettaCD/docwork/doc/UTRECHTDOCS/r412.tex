\documentstyle[12pt]{article}

\pagestyle{empty}
\setlength{\parindent}{0in}
\setlength{\textheight}{57pc}
\setlength{\textwidth}{35pc}
\setlength{\topmargin}{-2pc}
\setlength{\oddsidemargin}{0.6cm}
\renewcommand{\baselinestretch}{1}
\begin{document}
\section*{De vertaling van "getal"}

TRANSFER\\ 

\begin{tabular}{|lllll|} \hline
Dutch && IL & &  English \\ \hline\hline

singular & $\leftrightarrow$  & S  & $\leftrightarrow$ & singular \\ \hline
 
plural/mass & $\leftrightarrow$  &  PM  & $\leftrightarrow$  & plural/mass
 \\ \hline
\end{tabular}\\ \\

CRUCIALE ATTRIBUTEN:
\begin{description}
\item[.number] voour NOUN; waarden: singular en plural
\item[.pluralforms] voor SUBNOUN;  met name de waarden Noplural en OnlyPlural
\item[.posscomas] voor SUBNOUN; waarden [count] en [mass]
([count] voor de nouns die met het indefiniete
 lidwoord {\em een} combineren, [mass] voor de overige nouns)
\end{description}


VOORBEELDEN\\

\begin{tabular}{|l|l|l|} \hline
Dutch         & IL  &                    English \\ \hline\hline

boek          & S &  book\\        
boeken       & PM &    books\\ \hline

vent               & S &     bloke \\
kerels              & PM &      blokes\\ \hline

schoen              & S&    shoe\\
schoeisel, schoenen              & PM &     shoes\\ \hline
               
politie (noplur, mass) & PM & police (onlyplur, mass)\\ \hline

speelgoed,speeltjes              & PM &    toys\\ 
?speeltje              & S & toy  \\ \hline

melk              &PM &    milk\\ \hline

kaas1              & PM &     cheese1\\
kaas2              & S &     cheese2\\ \hline

brood1              & PM  &    bread1\\
brood2              &  S &    (loaf of bread2)\\
broden2             & PM &    (loafs of bread2)\\ \hline

hersenen1 (onlyplur, mass)               & PM &   brains1  \\
((een stel) hersenen2)                   & S &   (a) brain2 \\
((twee stel) hersenen2)              & PM &    (two) brains2\\ \hline
 
\end{tabular}
batim}

\end{document}
