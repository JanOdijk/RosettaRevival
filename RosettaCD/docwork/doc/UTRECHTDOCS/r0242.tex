\documentstyle{Rosetta}
\begin{document}
   \RosTopic{Linguistics}
   \RosTitle{Notulen Linguistenvergaderingen 12-11-'87 en 19-11-'87}
   \RosAuthor{Margreet Sanders}
   \RosDocNr{0242}
   \RosDate{November 20, 1987}
   \RosStatus{concept}
   \RosSupersedes{-}
   \RosDistribution{Linguists, Joep Rous}
   \RosClearance{Project}
   \RosKeywords{minutes}
   \MakeRosTitle
%
%
\section{Linguistenvergadering 12 november}

\begin{description}
\item [Aanwezig:] Jan, Harm, Andr\'{e}, Elly, Franciska, Lisette en Margreet 
(not.)
\end{description}

\noindent
Drie mededelingen van Jan O.:
\begin{enumerate}
  \item Er is een nieuw woordenboek-file toegevoegd: {\bf CONJ.dict}
  \item Bij het {\bf vullen van attributen} in de woordenboeken moet je met je 
initialen bijhouden wat je gevuld hebt
  \item De categorie SENTENCE onder CONJSENT kan niet afgeschaft worden. Dit 
vanwege de 
surface-parser, die SENTENCE expliciet gebruikt in zijn reguliere expressie. 
Afschaffing leek handig, omdat 
er dan niet een extra attribuut hoeft te zijn voor SENTENCE om het 
voegwoord dat onder SENTENCE staat, en dat bekend moet zijn voor de surface 
parser, te specifi\"{e}ren.
Daar dit dus niet blijkt te kunnen, wordt nu maar de {\bf CONJSENT-knoop 
afgeschaft}. De definitie van SENTENCE wordt
aangepast met het extra attribuut voor het voegwoord, en met de mogelijk een 
eventuele PREP (die eerst pas onder de CONJSENT kwam) 
meteen onder de SENTENCE te laten vallen. 
\end{enumerate}

\section{linguistenvergadering 19 november}

\begin{description}
\item [Aanwezig:] Jan, Harm, Andr\'{e}, Franciska, Lisette en Margreet (not.)
\item [Afwezig:] Elly
\end{description}

Wederom drie mededelingen van Jan Odijk:
\begin{enumerate}
  \item Omdat veranderingen in het {\bf (aux)domein} heel veel integratie-tijd kosten 
(meer dan 12 uur), moeten wijzigingen worden opgespaard bij \'{e}\'{e}n nieuw te 
cre\"{e}ren user en bv.\ {\bf alleen in het weekend geintegreerd} worden.
  \item De {\bf syntax van het auxdomein} is veranderd: er hoeft geen key meer 
gedefi\-nieerd te worden. In plaats daarvan moet nu alleen een constante-naam 
worden gedefinieerd (bv.\ {\em kommapunckey\/}), die dan tevens in het 
betreffende woordenboek op de juiste plaats moet worden gedeclareerd (bv.\ , : 
30400 = {\em kommapunckey\/})
  \item Carel is het programma {\bf WEB} aan het uittesten. Dit programma maakt het 
mogelijk in \'{e}\'{e}n file een Pascal-programma te combineren met de 
bijbehorende \LaTeX-documentatie. Voordeel hiervan is dat de documentatie 
direct gekoppeld is aan de source ervoor. Het maken van een dergelijk programma 
voor andere talen dan 
Pascal lijkt mogelijk en zinvol. Bekijken van het resultaat van zo'n combi-file 
in verschillende windows is in principe al tijdens 
het cre\"{e}ren ervan mogelijk,
mits de juiste apparatuur beschikbaar is.
\end{enumerate}


\end{document}
