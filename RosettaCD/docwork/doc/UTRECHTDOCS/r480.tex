% include  eof
\documentstyle{Rosetta}
\begin{document}
   \RosTopic{Rosetta3.doc.linguistics.Dutch}
   \RosTitle{Dutch M-rules:subgrammar NPformation}
   \RosAuthor{Franciska de Jong}
%Lisette Appelo}
   \RosDocNr{480}
   \RosDate{December 6, 1991}
   \RosStatus{approved}
   \RosSupersedes{}
%concept of September 4, 1989}
   \RosDistribution{Project}
   \RosClearance{Project}
   \RosKeywords{Dutch, M-rules, NPformation}
   \MakeRosTitle
%
%
\input{[dejong.mrules]mrudocdef}
\input{[dejong]definitions}
\hyphenation{syn-cate-go-re-ma-tic}
\section{Introduction}

Other than the subgrammars deriving clausal structures, 
the subgrammars crucial for the derivation of NPs
each pertain to a relatively isolated part of phrase structure.
This document deals with the rules accounting for the 
configuration dominated by the node NP.

A head is not obligatory for S-trees of the category NP 
but if there is one it is either a CN, a category that may be preceded by 
a determiner sisternode (DETP or otherwise), or the NP is a determinerless 
phrase headed by a pronominal expression or a proper name. 
The subgrammar CNformation generating the level CN is discussed in 
doc:r482 (FdeJ; in 
preparation).  
Together NPformation and CNformation
are the two major subgrammars involved in the generation of NPs.

Due to the relatively close relation between the derivational 
history of NPs and their S-tree structure
the division of labour between the sgs CNformation and NPformation 
can be pictured strightfowardly. Separate sgs deal with the derivation of 
characteristic input, such as DETPformation, and the various sgs yielding
ouput that may serve as input for the modification and complemenation of 
NOUNs.\\


\begin{verbatim}
               NP             -output of NPformation
           /       \
    detrel-
   expression        CN       -output of sg CNformation
  (optional)          |
                     NOUN
                      |
                    SUBNOUN   -output of derivation-subgrammar
                      |
                     BNOUN
\end{verbatim}

\noindent
Separate documentation is available on the treatment of 
possessive modification. Cf. doc:r413 (FdeJ).\\
Documentation on the treatment of number is incorporated 
in the document describing the subgrammar CNformation, doc:r482 (FdeJ).

\section{Phenomena not covered}
Not incorporated: \\
\begin{itemize}
  \item 
rules building NPs with a predeterminer. Example:
{\em al die boeken}.
Idea: predeterminers are modifiers to DETPs. Consequently such NPs should be 
derived by combining a complex DETP (e.g. {\em al die}, with $al$ as a modifier
to $die$) and a CN. 

The determiner {\em alle} is presently treated as a basic expression of 
category BDET. In
order to account for the relation between {\em alle} and {\em al de } it may be 
be needed to introduce it via rules.\\


  \item 
rules to derive NPs with a  measure NP 
modifying an adjectival form with a genitival -$s$,
such as {\em veel moois}, {\em wat lekkers}, {\em iets goeds}.
\end{itemize}





For various reasons, 
more or less understood,
the following Noun Phrase constructions are not covered.

\begin{itemize}
  \item Nouns with a non-lexicalized head:
\begin{description}
  \item de mooiste drie 
  \item de mijne
  \item die twee van Jan
  \item een van die twee 
\end{description}
  \item infinitives used as a noun
\begin{description}
  \item het koken van aardappelen
  \item het leveren van materiaal
\end{description}
  \item pronimal head + ADJ with -s
\begin{description}
  \item iets lekkers
  \item veel moois
\end{description}
  \item postnominal adverbial modifiers
\begin{description}
  \item het huis hierboven
  \item die jongen daar
\end{description}
  \item postnominal modifiers expressing comparison with {\em als}
\begin{description}
  \item een stad als Eindhoven
\end{description}
  \item idiomatic determiners
\begin{description}
  \item {\em een beetje} kaas
\end{description}
  \item complex determiners
\begin{description}
  \item alle drie de
  \item rond de twintig
\end{description}
  \item "een schat van een kind"
\begin{description}
  \item een schat van een kind
  \item kanjers van vissen
\end{description}

\end{itemize}





\section{Comments on certain NP-attributes}

Some attributes 
that have been chosen for special reasons have not been commented upon in the 
rule documentation:

\begin{description}
  \item[cases]\mbox{}
In contrast to English and Spanish, 
the Dutch version of RC\_NPformation generates NPs with the value [Nominative]
for the attribute .$cases$. As a consequence in analysis 
all Dutch NPs must be assigned the 
value [Nominative] before they can be decomposed. The case-assignment rules for 
Dutch subject NPs, that are assigned the value [Nominative] in S-Parser,  must therefore be asymmetric in order to 
avoid the risk of being circular. 
  \item[syntquant]\mbox{}

Presently no crucial reference is made to this attribute.
  \item[specQ]\mbox{}


The value for this attribute is provisonnally set to yesspec.
  \item[generic]\mbox{}

All present rules of RC\_NPformation generate NPs with the value $omegavalue$ 
for the attribute .$generic$. Uptill now, all paths through the grammar 
presuppose that sentences are non-generic. Because the Surface Parser is not 
supposed to be able to distinguish the various alternatives, and as the M-rules 
are presently not distinguishing betweeen generic 
and non-generic contexts,
in analysis the 
desubstitution of NP-arguments and other rules taking NPs as argument, require 
NPs with the value $omegageneric$. After application of these rules, the NP
is assigned the value $nogeneric$. In generation, the value for .$generic$ is 
$omegageneric$ after substitution. As a consequence it is not possible yet to 
translate sentences like {\em vogels hebben vleugels} or {\em bloemen houden 
van mensen} as statments involving generic arguments. Especially the 
translation into Spanish, a language that uses different NPs for generic and 
non-generic use,
may therefore be incorrect. 
\end{description}

\section{Subgrammar Specification}

The formation of NPs is dealt with by 
the subgrammar NPformation:\\
\begin{description}
  \item[Head] CN, NP, PERSPRO, INDEFPRO, WHPRO, PROPERNOUN, CN, CARDINAL 
  \item[Export] NP
  \item[Import] ADVP, ADJPPROP, DETP, NP, PREPPROP, SENTENCE
%
  \item[File] dutch:npsubgrammars.mrule (mrules67)
\end{description}

\section{Control Expression}
The control expression is defined as follows:

\begin{verbatim}
 ( RNPFORMATION1/15  | RNPFORMATION2/16  | RNPFORMATION3/17  
  | RNPFORMATION3a/173  | RNPFORMATION4a/14   | RNPFORMATION4/18  
  | RNPFORMATION5/19 | RNPFORMATION6/20   | RNPFORMATION7/21  
  | RNPFORMATION8/22  | RNPFORMATION9/23  | RNPFORMATION10/24 
  | RNPFORMATION11/25 | RNPFORMATION11a/58 | RNPFORMATION12/26 
  | RNPFORMATION13/27 | RNPpartitiveformation1/28 | RNPpartitiveformation2/29 
  | RNPFORMATION14/14 | RNPFORMATION17/105 | RcardNP/107) 

. [Rpndim/108]

. FpostunitNP/106
 
. (RNPPresentSuperdeixis1/100 | RNPPastSuperdeixis1/101 
  | RNPPresentSuperdeixis2/102 | RNPPastSuperdeixis2/103)

. [ RNPWHMODPOSS/30 | RNPPROPERNOUNMODPOSS/31  | RNPINDEFMODPOSS/32 |
   RnonCNMODRELSENT1/18 
  ] 

. [RNPapposition/104]

. {TNPnoTOomega1/46}
. FpostNPnoTOomega1/47

. {TNPnoTOomega2/48}
. FpostNPnoTOomega2/49

. FpreNPassignEform/38

. {TNPAssignEform1/33 | TNPAssignEform2/37 | 
  TNPAssignEform3/54 | TNPAssignEform4/55 }

. FpostNPassignEform/38

. {Tzijnddel/30}
. (Fzijnddel/31)

. FNPPreENdeletion/34 . {(TNPENdeletion1/35 |TNPENdeletion2/351)} 
. FNPPostENdeletion/36 
 
. {TNPCNdeletion1/42} . FpostNPCNdeletion1/44

. {TNPCNdeletion2/43} . FpostNPCNdeletion2/45

. [TNPhop/1]

. FpreNPadjqpcomplextrapos1/59 . [TNPadjqpcomplextrapos1/60]
. FpostNPadjqpcomplextrapos1/61

. FpreNPQPcomplextrapos/39 . [TNPQPcomplextrapos/40] . FpostNPQPcomplextrapos/41

. [RNPMODADVP/29]

. [RNPargmodsubst1/50 | RNPargmodsubst2/51 ]

. [TNPposTOpostmod/52] 
.  FNPpostopostmod/53 

. FpreNParticleIntro2/56 . {TNParticleIntro2/57}

\end{verbatim}

\section{Rules and transformations}


\begin{mruleclass}{NPformation}
\begin{classdescr}
\kind obligatory rule class
%\kind \nokind
\classtask The formation of the syntactic level NP
\classremarks The semantics of most rules is indicated vaguely by the 
the notion NPformation. This is to be understood as referring to a 
certain operation in terms of semantic types. 
\begin{filters}
\item 
\begin{member}
\rulename FpostunitNP
\ruletask To block NP's with a plural unitnoun head 
with singular morphology
when there is no plural 
count indefinite determiner or modifier preceding it.
Examples: *{\em een van de uur, *de uur, *uur, 
*een van Jans' jaar, *mijn uur, *vele uur}
\file dutch:npformation1.mrule (mrules46)
\remarks
Note that this filter is neither a speedrule, nor a filter that is to
check the correct application of an optional rule class. 
It filters out a specific subset of illformed NPs that are 
output of RC:NPformation. 
The reason to use the filter mechanism for this purpose is that 
it is the only way to express the restriction on the use of the singular 
paraphrase of plural unitnouns such as {\em uur} in a generalized way.
The alternative would be to complicate at least four rules of RC:NPformation
in order to exclude the formation of the illformed examples. 
It would require more elaborate models and the addition of 
rather complicated matchconditions. \\
\end{member}

\end{filters}

\nospeedrules
\noplannedrules
\norulesnotince
\begin{comments}
\end{comments}
\rulelist

\end{classdescr}

\begin{members}
\begin{member}
\rulename RNPformation1
\ruletask
Making an NP out of a CN containing a NOUN, and a DETP headed by:
DEMADJ,
NUM,
DET,
QP or
DETP (partitives) 
\file dutch:npformation1.mrule (mrules46)
\semantics NPformation
\example 
\begin{enumerate}
\item 
DETP headed by DEMADJ\\
$ die + (aardige) jongen(s)  \rightarrow
die (aardige) jongens\\
die + ( zoete) suiker \rightarrow  die (zoete) suiker\\
dat/dit + ( mooie) boek \rightarrow  dat/dit (mooie) boek\\
dat/dit$ + ({\em doffe}) $zilver \rightarrow  dat/dit$ ({\em doffe}) $zilver\\
deze + (oude) kaas/solist \rightarrow   deze (oude) kaas solist\\
dit + ( mooie) boek \rightarrow  dat (mooie) boek$\\
\item
DETP headed by NUM\\
{\em drie} + {\em (kleine) kleuters} $\rightarrow$ {\em drie (kleine) kleuters}\\
\item
DETP headed by DET\\
$ elke/iedere + (oude) man \rightarrow elke\ (oude)\ man$\\
 ('eForm' vs. 'noform' count/singular )\\
$ieder + (nieuw) boek \rightarrow ieder\  (nieuw)\ boek\\
alle/sommige + (nieuwe) boeken \rightarrow alle/sommige\ \  (nieuwe)\ \  boeken\\
veel + boeken/kaas  \rightarrow veel\ \  boeken/kaas\\
(always 'noform'; count/plural vs. mass/singular)\\
vele  + boeken \rightarrow veel\ \  boeken $\\
(always 'noform'; always count/plural )
\item
DETP headed by QP\footnote{The relation between QP {\em vele/veel} and
DET {\em vele} will be discussed in doc:r483 (FdeJ: QPformation).}\\
$ genoeg + boeken \rightarrow genoeg\  boeken\\
weinig + kaas \rightarrow weinig\ kaas$ \\
\item
DETP headed by DETP\\
$twee/sommige\  \ van\ \  die + (oude) boeken \rightarrow twee/sommige\ \ van\ \ 
die\ \  (oude)\ \  boeken$\\
\end{enumerate}
\remarks 


\begin{enumerate}
  \item 
The difference between QP (headed by Q) and DET (headed by BDET) is 
made in order to be able to account for the fact that all QP-determiners may 
also occur as non-determiners. E.g. {\em Jan had {\bf genoeg} geslapen}, {\em 
Jan komt {\bf veel} vaker dan Piet}.
  \item
Due to an incorrect value for partitive DETPs for the attribute 
.$definite$, the derivation of partitive NPs out of a partitive DETP and 
a CN containg a numerical modifier is blocked.  Example: {\em twee van die drie
boeken}. This can be remedied by changing the relevant rule of subgrammar 
DETPformation. However, in order to be able to exclude the possibility of 
recursion within the partitive DETP probably a new value for .$definite$ is 
required.
\end{enumerate}
\end{member}
\begin{member}
\rulename RNPformation2
\ruletask
The formation of an NP out of a CN with (a) an empty head (EN), and a DETP.
as bare NP.
\file dutch:npformation1.mrule (mrules46)
\semantics NPformation 
\example\mbox{}\\
\begin{enumerate}
  \item 
         EN                 $\rightarrow$ $drie$ EN, {\em enkele} EN, 
{\em de meeste} EN
  \item
         $gele$ EN            $\rightarrow$ {\em drie gele} EN, 
{\em alle gele} EN
  \item
         $honderd$ EN         $\rightarrow$ {\em alle honderd} EN
  \item
         EN {\em met een rietje}  $\rightarrow$ $drie$ EN {\em met een rietje}
\end{enumerate}
         excluded: $beide$ EN, $die$ EN $daar$; cf. RNPformation11\\
         excluded: $die$ EN {\em met een strikje}; cf. Rnpformation3a
\remarks\mbox{}
\begin{enumerate}
\item 
The occurrence of {\em alle, elke, iedere}, etc. as bare determiner is blocked 
by a condition making reference to the attribute $.definite$.
\item 
There is a problem with {\em beide}. Compare:
\begin{verbatim}
    *alle
    *alle met een strikje
versus
     beide
    *beide met een strikje
\end{verbatim}
If {\em beide} is considered definite (DETPrec.definite =def)  
{\em beide} will incorrectly be blocked as a bare DETP, while 
the ungrammatical {\em beide met een strikje} suggests that {\em beide} and 
{\em alle} should be treated on  a par.

A (not implemented yet) solution might be to assign {\em beide} two categories: 
INDEFPRO and BDET
Assuming that {\em beide} is def, NPformation1 will block the derivation of 
both {\em beide} and *{\em beide met een strikje}.
NPformation7  will derive {\em beide} as NP headed by an INDEFPRO.
\item 
The attribute NPhead indicates whether the properties of the context eventually 
determine whether an NP may occur without a nominal head. 
In case the value is $enoknp$ is may be substituted freely into 
sentential contexts. In case the value is $ennp$ substitution
requires the presence of  {\em er}.
\item 
The two subrules distinguish between two kinds of empty headed NPs:\\
SUBRULE  1: prenominal modifier present, NPrec.NPhead = enokNP: 
only without {\em er}; Examples: {\em twee gele, die mooie}\\
   SUBRULE  2a: no prenominal modiifer, 
count IN NPrec.posscomas and NPrec.NPhead = enNP: 
kan niet zonder {\em er}; Examples: {\em 
twee, vele, de meeste}\\
            2b:mass IN NPrec.posscomas and NPrec.NPhead = enokNP: 
     kan zonder {\em er}: {\em veel}).
\item 
    It should be investigated whether bare demonstratives (always 
    definite!!) should be dealt with by this rule (which would possibly
    require an extra attribute), or by another one.
\end{enumerate}

\end{member}
\begin{member}
\rulename RNPformation3
\ruletask
Syncategorematic introduction of {\em de} in case of an empty CN head.
\file dutch:npformation1.mrule (mrules46)
\semantics NPformation
\example 
\begin{enumerate} 
  \item
plural: {\em (drie) (dure) EN} $\rightarrow$ 
{\em {\bf de} (drie) (dure) EN}
\item 
 singular: {\em  gele EN} $\rightarrow$ 
{\em de gele EN})
\end{enumerate}
\remarks No remarks
\end{member}
\begin{member}
\rulename RNPformation3a
\ruletask
Syncategorematic introduction of DEMADJs 
in case of a CN with an empty head and 
a postnominal PREPPmodifier. 
The rule has the same meaning as RNPformation3 which 
introduces a definite article to a CN with EN head.
\file dutch:npformation1.mrule (mrules46)
\semantics NPformation
\example 
\begin{enumerate}
  \item 
plural: {\em (drie) (dure) EN van Jan/met een strikje } 
$\rightarrow$ 
{\em {\bf die} (drie) (oude) EN van Jan/met een strikje} 
\item
singular: {\em gele EN van mij/in de hoek} $\rightarrow$ 
 {\em {\bf dat} gele van 
mij/in de hoek})
\end{enumerate}
\remarks\mbox{}
\begin{enumerate}
\item 
Mapping onto Spanish:
both {\em los de mi} and {\em estos de mi} are correct, though there is a 
slight distinction in meaning between these forms.
\item A better solution: deriving the Dutch form with a DEMADJ rather than 
an article with a transformation. 
\item A better understanding of the translation relation for expressions of 
this kind requires an elaborated theory of discourse phenomena.
\end{enumerate}

\end{member}
\begin{member}
\rulename RNPformation4
\ruletask
Syncategorematic introduction of definite articles {\em de, het}
in case of non empty NP-head,
possibly  modified. 
\file dutch:npformation1.mrule (mrules46)
\semantics formation of a definite NP (LNPformationdef)
\example all nouns
\remarks No remarks
\end{member}
\begin{member}
\rulename RNPformation4a
\ruletask
Syncategorematic introduction of indefinite article {\em een} 
in case of non empty NP-head,
possibly  modified. 
\file dutch:npformation1.mrule (mrules46)
\semantics formation of an indefinite NP  (LNPformationindef)
\example all count nouns
\remarks 
\begin{enumerate}
  \item
Numerical {\em een} is supposed to be distinguished from article {\em een}
by the use of accents ({\em \'{e}\'{e}n}). It is introduced via NPformation1.
  \item
A separate transformation is still needed to derive {\em 'n} rather 
than {\em een}.
\end{enumerate}

\end{member}
\begin{member}
\rulename RNPformation5
\ruletask Replacing a postnominal possessor-expression in $posrel$ (modification of a CN 
containing a NOUN), by a prenominal possessor-expression in $detrel$. 
\file dutch:npformation1.mrule (mrules46)
\semantics formation of a definite NP (LNPformationdef)
\example 
CN[head/$boek$ posrel/$ik$] $\rightarrow$ 
NP[detrel/{\em mijn}, CN[head/{\em boek}] etc.
\remarks\mbox{}
The NP {\em iemands boek} should be def(inite), considering:
*{\em Er ligt iemands boek op tafel}.\\
The NP {\em wiens boek} should be indef(inite), considering:
*{\em Wiens boek ligt er op tafel}.\\
For Rosetta the existential sentence is taken as a diagnostic context 
in determining the value for the attribute $definite$.
\end{member}
\begin{member}
\rulename RNPformation6
\ruletask creating an NP-node out of a WHPRO
\file dutch:npformation3.mrule (mrules110)
\semantics NPformation
\example {\em wie, wat}
\remarks No remarks
\end{member}
\begin{member}
\rulename RNPformation7
\ruletask creating an NP-node out of an INDEFPRO
\file dutch:npformation3.mrule (mrules110)
\semantics NPformation
\example $iets$, $iemand$, $niets$, $niemand$
\remarks
\begin{enumerate}
\item  Indefpro's with incorporated negation should only be translated locally.
This rule is not applicable to {\em niemand} or {\em niets}. Substitution of 
negated INDEFPROs is prohibited.
\end{enumerate}

\end{member}
\begin{member}
\rulename RNPformation8
\ruletask creating an NP-node out of a PERSPRO
\file dutch:npformation3.mrule (mrules110)
\semantics NPformation
\example  all perspro's
\remarks\mbox{}
The attribute $sexes$ keeps the default value (feminine), since only 
$gender$
is used for PERSPRO's.

\end{member}
\begin{member}
\rulename RNPformation9
\ruletask creating an NP out of a PROPERNOUN. 

\file dutch:npformation3.mrule (mrules110)
\semantics NPformation
\example $Jan$ 
\remarks 
PROPERNOUNS may also become NP via the alternative derivation involving the 
creation of a SUBNOUN-level dominating PROPERNOUN. 
This derivation is needed in order to deal with 'nominalized' PROPERNOUNs as in 
{\em de beide Duitslanden} en {\em welke Jan}.
Inherent plural PROPERNOUNS with a determiner such as {\em de Antillen} can not 
be dealt with yet. An additional attribute seems to be required here.
\end{member}

\begin{member}
\rulename RNPformation10
\ruletask making a bare NP out of a  CN
\file dutch:npformation1.mrule (mrules46)
\semantics NPformation
\example 
\mbox{}\\
\begin{enumerate}
\item
CN[{\em (mooie) boeken}] 
$\rightarrow$ 
NP[head/CN[{\em (mooie) boeken}]]
\item
 CN[{\em (lekkere) kaas}]
$\rightarrow$ 
NP[head/CN[{\em (lekkere) kaas}]]
\end{enumerate}
\remarks No remarks
\end{member}

\begin{member}
\rulename RNPformation11
\ruletask
The formation of an NP out of a CN with (a) an empty head (EN) preceded by a
modifying ADJP or prenominal participle, and article {\em een}.
\file dutch:npformation3.mrule (mrules110)
\semantics formation of an indefinite NP  (LNPformationindef)
\example
CN[$gele$ EN]
$\rightarrow$ 
{\em een gele EN}; 
CN[$werkende$ EN]
$\rightarrow$ 
{\em een werkende EN}
\remarks No remarks
\end{member}
\begin{member}
\rulename RNPformation11a
\ruletask
The formation of an NP out of an indefinite article and a
  CN with an empty head (EN), followed by a 
modifying PREPP.
NB. NPhead= ennp; kan niet zonder {\em er}.

\file dutch:npformation3.mrule (mrules110)
\semantics NPformation
\example
         CN[EN met een rietje]
$\rightarrow$  
{\em een EN met een rietje}

\remarks This rule should be deleted from the control expression. 
The string
{\em een EN met een rietje} is only wellformed with {\em een} interpreted as 
numeral rather than article.
\end{member}


\begin{member}
\rulename RNPformation12
\ruletask Making a yeshuman NP out of a DETP + EN 
\file dutch:npformation1.mrule (mrules46)
\semantics NPformation
\example {\em velen, allen, sommigen, enkelen}\\
\remarks\mbox{}
This rule can probably be incorporated into RNPformation2.\\

\end{member}

\begin{member}
\rulename RNPformation13
\ruletask Making an NP-node out of a DEMPRO, singular ($dat$, $dit$).
\file dutch:npformation3.mrule (mrules110)
\semantics NPformation
\example {\em dat}, {\em dit}
\remarks 
$die$ can also occur as a DEMPRO, but then it is introduced 
syncategorematically, e.g. in: 
{\em  Die jongen, {\em die} heb ik gezien}.
\end{member}
\begin{member}
\rulename NPformation14
\ruletask creating an NP out of a  RECIPRO
\file dutch:npformation3.mrule (mrules110)
\semantics NPformation
\example {\em elkaar}
\remarks No remarks
\end{member}

\begin{member}
\rulename Rnpformation17
\ruletask
\ruletask creating an NP out of a BIGPRO
\file dutch:npformation3.mrule (mrules110)
\semantics NPformation
\example all BIGPRO's
\remarks No remarks
\end{member}

\begin{member}
\rulename RNPpartitiveformation1
\ruletask Formation of an partitive NP, out of a CN and a DETP, plus
syncategorematic introduction of $van$ + $de$. 
\file dutch:npformation3.mrule (mrules110)
\semantics NPformation
\example {\em drie} + CN[($nieuwe$) $boeken$] $\rightarrow$ {\em 
drie van de (nieuwe) 
boeken}
\remarks No remarks

\end{member}
\begin{member}
\rulename RNPpartitiveformation2
\ruletask Formation of an partitive NP, out of a DETP and a CN with 
possessor expression in 
$posrel$,
plus
syncategorematic introduction of {\em van}.
\file dutch:npformation3.mrule (mrules110)
\semantics NPformation
\example 
\begin{enumerate}
  \item 
$twee$ + CN[head/{\em broers} posrel/{\em ik}]
$\rightarrow$
{\em twee van mijn broers}
  \item
$veel$ + CN[head/{\em werk} posrel/{\em Jan}]
$\rightarrow$
{\em veel van Jans werk}
\end{enumerate}

\remarks Cf. also doc:r413 (FdeJ)
\end{member}
\begin{member}
\rulename RcardNp
\ruletask Formation of NP out of arabic digital expression
\file dutch:npformation3.mrule (mrules110)
\semantics NPformation
\example 3, 100, 1991
\remarks No remarks
\end{member}
\end{members}
\end{mruleclass}


\begin{mruleclass}{RC\_pndim}
\begin{classdescr}
\kind optional rule class
\classtask Accounting for diminutive PROPERNOUNs
\classremarks
\nofilters
\nospeedrules
\noplannedrules
\norulesnotince
\rulelist
\end{classdescr}
\begin{members}
\begin{member}
\rulename Rpndim
\ruletask Making a diminutive PROPERNOUN
\file dutch:verbderivation.mrule  (mrules60)
\semantics unclear
\example $Jan$ $\rightarrow$
$Jantje $
\remarks No remarks
\end{member}
\end{members}
\end{mruleclass}


\begin{mruleclass}{RC\_NPsuperdeixis}
\begin{classdescr}
\kind obligatory rule class
%\kind \nokind
\classtask Accounting for deixis
\classremarks For further comment, cf the general documentation on superdeixis.
\nofilters
\nospeedrules
\noplannedrules
\norulesnotince
\begin{comments}
\end{comments}
\rulelist
\end{classdescr}

\begin{members}
\begin{member}
\rulename RNPPresentSuperdeixis1
\ruletask Account for the value of NPrec.superdeixis if there is a nominal CN.
\file dutch:npformation2.mrule (mrules47)
\semantics The `indirect' relation between Rs and S is simultanuous: PRESENT.

\example
\remarks No remarks
\end{member}
\begin{member}
\rulename RNPPastSuperdeixis1
\ruletask Account for the value of NPrec.superdeixis if there is a nominal CN.
\file dutch:npformation2.mrule (mrules47)
\semantics The `indirect' relation between Rs and S is before: PAST.

\example
\remarks No remarks
\end{member}
\begin{member}
\rulename RNPPresentSuperdeixis2
\ruletask Account for the value of .superdeixis for non-CN NPs.
\file dutch:npformation2.mrule (mrules47)
\semantics The `indirect' relation between Rs and S is simultanuous: PRESENT.
\example
\remarks No remarks
\end{member}
\begin{member}
\rulename RNPPastSuperdeixis2
\ruletask Account for the value of .superdeixis for non-CN NPs.
\file dutch:npformation2.mrule (mrules47)
\semantics The `indirect' relation between Rs and S is before: PAST.
\example
\remarks No remarks
\end{member}
\end{members}
\end{mruleclass}

\begin{mruleclass}{RC\_nonCNmodification}
\begin{classdescr}
\kind optional rule class
%\kind \nokind
\classtask Accounting for modification of non-CN NPheads
\classremarks
\nofilters
\nospeedrules
\noplannedrules
\norulesnotince
\begin{comments}
\end{comments}
\rulelist
\end{classdescr}
\begin{members}
\begin{member}
\rulename RINDEFmodposs
\ruletask 
Introduction of postnominal possessive $van$-modifiers to 
NPs
with INDEFPRO as the head.
\file dutch:npformation2.mrule (mrules47)
\semantics possessive modification 
\example NP[$ieder$] + $wij$ (=possessor NP) $\rightarrow$ {\em ieder van ons}
\remarks Cf. also doc:r413 (FdeJ)
\end{member}
\begin{member}
\rulename RWHmodposs
\ruletask 
Introduction of postnominal possessive $van$-modifiers to 
NPs
with WHPRO as the head.
\file dutch:npformation2.mrule (mrules47)
\semantics possessive modification 
\example NP[$wie$] + $wij$ (=possessor NP) $\rightarrow$ {\em wie van ons}
\remarks Cf. also doc:r413 (FdeJ)
\remarks No remarks
\end{member}
\begin{member}
\rulename PROPERNOUNmodposs
\ruletask 
Introduction of postnominal possessive $van$-modifiers to 
NPs
with PROPERNOUN as the head.
\file dutch:npformation2.mrule (mrules47)
\semantics possessive modification 
\example NP[$Jan$] + $wij$ (=possessor NP) $\rightarrow$ {\em Jan van MArie}
\remarks Cf. also doc:r413 (FdeJ)
\end{member}
\begin{member}
\rulename RnonCNmodRELSENT1
\ruletask Modification of nonCNs
by a restrictive, finite, relative SENTENCE.

\file dutch:cnmodification.mrule (mrules109)
\semantics modification
\example 
\mbox{}\\
\begin{enumerate}
  \item

original noncnmodrelsent1:\\
{\em Jan}  + [x2 {\em is nooit ziek }] $\rightarrow$ [{\em Jan}, [{\em 
die nooit ziek is}],]\\
{\em iemand} + [x2 {\em is ziek}] $\rightarrow$ [{\em iemand} [{\em die ziek is
}],]\\
{\em iets} + [x2 {\em is lekker}] $\rightarrow$ [{\em iets} [{\em 
wat lekker is}],]

  \item 
original noncnmodrelsent2:\\
{\em iets waarvan ik wekelijks droom}

  \item 
original noncnmodrelsent3:\\
{\em iets waar ik van droom}

  \item 
original noncnmodrelsent5:\\
{\em iemand van wie ik droom}
\end{enumerate}

\remarks
\begin{enumerate}
  \item 
Present rule replaces the original noncnmodrelsentrules (1-3, 5). Cf. examples.
  \item 
In analysis the absence 
of a comma behind the  relative sentence is accepted, but in generation
a comma is always generated. 
\item The apparently restrictive modification in examples such as {\em iemand 
die ziek is} suggests that for INDEFPROs and 
perhaps also WHPROs a NP-derivation 
should be available involving a CN-level. 
\end{enumerate}
\end{member}

\end{members}
\end{mruleclass}

\begin{mruleclass}{RC\_NPapposition}
\begin{classdescr}
\kind optional rule class
%\kind \nokind
\classtask Accounting for appositions (not: appositives)
\classremarks
\nofilters
\nospeedrules
\noplannedrules
\norulesnotince
\begin{comments}
\end{comments}
\rulelist
\end{classdescr}
\begin{members}
\begin{member}
\rulename RNPapposition
\ruletask To insert an appositional NP into the NP-structure.
\file dutch:npformation2.mrule (mrules47)
\semantics non-restrictive modification
\example 

\begin{enumerate}
\item
NP[{\em Jan}] + NP[{\em mijn buurman}] $\rightarrow$ Jan, mijn buurman
\item
NP[{\em een buurman van mij}] + NP[{\em Jan}] $\rightarrow$ 
 een buurman van mij, Jan
\item
NP[{\em maandag}] + NP[{\em 15 februari 1991}] $\rightarrow$ 
maandag, 15 februari 1991
\end{enumerate}

\remarks No remarks
\end{member}
\end{members}
\end{mruleclass}

\begin{mruleclass}{TC\_NPnoTOomega + TC\_NPassignEform} 
\begin{classdescr}
\kind obligatory transformation classes
%check
%\kind \nokind
\classtask Accounting for the inflection on prenominal modifiers
\classremarks
TC\_NPnoTOomega introduces an auxiliary "defaultvalue" {\em omegaform}, 
which is needed 
to allow the iterative application of TC:NPassigeform. 
It would have been more elegant to assign the value {\em omegaform}
to ADJs, ADJPs and VERBs in the rules that introduce these categories 
(e.g. startrules). However this would require a large amount 
of adaptation in the 
rest of the grammar, in order to guarantee that the default value is reset 
before G-MORPH applies. 

\begin{filters}
\begin{members}
\begin{member}
\rulename FpostNPnoTOomega1
\ruletask 
To guarantee the correct application of TnoTOomega1.
\file dutch:npformation2.mrule (mrules47)
\end{member}
\begin{member}
\rulename FpostNPnoTOomega2
\ruletask
To guarantee the correct application of TnoTOomega2.
\file dutch:npformation2.mrule (mrules47)
\end{member}

\begin{member}
\rulename FpostNPAssignEform
\ruletask To guarantee the correct application of TC:NPAssignEform in 
generation.
\file dutch:npcnvaria.mrule (mrules130)
\end{member}
\end{members}
\end{filters}
\begin{speedrules}
\begin{members}
\begin{member}
\rulename FpreNPassignEform 
\ruletask  Speed filter. Blocks the analysis of NPs containing an
adjective that is not assigned the default value {\em omegaform} for
the attribute {\bf eORenForm}.

\file dutch:npcnvaria.mrule (mrules130)
\end{member}
\end{members}
\end{speedrules}
\noplannedrules
\norulesnotince
\begin{comments}
\end{comments}
\rulelist
\end{classdescr}
\begin{members}
\begin{member}
\rulename TNPnoTOomega1
\ruletask ADJ en ADJP: 
to change the default value for REC.eorenform from {\em Noform}
into {\em omegaform}.
\file dutch:npformation2.mrule (mrules47)
\semantics \nosemantics
\example all adjectives that occur as the head of a 
prenominal modifier
\remarks No remarks

\end{member}
\begin{member}
\rulename TNPnoTOomega2
\ruletask
to change the default value for REC.eorenform from {\em Noform}
into {\em omegaform}.
\file dutch:npformation2.mrule (mrules47)
\semantics \nosemantics
\example all participle VERBS 
that occur as  a prenominal modifier
\remarks No remarks
\end{member}

\begin{member}
\rulename TNPAssignEform1
\ruletask Account for the inflection on adjectives.
\file dutch:npformation2.mrule (mrules47)
\semantics \nosemantics
\example\mbox{}\\
\begin{enumerate} 
  \item
(subrule 1) determinerless NP: {\em lekkere appels, lekker brood, lekkere kaas}
  \item
(subrule 2) indefinite NP with determiner: {\em veel lekker brood, een 
lekkere hap, veel lekkere 
hapjes}
  \item
(subrule 3) (a)def NP with DET as head of DETP: {\em elk lekker brood, iedere 
lekkere hap, alle lekkere 
hapjes}
  \item
(subrule 4) (a)def NP, head of DETP is not DET: {\em 
het lekkere brood, mijn lekkere pap,
vaders lekkere hapjes}
  \item
(subrule 5) (in)def NP with POSSADJ as head of DETP: {\em wiens lekkere brood, 
wiens lekkere pap, wiens 
lekkere hapjes,
ons/zijn lekkere brood, onze/haar lekkere pap, onze/uw hapjes}
  \item
(subrule 6) def NP with possessive NP as determiner: {\em 
Jans/mijn vaders lekkere brood, Jans/mijn 
vaders lekkere pap,
Jans/mijn vaders lekkere hapjes}
\end{enumerate}

\remarks No remarks
\end{member}
\begin{member}
\rulename TNPAssignEform2
\ruletask Accounting for inflection 
on adjectives in case there is no nominal 
head.
\file dutch:npformation2.mrule (mrules47)
\semantics \nosemantics
\example alle lekkere, die mooie, de kale, een houten. 
\remarks  No remarks
\end{member}
\begin{member}
\rulename TNPAssignEform3
\ruletask Account for the inflection of participles
\file dutch:npformation2.mrule (mrules47)
\semantics \nosemantics
\example \mbox{}\\
\begin{enumerate} 
  \item
(subrule 1) determinerless NP: 
{\em stinkende/gekochte appels; rijzend/verrijkt brood, 
rijpende/afgedankte kaas}
  \item
(subrule 2) indefinite NP with determiner: 
{\em veel schimmelend/verpakt brood, een stinkende/
afgeprijsde kaas; wat rottende/gepikte eieren}
  \item
(subrule 3) (a)def NP with DET as head of DETP: 
{\em elk schimmelend/beschimmeld brood, iedere 
schimmelende/beschimmelde hap; alle vetmakende/gekochte hapjes}
  \item
(subrule 4) (a)def NP, head of DETP is not DET: 
{\em het smakende/gekookte brood, 
mijn kokende/afgekoelde pap; vaders uitnodigende/uitgestalde hapjes}
  \item
(subrule 5) (in)def NP with {\em wiens} as head of DETP: 
{\em wiens schimmelende/beschimmelde brood, wiens 
overkokende/overgekookte pap; wiens oprakende/opgewarmde hapjes}
\end{enumerate}





\remarks 
\begin{enumerate}
(This text is partly written by Margreet Sanders)\\
As there is no attribute .eformation for VERBs, the set of participles 
that can have inflection is to be reconstructed by means of the attribute
`conjclasses'. The verbs in conjclasses [5,6,7,8,11,12,15,16] have a past 
participle that always ends in {\em -en\/}, and thus never take an extra -e
ending. Classes [3,4,9,10] are regular weak verbs or have a regular weak past 
part, and may take en e-ending on their past part. Class 13 does not have a 
past part (placht etc.), class 14 is the extra
class for 0th and 1st person d-deletion (hou), and class 0 is for composite 
defective verbs (zweefvliegen). These three classes have not been excluded from
the present rules for past parts, and should be stopped elsewhere. 

The problems are caused by conjclasses 1 and 2, for irregular verbs. There is 
no way to see whether these verbs end in {\em -en\/} or in something else. The
verbs of class 1 as mentioned in doc.\ 134 (Rosetta3 Dutch Morphology: 
Inflection. Comments; written by Harm Smit) mostly end in {\em -en\/}, except the 
following: {\em geweest, gehad, gekund, gestaan; gewild, gezegd, gedaan, gezien
\/}. The former of this group will probably not be used attributively anyway, 
and can be disregarded. Therefore, it was decided to treat the verbs of 
conjclass 1 as members of the {\em no -e ending\/} set. This implies that the 
latter 4 verbs will be generated without any {\em -e\/} ending. To prevent the 
situation where one is forced to enter a wrong input, 
verbs of conjclass 1 and 2 are always accepted in analysis, both with 
and without e-ending.
Verbs of class 2 are also always generated without final -e, since it is 
unclear how many and which verbs are in this class, and since the examples 
provided in doc.\ 134 often end in -en. If a full list of class 2 verbs is 
available, this decision may be reconsidered.

To solve the problem of eFormation of past parts, morphology must be changed to 
separate classes 1 and 2 in -en past parts and other past parts.
\end{enumerate}

\end{member}
\begin{member}
\rulename TNPAssignEform4
\ruletask Spell out flection -e on particple VERBs 
in case there is no nominal 
head.
\file dutch:npformation2.mrule (mrules47)
\semantics \nosemantics
\example {\em alle ooit verkochte, de in de kast gelegde}
\remarks 
Cf. TNPassigneform3

\end{member}
\end{members}
\end{mruleclass}


\begin{mruleclass}{TC\_zijnddeletion}
\begin{classdescr}
\kind optional transformation class
%\kind \nokind
\classtask Deleting the present participle {\em zijnd}.
\classremarks
\begin{filters}
\begin{members}
\begin{member}
\rulename Fzijnddel
\ruletask To guarantee the correct application of Tzijnddel in 
generation.
\file dutch:npformation3.mrule (mrules110)
\end{member}
\end{members}
\end{filters}
\nospeedrules
\noplannedrules
\norulesnotince
\begin{comments}
\end{comments}
\rulelist
\end{classdescr}

\begin{members}
\begin{member}
\rulename Tzijnddel
\ruletask
\file dutch:npformation3.mrule (mrules110)
\semantics \nosemantics
\example
\remarks No remarks
\end{member}
\end{members}

\end{mruleclass}

\begin{mruleclass}{TC\_NPENdeletion + TC\_NPCNdeletion}
\begin{classdescr}
\kind optional transformation  classes
%\kind \nokind
\classtask Accounting for the deletion of abstract nominal heads.
\classremarks
\begin{filters}
\begin{members}

\begin{member}
\rulename FNPPostENdeletion
\ruletask To guarantee the correct application of 
TC\_NPENdeletion
\file dutch:npformation2.mrule (mrules47)
\end{member}
\begin{member}
\rulename FpostNPCNdeletion1
\ruletask To guarantee the correct application of 
TNPCNdeletion1.
\file dutch:npformation2.mrule (mrules47)
\end{member}

\begin{member}
\rulename FpostNPCNdeletion2
\ruletask To guarantee the correct application of 
TNPCNdeletion2.
\file dutch:npformation2.mrule (mrules47)
\end{member}
\end{members}

\end{filters}
\begin{speedrules}

\begin{members}
\begin{member}
\rulename FNPPreENdeletion
\ruletask To speed up analysis
\file dutch:npformation2.mrule (mrules47)
\end{member}
\end{members}

\end{speedrules}
\noplannedrules
\norulesnotince
\begin{comments}
\end{comments}
\rulelist
\end{classdescr}



\begin{members}
\begin{member}
\rulename TNPENdeletion
\ruletask Deletion of count EN.
\file dutch:npformation2.mrule (mrules47)
\semantics \nosemantics
\example de gele (EN), deze twee (EN), 
die (EN) van de markt,  de vijf (EN) die 
gisteren afwezig waren, die (EN) van mij.
\remarks No remarks
\end{member}
\begin{member}
\rulename TNPENdeletion2
\ruletask Deletion of mass EN.
\file dutch:npformation2.mrule (mrules47)
\semantics \nosemantics
\example het schone (EN), de jonge (EN), 
                            die (EN) van de markt, die (EN) van Jan
\remarks No remarks
\end{member}


\begin{member}
\rulename TNPCNdeletion1
\ruletask Deletion of count empty CN.
\file dutch:npformation2.mrule (mrules47)
\semantics \nosemantics
\example beide, sommige(n), NB. niet: alle (cf. problems)
\remarks
\begin{enumerate}
\item By requiring a DET head of the DETP in detrel application of this rule
to numerals is excluded. So this subgrammar does not generate {\em twee} as an NP.
The occurrence of such NPs requires the occurrence of {\em er}
in the same sentence. This is checked 
in the sentential 
subgrammars. 
Note that {\em vele/veel/weinig} are BDET as well as NUM.
So for these forms 
a derivation via subgrammar NPformation
is available. Cf. also doc:r483 (FdeJ).
\item problem1:\\ The rule should not apply to {\em alle}. This is not 
guaranteed yet. 
It should however apply to {\em allen}.
\item problem2:\\ The rule should also apply to {\em vele} and 
{\em velen} in the "hidden
partitive" interpretation. This partitive "vele" is not indefinite. It does not
require the occurrence of partitive {\em er}, nor does it allow the occurrence
of existential {\em er}. It is not related to {\em veel}. Its distribution
seems restricted: {\em vele} as an NP with an empty head occurs only in
topicalized position. Examples: {\em vele hebben een gat in de bodem} and {\em
vele zag hij leeg op een plank staan} (both with respect to a set of boxes
already mentioned inthe context or introduced otherwise). 

One way to deal with
this not-indefinite non-numeral is to consider it a BDET with the value adef
for .definite. Its distributional peculuarities might require a special 
value for NPhead (e.g. veleNP), to which the substitution rule can refer. 
This is not 
implemented yet as it would give rise to many ambiguities as long as 
the restrictions are not properly stated. 
\end{enumerate}

\end{member}
\begin{member}
\rulename TNPCNdeletion2
\ruletask Deletion of empty mass CN.
\file dutch:npformation2.mrule (mrules47)
\semantics \nosemantics
\example {\em veel, weinig, bijna genoeg}
\remarks For the status of {\em veel}, cf. also doc:r483 (FdeJ)
\end{member}

\end{members}
\end{mruleclass}


\begin{mruleclass}{TC\_NPhop}
\begin{classdescr}
\kind optional transformation  class
%\kind \nokind
\classtask Accounting for hopping DETPs.
\classremarks
\nofilters
\nospeedrules
\noplannedrules
\norulesnotince
\begin{comments}
\end{comments}
\rulelist
\end{classdescr}

\begin{members}
\begin{member}
\rulename TNPhop
\ruletask To account for the occurrence of postnominal determiners (QPs).
\file dutch:npformation1.mrule (mrules46)
\semantics \nosemantics
\example detrel/$genoeg$ head/$tijd \rightarrow$ head/$tijd$ hoprel/$genoeg$
\remarks No remarks
\end{member}
\end{members}
\end{mruleclass}

\begin{mruleclass}{TC\_NPadjqpcomplextraposition}
\begin{classdescr}
\kind optional transformation class
\classtask Extraposition 
of complements out of ADJP 
\classremarks
\begin{filters}

\begin{members}
\begin{member}
\rulename FpostNPadjqpcomplextrapos
\ruletask To guarantee the correct application of TNPadjqpcomplextrapos.
\file dutch:npformation2.mrule (mrules47)
\end{member}
\end{members}
\end{filters}
\begin{speedrules}
\begin{members}
\begin{member}
\rulename FpreNPadjqpcomplextrapos
\ruletask To speed up analysis.
\file dutch:npformation2.mrule (mrules47)
\end{member}
\end{members}
\end{speedrules}

\noplannedrules

\norulesnotince

\rulelist

\end{classdescr}

\begin{members}
\begin{member}
\rulename TNPadjqpcomplextrapos1
\ruletask Extraposition in the NP of the complements of QPs in 
ADJP.
\file dutch:npformation2.mrule (mrules47)
\semantics \nosemantics
\example
\begin{enumerate}
  \item 
$een$ [{\em mooier dan dit}] $boek$
$\rightarrow$
{\em  een mooier boek dan dit}
  \item
$het$ [{\em mooiste van alles}] EN 
$\rightarrow$
{\em het mooiste EN van alles}
  \item
 $een$ [{\em even mooi als dit}] $boek$ 
$\rightarrow$ 
{\em een even mooi boek als dit}
\end{enumerate}
\remarks No remarks
\end{member}
\end{members}
\end{mruleclass}

\begin{mruleclass}{TC\_NPQPcomplextraposition}
\begin{classdescr}
\kind optional transformation class
\classtask Extraposition of complements of QP in $detrel$
\classremarks
\begin{filters}

\begin{members}
\begin{member}
\rulename FpostNPQPcomplextrapos
\ruletask to guarantee the correct application of TNPQPcomplextrapos
\file dutch:npformation2.mrule (mrules47)
\end{member}
\end{members}
\end{filters}
\begin{speedrules}
\begin{members}
\begin{member}
\rulename FpreNPQPcomplextrapos
\ruletask to speed up analysis
\file dutch:npformation2.mrule (mrules47)
\end{member}
\end{members}
\end{speedrules}

\noplannedrules

\norulesnotince

\rulelist

\end{classdescr}

\begin{members}

\begin{member}
\rulename TNPqpcomplextrapos
\ruletask Extraposition in the NP of the complements of QPs in 
DETP.
\file dutch:npformation2.mrule (mrules47)
\semantics \nosemantics
\example\mbox{}
\begin{enumerate}
  \item 
[{\em meer dan melk}] {\em kaas}
$\rightarrow$
{\em meer kaas dan melk}
  \item
[{\em meer dan hij}] {\em boeken}
$\rightarrow$
{\em meer boeken dan hij}
\end{enumerate}
\remarks No remarks
\end{member}
\end{members}
\end{mruleclass}
\begin{mruleclass}{RC\_NPmodADV}
\begin{classdescr}
\kind optional rule class
%\kind \nokind
\classtask Accounting for the adverbial modification of the NP as a whole
\classremarks
\nofilters
\nospeedrules
\noplannedrules
\norulesnotince
\begin{comments}
\end{comments}
\rulelist
\end{classdescr}

\begin{members}
\begin{member}
\rulename NPmodADVP
\ruletask Modifying NPs by ADVs with preXPadv IN subcs.
\file dutch:npformation2.mrule (mrules47)
\semantics modification
\example ook het kind, zelfs dit boek, ook deze, zelfs hij,
ook Piet. 
\remarks Formation of e.g. *{\em zelfs iemand} and  {\em *ook wie} is blocked
\end{member}
\end{members}
\end{mruleclass}

\begin{mruleclass}{RC\_NPargmodsubst}
\begin{classdescr}
\kind optional rule class
%\kind \nokind
\classtask Substitution of NP complements
\classremarks
\nofilters
\nospeedrules
\noplannedrules
\norulesnotince
\begin{comments}
\end{comments}
\rulelist
\end{classdescr}

\begin{members}

\begin{member}
\rulename RNPargmodsubst1
\ruletask Substitution of NPVAR in postnominal PP with $prepobjrel$.
\file dutch:npcnvaria.mrule (mrules130)
\semantics restrictive modification
\example {\em de vraag} + {\em het antwoord op x1} $\rightarrow$ {\em 
het antwoord op de vraag}
\remarks No remarks
\end{member}
\begin{member}
\rulename RNPargmodsubst2
\ruletask Substitution of VARs for sentential complements to NPs
\file dutch:npcnvaria.mrule (mrules130)
\semantics specifying the 'content' of the noun-denotation
\example {\em of het regent} + 
{\em de vraag x1} $\rightarrow$ {\em de vraag of het regent}
\remarks The construction delat with here is sometimes referred to 
as 'appositive'.
\end{member}
\end{members}
\end{mruleclass}


\begin{mruleclass}{TC\_NPposTOpost}
\begin{classdescr}
\kind optional transformation class
%\kind \nokind
\classtask Replacing $posrel$ by $postmodrel$ in casde of a {\em van}-PREPP.
\classremarks

\begin{filters}
\begin{members}
\begin{member}
\rulename FNPpostopostmod
\ruletask To guarantee the correct application of TNPposTOpostmod.
\file dutch:npcnvaria.mrule (mrules130)
\end{member}
\end{members}

\end{filters}
\nospeedrules
\noplannedrules
\norulesnotince
\begin{comments}
\end{comments}
\rulelist
\end{classdescr}


\begin{members}
\begin{member}
\rulename TNPposTOpostmod
\ruletask To replace posrel by postmodrel in case of a PREPP modifier with 
prepkey = vanprepkey.
\file dutch:npcnvaria.mrule (mrules130)
\semantics \nosemantics
\example 
{\em de jas} posrel/{\em van Jan} 
$\rightarrow$ 
{\em de jas} postmodrel/{\em van Jan} 
\remarks This rule is incorporated for efficiency of the surface parser.
\end{member}
\end{members}
\end{mruleclass}


\begin{mruleclass}{TC\_NParticleintroduction}
\begin{classdescr}
\kind optional transformation class
%\kind \nokind
\classtask Introducing an article in  analysis
\classremarks
\nofilters

\begin{speedrules}
\begin{members}
\begin{member}
\rulename FNPpostopostmod
\ruletask To speed up analysis
\file dutch:npformation4.mrule (mrules120)
\end{member}
\end{members}
\end{speedrules}

\noplannedrules
\norulesnotince
\begin{comments}
\end{comments}
\rulelist
\end{classdescr}


\begin{members}
\begin{member}
\rulename TNParticleintro2
\ruletask 
To introduce in analysis 
the definite article in determinerless singular count definite NPs 
containing a specifying proper name. 
\file dutch:npformation4.mrule (mrules120)
\semantics \nosemantics
\example
\begin{enumerate}
  \item 
analytically: {\em hoofdredacteur Jansen}
$\rightarrow$ 
{\em de hoofdredacteur Jansen}

  \item 
analytically: 
{\em  station Eindhoven}
$\rightarrow$ 
{\em  het station Eindhoven}

\end{enumerate}
\remarks As it is not clear yet in which contexts
the definite
article can be omitted, it is not deleted in generation.
\end{member}
\end{members}
\end{mruleclass}

\end{document}




