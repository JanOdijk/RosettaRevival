
   \documentstyle{Rosetta}
   \begin{document}
      \RosTopic{formalism}
      \RosTitle{Surface Rule Compiler}
      \RosAuthor{Ren\'{e} Leermakers}
      \RosDocNr{0164}
      \RosDate{4-12-86}
      \RosStatus{concept}
      \RosSupersedes{-}
      \RosDistribution{Project}
      \RosClearance{Project}
      \RosKeywords{Surface grammar, notation, compiler}
      \MakeRosTitle
\section{Introduction}
In this section I describe the surface rule compiler. It was devised in order
to enable the rule-writers to test the rules quickly. In a later stage, Carel
will implement a more ambitious compiler. This will involve some minor changes
of notation, as I had to deviate a little from the notation Carel proposed in
his 'engineeral' thesis.

The present compiler truly compiles only
the rule regular expressions, and transduces the remaining part of the rule in
a simple way.

For rule-writers, the main part of this document is the example of section 2.
Section 3 gives the basis grammar for the regular expressions of surface rules,
section 4 lists the macros used in the condition/action parts of the rules and
section 5 displays the output as it is generated from the rule of section 2.
Lastly the last section gives the attribute grammar actually used in the
compiler.
\section{Syntax by Example}
Because of the transducing part of the compiler, no complete syntax of the
rule notation can be given. However, the next section contains a description
that is as formal as possible. Here we give a sample input file, with a
Rosetta2 surface rule that should contain all relevant features. The file is
to be called SURFRULES.SUR.
\begin{verbatim}

%ENGLISH
&

%PPrule
RegularExpression:
  PP=[PREP/1].NP/2.{preporpart}
  preporpart = PREP/4|PART/3
ConditionsAndActions:
  VAR moodvar::moodtype;
      prepfound::BOOLEAN;
      exppostpkey::baskeytype;
      prepkeyvar::baskeytype;

BEGIN
<*
 HINIT:BEGIN
         moodvar:=omegamood;prepfound:=false;exppostpkey:=0;prepkeyvar:=0
       END;
 1    :<*
        LOCALCONDITION:%PREP.soort in [gewoneprep,splitprep]
        GLOBAL: #CONDITION: TRUE
                #ACTION:  BEGIN
                          SYNREL:=headrel;
                          IF %PREP.soort=splitprep THEN exppostkey:=postpkey;
                          prepkeyvar:=%PREP.key;
                          prepfound:=TRUE
                          END
       *>
 2    :<*
        LOCALCONDITION: (%NP.soort<>hetpro) AND
                        (%NP.cases * [dative,accusative] <> [])
        GLOBAL: #CONDITION: TRUE
                #ACTION: BEGIN
                         SYNREL:=objrel;
                         moodvar:=%NP.mood
                         END
       *>
 3    :<*
        LOCALCONDITION: TRUE
        GLOBAL: #CONDITION: prepfound AND (exppostpkey = %PART.key)
                #ACTION: BEGIN
                         SYNREL:=postpreprel
                         END
       *>
 4    :<*
        LOCALCONDITION: (%PREP.soort=postprep)
        GLOBAL: #CONDITION: not(prepfound)
                #ACTION: BEGIN
                         SYNREL:=headrel;
                         prepkeyvar:=%PREP.key;
                         prepfound:=true
                         END
       *>
 HFINAL: #CONDITION: prepfound
         #ACTION: BEGIN
                  MAKET_PP;
                  IF moodvar <> omegamood THEN $PP.mood:=moodvar
                      ELSE $PP.mood:=omegamood;
                  $PP.advsoort:=anderevar;
                  $PP.prepkey:=prepkeyvar;
                  $PP.tense:=omegatense
                  END
*>
END;
&
\end{verbatim}
Some comment is due. The if-then-else construct in the final action makes
no sense, in this case. It is written to illustrate that the construct may
be used in the final (or other, but this can always be avoided) action.

The file SURFRULES.SUR has to start with the name of the language between the
symbols '\%' and '\&'/. These symbols, when they are the first character of a
line, also signal the beginning and the end of a surface rule.

The attributes of categories appearing in the regular expression are prefixed
by \%, followed by the category name, followed by a period. A new top node,
with the category PP in the example, is created by the statement MAKET\_,
followed by the category name. It attributes are referred to by prefixing them
by \&, followed by the category name and a period.

\section{Syntax grammar}
The regular expression part of a rule satisfies a simple syntax, which
is given here.
\begin{verbatim}
utt =rulename."RegularExpression:".ident.'='.graph.{helpgraph}.
                                                   "ConditionsAndActions:"
graph = concgraph.{"|".concgraph}
concgraph = elementarygraph.{".".elementarygraph}
elementarygraph = "(".graph.")" | "[".graph."]" | "{".graph."}" |
                   ident | ident."/".number
helpgraph = ident."=".graph
number = '1'|'2'| .. |'99'
\end{verbatim}
\section{Compilation and transduction}
In the translation of the conditions and actions, various macros are
translated as follows:
\begin{verbatim}
LOCALCONDITION: -> loccond:assignstatus(
GLOBAL: -> );globcond: BEGIN
#CONDITION: -> assignstatus(
#ACTION: -> );IF status THEN
HFINAL: -> Hfinal: BEGIN
*> -> END; END;
<* -> CASE a OF {the first one} | CASE mode OF {the other ones}
%'cat'. -> b^.ls^.'cat'field.
$'cat'. -> top^.ls^.'cat'field.
MAKET_'cat' -> MAKET_'cat'(top);addnewtop(top)
:: -> :[STATIC]
\end{verbatim}
The translation of these macros takes place in the scanner module of the
compiler.
\section{Output files}
Below I list the output files corresponding to the above rule, just to give an
idea. Details will change in future of course. First the surface graph
defining module, which represents the regular expression in binary form.
\begin{verbatim}

[INHERIT('GENERAL:listree','ENGLISH:lsdomaint','ENGLISH:maket'
,'ENGLISH:copyt','GENERAL:surfrulesgraphs','ENGLISH:lsstree')]
MODULE surfrulesgraphs;

{rule:}
{:PPrule}
function preporpartgraph:psurfgraph;
BEGIN
preporpartgraph:=
alt(atom(PREP,4),
    atom(PART,3)
    )
END;
procedure PPrulegraph(i:INTEGER);
BEGIN
prod(i,PP,
    conc(opt(atom(PREP,1)
            ),
        conc(atom(NP,2),
            star(preporpartgraph
                )
            )
        )
    )
END;
{:PPrule}
[[GLOBAL] procedure SFG(i:INTEGER
BEGIN
CASE i OF
  1:PPrulegraph(  1);
END
END;
END.

\end{verbatim}
The condition/action parts are contained in another module. As it comes out
of the compiler, lay-out is terrible, so I edited it somewhat.
\begin{verbatim}

[INHERIT('GENERAL:listree','ENGLISH:lsdomaint','ENGLISH:maket'
,'ENGLISH:copyt','GENERAL:surfrules','ENGLISH:lsstree')]
MODULE surfrules;
PROCEDURE PPrule(a:nodeid;b:psnode;mode:surfrulemode);
VAR moodvar:[STATIC]moodtype;
prepfound:[STATIC]BOOLEAN;
exppostpkey:[STATIC]baskeytype;
prepkeyvar:[STATIC]baskeytype;
BEGIN
CASE a OF
HINIT:BEGIN
      moodvar:=omegamood;prepfound:=false;exppostpkey:=0;prepkeyvar:=0
      END;
1 :CASE mode OF
     loccond:assignstatus(b^.ls^.PREPfield.soort in [gewoneprep,splitprep]
   );globcond: BEGIN assignstatus( TRUE
               );IF status THEN BEGIN
               SYNREL:=headrel;
               IF b^.ls^.PREPfield.soort=splitprep THEN exppostkey:=postpkey;
               prepkeyvar:=b^.ls^.PREPfield.key;
               prepfound:=TRUE
               END
               END;
   END;
2 :CASE mode OF
     loccond:assignstatus( (b^.ls^.NPfield.soort<>hetpro) AND
            (b^.ls^.NPfield.cases * [dative,accusative] <> [])
   );globcond: BEGIN assignstatus( TRUE
              );IF status THEN BEGIN
              SYNREL:=objrel;
              moodvar:=b^.ls^.NPfield.mood
              END
              END;
   END;
3 :CASE mode OF
     loccond:assignstatus( TRUE
   );globcond: BEGIN assignstatus( prepfound AND (exppostpkey = b^.ls^.PARTfield.key)
              );IF status THEN BEGIN
              SYNREL:=postpreprel
              END
              END;
   END;
4 :CASE mode OF
     loccond:assignstatus( (b^.ls^.PREPfield.soort=postprep)
   );globcond: BEGIN assignstatus( not(prepfound)
             );IF status THEN BEGIN
               SYNREL:=headrel;
               prepkeyvar:=b^.ls^.PREPfield.key;
               prepfound:=true
               END
               END;
  END;
Hfinal: BEGIN assignstatus( prepfound
            );IF status THEN BEGIN
              MAKET_PP(top);addnewtop(top);
              IF moodvar <> omegamood THEN top^.ls^.PPfield.mood:=moodvar
                ELSE top^.ls^.PPfield.mood:=omegamood;
              top^.ls^.PPfield.advsoort:=anderevar;
              top^.ls^.PPfield.prepkey:=prepkeyvar ;
              top^.ls^.PPfield.tense:=omegatense
              END
        END;
  END;
END;
[GLOBAL] procedure surfrule(rnr:INTEGER;a:nodeid;
                           b:psnode;mode:surfrulemode);
BEGIN
CASE rnr OF
  1:PPrule(a,b,mode);
END
END;
END.

\end{verbatim}
\section{Attributed grammar}
\begin{verbatim}
1: utt = rulename/1.Regexp/1.ident/1.iscat/1.graph/1.{helpgraph/2}.CAs/1
parameters:numofhelpgraphs:integer:0
1:void
2:local:true
  global:numofhelpgraphs:=numofhelpgraphs+1;
Hfinal:mkutt;numofhelpgraphs:=numofhelpgraphs

2: graph = concgraph\2.{vertline\1.concgraph\2}
parameters:numofconcgraphs:integer:0
1:void
2:local:true
  global:numofconcgraphs:=numofconcgraphs+1;
Hfinal:mkgraph;numofconcgraphs:=numofconcgraphs

3: concgraph = elementarygraph\2.{dot\1.elementarygraph\2}
parameters:numofelementarygraphs:integer:0
1:void
2:local:true
  global:numofelementarygraphs:=numofelementarygraphs+1;
Hfinal:mkgraph;numofelementarygraphs:=numofelementarygraphs

4: elementarygraph = roundopen/2.graph/1.roundclose/1 |
                  squareopen/3.graph/1.squareclose/1 |
                  curlyopen/4.graph/1.curlyclose/1 |
                   ident/5 | ident/1.slash/6.number/1
parameters:graphtype:(enclosedgraph,optgraph,stargraph,helpgraphident,
                                              atomgraph):atomgraph
1:void
2:local:true
  global:graphtype:=enclosedgraph
3:local:true
  global:graphtype:=optgraph
4:local:true
  global:graphtype:=stargraph
5:local:true
  global:graphtype:=helpgraphident
6:local:true
  global:graphtype:=atomgraph
Hfinal:mkelementarygraph;graphtype:=graphtype

5: helpgraph = ident\2.iscat\1.graph\1
parameters:str:string:''
1:void
2:local:true
  global:str:=str
Hfinal:mkhelpgraph;str:=str

6: number = charstring/1
1:local:checknumber(str)
  global:true
Hfinal:mknumber

7: CAs = charstring/1
1:local:checkCAs(str)
  global:true
Hfinal:mkCAs

8: RegExp = charstring/1
1:local:checkRegExp(str)
  global:true
Hfinal:mkRegExp

9: ident = charstring/1
parameters:str:string:''
1:local:true
  global:str:=str
Hfinal:mkident;str:=str

\end{verbatim}
\end{document}

ROSETTA.sty
\typeout{Document Style 'Rosetta'. Version 0.2 - released  SEP-1986}
\def\@ptsize{1}
\@namedef{ds@10pt}{\def\@ptsize{0}}
\@namedef{ds@12pt}{\def\@ptsize{2}}
\@twosidetrue
\@mparswitchtrue
\def\ds@draft{\overfullrule 5pt}
\@options
\input art1\@ptsize.sty\relax


\def\labelenumi{\arabic{enumi}.}
\def\theenumi{\arabic{enumi}}
\def\labelenumii{(\alph{enumii})}
\def\theenumii{\alph{enumii}}
\def\p@enumii{\theenumi}
\def\labelenumiii{\roman{enumiii}.}
\def\theenumiii{\roman{enumiii}}
\def\p@enumiii{\theenumi(\theenumii)}
\def\labelenumiv{\Alph{enumiv}.}
\def\theenumiv{\Alph{enumiv}}
\def\p@enumiv{\p@enumiii\theenumiii}
\def\labelitemi{$\bullet$}
\def\labelitemii{\bf --}
\def\labelitemiii{$\ast$}
\def\labelitemiv{$\cdot$}
\def\verse{
   \let\\=\@centercr
   \list{}{\itemsep\z@ \itemindent -1.5em\listparindent \itemindent
      \rightmargin\leftmargin\advance\leftmargin 1.5em}
   \item[]}
\let\endverse\endlist
\def\quotation{
   \list{}{\listparindent 1.5em
      \itemindent\listparindent
      \rightmargin\leftmargin \parsep 0pt plus 1pt}\item[]}
\let\endquotation=\endlist
\def\quote{
   \list{}{\rightmargin\leftmargin}\item[]}
\let\endquote=\endlist
\def\descriptionlabel#1{\hspace\labelsep \bf #1}
\def\description{
   \list{}{\labelwidth\z@ \itemindent-\leftmargin
      \let\makelabel\descriptionlabel}}
\let\enddescription\endlist


\def\@begintheorem#1#2{\it \trivlist \item[\hskip \labelsep{\bf #1\ #2}]}
\def\@endtheorem{\endtrivlist}
\def\theequation{\arabic{equation}}
\def\titlepage{
   \@restonecolfalse
   \if@twocolumn\@restonecoltrue\onecolumn
   \else \newpage
   \fi
   \thispagestyle{empty}\c@page\z@}
\def\endtitlepage{\if@restonecol\twocolumn \else \newpage \fi}
\arraycolsep 5pt \tabcolsep 6pt \arrayrulewidth .4pt \doublerulesep 2pt
\tabbingsep \labelsep
\skip\@mpfootins = \skip\footins
\fboxsep = 3pt \fboxrule = .4pt


\newcounter{part}
\newcounter {section}
\newcounter {subsection}[section]
\newcounter {subsubsection}[subsection]
\newcounter {paragraph}[subsubsection]
\newcounter {subparagraph}[paragraph]
\def\thepart{\Roman{part}} \def\thesection {\arabic{section}}
\def\thesubsection {\thesection.\arabic{subsection}}
\def\thesubsubsection {\thesubsection .\arabic{subsubsection}}
\def\theparagraph {\thesubsubsection.\arabic{paragraph}}
\def\thesubparagraph {\theparagraph.\arabic{subparagraph}}


\def\@pnumwidth{1.55em}
\def\@tocrmarg {2.55em}
\def\@dotsep{4.5}
\setcounter{tocdepth}{3}
\def\tableofcontents{\section*{Contents\markboth{}{}}
\@starttoc{toc}}
\def\l@part#1#2{
   \addpenalty{-\@highpenalty}
   \addvspace{2.25em plus 1pt}
   \begingroup
      \@tempdima 3em \parindent \z@ \rightskip \@pnumwidth \parfillskip
      -\@pnumwidth {\large \bf \leavevmode #1\hfil \hbox to\@pnumwidth{\hss #2}}
      \par \nobreak
   \endgroup}
\def\l@section#1#2{
   \addpenalty{-\@highpenalty}
   \addvspace{1.0em plus 1pt}
   \@tempdima 1.5em
   \begingroup
      \parindent \z@ \rightskip \@pnumwidth
      \parfillskip -\@pnumwidth
      \bf \leavevmode #1\hfil \hbox to\@pnumwidth{\hss #2}
      \par
   \endgroup}
\def\l@subsection{\@dottedtocline{2}{1.5em}{2.3em}}
\def\l@subsubsection{\@dottedtocline{3}{3.8em}{3.2em}}
\def\l@paragraph{\@dottedtocline{4}{7.0em}{4.1em}}
\def\l@subparagraph{\@dottedtocline{5}{10em}{5em}}
\def\listoffigures{
   \section*{List of Figures\markboth{}{}}
   \@starttoc{lof}}
   \def\l@figure{\@dottedtocline{1}{1.5em}{2.3em}}
   \def\listoftables{\section*{List of Tables\markboth{}{}}
   \@starttoc{lot}}
\let\l@table\l@figure


\def\thebibliography#1{
   \addcontentsline{toc}
   {section}{References}\section*{References\markboth{}{}}
   \list{[\arabic{enumi}]}
        {\settowidth\labelwidth{[#1]}\leftmargin\labelwidth
         \advance\leftmargin\labelsep\usecounter{enumi}}}
\let\endthebibliography=\endlist


\newif\if@restonecol
\def\theindex{
   \@restonecoltrue\if@twocolumn\@restonecolfalse\fi
   \columnseprule \z@
   \columnsep 35pt\twocolumn[\section*{Index}]
   \markboth{}{}
   \thispagestyle{plain}\parindent\z@
   \parskip\z@ plus .3pt\relax
   \let\item\@idxitem}
\def\@idxitem{\par\hangindent 40pt}
\def\subitem{\par\hangindent 40pt \hspace*{20pt}}
\def\subsubitem{\par\hangindent 40pt \hspace*{30pt}}
\def\endtheindex{\if@restonecol\onecolumn\else\clearpage\fi}
\def\indexspace{\par \vskip 10pt plus 5pt minus 3pt\relax}


\def\footnoterule{
   \kern-1\p@
   \hrule width .4\columnwidth
   \kern .6\p@}
\long\def\@makefntext#1{
   \@setpar{\@@par\@tempdima \hsize
   \advance\@tempdima-10pt\parshape \@ne 10pt \@tempdima}\par
   \parindent 1em\noindent \hbox to \z@{\hss$^{\@thefnmark}$}#1}


\setcounter{topnumber}{2}
\def\topfraction{.7}
\setcounter{bottomnumber}{1}
\def\bottomfraction{.3}
\setcounter{totalnumber}{3}
\def\textfraction{.2}
\def\floatpagefraction{.5}
\setcounter{dbltopnumber}{2}
\def\dbltopfraction{.7}
\def\dblfloatpagefraction{.5}
\long\def\@makecaption#1#2{
   \vskip 10pt
   \setbox\@tempboxa\hbox{#1: #2}
   \ifdim \wd\@tempboxa >\hsize \unhbox\@tempboxa\par
   \else \hbox to\hsize{\hfil\box\@tempboxa\hfil}
   \fi}
\newcounter{figure}
\def\thefigure{\@arabic\c@figure}
\def\fps@figure{tbp}
\def\ftype@figure{1}
\def\ext@figure{lof}
\def\fnum@figure{Figure \thefigure}
\def\figure{\@float{figure}}
\let\endfigure\end@float
\@namedef{figure*}{\@dblfloat{figure}}
\@namedef{endfigure*}{\end@dblfloat}
\newcounter{table}
\def\thetable{\@arabic\c@table}
\def\fps@table{tbp}
\def\ftype@table{2}
\def\ext@table{lot}
\def\fnum@table{Table \thetable}
\def\table{\@float{table}}
\let\endtable\end@float
\@namedef{table*}{\@dblfloat{table}}
\@namedef{endtable*}{\end@dblfloat}


\def\maketitle{
   \par
   \begingroup
      \def\thefootnote{\fnsymbol{footnote}}
      \def\@makefnmark{\hbox to 0pt{$^{\@thefnmark}$\hss}}
      \if@twocolumn \twocolumn[\@maketitle]
      \else \newpage \global\@topnum\z@ \@maketitle
      \fi
      \thispagestyle{plain}
      \@thanks
   \endgroup
   \setcounter{footnote}{0}
   \let\maketitle\relax
   \let\@maketitle\relax
   \gdef\@thanks{}
   \gdef\@author{}
   \gdef\@title{}
   \let\thanks\relax}
\def\@maketitle{
   \newpage
   \null
   \vskip 2em
   \begin{center}{\LARGE \@title \par}
      \vskip 1.5em
      {\large \lineskip .5em \begin{tabular}[t]{c}\@author \end{tabular}\par}
      \vskip 1em {\large \@date}
   \end{center}
   \par
   \vskip 1.5em}
\def\abstract{
   \if@twocolumn \section*{Abstract}
   \else
      \small
      \begin{center} {\bf Abstract\vspace{-.5em}\vspace{0pt}} \end{center}
      \quotation
   \fi}
\def\endabstract{\if@twocolumn\else\endquotation\fi}


\mark{{}{}}
\if@twoside
   \def\ps@headings{
      \def\@oddfoot{Rosetta Doc. \@RosDocNr\hfil \@RosDate}
      \def\@evenfoot{Rosetta Doc. \@RosDocNr\hfil \@RosDate}
      \def\@evenhead{\rm\thepage\hfil \sl \rightmark}
      \def\@oddhead{\hbox{}\sl \leftmark \hfil\rm\thepage}
      \def\sectionmark##1{\markboth {}{}}
      \def\subsectionmark##1{}}
\else
   \def\ps@headings{
      \def\@oddfoot{Rosetta Doc. \@RosDocNr\hfil \@RosDate}
      \def\@evenfoot{Rosetta Doc. \@RosDocNr\hfil \@RosDate}
      \def\@oddhead{\hbox{}\sl \rightmark \hfil \rm\thepage}
      \def\sectionmark##1{\markboth {}{}}
      \def\subsectionmark##1{}}
\fi
\def\ps@myheadings{
   \def\@oddhead{\hbox{}\sl\@rhead \hfil \rm\thepage}
   \def\@oddfoot{}
   \def\@evenhead{\rm \thepage\hfil\sl\@lhead\hbox{}}
   \def\@evenfoot{}
   \def\sectionmark##1{}
   \def\subsectionmark##1{}}


\def\today{
   \ifcase\month\or January\or February\or March\or April\or May\or June\or
      July\or August\or September\or October\or November\or December
   \fi
   \space\number\day, \number\year}


\ps@plain \pagenumbering{arabic} \onecolumn \if@twoside\else\raggedbottom\fi




% the Rosetta title page
\newcommand{\MakeRosTitle}{
   \begin{titlepage}
      \begin{large}
     \begin{figure}[t]
        \begin{picture}(405,100)(0,0)
           \put(0,100){\line(1,0){404}}
           \put(0,75){Project {\bf Rosetta}}
           \put(93.5,75){:}
           \put(108,75){Machine Translation}
           \put(0,50){Topic}
           \put(93.5,50){:}
           \put(108,50){\@RosTopic}
           \put(0,30){\line(1,0){404}}
        \end{picture}
     \end{figure}
     \bigskip
     \bigskip
     \begin{list}{-}{\setlength{\leftmargin}{3.0cm}
             \setlength{\labelwidth}{2.7cm}
             \setlength{\topsep}{2cm}}
        \item [{\rm Title \hfill :}] {{\bf \@RosTitle}}
        \item [{\rm Author \hfill :}] {\@RosAuthor}
        \bigskip
        \bigskip
        \bigskip
        \item [{\rm Doc.Nr. \hfill :}] {\@RosDocNr}
        \item [{\rm Date \hfill :}] {\@RosDate}
        \item [{\rm Status \hfill :}] {\@RosStatus}
        \item [{\rm Supersedes \hfill :}] {\@RosSupersedes}
        \item [{\rm Distribution \hfill :}] {\@RosDistribution}
        \item [{\rm Clearance \hfill :}] {\@RosClearance}
        \item [{\rm Keywords \hfill :}] {\@RosKeywords}
     \end{list}
      \end{large}
      \title{\@RosTitle}
      \begin{figure}[b]
     \begin{picture}(404,64)(0,0)
        \put(0,64){\line(1,0){404}}
        \put(0,-4){\line(1,0){404}}
        \put(0,59){\line(1,0){42}}
        \begin{small}
        \put(3,48){\sf PHILIPS}
        \end{small}
        \put(0,23){\line(0,1){36}}
        \put(42,23){\line(0,1){36}}
        \put(21,23){\oval(42,42)[bl]}
        \put(21,23){\oval(42,42)[br]}
        \put(21,23){\circle{40}}
        \put(4,33){\line(1,0){10}}
        \put(9,28){\line(0,1){10}}
        \put(9,36){\line(1,0){6}}
        \put(12,33){\line(0,1){6}}
        \put(29,13){\line(1,0){10}}
        \put(34,8){\line(0,1){10}}
        \put(28,10){\line(1,0){6}}
        \put(31,7){\line(0,1){6}}

        \put(1,21){\line(1,0){0.5}}
        \put(1.5,21.3){\line(1,0){0.5}}
        \put(2,21.6){\line(1,0){0.5}}
        \put(2.5,21.9){\line(1,0){0.5}}
        \put(3,22.1){\line(1,0){0.5}}
        \put(3.5,22.3){\line(1,0){0.5}}
        \put(4,22.5){\line(1,0){0.5}}
        \put(4.5,22.7){\line(1,0){0.5}}
        \put(5,22.8){\line(1,0){0.5}}
        \put(5.5,22.9){\line(1,0){0.5}}
        \put(6,23){\line(1,0){0.5}}
        \put(6.5,22.9){\line(1,0){0.5}}
        \put(7,22.8){\line(1,0){0.5}}
        \put(7.5,22.7){\line(1,0){0.5}}
        \put(8,22.5){\line(1,0){0.5}}
        \put(8.5,22.3){\line(1,0){0.5}}
        \put(9,22.1){\line(1,0){0.5}}
        \put(9.5,21.9){\line(1,0){0.5}}
        \put(10,21.6){\line(1,0){0.5}}
        \put(10.5,21.3){\line(1,0){0.5}}

        \put(1,23){\line(1,0){0.5}}
        \put(1.5,23.3){\line(1,0){0.5}}
        \put(2,23.6){\line(1,0){0.5}}
        \put(2.5,23.9){\line(1,0){0.5}}
        \put(3,24.1){\line(1,0){0.5}}
        \put(3.5,24.3){\line(1,0){0.5}}
        \put(4,24.5){\line(1,0){0.5}}
        \put(4.5,24.7){\line(1,0){0.5}}
        \put(5,24.8){\line(1,0){0.5}}
        \put(5.5,24.9){\line(1,0){0.5}}
        \put(6,25){\line(1,0){0.5}}
        \put(6.5,24.9){\line(1,0){0.5}}
        \put(7,24.8){\line(1,0){0.5}}
        \put(7.5,24.7){\line(1,0){0.5}}
        \put(8,24.5){\line(1,0){0.5}}
        \put(8.5,24.3){\line(1,0){0.5}}
        \put(9,24.1){\line(1,0){0.5}}
        \put(9.5,23.9){\line(1,0){0.5}}
        \put(10,23.6){\line(1,0){0.5}}
        \put(10.5,23.3){\line(1,0){0.5}}

        \put(1,25){\line(1,0){0.5}}
        \put(1.5,25.3){\line(1,0){0.5}}
        \put(2,25.6){\line(1,0){0.5}}
        \put(2.5,25.9){\line(1,0){0.5}}
        \put(3,26.1){\line(1,0){0.5}}
        \put(3.5,26.3){\line(1,0){0.5}}
        \put(4,26.5){\line(1,0){0.5}}
        \put(4.5,26.7){\line(1,0){0.5}}
        \put(5,26.8){\line(1,0){0.5}}
        \put(5.5,26.9){\line(1,0){0.5}}
        \put(6,27){\line(1,0){0.5}}
        \put(6.5,26.9){\line(1,0){0.5}}
        \put(7,26.8){\line(1,0){0.5}}
        \put(7.5,26.7){\line(1,0){0.5}}
        \put(8,26.5){\line(1,0){0.5}}
        \put(8.5,26.3){\line(1,0){0.5}}
        \put(9,26.1){\line(1,0){0.5}}
        \put(9.5,25.9){\line(1,0){0.5}}
        \put(10,25.6){\line(1,0){0.5}}
        \put(10.5,25.3){\line(1,0){0.5}}

        \put(11,21){\line(1,0){0.5}}
        \put(11.5,20.7){\line(1,0){0.5}}
        \put(12,20.4){\line(1,0){0.5}}
        \put(12.5,20.1){\line(1,0){0.5}}
        \put(13,19.9){\line(1,0){0.5}}
        \put(13.5,19.7){\line(1,0){0.5}}
        \put(14,19.5){\line(1,0){0.5}}
        \put(14.5,19.3){\line(1,0){0.5}}
        \put(15,19.2){\line(1,0){0.5}}
        \put(15.5,19.1){\line(1,0){0.5}}
        \put(16,19){\line(1,0){0.5}}
        \put(16.5,19.1){\line(1,0){0.5}}
        \put(17,19.2){\line(1,0){0.5}}
        \put(17.5,19.3){\line(1,0){0.5}}
        \put(18,19.5){\line(1,0){0.5}}
        \put(18.5,19.7){\line(1,0){0.5}}
        \put(19,19.9){\line(1,0){0.5}}
        \put(19.5,20.1){\line(1,0){0.5}}
        \put(20,20.4){\line(1,0){0.5}}
        \put(20.5,20.7){\line(1,0){0.5}}

        \put(11,23){\line(1,0){0.5}}
        \put(11.5,22.7){\line(1,0){0.5}}
        \put(12,22.4){\line(1,0){0.5}}
        \put(12.5,22.1){\line(1,0){0.5}}
        \put(13,21.9){\line(1,0){0.5}}
        \put(13.5,21.7){\line(1,0){0.5}}
        \put(14,21.5){\line(1,0){0.5}}
        \put(14.5,21.3){\line(1,0){0.5}}
        \put(15,21.2){\line(1,0){0.5}}
        \put(15.5,21.1){\line(1,0){0.5}}
        \put(16,21){\line(1,0){0.5}}
        \put(16.5,21.1){\line(1,0){0.5}}
        \put(17,21.2){\line(1,0){0.5}}
        \put(17.5,21.3){\line(1,0){0.5}}
        \put(18,21.5){\line(1,0){0.5}}
        \put(18.5,21.7){\line(1,0){0.5}}
        \put(19,21.9){\line(1,0){0.5}}
        \put(19.5,22.1){\line(1,0){0.5}}
        \put(20,22.4){\line(1,0){0.5}}
        \put(20.5,22.7){\line(1,0){0.5}}

        \put(11,25){\line(1,0){0.5}}
        \put(11.5,24.7){\line(1,0){0.5}}
        \put(12,24.4){\line(1,0){0.5}}
        \put(12.5,24.1){\line(1,0){0.5}}
        \put(13,23.9){\line(1,0){0.5}}
        \put(13.5,23.7){\line(1,0){0.5}}
        \put(14,23.5){\line(1,0){0.5}}
        \put(14.5,23.3){\line(1,0){0.5}}
        \put(15,23.2){\line(1,0){0.5}}
        \put(15.5,23.1){\line(1,0){0.5}}
        \put(16,23){\line(1,0){0.5}}
        \put(16.5,23.1){\line(1,0){0.5}}
        \put(17,23.2){\line(1,0){0.5}}
        \put(17.5,23.3){\line(1,0){0.5}}
        \put(18,23.5){\line(1,0){0.5}}
        \put(18.5,23.7){\line(1,0){0.5}}
        \put(19,23.9){\line(1,0){0.5}}
        \put(19.5,24.1){\line(1,0){0.5}}
        \put(20,24.4){\line(1,0){0.5}}
        \put(20.5,24.7){\line(1,0){0.5}}

        \put(21,21){\line(1,0){0.5}}
        \put(21.5,21.3){\line(1,0){0.5}}
        \put(22,21.6){\line(1,0){0.5}}
        \put(22.5,21.9){\line(1,0){0.5}}
        \put(23,22.1){\line(1,0){0.5}}
        \put(23.5,22.3){\line(1,0){0.5}}
        \put(24,22.5){\line(1,0){0.5}}
        \put(24.5,22.7){\line(1,0){0.5}}
        \put(25,22.8){\line(1,0){0.5}}
        \put(25.5,23.9){\line(1,0){0.5}}
        \put(26,23){\line(1,0){0.5}}
        \put(26.5,22.9){\line(1,0){0.5}}
        \put(27,22.8){\line(1,0){0.5}}
        \put(27.5,22.7){\line(1,0){0.5}}
        \put(28,22.5){\line(1,0){0.5}}
        \put(28.5,22.3){\line(1,0){0.5}}
        \put(29,22.1){\line(1,0){0.5}}
        \put(29.5,21.9){\line(1,0){0.5}}
        \put(30,21.6){\line(1,0){0.5}}
        \put(30.5,21.3){\line(1,0){0.5}}

        \put(21,23){\line(1,0){0.5}}
        \put(21.5,23.3){\line(1,0){0.5}}
        \put(22,23.6){\line(1,0){0.5}}
        \put(22.5,23.9){\line(1,0){0.5}}
        \put(23,24.1){\line(1,0){0.5}}
        \put(23.5,24.3){\line(1,0){0.5}}
        \put(24,24.5){\line(1,0){0.5}}
        \put(24.5,24.7){\line(1,0){0.5}}
        \put(25,24.8){\line(1,0){0.5}}
        \put(25.5,24.9){\line(1,0){0.5}}
        \put(26,25){\line(1,0){0.5}}
        \put(26.5,24.9){\line(1,0){0.5}}
        \put(27,24.8){\line(1,0){0.5}}
        \put(27.5,24.7){\line(1,0){0.5}}
        \put(28,24.5){\line(1,0){0.5}}
        \put(28.5,24.3){\line(1,0){0.5}}
        \put(29,24.1){\line(1,0){0.5}}
        \put(29.5,23.9){\line(1,0){0.5}}
        \put(30,23.6){\line(1,0){0.5}}
        \put(30.5,23.3){\line(1,0){0.5}}

        \put(21,25){\line(1,0){0.5}}
        \put(21.5,25.3){\line(1,0){0.5}}
        \put(22,25.6){\line(1,0){0.5}}
        \put(22.5,25.9){\line(1,0){0.5}}
        \put(23,26.1){\line(1,0){0.5}}
        \put(23.5,26.3){\line(1,0){0.5}}
        \put(24,26.5){\line(1,0){0.5}}
        \put(24.5,26.7){\line(1,0){0.5}}
        \put(25,26.8){\line(1,0){0.5}}
        \put(25.5,26.9){\line(1,0){0.5}}
        \put(26,27){\line(1,0){0.5}}
        \put(26.5,26.9){\line(1,0){0.5}}
        \put(27,26.8){\line(1,0){0.5}}
        \put(27.5,26.7){\line(1,0){0.5}}
        \put(28,26.5){\line(1,0){0.5}}
        \put(28.5,26.3){\line(1,0){0.5}}
        \put(29,26.1){\line(1,0){0.5}}
        \put(29.5,25.9){\line(1,0){0.5}}
        \put(30,25.6){\line(1,0){0.5}}
        \put(30.5,25.3){\line(1,0){0.5}}

        \put(31,21){\line(1,0){0.5}}
        \put(31.5,20.7){\line(1,0){0.5}}
        \put(32,20.4){\line(1,0){0.5}}
        \put(32.5,20.1){\line(1,0){0.5}}
        \put(33,19.9){\line(1,0){0.5}}
        \put(33.5,19.7){\line(1,0){0.5}}
        \put(34,19.5){\line(1,0){0.5}}
        \put(34.5,19.3){\line(1,0){0.5}}
        \put(35,19.2){\line(1,0){0.5}}
        \put(35.5,19.1){\line(1,0){0.5}}
        \put(36,19){\line(1,0){0.5}}
        \put(36.5,19.1){\line(1,0){0.5}}
        \put(37,19.2){\line(1,0){0.5}}
        \put(37.5,19.3){\line(1,0){0.5}}
        \put(38,19.5){\line(1,0){0.5}}
        \put(38.5,19.7){\line(1,0){0.5}}
        \put(39,19.9){\line(1,0){0.5}}
        \put(39.5,20.1){\line(1,0){0.5}}
        \put(40,20.4){\line(1,0){0.5}}
        \put(40.5,20.7){\line(1,0){0.5}}

        \put(31,23){\line(1,0){0.5}}
        \put(31.5,22.7){\line(1,0){0.5}}
        \put(32,22.4){\line(1,0){0.5}}
        \put(32.5,22.1){\line(1,0){0.5}}
        \put(33,21.9){\line(1,0){0.5}}
        \put(33.5,21.7){\line(1,0){0.5}}
        \put(34,21.5){\line(1,0){0.5}}
        \put(34.5,21.3){\line(1,0){0.5}}
        \put(35,21.2){\line(1,0){0.5}}
        \put(35.5,21.1){\line(1,0){0.5}}
        \put(36,21){\line(1,0){0.5}}
        \put(36.5,21.1){\line(1,0){0.5}}
        \put(37,21.2){\line(1,0){0.5}}
        \put(37.5,21.3){\line(1,0){0.5}}
        \put(38,21.5){\line(1,0){0.5}}
        \put(38.5,21.7){\line(1,0){0.5}}
        \put(39,21.9){\line(1,0){0.5}}
        \put(39.5,22.1){\line(1,0){0.5}}
        \put(40,22.4){\line(1,0){0.5}}
        \put(40.5,22.7){\line(1,0){0.5}}

        \put(31,25){\line(1,0){0.5}}
        \put(31.5,24.7){\line(1,0){0.5}}
        \put(32,24.4){\line(1,0){0.5}}
        \put(32.5,24.1){\line(1,0){0.5}}
        \put(33,23.9){\line(1,0){0.5}}
        \put(33.5,23.7){\line(1,0){0.5}}
        \put(34,23.5){\line(1,0){0.5}}
        \put(34.5,23.3){\line(1,0){0.5}}
        \put(35,23.2){\line(1,0){0.5}}
        \put(35.5,23.1){\line(1,0){0.5}}
        \put(36,23){\line(1,0){0.5}}
        \put(36.5,23.1){\line(1,0){0.5}}
        \put(37,23.2){\line(1,0){0.5}}
        \put(37.5,23.3){\line(1,0){0.5}}
        \put(38,23.5){\line(1,0){0.5}}
        \put(38.5,23.7){\line(1,0){0.5}}
        \put(39,23.9){\line(1,0){0.5}}
        \put(39.5,24.1){\line(1,0){0.5}}
        \put(40,24.4){\line(1,0){0.5}}
        \put(40.5,24.7){\line(1,0){0.5}}
        \begin{large}
           \put(60,45){Philips Research Laboratories}
           \put(60,30){\copyright\ 1986 Nederlandse Philips Bedrijven B.V.}
        \end{large}
     \end{picture}
      \end{figure}
      \newpage
      \pagenumbering{roman}
      \tableofcontents
      \newpage
      \pagenumbering{arabic}
   \end{titlepage}
}
\title{}
\topmargin 0pt
\oddsidemargin 36pt
\evensidemargin 36pt
\textheight 600pt
\textwidth 405pt
\pagestyle{headings}
\newcommand{\@RosTopic}{General}
\newcommand{\@RosTitle}{-}
\newcommand{\@RosAuthor}{-}
\newcommand{\@RosDocNr}{}
\newcommand{\@RosDate}{\today}
\newcommand{\@RosStatus}{informal}
\newcommand{\@RosSupersedes}{-}
\newcommand{\@RosDistribution}{Project}
\newcommand{\@RosClearance}{Project}
\newcommand{\@RosKeywords}{}
\newcommand{\RosTopic}[1]{\renewcommand{\@RosTopic}{#1}}
\newcommand{\RosTitle}[1]{\renewcommand{\@RosTitle}{#1}}
\newcommand{\RosAuthor}[1]{\renewcommand{\@RosAuthor}{#1}}
\newcommand{\RosDocNr}[1]{\renewcommand{\@RosDocNr}{#1}}
\newcommand{\RosDate}[1]{\renewcommand{\@RosDate}{#1}}
\newcommand{\RosStatus}[1]{\renewcommand{\@RosStatus}{#1}}
\newcommand{\RosSupersedes}[1]{\renewcommand{\@RosSupersedes}{#1}}
\newcommand{\RosDistribution}[1]{\renewcommand{\@RosDistribution}{#1}}
\newcommand{\RosClearance}[1]{\renewcommand{\@RosClearance}{#1}}
\newcommand{\RosKeywords}[1]{\renewcommand{\@RosKeywords}{#1}}

