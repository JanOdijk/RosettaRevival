
\documentstyle{Rosetta}
\begin{document}
   \RosTopic{General}
   \RosTitle{Notulen Groepsvergadering 6-11-89}
   \RosAuthor{Andr\'{e} Schenk}
   \RosDocNr{415}
   \RosDate{December 4, 1989}
   \RosStatus{approved}
   \RosSupersedes{-}
   \RosDistribution{Project}
   \RosClearance{Project}
   \RosKeywords{minutes}
   \MakeRosTitle
%
%
\begin{description}
\item[Aanwezig:] Lisette Appelo, Franciska de Jong, Jan Landsbergen, Ren\'{e}
                 Leermakers, Elly van Munster,
                 Jan Odijk, Joep Rous, 
                 Andr\'{e} Schenk,  Harm Smit
\item[Afwezig:] Margreet Sanders


\item[Agenda:]\mbox{}
  \begin{enumerate}
  \item Notulen
  \item Diversen
  \item Marktonderzoek etc.
  \item Voortgang
  \item Rondvraag
  \item Praatje Ren\'{e}
  \end{enumerate}
\end{description}

\section{Notulen}
De notulen van de vorige vergadering werden goedgekeurd. 

\section{Diversen}
- Jan L. bedankt iedereen voor de aanwezigheid bij zijn oratie en voor de 
medewerking bij het afscheid van Wijnand.

- Het persbericht is uit. Er is een copy bij Jan L.

- De kamer van Wijnand wordt voorlopig gebruikt door Loek Nijman. Als hij weer 
verhuist is de kamer weer van ons.

- Er zijn nieuwe ITLI publicaties aangekomen. Deze liggen bij Lisette.

- In het eerste halfjaar van 1990 wordt de vaste vergadertijd de tweede en de 
vierde maandagmiddag van de maand.

- Het overleg tussen research en de productdivisies heeft als resultaat gehad 
dat twee van de zes medewerkers zullen worden betaald door consumer
electronics.

- 22 november is een introductiedag voor nieuwe medewerkers. Van 14.45 - 15.25 
moeten wij een groepje van max. 10 personen bezighouden. Ren\'{e} en Lisette 
zullen dit verzorgen.

- Joep zei, dat hij vond dat er te weinig inhoudelijke zaken besproken 
werden in de vergaderingen. Ren\'{e} vroeg toen of Joep de vergadering saai 
vond. Joep kon niet precies onder woorden brengen wat hij bedoelde, maar hij 
vond de vergaderingen niet echt saai; hij vond dat er te weinig nieuwe 
voorstellen gedaan werden. Jan L. zei 
dat het vanzelf zou veranderen als er weer stukken besproken zouden gaan 
worden in de vergadering. Joep was het hier mee eens. Aangezien dit in principe
de normale
toestand is werd er besloten deze te laten zoals hij is. 

- De adressenlijst van Rosetta zit in de database bij Fred. Toevoegingen of 
wijzigingen moeten aan Fred doorgegeven worden.

- Franciska schrijft het stuk voor Informatie.
\section{Marktonderzoek etc.}

Het marktonderzoek is afgerond. De heer Bushing van M und W Test heeft
de resultaten aan Jaap Berkhoff, Jan L. en Andr\'{e} gepresenteerd. De 
resultaten zijn goed en voldoende om de stuugroep en evt. anderen te overtuigen 
van IMT. 23 november zal er -deo volente- een beslissing genomen worden door de 
stuurgroep.

Jaap B. stelt voor om niet het plan van vorig jaar door te voeren wat vrij 
groot was opgezet met meerdere talenparen. Hij stelt voor om aan \'{e}\'{e}n 
talenpaar te gaan werken waar snel resultaat uitkomt.

De Hoog en Waumans zullen van tevoren worden gepolst. Als deze het eens zijn is 
er een redelijke kans dat het zal lukken.

Het prototype moet op een 386 PC gaan draaien. Een mogelijkheid is om eerst op 
SUNs het systeem te ontwikkelen en daarna te converteren naar PC.


\section{Voortgang}
Het maken van de surface parser voor het Engels gaat voorspoedig. Voor het eind 
van het jaar is er een draaiende en gedeeltelijk geteste versie.

Volgens Ren\'{e} zijn er nog probleempjes met de effici\"{e}nte surface parser. 
Lange zinnen kan hij nog niet aan en de effici\"{e}ntie verbetering kan dus nog 
niet getest worden.

Harm is bezig met de Nederlandse en Engelse adjectieven.

Jan O. wil een vergaande verandering doorvoeren m.b.t. empty's. Dit wordt 
goedgekeurd. Jan O. zal het geheel co\"{o}rdineren.

Elly: de Spaanse werkwoorden (de eerste betekenis) zijn af. Elena is nu bezig
met de nouns. 

Joep, Franciska en Andr\'{e} gaan Rosetta3D co\"{o}rdineren. Zij zullen bepalen 
wat er moet gebeuren en wie dat gaat doen.


\section{Rondvraag}

Er waren geen vragen.

\section{Praatje}

Ren\'{e} hield een praatje over een stuk van Moore.
\end{document}


