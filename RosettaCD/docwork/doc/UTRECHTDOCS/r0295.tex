
\documentstyle{Rosetta}
\begin{document}
   \RosTopic{Rosetta3.Linguistics.Minutes}
   \RosTitle{Notulen Linguistenvergadering 08-12-88}
   \RosAuthor{Andr\'{e} Schenk}
   \RosDocNr{0295}
   \RosDate{December 15, 1988}
   \RosStatus{approved}
   \RosSupersedes{-}
   \RosDistribution{Project}
   \RosClearance{Project}
   \RosKeywords{minutes}
   \MakeRosTitle
%
%
\begin{description}
\item[Aanwezig:] Lisette Appelo, Franciska de Jong, Elly van Munster,
                 Jan Odijk, Margreet Sanders,
                 Andr\'{e} Schenk,  Harm Smit
\item[Afwezig:]


\item[Agenda:]\mbox{}
  \begin{enumerate}
  \item Notulen
  \item Prep en Part
  \item Opsplitsen in soorten van argumenten
  \item Bespreking van The Filling of BNOUN Entries
  \end{enumerate}
\end{description}

\section{Notulen}
De notulen van de vorige vergadering werden met enkele kleine wijzigingen
goedgekeurd. 

\section{Prep en Part}
Jan O. heeft het stuk Prep en Part uitgedeeld, dat gelezen moet worden voor de 
volgende vergadering.

\section{Opsplitsen in soorten van argumenten}
Jan O. stelt voor om wat we nu argumenten noemen in Rosetta4 op te splitsen in
3 soorten. De voordelen hiervan zijn dat het aantal patterns wordt gereduceerd
en dat het aantal zaken dat met empties wordt beregeld wordt gereduceerd. 

De eerste verdeling is in adjuncten en argumenten.

Adjuncten kunnen bij een werkwoord maar worden niet geintroduceerd met een 
variabele in de startregels. De variabele wordt alleen geintroduceerd, in 
aparte regels, als er een adjunct in de zin staat. Bijv. de resultatieve 
bepaling in {\em hij verft de deur groen} wordt dan behandeld als adjunct.

De argumenten worden opgesplitst in primaire en secundaire. Primaire argumenten 
zijn verplicht of optioneel, afhankelijk van het werkwoord, en secundaire zijn
optioneel. Een voorbeeld van een secundair argument is het voorzetselvoorwerp 
in {\em kijken naar}. Deze klasse expressies kenmerkt zich door het feit dat de 
betekenis van het simpele werkwoord bewaard blijft in de complexe expressie.

Een noodzakelijke voorwaarde voor het toekennen van de status van secundair
argument of adjunct aan een argument van een werkwoord is het feit of iets in
alle talen een adjunct is. Als dat niet het geval is moet het een secundair
argument worden. Verder moeten adjuncten locaal vertaald kunnen worden. 

Benefactieven kunnen nu ook behandeld worden als adjunct, bijv. in {\em ik koop 
Jan een boek}.

Een consequentie is dat expressies als {\em rekenen op} en {\em houden van} 
waarin de betekenis van het werkwoord niet behouden is zoals in {\em kijken 
naar} als idioom behandeld moeten worden.

\section{Bespreking van The Filling of BNOUN Entries}
Het document R0284: The Filling of BNOUN Entries van Franciska werd besproken. 
Opmerkingen en aanvullingen zullen in een volgende versie verwerkt worden.

\end{document}


