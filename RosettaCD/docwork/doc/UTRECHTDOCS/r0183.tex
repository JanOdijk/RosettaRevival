\documentstyle{Rosetta}
\begin{document}
   \RosTopic{General}
   \RosTitle{Notulen Rosetta vergadering 9-3-1987}
   \RosAuthor{Harm Smit}
   \RosDocNr{0183}
   \RosDate{\today}
   \RosStatus{approved}
   \RosSupersedes{-}
   \RosDistribution{Project}
   \RosClearance{Project}
   \RosKeywords{minutes}
   \MakeRosTitle
\begin{itemize}
  \item {\bf aanwezig}: Lilian Kopinga, Ans Post, Ren\'{e} Leermakers, 
             Jeroen Medema, Joep Rous, Jan Landsbergen, Andr\'{e} Schenk, 
             Lisette Appelo, Natalia Grygierczyk, Carel Fellinger, Jan Odijk, 
             Elly van Munster, Harm Smit, Jan Stevens, Margreet Sanders,
             Chris Hazenberg.
  \item {\bf afwezig}: Franciska de Jong.
  \item {\bf Agenda}:
    \begin{enumerate}
       \item Opening en notulen
       \item Diverse mededelingen
       \item Micro VAX
       \item Besproken en/of nieuw verschenen documenten
       \item Rondvraag en sluiting
    \end{enumerate}
  \item Aansluitend vertelden Jeroen Medema en Harm Smit iets over hun werk aan
        de Van Dale bestanden.
\end{itemize}
\section {Opening en notulen}
De notulen van de vorige vergadering werden met een kleine wijziging aangenomen.
\section {Diverse mededelingen}
\begin{enumerate}
  \item Lisette en Jan O. krijgen aansluitend op hun huidige contract het 
        aanbod in dienst te treden van Philips.
  \item Dhr. Goyer (vertegenwoordiger TDS in Raad van bestuur) van Philips 
        heeft met Jan L. een gesprek gehad over ROSETTA. Hij was vrij positief 
        over mogelijke toekomstige toepassingen van Rosetta in de industrie.
  \item Dhr. Pannenborg van Philips, die in het `Eurotra evaluatie-comit\'{e}'
        zitting heeft, is bij Jan L. op bezoek geweest om diens mening over 
        Eurotra te horen. Tevens heeft dhr. Pannenborg gevraagd of Jan iemand
        kende die deskundig was op gebied van automatisch vertalen en in het 
        comit\'{e} zitting zou kunnen nemen. Op voorstel van Lisette
        en Andr\'{e} heeft Jan daarop Pierre Isabelle genoemd.
  \item ROSETTA zal nu definitief {\em geen} bijdrage leveren aan de VIDEO 
        WRITER. De
        HIS heeft inmiddels naast dhr. Konst ook contact met het bedrijf INK in
        Amsterdam over woordenboeken voor de VIDEO WRITER. Uit algemene 
        belangstelling zullen Jan L. en Harm bij INK op bezoek gaan.
  \item Dhr. Pennings van het NOBIN zal met Jan komen praten over het 
        `meerjarenplan automatisch vertalen'. Dit rapport ligt ter inzage bij 
        Jan.
  \item {\bf Oproep 1:} Jan doet een dringend verzoek om suggesties en 
        idee\"{e}n m.b.t. tot natuurlijke taal in verband met 
        mogelijke subsidies in het van kader van SPIN.
  \item {\bf Oproep 2:} Daarnaast het verzoek eens na te denken over het maken 
        van een selectie van circa 10.000 woorden voor de woordenboeken van 
        ROSETTA3. Dit is de hoeveelheid woorden waarvoor het nog mogelijk zal 
        zijn de attributen {\bf goed} te vullen (hetgeen deels met de hand zal 
        moeten geschieden). Van de overige woorden zullen de attributen
        automatisch gevuld worden. Tot nu toe is al eens onderzocht of het 
        onderwerp `muziek' zich hiervoor leende; daartoe is zowel onderzocht
        wat de in Van Dale gebruikte toevoeging `muz.' opleverde (dit leverde
        voornamelijk exotische woorden op) en wat het zoeken naar de string 
        `muz' willekeurig in het lemma als resultaat gaf (dit leverde in ieder
        geval een interessantere verzameling woorden op).
  \item Tijdens de volgende vergadering zal iedereen weer een kort overzicht
        (1 tot 5 minuten durend) geven van zijn of haar werkzaamheden.
  \item {\bf Lisette} deelt mede in het bezit te zijn van een Reference Manual 
        over het nieuwste framework van Eurotra. Het ligt bij haar ter inzage.
  \item {\bf Joep} vertelt dat Loek een brief naar dhr. Bosma heeft gestuurd 
        met de
        mededeling dat de informatici (d.w.z. Carel, Jeroen, Joep, Ren\'{e}) 
        in verband met de slechte performance van de VAX ook 's~avonds gaan 
        werken. In verband hiermee wordt verzocht integration batches pas na
        23.00 uur op te starten.

        Verder zullen de informatici hun documentatie gaan verbeteren; in 
        principe is besloten dat hieraan steeds de eerste week van iedere
        maand gewerkt wordt.

        Tenslotte meldt Joep dat er een nieuw nummer van {\em Computers and 
        Translation} uit is; hierin staan o.a. een artikel van Schubert over DLT
        en een artikel van Joep en Ren\'{e}.
\end{enumerate}
\section {Micro VAX}
Het blijkt niet zo eenvoudig te zijn de Micro VAX nuttig in te zetten: de meeste
idee\"{e}n over het gebruik van het apparaat voor ROSETTA blijken te stuiten op
dure aanpassingen. De meest geschikte toepassing voor ons lijkt te zijn het 
gebruik van de Micro VAX als `\LaTeX-machine'. Hierover zal verder worden 
nagedacht. 

\section {Besproken en/of nieuw verschenen documenten}

\begin{itemize}
  \item {\bf besproken}: niets.
  \item {\bf verschenen}: document 180 van Joep (CF Control grammars:
        Definition of M-PARSER and M-GENERATOR); dit document zal op maandag
        27 april na de vergadering besproken worden.

        Het nieuwe NEHEM-verslag zal binnenkort de deur uitgaan. Het is in 
        kleine kring besproken en wordt binnenkort verspreid.

\end{itemize}
\section {Rondvraag en sluiting}
\begin{enumerate}
   \item {\bf Lisette} vraagt hoe het zit met de praatjes over {\em andere}
         vertaalsystemen; er wordt besloten hiermee weer te beginnen op 
         maandag 13 april.

         Tevens vraagt zij zich af of er (i.v.m. met de aanbiedingen aan Jan O. 
         en haar in dienst van Philips te komen) ook nieuws is over de geplande
         verlenging van het projekt. Jan kon met betrekking tot de verlenging
         op dit moment niets concreets melden; v\'{o}\'{o}r de zomer moet 
         hierover een beslissing vallen.

   \item {\bf Margreet} vraagt of zij nog ergens de door haar gevonden 
         spelfouten en haar opmerkingen betreffende het in de vorige maand 
         als document verspreide verslag aan Van Dale kwijt kan. Daar dit 
         verslag al verstuurd is, lijkt dit niet meer erg nuttig.
          
   \item Aan de orde komt het punt of de notulen van de regelmatig gehouden 
         lingu\"{\i}sten vergaderingen ook als document moeten worden verspreid.
         Besloten wordt dit inderdaad te doen, maar omdat het meestal erg korte
         stukken betreft zullen deze notulen regelmatig groepsgewijs gebundeld 
         worden.
\end{enumerate}
\end{document}
